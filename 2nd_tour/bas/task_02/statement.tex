\assignementTitle{Оценка параметров функции ШИМ – тяга винтомоторной группы БПЛА}{25}{}

В результате испытаний винтомоторной группы БПЛА самолетного типа получена экспериментальная 
зависимость тяги $T= {T_1, T_2, \dots, T_N}$ от длительности импульса ШИМ-сигнала,
поступающего на вход регулятора оборотов $x = {x_1, x_2,
\dots ,x_N}$, где \linebreak $N$ — объем выборки. Модельная зависимость имеет вид:

$T= ax^2 + bx + c.\space(1)$

\putImgWOCaption{14cm}{1}

Пример экспериментальной
зависимости и ее аппроксимации с помощью уравнения (1).

Напишите программу, определяющую по имеющимся
экспериментальным замерам коэффициенты $a$, $b$, $c$  модельной зависимости (1) 
такие, что их значения минимизируют суммарный квадрат отклонения модельной
зависимости от экспериментальной:

$$J=\frac{1}{2} \sum\limits_{i=1}^N(T_i-(ax_i^2+bx_i+c))^2 \to min_{a,b,c}.\space (2) $$

В качестве ответа указать значение критерия (2) с точностью до $10^{-6}$ для найденных коэффициентов 
$a$, $b$, $c$.

\inputfmtSection

В первой строке целое число $N (3\leq N\leq 20)$  — количество замеров.

В следующей строке через пробел $N$  вещественных чисел $x_i\space (1200\leq x_i \leq 1800$,\linebreak $i = 1...N)$. 

В следующей строке через пробел $N$  вещественных чисел $T_i\space (500\leq T_i \leq 25000$, \linebreak $i = 1...N)$.

\outputfmtSection

Единственное вещественное число — значение  $J$  с точностью не ниже $10^{-6}$.

\sampleTitle{1}

\begin{myverbbox}[\small]{\vinput}
    13
    1200.0000000000 1250.0000000000 1300.0000000000 1350.0000000000 
    1400.0000000000 1450.0000000000 1500.0000000000 1550.0000000000 
    1600.0000000000 1650.0000000000 1700.0000000000 1750.0000000000 
    1800.0000000000 
    
    6.9595774289 46.0021893921 122.2574184567 236.7533960937 
    387.2145491474 576.6941427344 803.2261672539 1068.1703676494 
    1368.2455215637 1705.5615035039 2082.3674051067 2493.9881033156 
    2944.9290232887
\end{myverbbox}
\begin{myverbbox}[\small]{\voutput}
    2.3959750295
\end{myverbbox}
\inputoutputTable

\solutionSection

Для решения задачи применим метод наименьших квадратов (МНК). Запишем необходимые условия экстремума 
заданного функционала: $\frac{\partial J}{\partial a}=0$,   $\frac{\partial J}{\partial b}=0$,   $\frac{\partial J}{\partial c}=0$. 
Здесь $\frac{\partial}{\partial k}$, $k=\{a,b,c\}$ – оператор частной производной. При взятии частных производных 
переменной считается только та величина, по которой берется производная, остальные величины дифференцируются 
как константы. Для заданного критерия получим:
$$\frac{\partial J}{\partial a}=\frac{\partial }{\partial a} \left\{ \frac{1}{2} \sum_{i=1}^N [T_i-(ax_i^2+bx_i+c)]^2 \right\}=$$
    $$=\sum_{i=1}^N [(T_i-ax_i^2-bx_i-c)(-x_i^2 )] = \sum_{i=1}^N x_i^4 a+\sum_{i=1}^N x_i^3 b+\sum_{i=1}^N x_i^2 c-\sum_{i=1}^N x_i^2 T_i =0,$$
$$\frac{\partial J}{\partial b}=\frac{\partial }{\partial b} \left\{ \frac{1}{2} \sum_{i=1}^N [T_i-(ax_i^2+bx_i+c)]^2 \right\}=$$
    $$=\sum_{i=1}^N [(T_i-ax_i^2-bx_i-c)(-x_i )] = \sum_{i=1}^N x_i^3 a+\sum_{i=1}^N x_i^2 b+\sum_{i=1}^N x_i c-\sum_{i=1}^N x_i T_i =0,$$
$$\frac{\partial J}{\partial c}=\frac{\partial }{\partial c} \left\{ \frac{1}{2} \sum_{i=1}^N [T_i-(ax_i^2+bx_i+c)]^2 \right\}=$$
    $$=\sum_{i=1}^N [(T_i-ax_i^2-bx_i-c)(-1)] = \sum_{i=1}^N x_i^2 a+\sum_{i=1}^N x_i b+Nc-\sum_{i=1}^N T_i =0.$$

Полученные три уравнения представляют собой систему линейных алгебраических уравнений (СЛАУ) относительно параметров a,b,c. Введем следующие замены переменных: 

$$a_{11}=\sum_{i=1}^N x_i^4, \space a_{12}=\sum_{i=1}^N x_i^3, \space a_{13}=\sum_{i=1}^N x_i^2, \space b_1=\sum_{i=1}^N x_i^2 T_i,$$
$$a_{21}=\sum_{i=1}^N x_i^3, \space a_{22}=\sum_{i=1}^N x_i^2, \space a_{23}=\sum_{i=1}^N x_i, \space b_2=\sum_{i=1}^N x_i T_i,$$
$$a_{31}=\sum_{i=1}^N x_i^2, \space a_{32}=\sum_{i=1}^N x_i, \space a_{33}=N, \space b_3=\sum_{i=1}^N T_i.$$

В новых обозначениях СЛАУ примет вид:
$$a_{11} a+a_{12} b+a_{13} c=b_{1},$$
$$a_{21} a+a_{22} b+a_{23} c=b_{2},$$
$$a_{31} a+a_{32} b+a_{33} c=b_{3}.$$

Оптимальные в смысле заданного критерия параметры находятся как решение полученной системы уравнений. При использовании метода Крамера решение можно представить в следующем виде:
$$a^*=\frac{\Delta_1}{\Delta},b^*=\frac{\Delta_2}{\Delta},c^*=\frac{\Delta_3}{\Delta},$$
где
$$\Delta=a_{11} (a_{22} a_{33}-a_{23} a_{32})-a_{12} (a_{21} a_{33}-a_{23} a_{31})+a_{13} (a_{21} a_{32}-a_{22} a_{31}),$$
$$\Delta_1=b_1 (a_{22} a_{33}-a_{23} a_{32})-b_2 (a_{12} a_{33}-a_{13} a_{32})+b_3 (a_{12} a_{23}-a_{13} a_{22}).$$
$$\Delta_2=-b_1 (a_{21} a_{33}-a_{23} a_{31})+b_2 (a_{11} a_{33}-a_{13} a_{31})-b_3 (a_{11} a_{23}-a_{13} a_{21}),$$
$$\Delta_3=b_1 (a_{21} a_{32}-a_{22} a_{31})-b_2 (a_{11} a_{32}-a_{12} a_{31}) + b_3 (a_{11} a_{22}-a_{12} a_{21}).$$

Значение критерия для найденных коэффициентов вычисляется как
$$J=\frac{1}{2} \sum_{i=1}^N (T_i-a^* x_i^2-b^* x_i-c^* )^2 .$$

Программная реализация сводится к последовательному выполнению описанных шагов.

\includeSolutionIfExistsByPath{2nd_tour/bas/task_02}