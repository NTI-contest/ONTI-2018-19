\assignementTitle{Упрежденный разворот БПЛА самолетного типа}{25}

БПЛА совершает горизонтальный полет. Программа полета БПЛА задается при
помощи массива промежуточных пунктов маршрута (ППМ), где каждый  $i$-й ППМ
представляет собой точку в формате $\{\lambda_i, \varphi_i\}$ , где , $\lambda_i$  — долгота ППМ, 
$\varphi_i$   — широта ППМ, измеряемые в формате: градусы, минуты, секунды.
Массив ППМ кодирует кусочно-линейную траекторию, которой БПЛА, ввиду
ограничений на минимальный радиус разворота, не может следовать идеально точно
в точках поворота. Поэтому переход с одной линии пути на следующую происходит
по линии упрежденного разворота, представляющую собой окружность, вписанную в
угол между последовательными линиями пути (красная линия, рис. 1).

\putImgWOCaption{10cm}{1}

Рис 1. Линия упрежденного
разворота при смене линий пути.

Напишите программу, определяющую координаты точек  $A$, $B$, $O$  по известным координатам трех ППМ и
радиусу разворота БПЛА.

Гарантируется возможность выполнения описанного
маневра. Также гарантируется, что точка  $A$  находится на линии между ППМ 1 и ППМ
2, а точка $B$  — между ППМ 2 и ППМ 3. При пересчете долготы и широты в
метрические координаты, использовать коэффициенты пересчета, найденные для точки
ППМ 2. При нахождении этих коэффициентов считать Землю шаром с радиусом 6371 
км.

Координаты точек даны в географической системе координат. Для решения 
задачи перейти в нормальную земную, прикреплённую к точке ППМ 2. 
Коэффициенты пересчёта географических координат в метрические вычислить для точки ППМ 2. 
Задачу решить в полученной плоской (нормальной земной) системе координат. Для обратного 
перехода в географические координаты использовать те же найденные коэффициенты.

(Литература: ГОСТ 20058-80 динамика ЛА в атмосфере)

\inputfmtSection

В первой строке подается радиус разворота $R$  [м].

Затем в следующих 3 строках по 3 числа:  $DD$, $MM$, $SS$  — координаты (часы, минуты, секунды) 
соответствующих ППМ.

Координаты даны в формате (Долгота, Широта).

\outputfmtSection

Три строки: соотвествующие координаты точек $A$, $B$, $O$  в формате $DD$ $MM$ $SS$.

Выводить координаты следует в том же формате (Долгота, Широта).

Градусы и минуты выводить как целые числа, секунды с точностью до 5 знаков после запятой.

\begin{myverbbox}[\small]{\vinput}
    9956.2972006643
    51 6 19.4687440812 44 25 37.9940626228
    51 7 17.0494705038 44 28 11.6165654469
    51 9 19.9157763723 44 29 43.5638163330
\end{myverbbox}
\begin{myverbbox}[\small]{\voutput}
    51 6 47.2291330212 44 26 52.0573941105
    51 8 36.6927585101 44 29 11.2177826557
    51 14 3.5882547879 44 25 28.7662660863
\end{myverbbox}
\inputoutputTable

%\includeSolutionIfExistsByPath{2nd_tour/bas/task_03}