\assignementTitle{Упрежденный разворот БПЛА самолетного типа}{25}{}

БПЛА совершает горизонтальный полет. Программа полета БПЛА задается при
помощи массива промежуточных пунктов маршрута (ППМ), где каждый  $i$-й ППМ
представляет собой точку в формате $\{\lambda_i, \varphi_i\}$ , где , $\lambda_i$  — долгота ППМ, 
$\varphi_i$   — широта ППМ, измеряемые в формате: градусы, минуты, секунды.
Массив ППМ кодирует кусочно-линейную траекторию, которой БПЛА, ввиду
ограничений на минимальный радиус разворота, не может следовать идеально точно
в точках поворота. Поэтому переход с одной линии пути на следующую происходит
по линии упрежденного разворота, представляющую собой окружность, вписанную в
угол между последовательными линиями пути (красная линия).

\putImgWOCaption{7cm}{1}

\begin{center}
Линия упрежденного
разворота при смене линий пути.
\end{center}

Напишите программу, определяющую координаты точек  $A$, $B$, $O$  по известным координатам трех ППМ и
радиусу разворота БПЛА.

Гарантируется возможность выполнения описанного
маневра. Также гарантируется, что точка  $A$  находится на линии между ППМ 1 и ППМ
2, а точка $B$  — между ППМ 2 и ППМ 3. При пересчете долготы и широты в
метрические координаты, использовать коэффициенты пересчета, найденные для точки
ППМ 2. При нахождении этих коэффициентов считать Землю шаром с радиусом 6371 
км.

Координаты точек даны в географической системе координат. Для решения 
задачи перейти в нормальную земную, прикреплённую к точке ППМ 2. 
Коэффициенты пересчёта географических координат в метрические вычислить для точки ППМ 2. 
Задачу решить в полученной плоской (нормальной земной) системе координат. Для обратного 
перехода в географические координаты использовать те же найденные коэффициенты.

(Литература: ГОСТ 20058-80 динамика ЛА в атмосфере)

\inputfmtSection

В первой строке подается радиус разворота $R$  [м].

Затем в следующих 3 строках по 3 числа:  $DD$, $MM$, $SS$  — координаты (часы, минуты, секунды) 
соответствующих ППМ.

Координаты даны в формате (Долгота, Широта).

\outputfmtSection

Три строки: соотвествующие координаты точек $A$, $B$, $O$  в формате $DD$ $MM$ $SS$.

Выводить координаты следует в том же формате (Долгота, Широта).

Градусы и минуты выводить как целые числа, секунды с точностью до 5 знаков после запятой.

\sampleTitle{1}

\begin{myverbbox}[\small]{\vinput}
    9956.2972006643
    51 6 19.4687440812 44 25 37.9940626228
    51 7 17.0494705038 44 28 11.6165654469
    51 9 19.9157763723 44 29 43.5638163330
\end{myverbbox}
\begin{myverbbox}[\small]{\voutput}
    51 6 47.2291330212 44 26 52.0573941105
    51 8 36.6927585101 44 29 11.2177826557
    51 14 3.5882547879 44 25 28.7662660863
\end{myverbbox}
\inputoutputTable

\solutionSection

Для решения задачи требуется привести все координаты к единому метрическому масштабу, что можно сделать в 2 
этапа. Сначала проведем преобразование координат из формата $DD.MM.SS$ к градусам и их долям $dd$:
$$dd=DD+\frac{1}{60} MM+\frac{1}{3600} SS,$$
где $dd=\{\phi_i,\lambda_i\},i=1..3$, $DD$,$MM$,$SS$ – градусы, минуты и секунды координат соответствующей $i$-ой точки. 

Для пересчета угловых координат в метрические принимаются два допущения:
\begin{enumerate}
    \item Земля имеет форму идеального шара,
    \item В окрестности опорной точки (относительно которой рассчитываются коэффициенты преобразования) 
    кривизна поверхности бесконечно мала.
\end{enumerate}

За опорную точку, согласно условию задачи, принимается точка ППМ2. Тогда из геометрических соображений
\putImgWOCaption{8cm}{2}

\begin{center}
    Связь географической и нормальной систем координат.
\end{center}

\noindent получим следующие соотношения:
$$K_\phi=\frac{2\pi R_E}{360}=111194.9 \text{м/град}$$
$$K_\lambda=\frac{2\pi R_E^\phi}{360}=\frac{2\pi R_E  cos\phi_0}{360}$$
$$X_i=K_\phi (\phi_i-\phi_0)$$
$$Z_i=K_\lambda (\lambda_i-\lambda_0)$$
с помощью которых координаты исходных ППМ переводятся к метрическому формату, и геометрия задачи сводится к 
геометрии на плоскости. 

\putImgWOCaption{9cm}{3}

\begin{center}
    Геометрия упрежденного разворота в горизонтной плоскости.
\end{center}

Очевидно, что точка ППМ2, являющаяся началом метрической системы координат, будет иметь координаты $\{0,0\}$

Из равенства треугольников \{А ППМ2 О\} и \{В ППМ2 О\} следует, что \linebreak 
$[A$ $\text{ППМ}2]$ = [B ППМ2] = $\delta$ – линейное 
упреждение разворота, которое можно выразить через изменение курса $\Delta  \Psi$ и радиус разворота R:
$$\delta=R \space tan \frac{|\Delta \Psi|}{2}.$$

Введем систему координат $O'XZ$ с центром в точке ППМ2. Пусть $\overrightarrow{v_{21}}=[x_{21}  z_{21} ]^T$, 
$\overrightarrow{v_{23}}=[x_{23}  z_{23} ]^T$ – векторы от точки ППМ2 до ППМ1 и ППМ3 соответственно. $T$~–~операция транспонирования. 
Для этих векторов справедливы соотношения скалярного и векторного произведения:
$$\overrightarrow{v_{23}} \cdot \overrightarrow{v_{21}}=\Vert \overrightarrow{v_{23}}\Vert\Vert\overrightarrow{v_{21}} \Vert cos(\pi-\Delta \Psi),$$
$$\overrightarrow{v_{23}}  \times  \overrightarrow{v_{21}}=\Vert\overrightarrow{v_{23}} \Vert\Vert\overrightarrow{v_{21}} \Vert sin(\pi -\Delta \Psi),$$
где $\Vert \cdot \Vert$ - норма (длина) вектора. Поделив второе выражение на первое и выразив $\Delta \Psi$ получим:
$$\Delta \Psi = \pi - atan2 \frac{\overrightarrow{v_{23}} \times \overrightarrow{v_{21}}}{\overrightarrow{v_{23}} \cdot \overrightarrow{v_{21}}},$$
где для плоского случая:
$$ \overrightarrow{v_{23}} \cdot \overrightarrow{v_{21}}=x_{23} x_{21}+z_{23} z_{21},$$
$$ \overrightarrow{v_{23}} \times \overrightarrow{v_{21}}=x_{23} z_{21}-x_{21} z_{23}.$$

Для корректного машинного расчета необходимо, чтобы угол находился в пределах $-\pi..\pi$. Гарантировать это 
можно следующей простой операцией:
$$ \Delta \Psi := atan2 \frac{sin\Delta \Psi}{cos\Delta \Psi '},$$
где <<$:=$>> обозначает присвоение.

Метрические координаты точек А, В находятся масштабированием векторов:
$$\overrightarrow {O' A}= [x_A z_A ]^T=\delta \frac{\overrightarrow{v_{21}}}{\Vert\overrightarrow{v_{21}} \Vert} ,$$
$$\overrightarrow{O' B}=[x_B z_B ]^T=\delta \frac{\overrightarrow{v_{23}}}{\Vert\overrightarrow{v_{23}} \Vert} .$$

Точку О предлагается найти как пересечение линий АО и ВО. Запишем их уравнения в нормальной земной 
системе координат:
$$\left\{ \begin{array}{l}
    a_{11} X+a_{12} Z=b_1\\
    a_{21} X+a_{22} Z=b_2
    \end{array}
    \right.
$$
где $a_{11}=x_{21}$, $a_{12}=z_{21}$,$a_{21}=x_{23}$,$a_{22}=z_{23}$,$b_1=x_{21} x_A+z_21 z_A$,$b_2=x_{23} x_B+z_{23} z_B$,
$X=x_O$,$Z=z_O$. Две прямые согласно условию задачи имеют единственную точку пересечения – $O$, которую можно 
найти как решение полученной системы уравнений:
$$x_O=\frac{b_1 a_{22}-b_2 a_{12}}{a_{11} a_{22}-a_{12} a_{21}},$$
$$z_O=\frac{b_2 a_{11}-b_1 a_{21}}{a_{11} a_{22}-a_{12} a_{21}}.$$

Для осуществления преобразований к требуемому формату ответа воспользуемся формулами. которые получаются на 
основе прямых преобразований. Для точек \linebreak $k=\{A,B,O\}$ переход в угловые координаты:
$$\phi_k=\phi_0+\frac{x_k}{K_\phi},$$
$$\lambda_k=\lambda_0+\frac{z_k}{K_\lambda}.$$

Переход к формату $DD.MM.SS$:
$$DD_k=\lfloor dd_k \rfloor,$$
$$MM_k=\lfloor 60(dd_k-DD_k)\rfloor,$$
$$SS_k=3600(dd_k-DD_k-\frac{1}{60} MM_k ).$$
где $\lfloor  \cdot \rfloor$ - округление в меньшую сторону.

Программная реализация сводится к последовательному выполнению описанных шагов.

\includeSolutionIfExistsByPath{2nd_tour/bas/task_03}