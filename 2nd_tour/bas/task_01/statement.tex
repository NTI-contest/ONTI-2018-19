\assignementTitle{Высота и скорость полёта}{15}{}

Приемник воздушного давления (ПВД) выдает информацию о полном 
$P_{\text{полн}}$ и статическом $P_{\text{стат}}$ давлении. Напишите программу, 
определяющую высоту $H$, скорость полета $V$ и число Маха $M$ согласно модели
стандартной атмосферы, считая, что полет проходит не выше тропосферы.

\inputfmtSection

Два вещественных числа через пробел $P_{\text{полн}}$ и $P_{\text{стат}}$ — значения полного и статического давления в Паскалях.

\outputfmtSection

Три вещественных числа через пробел $H$, $V$, $M$ — высота в метрах, скорость в~м/с, и число 
Маха с точностью не ниже $10^{-3}$.

\sampleTitle{1}

\begin{myverbbox}[\small]{\vinput}
    137842.5939207129 40814.5390179144
\end{myverbbox}
\begin{myverbbox}[\small]{\voutput}
    7039.2966692834 575.1731540867 1.8430291164
\end{myverbbox}
\inputoutputTable

\solutionSection

В теоретической справке к задаче даны следующие исходные соотношения:

$$T=T_0+ \beta H,$$
$$\frac{P}{P_0} =\left(\frac{T}{T_0}\right)^{-\frac{g}{\beta R}},$$
$$\frac{\rho}{\rho_0} =\left(\frac{T}{T_0}\right)^{-\frac{g}{\beta R} - 1},$$
где $H$ – высота [м], $T$ – температура воздуха на высоте $H$ [К], $P$ – статическое давление воздуха на 
высоте~$H$~[Па], $\rho$ – плотность воздуха на высоте $H$ [кг/м$^3$], $T_0=288.15$, $P_0=101325$,$\rho_0=1.225$ – 
значения соответствующих величин на уровне моря ($H=0$), $g=9.81$ – ускорение свободного падения [м/с$^2$],  
$\beta =\frac{\partial T}{\partial H}=-0.0065$~– температурный градиент [К/м] (значение указано для стратосферы), 
$R=287$ – газовая постоянная воздуха [Дж/кг/К]. 

Подставим зависимость температур от высоты в формулу для давлений и проведем вывод формулы для нахождения высоты 
по статическому давлению:
$$\frac{P}{P_0} =\left(\frac{T_0+ \beta H}{T_0}\right)^{-\frac{g}{\beta R}},$$
$$\frac{P}{P_0} =\left(1+ \frac{\beta H}{T_0} \right)^{-\frac{g}{\beta R}},$$
$$1+ \frac{\beta H}{T_0} =\left(\frac{P}{P_0} \right)^{-\frac{\beta R}{g}},$$
$$H=\frac{T_0}{\beta}\left[\left(\frac{P}{P_0}\right)^{-\frac{\beta R}{g}}-1 \right].$$

Вычислив значение высоты, по исходным формулам можно вычислить температуру и плотность. 

Скорость звука рассчитывается на основе температуры с использованием следующего соотношения:
$$a=\sqrt{\gamma RT},$$
где $\gamma=1.4$ – показатель адиабаты воздуха.

Для определения скорости полета воспользуемся законом Бернулли:
$$P_\text{полн}=P+\frac{\rho V^2}{2},$$
откуда выражая V получим:
$$V=\sqrt{\frac{2P_\text{дин}}{\rho}},$$
где $P_\text{дин}=P_\text{полн}-P$.

Число Маха есть отношение скорости полета к местной скорости звука:

$$M=\frac{V}{a}.$$

Таким образом были получены формулы для нахождения высоты, скорости полета и числа Маха на основе модели МСА. 
Программная реализация в образце решения основана на объектно-орииентированном подходе. Созданы два класса: 
Atmosisa (модель стандартной атмосферы), PitotTube (модель приемника воздушного давления). Первый класс 
проводит вычисления параметров атмосферы на основе статического давления, второй – скорости и числа Маха 
на основе параметров атмосферы и динамического давления.

\includeSolutionIfExistsByPath{2nd_tour/bas/task_01}