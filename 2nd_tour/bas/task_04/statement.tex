\assignementTitle{Определение координат наземного объекта по камере}{35}{}

Камера жестко закреплена на корпусе БПЛА и ее оптическая ось сонаправлена с
продольной осью $X^{BF}$ связанной системы координат БПЛА (Body Frame~—~$BF$). Алгоритм технического зрения распознает искомый объект на снимке (точка~$N$) и вычисляет его координаты в системе пиксельных
координат камеры $PCF$ (Pixel Camera Frame) в виде целочисленных пикселей ($N^{PCF}_X, N^{PCF}_Y$). 
Напишите программу, которая по известным 

\begin{itemize}
    \item координатам точки в системе пиксельных координат камеры ($PCF$: \linebreak $N^{PCF}_X, N^{PCF}_Y$ ) [пиксели],
    \item углам ориентации БПЛА: рыскание $ \psi $, крен $ \gamma $, тангаж $ \vartheta $ [град]
    \item местоположению БПЛА в нормальной земной системе координат $EF$ (Earth Frame): $S_X^{EF}$ 
    (соответствует широте), 
    $ S_Y^{EF} $(соответствует высоте), \linebreak $ S_Z^{EF} $(соответствует долготе) [м].
\end{itemize}

определит координаты искомого объекта $ M $ в нормальной земной системе координат $ EF $: $ M_X^{EF} $, $ M_Y^{EF} $[м].
 
При решении считать, что: координаты БПЛА и координаты точки $ S $ совпадают, оптическая ось 
камеры проходит через центр снимка, точка $ M $ гарантированно попадает в поле зрения камеры, 
подстилающая поверхность плоская и высота точки М равна 0, искажения отсутствуют.

\putImgWOCaption{12cm}{1}

\begin{center}
Используемые системы координат.
\end{center}

Характеристики камеры:

\begin{itemize}
    \item разрешение $RES_X \times RES_Y = 2592 х 1944$,
    \item размер одного пикселя $s_x = s_y = 1.4 \text{мкм}$,
    \item фокусное расстояние $ F = 3.6$ мм (между точкой $S$ и \\ плоскостью $O^{PCF}X^{PCF}Y^{PCF}) $.
\end{itemize}

\inputfmtSection

$N_X^{PCF}, N_Y^{PCF}, \psi, \vartheta, \gamma, S_X^{EF}, S_Y^{EF}, S_Z^{EF}$ 

\outputfmtSection

$M_X^{EF}, M_Z^{EF}$. Ответ дать округленным до 1 м по правилам математического округления.

\sampleTitle{1}

\begin{myverbbox}[\small]{\vinput}
    2319.0000000000 160.0000000000 170.6723227638 -33.2580034791 
    13.6495559557 1805.1698355052 89.1656239509 -7267.9830317087
\end{myverbbox}
\begin{myverbbox}[\small]{\voutput}
    1754 -7297
\end{myverbbox}
\inputoutputTable

\solutionSection

Поскольку точки $M$,$S$,$N$ находятся на одной линии и точка $M$ принадлежит плоскости 
$O^{EF} X^{EF} Z^{EF}$, то для нахождения ее координат достаточно найти точку пересечения прямой $SN$ и плоскости 
$O^{EF} X^{EF} Z^{EF}$. Для этого необходимо перевести координаты точки $N$ в систему координат двух других точек – 
нормальную земную.

Во-первых, проведем масштабирование координат камеры для перехода к метрическому формату:
$$N_x^{MCF}=s_x N_x^{PCF},$$
$$N_y^{MCF}=s_y N_y^{PCF},$$
$$N_z^{MCF}=0,$$
где $MCF$  обозначает Metric Camera Frame. Ее ось дополняет $N_x^{MCF}$ и $N_y^{MCF}$ до правой тройки векторов.

Во-вторых, система координат $O^{MCF} X^{MCF} Y^{MCF} Z^{MCF}$ связана со строительной системой координат 
$O^{BF} X^{BF} Y^{BF} Z^{BF}$ параллельным переносом:
$$N_x^{BF}=F$$
$$N_y^{BF}=-N_x^{MCF}+0.5s_x RES_x$$
$$N_z^{BF}=-N_y^{MCF}+0.5s_y RES_y$$
в результате которого становятся известными координаты точки $N$ в строительной (связанной) системе координат.

Третье заключительное преобразование состоит из двух этапов: вращение и параллельный перенос. 
Под вращением понимается перевод координат связанной системы в систему координат, оси которой 
сонаправлены осям нормальной земной $O^{EF} X^{EF} Y^{EF} Z^{EF}$, но с центром в точке $S$. 
Параллельный перенос смещает начало координат в точку $O^{EF}$. Преобразование имеет вид:
$$N_x^{EF}=S_x^{EF}+a_{11} N_x^{BF}+a_{12} N_y^{BF}+a_{13} N_z^{BF}$$
$$N_y^{EF}=S_y^{EF}+a_{21} N_x^{BF}+a_{22} N_y^{BF}+a_{23} N_z^{BF}$$
$$N_z^{EF}=S_z^{EF}+a_{31} N_x^{BF}+a_{32} N_y^{BF}+a_{33} N_z^{BF}$$
где

\begin{tabular}{ccc}
    $a_{11}=c\psi c \vartheta$ & $a_{21}=s \vartheta$ & $a_{31}=-s\psi c \vartheta$ \\
    $a_{12}=s\psi s \gamma-c\psi s \vartheta c \gamma$ & $a_{22}=c \vartheta c \gamma$ & $a_{32}=c\psi s \gamma-s\psi s \vartheta c \gamma$ \\
    $a_{13}=s\psi c \gamma-c\psi s \vartheta s \gamma$ & $a_{23}=-c \vartheta s \gamma$ & $a_{33}=c\psi c \gamma-s\psi s \vartheta s \gamma$ \\
\end{tabular}

\noindent и введены обозначения $s(\cdot)=sin(\cdot), c(\cdot)=cos(\cdot)$. Углы крена, тангажа и рыскания показаны на рисунке.

\putImgWOCaption{8cm}{2}

\begin{center}
    Углы ориентации ЛА.    
\end{center}

Теперь, когда координаты точек известны в общей для них системе координат найдем координаты 
точки используя каноническое уравнение прямой в пространстве:
$$\frac{M_x^{EF}-S_x^{EF}}{N_x^{EF}-S_x^{EF}}=\frac{M_y^{EF}-S_y^{EF}}{N_y^{EF}-S_y^{EF}}=\frac{M_z^{EF}-S_z^{EF}}{N_z^{EF}-S_z^{EF}}$$

Из условия $M_z^{EF}=0$ получим решение задачи:
$$M_x^{EF}=S_x^{EF}-\frac{N_x^{EF}-S_x^{EF}}{N_y^{EF}-S_y^{EF}} S_y^{EF}$$
$$M_z^{EF}=S_z^{EF}-\frac{N_z^{EF}-S_z^{EF}}{N_y^{EF}-S_y^{EF}} S_y^{EF}$$

Программная реализация сводится к последовательному выполнению описанных шагов.

\includeSolutionIfExistsByPath{2nd_tour/bas/task_04}