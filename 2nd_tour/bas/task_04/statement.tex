\assignementTitle{Определение координат наземного объекта по камере}{35}

Камера жестко закреплена на корпусе БПЛА и ее оптическая ось сонаправлена с
продольной осью $XBF$ связанной системы координат БПЛА (Body Frame — $BF$). Алгоритм технического зрения распознает искомый объект на снимке (точка $N$) и вычисляет его координаты в системе пиксельных
координат камеры $PCF$ (Pixel Camera Frame) в виде целочисленных пикселей ($N^{PCF}_X, N^{PCF}_Y$). 
Напишите программу, которая по известным 

\begin{itemize}
    \item координатам точки в системе пиксельных координат камеры ($PCF$: \linebreak $N^{PCF}_X, N^{PCF}_Y$ ) [пиксели],
    \item углам ориентации БПЛА: рыскание $ \psi $, крен $ \gamma $, тангаж $ \vartheta $ [град]
    \item местоположению БПЛА в нормальной земной системе координат $EF$ (Earth Frame): $ S_X^{EF}$ 
    (соответствует широте), 
    $ S_Y^{EF} $(соответствует высоте), \linebreak $ S_Z^{EF} $(соответствует долготе) [м].
\end{itemize}

определит координаты искомого объекта $ M $ в нормальной земной системе координат $ EF $: $ M_X^{EF} $, $ M_Y^{EF} $[м].
 
При решении считать, что: координаты БПЛА и координаты точки $ S $ совпадают, оптическая ось камеры проходит через центр снимка, точка $ M $ гарантированно попадает в поле зрения камеры, подстилающая поверхность плоская и высота точки М равна 0, искажения отсутствуют.

\putImgWOCaption{12cm}{1}

Рис. 1. Используемые системы координат.

Характеристики камеры:

\begin{itemize}
    \item разрешение $RES_X x RES_Y = 2592 х 1944$,
    \item размер одного пикселя $s_x = s_y = 1.4 \text{мкм}$,
    \item фокусное расстояние $ F = 3.6 \text{мм}$ (между точкой $S$ и \\ плоскостью $O^{PCF}X^{PCF}Y^{PCF}) $.
\end{itemize}

\inputfmtSection

$N_X^{PCF}, N_Y^{PCF}, \psi, \vartheta, \gamma, S_X^{EF}, S_Y^{EF}, S_Z^{EF}$ 

\outputfmtSection

$M_X^{EF}, M_Z^{EF}$. Ответ дать округленным до 1 м по правилам математического округления.

\begin{myverbbox}[\small]{\vinput}
    2319.0000000000 160.0000000000 170.6723227638 -33.2580034791 
    13.6495559557 1805.1698355052 89.1656239509 -7267.9830317087
\end{myverbbox}
\begin{myverbbox}[\small]{\voutput}
    1754 -7297
\end{myverbbox}
\inputoutputTable

%\includeSolutionIfExistsByPath{2nd_tour/bas/task_04}