\assignementTitle{Белые или чёрные?}{15}

Winning and losing isn't everything;

sometimes, the journey is just as important as the outcome.

Alex Morgan

В первой задаче вам предстоит анализировать шахматные партии. Вам дан датасет, в котором содержатся описания шахматных партий за некоторый период времени с одного сайта для онлайн-игры в шахматах. Вам гарантируется, что каждая партия закончилась победой либо чёрных, либо белых. Ваша задача будет состоять в том, чтобы для каждой партии ответить, какая сторона победила.

\inputfmtSection

Вам надо будет скачать датасет с играми (\url{https://drive.google.com/file/d/1IsebLv8-AcLoBbIbl3gCVJ7cu_oo0eQV/view}). Он имеет типичный для машинного обучения формат .csv. Про каждую партию вам известна следующая информация:

\begin{enumerate}
    \item 'id' -- уникальный идентификатор партии
    \item 'rated' -- True/False -- была ли партия рейтинговой
    \item 'created\_at' -- время начала партии
    \item 'last\_move\_at' -- во сколько был совершён последний ход
    \item 'turns' -- сколько ходов было в партии
    \item 'white\_id' -- уникальный идентификатор игрока белыми
    \item 'white\_rating' -- рейтинг игрока белыми
    \item 'black\_id' -- уникальный идентификатор игрока чёрными
    \item 'black\_rating' -- рейтинг игрока чёрными
\end{enumerate}

\outputfmtSection

В систему вам необходимо сдать файл, в котором будет один столбец с заголовком ‘winner’, в каждой ячейке которого будет написано ‘black’ или ‘white’. Пример посылки. (\url{https://drive.google.com/file/d/1NE48oM5IuBGLbrJ1Dk9Noti4mdsmmj8X/view?usp=sharing})

\markSection

За эту задачу вы можете получить 0 или 15 баллов: 15 -- в случае, если ваш ответ полностью верен, 0 -- в противном случае.

%\includeSolutionIfExistsByPath{2nd_tour/big_data/task_01}