\assignementTitle{Это легковушка? Это автобус?}{15}

В этой задаче вам предстоит по данным GPS-навигации определять тип средства передвижения -- это автобус или легковой автомобиль?

\inputfmtSection

В качестве тренировочной выборки вам даны некоторые данные GPS-навигации о движении транспорта в Красноярске.\\ \url{https://drive.google.com/open?id=1QdZJT05NiqgCDPYrPHJBWCK2SeORK2Fo6HN-ODT7cgE}

В качестве признаков имеются средняя скорость, время в пути, пройденная дистанция,  rating -- оценочный параметр, выражающий насколько данное средство передвижения удобно для перемещения по городу(3 -- удобно, 2 -- нормально, 1 -- плохо), а также ответ --  $'car'$ или $'bus'$.

Помимо этого вам дана тестовая выборка(test.csv), набор столбцов которой отличается от 
обучающей только отсутствие столбца car\_or\_bus. Как вы уже могли догадаться, ваша задача -- 
обучиться на обучающей выборке и научиться предсказывать для тестовой выборки тип каждого из 
указанных в ней средств передвижения (car или bus).




\outputfmtSection

Вам нужно сдать в систему файл, в котором для каждой строки из тестовой выборки указан предполагаемый вами ответ (car/bus). Пример посылки. (\url{https://drive.google.com/file/d/1dMOJiqO_UU7ey4o2dcVjhS23pfbKaea5/view?usp=sharing})

\markSection

Чем больше правильных ответов вы дадите, тем больше баллов вы заработаете. 
Назовём accuracy\_score величину, равную отношению данных правильно ответов к общему количеству строк в 
test.csv. 

Тогда:

\begin{enumerate}
    \item Если 0.6  <= accuracy\_score <= 0.85, то вы получаете за задачу 10 баллов
    \item Если 0.85 <= accuracy\_score <= 0.95, то вы получаете за задачу 20 баллов
    \item Если 0.95 <= accuracy\_score <= 1.0, то вы получаете за задачу 30 баллов
\end{enumerate}

Обратите внимание, у вас есть только 5 попыток.

%\includeSolutionIfExistsByPath{2nd_tour/big_data/task_02}