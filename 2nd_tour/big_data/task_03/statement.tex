\assignementTitle{О чём эта новость?}{60}{}

В этой задаче вам предстоит для некоторого набора новостных заголовков и текстов определить тематику новости.

Новость может быть одного из 7 типов:

\begin{enumerate}
    \item Мир -- “world”
    \item Наука -- “science”
    \item Культура -- “culture”
    \item Экономика -- “economy”
    \item Политика -- “politics”
    \item Общество -- “society”
    \item Религия -- “religion”
\end{enumerate}

Особенностью данной задачи является то, что обучающий датасет и модель вам предстоит сделать самим.

Для сбора и структурирования датасета новостей жюри рекомендует использовать язык программирования 
Python и фреймворк Scrapy. Небольшой туториал по этой библиотеке вы можете найти на официальном 
сайте фреймворка (\url{https://doc.scrapy.org/en/latest/intro/tutorial.html}).

\inputfmtSection

Вам дан датасет с новостями (test.csv), в котором содержатся заголовки и тексты каждой статьи из подборки жюри (все эти новости реальные и взяты из архивов новостных сайтов разных годов).

\outputfmtSection

Вы должны сдать файл, в котором для каждой новостной статьи указана одна из тематик из списка выше. 
Правильный формат сдаваемого файла вы можете увидеть в файле (sample\_submission.csv\\ \url{https://drive.google.com/file/d/1EX5eUmNjgr5E3YpjCuZ6dmfohVIh1Izq/view?usp=sharing}).

\markSection

Всего за эту задачу вы можете набрать до 60 баллов.

Определим accuracy\_score как долю новостей, которые вы классифицировали правильно к общему количеству новостей в test.csv. Чем эта метрика выше, тем больше баллов вы получите, а именно:

\begin{enumerate}
    \item Если 0.3 <= accuracy\_score <= 0.5, то вы получаете 10 баллов
    \item Если 0.5 <= accuracy\_score <= 0.6, то вы получаете 20 баллов.
    \item Если 0.6 <= accuracy\_score <= 0.7, то вы получаете 30 баллов.
    \item Если 0.7 <= accuracy\_score <= 0.8, то вы получаете 40 баллов.
    \item Если 0.8 <= accuracy\_score <= 0.9, то вы получаете 50 баллов.
    \item Если 0.9 <= accuracy\_score <= 1.0, то вы получаете 60 баллов.
\end{enumerate}

%\includeSolutionIfExistsByPath{2nd_tour/big_data/task_03}