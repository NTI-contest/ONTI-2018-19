\assignementTitle{Орбитальная механика}{6}{}

Спутник, выполняющий съёмку Земли, обращается по круговой низкой околоземной орбите с высотой $Н=500$ км. 
Разрешение получаемых снимков при этом $a=1$~м. Так как значительную часть времени он находится вне зоны 
радиовидимости наземных антенн приёма данных, то для ускорения передачи снимков на Землю она ведётся 
через группировку геостационарных спутников. Высота геостационарной орбиты составляет $H_\text{ГСО} = 35794$ 
км при среднем радиусе Земли $R_{\text{зем}} = 6730$ км. 

а) Сколько таких ретрансляторов Р нужно, чтобы в каждый момент времени фотоспутник Ф был в зоне радиовидимости антенны спутника-ретранслятора Р? Считать, что передача данных возможна, если угол между прямой, соединяющей спутник Ф и ретранслятор, и прямой, 
соединяющей ретранслятор и центр Земли, составляет не более 5.73 градусов. Все спутники-ретрансляторы 
расположены в вершинах правильного многоугольника, вписанного в окружность геостационарной орбиты.

б) В какой-то момент заказчик снимков попросил снять небольшую площадь на Земле в точке П с разрешением 0.7 $\cdot$а. 
Для этого спутник Ф переводится на эллиптическую орбиту с перигеем над точкой П. На сколько нужно затормозить спутник Ф для этого? 
Принять, что спутник тормозится включением на очень малое время реактивного двигателя, сопло, которого направлено вперёд.

в) Как и насколько изменится период обращения спутника Ф? Укажите ответ со знаком +, если период увеличился и со знаком -, если уменьшился.

\subsubsection*{Критерий оценки}

П. “А”: Точный ответ: 2 балла

П. “Б”, Точный ответ в м/с с округлением до целого: 2 балла

П. “В”, Точный ответ в секундах с округлением до 0.5 с: 2 балла

\solutionSection

\begin{enumerate}
    \item[a)] Нарисуем схему взаимного расположения спутников и Земли $\alpha$
 
    \putImgWOCaption{10cm}{1}

    Тогда задача сводится к нахождению угла $\theta$ на основе известного угла $\alpha=5.73^\circ$. 
    Из рисунка ясно, что каждый спутник-ретранслятор обслуживает часть орбиты фотоспутника с углом дуги $2\theta$.  
    Запишем для данной схемы систему из двух уравнений и двух неизвестных:
    угла $\theta$ и отрезка $L$. Для второго уравнения используем теорему синусов.
    
    $$\left\{
        \begin{array}{l}
        (H_\text{ГСО}+R_\text{зем})=(H+R_\text{зем})\cdot cos\theta +L\cdot cos\alpha\\
        \frac{L}{sin\theta}=\frac{H+R_\text{зем}}{sin\alpha}
        \end{array} \right.        
    $$
    
    Избавимся от неизвестной дальности $L$:
    $$L=\frac{H+R_\text{зем}}{sin\alpha} \cdot sin\theta$$

    Тогда в первом уравнении останется неизвестным только искомый угол $\theta$:
    $$(H_\text{ГСО}+R_\text{зем} )=(H+R_\text{зем} )\cdot cos\theta +\frac{H+R_\text{зем}}{sin\alpha} \cdot sin\theta \cdot  cos\alpha$$
    
    Упростим:
    $$(H_\text{ГСО}+R_\text{зем} )=(H+R_\text{зем} )\cdot cos\theta + \frac{1}{tg\alpha} \cdot (H+R_\text{зем} )\cdot sin\theta$$ 
    
    Сведём уравнение к одной переменной $tg\theta$ при помощи подсказок из Указания:
    $$(H_\text{ГСО}+R_\text{зем} )=(H+R_\text{зем} )\cdot  \frac{1}{\sqrt{1+tg^2 \theta}}+\frac{1}{tg\alpha} \cdot (H+R_\text{зем} )\cdot  \frac{tg \theta}{\sqrt{1+tg^2 \theta}}$$  
    
    Сделаем замену переменной  $х= tg\theta$. 

    Также обозначим для упрощения $\frac{(H_\text{ГСО}+R_\text{зем} )}{(H+R_\text{зем} )}=A$

    Тогда:
    $$A=\frac{1}{\sqrt{1+x^2}}+\frac{1}{tg\alpha} \cdot  \frac{x}{\sqrt{1+x^2}}$$

    Преобразуем:
    $$A\cdot \sqrt{1+x^2}=1+\frac{1}{tg\alpha} \cdot x$$

    Возведём в квадрат:
    $$A^2 \cdot (1+x^2 )=1+2\cdot  \frac{1}{tg\alpha} \cdot x+\left(\frac{1}{tg\alpha}\right)^2 \cdot x^2$$

    Пришли к квадратному уравнению:
    $$\left(\left(\frac{1}{tg\alpha}\right)^2-A^2\right)\cdot x^2+2\cdot  \frac{1}{tg\alpha} \cdot x+(1-A^2 )=0$$
    
    Его решение, очевидно, должно быть положительным, так как угол $\theta>0$. Тогда в формуле для корней выбираем знак «+»:
    $$x=\frac{-2\cdot  \frac{1}{tg\alpha}+\sqrt{\left({2}{tg\alpha}\right)^2-4\cdot \left(\left(\frac{1}{tg\alpha}\right)^2-A^2 \right)\cdot (1-A^2)}}{2\cdot \left(\left(\frac{1}{tg\alpha}\right)^2-A^2\right)}$$ 

    Подставляем числа:
    $$tg\alpha \approx \alpha=\frac{5.73}{57.3}=0.1$$
    $$A=((H_\text{ГСО}+R_\text{зем} ))/((H+R_\text{зем} ) )=((35794+6370))/((500+6370) )=6,137409$$
    
    И находим $ х= tg\theta$
    $$x=\frac{-2\cdot 10+\sqrt{(20^2-4\cdot (10^2-37.66779)\cdot (1-37.66779)}}{2\cdot (10^2-37.66779)}=$$
    $$=\frac{-2\cdot 10+√(20^2-4\cdot (10^2-37.66779)\cdot (1-37.66779) ))}{2\cdot (10^2-37.66779)} =(-2\cdot 10+√(400+4\cdot 62,33221\cdot 36,66779))/2\cdot 62,33221=(-2\cdot 10+√9542,33754)/124,66442=(-2\cdot 10+97,6849)/124,66442=77,6849/124,66442=0,62315$$
    Определим угол \theta:
    \theta=arctg x=arctg (0,62315)=31,929 градусов
    
    Фотоспутник будет видеть один и тот же ретранслятор на протяжении дуги своей орбиты в 2\theta. Тогда количество необходимых ретрансляторов можно определить как:
    N_ретр=360^/2\cdot \theta=360^/(2\cdot arctg x)=360^/(2\cdot 31,929 )=5,64
    Очевидно, что число спутников не может быть дробным. Значит, ретрансляторы будут работать с небольшим перекрытием зон радиовидимости, а их нужно 6 штук.
    
    б) Так как по условию, разрешение снимка для точки П должно быть 0,7∙а, 
    то и высота орбиты в перигее должна быть 0,7∙Н. Тогда с учётом  второго закона Кеплера и закона сохранения момента импульса:
    (V_орб-ΔV)·(R_\text{зем}+H)=V_перигей·(R_\text{зем}+0,7·H)
    Где V_орб – скорость спутника на первоначальной круговой орбите с высотой Н
    
    С другой стороны, по закону сохранения энергии, 
    m·(V_орб-ΔV)^2/2-(G·M_\text{зем}·m)/(R_\text{зем}+H)=m·(V_перигей )^2/2-(G·M_\text{зем}·m)/(R_\text{зем}+0,7·H)
    Тогда
    (V_орб-ΔV)^2/2-(G·M_\text{зем})/(R_\text{зем}+H)=((V_орб-ΔV)·((R_\text{зем}+H))/((R_\text{зем}+0,7·H) ))^2/2-(G·M_\text{зем})/(R_\text{зем}+0,7·H)
    
    После преобразования и домножения на 2:
    -(V_орб-ΔV)^2+((V_орб-ΔV)·((R_\text{зем}+H))/((R_\text{зем}+0,7·H) ))^2==2·G·M_\text{зем}·(1/(R_\text{зем}+0,7·H)-1/(R_\text{зем}+H))
    (V_орб-ΔV)^2·((R_\text{зем}+H)^2/(R_\text{зем}+0,7·H)^2 -1)=2·G·M_\text{зем}·(1/(R_\text{зем}+0,7·H)-1/(R_\text{зем}+H))
    
    Приведём дробь справа к общему знаменателю:
    
    (V_орб-ΔV)^2·((R_\text{зем}+H)^2/(R_\text{зем}+0,7·H)^2 -1)=2·G·M_\text{зем}·((R_\text{зем}+H-R_\text{зем}-0,7·H)/((R_\text{зем}+0,7·H)·(R_\text{зем}+H) ))
    
    (V_орб-ΔV)^2·(((R_\text{зем}+H)^2-(R_\text{зем}+0,7·H)^2)/(R_\text{зем}+0,7·H)^2 )==2·G·M_\text{зем}·((0,3·H)/((R_\text{зем}+0,7·H)·(R_\text{зем}+H) ))
    
    
    Раскроем разность квадратов слева в числителе:
    (V_орб-ΔV)^2·((2·R_\text{зем}·H+H^2-2·0,7·R_\text{зем}·H-0,49·H^2)/(R_\text{зем}+0,7·H)^2 )==2·G·M_\text{зем}·((0,3·H)/((R_\text{зем}+0,7·H)·(R_\text{зем}+H) ))
    
    (V_орб-ΔV)^2·((0,6·R_\text{зем}·H+0,51·H^2)/((R_\text{зем}+0,7·H) ))=2·G·M_\text{зем}·((0,3·H)/((R_\text{зем}+H) ))
    
    (V_орб-ΔV)^2·((0,6·R_\text{зем}+0,51·H)/((R_\text{зем}+0,7·H) ))=2·G·M_\text{зем}·(0,3/((R_\text{зем}+H) ))
    
    Тогда для скорости спутника сразу после выдачи тормозного импульса
    V_орб-ΔV=√((2·G·M_\text{зем}·(0,3/((R_\text{зем}+H) )))/((0,6·R_\text{зем}+0,51·H)/((R_\text{зем}+0,7·H) )))
    Итак,
    ΔV=V_орб-√((2·G·M_\text{зем}·(0,3/((R_\text{зем}+H) )))/((0,6·R_\text{зем}+0,51·H)/((R_\text{зем}+0,7·H) )))=√((G·M_\text{зем})/(R_\text{зем}+H))-√((2·G·M_\text{зем}·(0,3/((R_\text{зем}+H) )))/((0,6·R_\text{зем}+0,51·H)/((R_\text{зем}+0,7·H) )))==√((6,67·10^(-11)·5,97·10^24)/(1000·(6370+500) ))-√(((2·6,67·10^(-11)·5,97·10^24)/1000·(0,3/(6370+500)))/((0,6·6370+0,51·500)/(6370+350)))=7613,28-√(((2·6,67·10^(-11)·5,97·10^24)/10^3 ·4,3668·10^(-5))/0,606696)=7613,28-√((2·6,67·5,97·10^5·4,3668)/0,606696)=7613,28-7571,14=42,14  м/с
    
    в) Для старой, круговой орбиты, большая полуось эллипса равна R\text{зем} + Н . 
    Для новой, эллиптической орбиты, большая полуось эллипса будет: 
    
    0,5∙(2∙R_\text{зем}+H+0,7∙H)=R_\text{зем}+0,85∙H
    
    Тогда по третьему закону Кеплера:
    T_нов/T_круг =((R_\text{зем}+0,85∙H)/(R_\text{зем}+H))^1,5
    Отсюда: 
    T_нов=T_круг ((R_\text{зем}+0,85∙H)/(R_\text{зем}+H))^1,5=2π (R_\text{зем}+H)^1,5/√(G∙M_\text{зем} ) ((R_\text{зем}+0,85∙H)/(R_\text{зем}+H))^1,5
    
    Тогда разность периодов обращения:
    T_нов-T_круг=2π (R_\text{зем}+H)^1,5/√(G∙M_\text{зем} ) (((R_\text{зем}+0,85∙H)/(R_\text{зем}+H))^1,5-1)=5667∙((6795/6870)^1,5-1)=-92,54 с
    
\end{enumerate}

\answerMath{}