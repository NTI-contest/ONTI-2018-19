\assignementTitle{Орбитальная механика}{6}

Спутник, выполняющий съёмку Земли, обращается по круговой низкой околоземной орбите с высотой $Н=500$ км. 
Разрешение получаемых снимков при этом $a=1$ м. Так как значительную часть времени он находится вне зоны 
радиовидимости наземных антенн приёма данных, то для ускорения передачи снимков на Землю она ведётся 
через группировку геостационарных спутников. Высота геостационарной орбиты составляет $\text{НГСО} = 35794$ 
км при среднем радиусе Земли $R_{\text{зем}} = 6730$ км. 

а) Сколько таких ретрансляторов Р нужно, чтобы в каждый момент времени фотоспутник Ф 
был в зоне радиовидимости антенны спутника-ретранслятора Р? Считать, что передача данных возможна, 
если угол между прямой, соединяющей спутник Ф и ретранслятор, и прямой, соединяющей ретранслятор и центр 
Земли, составляет не более 5,73 градусов. Все спутники-ретрансляторы расположены в вершинах правильного 
многоугольника, вписанного в окружность геостационарной орбиты.

б) В какой-то момент заказчик снимков попросил снять небольшую площадь на Земле в точке П с разрешением 0,7 $\cdot$а. 
Для этого спутник Ф переводится на эллиптическую орбиту с перигеем над точкой П. На сколько нужно затормозить спутник Ф для этого? 
Принять, что спутник тормозится включением на очень малое время реактивного двигателя, 
сопло, которого направлено вперёд.

в) Как и насколько изменится период обращения спутника Ф? Укажите ответ со знаком +, 
если период увеличился и со знаком -, если уменьшился.

\subsubsection*{Критерий оценки}

П. “А”: Точный ответ: 2 балла

П. “Б”, Точный ответ в м/с с округлением до целого: 2 балла

П. “В”, Точный ответ в секундах с округлением до 0,5 с: 2 балла