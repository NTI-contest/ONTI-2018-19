\assignementTitle{Оптика}{13}{}

Cпутник вращается по круговой орбите высотой $H=500$ км. Съёмка Земли выполняется с помощью оптико-электронной. 
Фотоприёмник представляет собой матрицу размером $4000 \times 1200$ пикселей, широкая сторона матрицы 
расположена поперёк направления полёта. Размер пикселя матрицы составляет $\delta=10$ мкм.

\begin{enumerate}
\item Каким должно быть фокусное расстояние объектива, чтобы при съёмке с креном (отклонением оси 
объектива от вертикали) до $\theta=30^{\circ}$ разрешение снимков было не хуже 5 метров. Принять, 
что разрешение снимка определяется как наибольший размер проекции пикселя, соответствующего центру поля зрения. 
В случае, если размеры сторон проекции пикселя для одного и того же направления отличаются, рассмотреть их среднее.

\putImgWOCaption{10cm}{1}

Поперечный разрез, спутник летит ОТ наблюдателя рисунка. Так как размер пикселя 
$\delta << f$ (фокусное расстояние), то выполняется примерное равенство углов $\theta' = \theta$.

\item Для обеспечения съёмки цветных изображений на фотоприёмную матрицу наклеили синий, зелёный и 
красный светофильтры. Каждый из них покрывает область матрицы с размером $4000 \times 400$ пикселей. Центральные 
длины волн этих цветовых диапазонов 490; 560; 660 нм соответственно, ширина диапазонов 80 нм для каждого 
(то есть диапазоны длин волн 450...530 нм, 520...600 нм, 620...700 нм). Какой должна быть светосила объектива, 
чтобы при съёмке в надир местности с коэффициентом отражения $r = 0.3$ в каждом из трёх цветовых диапазонов 
изображение не было недоэкспонированным. Недоэкспонированным является снимок, где в пикселе сигнал 
составляет менее $N = 10000$ электронов. Светимость поверхности $E_{\text{пов}}$ [Вт/м2] и поток, падающий на 
фотоприёмник $E_{\text{фок}}$ [Вт/м2], связаны соотношением 

$$E_{\text{фок}}=0.25 \cdot (D/f)2 \cdot E_{\text{пов}} \cdot K_{\text{опт}} \cdot K_{\text{атм}}$$

Принять, что средняя спектральная облучённость поверхности Земли в месте съёмки составляет $e_{\text{зем}} = 1$ Вт/(м$^2 \cdot $нм).

В среднем на 10 фотонов, попавших на пиксель фотоприёмника, приходится 4 сгенерированных электрона в 
пикселе. Коэффициент пропускания объектива $K_{\text{опт}} = 0.8$, коэффициент пропускания атмосферы $K_{\text{атм}} = 0.7$.

\item Функцию передачи контраста камеры на больших пространственных частотах можно оценить как:

$$CTF_{\text{системы}}(v) = \frac{4}{\pi} \cdot MTF_{\text{системы}}(v) =
\frac{4}{\pi} \cdot MTF_{\text{фотоприемника}}(v) \cdot MTF_{\text{объектива}}(v) = $$
$$ = \frac{4}{\pi} \cdot \frac{sin(\pi \cdot v \cdot \delta)}{\pi \cdot v \cdot \delta} \cdot \left(1 - \frac{v}{v_{\text{гр}}} \right)$$
 
Где граничная пространственная частота объектива на длине волны $\lambda$ определяется как:
 
$$v_{\text{гр}} = \frac{1}{\lambda} \cdot \frac{D}{f}$$

Каким будет контраст изображения группы гаражей, стоящих параллельно на земле с 
альбедо 0.1? Крыши гаражей покрыты крашеными стальными листами с альбедо 0.5. 
Рассматривается снимок в красном диапазоне (длина волны - 660 нм). Ширина 
гаражей равна расстоянию между ними и составляет $a = 3.8$ метра. Съёмка ведётся в 
надир с высоты $H = 500$ км.

\end{enumerate}

\subsubsection*{Начальные данные}

Постоянная планка, h, Дж $\cdot$ с: $6.63 \cdot 10^{-34}$

Скорость света, c, м/с: $3 \cdot 10^8$

Время выдержки, мс: 0.53

\subsubsection*{Критерий оценки}

П. 1: Точный ответ, с округлением до сотых в метрах: 2 балла

П. 2: Точный ответ с округлением до тысячных: 5 баллов

П. 3: Точный ответ с округлением до сотых: 6 баллов

\solutionSection

\begin{enumerate}
\item Так как высота орбиты $H<<R_\text{зем}$, а угол 30 градусов достаточно мал, то поверхность Земли можно считать плоской. 

Очевидно, что в случае съёмки с креном квадратному пикселю матрицы соответствует его трапециевидная проекция на поверхность 
Земли. Пусть $X$ – направление вдоль полёта, а $Y$ – поперёк полёта.  

Тогда для центра поля зрения размеры трапеции (средняя линия $a_x$ и высота~$a_y$) определяются через дальность наблюдения $L_\text{набл}$ как:
$$a_x=\frac{L_\text{набл}}{f}\cdot \delta $$
$$a_y=\frac{L_\text{набл}}{f}\cdot \delta \cdot \frac{1}{cos\theta}$$ 

Тогда
$$a_y=\frac{L_\text{набл}}{f} \cdot \delta \cdot \frac{1}{cos\theta^{'}} \sim \frac{L_\text{набл}}{f}\cdot \delta \cdot \frac{1}{\cos\theta} =\frac{H}{f}\cdot \delta \cdot \frac{1}{(cos\theta )^2} $$
 
Интерес представляет $a_y$ , так как она больше $a_x$:

$$a_y=\frac{L_\text{набл}}{f} \cdot \delta \cdot \frac{1}{cos\theta} =\frac{H}{f}\cdot \delta \cdot \frac{1}{(cos\theta )^2}$$

Тогда искомое фокусное расстояние:

$$f=\frac{H}{a_y} \cdot \delta \cdot \frac{1}{(cos\theta )^2} =\frac{500\cdot 10^3}{5}\cdot 10\cdot 10^{-6}\cdot \frac{4}{3}=1.33 \: \text{м}$$

\item Для спектрального диапазона с центральной длиной волны $\lambda$ и шириной $\Delta \lambda$ число упавших на пиксель фотонов будет
$$N_\text{фот}=\frac{N}{0.4}$$

Тогда общая энергия, облучившая пиксель фотоприёмника за время накопления сигнала:
$$W=N_\text{фот}\cdot \frac{h\cdot c}{\lambda}=\frac{N}{0.4}\cdot \frac{h\cdot c}{\lambda}$$

С учётом связи светимости поверхности Земли и облучённости фокальной плоскости имеем при времени накопления $T$:
$$W=E_\text{фок}\cdot \delta^2\cdot T =  \frac{1}{4}  \cdot \left(\frac{D}{f}\right)^2 \cdot (r\cdot e_\text{зем}\cdot \Delta \lambda)\cdot K_\text{опт}\cdot K_\text{атм}\cdot \delta^2\cdot T$$

Тогда:
$$\frac{N}{0.4}\cdot \frac{h\cdot c}{\lambda}=  \frac{1}{4}  \cdot \left(\frac{D}{f}\right)^2\cdot (r\cdot e_\text{зем}\cdot \Delta \lambda)\cdot K_\text{опт}\cdot K_\text{атм}\cdot \delta ^2\cdot T$$

Отсюда можно выразить искомую минимальную светосилу для всех трёх спектральных диапазонов с учётом полученного в п.2 времени накопления Т:
$$\left(\frac{D}{f}\right)^2=10\cdot N\cdot \frac{h\cdot c}{\lambda}\cdot \frac{1}{r\cdot e_\text{зем}\cdot \Delta \lambda\cdot K_\text{опт}\cdot K_\text{атм}\cdot \delta^2\cdot T}=$$
$$=10\cdot 10^4\cdot \frac{6.63\cdot 10^{-34}\cdot 3\cdot 10^8}{\lambda}\cdot \frac{1}{0.3\cdot 1\cdot 80\cdot 0.8\cdot 0.7\cdot 10^{-10}\cdot 5.31\cdot 10^{-4}}=$$
$$=10\cdot 10^4\cdot \frac{6.63\cdot 10^{-34}\cdot 3\cdot 10^8}{\lambda}\cdot \frac{1}{0.3\cdot 1\cdot 80\cdot 0.8\cdot 0.7\cdot 10^{-10}\cdot 5.31\cdot 10^{-4} }=$$
$$=\frac{2.787\cdot 10^{-8}}{\lambda}\left.\right|_{\lambda=490\cdot 10^{-9}; 560\cdot 10^{-9}  ; 660\cdot 10^{-9}  }=0.0568; \: 0.0498; \: 0.0422$$

Так как речь идёт о минимальном сигнале, то нужно ориентироваться на наибольшее требуемое значение светосилы среди всех трёх спектральных каналов.

\item Контраст сцены составляет
$$k_\text{сцены}=\frac{\rho_\text{крыши}-\rho_\text{земли}}{\rho_\text{крыши}+\rho_\text{земли}}=\frac{0.5-0.1}{0.5+0.1}=\frac{2}{3}$$

Ширина изображения одного гаража составляет:

$$d=\frac{f}{H}\cdot a$$

Пространственная частота изображения группы гаражей может быть найдена как:
$$v=\frac{1}{2\cdot d}=\frac{1}{2\cdot \frac{f}{H} \cdot a}=\frac{1}{2\cdot \frac{1.333}{500\cdot 10^3}\cdot 3.8}=49354 \: \text{(пар линий)/м}$$

Граничная пространственная частота на центральной длине волны красного канала объектива с учётом найденной ранее светосилы:
$$v_\text{гр}=\frac{1}{\lambda}\cdot  \left( \frac{D}{f} \right) =\frac{1}{660\cdot 10^{-9}}\cdot \sqrt{0.0568}=361102 \: \text{(пар линий)/м}$$

Тогда контраст изображения примет вид:
$$k_\text{изобр}=CTF_\text{системы}(v)\cdot k_\text{сцены}=\frac{4}{\pi} \cdot \frac{sin(\pi \cdot v\cdot \delta )}{\pi \cdot v\cdot \delta}\cdot \left(1-\left(\frac{v}{v_\text{гр}} \right) \right)\cdot k_\text{сцены}=$$
$$=\frac{4}{3.14}\cdot \frac{sin(3.14\cdot 49354\cdot 10^{-5})}{3.14\cdot 49354\cdot 10^{-5}}\cdot \left(1-\left(\frac{49354}{361102}\right)\right)\cdot \frac{2}{3}=$$
$$=\frac{4}{3.14}\cdot 0.6451\cdot 0.8633\cdot \frac{2}{3}=0.47$$

\end{enumerate}

\answerMath{1. 1.33; 2. 0.0568; 3. 0.47}