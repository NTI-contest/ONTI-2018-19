\assignementTitle{Оптика}{13}{}

Cпутник вращается по круговой орбите высотой $H=500$ км. Съёмка Земли выполняется с помощью оптико-электронной. 
Фотоприёмник представляет собой матрицу размером $4000 \times 1200$ пикселей, широкая сторона матрицы 
расположена поперёк направления полёта. Размер пикселя матрицы составляет $\delta=10$ мкм.

1) Каким должно быть фокусное расстояние объектива, чтобы при съёмке с креном (отклонением оси 
объектива от вертикали) до $\theta=30^{\circ}$ разрешение снимков было не хуже 5 метров. Принять, 
что разрешение снимка определяется как наибольший размер проекции пикселя, соответствующего центру поля зрения. 
В случае, если размеры сторон проекции пикселя для одного и того же направления отличаются, рассмотреть их среднее.

\putImgWOCaption{10cm}{1}

Рисунок к задаче 1: поперечный разрез, спутник летит ОТ наблюдателя рисунка. Так как размер пикселя 
$\delta << f$ (фокусное расстояние), то выполняется примерное равенство углов $\theta' = \theta$.

2) Для обеспечения съёмки цветных изображений на фотоприёмную матрицу наклеили синий, зелёный и 
красный светофильтры. Каждый из них покрывает область матрицы с размером $4000 \times 400$ пикселей. Центральные 
длины волн этих цветовых диапазонов 490; 560; 660 нм соответственно, ширина диапазонов 80 нм для каждого 
(то есть диапазоны длин волн 450...530 нм, 520...600 нм, 620...700 нм). Какой должна быть светосила объектива, 
чтобы при съёмке в надир местности с коэффициентом отражения $r = 0.3$ в каждом из трёх цветовых диапазонов 
изображение не было недоэкспонированным. Недоэкспонированным является снимок, где в пикселе сигнал 
составляет менее $N = 10000$ электронов. Светимость поверхности $E_{\text{пов}}$ [Вт/м2] и поток, падающий на 
фотоприёмник $E_{\text{фок}}$ [Вт/м2], связаны соотношением 

$$E_{\text{фок}}=0,25 \cdot (D/f)2 \cdot E_{\text{пов}} \cdot K_{\text{опт}} \cdot K_{\text{атм}}$$

Принять, что средняя спектральная облучённость поверхности Земли в месте съёмки составляет $e_{\text{зем}} = 1$ Вт/(м$^2 \cdot $нм).

В среднем на 10 фотонов, попавших на пиксель фотоприёмника, приходится 4 сгенерированных электрона в 
пикселе. Коэффициент пропускания объектива $K_{\text{опт}} = 0.8$, коэффициент пропускания атмосферы $K_{\text{атм}} = 0.7$.

3) Функцию передачи контраста камеры на больших пространственных частотах можно оценить как:

$$CTF_{\text{системы}}(v) = \frac{4}{\pi} \cdot MTF_{\text{системы}}(v) =
\frac{4}{\pi} \cdot MTF_{\text{фотоприемника}}(v) \cdot MTF_{\text{объектива}}(v) = $$
$$ = \frac{4}{\pi} \cdot \frac{sin(\pi \cdot v \cdot \delta)}{\pi \cdot v \cdot \delta} \cdot \left(1 - \frac{v}{v_{\text{гр}}} \right)$$
 
Где граничная пространственная частота объектива на длине волны $\lambda$ определяется как:
 
$$v_{\text{гр}} = \frac{1}{\lambda} \cdot \frac{D}{f}$$

Каким будет контраст изображения группы гаражей, стоящих параллельно на земле с 
альбедо 0.1? Крыши гаражей покрыты крашеными стальными листами с альбедо 0.5. 
Рассматривается снимок в красном диапазоне (длина волны - 660 нм). Ширина 
гаражей равна расстоянию между ними и составляет $a = 3.8$ метра. Съёмка ведётся в 
надир с высоты $H = 500$ км.

\subsubsection*{Начальные данные}

Постоянная планка, h, Дж $\cdot$ с: $6.63 \cdot 10^{-34}$

Скорость света, c, м/с: $3 \cdot 108$

Время выдержки, мс: 0.53

\subsubsection*{Критерий оценки}

П. 1: Точный ответ, с округлением до сотых в метрах: 2 балла

П. 2: Точный ответ с округлением до тысячных: 5 баллов

П. 3: Точный ответ с округлением до сотых: 6 баллов