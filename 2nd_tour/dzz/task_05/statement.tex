\assignementTitle{Программа полета}{42}{}

Миссией космического аппарата является съемка г. Москвы и г. Красноярска. При этом съемка каждого города должна происходить в период времени с 9:45 по 12:45 по местному времени. На орбите аппарат оказывается 12 апреля 2019 года в 00:00:00 по UTC.

Координатами центра г. Москвы считать: 55.753960,  37.620393
Координатами центра г. Красноярска считать: 56.010569, 92.852545

Программа полета всегда ориентирует спутник в надир. Запаса энергии аппарата хватает на 180 секунд работы камеры в течении первых трех суток полета. Периоды работы камеры в разное время суммируются.

Съемка считается произведенной, если аппарат находился над точкой, удаленной не более, чем на 50 км от центра города, и камера была включена.

Для решения задачи требуется:


\begin{enumerate}
    \item Определить и задать параметры орбиты, удовлетворяющей условиям съемки
    \item Задать программу управления полезной нагрузкой так, чтобы съемка производилась успешно в течение первых трех суток полета    
\end{enumerate}

\subsubsection*{Начальные данные}

Радиус Земли, м: 6371008.8

Гравитационный параметр, $\mu = G \cdot M$, м$^3/$с$^2$: $3.986004418 \cdot 10^{14}$

Средняя скорость вращения Земли, об/сут: $1.00273781191135448$

Начальный угол вращения Земли: $199.91^{\circ}$

\putImgWOCaption{13cm}{1}

В этой миссии на навигатор наложено ограничение: максимальная высота его работы - 3000 км над поверхностью Земли. При расчет орбиты учитывать, что орбита не подвержена прецессии, а влияние атмосферы на движение КА отсутствует.

\subsubsection*{Критерий оценки}

\begin{itemize}
    \item Аппарат не менее 1 раза в сутки смог провести съемку одного из городов в указанный период времени с отклонением от указанных координат центра не более чем на 25 км: 4 балла
    \item Аппарат не менее 1 раза смог провести съемку одного из городов в указанный период времени с отклонением от указанных координат не более чем на 50 км: 3 балла
\end{itemize}

\solutionSection

В отличие от предыдущей задачи здесь необходимо провести съемку уже двух городов. 
Обратите внимание, что широта двух городов близка по значению. Поэтому, чтобы успешно выполнить миссию, 
необходимо подобрать орбиту так, чтобы между первым и вторым городом при проходе спутника укладывалось 
целое число витков.

Для простоты также определимся, что будем работать с полярной круговой орбитой, а следовательно:

$i = 90^\circ$

$e = 0$

$\omega = 0$

Первой точкой съёмки будем рассматривать г. Красноярск, т. к. он лежит немного севернее г. Москвы.

Долгота восходящего узла определяется также как и в предыдущей задаче:

$$\alpha = 199.91^\circ + \frac{1.00273781191135448 \cdot 360^\circ}{24/t} = 260.0742687^\circ$$
$$\omega = \alpha + 92.852545 = 352.9268137^\circ$$
где $t=4$ часам, т. к. спутник начинает движение в 00:00 по UTS, что соответствует 07:00 часам местному времени 
г. Красноярск. Так как промежуток времени достаточно большой, необязательно брать строго 4 часа.

Теперь встает вопрос о том, как правильно подобрать радиус орбиты. Как говорилось в предыдущей задаче, 
орбита должна быть кратной вращению Земли. Значит имеем:
$$T=\frac{T_\text{зв}}{N}$$

При этом за то время, что Земля проворачивается от Красноярска к Москве, аппарат также должен совершить 
целое число витков вокруг Земли (в принципе можно не строго целое число витков, но число виктов должно 
быть близко к целому), то есть:
$$T\cdot k=\frac{\Delta \lambda}{\omega_\text{з}} $$

Из двух уравнений можно получить соотношение $k(N)$. Таблица с перебором значений $N$ и соответствующим ему 
значением $k$ выглядит следующим образом:

\begin{tabular}{ |l |l |l |}
    N	&k	&h, км\\
    \hline\\
    11	& 1.687649088	& 8524.746\\
    \hline
    13	& 1.994494377	& 7626.308\\
    \hline
    15	& 2.301339666	& 6932.380\\
    \hline
    17	& 2.841093728	& 6377.405\\
    \hline
\end{tabular}

Формула для вычисления высоты:
$$a=\sqrt[3]{\mu\left(\frac{T}{2\pi}\right)^2}$$

Из приведённых соотношений видно, что кратность от 17 и 
более недопустима по высоте, а от 10 и менее ведет к 
возрастанию высоты орбиты, что нежелательно, так как 
при высоте от 3000 км навигатор перестает функционировать.

Наилучшим соотношением для нас будет пара чисел $N=13$ и с 
округлением $k=2$. Соответственно, значение большой полуоси 
$a=7626.308$.

Осталось определить значение средней аномалии, что по 
порядку действий совпадает с предыдущей задачей:
$$N = \frac{t \cdot 60 \cdot 60}{T} = 2.172601086$$

Также берём дробную часть:
$$\phi = 360^\circ \cdot 0.671231605 = 62.13639096^\circ$$
$$M_0 = 56.010569^\circ -  62.13639096^\circ = -6.12582196^\circ$$
$$M_0 = 353.8741780^\circ$$

Алгоритм управления похож на тот, что рассматривался в задаче 
«Управление полезной нагрузкой».

Для начала также вычислим допустимую угловую погрешность:
$$\alpha=\frac{180^\circ\cdot L}{2\pi \cdot R_\text{зем}}$$
$$\alpha = 0.45^\circ$$

И допишем алгоритм:

\begin{minted}[fontsize=\footnotesize, linenos]{python}
    import math
    camera = spacecraft.device("camera").function(0)
    navigator = spacecraft.device("navigator").function(0)


    def on_target(flight_time, coords):
        #TODO: Your code here
        if (math.fabs(coords[0] - 55.753960) < 0.45 and 
            math.fabs(coords[1] - 37.620393) < 0.45):
            return True
        if (math.fabs(coords[0] - 56.010569) < 0.45 and 
            math.fabs(coords[1] - 92.852545) < 0.45):
            return True
        return False


    def loop():
        flight_time = runtime.flight_time()

        location = navigator.location()
        coords = (location[0], location[1])

        if on_target(flight_time, coords):
            if not camera.enabled():
                camera.enable()
        else:
            if camera.enabled():
                camera.disable()

\end{minted}

\answerMath{$e = 0$, $i = 90^\circ$, $w = 0^\circ$, 
$\Omega = 352.9268137^\circ$, $а = 7626.308$ км, 
$M_0 = 353.8741780^\circ$.}