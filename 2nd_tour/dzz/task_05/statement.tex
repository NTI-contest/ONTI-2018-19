\assignementTitle{Программа полета}{42}

Миссией космического аппарата является съемка г. Москвы и г. Красноярска. При этом съемка каждого города должна происходить в период времени с 9:45 по 12:45 по местному времени. На орбите аппарат оказывается 12 апреля 2019 года в 00:00:00 по UTC.

Координатами центра г. Москвы считать: 55.753960,  37.620393
Координатами центра г. Красноярска считать: 56.010569, 92.852545

Программа полета всегда ориентирует спутник в надир. Запаса энергии аппарата хватает на 180 секунд работы камеры в течении первых трех суток полета. Периоды работы камеры в разное время суммируются.

Съемка считается произведенной, если аппарат находился над точкой, удаленной не более, чем на 50 км от центра города, и камера была включена.

Для решения задачи требуется:


\begin{enumerate}
    \item Определить и задать параметры орбиты, удовлетворяющей условиям съемки
    \item Задать программу управления полезной нагрузкой так, чтобы съемка производилась успешно в течение первых трех суток полета    
\end{enumerate}

\subsubsection*{Начальные данные}

Радиус Земли, м: 6371008.8

Гравитационный параметр, $\mu = G \cdot M$, м$^3/$с$^2$: $3.986004418 \cdot 10^{14}$

Средняя скорость вращения Земли, об/сут: $1.00273781191135448$

Начальный угол вращения Земли: $199.91^{\circ}$

\putImgWOCaption{13cm}{1}

В этой миссии на навигатор наложено ограничение: максимальная высота его работы - 3000 км над поверхностью Земли. При расчет орбиты учитывать, что орбита не подвержена прецессии, а влияние атмосферы на движение КА отсутствует.

\subsubsection*{Критерий оценки}

\begin{itemize}
    \item Аппарат не менее 1 раза в сутки смог провести съемку одного из городов в указанный период времени с отклонением от указанных координат центра не более чем на 25 км: 4 балла
    \item Аппарат не менее 1 раза смог провести съемку одного из городов в указанный период времени с отклонением от указанных координат не более чем на 50 км: 3 балла
\end{itemize}