\assignementTitle{Расчет орбиты}{30}

Наноспутник формата 3U массой 2,96 кг может быть размещен на произвольной орбите 1 апреля 2019 года в 21:00 (МСК). Задайте параметры орбиты через кеплеровы элементы так, чтобы 2-го, 3-го и 4-го апреля в 9:00 каждого дня аппарат прошел над Олимпийским парком Сочи.

Координатами Олимпийского парка считать GPS - координаты: 43.405398, 39.954817

При расчет орбиты учитывать, что орбита не подвержена прецессии, а влияние атмосферы на движение КА отсутствует.

\subsubsection*{Начальные данные}

Радиус Земли, м: $6371008.8$

Гравитационный параметр, $\mu = G \cdot M$, м$^3/$с$^2$: $3.986004418 \cdot 10^{14}$

Средняя скорость вращения Земли, об/сут: 1.00273781191135448

Начальный угол вращения Земли: 9$9.81^{\circ}$

\subsubsection*{Критерий оценки}

\begin{itemize}
    \item Аппарат в любой из дней проходит с отклонением до 100 км от цели и до 60 мин от нужного времени - 2 балла;
    \item Аппарат в любой из дней проходит с отклонением до 100 км от цели и до 15 мин от нужного времени - 3 балла;
    \item Аппарат в любой из дней проходит с отклонением до 25 км от цели и до 60 мин от нужного времени - 2 балла;
    \item Аппарат в любой из дней проходит с отклонением до 25 км от цели и до 15 мин от нужного времени - 3 балла;
\end{itemize}

При выполнении нескольких условий баллы суммируются.
