\assignementTitle{Расчет орбиты}{30}{}

Наноспутник формата 3U массой 2,96 кг может быть размещен на произвольной орбите 1 апреля 2019 года в 21:00 (МСК). Задайте параметры орбиты через кеплеровы элементы так, чтобы 2-го, 3-го и 4-го апреля в 9:00 каждого дня аппарат прошел над Олимпийским парком Сочи.

Координатами Олимпийского парка считать GPS - координаты: 43.405398, 39.954817

При расчет орбиты учитывать, что орбита не подвержена прецессии, а влияние атмосферы на движение КА отсутствует.

\subsubsection*{Начальные данные}

Радиус Земли, м: $6371008.8$

Гравитационный параметр, $\mu = G \cdot M$, м$^3/$с$^2$: $3.986004418 \cdot 10^{14}$

Средняя скорость вращения Земли, об/сут: 1.00273781191135448

Начальный угол вращения Земли: 9$9.81^{\circ}$

\subsubsection*{Критерий оценки}

\begin{itemize}
    \item Аппарат в любой из дней проходит с отклонением до 100 км от цели и до 60 мин от нужного времени - 2 балла;
    \item Аппарат в любой из дней проходит с отклонением до 100 км от цели и до 15 мин от нужного времени - 3 балла;
    \item Аппарат в любой из дней проходит с отклонением до 25 км от цели и до 60 мин от нужного времени - 2 балла;
    \item Аппарат в любой из дней проходит с отклонением до 25 км от цели и до 15 мин от нужного времени - 3 балла;
\end{itemize}

При выполнении нескольких условий баллы суммируются.
\solutionSection
Сразу определимся с типом нашей орбиты: мы можем выбрать круговую полярную орбиту, так как в условии сказано, что орбита не подвержена прецессии. Это значительно упрощает задачу. 
Наклонение полярной орбиты составляет 90 градусов, а эксцентриситет окружности равен нулю. Так как орбита круговая, мы можем задать аргумент перицентра, также равный нулю, чтобы перицентр оказался на экваторе.

$i = 90^\circle$
$e = 0$
$ω = 0$

Для простоты мы взяли за плоскость отсчёта плоскость экватора Земли, но элементы орбит считаются в небесной системе координат. И, если совмещением плоскости отсчёта с плоскостью экватора мы добились того, что оси Z небесной и земной систем отсчёта совпадают, то всё равно остаётся вращение Земли вокруг своей оси, а значит, оси X и Y вращаются относительно небесной системы отсчёта. 

Ось Х в земной системе отсчёта лежит в плоскости нулевого меридиана, а в условии указан начальный угол вращения, равный 99,81° - это есть угол между направлениями на нулевой меридиан Земли и на точку весеннего равноденствия, и это понадобится нам для вычисления долготы восходящего узла. 
Долгота Олимпийского парка, по сути, и есть долготой восходящего узла в земной системе отсчёта, нам осталось учесть начальный угол вращения Земли и её скорость вращения.
Аппарат начинает своё движение 1 апреля в 21:00, а над Олимпийским парком должен пролететь 2 апреля в 9:00, то есть, пройдёт 12 часов. Переведём скорость вращения Земли из оборотов в сутки в градусы в сутки. 12 часов - это половина суток, значит угол вращения Земли в 9:00 2 апреля будет составлять:

$$α = 99,81^\circle + frac{1,00273781191135448 \cdot 360^\circle} {2} = 280,3028061440438064^\circle$$

Тогда долгота восходящего узла аппарата будет равна сумме угла вращения Земли 2 апреля в 9:00 и долготы Олимпийского парка:

$$Ω = α + 39,954817^\circle = 320,2576231440438064^\circle$$

Если долгота восходящего узла превысила бы по модулю 360 градусов, то следовало бы отнять или прибавить 360 градусов от полученного числа.

Давайте теперь определимся с большой полуосью орбиты. Её можно вычислить через период по известной из небесной механики формуле:
$$a=\sqrt[3]{(μfrac{T}{2\pi}^2)}$$
Из-за того, что прецессии орбит нет, мы можем утверждать, что каждые звёздные сутки аппарат будет пролетать через одну и ту же точку. Поэтому мы можем выбрать период вращения аппарата по орбите таким образом, чтобы в звёздные сутки укладывалось целое число периодов, например, 15. Известно, что в звёздных сутках 86164 секунды, тогда:
$$T = frac{86164 c} {15} = 5744,2(6) c$$
Считаем большую полуось:
$$a=\sqrt[3]{3.9860044181014 (frac{5744.2(6)}{2\pi})^2 }=6932.3808 км$$

Осталось найти среднюю аномалию. Как вы помните, средняя аномалия отсчитывается от экватора, и наш спутник должен начать своё движение в такой точке, чтобы 12 часов оказаться на широте Олимпийского парка. Сначала найдём количество витков, которое совершит аппарат за 12 часов, не забываем перевести часы в секунды:
$$N = 12 * 60 * 60 / T = 7,5205422218095724432477600854185$$

За каждый виток аппарат совершит один оборот в 360 градусов, поэтому будем учитывать только дробную часть, чтобы найти угол, на который сместится аппарат за 12 часов - от своей начальной точки до целевой:

$$φ = 360^\circle \cdot 0,5205422218095724432477600854185 = 187,39519995144607956919363075066^\circle$$

Итак, чтобы оказаться в нужной точке через 12 часов, необходимо вычесть из широты Олимпийского парка тот угол, на который сместится аппарат на своей орбите через эти 12 часов:

$$M0 = 43,405398^\circle -  187,39519985144607956919363075066^\circle = -143,989802^\circle$$

Средняя аномалия орбиты в итоге: $M0 = 216,01^\circle$

