\assignementTitle{Управление полезной нагрузкой}{9}

Студенческий спутник ДЗЗ размещен на орбите, заданной следующими кеплеровыми элементами:

\begin{itemize}
    \item большая полуось: 6838.5 км,
    \item наклонение: 97.2675 градусов,
    \item аргумент перицентра:  121.0227 градусов
    \item эксцентриситет:  0.0010868
    \item долгота восходящего узла:  42.0376 градусов
    \item средняя аномалия: 301.0 градусов
\end{itemize}

Аппарат начинает движение по Орбите 12 декабря 2018 года в 21:00 UTC. 
Программа полета ориентирует спутник всегда в надир.

Необходимо ввести программу управления полезной нагрузкой согласно предоставленному шаблону так, 
чтобы в течении первых суток снять следующие точки:

\begin{enumerate}
    \item Самара (Самарская область, Россия) (GPS - координаты: 53.1774, 50.1165)
    \item Петергоф (Ленинградская область, Россия) (GPS - координаты: 59.8783, 29.9000)
    \item Топика (Канзас, США)  (GPS - координаты: 39.0256, -95.6834)
\end{enumerate}

Запаса энергии аппарата хватает на 60 секунд работы камеры. Периоды работы камеры в разное время суммируются.

Съемка считается произведенной, если аппарат находился над точкой, удаленной не более, чем на 5 км от заданной, 
и камера была включена.

Радиус Земли, м: 6371008.8

Гравитационный параметр,$ \mu = G \cdot M$, м$^3$/с$^2$: $3.986004418 \cdot 10^{14}$

Средняя скорость вращения Земли, об/сут: 1.00273781191135448

Начальный угол вращения Земли: $36.51^{\circ}$

\subsubsection*{Начальные данные}

Постоянная планка, h, Дж $\cdot$ с: $6.63 \cdot 10^{-34}$

Скорость света, c, м/с: $3 \cdot 108$

Время выдержки, мс: 0.53

\putImgWOCaption{13cm}{1}

\subsubsection*{Критерий оценки}

\begin{itemize}
    \item Аппарат не менее 1 раза находился на расстоянии, не более чем на 5 км удаленном от одной из точек съемки, и камера была включена - 2 балла
    \item Аппарат не менее 1 раза находился на расстоянии, не более чем на 15 км удаленном от одной из точек, и камера была включена - 1 балл
\end{itemize}

При выполнении нескольких условий баллы суммируются.
