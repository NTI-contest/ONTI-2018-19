\assignementTitle{Предначертанный отказ}{4}{}

Обратите внимание, что сама по себе задача несложная, но её условие нужно читать очень внимательно!

\subsection*{Послесловие}

Эта задача — модель энергосистемы, которая демонстрирует один из возможных путей коллапса энергетики. Для проектирования развития энергосистем очень важно такие пути знать, чтобы держаться от них подальше.

Модель в этой задаче нарочно простая, чтобы условие задачи влезало на полстраницы; в реальности и энергетические компании устроены по-другому, и потребители ведут себя намного сложнее, и их потребление есть величина не постоянная, и т.д.

Любая модель отражает лишь малую часть моделируемой системы, и даже если какую-то черту системы включить в модель технически легко, нужно ещё и утвердительно ответить на вопрос «Важна ли эта черта для интересующих нас эффектов»? И если ответ содержит сомнение, то влияние этой черты на модель нужно изучать отдельно. Особенно если вас интересуют качественные характеристики модели, а черта влияет только на количественные. Чем модель будет проще, тем легче её будет изучать и разбирать её поведение по косточкам. Развитие модели — это не доведение её до похожести на настоящую систему, а изучение того, что именно приводит к возникновению или исчезновению эффекта, и какие черты настоящей системы влияют на этот эффект в худшую или лучшую сторону.

Имеется энергосистема, в которой находится известное число (100) потребителей электроэнергии.
Мощность каждого потребителя известна и постоянна в течение всего периода моделирования.

Каждый месяц потребитель может сделать одно из следующих действий:

\begin{enumerate}
    \item Установить солнечную панель на 2 кВт мощности за 10000 рублей.
    \item Установить солнечную панель на 6 кВт мощности за 25000 рублей.
    \item Установить дешёвый энергетический порт на 5 кВт установленной мощности за 15000 рублей.
    \item Установить дорогой энергетический порт на 5 кВт установленной мощности за 20000 рублей.    
\end{enumerate}

\begin{itemize}
    \item С помощью энергетического порта потребитель может продавать излишек своей электроэнергии другим потребителям за $50\%$ цены от цены сетевой компании.
    \item Солнечные панели нужно считать постоянными источниками мощности (как будто их график генерации сглажен иделальными накопителями энергии).
    \item Пользователь принимает действие, если при текущей цене сетевой компании на электроэнергию покупка окупится не более чем через три года. Если пользователь может предпринять несколько действий, он предпринимает то из них, которое окупается быстрее (а лучше ещё и стоит меньше).
    \item Сетевая компания несёт фиксированные издержки на содержание энергосистемы в 1 млн. рублей в месяц (или 12 миллионов за 365 дней), плюс 1,5 рублей за каждый произведённый кВт$\cdot$час.
    \item Тариф на электроэнергию сетевая компания устанавливает таким, чтобы при текущем уровне потребления сетевая компания имела прибыль в $5\%$.
    \item Новый тариф устанавливается каждый год (раз в 12 месяцев, в том числе и в самом начале моделирования).
    \item Если в системе установленная мощность дешёвых энергетических портов окажется больше $30\%$ от суммарного потребления мощности из сети (суммарной мощности, которую объекты потребляют от сетевой компании и других потребителей, на не из собственных солнечных панелей), в ней произойдёт авария из-за низкого качества передаваемой ими электроэнергии.
    \item Если установка пользователем дешёвого порта приведёт превышению установленной мощности дешёвых энергетических портов в сети показателя в $10\%$ от суммарного потребления мощности из сети, сетевая компания запретит установку, и пользователь придётся исключить это действие из рассмотрения.
    \item Потребители принимают решения независимо друг от друга в том порядке, в котором они перечислены в исходных данных.
    \item Компания устанавливает цену раньше, чем потребители принимают решения о покупках.
\end{itemize}

\inputfmtSection

Мощности потребителей в кВт$\cdot$ч, массив из 100 чисел с плавающей точкой, от 5 до 500.

\outputfmtSection

Количество месяцев до аварии, целое число, например, 97.

\subsubsection*{Пример входных данных}

\noindent[91.05173216589283, 346.59744306410516, 79.340611380049, 6.301441137920902, \\ 
34.63321051774173, 283.7442062865518, 462.85639563921933, 18.35969426088177, \\
114.42284118915013, 348.2511151684352, $\cdots$ ]

\subsubsection*{Пример выходных данных}

\noindent{97}

\subsubsection*{Ограничения вычислительных ресурсов}

Время выполнения программы на сервере не более 15 секунд.

Требуемая память на сервере не более 256 мегабайт.

\solutionSection

Для решения достаточно внимательно прочитать условие и реализовать описанный в нём алгоритм. Задача не требует применения специфических знаний, но имеет значительную когнитивную трудность в виде высоких требований к внимательности. Ограничения вычислительных ресурсов влияют только при выборе катастрофически неэффективных алгоритмов, например, самостоятельной сортировки вместо использования встроенных в язык функций.

\includeSolutionIfExistsByPath{2nd_tour/ies/task_02}