\assignementTitle{Умная сетка}{2.2}{}

В жилом доме построена инновационная система энергопотребления, где все квартиры соединены в 
общую энергосеть и могут обмениваться электроэнергией (купленной, полученной с солнечных батарей, 
накопленной в энергонакопителе, \linebreak и т.п.) таким образом, что каждая квартира получает необходимое энергоснабжение. Даже в том случае, когда какой-то квартире энергии не хватает, всегда можно взять энергию из внешней энергосистемы.

Пусть каждая квартира имеет на данный момент излишек или недостаток энергии.

Если у квартиры излишек электроэнергии, то она раздаёт энергию соседям в определённом владельцами квартиры соотношении, выражаемом максимальной долей излишка, которую квартира может передать каждому из соседей. Даже если кто-то из соседей не может принять переданный им излишек (им уже хватает энергии), то он всё равно не перераспределяется.

Если у квартиры недостаток электроэнергии, то она берёт энергию из выделенных её соседями долей, но в пределах, ограниченных соседями, таковы технические ограничения. Если энергии соседей тоже недостаточно, она берёт энергию из внешней энергосети.

По заданным излишкам и недостаткам, а также настройкам соотношений необходимо определить, сколько энергии придётся докупить с внешней сети, чтобы дорогие соседи смогли дальше пить чай и смотреть новости.

\inputfmtSection

Два массива. В первом находятся текущие избытки или недостатки энергии у каждого из соседей. Во втором — установленные доли излишков, передаваемых соседям. Если этот массив назвать foo, то foo[3][4] == 0.5 означает, что сосед в квартире номер 3 передаст соседу в квартире номер 4 не больше половины своего избытка энергии (даже если остальной избыток передать будет некому).

Сумма значений по первому индексу массива всегда равна 1.

Размеры массивов согласованы: если первый имеет размер 100, то второй — 100 на 100.

Число квартир находится в диапазоне от 50 до 150.

\outputfmtSection

Число с плавающей точкой.

\subsubsection*{Пример входных данных}
\noindent([25.764437694733104, -19.65574253924039, 23.252448869230683, -10.338138947636265, \\
12.335993891889927, 6.486647051427877, -0.9219876347859852, $\cdots$], [[0.16890061330053335, \\
0.18540184620782904, $\cdots$]])

\subsubsection*{Пример выходных данных}

97.7
(не является ответом к примеру входных данных)

\subsubsection*{Ограничения вычислительных ресурсов}

Время выполнения программы на сервере не более 15 секунд.

Требуемая память на сервере не более 256 мегабайт.

\solutionSection

Для решения задачи достаточно однократно вычислить, сколько мощности соседи передают друг другу, а затем просуммировать недостатки (если есть) по всем квартирам. Ограничения вычислительных ресурсов несущественны.



\includeSolutionIfExistsByPath{2nd_tour/ies/task_07}