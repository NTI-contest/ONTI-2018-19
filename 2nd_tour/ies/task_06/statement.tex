\assignementTitle{Золотое солнце}{1}

Вам даны почасовые данные за 30 дней по яркости солнца (в килолюксах) и потребляемой мощности (в кВт) для некоторого потребителя (всего 1440 чисел).

У потребителя установлена большая солнечная батарея, для которой зависимость мощности от яркости солнца линейна и составляет 1,5 кВт при 10 килолюксах.

Нужно выяснить, сколько денег пользователь сэкономил за счёт солнечной батареи за рассматриваемый месяц.

В течение часа яркость солнца нужно считать постоянной.

Стоимость электроэнергии 10 руб. за кВт$\cdot$час.

Отдавать излишки энергии в сеть пользователь не может.

\subsubsection*{Пример выходных данных}

4081.5

\subsubsection*{Пример входных данных}
Это просто пара из двух массивов по 30*24 чисел каждый. Первый массив — данные по яркости солнца, в клк, второй — по потреблению, в кВт.

\noindent([0.0, 0.0, 0.0, 0.0, 0.0, 0.0, 1.3457030373754966, 3.0065015553528815, 4.26458504488992, \linebreak 
6.337972841901577, 6.398630888429304, 10.645502850939069, 13.663080234839168, \\
14.997035523241964, 15.265556118680628, 13.238674356373597, 16.509038655033837, \\
13.221606012902962, 11.691021062908755, 11.265662725768626, 6.511092267415828, \\
4.822169622939845, 3.1900099497706904, 0.7378545946770569, $\cdots$ ])

%\includeSolutionIfExistsByPath{2nd_tour/ies/task_06}