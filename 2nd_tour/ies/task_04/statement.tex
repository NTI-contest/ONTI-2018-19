\assignementTitle{Светлое наследие Теслы}{2}

Никола Тесла мечтал передавать энергию по воздуху с помощью особых башен (посмотрите на башню Ворденклиф, чтобы было легче представить), создающих высокое напряжение, которое затем передаётся без проводов, прямо по воздуху в приёмники.

Вам поручены работы по проекту строительства подобной башни. Она должна обеспечивать электроэнергией небольшое поселение, каждое из зданий которого будет оборудовано приёмником. Эффективная площадь антенны каждого приёмника 2 квадратных метра (это площадь, которую бы имел идеальный приёмник для приёма той же мощности, что и реальный), и для простоты будем считать, что она не зависит от расстояния до приёмника. Планируемая высота башни обеспечивает подъём на 50 м над приёмниками.

Известны координаты будущих приёмников на плоскости (для удобства переведённые в локальную декартову систему). Нужно определить координаты для постройки башни, выполняющие следующие условия: 

\begin{itemize}
    \item Мощность излучателя минимальна.
    \item Все приёмники принимают не менее 1 кВт.
\end{itemize}

Принимаемая приёмниками мощность обратно пропорциональна квадрату расстояния до излучателя.

В этой части нужно вывести только координаты башни.

За эту часть начисляется 1,3 балла.

\inputfmtSection

Нна первой строке целое число приёмников $N$, затем $N$ строк с вещественными числами 
$x_i$ и $y_i$ через пробел. Это координаты приёмников. Единица измерения — метр.

\outputfmtSection

Два вещественных числа $x_c$, $y_c$ через пробел — координаты башни.

Ответ считается верным, если вычисленные координаты отдалены от эталонных менее, чем на 1 условный метр

Число приёмников изменяется от 4 до 20.

Координаты изменяются от -1000 до 1000.

\begin{myverbbox}[\small]{\vinput}
    4
    1.2 -1.3
    -501.33 1281.41
    -1 -1
    1 1
\end{myverbbox}
\begin{myverbbox}[\small]{\voutput}
    0.0001 10.000
\end{myverbbox}
\inputoutputTable


В этой части нужно вывести только мощность башни.

За эту часть начисляется 0,7 балла.

\inputfmtSection

На первой строке целое число приёмников $N$, затем $N$ строк с вещественными 
числами $x_i$ и $y_i$ через пробел. Это координаты приёмников. Единица измерения — метр.

\outputfmtSection

Вещественное число, мощность излучателя в кВт.

Ответ считается верным, если вычисленная мощность приёмника отличается от эталонной менее, чем на 10 Вт.

Число приёмников изменяется от 4 до 20.

Координаты изменяются от -1000 до 1000.

\begin{myverbbox}[\small]{\vinput}
    4
    1.2 -1.3
    -501.33 1281.41
    -1 -1
    1 1
\end{myverbbox}
\begin{myverbbox}[\small]{\voutput}
    250200.002
\end{myverbbox}
\inputoutputTable

%\includeSolutionIfExistsByPath{2nd_tour/ies/task_04}