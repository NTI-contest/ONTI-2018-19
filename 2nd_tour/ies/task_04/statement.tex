\assignementTitle{Светлое наследие Теслы}{2}{}

Никола Тесла мечтал передавать энергию по воздуху с помощью особых башен (посмотрите на башню Ворденклиф, чтобы было легче представить), создающих высокое напряжение, которое затем передаётся без проводов, прямо по воздуху в приёмники.

Вам поручены работы по проекту строительства подобной башни. Она должна обеспечивать электроэнергией небольшое поселение, каждое из зданий которого будет оборудовано приёмником. Эффективная площадь антенны каждого приёмника 2 квадратных метра (это площадь, которую бы имел идеальный приёмник для приёма той же мощности, что и реальный), и для простоты будем считать, что она не зависит от расстояния до приёмника. Планируемая высота башни обеспечивает подъём на 50 м над приёмниками.

Известны координаты будущих приёмников на плоскости (для удобства переведённые в локальную декартову систему). Нужно определить координаты для постройки башни, выполняющие следующие условия: 

\begin{itemize}
    \item Мощность излучателя минимальна.
    \item Все приёмники принимают не менее 1 кВт.
\end{itemize}

Принимаемая приёмниками мощность обратно пропорциональна квадрату расстояния до излучателя.

\subsubsection*{Часть 1}

В этой части нужно вывести только координаты башни.

За эту часть начисляется 1,3 балла.

\subsubsection*{Часть 2}

В этой части нужно вывести только мощность башни.

За эту часть начисляется 0,7 балла.

\inputfmtSection

На первой строке целое число приёмников $N$, затем $N$ строк с вещественными числами 
$x_i$ и $y_i$ через пробел. Это координаты приёмников. Единица измерения — метр.

\outputfmtSection

\subsubsection*{Часть 1}

Два вещественных числа $x_c$, $y_c$ через пробел — координаты башни.

Ответ считается верным, если вычисленные координаты отдалены от эталонных менее, чем на 1 условный метр

Число приёмников изменяется от 4 до 20.

Координаты изменяются от -1000 до 1000.

\begin{myverbbox}[\small]{\vinput}
    4
    1.2 -1.3
    -501.33 1281.41
    -1 -1
    1 1
\end{myverbbox}
\begin{myverbbox}[\small]{\voutput}
    0.0001 10.000
\end{myverbbox}
\inputoutputTable

\subsubsection*{Часть 2} 

Вещественное число, мощность излучателя в кВт.

\begin{myverbbox}[\small]{\vinput}
    4
    1.2 -1.3
    -501.33 1281.41
    -1 -1
    1 1
\end{myverbbox}
\begin{myverbbox}[\small]{\voutput}
    250200.002
\end{myverbbox}
\inputoutputTable

\subsection*{Ограничения вычислительных ресурсов}

Время выполнения программы на сервере не более 15 секунд.

Требуемая память на сервере не более 256 мегабайт.


\solutionSection

Задачу можно решить как методом скорейшего спуска, так и алгебраически: достаточно, например, перебрать все тройки точек из множества входящих в выпуклую границу всех точке. Для каждой из трёх точек нужно проверить, что описанная вокруг их треугольника окружность удовлетворяет всем ограничениям условия, и найти минимальную. Её центр является ответом на первую часть. Для получения ответа ко второй части нужно найти расстояние от передатчика до самого крайнего приёмника — это гипотенуза прямоугольного треугольника с катетами, равными 50 метрам и радиусу найденной ранее окружности. Далее нужно найти площадь сферы с радиусом в найденное расстояние от передатчика до самого удалённого приёмника, и исходя из того, что 2 квадратных метра этой площади несут мощность в 1кВт, найти мощность, распределённую по всей сфере. Это и есть ответ второй части. Ограничения вычислительных ресурсов несущественны.

\codeExample

\subsubsection*{Часть 1}

Ниже представлено решение на языке Python3

\inputminted[fontsize=\footnotesize, linenos]{python}{2nd_tour/ies/task_04/source_1.py}

\subsubsection*{Часть 2}

Ниже представлено решение на языке Python3

\inputminted[fontsize=\footnotesize, linenos]{python}{2nd_tour/ies/task_04/source_2.py}
