\assignementTitle{Мысленный аукцион}{10}{}

\subsubsection*{Голандский аукцион}

Существует много схем аукционов. Как правило, они эквивалентны друг другу, но на практике отличия 
могут быть существенными. В схеме «голландского аукциона» когда объявляется лот, на специальном 
табло загорается цена, которая со временем снижается (и довольно быстро). У участников аукциона 
есть кнопка, нажав которую, они приобретут лот по той цене, которая в момент нажатия высвечивается 
на табло. Преимущества такой схемы аукциона в том, что розыгрыши идут очень быстро и за фиксированное 
время, и можно даже разыгрывать несколько лотов одновременно. Недостатком является то, что у участников 
очень мало времени на принятие решения, поэтому эта схема подходит в основном для большого числа 
однородных, но различных лотов. Например, цветов.

\putImgWOCaption{7cm}{1}

\subsubsection*{Вероятностные распределения}

Вероятностные распределения — это характеристика случайной величины, по сути являющаяся её определением, которая говорит о том, какой исход случайной величины с какой вероятностью произойдёт.

Рассмотрим, например, случайную величину «число очков на шестигранном игральном кубике». Возможные исходы — 1, 2, 3, 4, 5 и 6; вероятности всех исходов одинаковы и равны 1/6. В виде графика это распределение будет выглядеть вот так:

\putImgWOCaption{10cm}{2}

Рассмотрим в качестве второго примера не идеальный, а реальный кубик, такой, как на рисунке:

\putImgWOCaption{7cm}{3}

Из-за того, что очки на нём нанесены в виде больших углублений, у него смещён центр тяжести, и он имеет такое распределение вероятностей:

\putImgWOCaption{10cm}{4}

Вероятности исходов у такого кубика равны (по результатам 100 бросков): \linebreak для 1 — 0,24, для 2 — 0,18, для 3 — 0,17, для 4 — 0,17, для 5 — 0,15, \linebreak для 6 — 0,10.

Если мы вычислим среднее ожидаемое число очков на каждом их этих кубиков (математические ожидания соответствующих случайных величин), то получим следующее.

Для правильного кубика это 

$1 \times { 1 \over 6 } + 2 \times { 1 \over 6 } + 3 \times { 1 \over 6 } + 4 \times { 1 \over 6 }+5  \times { 1 \over 6 } +6 \times { 1 \over 6 } = 3,5$ 

Для нашего неправильного кубика это 

$1 \times 0,24 + 2 \times 0,18 + 3 \times 0,17 + 4 \times 0,17 + 5\times 0,15 + 6\times 0,10 = 3,08$  

\subsubsection*{Условие задачи}

У вас есть три корзинки с тюльпанами, которые вы выставляете на голландский аукцион со стартовой ценой в 49 и шагом 1.

В аукционе участвует всего 5 участников, и у вас есть предположения о ценах, которую каждый участник готов заплатить за каждый лот.

Всего у вас, естественно, 15 таких предположений, и каждое из них представляет собой вероятностное распределение на ценах от 0 до 49.

Если два или более участников объявят одинаковую цену, то кто-то из них нажмёт кнопку чуть раньше.

Вероятность этого одинакова для всех участников.

Ноль~— допустимая цена, и если все выбрали такую ставку, то кому-то корзинка достанется бесплатно.

Каждый участник купит не больше одной корзинки.

Исходя из ваших предположений об участниках, найдите, на сколько процентов можно поднять выручку от аукциона изменением порядка лотов по сравнению с начальным.

\inputfmtSection

Трёхмерный массив, первый индекс которого — участник аукциона (его номер), второй — номер корзинки с цветами (в изначальной расстановке), третий — цена. Значение, которое хранится в массиве — вероятность того, что этот участник решит купить эту корзинку именно по этой цене.

\outputfmtSection

Число с плавающей точкой, например, 0.012182432410846942.

Обратите внимание, что ответ нужно выдать в процентах.

Требуемая точность — хотя бы 1e-10.

\subsubsection*{Пример входных данных}

\noindent{[[[0.0, 0.0, 0.00037280466184606246, 0.0006687013716653295, 0.0014910532789078324, \linebreak 0.0022033598533922837, 0.0033672906128430487, 0.004827078289717003, \\
0.006056792296892899, 0.007235026230088179, 0.00894220596189969, $\cdots$ ]]]}

\subsubsection*{Пример выходных данных}

\noindent{0.012182432410846942}

\subsubsection*{Ограничения вычислительных ресурсов}

Время выполнения программы на сервере не более 30 секунд.

Требуемая память на сервере не более 256 мегабайт.

\solutionSection

Для решения задачи нужно выделить и перебрать вероятности всех элементарных событий — действительных наборов ставок. В паре с фиксированной последовательностью лотов это позволяет получить вероятностное распределение результатов аукциона (при заданной последовательности лотов) и вычислить его математическое ожидание. Перебрав все 120 последовательностей, несложно получить ответ. Ограничения вычислительных ресурсов.

\includeSolutionIfExistsByPath{2nd_tour/ies/task_01}