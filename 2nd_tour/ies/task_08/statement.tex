\assignementTitle{PageRank на минималках}{2.4}{}

Пользователь читает Википедию по следующему алгоритму:

\begin{enumerate}
    \item Вначале у него открыта вкладка браузера с главной страницей.
    \item Пользователь читает страницу на текущей вкладке и начисляет ей 10 очков рейтинга.
    \item Он открывает все ссылки на страницы внутри Википедии, которые есть на странице в текущей вкладке браузера (кроме тех, что ведут на неё саму). Если пользователь натыкается на ссылку, которая уже открыта в какой-то вкладке, он не открывает эту ссылку, но странице, на которую она ведёт, начисляет 1 очко рейтинга. Ссылки открываются в том порядке, в котором они перечислены во входных данных. Затем он закрывает текущую вкладку.
    \item Если открытых вкладок больше нет, он возвращается на главную страницу.
    \item Если число открытых вкладок < 50, пользователь закрывает текущую вкладку и переходит к п.2
    \item Если открыто >= 50 вкладок, то пользователь начисляет каждой вкладке, кроме последней, 5 очков и закрывает их.
\end{enumerate}

Затем он переходит к п.2

По заданной структуре графа ссылок Википедии найдите страницу, которой пользователь присвоит 
наибольший рейтинг.

Если даже вы немного ошибётесь и укажете страницу, рейтинг которой отличается от рейтинга 
наилучшей меньше, чем на $1\%$, ответ будет засчитан как правильный.

\inputfmtSection

В «Википедии» 500 страниц и примерно 11000 ссылок между ними (страницы могут иногда ссылаться сами на себя).

Каждая страница имеет номер от 0 до 499. Главная страница имеет номер 0.

Каждая ссылка представлена парой $(a,b)$, где $a$ — номер страницы, на которой ссылка расположена, 
а $b$ — номер страницы, на которую она ведёт. Например, $(0,1)$ — ссылка с главной страницы на страницу 
с номером 1.

\outputfmtSection

Номер самой «рейтинговой» страницы, например, 404.

\subsubsection*{Пример входных данных}

\noindent[(0, 200), (0, 399), (0, 499), (0, 303), (0, 104), (0, 90), (0, 60), (0, 410), 
(0, 272), (0, 9), (336, 109), (387, 139), (78, 77), (480, 225), (273, 189), (58, 395), 
$\cdots$ ]

\subsubsection*{Пример выходных данных}

404

\subsubsection*{Ограничения вычислительных ресурсов}

Время выполнения программы на сервере не более 15 секунд.

Требуемая память на сервере не более 256 мегабайт.

\solutionSection

В задаче описан алгоритм, который нужно реализовать. Нетривиальным является то, что в нём отсутствует критерий остановки. Его нужно подобрать самостоятельно, например, «на глаз» подобрать константу числа итераций алгоритма, после которых ранжировка лидеров не меняется. Для этого можно использовать самостоятельно сгенерированные примеры. Ограничения вычислительных ресурсов достаточно слабы, чтобы проходили решения, в которых критерий остановки выбран очень консервативно (допускаются решения в 10 раз медленнее авторского решения).

\includeSolutionIfExistsByPath{2nd_tour/ies/task_08}