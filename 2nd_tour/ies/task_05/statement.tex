\assignementTitle{Принудительная добровольность}{5}{}

У некоторой энергетической компании есть «головная боль» в виде загородного посёлка, сети в котором 
очень сильно перегружены, что приводит к большим издержкам на их обслуживание. Модернизация 
сетей в посёлке обойдётся очень дорого и в обозримом будущем не окупится. Компания пытается улучшить 
ситуацию, предложив потребителям в посёлке такие тарифы, при которых и пиковое потребление электроэнергии 
снизится, и потребители с радостью на них перейдут, потому что станут платить меньше. Вам предстоит 
рассчитать такие тарифы.

У вас имеется модель «типичного» потребителя в доме данного посёлка, у которого имеется «типичный» 
график почасового потребления электроэнергии.Часть потребления постоянна и от тарифа не зависит, 
а часть может изменяться. В нашей модели, для простоты, изменение состоит в том, что график 
изменяемого потребления может смещаться во времени на произвольное число часов.

Например, если графики потребления выглядят так:

\putImgWOCaption{12cm}{1}
...и для суммарного потребления — так (здесь синий и красный графики с картинки выше сложены):

\putImgWOCaption{12cm}{2}
...то после смещения переменной составляющей потребления на 5 часов, эти графики превратятся в такие:

\putImgWOCaption{12cm}{3}

\putImgWOCaption{12cm}{4}

Владелец дома платит за электроэнергию по двухзонному тарифу: с 12 часов (включительно) до 16 
(не включительно) и с 23 (включительно) до 6 (не включительно) часов цена составляет 3 р./кВт$\cdot$ч, в 
остальное время — 6. По данным из примера (на следующей странице) можно подсчитать, что модельный 
потребитель платит за электроэнергию 45 546,89 р./год (за 365 дней).

\begin{itemize}
    \item Каждый кВт пиковой мощности (максимальной мощности за день) приносит компании 40 000 р. 
    убытков в год.
    \item Потребитель без проблем согласится перейти на новый тариф только в том случае, если после изменения графика переменной части потребления он будет платить хотя бы на 500 р./год меньше, чем он платит сейчас.
    \item Если при новых тарифах прибыль компании уменьшится, она на это изменение не пойдёт.
    \item Если по новому тарифу для потребителя без изменения графика потребления годовая стоимость энергии увеличится менее, чем на 1000 рублей, он не будет изменять своего поведения.
    \item Если потребитель может изменить график потребления несколькими способами, он сделает это наиболее удобным для себя способом; поскольку этот способ неизвестен, приходится исходить из того, что он будет вести себя наименее выгодным для энергокомпании образом.
\end{itemize}

Какие цены должна предложить энергокомпания, чтобы максимально увеличить свою прибыль? Если удовлетворяющего заданным ограничениям решения нет, то нужно вывести значения действующего тарифа. 
Решение состоит из двух чисел — цен за кВт$\cdot$час ночью-днём и утром-вечером.

Задача может иметь несколько правильных ответов и ещё больше похожих на правильные.
Для решения задачи вам нужно привести любой правильный ответ.

\inputfmtSection

Два массива, каждый из 24 чисел (почасовые значения потребляемой энергии \linebreak в кВт$\cdot$ч за сутки). 

Первый массив — постоянная составляющая потребления, второй — переменная. Отсчёт идёт с полуночи.

\outputfmtSection

Пара чисел — тарифы ночью-днём и утром-вечером в копейках за кВт$\cdot$ч.

\subsubsection*{Пример ответа}

Для тарифов в 2р. 53коп. и 8р. 47коп ответ выглядит вот так (это не ответ к примеру входных данных):

(253,847)

\subsubsection*{Пример входных данных}
\noindent([1.2532064880926013, 0.9277330171025241, 0.3830082504508084, 0.310412919612098, \\
0.375401744425605, 0.46924600370281877, 0.43025390820572046, $\cdots$])

%\includeSolutionIfExistsByPath{2nd_tour/ies/task_05}