\assignementTitle{}{35}{}

Есть анимация движения роя из $N$ коптеров, для каждого из которых в любой момент времени известны координаты $x$, $y$, $z$. Нумерация кадров начинается с 0, количество кадров анимации равно $M$. Необходимо определить не сближаются ли какие-либо 2 дрона более, чем на 2 метра, и, если сближаются, то определить расстояние максимального сближения и кадр, на котором это сближение произошло. Если критического сближения не произошло, вернуть значение '-1'. Перемещения дронов между точками анимации считать мгновенным (без учета траектории полета между точками; считать, что коптеры мгновенно оказываются в конечных точках).

\inputfmtSection

Во входных данных указаны значения $N$ и $M$, два перевода строки, затем списки позиций дронов для каждого кадра анимации. Списки для каждого из дронов разделены двумя переводами строки.

\outputfmtSection

Необходимо вывести минимальное значение расстояния и соответствующий кадр анимации, либо значение -1.

Погрешности, которые могут возникнуть в числах с плавающей точкой, учитываются в проверке (предусмотрен допуск).

\sampleTitle{1}

\begin{myverbbox}[\small]{\vinput}
    23 16

    -0.645 1.541 1.762
    -2.018 2.315 2.882
    -2.493 1.866 3.234
    ...
\end{myverbbox}
\begin{myverbbox}[\small]{\voutput}
    0.7286206145862202 0
\end{myverbbox}
\inputoutputTable

Необходимо вывести минимальное значение расстояния и соответствующий кадр анимации, либо значение -1 (кадры анимации считаются с 0).

Погрешности, которые могут возникнуть в числах с плавающей точкой, учитываются в проверке (предусмотрен допуск).

\solutionSection

Для начала вспомним формулу из геометрии, позволяющую вычислить наименьшее расстояние между двумя точками в пространстве 
$$s = \sqrt{(x-x_0)^2+(y-y_0)^2+(z-z_0)^2} \qquad (1)$$
На языке Python её можно описать следующим образом:
$$math.sqrt((x-x0)**2+(y-y0)**2+(z-z0)**2)$$

Для того, чтобы найти максимальное расстояние, на которое приблизятся 2 дрона, необходимо поочередно вычислить расстояние между коптерами на каждом кадре анимации с помощью формулы 1. 

Для этого все строки считаем в массив coordinates=N*M*3 и будем вычислять расстояния между точками  coordinates[i,k] и  coordinates[j,k], где k - номер кадра (фрейма), i, j - номера сравниваемых точек.

Если расстояние dist между сравниваемыми точками меньше последнего найденного минимального расстояния, присваиваем это значение переменной min\_distance и запоминаем номер кадра, на котором зафиксировано критическое сближение между дронами. 
В вывод передаем значение “-1”, если ни на одном кадре не произошло критического сближение (более, чем на 2 метра) или (min\_distance, frame) - минимальное расстояние между дронами и номер кадра.

\includeSolutionIfExistsByPath{2nd_tour/ats/task_03}