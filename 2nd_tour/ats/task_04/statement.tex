\assignementTitle{}{}{}

Распознавание сигналов светофоров

В этом задании вы сможете использовать свои знания по компьютерному зрению для создания классификатора изображений светофора! Вам будут даны изображения светофора, на каждом из которых горит только один из трех сигналов: красный, желтый или зеленый.

Система классификации

Мы подготовили для вас программу на языке Python, в которой вы будете предварительно обрабатывать изображения, выделять особенности, которые помогут находить отличия в видах изображения, использовать эти особенности, чтобы разделить изображения по трём категориям: светофор с красным, желтым или зелёным сигналом.

Этапы работы:

1. Загрузка и визуализация данных

В любой задаче по классификации сначала необходимо ознакомиться с данными: Вам нужно будет загрузить изображения сигналов светофоров и визуализировать их!

2. Предварительная обработка

Входные изображения и выходные метки (labels) должны быть стандартизированы:, все входные данные должны быть одного типа и одного размера, а выходные данные должны быть числовой меткой. Так вы сможете проанализировать все входные изображения одним и тем же способом и предугадать, чего следует ожидать для нового изображения.

3. Выделение особенностей

Теперь необходимо выделить особенности в каждом изображении и классифицировать их. Поле для творчества безгранично: объекты могут быть как одномерными массивами (векторами), так и отдельными значениями, которые дают некую информацию об изображении, чтобы помочь вам классифицировать его как красный, желтый или зеленый сигнал светофора. 

\putImgWOCaption{16cm}{1}

4. Ошибки классификации и визуализации

Наконец, вы создали функцию, которая использует найденные вами особенности для классификации любого сигнала светофора. Задача функции будет заключаться в том, чтобы получать на вход изображение и выводить метку (вектор из 3 значений). Вы можете сравнить прогнозируемую метку с истинной меткой и определить точность классифицируемой вами модели.

5. Оцените свою модель

Чтобы ваша работа была засчитана, классификатор должен иметь точность > 70\%. Вероятнее всего, вам нужно будет повысить точность вашего классификатора путем изменения имеющихся особенностей изображения или добавления новых. Чем выше точность, тем больше баллов вы получаете за решение.