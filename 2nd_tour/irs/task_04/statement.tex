\assignementTitle{Планирование движения в лабиринте}{15}{}

Роботы в количестве $N$, собранные по дифференциальной схеме, движутся по заранее известному лабиринту и
представленному на рисунке.

Необходимо доехать из начального в конечный сектор каждым роботом, если они движутся одновременно и параллельно и имеют следующие команды:

\begin{enumerate}
    \item $F$ - проезд робота в следующий по ходу движения сектор;
    \item $L$ - поворот робота в данном секторе налево и проезд в следующий по ходу движения сектор;
    \item $R$ - поворот робота в данном секторе направо и проезд в следующий по ходу движения сектор.
\end{enumerate}

Считать что роботы поворачиваются мгновенно, а после одновременно начинают движение вперёд с одинаковой скоростью. Робот является цилиндром и занимает 2/3 площади клетки. Каждое перемещение он начинает и заканчивает в центре клетки.  

Также гарантируется, что данных команд будет достаточно для выполнения задачи. При движении столкновение не допускается и перемещение необходимо осуществлять так, чтобы роботы получили наименьшее число команд. Иначе говоря, необходимо, чтобы сумма длин строк, содержащих команды передаваемые на робота в хронологическом порядке, без пробелов и запятых, была минимально возможной.

\putImgWOCaption{8cm}{1}

Структура лабиринта для решения задачи

\inputfmtSection

Первая строка содержит 1 целое число: $N$ - количество роботов на поле $(1 \leq N \leq 3)$.

Далее идет $N$ строк, содержащие координаты старта и финиша каждого робота, а также направление робота при старте, т.е. одна строка имеет следующую структуру: $x_s$, $y_s$, $dir$, $x_f$, $y_f$, где:

\begin{itemize}
    \item $x_s$ - координаты старта данного робота по оси $X$;
    \item $y_s$ - координаты старта данного робота по оси $Y$;
    \item $dir$ - направление робота при старте:
    \begin{itemize}
        \item $U$ - робот направлен вверх;
        \item $L$ - робот направлен влево;
        \item $D$ - робот направлен вниз;
        \item $R$ - робот направлен вправо;
    \end{itemize}
    \item $x_f$ - координаты финиша данного робота по оси $X$;
    \item $y_f$ - координаты финиша данного робота по оси $Y$;
\end{itemize}

Все данные указаны через пробел, числа являются целыми.

\outputfmtSection

$N$ строк, каждая строка содержит команды, передаваемые на данного робота в хронологическом порядке (самая первая команда расположена в начале строки), без пробелов и запятых.

Примечание: решение должно считывать из стандартного ввода и выводить результат в стандартный вывод. 
Примеры входных данных представлены по ссылке (\url{http://bit.ly/2KtN4z9}) и не выводятся в самом задании, поскольку займут много места на экране.

%\includeSolutionIfExistsByPath{2nd_tour/irs/task_04}