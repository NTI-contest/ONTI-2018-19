\solutionSection

Одним из вариантов решения данной задачи является составление графа, где вершины – состояние позиций роботов внутри лабиринта, а ребра - действия всех роботов, приведшие к переходу в это состояние. 

Например, если по лабиринту перемещаются три робота, то ребро "FRF" определяет, что следующее состояние получается посредством перемещения первого робота прямо, второго - поворотом направо, третьего - также проездом прямо. Необходимо помнить, что согласно условиям задачи, поворот направо/налево - это поворот в текущем секторе направо/налево и проезд прямо. 

При построении новых ребер необходимо учитывать, что роботы не должны сталкиваться. Также возможны такие ребра, где один из роботов стоит - т.е. в одном из предыдущих состояний он приехал в точку финиша.

На рис. \ref{fig:02_maze_movement_01} изображен пример графа для очень простого лабиринта и двух роботов. 
Очевидно, в таком графе можно применить поиск в ширину для нахождения вершины, когда все роботы находятся в требуемом секторе. Последовательность ребер из базового состояния в финишное состояние - будет кодировать путь каждого робота.

\begin{figure}[h!]
	\centering
	\includegraphics[width=0.8\linewidth]{2nd_tour/irs/task_04/02_maze_movement_01.pdf}
	\caption{Граф перемещения роботов}
	\label{fig:02_maze_movement_01}
\end{figure}


\codeExample

%\inputPythonSource
\inputCPPSource
