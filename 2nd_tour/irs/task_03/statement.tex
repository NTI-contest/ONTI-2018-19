\assignementTitle{Распознавание ARTag маркера}{10}{}

Для определения своего местоположения квадрокоптер использует камеру, снимающую поверхность, над которой перемещается робот. Изображение с камеры приходит в управляющую программу в виде набора $height \times width$ точек, где каждая точка закодирована в RGB-формате.

\putImgWOCaption{8cm}{1}

На поверхность нанесены ARTag маркеры (\url{https://inside.mines.edu/~whoff/courses/EENG512/lectures/other/ARTag.pdf}) 
Элементы маркера, расположенные по его границе - всегда черные. Четыре элемента, находящиеся в углах внутреннего $3\times 3$ квадрата определяют ориентацию маркера таким образом, что только элемент в нижнем правом углу квадрата - белый. Центральный элемент квадрата используется для проверки четности (parity check): если количество единичных бит в двоичной записи закодированного в маркере числа нечетное, то он черный. Оставшиеся $4$ элемента маркера кодируют число по следующему правилу: если элемент черный, то в он обозначает $1$, если белый, то $0$ при этом самый первый элемент --- старший бит закодированного числа. Элементы пронумерованы сверху вниз, слева направо (см. рис.).

Например, на маркере с рис. закодировано число $0011_2$, что эквивалентно $3_{10}$.

\putImgWOCaption{8cm}{2}

Маркер с закодированным значением - $0011_2$.

Поскольку квадрокоптер не перемещается постоянно параллельно поверхности, то изображения ARTag маркера, получаемые с камеры, получаются в виде неправильного выпуклого четырехугольника, а непостоянные условия освещенности изменяют фокус, тон и добавляют блики на изображение. Также квадрокоптеру не всегда удается полностью захватить маркер, и необходимо обрабатывать данные с нескольких снимков.

Поскольку направление запуска квадрокоптера заранее неизвестно, то ориентация маркеров заранее неизвестна, но его изображение таково, что оно по каждой из осей $X, Y, Z$ относительно оптической оси камеры не превышает 25 градусов.

Напишите программу для определения закодированного в маркере числа.

\inputfmtSection

Входной файл состоит из нескольких строк.

Первая строка содержит 3 целых числа: $N$, $height$, $width$:

\begin{itemize}
    \item $N$ - количество замеров камерой ($1 \leq N \leq 100$);
    \item $height$ - разрешение изображения по высоте ($10 \leq height \leq 1000$);
    \item $width$ - разрешение изображения по ширине ($10 \leq width \leq 1000$).
\end{itemize}

Далее расположено $N$ блоков $ height $ строк.

В каждом блоке расположено изображение размером $height \times width$ ввиде шестнадцатиричных чисел слева 
направо сверху вниз. Данные числа имеют следующий вид: $RRGGBB$, где $RR$ - 16тиричное число $R$ составляющей 
данного элемента матрицы, $GG$ и $BB$ - это 16тиричные числа $G$ и $B$ составляющих соответственно .

Все числа являются целыми.

\outputfmtSection

Вывести одно целое число в десятичной системе счисления, которое было закодированное на маркере. 
В случае если невозможно определить число, следует вывести -1.

Примечание: решение должно считывать из стандартного ввода и выводить результат в стандартный вывод. 
Примеры входных данных представлены по ссылке (\url{http://bit.ly/2DHww5w}) и не выводятся в самом задании, поскольку займут много места на экране.

%\includeSolutionIfExistsByPath{2nd_tour/irs/task_03}