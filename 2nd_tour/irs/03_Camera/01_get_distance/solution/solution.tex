\solutionSection

Составим схему по данным из условий задачи (рис. \ref{fig:03_camera_01_01}):
\begin{enumerate}
	\item Высота установки камеры, $h$
	\item Угол отклонения камеры от линии горизонта, $\alpha$
	\item Угол обзора камеры, $\beta$
	\item Разрешение камеры по высоте, $height$
\end{enumerate}

\begin{figure}[H]
	\centering
	\includegraphics[width=0.7\linewidth]{2nd_tour/irs/03_Camera/01_get_distance/solution/03_camera_01_01.pdf}
	\caption{}
	\label{fig:03_camera_01_01}
\end{figure}

Теперь вполне очевидно, что искомое значение $x$ - это катет $b$ прямоугольного треугольника $\triangle ABC$ (рис. \ref{fig:03_camera_01_02})

\begin{figure}[H]
	\centering
	\includegraphics[width=0.7\linewidth]{2nd_tour/irs/03_Camera/01_get_distance/solution/03_camera_01_02.pdf}
	\caption{}
	\label{fig:03_camera_01_02}
\end{figure}

Как вариант, найти длину катета $b$ можно через $tg(\angle CAB)$, тогда пусть $\angle CAB = \omega$.\\
Для нахождения угла $\omega$ нам надо найти углы $\theta$ и $\gamma$. \\
Для нахождения угла $\theta$ нам надо знать расстояние $x_1$. В нашем случае, расстояние $x_1$ - это высота в пикселях от нижней границы изображения до нижней границы предмета на изображении. см. рис.\ref{fig:03_camera_01_03}.

\begin{figure}[H]
	\centering
	\includegraphics[width=0.7\linewidth]{2nd_tour/irs/03_Camera/01_get_distance/solution/03_camera_01_03.pdf}
	\caption{}
	\label{fig:03_camera_01_03}
\end{figure}

Нахождение расстояния $x_1$:
По условиям задачи известно, что искомый объект на изображении - "однотонный".
Алгоритм поиска однотонного изображения заключается в поиске близлежащих пикселей одинакового цвета. Для этого используется построчное сканирование изображения с применением алгоритма BFS к каждому непросмотренному пикселю. При этом надо вести учет уже просмотренных ранее пикселей через BFS. Начальный пиксель находится в левом верхнем углу.

После обработки всего изображения сравниваем площади полученных объектов и находим объект с площадью, максимально приближенной к задаваемой в условии задачи.

Далее у этого объекта берем значения пикселя с ближайшими координатами к нижней границе изображения, рис. \ref{fig:03_camera_01_04}.
\\
\begin{figure}[H]
	\centering
	\includegraphics[width=0.7\linewidth]{2nd_tour/irs/03_Camera/01_get_distance/solution/03_camera_01_04.pdf}
	\caption{}
	\label{fig:03_camera_01_04}
\end{figure}

Находим расстояние $x_1$:
\begin{align*}
x_1 = height - \text{максимальная координата Y объекта} = 1080 - 720 = 320 \text{ пикселей}
\end{align*}

Угол $\theta$ найдем через пропорцию:
\begin{equation*}
\frac{\beta}{\theta} = \frac{height}{x_1} \qquad \Rightarrow \qquad \theta = \frac{x_1 \cdot \beta}{height} \qquad \Rightarrow \qquad \theta = \frac{320 \cdot 49}{1080} = 14.52 ^\circ
\end{equation*}

Угол $\gamma$ в нашем случае - это сумма угла $\alpha$ и половины угла $\beta$:
\begin{align*}
\gamma = \alpha + \frac{\beta}{2} \qquad \Rightarrow \qquad \gamma = 5 +  \frac{49}{2} = 29.5^\circ
\end{align*}

И, наконец, найдем угол $\omega$:
\begin{align*}
\omega = 90 - \gamma + \theta \qquad \Rightarrow \qquad \omega = 90 - 29.5 + 14.52 = 75.02^\circ
\end{align*}

Теперь найдем $X$, но предварительно переведем градусы в радианы:
\begin{align*}
\omega' = \frac{\omega \cdot \pi}{180} \qquad \Rightarrow \qquad \omega' = \frac{95 \cdot \pi}{180} = 1,309346005 \text{ радиан} \\ \\
x = h \cdot tg(\omega') \qquad \Rightarrow \qquad x = 95 \cdot tg(1,309346005) = 355 \text{ мм}
\end{align*}

\textbf{Ответ: расстояние от камеры до объекта равно 355 мм}

\codeExample

\inputPythonSource
%\inputCPPSource
