\assignementTitle{Определения расстояния до препятствия}{5}

Камера статично установлена на высоте $h$ мм и отклонена на $\alpha^\circ$ от горизонта.
Она имеет разрешение $height\times width$ и угол обзора $\beta^\circ$.
Поясняющая картинка представлена на рисунке \ref{fig:cameraExample}.

\putImgForRef{8cm}{2nd_tour/irs/03_Camera/01_get_distance/camera_example}
{Схема установки камеры}{fig:cameraExample}


Необходимо найти расстояние до предмета, установленного на полигоне.
Гарантируется, что предмет находится в поле зрения камеры, является однотонным.
Его минимальная площадь на изображении равна $s \%$.
Также известно, что помимо требуемого предмета
в камеру попадает поверхность полигона и(или) борта.
Поверхность полигона не однородная, а содержащая различные цвета в хаотическом порядке.

Пример изображения представлен на рис.  \ref{fig:getDistanceImageExample},

\putImgForRef{8cm}{2nd_tour/irs/03_Camera/01_get_distance/field_example}
{Пример изображения с камеры}{fig:getDistanceImageExample}


\inputfmtSection

Первая строка входного файла содержит 6 чисел: $h$,~ $\alpha$,~ $\beta$,~ $height$,~ $width$,~$s$:
\begin{itemize}
    \item $h$ --- высота установки камеры в мм ($10 \leq h \leq 100$);
    \item $\alpha$ --- угол в градусах,  под которым расположена камера относительно горизонта ($0^\circ \leq \alpha \leq 45^\circ$);
    \item $\beta$ --- угол обзора камеры в градусах ($30^\circ \leq \alpha \leq 180^\circ$);
    \item $height$ --- разрешение изображения по высоте ($10 \leq height \leq 2000$);
    \item $width$ --- разрешение изображения по ширине ($10 \leq width \leq 2000$);
    \item $s$ --- минимальная площадь предмета в $\%$ от всего изображения, которое измеряется в пикселях ($1 \leq s \leq 100$).
\end{itemize}

На следующих $height$ строках расположено изображение размером $height\times width$ в виде шестнадцатиричных чисел.
Данные числа имеют следующий вид: $RRGGBB$, где $RR$ - 16тиричное число $R$ составляющей данного элемента
матрицы, $GG$ и $BB$ - 16тиричные числа $G$ и $B$ составляющих, соответственно.

Все числа являются целыми.


\outputfmtSection

Одна строка, в которой указано число --- расстояние до предмета в мм. Допускается погрешность в 10 мм.


\exampleSection

Примеры входных данных и ответов к ним можно найти по сслыке \url{http://bit.ly/2FCbgAU}.


\includeSolutionIfExistsByPath{2nd_tour/irs/03_Camera/01_get_distance/solution}