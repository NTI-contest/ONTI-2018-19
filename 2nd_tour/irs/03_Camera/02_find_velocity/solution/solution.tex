\solutionSection

Для примера рассмотрим схему установки камеры (рис. \ref{fig:02_find_velocity_01}), снимки с камеры (рис. \ref{fig:02_find_velocity_02}) и набор данных:
\begin{itemize}
	\item Количество замеров камерой, $N = 3$
	\item Высота установки камеры, $h = 100$ мм
	\item Угол обзора камеры, $\alpha = 30^\circ$
	\item Разрешение снимка камеры по высоте, $height = 15$ пикселей
	\item Разрешение снимка камеры по ширине, $width = 15$ пикселей
\end{itemize}

\begin{figure}[h!]
	\centering
	\includegraphics[width=0.7\linewidth]{2nd_tour/irs/03_Camera/02_find_velocity/solution/02_find_velocity_01.pdf}
	\caption{Схема установки камеры с исходными данными}
	\label{fig:02_find_velocity_01}
\end{figure}

\begin{figure}[h!]
	\centering
	\includegraphics[width=0.7\linewidth]{2nd_tour/irs/03_Camera/02_find_velocity/solution/02_find_velocity_02.pdf}
	\caption{Пример снимков (1 цв.квадрат = 1 пиксель) с камеры размером 15х15 пикселей}
	\label{fig:02_find_velocity_02}
\end{figure}

Для решения задачи сначала вычисляем перемещения робота через смещения пикселей на снимках (рис. \ref{fig:02_find_velocity_03}). Сравниваем по два смежных снимка: $снимок_{i-1} и снимок_i$. После обработки всех снимков получаем конечное смещение в пикселях.

\begin{figure}[h!]
	\centering
	\includegraphics[width=0.7\linewidth]{2nd_tour/irs/03_Camera/02_find_velocity/solution/02_find_velocity_03.pdf}
	\caption{Вычисление перемещения робота}
	\label{fig:02_find_velocity_03}
\end{figure}

\newpage

Для пересчета смещения из пикселей в мм рассмотрим схему на рис. \ref{fig:02_find_velocity_04} и переведем полученные относительные координаты $(x; y)$ в мм.

\begin{figure}[h!]
	\center
	\includegraphics[width=0.4\linewidth]{2nd_tour/irs/03_Camera/02_find_velocity/solution/02_find_velocity_04.pdf}
	\caption{}
	\label{fig:02_find_velocity_04}
\end{figure}


Вычислим размерность 1 пикселя для ширины и высоты снимка (формулы \ref{eq:02_find_velocity_01} и \ref{eq:02_find_velocity_02}). \\

\begin{equation}
pixel\_width\_mm = \frac{h \cdot tg(\beta) \cdot 2}{width} \Rightarrow
\frac{100 \cdot tg(0.2618) \cdot 2}{15} = 3.57 \text{ мм}
\label{eq:02_find_velocity_01}
\end{equation}

\begin{equation}
pixel\_height\_mm = \frac{h \cdot tg(\beta) \cdot 2}{height} \Rightarrow
\frac{100 \cdot tg(0.2618) \cdot 2}{15} = 3.57 \text{ мм}
\label{eq:02_find_velocity_02}
\end{equation}


Переводим пиксели в мм:
\begin{equation*}
\begin{aligned}
& x = x \cdot 3.57 = -1 \cdot 3.57 = -3.6 \text{(мм)}\\
& y = y \cdot 3.57 = -2 \cdot 3.57 = -7.1 \text{(мм)}
\end{aligned}
\end{equation*}

\textbf{Ответ: } -3.6 -7.1



\codeExample

\inputPythonSource
%\inputCPPSource
