\assignementTitle{Запуск вcенаправленной тележки}{5}{}

Робот, моторы которого расположены под углом в $120^{\circ}$ друг к другу, 
движется в некотором направлении, пока на моторы подается мощность. Каждый мотор работает некоторое время: 
$t_1$, $t_2$, $t_3$, соответственно. На валах моторов закреплены омниколёса 
(\url{https://en.wikipedia.org/wiki/Omni_wheel}). С некоторой кинематической моделью робота можно 
познакомиться по ссылке: 
\url{https://bharat-robotics.github.io/blog/kinematic-analysis-of-holonomic-robot}.

Необходимо определить новые координаты центра робота, если в момент начала движения он находился в точке 
и один из двигателей находился на оси $Y$, в направлении положительной части (см. рис.). 
Расположение моторов является постоянным и соответствует рисунку.

Считать, что мощность на все моторы подаётся одновременно и достигается мгновенно. Также считать что 
происходит движение без поворотов, иначе говоря робот двигается только прямолинейно в любом направлении. 
Колёса вращаются без проскальзывания.

\putImgWOCaption{8cm}{1}

Расположение моторов робота относительно центра глобальной системы отсчета в начальный момент времени.

\inputfmtSection

Одна строчка, состоящая из 7-ми чисел: $d,$ $w_1,$ $t_1,$ $w_2,$ $t_2,$ $w_3,$ $t_3$, разделёнными пробелами, где

\begin{itemize}
    \item диаметр колёс в мм ($30 \leq d \leq 100$); $w_1, w_2, w_3$ -  скорости вращения моторов рад/с ($-2$ $\leq w_1, w_2, w_3 \leq 2$ )
    \item $t_1, t_2, t_3$ - время работы каждого мотора в с ($10 \leq t_1, t_2, t_3 \leq 1000$).
    \item Диаметр колёс и время движения - целые числа, скорости вращения - вещественные.
\end{itemize}

\outputfmtSection

Одна строка, содержащая два целых числа, через пробел - координаты центра робота в мм, где он закончил своё движение.

Допускается погрешность в 1 мм по каждой из координат.

\subsubsection*{Примечание}

Данные в тестах подобраны таким образом, что нет варианта движения, когда робот движется вокруг какой-то точки.

Примечание: решение должно считывать из стандартного ввода и выводить результат в стандартный вывод. Примеры входных данных представлены по 
ссылке \url{http://bit.ly/2RdizA3} и не выводятся в самом задании, поскольку займут много места на экране.

\includeSolutionIfExistsByPath{2nd_tour/irs/task_01}