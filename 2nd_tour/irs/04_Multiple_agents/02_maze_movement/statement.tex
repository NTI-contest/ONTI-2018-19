\assignementTitle{Планирование движения в лабиринте}{15}

Роботы в количестве $N$, собранные по дифференциальной схеме, движутся по заранее известному лабиринту и
представленному на рис. \ref{fig:mazeStructureExample}.

Необходимо доехать из начального в конечный сектор каждым роботом, если они
движутся одновременно и параллельно и имеют следующие команды:
\begin{itemize}
    \item $F$ --- проезд робота в следующий по ходу движения сектор;
    \item $L$ --- поворот робота в данном секторе налево и проезд в следующий по ходу движения сектор;
    \item $R$ --- поворот робота в данном секторе направо и проезд в следующий по ходу движения сектор.
\end{itemize}
Считать что роботы поворачиваются мгновенно, а после одновременно начинают движение вперёд с 
одинаковой скоростью.Робот является цилиндром и занимает 2/3 площади клетки. Каждое перемещение он начинает и 
заканчивает в центре клетки.

Также гарантируется, что данных команд будет достаточно для выполнения задачи. При движении столкновение не 
допускается и перемещение необходимо осуществлять так, чтобы роботы получили наименьшее число команд. Иначе 
говоря, необходимо, чтобы сумма длин строк, содержащих команды передаваемые на робота в хронологическом порядке, 
без пробелов и запятых, была минимально возможной.



\putImgForRef{8cm}{2nd_tour/irs/04_Multiple_agents/02_maze_movement/field_example}
{Структура лабиринта для решения задачи}{fig:mazeStructureExample}



\inputfmtSection


Первая строка содержит 1 целое число: $N$ --- количество роботов на поле $(1 \leq N \leq 3)$.

Далее идет $N$ строк, содержащие координаты старта и финиша каждого робота, а также
направление робота при старте, т.е. одна строка имеет следующую структуру:
$x_s$,~ $y_s$,~$dir$,~ $x_f$,~ $y_f$, где:
\begin{itemize}
    \item $x_s$ --- координаты старта данного робота по оси $X$;
    \item $y_s$ --- координаты старта данного робота по оси $Y$;
    \item $dir$ --- направление робота при старте:
        \begin{itemize}
            \item $U$ -- робот направлен вверх;
            \item $L$ -- робот направлен влево;
            \item $D$ -- робот направлен вниз;
            \item $R$ -- робот направлен вправо;
        \end{itemize}
    \item $x_f$ --- координаты финиша данного робота по оси $X$;
    \item $y_f$ --- координаты финиша данного робота по оси $Y$;
\end{itemize}

Все данные указаны через пробел, числа являются целыми.


\outputfmtSection

$N$ строк, каждая строка содержит команды, передаваемые на данного робота в хронологическом порядке
(самая первая команда расположена в начале строки), без пробелов и запятых.


\commentsSection

Дополнительные наборы входных данных доступны по ссылке \url{http://bit.ly/2KtN4z9}.

\exampleSection

\sampleTitle{1}

\begin{myverbbox}[\small]{\vinput}
    2
    6 7 U 0 2
    7 7 U 1 2
\end{myverbbox}
\begin{myverbbox}[\small]{\voutput}
    FLLRFFFFRFFFF
    FLRFFFLLRFFRL
\end{myverbbox}
\inputoutputTable

\sampleTitle{2}

\begin{myverbbox}[\small]{\vinput}
    2
    0 0 D 2 5
    0 1 R 2 1
\end{myverbbox}
\begin{myverbbox}[\small]{\voutput}
    FFFFFFLFL
    RFFFFFLFFLFFLRFRLFLL
\end{myverbbox}
\inputoutputTable




\includeSolutionIfExistsByPath{2nd_tour/irs/04_Multiple_agents/02_maze_movement/solution}