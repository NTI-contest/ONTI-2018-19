\assignementTitle{Определение лабиринта}{10}

Роботы в количестве $N$, собранные по дифференциальной схеме, оснащёны тремя дальномерами, направленными налево, прямо и
направо относительно направления движения робота.
Роботы движутся одновременно  из заранее известных секторов по неизвестному лабиринту, размером $K \times M$ секторов.

Между секторами могут быть стены, нахождение препятствия, внутри сектора недопустимо.
Отсчёт координат секторов начинается с $(0;0)$ в левом верхнем углу, $X$  возрастает вправо, $Y$ --вниз.

В процессе своих перемещений они получают значения с дальномеров, позволяющих узнать структуру лабиринта.
В результате полученных данных необходимо понять в какие сектора полигона может попасть первый робот, без
учёта нахождения на поле других роботов. В случае если данное значение невозможно определить,
следует вывести количество секторов, посещённых в процессе движения данным роботом.

\putImg{8cm}{2nd_tour/irs/04_Multiple_agents/01_maze_structure_detecting/field_example}
{Пример соревновательного полигона}

\putImg{5cm}{2nd_tour/irs/04_Multiple_agents/01_maze_structure_detecting/sensors_location}
{Расположение датчиков}


\inputfmtSection

Первая строка содержит 4 целых числа: $N$, $K$, $M$ и $i$:
\begin{itemize}
    \item $N$ --- количество роботов на поле $(2 \leq N \leq 5)$;
    \item $K$ --- ширина (по оси $X$) лабиринта в секторах $(4 \leq K \leq 20)$;
    \item $M$ --- длина (по оси $Y$) лабиринта в секторах $(4 \leq M \leq 20)$;
    \item $i$ --- количество показаний каждого робота.
\end{itemize}

Далее идет $N$  строк, содержащие координаты старта каждого робота, а также направление робота при старте,
т.е. одна строка имеет следующую структуру: $x_s$, $y_s$, $dir$, где:
\begin{itemize}
    \item $x_s$ --- координаты старта данного робота по оси $X$;
    \item $y_s$ --- координаты старта данного робота по оси $Y$;
    \item $dir$ --- направление робота при старте:
    \begin{itemize}
        \item $U$ --- робот направлен вверх;
        \item $L$ --- робот направлен влево;
        \item $D$ --- робот направлен вниз;
        \item $R$ --- робот направлен вправо;
    \end{itemize}
\end{itemize}

Далее идет $N$ блоков по $i$ строк.
На каждой строке находится действие робота и показания всех датчиков после этого действия через пробел:
$Movement, ~ D_l,~ D_f, ~D_r$:
\begin{itemize}
    \item $Movement$ --- выполненное действие робота:
    \begin{itemize}
        \item $F$ --- проезд робота в следующий по ходу движения сектор;
        \item $L$ --- поворот робота в данном секторе налево;
        \item $R$ --- поворот робота в данном секторе направо;
    \end{itemize}
    \item $D_l$ --- показания датчика расстояния, направленного влево:
    \begin{itemize}
        \item $1$ --- присутствует препятствие;
        \item $0$ --- отсутствует препятствие;
    \end{itemize}
    \item $D_f$ --- показания датчика расстояния, направленного вперёд:
    \begin{itemize}
        \item $1$ --- присутствует препятствие;
        \item $0$ --- отсутствует препятствие;
    \end{itemize}
    \item $D_r$ --- показания датчика расстояния, направленного вправо:
    \begin{itemize}
        \item $1$ --- присутствует препятствие;
        \item $0$ --- отсутствует препятствие.
    \end{itemize}
\end{itemize}



\outputfmtSection

Одно число - количество секторов, которое возможно посетить. Если невозможно определить, то вывести число
секторов посещённых первым роботом.


\exampleSection

Примеры входных данных и ответов к ним можно найти по сслыке \url{http://bit.ly/2LK5FrH}.





\includeSolutionIfExistsByPath{2nd_tour/irs/04_Multiple_agents/01_maze_structure_detecting/solution}