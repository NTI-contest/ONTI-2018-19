\assignementTitle{Определение вектора перемещения робота в заданном виде}{20}{}

Робот, собранный по дифференциальной схеме, оборудован камерой. Камера установлена так, что смотрит 
вниз перпендикулярно поверхности, по которой движется робот. Известны характеристики камеры такие, 
как разрешение ($height \times width$), угол обзора $\alpha$, высота расположения над плоскостью движения. 
Верхний край кадра находится дальше от робота, чем нижний край кадра.

Есть последовательность кадров, сделанных камерой через равные промежутки времени. Всего кадров было сделано $N$.

\putImgWOCaption{10cm}{1}

Вид робота. Поясняющая картинка

Зная, что робот в каждый момент движется с постоянной линейной $(v)$ и постоянной угловой $(\omega)$ скоростью, необходимо определить перемещение по $ X $ и по $Y$ в мм.

\inputfmtSection

Входной файл состоит из нескольких строк.Первая строка содержит 5 целых чисел: $N$, $h$, $\alpha$, $height$, $width$:

\begin{itemize}
    \item $N$ - количество замеров камерой ($2 \leq N \leq 16$);
    \item $h$ - высота установки камеры в мм ($10 \leq h \leq 2000$);
    \item $\alpha$ - угол обзора камеры в градусах ($15^\circ \leq \alpha \leq 175^\circ$);
    \item $height$ - разрешение изображения по высоте ($5 \leq height \leq 16$);
    \item $width$ - разрешение изображения по ширине ($5 \leq width \leq 16$).
\end{itemize}

Далее расположено $N$ строк, на каждой из которых расположено изображение размером $height\times width$ 
в виде шестнадцатиричных чисел слева направо, сверху вниз. Данные числа имеют следующий вид: $RRGGBB$, 
где $RR$ - 16тиричное число $R$ составляющей данного элемента матрицы, $GG$  и $BB$ - это 16тиричные числа $G$ и $B$ составляющих 
соответственно.

Все числа являются целыми.

\outputfmtSection

Вывести пару вещественных чисел с точностью до десятых - перемещение по $X$ и по $Y$ в мм в данном порядке через пробел.

Примечание: решение должно считывать из стандартного ввода и выводить результат в стандартный вывод. 
Примеры входных данных представлены по ссылке (\url{http://bit.ly/2EP1c5g}) и не выводятся в самом задании, поскольку займут много места на экране.

\includeSolutionIfExistsByPath{2nd_tour/irs/task_12}