\solutionSection

Для решения задачи в симуляторе TRIK-Studio определим следующую конфигурацию моторов и датчиков:

\begin{center}
	\begin{tabular}{|l|c|}
		\hline 
		\rule{0pt}{2.5ex} \textbf{Оборудование} & \textbf{Порт} \\ 
		\hline 
		\rule{0pt}{2.5ex} Правый мотор & M4 \\ 
		\hline 
		\rule{0pt}{2.5ex} Левый мотор & M3 \\ 
		\hline 
		\rule{0pt}{2.5ex} Датчик расстояния, направленный вперед & A1 \\ 
		\hline 
		\rule{0pt}{2.5ex} Датчик расстояния, направленный влево & A2 \\ 
		\hline 
	\end{tabular} 
\end{center}

Для перемещения в лабиринте нам необходимо подготовить функции:
\begin{itemize}
	\item Проезд робота вперед на заданное расстояние
	\item Разворот робота на заданный угол (по энкодерам или гироскопу)
	\item Динамическое вычисление координат текущего сектора лабиринта (X; Y) и направления робота (азимут)
	\item Движение робота в лабиринте по правилу ''левой руки''
\end{itemize}


Алгоритмы ''прямолинейного движения на заданное расстояние'' и ''разворот робота на месте'' являются базовыми для начальной робототехники и подробно рассматривать мы их не будем.

Алгоритм ''левой руки'':
Для движения по лабиринту по правилу ''левой руки'' на роботе установлено 2 датчика расстояния: первый - спереди, второй – слева по ходу движения.

Псевдокод правила ''левой руки'':
\begin{enumerate}
	\item \textit{Если слева пусто (датчик возвращает расстояние больше, чем длина одного сегмента лабиринта), то:}
	\begin{enumerate}
		\item Поворот налево на 90 градусов
		\item Проезд прямо на 1 сегмент
	\end{enumerate}	
	\item \textit{иначе:}
	\begin{itemize}
		\item если спереди пусто (датчик возвращает расстояние больше, чем длина одного сегмента), то:
	\end{itemize}
	проезд прямо на 1 сегмент
	иначе:
	поворот направо на 90 градусов
\end{enumerate}


Дискретная одометрия:
Для вычисления азимута будем использовать следующее кодирование сторон света (см. рис.\ref{fig:04_discrete_odometry_01})

\putImgForRef{0.3\linewidth}
             {2nd_tour/irs/06_in_simulatior/04_discrete_odometry/solution/04_discrete_odometry_01}
             {Определение числовых вариантов сторон света}
             {fig:04_discrete_odometry_01}

Вполне очевидно, что азимут робота меняется только при поворотах, а координаты сектора меняются только при перемещении между ними, но изменение координат зависит от азимута.

По условиям задачи робот в начальный момент времени установлен в направлении ''восток''. Полные координаты робота в момент старта $[x, y, azimut]$ показаны на рис. \ref{fig:04_discrete_odometry_02} :

\putImgForRef{0.6\linewidth}
             {2nd_tour/irs/06_in_simulatior/04_discrete_odometry/solution/04_discrete_odometry_02}
             {Расположение робота на старте и начальные координаты робота [$x$, $y$, азимут]}
             {fig:04_discrete_odometry_02}

Далее, согласно алгоритму ''левой руки'', робот проверяет возможность перемещения влево. В нашем случае - ''пусто'', т.е. робот поворачивает налево и проезжает вперед 1 сектор.
При повороте налево меняется азимут с $1$ на $0$.
При проезде прямо меняются координаты $(X; Y)$ в зависимости от азимута.
Не забываем, что по условиям задачи робот на старте направлен вдоль оси $Y$, а ось $X$ направлена вправо от робота (см. табл.\ref{table:azimut})
\begin{table}
	\begin{center}
		\begin{tabular}{|c|c|c|c|}
			\hline 
			\multicolumn{4}{|c|}{\textbf{Азимут}} \\ 
			\hline 
			0 & 1 & 2 & 3 \\ 
			\hline 
			X-1 & Y+1 & X+1 & Y-1 \\ 
			\hline 
		\end{tabular}
		\caption{Пересчет координат робота (при движении ''вперед''), в зависимости от текущего азимута}
		\label{table:azimut}
	\end{center}
\end{table}

Подсчет посещенных секций делаем только при перемещении между ними (при поворотах секции не меняются).

Когда количество посещенных секций будет равно заданному значению - останавливаем робота и выводим на экран координаты робота. 
Обращайте внимание, в какое место экрана и в каком виде должен выводиться ответ.

Пример выполнения задания с количеством секций = 17 показан на рис.\ref{fig:04_discrete_odometry_03}.

\putImgForRef{0.6\linewidth}
             {2nd_tour/irs/06_in_simulatior/04_discrete_odometry/solution/04_discrete_odometry_03}
             {Перемещения робота по правилу ''левой руки'' с номерами посещенных секторов}
             {fig:04_discrete_odometry_03}

Для нашего примера робот остановится в координатах $(-3; 2)$ с азимутом $3$ (запад).

Ответ выводим на экран TRIK'а: \textbf{(-2, 3)}

\codeExample

\inputJSSource