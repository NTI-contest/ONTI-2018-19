\assignementTitle{Определение собственных координат}{10}

Робот собранный по дифференциальной схеме оснащен двумя инфракрасными датчиками расстояния. Один из датчиков расстояния установлен так, что показывает расстояние до препятствий прямо по курсу робота. Второй датчик расстояния направлен влево.

\putImgForRef{8cm}{2nd_tour/irs/06_in_simulatior/04_discrete_odometry/maze-002}
{Пример начального расположения робота}{fig:discreteOdometryStartPos}

Для решения задачи робот запускается на поле, состоящим из 8х8 квадратных секторов. На поле между некоторыми секторами установлены препятствия для ограничения перемещения робота.

Роботу необходимо проехать некоторое количество секторов, заданное через входной файл \textit{input.txt}, по правилу ''левой руки'', после остановиться и вывести на экран относительные координаты, т.е. свои координаты относительно точки старта, в формате $(X, Y)$, где начало отсчета - точка старта, направление оси $Y$ совпадает с первоначальным направлением движения, ось $X$ направлена вправо перпендикулярно оси $Y$.

\begin{center}
\noindent
\textbf{Конфигурация робота}
\end{center}

\twoitems{Подключение моторов}{Левый мотор - порт M3;}{Правый мотор - порт M4.}
\twoitems{Подключение датчиков}{Датчик расстояния, направленный вперед - порт A1}
{Датчик расстояния, направленный влево - порт A2}

\inputfmtSection

Входной файл содержит только одну строчку. В строке - целое число $N$ \linebreak $(5 \leq N \leq 40)$, определяющее количество секторов, которое необходимо проехать. Сектор старта не учитывается в подсчёте.

\begin{center}
\noindent
\textbf{Ограничения}
\end{center}

Робот не должен выполнять задание дольше 3 минут.

\exampleSection

Для лабиринта, представленного на рис. \ref{fig:discreteOdometryStartPos}, и при числе $7$ во входных данных, робот должен остановиться в соответствии с рис. \ref{fig:discreteOdometryFinishPos} и вывести на экран $(4,1)$.

\putImgForRef{8cm}{2nd_tour/irs/06_in_simulatior/04_discrete_odometry/maze-002-finish}
{Расположение робота, стартовавшего из позиции на рис. \ref{fig:discreteOdometryStartPos} и проехавшего $7$ секторов}{fig:discreteOdometryFinishPos}

\includeSolutionIfExistsByPath{2nd_tour/irs/06_in_simulatior/04_discrete_odometry/solution}