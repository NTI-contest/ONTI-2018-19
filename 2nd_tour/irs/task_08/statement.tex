\assignementTitle{Определение собственных координат}{10}

Робот собранный по дифференциальной схеме оснащен двумя инфракрасными датчиками расстояния. Один из датчиков расстояния установлен так, что показывает расстояние до препятствий прямо по курсу робота. Второй датчик расстояния направлен влево.

\putImgWOCaption{10cm}{1}

Для решения задачи робот запускается на поле, состоящим из 8х8 квадратных секторов. На поле между некоторыми секторами установлены препятствия для ограничения перемещения робота.

Роботу необходимо проехать некоторое количество секторов, заданное через входной файл input.txt, по правилу левой руки, после остановиться и вывести на экран относительные координаты, т.е. свои координаты относительно точки старта, в формате (X,Y), где начало отсчета - точка старта, направление оси У совпадает с первоначальным направлением движения, ось Х направлена вправо перпендикулярно оси У.

\subsection*{Конфигурация робота}

Подключение моторов:

\begin{itemize}
    \item Левый мотор - порт M3;
    \item Правый мотор - порт M4.
\end{itemize}

Подключение датчиков:

\begin{itemize}
    \item Датчик расстояния, направленный вперед - порт A1
    \item Датчик расстояния, направленный влево - порт A2
\end{itemize}


\inputfmtSection
 
Входной файл содержит только одну строчку. В строке - целое число $N (5 \leq N \leq 40)$, определяющее количество секторов, которое необходимо проехать.

Пример входных данных для рисунка:

7

Пример входных данных для TRIK studio:

17

\subsubsection*{Комментарии}

Для лабиринта, представленного на рис и используя пример входных данных. Робот должен остановиться в соответствии с рис и вывести на экран (4,1).

В случае если ваше решение не проходит тесты, рекомендуется удостовериться в этом несколько раз (до 5 попыток).  Данная проблема возникает из-за проверяющей среды.

%\includeSolutionIfExistsByPath{2nd_tour/irs/task_08}