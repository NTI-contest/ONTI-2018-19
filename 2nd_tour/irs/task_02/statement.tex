\assignementTitle{Определение расстояния до препятствия}{5}{}

Камера статично установлена на высоте $h$ мм и отклонена на $\alpha^\circ$ от горизонта. Она имеет разрешение $height\times width$ и угол обзора $\beta^\circ$. Поясняющая картинка представлена на рисунке.

\putImgWOCaption{8cm}{1}

\subsubsection*{Схема установки камеры}

Необходимо найти расстояние до предмета, установленного на полигоне. Гарантируется, что предмет находится в поле зрения камеры, и является однотонный. Его минимальная площадь на изображении равна $s \%$ .

Также известно, что помимо требуемого предмета в камеру попадает поверхность полигона и(или) борта. Поверхность полигона и бортов не однородная, а содержащая различные цвета в хаотическом порядке.

\putImgWOCaption{8cm}{2}

Пример изображения с камеры

\inputfmtSection

Первая строка входного файла содержит 6 чисел: $h$,  $\alpha$, $\beta$, $height$, $width$, $s$:

\begin{itemize}
    \item $h$ - высота установки камеры в мм ($10 \leq h \leq 100$);
    \item $\alpha$ - угол в градусах, под которым расположена камера относительно горизонта ($0^\circ \leq \alpha \leq 45^\circ$);
    \item $\beta$ - угол обзора камеры в градусах ($30^{\circ} \leq \alpha \leq 180^{\circ}$);
    \item $height$ -  разрешение изображения по высоте ($10 \leq height \leq 2000$);
    \item $width$ - разрешение изображения по ширине ($10 \leq width \leq 2000$);
    \item $s$ -  минимальная площадь предмета в $\% $ от всего изображения, которое измеряется в пикселях ($ 1 \leq s \leq 100 $).
\end{itemize}

На следующих $height$ строках расположено изображение размером $height\times width$ в виде шестнадцатиричных чисел.

Данные числа имеют следующий вид: $RRGGBB$, где $RR$ - 16тиричное число $R$ составляющей данного элемента<br>матрицы, $GG$ и $BB$ это 16тиричные числа $G$ и $B$ составляющих соответственно.

Все числа являются целыми.

\outputfmtSection

Одна строка, в которой указано число - расстояние до предмета в мм. Допускается погрешность в 10 мм.

Примечание: решение должно считывать из стандартного ввода и выводить результат в стандартный вывод. Примеры входных данных представлены по ссылке 
\url{http://bit.ly/2FCbgAU} и не выводятся в самом задании, поскольку займут много места на экране.

\includeSolutionIfExistsByPath{2nd_tour/irs/task_02}