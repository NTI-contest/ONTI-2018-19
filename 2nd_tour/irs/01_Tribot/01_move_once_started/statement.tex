\assignementTitle{Запуск всенаправленной тележки}{5}

Робот, моторы которого расположены под углом в $120^{\circ}$ друг к другу,
движется в некотором направлении, пока на моторы подается мощность.
Каждый мотор работает некоторое время: $t_1$,~$t_2$,~$t_3$, соответственно.
На валах моторов закреплены омниколёса (\url{https://en.wikipedia.org/wiki/Omni_wheel}).
С некоторой кинематической моделью робота можно познакомиться по ссылке:
\url{https://bharat-robotics.github.io/blog/kinematic-analysis-of-holonomic-robot/}.

Необходимо определить новые координаты центра робота, если в момент начала движения он находился в точке
$(0,0)$ и один из двигателей находился на оси $Y$, в направлении положительной части (см. рис. \ref{fig:sampleOnceStarted}).
Расположение моторов является постоянным и соответствует рисунку \ref{fig:sampleOnceStarted}.

Считать, что мощность на все моторы подаётся одновременно и достигается мгновенно.
Также считать что происходит движение без поворотов,
иначе говоря робот двигается только прямолинейно в любом направлении.
Колёса вращаются без проскальзывания.

Данные в тестах подобраны таким образом, что нет вариата движения, когда робот движется вокруг какой-то точки.


\putImgForRef{8cm}{2nd_tour/irs/01_Tribot/tribot_configuration}
{Расположение моторов робота относительно центра глобальной системы отсчета в начальный момент времени}{fig:sampleOnceStarted}


\inputfmtSection

Одна строчка, состоящая из 7ми чисел: $d,$~$w_1,$~$t_1,$~$w_2,$~$t_2,$~$w_3,$~$t_3$, разделёнными пробелами, где
\begin{itemize}
    \item $d$ --- диаметр колёс в мм ($30 \leq d \leq 100$);
    \item $w_1, w_2, w_3$ --- скорости вращения моторов в рад/с ($-2$ $\leq w_1, w_2, w_3 \leq 2$);
    \item $t_1, t_2, t_3$ --- время работы каждого мотора в с ($10 \leq t_1, t_2, t_3 \leq 1000$).
\end{itemize}

Диаметр колёс и время движения --- целые числа, скорости вращения --- вещественные.

\commentsSection

Дополнительные наборы входных данных доступны по \url{http://bit.ly/2RdizA3}{ссылке}.

\outputfmtSection


Одна строка, содержащая два целых числа  через пробел  --- координаты центра робота в мм,
где он закончил своё движение.
Допускается погрешность в 1 мм по каждой из координат.

\exampleSection

\sampleTitle{1}


\begin{myverbbox}[\small]{\vinput}
    35 1 15 0 0 -1 15
\end{myverbbox}
\begin{myverbbox}[\small]{\voutput}
    196.875 -113.666
\end{myverbbox}
\inputoutputTable

\sampleTitle{2}

\begin{myverbbox}[\small]{\vinput}
    40 1.4 100 -1.4 100 1.4 100
\end{myverbbox}
\begin{myverbbox}[\small]{\voutput}
    2800 2424.87
\end{myverbbox}
\inputoutputTable


\includeSolutionIfExistsByPath{2nd_tour_progr/01_Tribot/01_move_once_started/solution}