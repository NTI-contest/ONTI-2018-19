\solutionSection

В данной задаче модель всенаправленной тележки аналогична модели в предыдущей задаче (рис. \ref{fig:01_move_multi_01}) и по условиям задачи тележка движется прямолинейно, т.е. мы можем использовать уравнения кинематики из прошлой задачи. 

\begin{figure}[h!]
	\centering
	\includegraphics[width=0.7\linewidth]{2nd_tour/irs/01_Tribot/02_move_multiple_starting/solution/01_move_multi_01}
	\caption{Схема всенаправленной тележки}
	\label{fig:01_move_multi_01}
\end{figure}

Рассмотрим отличия от предыдущей задачи:
\begin{itemize}
	\item Время вращения всех моторов постоянное и одинаковое, $t$ с.
	\item У моторов (у всех одновременно) происходит изменение угловых скоростей, $N$ раз за все время движения $t$
\end{itemize}

Время, используемое в этой формуле, вычисляется делением всего временного интервала, в течение которого двигается робот на количество измерений (формула \ref{eq:01_move_multi_01}).

\begin{equation}
	dt = \frac{t}{N}
	\label{eq:01_move_multi_01}
\end{equation}

Таким образом, чтобы получить итоговые координаты центра тележки, нужно итеративно (через время $dt$) производить вычисление новых координат, подставляя в каждой итерации очередное значение мощности для каждого мотора.

\codeExample

%\inputPythonSource
\inputCPPSource