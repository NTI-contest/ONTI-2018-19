\assignementTitle{Подсчет черных линий}{10}{}

Перед запуском робот установлен перед последовательностью отрезков черных линий. Первый отрезок находится перпендикулярно направлению движения робота, остальные отрезки - параллельны первому.

Робот оснащен одним датчиком освещенности. Напишите программу для остановки робота над черной линией, чей номер передается в программу через входной файл input.txt.

\subsubsection*{Конфигурация робота}

Подключение моторов:

\begin{enumerate}
    \item Левый мотор - порт M3;
    \item Правый мотор - порт M4.
\end{enumerate}

Подключение датчиков:

\begin{enumerate}
    \item Датчик освещенности подключен к порту A1.
\end{enumerate}


\inputfmtSection
 
Входной файл содержит только одну строчку. В строке - целое число $N (1 \leq N \leq 20)$, определяющее номер линии, на которой необходимо остановиться. После остановки на нужной чёрной линии, робот должен остановиться и вывести на экран слово «finish».

\subsubsection*{Комментарии}

Робот считается остановившимся над линией, если после остановки любая его часть (но не датчик освещенности) находится над черной линией.

Ширина черных линий и расстояние между ними заранее неизвестны. Ширина черной линии может варьироваться от линии к линии. Расстояние между двумя соседними черными линиями также может быть непостоянно.

%\includeSolutionIfExistsByPath{2nd_tour/irs/task_08}