\assignementTitle{Определение цветных цилиндров}{15}


Робот, собранный по дифференциальный схеме и оснащенный дальномером, направленным прямо по ходу движения,
перемещается по робототехническому полигону.
На данном полигоне установлены цилиндры различного цвета. Робот также является цилиндром.

В процессе своего передвижения по полигону (см рис. \ref{fig:colourCylindersExample})
робототехническое устройство измерило расстояние до возникающих прямо препятствий
и их цвет в шестнадцатиричном формате вида $RRGGBB$, где $RR$ - 16тиричное число $R$ составляющей данного элемента
матрицы, $GG$ и $BB$ - 16тиричные числа $G$ и $B$ составляющих, соответственно.

\putImgForRef{8cm}{2nd_tour/irs/02_Beacons/02_colour_cylinders/beacons2}
{Пример соревновательного полигона}{fig:colourCylindersExample}

Необходимо определить между какими цилиндрами робот не сможет проехать, если известен маршрут робота и показания
датчика в процессе движения.
Размер цилиндров и робота: $20$ см в диаметре.
Дальномер возвращает расстояние до цилиндра, считая от оси вращения робота,
находящейся в центре между колёсами.
Также гарантируется, что робот измерил каждый цилиндр не менее трех раз, и разница между этими измерениями составляет
минимум $5$ дуговых градусов (при измерении по дуге цилиндра).

\inputfmtSection

Первая строка содержит 2 целых числа: $N$ и $dT$:
\begin{itemize}
    \item $N$ --- количество замеров ($10 \leq N \leq 10000$);
    \item $dT$ --- пауза между измерениями в мс ($0 \leq dT \leq 10000$).
\end{itemize}

Далее идёт $N$ строк. Каждая строка имеет следующую структуру:
$Vel_{linear}$, \linebreak $Vel_{angular}$, $Distance$, $Color$, в которой все числа разделены пробелами, где:
\begin{itemize}
    \item $Vel_{linear}$ --- линейная скорость в см/с, с которой движется робот в данный момент ($-100 \leq Vel_{linear} \leq 100$);
    \item $Vel_{angular}$ --- угловая скорость в рад/мс, с которой движется робот в данный момент ($-10 \leq Vel_{angular} \leq 10$);
    \item $Distance$ --- показания датчика расстояния в см в диапазоне ``5--255'' см, если предмет вне зоны видимости, то
    будет выведено $255$;
    \item $Color$ --- цвет обнаруженного циллиндра в шестнадцатиричном формате вида $RRGGBB$.
    В случае если цилиндр не обнаружен, будет выведено ``000000''.
\end{itemize}

Значения скоростей и показания датчика являются вещественными числами. Моторы достигают скоростей мгновенно. Робот движется без проскальзывания.
Значения цвета --- целые числа.


\outputfmtSection

Одна строка, в которой указана пара цветов цилиндров в шестнадцатиричной записи через пробел, в порядке возрастания чисел.
В случае если таких пар несколько, то каждую пару следует выводить в отдельной строке.
Несколько строк следует выводить в порядке возрастания первых чисел, а в случае их равенства, в порядке возрастания вторых.


\exampleSection

Примеры входных данных и ответов к ним можно найти по сслыке \url{http://bit.ly/2UoC22Z}.



\includeSolutionIfExistsByPath{2nd_tour/irs/02_Beacons/02_colour_cylinders/solution}