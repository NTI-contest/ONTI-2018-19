\assignementTitle{Определение размера препятствия}{15}

Робот, собранный по дифференциальной схеме, передвигается по полигону.
Он оснащён приёмником и дальномером.
Приёмник позволяет измерять расстояния до установленных на поле и находящихся в прямой видимости маяков.
Дальномер направлен влево по ходу движения робота и позволяет получать информацию
о находящихся препятствиях в данном направлении.
На полигоне находятся препятствие и $K$ маяков, координаты которых передаются через входной файл.
Гарантируется, что маяки не мешают роботу перемещаться и не попадают в поле зрения дальномера.
Пример полигона представлен на рис. \ref{fig:beaconFieldExample}.

Необходимо найти площадь препятствия, расположенного на полигоне.
Известно что, приёмник возвращает значения расстояний до препятствия в мм, считая от оси вращения робота,
находящейся в центре между колёсами. Дальномер расположен в том же месте, что и приёмник.

\putImgForRef{8cm}{2nd_tour_progr/02_Beacons/01_beacon_searching/beacons}
{Пример соревновательного полигона}{fig:beaconFieldExample}

\inputfmtSection

Первая строка содержит 3 целых числа: $K$ и $N$ и $dT$:
\begin{itemize}
    \item $K$ --- количество маяков ($10 \leq K \leq 100$);
    \item $N$ --- количество замеров ($10 \leq N \leq 10~000$);
    \item $dT$ --- пауза между измерениями в мс ($0 \leq dT \leq 10~000$).
\end{itemize}

Далее идёт $K$ строк. Каждая строка имеет следующую структуру:
$i$,~ $x_i$,~ $y_i$, в которой все числа вещественные и разделены пробелами, где:
\begin{itemize}
    \item $i$ --- номер маяка по порядку ($0 \leq i \leq 10~000$);
    \item $x_i$ --- координата $x$ $i$-того маяка ($-10~000 \leq x_i \leq 10~000$);
    \item $y_i$ --- координата $y$ $i$-того маяка ($-10~000 \leq y_i \leq 10~000$).
\end{itemize}

Далее идёт $N$ строк. Каждая строка содержит:
\begin{itemize}
    \item $d$ --- целое число, показание дальномера в мм ($0 \leq d \leq 1~000$), в случае если предмет
    находится вне поля видимости дальномера, то его значение будет равно $1~000$;
    \item $Value_{1} ~Value_2 ~\dots ~Value_k$ --- вещественные числа, показания, получаемые приёмником с каждого маяка.
    В случае, если маяк находится вне зоны видимости, то соответствующее значение будет равно $-1$.
\end{itemize}

Все числа указаны через пробел.

\outputfmtSection

Одна строка, содержащая одно целове число -- площадь препятствия в мм$^2$.
Допускается погрешность в $\pm 1$ м$^2$.

\exampleSection

Примеры входных данных и ответов к ним можно найти по \href{http://bit.ly/2Re5l9N}{данной} сслыке.




\includeSolutionIfExistsByPath{2nd_tour_progr/02_Beacons/01_beacon_searching/solution}