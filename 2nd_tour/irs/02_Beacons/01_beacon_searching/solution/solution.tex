\solutionSection

Для определения границ препятствия сначала определим координаты робота в момент замеров. Для этого мы найдем точки пересечения двух окружностей, образованных от центров маяков, которые "видят" робота в данный момент времени. Схема представлена на рис.\ref{eq:02_beacons_01_01}, где: \\

\begin{itemize}
	\item $M_1$ - маяк $N$, установленный в координатах $(a_1; b_1)$
	\item $M_2$ - маяк $K$, установленный в координатах $(a_2; b_2)$
	\item $r_1$ - радиус первой окружности (расстояние от маяка $N$ до центра робота)
	\item $r_2$ - радиус второй окружности (расстояние от маяка $K$ до центра робота)
\end{itemize}

\begin{figure}[h!]
	\centering
	\includegraphics{2nd_tour_progr/02_Beacons/01_beacon_searching/solution/02_beacons_01_01.pdf}
	\caption{}
	\label{fig:02_beacons_01_01}
\end{figure}

Сначала найдем расстояние между центрами маяков: $d$ $(M_1$;$M_2)$ (можно включить проверку отсутствие решений: $d = |M_1 - M_2|$, при $d > r_1 + r_2$ (круги не пересекаются) или $d < |r_2 - r_1|$ (одна окружность находится внутри другой)).
На схеме (рис.\ref{eq:02_beacons_01_01}) видно, что $r_1$ и $r_2$ - катеты $\triangle M_1M_2K_1$, а искомая сторона $M_1M_2$ - гипотенуза. Исходя из теоремы Пифагора, найдем гипотенузу по формуле (\ref{eq:02_beacons_01_01}).

\begin{equation}
	d = \sqrt{{(a_2 - a_1)}^2 + {(b_2 - b_1)}^2}
	\label{eq:02_beacons_01_01}
\end{equation}

Далее нам надо найти $r_2$, $M_1S = d_1$; $M_2S = d_2$.

\[
\left\{ 
	\begin{aligned}
		& d_1 + d_2 = d \\
		& h = \sqrt{{r_1}^2 - {d_1}^2} = \sqrt{{r_2}^2 - {d_2}^2}
	\end{aligned}
	\right.
	\qquad \Rightarrow \qquad
	\left\{ 
	\begin{aligned}
		& d_1 + d_2 = d \\
		& {r_1}^2 - {d_1}^2 = {r_2}^2 - {d_2}^2
	\end{aligned}
\right.
\qquad	\Rightarrow
\]
\[
	\left\{ 
		\begin{aligned}
			& d_1 + d_2 = d \\
			& {d_2}^2 - {d_1}^2 = {r_2}^2 - {r_1}^2
		\end{aligned}
		\right.
		\qquad \Rightarrow \qquad
		\left\{ 
		\begin{aligned}
			& d_1 + d_2 = d \\
			& (d_2 - d_1)(d_2 + d_1) = {r_2}^2 - {r_1}^2
		\end{aligned}
	\right.
	\qquad \Rightarrow
\]
\[
	(d_2 - d_1) = \frac{{r_2}^2 - {r_1}^2}{d} \qquad \Rightarrow \qquad
	d_2 = \frac{{r_2}^2 - {r_1}^2}{d} + d_1 = \frac{{r_2}^2 - {r_1}^2}{d} + d - d_2 \qquad \Rightarrow
\]

\begin{equation}
	2\cdot d_2 = \frac{{r_2}^2 - {r_1}^2}{d} + d \qquad \Rightarrow \qquad 
	d_2 = \frac {{r_2}^2 - {r_1}^2}{2\cdot d} + \frac{d}{2}
	\label{eq:02_beacons_01_02}
\end{equation}

Теперь найдем катет $d_2$ в треугольнике $K_1M_2S$ по формуле (\ref{eq:02_beacons_01_02}).

Зная расстояния $d$ и $d_2$, находим $d_1$ по формуле (\ref{eq:02_beacons_01_03}):
\begin{equation}
	d_1 = d - d_2
	\label{eq:02_beacons_01_03}
\end{equation}

Зная катет $d_1$ и гипотенузу $r_1$ найдем катет $h$ в $\triangle M_1K_1S$ по формуле (\ref{eq:02_beacons_01_04}):

\begin{equation}
	h = \sqrt{{r_1}^2 - {d_1}^2}
	\label{eq:02_beacons_01_04}
\end{equation}

Далее находим координаты точки $S$ через формулу деления отрезка в данном отношении (через координаты $M_1$ и $M_2$ и отношения $k$):
\begin{eqnarray}
		k = \frac{d_1}{d_2} \nonumber \\
		x_s = \frac{a_1 + k \cdot a_2}{1 + k} \label{eq:02_beacons_01_05} \\
		y_s = \frac{b_1 + k \cdot b_2}{1 + k} \label{eq:02_beacons_01_06}
\end{eqnarray}

Далее найдем координаты обоих точек пересечения окружностей по формулам (\ref{eq:02_beacons_01_07}, \ref{eq:02_beacons_01_08}, \ref{eq:02_beacons_01_09}, \ref{eq:02_beacons_01_10}):

\begin{equation}
x_1 = \frac{h}{d} \cdot (b_1 - b_2)+x_s
\label{eq:02_beacons_01_07}
\end{equation}
\begin{equation}
y_1 = \frac{h}{d} \cdot (a_2 - a_1)+y_s
\label{eq:02_beacons_01_08}
\end{equation}
\begin{equation}
x_2 = -\frac{h}{d} \cdot (b_1 - b_2)+x_s
\label{eq:02_beacons_01_09}
\end{equation}
\begin{equation}
y_2 = -\frac{h}{d} \cdot (a_2 - a_1)+y_s
\label{eq:02_beacons_01_10}
\end{equation}

Для выбора координат из полученных $(x_1, y_1)$ и $(x_2, y_2)$ проверим значения по координатам 3 маяка (на выбор).

Для вычисления координат точки на препятствии, видимую датчиком расстояния, сначала найдем угол $\alpha$ робота по формуле (\ref{eq:02_beacons_01_11}):
\begin{equation}
	\alpha_{robot} = atan2(x_i - x_{i-1};  y_i - y_{i-1})
	\label{eq:02_beacons_01_11}
\end{equation}

после чего найдем угол дальномера $\beta$ (с учетом того, что дальномер направлен влево по ходу движения робота) по формуле (\ref{eq:02_beacons_01_12}):

\begin{equation}
	\beta = \alpha_{robot} + \frac{\pi}{2}
	\label{eq:02_beacons_01_12}
\end{equation}

координаты точки на препятствии, вычисляемые с показаний дальномера, находим по формулам (\ref{eq:02_beacons_01_13}, \ref{eq:02_beacons_01_14}):
\begin{equation}
	x_o = x_{i\;robot} + d \cdot cos(\beta)
	\label{eq:02_beacons_01_13}
\end{equation}
\begin{equation}
	y_o = y_{i\;robot} + d \cdot sin(\beta)
	\label{eq:02_beacons_01_14}
\end{equation}

где $d$ - показания с дальномера до препятствия.
\\

Результаты вычислений показаны на рис. \ref{fig:02_beacons_01_02}

\begin{figure}[h!]
	\centering
	\includegraphics{2nd_tour_progr/02_Beacons/01_beacon_searching/solution/02_beacons_01_02.pdf}
	\caption{Траектория движения робота и границы препятствия}
	\label{fig:02_beacons_01_02}
\end{figure}


По полученным координатам точек многоугольника находим площадь препятствия по формуле площади Гаусса (\ref{eq:02_beacons_01_15}):
\begin{equation}
	\begin{aligned}
		A & =\frac{1}{2} 
		\left|\sum_{i=1}^{n-1} x_i y_{i+1} + x_n y_1 - \sum_{i=1}^{n-1} x_{i+1} y_i -  x_1 y_n \right| \\ 
		& =
		\frac {1}{2} |x_1 y_2 + x_2 y_3 + \cdots + x_{n-1}y_n + x_n y_1 - x_2 y_1 - x_3 y_2- \cdots - x_n y_{n-1} - x_1 y_n |
	\end{aligned}
	\label{eq:02_beacons_01_15}
\end{equation}

где,
\begin{itemize}
	\item $A$ - площадь многоугольника
	\item $n$ - количество сторон многоугольника
	\item $(x_i, y_i)$ при $i = 1, 2, \dots n $ – координаты вершин многоугольника	
\end{itemize} 


В ответ выводим полученную площадь препятствия.


\codeExample

\inputPythonSource
%\inputCPPSource