\solutionSection

Решение задачи начнем с изучения структуры пакета UDP.\\
По условиям задачи длина пакета составляет 32 байта в шестнадцатеричном формате.
Каждый пакет содержит следующую информацию:
\begin{itemize}
	\item IP-адрес отправителя - 4 байта
	\item IP-адрес получателя - 4 байта
	\item Время снятия показаний координат устройства - 4 байта
	\item Координата по оси X в данный момент времени - 4 байта
	\item Координата по оси Y в данный момент времени - 4 байта
\end{itemize}

Из описания структура протокола UDP (для IPv4) нам известно, что IP-адреса отправителя и получателя находятся в начале пакета и занимают по 4 байта каждый (таблица \ref{fig:01_Found_intersects_01}).

\begin{table}[h!]
	\begin{center}
		\begin{tabular}{|c|c|c|c|c|c|c|c|c|c|c|c|c|c|c|c|c|c|c|c|c|c|c|}
			\hline
			байт   & \multicolumn{2}{c|}{0}  & \multicolumn{2}{c|}{1}  & \multicolumn{2}{c|}{2}  & \multicolumn{2}{c|}{3}  & \multicolumn{2}{c|}{4}  & \multicolumn{2}{c|}{5}  & \multicolumn{2}{c|}{6}  & \multicolumn{2}{c|}{7}  & \multicolumn{2}{c|}{8}  & \multicolumn{2}{c|}{9} & \multicolumn{2}{c|}{10} \\ \hline
			данные & \textbf{7} & \textbf{4} & \textbf{1} & \textbf{b} & \textbf{c} & \textbf{7} & \textbf{8} & \textbf{a} & \textbf{4} & \textbf{9} & \textbf{2} & \textbf{7} & \textbf{2} & \textbf{a} & \textbf{2} & \textbf{b} & \textbf{0} & \textbf{0} & 1          & 1         & 0          & 0          \\ \hline
			& \multicolumn{8}{c|}{IP отправителя}                                                                   & \multicolumn{8}{c|}{IP получателя}                                                                    &            &            &            &           &            &            \\ \hline
		\end{tabular}
		\label{table:01_Found_intersects_01}
		\caption{IP-адреса отправителя и получателя в пакете UDP}
	\end{center}
\end{table}

Нас интересует только IP-адрес отправителя. В нашем примере это $[74.1b.c7.8a]_{16} \Rightarrow [116.27.199.138]_{10}$ 

В середине пакета UDP находятся данные, которые нам не нужны в рамках нашей задачи.

Время снятия показаний $t$, координаты $X$ и $Y$ находятся в конце пакета UDP (таблица \ref{table:01_Found_intersects_02}).

\begin{table}[h!]
	\begin{center}
		\begin{tabular}{|c|c|c|c|c|c|c|c|c|c|c|c|c|c|c|c|c|c|c|c|c|c|c|c|c|}
			\hline
			байт   & \multicolumn{2}{c|}{20} & \multicolumn{2}{c|}{21} & \multicolumn{2}{c|}{22} & \multicolumn{2}{c|}{23} & \multicolumn{2}{c|}{24} & \multicolumn{2}{c|}{25} & \multicolumn{2}{c|}{26} & \multicolumn{2}{c|}{27} & \multicolumn{2}{c|}{28} & \multicolumn{2}{c|}{29} & \multicolumn{2}{c|}{30} & \multicolumn{2}{c|}{31} \\ \hline
			данные & \textbf{0} & \textbf{0} & \textbf{0} & \textbf{0} & \textbf{0} & \textbf{4} & \textbf{e} & \textbf{b} & \textbf{0} & \textbf{0} & \textbf{0} & \textbf{0} & \textbf{0} & \textbf{3} & \textbf{e} & \textbf{7} & \textbf{0} & \textbf{0} & \textbf{0} & \textbf{0} & \textbf{0} & \textbf{5} & \textbf{2} & \textbf{9} \\ \hline
			& \multicolumn{8}{c|}{время}                                                                            & \multicolumn{8}{c|}{координата X}                                                                     & \multicolumn{8}{c|}{координата Y}                                                                     \\ \hline
		\end{tabular}
		\caption{Время, координаты X и Y робота в пакете UDP}
		\label{table:01_Found_intersects_02}
	\end{center}
\end{table}

Обрабатываем все пакеты данных и отбираем в новый массив (список) IP-адрес отправителя, время, координату $X$, координату $Y$.

Далее по полученным данным проверяем пересечения с другими роботами. Для этого мы проверяем пересечение каждого отрезка пути текущего робота: $(x_i;y_i)$ и $(x_{i-1} y_{i-1})$ с каждым отрезком пути остальных роботов. Если отрезки пересекаются, значит и пути пересеклись. 

Если пересечений не обнаружено, то в качестве ответа выводим $-1$.

Если же пути пересекались, то выводим пару IP-адресов роботов в десятичном формате в порядке возрастания. В случае, если пересекались несколько пар роботов, то эти пары выводим в порядке первых пересечений траекторий роботов.\\

\begin{figure}[h!]
	\centering
	\includegraphics[width=0.7\linewidth]{2nd_tour/irs/task_05/01_Found_intersects_01.pdf}
	\caption{Пример не пересекающихся траекторий роботов}
	\label{fig:01_Found_intersects_01}
\end{figure}

В случае с траекториями движения роботов, как на рис. \ref{fig:01_Found_intersects_01} ответ: $-1$


\begin{figure}[h!]
	\centering
	\includegraphics[width=0.7\linewidth]{2nd_tour/irs/task_05/01_Found_intersects_02.pdf}
	\caption{Пример пересекающихся траекторий роботов}
	\label{fig:01_Found_intersects_02}
\end{figure}

В случае с траекториями движения роботов, как на рис. \ref{fig:01_Found_intersects_02} ответ: \\

\begin{verbatim}
219.166.239.45 222.52.3.223
90.13.181.142 219.166.239.45
102.25.208.228 219.166.239.45
\end{verbatim}


\codeExample

\inputPythonSource
%\inputCPPSource
