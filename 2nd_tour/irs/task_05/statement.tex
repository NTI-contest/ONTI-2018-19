\assignementTitle{Определение пересекающихся траекторий}{15}

Имеется набор байт - трафик, собранный на концентраторе, во время общения в локальной сети нескольких устройств, включая робототехнические устройства.

Известно, что робототехнические устройства общались по протоколу, построенному поверх UDP (\url{https://ru.wikipedia.org/wiki/UDP}) 
и реализованному следующим образом: три целых числа - $t_i$, $X_i$, $Y_i$ - каждое из которых занимает $4$ байта, 
где $t_i$ - время, в которое были сняты данные показатели координат, а $X_i$ и $Y_i$ - 
координаты местоположения робототехнической тележки в данный момент времени. При записи чисел использовалась нотация BigEndian 
(\url{https://en.wikipedia.org/wiki/Endianness}).

Необходимо определить IP-адреса (\url{https://ru.wikipedia.org/wiki/IPv4}) устройств, чьи траектории пересекались.

Считать, что между изменениями робототехнические тележки перемещались прямо. Гарантируется, что через данный концентратор проходит лишь необходимый трафик, т.е. отсутствуют пакеты, не относящиеся к данной задаче.

\inputfmtSection
 
Первая строка содержит 1 целое число: $N$ - количество переданых пакетов через концентратор  $(4 \leq N \leq 10^3)$.

Далее идут $N$ строк, каждая из которых содержит один пакет, переданный через концентратор в 
в шестнадцатеричном формате, который содержит $t_i$, $X_i$, $Y_i$, где:

\begin{itemize}
    \item $t_i$ - время в мс, в которое были сняты данные показатели координат - 
    $4$ байта $(0 \leq t_i < 2^{32})$;
    \item $X_i$ - координаты по оси $X$ местоположения робототехнической тележки в данный момент времени - $4$ байта $(0 \leq X_i < 2^{32})$;
    \item $Y_i$ - координаты по оси $Y$ местоположения робототехнической тележки в данный момент времени - $4$ байта $(0 \leq Y_i < 2^{32})$.
\end{itemize}

\outputfmtSection

Необходимо вывести два IP-адресса в десятичном формате через пробел в порядке их возрастания -- адреса устройств,
чьи траектории пересекались.

В случае если пересекались несколько пар роботов,  эти пары следуют в порядке первых пересечений траекторий
робототехнических устройств.

В случае если таких пересечений нет, следует вывести -1. 

Примечание: решение должно считывать из стандартного ввода и выводить результат в стандартный вывод. 
Примеры входных данных представлены по ссылке (\url{http://bit.ly/2QkcNzv}) и не выводятся в самом задании, поскольку займут много места на экране.

%\includeSolutionIfExistsByPath{2nd_tour/irs/task_05}