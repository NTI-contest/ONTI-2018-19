\assignementTitle{Управление всенаправленной тележкой}{10}

Робот, моторы которого расположены под углом в $120^{\circ}$ друг к другу, движется в заданном направлении 
некоторое время $t$. На валах моторов закреплены омниколёса (\url{https://en.wikipedia.org/wiki/Omni_wheel}). 
С некоторой кинематической моделью робота можно познакомиться по ссылке: (\url{https://bharat-robotics.github.io/blog/kinematic-analysis-of-holonomic-robot/}).

Необходимо определить новые координаты центра робота, если в момент начала движения он находился в точке 
$(0,0)$ и один из двигателей находился на оси $Y$, в направлении положительной части (см. рис.). 
Расположение моторов является постоянным и соответствует рисунку.

Считать, что мощность достигается мгновенно и может подаваться на все моторы одновременно. 
Колёса вращаются без проскальзывания. В случае когда на моторы ничего не подаётся, их скорость равна $0$.

\putImgWOCaption{8cm}{1}

Расположение моторов робота относительно центра глобальной системы отсчета в начальный момент времени

\inputfmtSection
 
Первая строчка содержит четыре целых числа через пробел - $d,$ $p,$ $t,$ $N$, где :

\begin{itemize}
    \item $d$ - диаметр колёс в мм ($30 \leq d \leq 100$);
    \item $p$ - длина оси (осевой балки) от центра робота до колеса в мм ($ 50 \leq p \leq 125$);
    \item $t$ - общее время работы моторов, в с ($10 \leq t \leq 1000$);
    \item $N$ - количество измерений ($1 \leq N \leq 1000$).
\end{itemize}

Далее идут $3$ строки -для первого, второго и третьего моторов, соответственно.

В каждой строке находится $N$ вещественных чисел через пробел - подаваемая на мотор скорость $w_i$, через равные промежутки времени. ($-2$ рад/с $\leq w_i \leq 2$ рад/с).

\outputfmtSection

Одна строка, содержащая два целых числа - координаты центра робота в мм, где он закончил своё движение, через пробел. Допускается погрешность в 1 мм по каждой из координат.

Примечание: решение должно считывать из стандартного ввода и выводить результат в стандартный вывод. 
Примеры входных данных представлены по ссылке (\url{http://bit.ly/2L3bdx3}) и не выводятся в самом задании, поскольку займут много места на экране.

%\includeSolutionIfExistsByPath{2nd_tour/irs/task_05}