\assignementTitle{Двойная идентификация}{5}{}

Специалист, разрабатывающий программное обеспечение, описывает процесс получения наличных денег в банкомате следующим образом:

\begin{enumerate}
    \item Банкомат с самого начала проверяет PIN-код владельца карты. Если PIN-код совпадает, то переходим к следующему пункту.
    \item Пользователь запрашивает определенную сумму.
    \item Банкомат делает запрос к банку, который отвественный за установку данного банкомата.
    \item Банк связывается с платежной системой (VISA, MasterCard, МИР) и передает ей номер карты, по которому определяется банк, выпустивший карту.
    \item Платежная система устанавливает сессию с авторизационным узлом, банка, который выпустил карту.
    \item Авторизационный узел, поскольку имеет доступ к базе банка, определяет для данной карты счет, остаток на счету и шлет ответ, можно ли выдать запрошенную сумму или нет.
\end{enumerate}

В случае, если карта того же банка, что и банкомат, то запроса к платежной системе не происходит.

Если карта магнитная или не работает считыватель чипа карты, то PIN-код в зашифрованном виде (pin block; \url{https://en.wikipedia.org/wiki/PIN_pad}) передается в авторизационный узел. При этом важно заметить, что время пребывания PIN-кода в памяти устройства, считывающего код должно быть минимально, чтобы избегать возможности взлома и поиска PIN-кода в памяти.

Представим, что есть необходимость сделать банкомат для блокчейн сети на базе Ethereum, который бы производил идентификацию пользователей в сети блокчейн с помощью лица. Очевидно, что в этом случае должна быть база данных, которая бы ставила в соответствие некоторый идентификатор человека, опознанного с помощью камеры, и приватного ключа, позволившего бы данному человеку авторизовывать себя в блокчейн сети.

Хранение приватного ключа на каком-то сервере - не очень хорошая идея. Поэтому необходим способ, который бы из идентификатора человека мог бы получить приватный ключ. При этом злоумышленник, даже зная алгоритм и идентификатор, не мог бы произвести подобное преобразвание. Тогда в алгоритме должно использоваться что-то, что не знает злоумышленник, но знает пользователь. Таким образом, снова появляется необходимость использовать PIN-код: пользователь идентифицирует себя с помощью камеры и распознавания лица, но авторизируется на доступ к своим средствам только с помощью PIN-кода.

В качестве примера, рассмотрим следующий алгоритм генерации приватного ключа $K$:
$$K = keccak256(keccak256(keccak256(keccak256(keccak256(''), I, P_1), I, P_2), I, P_3), I, P_4)$$

где $''$ - ''пустая'' последовательность байт, $I$ --- это идентификатор, который возвращает система распознавания по лицу, приведенный к длине в 16 байт, а ($P_1$, $P_2$, $P_3$, $P_4$) --- последовательно введенные четыре цифры PIN-кода, где каждая цифра представлена целым числом длиной 1 байт, $P_1$ --- цифра самого старшего разряда в PIN-коде (первая введенная цифра), а $P_4$ --- цифра самого младшего разряда в PIN-коде (последняя введенная цифра).

Напишите программу, которая бы по идентификатору человека и PIN-коду получала бы баланс счета в сети Sokol (тестовая сеть, совместимая с Ethereum Virtual Machine), принадлежащего данному человека.

\inputfmtSection

Первая строка содержит последовательность из 36 шестнадцатиричных символов –  идентификатор пользователя в виде UUID (Universally Unique Identifier; \url{https://ru.wikipedia.org/wiki/UUID})

Вторая строка содержит четырехзначное число - PIN-код, необходимый для доступа к счету пользователя.

\outputfmtSection

Введите одно число - баланс счета пользователя в Wei.

\subsubsection*{Комментарий}

Если вам неизвестны концепции приватного ключа и адреса сети Ethereum и баланса счета, то обратитесь к 
следующим задачам второго этапа профиля ''Программная инженерия финансовых технологий'' сезона 2016-2017 годов:

\begin{itemize}
    \item Криптография с открытым ключом \url{https://stepik.org/lesson/59926/step/3}
    \item Адреса Ethereum \url{https://stepik.org/lesson/59926/step/4}
    \item Получение баланса \url{https://stepik.org/lesson/62023/step/2}
\end{itemize}

Для конкатенации данных используйте тот же подход, что используется в языке Solidity при вызове abi.encodePacked (\url{https://solidity.readthedocs.io/en/v0.5.1/units-and-global-variables.html#abi-encoding-and-decoding-functions}).

Получение информации из тестовой сети Sokol может происходить без необходимости синхронизировать свой собственный узел сети. Вместо этого можно отправлять JSON-RPC запросы на URL: \url{https://sokol.poa.network}.

\begin{myverbbox}[\small]{\vinput} 
    a52b5033-35d1-4aa6-8190-72f0116edba3
    1741
\end{myverbbox}
\begin{myverbbox}[\small]{\voutput}
    83332335671782
\end{myverbbox}
\inputoutputTable

%\includeSolutionIfExistsByPath{2nd_tour/fintex/task_10}