\assignementTitle{Аффинные преобразования}{5}

Ранние алгоритмы распознования лиц требовали, чтобы распозноваемое лицо располагалось в положении ''анфас'' (\url{https://bit.ly/2yY7iwA}). 
Очевидно, что для такое условие можно соблюсти в тех случаях, когда есть возможность явно попросить 
распознаваемого встать определенным образом и смотреть в камеру. В большинстве случаев, применимых к реальной 
жизни, приходится работать с изображением лица, которое может быть повернуто относительно фокальной плоскости 
камеры.

Для решения данной проблемы, предлагалось использовать афинные преобразования (\url{https://bit.ly/2QxNoz5}, 
\url{https://compgraphics.info/2D/affine_transform.php}).

После применения преобразований ключевые точки лица (например, кончик носа, уголки губ и глаз) 
располагались на тех местах изображения, где их мог бы корректно обработать следующий уровень 
алгоритма распознования (\url{https://bit.ly/2zICQbH}).

\subsubsection*{Microsoft Face API}

Сервис распознования лиц Microsoft Face API (\url{https://azure.microsoft.com/en-us/services/cognitive-services/face/}) для каждого изображения лица, 
загруженного на сервис может вернуть набор данных в формате JSON, описывающий расположение на изображении 
ключевых точек лица.

\putImgWOCaption{13cm}{1}

Координаты ключевых точек из набора данных, возвращаемых MS Face API, 
указывают в каком месте изображения находится тот или иной элемент лица. Следовательно, если лицо повернуто, то, 
например, координата $X_{UnderLipBottom}$ нижней части нижней губы будет смещена влево или вправо относительно 
координаты $X_{NoseTip}$ кончика носа.

Напишите программу, которая бы определяла положение ключевых 
точек лица после применения афинных преобразований, если известно, что эталонные координаты ключевых точек 
eyeLeftOuter (внешний уголок левого глаза), noseTip (кончик носа)и eyeRightOuter 
(внешний уголок правого глаза) должны быть следующие:

\begin{itemize}
    \item eyeLeftOuter: (252, 331)
    \item noseTip: (520, 634)
    \item eyeRightOuter: (782, 321)
\end{itemize}

$P_{x,y}=145$.

\inputfmtSection

На вход приходит структура в формате JSON с координатами четырех ключевых точек лица. Среди них есть точно координаты для точек eyeLeftOuter, noseTip и eyeRightOuter, а также четверта точка - одна из тех, что представлены на рисунке выше (возвращаемые MS Face API).

\outputfmtSection

Выведите два целых числа в одной строке - координату $X$ и $Y$ четвертой ключевой точки, подразумевая, что к ней применено такое преобразование, при котором точки eyeLeftOuter, noseTip и eyeRightOuter расположатся в эталонных координатах.

\subsubsection*{Примечание}

При реализации подавления шума используйте плавающее окно $3 \times 3$. При получении не целых результатов - отбрасывайте дробную часть.

\begin{myverbbox}[\small]{\vinput} 
    [{"faceLandmarks": {"eyeLeftOuter": {"x": 224.1, "y": 199.0}, ...]
\end{myverbbox}
\begin{myverbbox}[\small]{\voutput}
    645 670
\end{myverbbox}
\inputoutputTable

%\includeSolutionIfExistsByPath{2nd_tour/fintex/task_03}