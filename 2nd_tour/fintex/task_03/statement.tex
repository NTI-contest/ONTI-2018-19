\assignementTitle{Аффинные преобразования}{3}

Ранние алгоритмы распознования лиц требовали, чтобы распозноваемое лицо располагалось в положении ''анфас'' (\url{https://bit.ly/2yY7iwA}).
Очевидно, что для такое условие можно соблюсти в тех случаях, когда есть возможность явно попросить распознаваемого встать определенным образом и смотреть в камеру.
В большинстве случаев, применимых к реальной жизни, приходится работать с изображением лица, которое может быть повернуто относительно фокальной плоскости камеры.

Для решения данной проблемы, предлагалось использовать афинные преобразования (\url{https://bit.ly/2QxNoz5}, \url{https://compgraphics.info/2D/affine_transform.php}).

\putImgWOCaption{15cm}{affine_transformations_for_face}

После применения преобразований ключевые точки лица (например, кончик носа, уголки губ и глаз) располагались на тех местах изображения, где их мог бы корректно обработать следующий уровень алгоритма распознования (\url{https://bit.ly/2zICQbH}).

\textbf{Microsoft Face API}

Сервис распознования лиц \textit{Microsoft Face API} (\url{https://azure.microsoft.com/en-us/services/cognitive-services/face/}) для каждого изображения лица, загруженного на сервис может вернуть набор данных в формате JSON, описывающий расположение на изображении ключевых точек лица.

\putImgWOCaption{8cm}{landmarks}

Координаты ключевых точек из набора данных, возвращаемых \textit{MS Face API}, указывают в каком месте изображения находится тот или иной элемент лица.
Следовательно, если лицо повернуто, то, например, координата $X_{UnderLipBottom}$ нижней части нижней губы будет смещена влево или вправо относительно координаты $X_{NoseTip}$ кончика носа.

Напишите программу, которая бы определяла положение ключевых точек лица после применения афинных преобразований, если известно, что эталонные координаты ключевых точек \textit{eyeLeftOuter} (внешний уголок левого глаза), \textit{noseTip} (кончик носа)и \textit{eyeRightOuter} (внешний уголок правого глаза) должны быть следующие:

\begin{table}[H]
\centering
\begin{tabular}{|l|c|}
\hline
eyeLeftOuter  & (252, 331) \\ \hline
noseTip       & (520, 634) \\ \hline
eyeRightOuter & (782, 321) \\
\hline
\end{tabular}
\end{table}

\inputfmtSection

На вход приходит структура в формате JSON с координатами четырех ключевых точек лица. Среди них есть точно координаты для точек \textit{eyeLeftOuter}, \textit{noseTip} и \textit{eyeRightOuter}, а также четверта точка - одна из тех, что представлены на рисунке выше (возвращаемые \textit{MS Face API}).

\outputfmtSection

Выведите два целых числа в одной строке - координату $X$ и $Y$ четвертой ключевой точки, подразумевая, что к ней применено такое преобразование, при котором точки \textit{eyeLeftOuter}, \textit{noseTip} и \textit{eyeRightOuter} расположатся в эталонных координатах.

\exampleSection

\sampleTitle{1}

\begin{myverbbox}[\small]{\vinput}
[{"faceLandmarks": {"eyeLeftOuter": {"x": 224.1, "y": 199.0}, 
"noseTip": {"x": 286.9, "y": 254.6}, "eyeRightOuter": {"x": 
355.2, "y": 195.6}, "noseRightAlarOutTip": {"x": 317.6, "y": 
261.1}}}]
\end{myverbbox}
\begin{myverbbox}[\small]{\voutput}
645 670
\end{myverbbox}
\inputoutputTable

\includeSolutionIfExistsByPath{2nd_tour/fintex/task_03/solution}
