\solutionSection

Несмотря на то, что рекоммендовано воспользоваться MS Face API, есть способ проще, который не требует обращений к сторонним сервисам.
Необходимо для каждого кадра сформировать хэш, который является его численным представлением и далее сравнивать эти хэши, чтобы найти разные кадры в видеоряде.
Даже если два кадра на видео выглядят одинаковыми, они могут содержать различающиеся пиксели, в силу специфики работы алгоритмов сжатия видео, например mp4. Значит, обычные криптографические алгоритмы не смогут нам помочь в решении данной задачи, так как они очень чувствительны к изменениям входных данных, и будут возвращать абсолютно разные хэши, даже если два изображения имеют всего один различающийся бит.
Поэтому, мы воспользуемся перцептивными хэшами, а именно алгоритмом хэша по среднему (average hash). Узнать больше про такие алгоритмы можно здесь: https://habr.com/post/120562/, https://www.pyimagesearch.com/2017/11/27/image-hashing-opencv-python/, www.hackerfactor.com/blog/index.php?/archives/529-Kind-of-Like-That.html.
Для работы с видеорядом, воспользуемся модулем cv2.

\codeExample

\inputPythonSource
