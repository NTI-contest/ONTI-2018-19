\assignementTitle{Простая обработка видео-потока}{3}

Современные системы распознавания лиц работают как со статическими изображениями, когда делается снимок, с которым работает система, так и с видео потоком, когда извлекаются кадры для распознавания.

Статические системы могут применяться там, где есть дополнительная проверка на подлинность человека - оператор банка или кассир могут удостовериться, что перед ними живой человек, а не фотография или манекен.

Системы, работающие с видео потоками, используются там, где нет возможности применять дополнительные проверки оператором-человеком, вместо этого алгоритм автоматически должен распознавать факты мошенничества.

Для выполнения следующего задания - рекомендация - познакомиться с сервисом \textit{MS Face API} (\url{https://docs.microsoft.com/ru-ru/azure/cognitive-services/face/overview}). Данный сервис работает через REST API. Это значит, что вся обработка и распознавание изображения происходит в облаке Azure, а программисту необходимо загружать туда данные или получать результаты, используя HTTP POST запросы.

Сервис MS Face API может следующее:
\begin{itemize}
  \item \textbf{Обнаружение лиц} --- выявляет лица на изображениях и возвращает координаты прямоугольника, в котором они расположены, извлекает ряд атрибутов, связанных с лицом, например поза, пол, возраст, положение головы. 
  \item \textbf{Проверка лиц} --- оценивает, принадлежат ли два лица одному человеку.
  \item \textbf{Поиск похожих лиц} --- сравнивает целевое лицо и набор потенциальных лиц для поиска и находит небольшое количество лиц, очень похожих на целевое.
  \item \textbf{Группировка лиц} --- делит неизвестные лица на несколько групп, основываясь на сходстве.
  \item \textbf{Идентификация личности} --- позволяет идентифицировать новое обнаруженное лицо путем сравнения с базой данных заранее загруженных групп лиц.
\end{itemize}

Напишите программу, которая бы позволяла выполняла обработку отдельных кадров видео-файла с фотографиям лиц.

\inputfmtSection

Ссылка на Google Drive, где необходимо скачать видео-файл.  

\outputfmtSection

Целое число --- сколько раз сменялись фотографии людей.

\exampleSection

\sampleTitle{1}

\footnotesize
\begin{myverbbox}[\small]{\vinput}
https://drive.google.com/open?id=16nKi8GgNgLLRd_Z0JjYMjBsaJi3Hp8oz
\end{myverbbox}
\begin{myverbbox}[\small]{\voutput}
190
\end{myverbbox}
\inputoutputTable
\normalsize

\commentsSection
Платформа Stepik позволяет отслеживать попытки решать задачу перебором. Такие попытки будут приводить к нулевым баллам за данную задачу.

\includeSolutionIfExistsByPath{2nd_tour/fintex/task_07/solution}
