\assignementTitle{Гистограммы направленных градиентов}{3}

Гистограммы направленных градиентов (Histagrams of oriented gradients, HOG, \url{https://bit.ly/2nBnUqS}) используются для обнаружения объектов на изображении.
Например, целиком человека (\url{https://bit.ly/2RE1eAi}) или его лица (\url{https://bit.ly/2zASMKI}).

Если кратко, то алгоритм построения гистограмм градиентов можно описать следующим образом
\begin{itemize}
  \item Для каждого пикселя изображения определяется вектор (градиент), определяющий в какую сторону происходит изменение яркости;
  \item Векторы группируются в наборы ($X \times X$);
  \item Внутри каждого набора градиенты, направленные примерно в одну сторону, объединяются - получается несколько групп градиентов - гистограмма, по которой можно сказать какое направление преобладает у пикселей одного набора;
  \item Гистограммы градиентов нормализуются по яркости, чтобы сделать гистограмму независимой от яркости изначального изображения;
  \item Нормализованные гистограммы градиентов сверяются с шаблоном (возможно, с применением алгоритмов машинного обучения), чтобы определить наличие на изначальном изображении объекта, определяемого шаблоном. 
\end{itemize}

Более подробно алгоритмы описаны на следующих ресурсах:
\begin{itemize}
  \item Статья о вычислении векторов (градиентов): \url{http://mccormickml.com/2013/05/07/gradient-vectors/}
  \item Статья о построении гистограммы градиентов и нормализации: \url{http://mccormickml.com/2013/05/09/hog-person-detector-tutorial/} 
  \item Пошаговое описание построения гистограммы градиентов для всей картинки: \url{https://www.learnopencv.com/histogram-of-oriented-gradients/}
\end{itemize}

Напишите программу, которая бы используя алгоритм построения гистограмм направленных градиентов, определала бы лежит или стоит человек на изображении.

\inputfmtSection

В первой строке входного файла содержится два числа $W$ и $H$ - размер изображение по горизонтали и вертикали ($100 \le W, H \le 200$).

Последующая строка содержит $W \times H$ наборов шестнадцатиричных символов. Каждый набор - пиксель, описанный в модели $RGB$, где первые (старшие) два символа кодируют красную компоненту пикселя, следующие два символа кодируют зеленую компоненту, а последние (младшие) два символа кодируют синюю компоненту.
Первые $W$ наборов - это первая строка изображения, следующие $W$ наборов - вторая строка и т.д.

\outputfmtSection

Выведите одно число:
\begin{itemize}
  \item $0$ -- если человека на картинке нет
  \item $1$ -- если человек стоит
  \item $2$ -- если человек лежит
\end{itemize}

\exampleSection

\sampleTitle{1}
\url{https://drive.google.com/file/d/11INAwaWig6HQ9UIZmp-jZRorBDdfQuZ6/view?usp=sharing}

\sampleTitle{2}
\url{https://drive.google.com/file/d/1GQoS3CXXJHV7kDIs4HorJM-H5x_xQlIh/view?usp=sharing}

\sampleTitle{3}
\url{https://drive.google.com/file/d/1p0JRD0sJyOgiIjR0cCMGnli3EZj4zvHL/view?usp=sharing}

\sampleTitle{4}
\url{https://drive.google.com/file/d/1cRUC6OWawj0X-t4XEY_sfuAoDMwMCQAX/view?usp=sharing}

\commentsSection

Если на изображении присутствует человек, то руки будут вытянуты вдоль тела, а ноги будут прямые.

Пол человека может быть любой.
Фон будет в среднем однородный (трава, листья, обои, камень и т.п.). Никаких других крупных объектов, кроме человека на изображении не будет.

\includeSolutionIfExistsByPath{2nd_tour/fintex/task_02/solution}
