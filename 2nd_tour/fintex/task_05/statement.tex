\assignementTitle{Язык структурированных запросов}{5}

С самых ранних времен финансовые отношения всегда ассоциировались с информацией. Объемы информации росли, уже невозможно было использовать только бухгалтерскую книгу (гроссбух, ledger). А затем, товаро-денежные отношения между участниками усложнились, что выборка информации, определение цен, подведение итогов и т.п. стала невозможна вручную. Между тем, типовые операции с данными, выполняемые в рамках таких отношений, демонстрировали очевидность их автоматизации.

Сначала появились реляционные базы данных (\url{https://bit.ly/2IZvBjL}). А затем для работы с 
данными в базах данных с реляционной моделью был придуман и стандартизирован язык структурированный запросов 
(structured query language, SQL; \url{https://bit.ly/2kip6IQ}).

С точки зрения языка SQL данные хранятся в таблицах. Команды языка SQL позволяют управлять таблицами, изменять их содержимое, читать данные из таблиц.

Например, команда

\begin{minted}[fontsize=\footnotesize, linenos]{sql}
    SELECT * FROM Payments;
\end{minted}

выведет все строки (записи) из таблицы Payments, а команда

\begin{minted}[fontsize=\footnotesize, linenos]{sql}
    SELECT Sender, Receiver FROM Payments WHERE (Value > 50);
\end{minted}

выведет только отправителя и получателя для тех записей, где сумма платежа больше 50.

Ознакомьтесь с основами языка SQL в курсе ''Базы данных'' \url{https://stepik.org/2614}, (модули 4 - 8) и найдите решение к следующей задаче.

\subsubsection*{Формулировка задачи}

В реляционной базе данных есть две таблицы. transactions указаны транзакции пользователей (три столбца: sender, recipient, val). Другая таблица result используется для промежуточного хранения результата (один столбец: res).

Вставьте в таблицу result только одну строку, в которой будет подсчитан конечный баланс пользователя ''Frank''. Начальный баланс пользователя считать 0.

Примечание: нужно написать такую одну сложносоставную команду на языке SQL, которая бы как запрашивала 
нужные данные из таблицы transactions, так и вставляла бы данные в таблицу result. Из-за особенностей 
проверки задач на SQL-запросы платформы Stepik, даже правильные ответы будут проверяться на полное решение. 
Т.е. ответы подобные

\begin{minted}[fontsize=\footnotesize, linenos]{sql}
    INSERT INTO result (res) VALUES (число);
\end{minted}

приниматься не будут.

\subsection*{Пример}

Чтобы попрактиковаться в составлении запроса, вставьте следующий код в онлайн-консоль для тестирования SQL команд (\url{https://www.tutorialspoint.com/execute_sql_online.php}):

\begin{minted}[fontsize=\footnotesize, linenos]{sql}
    BEGIN TRANSACTION;

    CREATE TABLE transactions (sender VARCHAR(20), recipient VARCHAR(20), val INT);
    INSERT INTO transactions (sender, recipient, val) VALUES ('Alice', 'Frank', 10);
    INSERT INTO transactions (sender, recipient, val) VALUES ('Alice', 'Frank', 20);
    INSERT INTO transactions (sender, recipient, val) VALUES ('Clare', 'Frank', 5);
    INSERT INTO transactions (sender, recipient, val) VALUES ('Clare', 'Alice', 7);
    INSERT INTO transactions (sender, recipient, val) VALUES ('Frank', 'Alice', 3);
    
    CREATE TABLE result (res INT);
    COMMIT;
    
    /* Данный запрос должен быть изменен на тот, что будет использоваться в 
       Stepik в качестве ответа */
    INSERT INTO result (res) VALUES(32);
    /* Конец запроса для Stepik */
    
    SELECT * FROM result;
\end{minted}

Исполните данный запрос, нажав кнопку ''Execute''.

Затем измените команду ''INSERT INTO result (res) VALUES (32);'' на ту, что будете использовать в качестве ответа.