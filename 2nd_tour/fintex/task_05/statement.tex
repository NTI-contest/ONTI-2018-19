\assignementTitle{Язык структурированных запросов}{3}

С самых ранних времен финансовые отношения всегда ассоциировались с информацией. Объемы информации росли, уже невозможно было использовать только бухгалтерскую книгу (гроссбух, ledger). А затем, товаро-денежные отношения между участниками усложнились, что выборка информации, определение цен, подведение итогов и т.п. стала невозможна вручную. Между тем, типовые операции с данными, выполняемые в рамках таких отношений, демонстрировали очевидность их автоматизации.     

Сначала появились реляционные базы данных (\url{https://bit.ly/2IZvBjL}). А затем для работы с данными в базах данных с реляционной моделью был придуман и стандартизирован язык структурированный запросов (structured query language, SQL, \url{https://bit.ly/2kip6IQ}).

С точки зрения языка SQL данные хранятся в таблицах. Команды языка SQL позволяют управлять таблицами, изменять их содержимое, читать данные из таблиц.

Например, команда

\begin{minted}[fontsize=\footnotesize]{sql}
SELECT * FROM Payments;
\end{minted}

выведет все строки (записи) из таблицы \texttt{Payments}, а команда 

\begin{minted}[fontsize=\footnotesize]{sql}
SELECT Sender, Receiver FROM Payments WHERE (Value > 50);
\end{minted}

выведет только отправителя и получателя для тех записей, где сумма платежа больше $50$.

Ознакомьтесь с основами языка SQL в курсе ''Базы данных'' (\url{https://stepik.org/2614}, модули $4$ - $8$) и найдите решение к следующей задаче:

В реляционной базе данных есть две таблицы. В одной таблице \texttt{transactions} указаны транзакции пользователей (три столбца: \texttt{sender}, \texttt{recipient}, \texttt{val}). Другая таблица (\texttt{result}) используется для промежуточного хранения результата (один столбец: \texttt{res}).

Вставьте в таблицу \texttt{result} только одну строку, в которой будет подсчитан конечный баланс пользователя \textit{''Frank''}. Начальный баланс пользователя считать 0.  

\commentsSection

Нужно нужно написать такую одну сложносоставную команду на языке SQL, которая бы как запрашивала нужные данные из таблицы \texttt{transactions}, так и вставляла бы данные в таблицу \texttt{result}. Из-за особенностей проверки задач на SQL-запросы платформы Stepik, даже правильные ответы будут проверяться на полное решение. Т.е. ответы подобные \texttt{INSERT INTO result (res) VALUES (число);} приниматься не будут. 

\exampleSection

\sampleTitle{1}

Чтобы попрактиковаться в составлении запроса, вставьте следующий код в онлайн-консоль для тестирования SQL команд \url{https://www.tutorialspoint.com/execute_sql_online.php}:

\begin{minted}[fontsize=\footnotesize]{sql}
BEGIN TRANSACTION;

CREATE TABLE transactions (sender VARCHAR(20), recipient VARCHAR(20), val INT);
INSERT INTO transactions (sender, recipient, val) VALUES ('Alice', 'Frank', 10);
INSERT INTO transactions (sender, recipient, val) VALUES ('Alice', 'Frank', 20);
INSERT INTO transactions (sender, recipient, val) VALUES ('Clare', 'Frank', 5);
INSERT INTO transactions (sender, recipient, val) VALUES ('Clare', 'Alice', 7);
INSERT INTO transactions (sender, recipient, val) VALUES ('Frank', 'Alice', 3);

CREATE TABLE result (res INT);
COMMIT;

/* Данный запрос должен быть изменен на тот, что будет использоваться в 
   Stepik в качестве ответа */
INSERT INTO result (res) VALUES(32);
/* Конец запроса для Stepik */

SELECT * FROM result;
\end{minted}

Исполните данный запрос, нажав кнопку ''Execute''.

Затем измените команду \texttt{INSERT INTO result (res) VALUES (32);} на ту, что будете использовать в качестве ответа.

\includeSolutionIfExistsByPath{2nd_tour/task_05/solution}
