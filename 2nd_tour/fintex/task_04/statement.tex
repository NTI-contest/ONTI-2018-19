\assignementTitle{Идентификация по лицу}{10}

Одним из подходов, используемым для распознования лиц, является измерение расстояний между ключевыми элементами лица - глазами, носом, ртом и т.п. Для этого между одними и теми же точками в эталонном и сравниваемом изображении считается расстояние (длина вектора). Если сумма таких расстояний большая, то, скорее всего, на изображениях лица двух разных людей.

При этом нужно помнить, что поскольку расстояние будет зависеть от масштаба изображения, а также от поворота лица относительно оптической оси камеры или фокальной плоскости, то исходный набор ключевых точек нужно нормализовать: центрировать, привести к одному масштабу и повернуть на один и тот же угол.

Напишите программу, которая бы по эталонному набору точек, описывающих лицо какого-то человека в формате MS Face API, определяла бы какой из трех дополнительных наборов ключевых точек лица принадлежит этому же человеку.

\inputfmtSection

На вход приходит структура в формате JSON, содержащяя четрые набора ключевых точек в в формате MS Face API.

\outputfmtSection

Выведите одно число $0$, $1$ или $2$, в зависимости от того какой из первых трех наборов ключевых точек описывает лицо того же человека, что и в послденем - четвертом наборе.

\begin{myverbbox}[\small]{\vinput} 
    [{"faceRectangle":{"top":228,"left":119,"width":311,"height":311},
    "faceLandmarks":{"pupilLeft":{"x":210,"y":306}, ... ]
\end{myverbbox}
\begin{myverbbox}[\small]{\voutput}
    0
\end{myverbbox}
\inputoutputTable

%\includeSolutionIfExistsByPath{2nd_tour/fintex/task_04}