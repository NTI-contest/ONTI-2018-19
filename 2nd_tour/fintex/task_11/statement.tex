\assignementTitle{UTXO в смарт-контракте}{15}{}

Модель UTXO (Unspent Transaction Output) известна тем, что используется в блокчейн Bitcoin. Все Bitcoin транзакции образуют связные списки: каждая транзакция, которая списывает средства со счета пользователя явно связана с другой транзакцией, в рамках которой эти средства на счете появились. При этом, очевидно, что транзакция начисляющая на счет средства, будет одновременно и транзакцией списывающей с какого-то счета средства. Исключением являются, так называемые, coinbase транзакции - транзакции начисляющие на счет пользователя средства, полученные в качестве вознаграждения за выпуск (майнинг) нового блока.

Поскольку для формирования данных одной транзакции на списание средств используется идентификатор другой транзакции, начислявшей средства, а данные полученной транзакции используются для вычисления хэша транзакции - ее идентификатора, то получается такой связный список, изменение элементов которого задним числом приведет к потере связности. Т.е. помимо связи на уровне блоков в блокчейн Bitcoin имеется связь на уровне транзакций.

Если транзакция, начисляющая средства на счет, еще не была использована для организации связного списка, а иными словами, даныые средства еще не списывались, то такая транзакция называется непотраченная (unspent). А имея в виду то, что внутри одной списывающей транзакции (input) можно начислять средства сразу на несколько счетов (каждое такое начисление - output), то список всех непотраченных начислений называет - непотраченные выходы траназкций (unspent transaction output).

Подробнее об входах и выходах в транзакциях Bitcoin можно почитать в книге ''Mastering Bitcoin'':

\begin{enumerate}
    \item Основы транзакций в Bitcoin (\url{https://bit.ly/2SKOOHs})
    \item Детальное описание транзакций (\url{https://bit.ly/2QRlIIQ})
    \item Про coinbase транзакции (\url{https://bit.ly/2SKoBJ8})
\end{enumerate}

Несмотря на сложности в виде необходимости обхода всей цепочки блоков для подсчета баланса счета, у UTXO модели есть ряд преимуществ:

\begin{itemize}
    \item Благодаря выстроенным цепочкам можно отслеживать какие средства откуда пришли и куда ушли. Даже не расскрывая принадлежность Bitcoin адресов частным лицам или организациям можно отслеживать и анализировать перемещение средств.
    \item Отдельные непотраченные выходы могут быть помечены, как использующиеся в платежных каналах (Lightening Network), а следовательно можно с одного аккаунта совершать параллельные оффчейн платежи без опасений двойных трат.
\end{itemize}

Технология, на которой построен блокчейн Ethereum, не подразумевает использование UTXO модели. Вместо этого используется модель состояний, когда перевод средств со счета на счет или изменение данных контракта формируют новое состояние конкретного аккаунта в частности и всего блокчейна в целом.Тем не менее, UTXO модель можно реализовать в ценностях, которые строятся на основе смарт-контрактов Ethereum, иногда такие ценности называют токенами. Тогда преимущества, перечисленные выше, будет применимы и к таким токенам.Ниже, представлен код на языке Solidity для токена, использующего модель UTXO.

\begin{minted}[fontsize=\footnotesize, linenos]{c++}
    pragma solidity ^0.5.1;

    contract UTXOBasedToken {
        event Transfer(bytes32 indexed tx_source, bytes32 indexed tx_address, address indexed recipient, uint256 value, uint256 vout);
    
        address owner;
    
        struct Transaction {
            address recipient;
            uint256 value;
        }
    
        uint256 coinbaseSeq = 0;
        mapping (bytes32 => Transaction) utxoPool;
        
        constructor() public {
            require(msg.sender != address(0));
            owner = msg.sender;
        }
    
        function transfer(bytes32 _txHash, uint256 _vout, address[] memory _recipients, uint256[] memory _values) public {
            require(_recipients.length == _values.length);
            require(_recipients.length<=20);
            uint256 total;
            bytes32 db_key = keccak256(abi.encodePacked(_txHash, _vout));
            require(utxoPool[db_key].recipient == msg.sender);
            uint256 utxo_value = utxoPool[db_key].value;
    
            bytes32 newTxHash = keccak256(abi.encodePacked(_txHash, _vout, _recipients, _values));
    
            for(uint256 vout=0; vout<_recipients.length; vout++) {
                require(_recipients[vout] != address(0));
                bytes32 new_db_key = keccak256(abi.encodePacked(newTxHash, vout));
                utxoPool[new_db_key] = Transaction(_recipients[vout], _values[vout]);
                require(total < total+_values[vout]);
                total += _values[vout];
                emit Transfer(_txHash, newTxHash, _recipients[vout], _values[vout], vout);
            }
    
            require(total == utxo_value);
            delete utxoPool[db_key];
        }
       
        function() external payable {
            require(msg.value > 0);
            bytes32 txHash = keccak256(abi.encodePacked(msg.sender, msg.value, coinbaseSeq));
            bytes32 db_key = keccak256(abi.encodePacked(txHash, uint256(0)));
            utxoPool[db_key] = Transaction(msg.sender, msg.value);
            coinbaseSeq++;
            emit Transfer(bytes32(0), txHash, msg.sender, msg.value, uint256(0));
        }
        
        function withdraw(uint256 _value) public {
            require(msg.sender == owner);
            require(address(this).balance >= _value);
            msg.sender.transfer(_value);
        }
    }
\end{minted}

Если известно, что данный контракт зарегистрирован по адресу\\ 0xe87a3686b0a42d66eee76d48c9a8307c27d14d1c (\url{https://blockscout.com/poa/sokol/address/0xe87a3686b0a42d66eee76d48c9a8307c27d14d1c}) 
в сети Sokol (тестовая сеть, совместимая с Ethereum Virtual Machine), 
напишите программу, которая бы позволяла для конкретного пользователя контракта отследить в каких блоках происходила покупка токенов, составляющих текущий баланс данного пользователя.

\inputfmtSection

Одна строка - адрес пользователя контракта - последовательность начинающаяся с 0x, за которыми следует 40 шестнадцатиричных символов.

\outputfmtSection

Строка - номера блоков, разделенные пробелом, перечисленные в порядке возрастания.

\subsubsection*{Комментарий}

Если вам неизвестна концепция токенов, то можете обратиться к задачае ''Получение баланса ERC-20 token'' (\url{https://stepik.org/lesson/62024/step/4}) второго этапа профиля ''Программная инженерия финансовых технологий'' сезона 2016-2017 годов.

Для решения данной задачи необходимо познакомиться с языком написания Ethereum смарт-контрактов Solidity (\url{https://solidity.readthedocs.io}).

Получение информации из тестовой сети Sokol может происходить без необходимости синхронизировать свой собственный узел сети. Вместо этого можно отправлять JSON-RPC запросы на URL: \url{https://sokol.poa.network}.

\begin{myverbbox}[\small]{\vinput} 
    0x2C7Cc50973b57b2b03f569643e4e604977D4F7fC
\end{myverbbox}
\begin{myverbbox}[\small]{\voutput}
    6070845 6071386 6071669 6071796
\end{myverbbox}
\inputoutputTable

%\includeSolutionIfExistsByPath{2nd_tour/fintex/task_11}