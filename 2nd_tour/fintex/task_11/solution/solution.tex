\solutionSection

Для решения задачи воспользуемся механизмом фильтров в библиотеке web3.

Первым делом, найдем все входящие транзакции пользователя, отфильтровав события \textit{Transfer} по \textit{\{recepient: givenAddress\}}. Затем, среди всех событий выберем только относящиеся к еще непотраченным транзакциям, то есть те, для которых существует запись в \textit{utxoPool}(Для этого можно использовать функцию \textit{web3.eth.}\\\textit{getStorageAt}). По всем оставшимся событиям \textit{Transfer} пройдем рекурсивно по цепочке до транзакции покупки токенов, используя фильтр событий по \textit{\{tx\_address: event.tx\_source\}}, пока \textit{tx\_source != 0x000...00}. Для получения конечного ответа получим все номера блоков, оставим только уникальные, запишем в возрастающем порядке.

\codeExample

\inputPythonSource