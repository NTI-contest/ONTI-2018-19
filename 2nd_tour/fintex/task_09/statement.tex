\assignementTitle{Угол поворота лица}{10}{}

Не всегда задача распознования лица сводится только к определению того, как расположены его части: нос, глаза, губы, брови. Иногда бывает необходимо ответить на вопрос - куда повернуто лицо. Например, таким образом можно сказать в какую сторону смотрит человек или общаются ли два человека между собой.

Если взять некоторые точки на лице человека (например, глаза, уголки губ, нос), то приняв их за вершины, можно построить многогранник.

\putImgWOCaption{10cm}{1}

Для простоты, допустим, что многогранник - пирамида, в основании которой - квадрат.

\putImgWOCaption{10cm}{2}

Будем счиать, что если лицо направлено в камеру, то основание пирамиды лежит в плоскости, параллельной фокальной плоскости камеры (параллельной плоскости светочувствительной матрицы). Очевидно, что проекция основания пирамиды на кадр - квадрат, а вершина пирамиды будет лежать на пересечении диагоналей квадрата.

Тогда, если человек повернет свое лицо, например, вверх, то проеция пирамиды на кадр изменится: квадрат может превратиться в просто многоугольник, а вершина пирамиды больше не будет лежать на диагоналях многугольника.

\putImgWOCaption{15cm}{3}

Напишите программу, которая бы по проекции правильной пирамиды на кадр могла бы определить на какой угол $\alpha$ она была повернута.

\inputfmtSection

В первой строке - целое число - высота $h$ правильной пирамиды в миллиметрах. 

В следующих пяти строках по два целых числа $x_i$, $y_i$ - координаты проекции пирамиды на плоскость кадра. 
Для простоты считать, что расстояние между центрами соседних пикселей - 1 миллиметр. Порядок следования вершин - 
неопределен.

\outputfmtSection

Введите одно число - угол поворота пирамиды относительно оси, проходившей до поворота через вершину пирамиды и центр квадрата ее основания. Угол укажите в радианах с точностью до сотых.

\begin{myverbbox}[\small]{\vinput} 
    230
    191 567
    84 371
    85 688
    397 380
    398 697
\end{myverbbox}
\begin{myverbbox}[\small]{\voutput}
    0.2600160690181619
\end{myverbbox}
\inputoutputTable

%\includeSolutionIfExistsByPath{2nd_tour/fintex/task_09}