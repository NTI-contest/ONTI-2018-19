\assignementTitle{Обнаружение состояния элементов лица}{10}{}

Чтобы еще более усилить алгоритмы распознования настоящего лица человека, можно просить пользователя не только изменять положение головы, но и подмигнуть, открыть и закрыть рот.

\putImgWOCaption{10cm}{1}

Если набор таких действий задать друг за другом, - например, "закройте правый глаз, 
откройте рот, откройте правый глаз, закройте левый глаз, закройте рот", а человек все 
корректно повторит, то это позволит не только убедиться, что перед нами живой человек, 
но и новые данные, получаемые в ходе этой процедуры позволят уточнить - правильно ли вообще произолшла идентификация - тот ли человек пытается пройти идентификацию.

\putImgWOCaption{10cm}{2}

Поскольку ограниченные возможности MS Face API уже не могут реализовать подобные проверки, найдите способ, используя библиотеку OpenCV, воплотить проверки открыт-закрыт у человека глаз, открыт или закрыт рот.

\inputfmtSection

В первой строке - одно из четырех возможных словосочетаний, определяющее какое действие будет проверяться от пользователя:

\begin{itemize}
    \item close right eye - закрыть правый глаз
    \item close left eye - закрыть левый глаз
    \item close both eyes - закрыть оба глаза
    \item open mouth - открыть рот
\end{itemize}

Следующая строка - ссылка на Google Drive, откуда нужно загрузить архив с фотографиями, которые нужно обработать.

\outputfmtSection

Выведите в одну строку последовательность из номеров фотографий, в которых человек выполняет заданное действие. Номера фотографий перечислены по возрастанию и разделены одним пробелом.

\begin{myverbbox}[\small]{\vinput} 
    close both eyes
    https://drive.google.com/open?id=1vfYapd7kfTqAk3iZ8ragBnnjGFf6QpNR
\end{myverbbox}
\begin{myverbbox}[\small]{\voutput}
    4 7 13 15 22 45 51 60 66 68 76 83
\end{myverbbox}
\inputoutputTable

\begin{myverbbox}[\small]{\vinput} 
    open mouth
    https://drive.google.com/open?id=1MrEQTSJakfnOUbylrgT50YBsl04wBDxj
\end{myverbbox}
\begin{myverbbox}[\small]{\voutput}
    0 2 13 19 22 29 33 35 43 48 54 67 74 75 80 82 86 91
\end{myverbbox}
\inputoutputTable

%\includeSolutionIfExistsByPath{2nd_tour/fintex/task_13}