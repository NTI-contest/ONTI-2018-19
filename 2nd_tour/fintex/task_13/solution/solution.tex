\solutionSection

Для решения задачи воспользуемся библиотекой dlib и shape-predictor'ом на 68 точек.

Для каждого глаза и рта определим нужные точки и вычислим AR (aspect ratio), по аналогии с этой статьёй: \url{https://www.pyimagesearch.com/2017/04/24/eye-blink-detection-opencv-python-dlib/}. Определим численные границы для глаз и рта, соответсвующие открытому и закрытому положению.

Для каждой фотографии вычислим необходимые параметры, сравним с граничными значениями, сделаем вывод о состоянии элементов лица.

\codeExample

\inputPythonSource
