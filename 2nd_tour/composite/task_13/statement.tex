\assignementTitle{}{10}{}

Диаметр наночастицы на изображении d полученном с помощью атомно-силового микроскопа (АСМ) составляет 30 нм. Определить истинный диаметр наночастицы в нанометрах при условии, что радиус зонда АСМ составляет R =10 нм. Дайте ответ с точностью до сотых.

\putImgWOCaption{10cm}{1}

\begin{center}
    Конволюция зонда при сканировании
\end{center}

\explanationSection

\putImgWOCaption{9cm}{2}

\begin{center}
    Геометрия зонда и частицы
\end{center}

Согласно рисунок:
$$(R+r)^2=(R-r)^2+r_c \qquad (1)$$
$$r_c=\frac{d}{2} \qquad  (2)$$
$$d_\text{ист}=2r \qquad  (3)$$

Раскрывая скобки в выражении (1):
$$2Rr=-2Rr+r_c^2 \qquad  (4)$$
$$r_c^2=4Rr  \qquad (5)$$

Следовательно, формула определения видимого радиуса наночастицы в случае, когда размеры зонда сравнимы с размерами наночастицы: 
$$r_c=2 \sqrt{Rr}, \qquad  (6)$$
где $r_c$ – радиус наночастицы на изображении, $R$ – радиус закругления зонда атомно-силового микроскопа, $r$ – истинный радиус наночастицы.

Выразим радиус наночастицы:
$$\sqrt{Rr}=\frac{r_c}{2} \qquad (7)$$
$$Rr=\left(\frac{r_c}{2}\right)^2 \qquad  (8)$$
$$r=\left(\frac{r_c}{2}\right)^2/R \qquad  (9)$$
$$r=\frac{\left(\frac{15}{2}\right)^2}{10}=\frac{225}{10}=5.625 \: \text{нм} \qquad  (10)$$

Следовательно, истинный диаметр 11.25 нм.

\answerMath{11.25 нм.}