\assignementTitle{}{3}{}

Какой из перечисленных типов связующих обладает наибольшим модулем упругости?

\begin{enumerate}
    \item Полиэфирные
    \item Эпоксидные
    \item Полиимидные
    \item Фенолформальдегидные
\end{enumerate}

\explanationSection

Среди полимерных связующих распространены четыре вида: полиэфирные, эпоксидные, фенолформальдегидные и 
полиимидные. Каждое из связующих обладает своими преимуществами и недостатками, и их применяют в зависимости от 
назначения будущего изделия. Модуль упругости связующего оказывает влияние на модуль упругости композита в целом, 
но главное, на что влияет связующее – теплостойкость композита.

\textbf{Полиэфирные} достаточно распространенные связующие, преимущества которых, в первую очередь – 
невысокая стоимость. С технологической точки зрения их отличает малое необходимое количество отвердителя 
при приготовлении связующего (несколько процентов массовой доли), что несколько усложняет технологию, т.к. 
даже несколько миллилитров отвердителя, добавленных сверх нормы изменят режим отверждения. Модуль упругости 
полиэфирных связующих на уровне 1.5-4.5 ГПа. Однако у таких связующих низкая теплостойкость (50-80 $^\circ C$).

\textbf{Эпоксидные} связующие наиболее распространены, их отличает хорошая технологичность, невысокая 
стоимость, химическая и коррозионная стойкость, модуль упругости на уровне 3-4.5 ГПа и теплостойкость выше, 
чем у полиэфирных (130-150~$^\circ C$) связующих. По соотношению стоимости связующего к модулю и теплостойкости наиболее оптимальны.

\textbf{Фенолформальдегидные} связующие обладают теплостойкостью на уровне эпоксидных связующих, и наибольшим модулем упругости (7-11 ГПа). Однако эти связующие очень ядовиты и не так технологичны, как эпоксидные.

\textbf{Полиимидные} связующие обладают наибольшей теплостойкостью (250-320 $^\circ C$), но и наибольшей 
стоимостью и среднем модулем упругости (3.2-5 ГПА).

\answerMath{4.}