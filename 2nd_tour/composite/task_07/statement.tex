\assignementTitle{}{3}{}

Укажите, какой из указанных типов волокон обладает наибольшим удельным модулем упругости?

\begin{enumerate}
    \item Борное
    \item Стеклянное
    \item Арамидное
    \item Углеродное
\end{enumerate}

\explanationSection

Армирующий компонент в виде волокон прямым и самым большим образом влияет на прочность композиционного материала. Существуют несколько видов волокон, которые используются в композитных технологиях, каждый из которых обладает своими преимуществами и недостатками. Важным показателем волокон является удельный модуль упругости, т.е. модуль упругости, отнесенный к единице массы. Чем больше модуль упругости, тем более легким получится изделие с применением этого волокна.

\textbf{Арамидные} волокна – это полимерные волокна, которые получают из высокомолекулярных соединений, известные под торговой маркой Кевлар (важно не путать, Кевлар – это именно торговая марка, а не название материала). Арамидные волокна обладают средним значением удельного модуля. Особенностью данных волокон является плохая адгезия к матрице, что ограничивает сферы их применения, однако они обладают очень высокими значениями удельной прочности, благодаря чему арамидные ткани используют в бронежилетах и бронекасках. 

\textbf{Углеродное} волокно обладает самым большим значением удельного модуля. Процесс изготовления углеродных волокон очень ресурсо- и трудоёмкий, включает в себя несколько стадий обработки: окисление, карбонизация, графитизация, что обуславливает их очень высокую стоимость. Углеродные волокна получают из вискозы, полиакрилнитрила и нефтяных пеков. Но это безальтернативный материал для изготовления лёгких и прочных конструкций, поэтому в основном углеродные волокна применяют в авиационной и ракетно-космической промышленности.

\textbf{Борные} волокна также обладают высоким удельным модулем упругости, однако всё же меньшим, чем у углеродных. Их стоимость сравнима со стоимостью углеродных, однако они менее технологичны. Изготавливаются методом газофазного осаждение атомов бора на вольфрамовую нить с выпадением соляной кислоты. В основном применяют в композиционном материале – бороалюминии.

\textbf{Стеклянные} волокна обладают самым низким значением удельного модуля. Однако у композитов на основе 
стеклянных волокон есть одно важное преимущество~– радиопрозрачность. Из стеклопластиков изготавливают головные 
обтекатели ракет и носовые части самолётов – элементы, под которыми находится радиооборудование. Также стеклянные волокна достаточно дешевые, поэтому широко применяются для внешней отделки автобусов, электропоездов, отделки салонов самолётов и т.д.

\answerMath{4.}