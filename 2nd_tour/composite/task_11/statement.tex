\assignementTitle{}{7}{}

Оцените адгезионную прочность (в МПа) в системе полимер-волокно, если известен диаметр волокна 6 мкм, глубина погружения волокна в полимер - 60 мкм. Испытание на адгезионную прочность производится выдергиванием волокна с силой 0.03 Н из слоя полимера.

\explanationSection

Как уже было сказано ранее, адгезия – один из важнейших параметров композита, оказывающий очень большое влияние на прочность материала. Для оценки адгезионной прочности используют метод выдергивания волокна, когда из слоя полимера строго по оси волокона его вытягивают, измеряя усилие, при котором волокно выскользнет из связующего.

\putImgWOCaption{8cm}{1}
\begin{center}
    Система полимер-волокно
\end{center}

Напряжение в материале определяется по следующей формуле:
$$\sigma=\frac{F}{S} \qquad (1)$$

Площадь контакта между волокном и матрицей:
$$S=2 \pi rh,  \qquad (2) $$
где $r=\frac{d}{2}$; $h$ – глубина погружения волокна в связующее (толщина слоя полимера).

Тогда, подставляя (2) в (1) и известные данные, получаем:
$$\sigma=\frac{0.03 \: \text{Н}}{2\cdot 3.14\cdot 3\cdot 10^{-6} \: \text{м} \: \cdot 60\cdot 10^{-6} \: \text{м}}=26 \: \text{МПа}  \qquad (3)$$

\answerMath{26 МПа.}