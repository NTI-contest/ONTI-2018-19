\assignementTitle{}{3}{}

Чем по большей степени не определяется прочность непрерывно армированных композиционных материалов?

\begin{enumerate}
    \item Прочностью матрицы
    \item Прочностью волокна
    \item Адгезионной прочностью волокон и матрицы
    \item Направлением армирования
\end{enumerate}

\explanationSection

Композиционный материал – это сложная многофазная структура, на прочность которой оказывают влияние различные те или иные параметры. Некоторые в большей степени, некоторые в меньшей. 

\textbf{Направлением армирования}. Композиционный материал анизотропный, т.е. его прочность в разных направлениях отличается. Структура многих композитов волокнистая, а волокна сопротивляются только одному виду нагружения – растяжение вдоль своей оси. Если все волокна в композите уложены в одну сторону, то при растяжении композита вдоль направления укладки волокон его прочность будет максимальная, а при растяжении поперек волокон прочность будет минимальная, причем различаться на порядок, т.к. волокна не натягиваются и не сопротивляются растяжению композита. Металлы изотропные, т.е. равнопрочные во всех направлениях. Не важно, в каком направлении мы прикладываем к нам нагрузку. Направление армирование – один из основных факторов прочности композита.

\textbf{Адгезионной прочностью волокон и матрицы}. Очень большой вклад в прочность композиционного материала вносит адгезия. Адгезия – характеризует взаимодействие между армирующим наполнителем, в частности, волокнами, и связующим. Адгезия – это молекулярное взаимодействие, т.е. взаимодействие поверхностных слоёв волокон и связующего на молекулярном уровне, связь за счет сил Ван-дер-Ваальса, полярных сил и иногда — взаимной диффузией волокон и матрицы. Кроме того, важно также и механическое сцепление взаимодействие волокон и матрицы. Как показывают исследования микрорельефа волокон, поверхность волокон шероховатая, что положительно сказывается на прочности композита в целом.

\textbf{Прочностью матрицы}. Роль матрицы в композиционном материале – распределение нагрузки между всеми волокнами. Кроме того, матрица зачастую является самым хрупким компонентом, поэтому прочность непрерывно армированных композиционных материалов в меньшей степени определяется именно прочностью матрицы.

\textbf{Прочностью волокон}. Волокна воспринимают нагрузку композиционного материала, и в первую очередь определяют прочность непрерывно армированных композиционных материалов. 

\answerMath{1.}