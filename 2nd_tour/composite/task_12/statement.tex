\assignementTitle{}{7}{}

Рассчитайте прочность при изгибе образца в форме бруска с длиной 32 мм, шириной 8 мм, толщиной 4 мм при нагрузке 1150 Н. Дайте ответ с точность до сотых.

\explanationSection

Испытания конструкций очень важны, и один из видов испытаний – это испытания на изгиб, которые позволяют понять, насколько прочный материал, из которого в последствии будет изготовлена та или иная конструкций. На испытательной машине можно нагрузить образец силой F, и, зная его геометрические размеры, необходимо найти изгибную прочность образца.
Для определения изгибной прочности воспользуемся следующей формулой:

$$\sigma_\text{и}=\frac{3PL}{bh^2} \qquad (1)$$

Подставляем исходные данные:
$$\sigma_\text{и}=\frac{3\cdot 1150 \: \text{Н} \: \cdot 32 \: \text{мм}}{8 \: \text{мм}\cdot 4^2  \: \text{мм}}=431.25 \: \text{МПа} \qquad (2)$$

\answerMath{431.25 МПа.}