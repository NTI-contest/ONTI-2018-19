\assignementTitle{}{10}{}

Диаметр наночастицы $d$ на изображении, полученном с помощью атомно-силового микроскопа составляет 100 нм. Известно, что истинный диаметр наночастицы $r$ составляет 40 нм. Определить радиус закругления зонда атомно-силового микроскопа. Дайте ответ с точностью до сотых.

\putImgWOCaption{10cm}{1}

\begin{center}
    Конволюция зонда при сканировании
\end{center}

\explanationSection

\putImgWOCaption{9cm}{2}

\begin{center}
    Геометрия зонда и частицы
\end{center}

Согласно рисунку:
$$(R+r)^2=(R-r)^2+r_c \qquad (1)$$
$$r_c=\frac{d}{2} \qquad (2)$$
$$d_\text{ист}=2r \qquad (3)$$

Раскрывая скобки в выражении (1):
$$2Rr=-2Rr+r_c^2 \qquad (4)$$
$$r_c^2=4Rr \qquad (5)$$

Следовательно, формула определения видимого радиуса наночастицы в случае, когда размеры зонда сравнимы с размерами наночастицы: 
$$r_c=2 \sqrt{Rr}, \qquad (6) $$
где $r_c$ – радиус наночастицы на изображении, $R$ – радиус закругления зонда атомно-силового микроскопа, $r$ – истинный радиус наночастицы.

Выразим радиус наночастицы:
$$\sqrt{Rr}=\frac{r_c}{2} \qquad (7)$$
$$Rr=\left(\frac{r_c}{2}\right)^2 \qquad (8)$$
$$R=\left(\frac{r_c}{2}\right)^2/r \qquad (9)$$
$$R=\frac{\left(\frac{50}{2}\right)^2}{20}=\frac{625}{20}=31.25 \: \text{нм} \qquad (10)$$

Следовательно, истинный диаметр 31.25 нм

\answerMath{31.25 нм.}