\assignementTitle{}{2}{}

Что из перечисленного отличает композиционный материал от традиционных?

\begin{enumerate}
    \item Высокая прочность и легкость
    \item Состоит из двух и более компонентов
    \item Наличие четкой границы между фазами
    \item Волокнистая структура
\end{enumerate}

\explanationSection

Композиционные материалы – это новые, искусственно созданные материалы, значительно отличающиеся от привычных нам металлов, пластиков, стекла, древесины и т.д. Вопрос посвящен ключевому отличию композитов от традиционных материалов.

\textbf{Волокнистая структура}. Да, многие композиты обладают волокнистой структурой, но это не является их отличительной особенностью. Есть композиты без волокнистой структуры, армированные частицами, например, сферопластики, у которых армирование маленькими стеклянными сферами диаметром в несколько микрон. При этом волокнистая структура присуща древесине и металлам, которые определенно не являются композиционными материалами.

\textbf{Состоит из двух и более компонентов}. Композиты, как известно, действительно состоят из двух и более компонентов, но многие металлы также являются материалами, состоящими из двух других. Сталь состоит из железа и углерода, бронза состоит из меди и олова, но они являются сплавами, и макрограница между составными компонентами в них отсутствует.

\textbf{Наличие четкой границы между фазами}. Макрограница между железом и углеродом в стали отсутствует, тогда как это является обязательным свойством композиционного материала. Не смотря на малый размер углеродных или стеклянных волокон, мы всё равно видим границу между волокнами и полимерным связующим в углепластике или стеклопластике, но не видим границы между медью и оловом в бронзе.

\textbf{Высокая прочность и легкость}. Не вызывает сомнений, что композиты – легкие и прочные материалы, однако нет оценки этой прочности и лёгкости. Титан тоже обладает высокими значениями прочности при малой массе, но, естественно, композитом не является.

\answerMath{3.}