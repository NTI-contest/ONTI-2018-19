\assignementTitle{}{15}{}

\putImgWOCaption{9cm}{1}

Определить жесткость $b_1$ в ГПа на растяжение трехслойного композитного стержня в направлении $\varphi = 0^{\circ}$.
Слои изготовлены из одного материала – однонаправленного композиционного материала c наполнителем в 
виде непрерывных волокон, $E_1=10$~ГПа, $E_1>>E_2$, т.е. $E2 \approx 0$.

\begin{tabular}{l l l}
    Номер слоя & Угол укладки, $\varphi^{\circ}$ & Объемная доля \\
    1 & 45 & 0.25 \\
    2 & -45 & 0.25 \\
    3 &  0  & 0.5
\end{tabular}\\

\explanationSection

Используя правило смеси для композита несложно определить 
значение модуля упругости слоистого материала, если все слои 
уложены в одну сторону. Однако, если слои уложены более чем в 
одном направлении, то определение модуля упругости по правилу 
смеси невозможно. Для такого слоистого материала модуль 
упругости будет выглядеть следующим образом:

$$\begin{bmatrix}
    \sigma_1\\
    \sigma_2\\
    \sigma_3\\
    \sigma_4\\
    \sigma_5\\
    \sigma_6
\end{bmatrix}
=
\begin{bmatrix}
    \begin{bmatrix}
        C_{11} & C_{12} & C_{13} & 0 & 0 & 0 \\
        C_{21} & C_{22} & C_{23} & 0 & 0 & 0 \\
        C_{31} & C_{32} & C_{33} & 0 & 0 & 0 \\
        0 & 0 & 0 & C_{44} & 0 & 0\\
        0 & 0 & 0 & 0 & C_{55} & 0\\
        0 & 0 & 0 & 0 & 0 & C_{66}
    \end{bmatrix}
\end{bmatrix}
\cdot
\begin{bmatrix}
    \epsilon_1\\
    \epsilon_2\\
    \epsilon_3\\
    \epsilon_4\\
    \epsilon_5\\
    \epsilon_6
\end{bmatrix}
$$

Решение с помощью «нитяной» модели композиционного материала 
позволит определить жесткость $b_1$ в ГПа на растяжение 
трехслойного композитного стержня в направлении $\phi = 0^\circ$. 
Жесткость $b_1$ представляет собой: 

$$b_1=g_{xx}-\frac{(g_{xy})^2}{g_{yy}} \qquad (2)$$
где $g_{xx}$, $g_{xy}$ и $g_{yy}$ – элементы матрицы жесткости.

Коэффициент $g_{xx}$ определяется следующим образом:
$$g_{xx}=\sum_{i=1}^n E_{1i} \cdot cos^4 \phi_i \cdot \delta_i \qquad (3)$$
Коэффициент gyy определяется следующим образом:
$$g_{yy}=\sum_{i=1}^n E_{1i} \cdot sin^4 \phi_i \cdot \delta_i \qquad (4)$$
Коэффициент gxy определяется следующим образом:
$$g_{xy}=\sum_{i=1}^n E_{1i} \cdot sin^2 \phi_i \cdot cos^2 \phi_i \cdot \delta_i \qquad (5)$$
Подставляем известные нам величины:
$$g_{xx}=E_1 \cdot \frac{1}{4} \cdot \frac{1}{4}+E_1 \cdot \frac{1}{4} \cdot \frac{1}{4}+E_1 \cdot 1 \cdot \frac{1}{2}=\frac{5}{8} E_1 \qquad (6)$$
$$g_{yy}=\frac{1}{8} E_1 \qquad (7)$$
$$g_{xy}=\frac{1}{8} E_1 \qquad (8)$$

Получает итоговый ответ:
$$b_1=\frac{1}{2} E_1=5 \: \text{ГПа}$$

\answerMath{5 ГПа.}