\assignementTitle{}{15}{}

\putImgWOCaption{9cm}{1}

Через трехслойную панель протекает тепловой поток плотностью
$q$. Площадь поверхности
панели перпендикулярна направлению теплового потока и равна $S$. Разница температур наружных
поверхностей равна $\Delta t$. Определите
теплопроводность слоя 2 в Вт/(м$\cdot$К), если теплопроводность слоев 1 и 3 равны соответственно
$\lambda_1$ и $\lambda_3$, толщины слоев 1, 2 и 3 равны $l_1$, $l_2$ и $l_3$ соответственно. 
Дайте ответ с точностью до сотых.

$q$ = 30 Вт/м$^2$, 

$S$ = 2 м$^2$, 

$\Delta t$ = 5, 

$\lambda_1$ = 20 Вт/(м$\cdot$К),

$\lambda_3$ = 8 Вт/(м$\cdot$К), 

$l_1$ = 0,05 м, 

$l_2$ = 0,3 м, 

$l_3$ = 0,15 м.

\explanationSection

Найдем общую (суммарную) теплопроводность панели, толщина 
которой равна $l=l_1+l_2+l_3$. Используем закон Фурье для плоской стенки:

$$\lambda=\frac{q \cdot l}{\Delta t}=3 \: [\text{Вт⁄(м} \cdot \text{К})] \qquad (1)$$

Для нахождения теплопроводности слоя 2 воспользуемся формулой суммы тепловых сопротивлений:

$$\frac{l}{\lambda}=\frac{l_1}{\lambda_1}+\frac{l_2}{\lambda_2}+\frac{l_3}{\lambda_3}, \qquad (2)$$

Откуда:

$$\lambda_2=\frac{l_2 \cdot \lambda \cdot \lambda_1 \cdot \lambda_3}{l \cdot \lambda_1 \cdot \lambda_3-l_1 \cdot \lambda \cdot \lambda_3-l_3 \cdot \lambda \cdot \lambda_1}=2.06 \: [\text{Вт⁄(м} \cdot \text{К})] \qquad (3)$$

\answerMath{2.06 Вт⁄(м $\cdot$ К).}