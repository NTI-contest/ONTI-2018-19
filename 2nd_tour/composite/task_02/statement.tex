\assignementTitle{}{2}{}

По какому параметру не классифицируют композиционные материалы? 

\begin{enumerate}
    \item По типу армирования
    \item По типу наполнителя
    \item По типу производства
    \item По типу связующего
\end{enumerate}

\explanationSection

В основе любой науки, а композиты заняли отдельную и важную нишу в материаловедении, лежит классификация. Вопрос посвящен классификации композиционных материалов, причем, классифицируют их по нескольким параметрам сразу.

\textbf{По типу наполнителя}. Композиционный материал состоит из связующего и наполнителя, по типу которого их и классифицируют. Важно не путать этот параметр с типом армирования, потому что тут речь идёт именно о материале, из которого изготовлен наполнитель. Наполнители подразделяют на углеродные, стеклянные, полимерные и другие наполнители.

\textbf{По типу связующего}. Второй компонент композиционного материала – связующее. Аналогично предыдущему варианту, здесь идёт классификация по материалу, из которого изготовлено связующее. Связующие подразделяют на полимерные, металлические, керамические, оксид-оксидные.

\textbf{По типу армирования}. Наполнители, помимо химического состава, также могут отличаться по своей форме, поэтому дополнительно композиты классифицируют по типу армирования. Армирование может быть частицами (сферопластики), короткими волокнами (композиты на основе стекломатов), и волокнами непрерывной длины (композиты на основе стеклянных, углеродных и других тканей).

\textbf{По типу производства}. Да, композиционные материалы возможно изготовить различными технологическими методами (контактное формование, автоклавное формование, вакуумная инфузия, RTM и др.), однако, данный параметр не входит в классификацию композитов, т.к. не оказывает влияние на структуру или состав композита, а влияет только на качество и технологичность конечного изделия. Другими словами, по типу производства можно классифицировать изделие, а не материал.

\answerMath{3.}