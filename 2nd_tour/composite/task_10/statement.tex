\assignementTitle{}{7}{}

Рассчитать время линейной намотки трубы в минутах, если
известны необходимая толщина трубы - 10 мм, длина трубы - 100  мм, скорость вращения оправки 20~об/мин, 
линейная плотность жгута 2000 г/км (текс), и плотность жгута 2.55 г/см$^3$,
объемная доля волокна $60\%$.

\explanationSection

\putImgWOCaption{7cm}{1}
\begin{center}
    Схема задачи
\end{center}

Площадь волокон равна:
$$f=  \frac{T}{\rho} \qquad (1)$$

Количество волокон, помещающихся в поперечном сечении трубы:
$$n=\frac{h\cdot L}{f}\cdot v_f \qquad (2)$$

Время, необходимое для намотки одного слоя:
$$t=\frac{n}{\omega} \qquad (3)$$

Следовательно, подставляя (1) в (2), и (2) в (3):
$$t=\frac{h\cdot L}{f\cdot \omega}=\frac{h\cdot L\cdot \rho}{f\cdot \omega\cdot T}\cdot v_f \qquad (4)$$

Подставляем исходные данные, переведенные в СИ:
$$t=\frac{0.01\cdot 0.1\cdot 2550}{0.02\cdot 20}\cdot 0.6=38.25 \: \text{мин} \qquad (4)$$

\answerMath{38.25 мин.}