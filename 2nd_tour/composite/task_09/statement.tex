\assignementTitle{}{7}{}

Рассчитать объемное содержание в процентах
армирующих волокон в углепластике, если известно, что в нём содержится $68\%$ по
массе углеродного волокна, масса образца 100 г, его плотность 1.45 г/см$^3$.
Плотность углеродного волокна 1.8 г/см$^3$. Недостающие данные для
расчета взять из справочника. Ответ округлите до целого.

\solutionSection

Согласно правилу смеси для композита модуль упругости материала можно оценить с помощью формулы:
$$E_k=E_f v_f+E_m v_m \qquad (1)$$

Для того, чтобы воспользоваться этой формулой необходимо знать 
объемную долю наполнителя $v_f$ и объемную долю связующего $v_m$.

Массовая доля волокон равна:
$$\omega_f=\frac{m_f}{m_k}, \qquad (2)$$
где $m_f$ – масса волокон, $m_k$ – масса всего композита.

Объёмная доля волокон равна:
$$v_f=\frac{V_f}{V_k}, \qquad (3)$$
где $V_f$ – объем волокон, $V_k$ – объем всего композита.

Объем волокна равен:
$$V_f=\frac{m_f}{\rho_f}, \qquad (4)$$

Объем композита равен:
$$V_k=\frac{m_k}{\rho_k}, \qquad (5)$$

Следовательно:
$$v_f=\frac{m_f}{\rho_f}   \frac{\rho_k}{m_k}, \qquad (6)$$

Подставляя (2) в (6) получаем:
$$v_f=\frac{\omega_{fm_k}}{\rho_f} \frac{\rho_k}{m_k}, \qquad (7)$$

Таким образом:
$$v_f = \frac{0.68 \cdot 1.45\text{г/см}^3}{1.8 \text{г/см}^3}=55\% \qquad (8)$$

\answerMath{55 \%.}