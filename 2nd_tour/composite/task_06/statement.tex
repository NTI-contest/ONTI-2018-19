\assignementTitle{}{3}{}

Какой тип течения присущ для эпоксидных связующих?

\begin{enumerate}
    \item Неньютоновские
    \item Дилатантный
    \item Ньютоновский
    \item Псевдопластичный
\end{enumerate}

\explanationSection

Один из основных технологических методов изготовления композитов – вакуумная инфузия, при котором связующее пропитывает преформу – сухую заготовку, выполненную из армирующего компонента. В случае сложной формы будущего изделия, например, если это поверхность кривизны второго или третьего порядка, геометрия расположения каналов подачи связующего и откачки воздуха неочевидна, и необходимо рассчитывать возможную траекторию течения связующего, для чего необходимо понимать механику его движения.

\textbf{Ньютоновский} тип течения присущ жидкостям, у которых скорость сдвига прямо пропорционально усилию сдвига. Яркий пример ньютоновской жидкости – вода.

\textbf{Неньютоновские} жидкости – это жидкости, у которых нет прямой зависимости между скоростью сдвига и усилием сдвига. При этом неньютоновские жидкости подразделяются на дилатантные и псевдопластичные. Ответ «Неньютоновский» был бы ошибочным, т.к. это общий класс жидкостей, а не частный.

\textbf{Дилатантные} жидкости – это жидкости, у которых вязкость возрастает при увеличении скорости деформации сдвига. Дилатантный эффект наблюдается в тех материалах, у которых плотно расположенные частицы перемешаны с жидкостью, заполняющей пространство между частицами. Примером дилатантной жидкости является крахмал с водой.

\textbf{Псевдопластичной} жидкостью называются жидкости, у которых вязкость жидкости уменьшается при увеличении напряжений сдвига. Эпоксидные смолы относятся именно к дилатантным жидкостям.

\answerMath{4.}