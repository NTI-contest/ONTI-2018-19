\assignementTitle{}{3}{}

По какой из приведенных формул определяется модуль упругости
при растяжении волокнистого композиционного материала, если известная прочность
наполнителя $E_f$, прочность связующего $E_m$ , объемная доля наполнителя $v_f$, объемная доля 
связующего $v_m$?

\begin{enumerate}
    \item $ E_k=E_f v_f+E_m v_m $
    \item $ E_k=E_f E_m+v_f v_m $
    \item $ E_k=E_f v_f-E_m v_m $
    \item $ E_k=E_f E_m-v_f v_m $
\end{enumerate}

\begin{enumerate}
    \item По формуле №1
    \item По формуле №2
    \item По формуле №3
    \item По формуле №4
\end{enumerate}

\explanationSection

Для того, чтобы изготавливать из материала то или иное изделие нужно знать характеристики этого материала, 
в частности, модуль упругости. Модуль упругости~– способность твёрдого тела сопротивляться растяжению при упругой деформации. Не существует паспортных или ГОСТовских данных о модуле упругости, например, углепластика, т.к. он зависит и от модуля упругости связующего, и от модуля упругости волокна, а эти параметры не постоянны, в свою очередь, зависят от производителя, химического состава, технологии производства. Для оценки модуля упругости композиционного материала используется правило смеси.
Рассмотрим некоторый элемент композиционного материала, т.н. представительный элемент.

\putImgWOCaption{9cm}{1}

\begin{center}
    Представительный элемент композита

    (1 – ось в направлении вдоль волокна; 2 – ось в направлении поперек волокон; \linebreak
    $h$~– толщина композита; $V_\text{в}$ – объем волокна в микромодели композита; $V_\text{м}$ – объем матрицы в микромодели композита; 
    $V$ – объем микромодели композита)
\end{center}

При растяжении такой модели вдоль волокон (направление 1) напряжения будут определяться по следующей формуле:
$$\sigma_k=\sigma_f v_f+\sigma_m v_m \qquad (1)$$

По закону Гука, напряжение есть произведение модуля упругости материала на его деформацию:
$$\sigma=E \epsilon \qquad (2)$$

Подставим выражение (2) в выражение (1):
$$E_k \epsilon_k=E_f \epsilon_f v_f+E_m \epsilon_m v_m \qquad (3)$$

Деформация композита в целом равна деформации матрица и деформации волокна, т.к. нет разрушения, и все элементы деформируются совместно:
$$\epsilon_k=\epsilon_f=\epsilon_m \qquad (4)$$

Следовательно:
$$E_k=E_f v_f+E_m v_m \qquad (5)$$

\answerMath{1.}
