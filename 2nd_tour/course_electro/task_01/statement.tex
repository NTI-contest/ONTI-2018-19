\assignementTitle{}{3}{}

В качестве фар машинки-робота использован осветительный светодиод LM561C, запитываемый от 2-х баночного LiPo 
аккумулятора (7.4 В). Последовательно со светодиодом, в цепь также включен ограничительный резистор. 
Используя приведенную ниже вольт-амперную характеристику светодиода, выберите номинал резистора так, 
чтобы ток через светодиод был по возможности ближе к 100 мА, но не превышал этого значения.

\putImgWOCaption{9cm}{1}

\begin{enumerate}
    \item[a.] Выберите номинал ограничительного резистора из стандартного ряда номиналов:
\end{enumerate}    
\begin{enumerate}
    \item 33 Ом
    \item 39 Ом
    \item 47 Ом
    \item 56 Ом
    \item 68 Ом
    \item 82 Ом
    \item 100 Ом
    \item 120 Ом
\end{enumerate}

\begin{enumerate}
    \item[б.] Максимальная рассеиваемая мощность этого резистора не должна быть менее чем 
\end{enumerate}

\begin{enumerate}
    \item 0.125 Вт
    \item 0.25 Вт
    \item 0.5 Вт
    \item 0.75 Вт
    \item 1 Вт
    \item 2 Вт
\end{enumerate}

\solutionSection

В этой задаче вам надо понимать закон Ома и знать, что такое вольт-амперная характеристика.

В отличие от, например, резисторов, светодиоды являются нелинейными электронными компонентами. Ток через светодиод почти не протекает при малых значениях напряжения, находится в рабочем диапазоне в очень узком диапазоне напряжений, а при дальнейшем увеличении напряжения резко нарастает, до перегрева и выхода светодиода из строя. График зависимости тока от напряжения называется вольт-амперной характеристикой устройства.  

Как видно из графика, требуемый ток в 100~мА получается при напряжении на диоде 2.85~В. Однако же на входе у нас 7.4~В, поэтому из них $U_R= 7.4 - 2.85 = 4.55$~B  должны падать на ограничительном резисторе. Поскольку резистор и светодиод соединены последовательно, ток через резистор составляет те же 100~мА. По закону Ома:
$$R = U_R/I = 4.55\:\text{В} / 0.1\:\text{А} = 45.5\: \text{Ом}$$

Однако резисторы изготавливаются определенных номиналов, и Stepik предлагает выбрать один из них (33, 39, 47, 56, 68 Ом).  Выбираем резистор с ближайшим большим значением - 47 Ом.  

Какова же должна быть максимальная рассеиваемая мощность этого резистора?  Мощность вычисляется по формуле:
$$P= U\cdot I = 4.55 \: \text{В} \cdot 0.1 \: \text{А} \approx 0.45 \: \text{Вт}$$

Подбираем ближайшее большее значение (0.125, 0.25, 0.5, 0.75, 1, 2 Вт):  0.5 Вт.

\answerMath{a - 3, б - 3.}