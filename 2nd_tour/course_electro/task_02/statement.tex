\assignementTitle{Простой blink}{2}{}

В этом шаге вам нужно будет изменить и проверить работу простейшего примера blink из 
стандартного набора примеров Arduino IDE, для этого следует изменить номер пина и интервал мигания, 
как указано в шаблоне!

Такие задачи стали возможны благодаря мини-симулятору Arduino, который имитирует действие 
некоторых функций библиотеки Arduino, изменяя состояние хранимых в памяти "пинов" и сообщая о 
каждом изменении платформе Stepik, чтобы проверить правильность вашего решения.  Вы пишете код 
КАК БЫ для Ардуино, на самом деле он выполняется в Степике, взаимодействуя с платформой через 
стандартный ввод-вывод, как и любая другая задача на программирования.

Вам НЕ НАДО трогать что-нибудь в шаблоне, кроме функций  setup() и loop().  
Разумеется, для решения более сложных задач могут понадобиться дополнительные 
функции или переменные, которые вы можете определить там же.  Можно использовать большинство 
стандартных библиотек C++, но никакие библиотеки, специфические для Arduino, применить не получится.  
Мини-симулятор настраивается под задачу, и в нем, как правило, не будут работать никакие функции, не 
относящиеся конкретно к данной задачи.

Функции библиотеки Arduino, которые можно использовать в этой задаче:

void digitalWrite(int pin, int  state): изменяет хранимое в памяти состояние пинов, и сообщает 
в Stepik о каждом изменении, вместе с отметкой времени.  Если был задан неверный номер пина, либо 
этот пин не был переведен в режим OUTPUT, либо состояние пина не изменилось, то ничего не происходит. 

void pinMode(int pin, int mode):  устанавливает режим работы пина:  INPUT, \linebreak OUTPUT или INPUT\_PULLUP. 

void delay(int n):  "Задержка" на n миллисекунд, а на самом деле - просто увеличение внутреннего 
счетчика времени. Мини-симулятор следит за временем, хотя, в отличие от реального микроконтроллера, 
счетчик времени увеличивается только и исключительно вызовами функции delay(). Любые операции, между 
которыми нет delay(), считаются выполняющимися мгновенно.  

Если вы хотите испытать свой код на реальной схеме с Ардуино,  просто перенесите в среду 
Ардуино написанный вами код и он, МОЖЕТ БЫТЬ, даже будет работать.  Разумеется, вам сначала 
придется собрать соответствующую описанию схему.

%\includeSolutionIfExistsByPath{2nd_tour/bas/task_01}