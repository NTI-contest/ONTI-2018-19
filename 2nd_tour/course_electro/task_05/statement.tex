\assignementTitle{Асинхронный маячок}{3}{}

К цифровым пинам контроллера Ардуино подключены 3 светодиода разных цветов (красный, желтый и зеленый).   Напишите
программу, которая будет одновременно мигать этими светодиодами, причем каждым - со своим периодом и скважностью.  
В начальный момент времени все 3 светодиода должны включиться одновременно. 

Входные данные мини-симулятор считывает автоматически, но вам нужно использовать переменные, как указано в шаблоне кода.  Как и в предыдущем примере, вам доступны функции pinMode(), digitalWrite() и delay().

\solutionSection

Задачка с тремя светодиодами требует совсем другого подхода:  мы должны использовать в конце функции loop() ровно одну короткую задержку (например, на 1 мс) и для каждого светодиода использовать отдельную переменную-счетчик,  которая увеличивается при каждом проходе цикла.  При заданных пороговых значениях счетчика светодиод может включаться или отключаться.  По сравнению с фиксированной задержкой в конце цикла, время выполнения остальной части кода (проверки счетчиков и управление пинами) пренебрежимо мало.

\includeSolutionIfExistsByPath{2nd_tour/course_electro/task_05}