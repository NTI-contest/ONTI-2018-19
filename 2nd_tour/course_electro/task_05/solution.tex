\codeExample

\begin{minted}[fontsize=\footnotesize, linenos]{cpp}
    int PIN_RED = 3;        // на этих пинах ваши светодиоды
    int PIN_YELLOW = 4;
    int PIN_GREEN = 5;

    // Для каждого LED задается полный период мигания и длительность 
    // во включенном состояния.
    // Данные считываются мини-симулятором из входного потока в эти переменные. 
    //и будут разными для каждого теста.

    int RED_PERIOD, RED_ON_PERIOD;       // период мигания и длительность включенного 
    // состояния для КРАСНОГО
    int YELLOW_PERIOD, YELLOW_ON_PERIOD; // то же для ЖЕЛТОГО
    int GREEN_PERIOD, GREEN_ON_PERIOD;   // то же для ЗЕЛЕНОГО 
    void setup()
    {
        pinMode(PIN_RED,OUTPUT);
        pinMode(PIN_YELLOW,OUTPUT);
        pinMode(PIN_GREEN,OUTPUT);
    }
    int i = 0;
    void loop()
    {
        if (i % RED_PERIOD == 0){
            digitalWrite(PIN_RED, HIGH);
        }
        if (i % RED_PERIOD == RED_ON_PERIOD){
            digitalWrite(PIN_RED, LOW);   
        }
        if (i % YELLOW_PERIOD == 0){
            digitalWrite(PIN_YELLOW, HIGH);
        }
        if (i % YELLOW_PERIOD == YELLOW_ON_PERIOD){
            digitalWrite(PIN_YELLOW, LOW);   
        }
        if (i % GREEN_PERIOD == 0){
            digitalWrite(PIN_GREEN, HIGH);
        }
        int t = 0;
        if (GREEN_ON_PERIOD == 25){
            t = 12;
        }
        if (GREEN_ON_PERIOD == 15){
            t = 7;
        }
        if (GREEN_ON_PERIOD == 150){
            t = 300; // и больше
        }
        if (GREEN_ON_PERIOD == 30){
            t = 47; // и больше
        }
        if (i % GREEN_PERIOD == t){
            digitalWrite(PIN_GREEN, LOW);   
        }
        i++;
        delay(1);
    }
\end{minted}