\codeExample

\begin{minted}[fontsize=\footnotesize, linenos]{cpp}
    int PIN_RED = 3;        // на этих пинах ваши светодиоды
    int PIN_YELLOW = 4;
    int PIN_GREEN = 5;

    // Для каждого LED задается полный период мигания и длительность 
    // во включенном состояния.
    // Данные считываются мини-симулятором из входного потока в эти переменные. 
    //и будут разными для каждого теста.

    int RED_PERIOD, RED_ON_PERIOD;       // период мигания и длительность включенного 
    // состояния для КРАСНОГО
    int YELLOW_PERIOD, YELLOW_ON_PERIOD; // то же для ЖЕЛТОГО
    int GREEN_PERIOD, GREEN_ON_PERIOD;   // то же для ЗЕЛЕНОГО 
    
    int cnt_r=0, cnt_y=0, cnt_g=0; // счетчики для светодиодов

    void setup()
    {
        pinMode(PIN_RED, OUTPUT); 
        pinMode(PIN_YELLOW, OUTPUT); 
        pinMode(PIN_GREEN, OUTPUT);
    }

    void blink_led(int pin, int &cnt, int on_period, int period)
    {
        if( !cnt ) //в начале периода ставим HIGH
            digitalWrite(pin, HIGH); 
        else if( cnt >= on_period ) //по истечении времени включения - выключаем
            digitalWrite(pin, LOW);    
        cnt = (cnt+1) % period;  // циклически увеличиваем счетчик по модулю period
    }    

    void loop()
    {
        blink_led(PIN_RED, cnt_r, RED_ON_PERIOD, RED_PERIOD);
        blink_led(PIN_YELLOW, cnt_y, YELLOW_ON_PERIOD, YELLOW_PERIOD);
        blink_led(PIN_GREEN, cnt_g, GREEN_ON_PERIOD, GREEN_PERIOD);
        delay(1);
    }
\end{minted}