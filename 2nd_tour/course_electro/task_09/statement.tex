Шаговые двигатели - основа самых разных координатных устройств, в том числе станков с ЧПУ, а среди них и самых распространенных - 3D-принтеров  (да и обычных бумажных принтеров тоже!)

В отличие об обычных электромоторов, шаговый двигатель не просто "вращается", а поворачивается на 
заданный угол, с весьма хорошей точностью, при подаче импульсов определенной формы, полярности и 
продолжительности на его 2 обмотки.  Для управления шаговыми двигателями  используют специальные '
микросхемы-драйверы.  Типичный драйвер управляется с микроконтроллера по 3-м пинам \linebreak ENABLE, DIR и STEP:   

ENABLE - включение двигателя.  Когда шаговый двигатель включен, ток протекает по обмоткам и блокирует движение ротора, так что прокрутить вал можно, только приложив значительное усилие.  Часто сигнал ENABLE для драйвера имеет инверсную полярность, т.е. двигатель включается при установке низкого уровня сигнала.

DIR  - уровень сигнала определяет направление вращения.

STEP -  каждый короткий импульс, подаваемый на этот пин, поворачивает вал шагового двигателя на 1 шаг.

Драйвер шагового двигателя несложно подключить к Ардуино прямо на макетной плате, но гораздо удобнее пользоваться специальными шилдами.   На следующей картинке показан"бутерброд" из Arduino UNO (внизу), CNC ShieldV3 и 4-х драйверов моторов сверху.  Это типичный расклад для любительского настольного фрезерного станка или плоттера.  Для 3D-принтеров чаще применяют аналогичный "бутерброд", но на базе Arduino Mega и шилда RAMPS.

\putImgWOCaption{7cm}{1}

\assignementTitle{Какие пины? ("Google it!")}{1}{}

Когда мы подключаем драйвер шаговых двигателей на макетной плате, то мы свободны в выборе пинов Arduino, которые будут им управлять.  Но если мы используем CNC Shield,  то вся "распиновка" определяется платой.  Найдите в интернете информацию о CNC Shield V3 для Arduino UNO и заполните пробелы.

\putImgWOCaption{7cm}{2}

\begin{enumerate}
    \item[a)] При использовании этой платой сигнал ENABLE: 
    \begin{enumerate}
        \item Отдельный для каждого мотора
        \item Общий для всех моторов
        \item Всегда включен
        \item Общий для каналов X и Y
    \end{enumerate}
    \item[б)] Сигнал STEP для канала X подключен к пину Arduino: 
    \begin{enumerate}
        \item D1
        \item D2
        \item D3
        \item D4
        \item D5
        \item D6
        \item D7
        \item D8
    \end{enumerate}
    \item[в)] Сигнал DIR для канала Z подключен к пину Arduino: 
    \begin{enumerate}
        \item D1
        \item D2
        \item D3
        \item D4
        \item D5
        \item D6
        \item D7
        \item D8
        \item D9
        \item D10
    \end{enumerate}
\end{enumerate}

\solutionSection

Решение этой задачи очевидно из условия. Ищем "CNC shield pinout"{}, и находим вот такую схему:

\putImgWOCaption{16cm}{3}

Посмотрев подробнее на подключение драйверов, мы видим, что:
\begin{itemize}
    \item Сигнал EN (а точнее - NOT ENABLE, т.к. драйверы включаются при подаче 0 на EN) разведен с пина 8 на все драйвера,
    \item Сигнал шаг X\_STEP подключен к пину 2, Y\_STEP к пину 3,  Z\_STEP к пину 4
    \item Сигнал направления X\_DIR подключен к пину 5 и т.д.
\end{itemize}

\putImgWOCaption{8cm}{4}

Этого достаточно, чтобы правильно ответить на вопрос задания.  На практике, следует иметь в виду, что разводка конкретной "китайской"\ платы может отличаться от найденной вами в интернете, поэтому крайне желательно проверять ее по дорожкам на плате и/или мультиметром.