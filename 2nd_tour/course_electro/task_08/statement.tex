\assignementTitle{Загадочный делитель напряжения}{1}{}

Делитель напряжения из двух резисторов разного номинала подключен между пинами D2 и D3, а средняя точка подключена к аналоговому пину A0, как показано на схеме:

\putImgWOCaption{10cm}{1}

Значение напряжения измеряется при разных комбинациях настроек пинов. Считать, что напряжение на пине в выходном режиме в точности равно либо 0, либо напряжению питания (хотя в реальных схемах, логический "0" всегда чуть выше 0V, а логическая "1" чуть ниже напряжения питания). Сопротивление внутреннего подтягивающего резистора (на любом пине) принять за 10 кОм.  

Нужно написать программу, которая рассчитывает все возможные напряжения, которые можно намерить на 0000000000A0, изменяя настройки пинов.  В этой задаче не используется мини-симулятор Ардуино, вы просто пишете код на любом удобном для вас языке.

\inputfmtSection
Два числа через пробел - сопротивления резисторов R1 и R2.

\outputfmtSection
На первой строке - число значений в ответе, на второй строке - последовательность целых чисел в диапазоне 0..1023, 
которые можно было бы получить функцией analogRead(A0) при различных настройках пинов.  

Все числа в ответе должны быть  на одной строке через пробел,  упорядочены по возрастанию,  
повторяющиеся значения удалены. 

Допускается отклонение вычисленных значений на 1 в любую сторону.

\solutionSection

Чтобы понять эту задачу, следует помнить, что любой пин Arduino может находится в 4-х разных состояниях:

\begin{tabular}{|p{6.5cm}|p{1.5cm}|p{6cm}|}
    \hline
    Состояние&Обозн.&Что происходит \\
    \hline
    pinMode(pin, OUTPUT);

    digitalWrite(pin,  HIGH);& 1& На пин подано напряжение питания\\
    \hline
    pinMode(pin, OUTPUT);

    digitalWrite(pin,  LOW); & 0 & На пин подается 0 (подключен GND) \\
    \hline
    pinMode(pin, INPUT); &I &Пин отключен от каких-либо напряжений \\
    \hline
    pinMode(pin, INPUT\_PULLUP);& U & Пин подключен к напряжению питания через подтягивающий резистор в 10K \\
    \hline
\end{tabular}
 
Все 4 перечисленных режима относятся как к пинам D2, D3, так и к пину A0, на котором производятся измерения.  Когда мы говорим про измерение аналогового сигнала функцией analogRead(), всегда предполагается, что аналоговый пин находится в режимах INPUT или INPUT\_PULLUP.  Однако ничего не мешает установить A0 в режим OUTPUT, подать на него LOW или HIGH и прочитать значение полученного напряжения, если это зачем-то вдруг понадобилось.

Итак, мы можем получить $4 \cdot 4 \cdot 4 = 64$ комбинации состояний 3х пинов. Многие из них дадут уникальные комбинации сопротивлений и уникальные значения измеряемого напряжения на средней точке делителя.  Следующая функция на Python вычисляет выходное напряжение при  различных комбинациях состояний трех пинов (обозначены '0', '1', 'U', 'I' - см. выше):

\includeSolutionIfExistsByPath{2nd_tour/course_electro/task_08}