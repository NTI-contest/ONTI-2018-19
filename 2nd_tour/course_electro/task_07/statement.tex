\assignementTitle{Аналоговые кнопки}{4}{}

Если нужно подключить к устройству несколько кнопок, а пины контроллера приходится экономить, то подключают цепь из нескольких кнопок и резисторов к одному аналоговому входу, так что при разных комбинациях нажатий получаются разные сопротивления цепи, и, соответственно, разные напряжения на  аналоговом входе.  

И вот, имеется такая схема:

\putImgWOCaption{10cm}{1}

Напишите программу, которая сразу после включения определяет нажатые кнопки и передает через последовательный порт одно десятичное число, 
соответствующее их комбинации.  Каждой из кнопок S1..S3 соответствует один бит в полученном числе.  Например, нажатые кнопки S1 и S2 должны 
давать число 3  (2+1).     Ожидаемая скорость передачи - 115200 бод.  Программа должна учитывать,  что сопротивления резисторов могут 
несколько (до $2.5\%$ в каждую сторону) отклоняться от  указанных на схеме значений.

В этом примере, мини-симулятор будет выполнять функцию loop() однократно для каждой комбинации нажатых кнопок.  Никаких  задержек 
и ожиданий в коде писать не надо.  Используются функции analogRead(), pinMode() и упрощенный Serial, как в предыдущих заданиях.

\solutionSection

Если нужно подключить к устройству несколько кнопок, а пины контроллера приходится экономить, то часто подключают цепь из нескольких кнопок и резисторов к одному аналоговому входу, так что при разных комбинациях нажатий получаются разные сопротивления цепи, и, соответственно, разные напряжения на  аналоговом входе.

В данной схеме "верхний" резистор делителя $R4=12$ кОм, а "нижняя" ветка делителя состоит из 3-х последовательно соединенных резисторов ($R1=2.7$ K, $R2=5.6$ К, $R3=12$ K), каждый из которых может быть шунтирован при нажатии соответствующей кнопки.  Для любой заданной комбинации кнопок (из 8 возможных) можно заранее вычислить напряжение на пине A0. Например, при замкнутых $S1$ и $S3$ в нижней части делителя останется только $R2=5.6$ К, и результат analogRead() будет равен:
$$A= R2 / (R2 + R4) \cdot 1023 = 5.6 / (5.6 + 12 ) \cdot 1023 \approx 325$$

С учетом 2.5\% допусков, диапазон значений analogRead() для этой комбинации кнопок составляет:  $A_{min} = A \cdot 0.975 \approx 317$,  $A_{max} = A \cdot 1.025 = 333$

Программа сравнивает полученное значение с каждым из рассчитанных таким образом диапазонов, определяя комбинацию нажатых кнопок.  (Полный текст программы не приводится).