\assignementTitle{Простейший омметр}{2}{}

Вам нужно реализовать на Arduino простейший омметр (измеритель сопротивления), как показано на схеме. 

\putImgWOCaption{9cm}{1}

Для этого к пину A0 подключен делитель напряжения, состоящий из прецизионного резистора 10 кОм "сверху" и измеряемого резистора "снизу".  После того, как очередной измеряемый резистор подключен к схеме, контроллер включается, программа на Arduino производит измерение и через последовательный порт выводит в виде числа значение сопротивления, в кОм, с округлением до целого и с переводом строки на конце.   

Для упрощения,  мы принимаем, что измеряемое сопротивление всегда целое число килоом и отсутствуют любые ошибки измерения. Все значения измеряемых резисторов находятся в диапазоне от 10 кОм до 90 кОм (т.е. допускающем их сравнительно точное измерение при 10кОм подтягивающем резисторе).


Напишите программу для мини-симулятора Arduino, которая выполняет измерения.

Данные для измерения поступают из входного потока и расшифровываются мини-симулятором, а вычисленные вашей программой и отправленные на Serial.println() значения передаются в выходной поток для проверки.  Мини-симулятор однократно вызывает функцию loop() для выполнения каждого измерения. 

Разрешается пользоваться функциями pinMode(), analogRead() ,  а также упрощенным классом Serial c методами begin(), print() и println().  Передача данных должна происходить на скорости 9600 бод.

\begin{myverbbox}[\small]{\vinput}
    511
\end{myverbbox}
\begin{myverbbox}[\small]{\voutput}
    10
\end{myverbbox}
\inputoutputTable

\solutionSection

Очевидно, что это задача на использование имеющегося в микроконтроллере ATMEGA аналого-цифрового преобразователя (АЦП).  На уровне библиотеки Arduino, чтение аналогового значения на входе АЦП выполняется функцией analogRead().  Эта функция возвращает значения в диапазоне от 0 (на входе - нулевое напряжение) до 1023 (напряжение питания).  При подключении на аналоговый вход делителя напряжения, изображенного на схеме,  напряжение на входе будет составлять:
$$V_x = V_{cc} \cdot R_x / (R_1 + R_x),$$ а значение, принятое с АЦП: $$X_{ADC} =  1023 \cdot  R_x / (R_1 + R_x)$$
Решая это уравнение относительно $R_x$ , получаем:  
$$R_x = R_1  \cdot X_{ADC} / (1023  - X_{ADC})$$ 
При реализации вычислений на C, всегда следует обращать внимание на правильность задания типов данных и на возможные ошибки округления.  Наш код может выглядеть так:

\begin{minted}[fontsize = \footnotesize, linenos]{cpp}
#include <math.h>
Float R1 = 10.0;
void loop()
{
    int x = analogRead(A0);
    float R = R1 * x / (1023 - x); 
    Serial.println(round(R));
}
\end{minted}

Разумеется, нельзя забывать и про правильную инициализацию как пинов, так и последовательного порта (хотя при старте Arduino все пины уже и так находятся в режиме INPUT, желательно прописать pinMode() просто для большей ясности):

\begin{minted}[fontsize = \footnotesize, linenos]{cpp}
void setup()
{
    pinMode(A0, INPUT);    
    Serial.begin(9600);
}
\end{minted}