\assignementTitle{}{5}

Вам надо спроектировать узел линейного перемещения для автомата, который раскладывает грузы по 8 ячейкам.  Ячейки выстроены в одну линию,  расстояние между соседними ячейками 40 мм.  В качестве привода применен сервомотор с шестерней и зубчатой рейкой, как показано на рисунке:

\putImgWOCaption{10cm}{1}

Сервопривод имеет вращательный момент (stall torque) 0.2 H*м  и диапазон поворота вала 270 градусов.  Какое максимальное усилие можно получить на зубчатой рейке, при условии, что узел линейного перемещения должен обеспечивать позиционирование схвата над центром любой из ячеек?   

Введите максимальное усилие, в Ньютонах (вводить только число),  с точностью не ниже 0.1Н.