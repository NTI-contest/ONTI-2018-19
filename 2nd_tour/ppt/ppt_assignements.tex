Во втором этапе вам будут предложены задания, которые не только проверят вашу готовность к финалу, но и помогут к нему подготовиться.  Задание финала будет связано с моделированием в САПР некоего устройства, изготовлением его деталей на станках с ЧПУ (3D-принтерах, лазерном и фрезерном),  сборкой и наладкой электронной начинки и, наконец, программированием полученного устройства.  Само проектируемое устройство будет некой разновидностью координатного устройства с ЧПУ. В задании понадобится также программировать на ПК (Python или C++), используя библиотеку машинного зрения OpenCV.

Команда должна состоять не более чем из 3-х человек, их роли условно обозначены как "конструктор, технолог, электронщик, программист". Желательно. чтобы каждый участник до какой-то степени сочетал несколько ролей,  например, в начале работы лучше иметь больше конструкторов, ближе к концу - больше электронщиков и программистов.  Совокупность знаний и умений участников команды должна включать:

\textbf{Конструктор/технолог:}

Умение работать в САПР (любые из: Autodesk Inventor, Fusion 360, PTC Creo, Компас 3D, другие - по согласованию с организаторами), опыт конструирования технических устройств. Понимание особенностей моделирования под конкретные технологии прототипирования.

Практический опыт изготовления изделий на 3D-принтере, лазерном станке, фрезерном станке с ЧПУ, пост-обработка с использованием ручного и станочного инструмента, сборка.   НЕ требуется обязательно уметь использовать все эти технологии, но разнообразие доступных технологий очевидно облегчает задачу и позволит быстрее получить качественный результат.

\textbf{Электронщик/программист:}

Знание электроники (Ардуино, шаговые двигатели, сервоприводы, работа с распространенными типами датчиков), пайка и монтаж.
Программирование микроконтроллеров (например, Ардуино).
Программирование на ПК, на любом высокоуровневом языке (предпочтительно - Python и/или C++)

\textbf{Общие навыки:}

Навыки командной работы (совместное планирование, распределение работ).
Структура задания. 

Задание 2-го тура состоит из задач в Stepik и практической части. В задачах на Степике имеются разделы для конструктора (проектирование в САПР) и для электронщика- программиста (задачи по основам электроники и по программированию Ардуино). Практическая часть предполагает конструирование и изготовление узла или упрощенного прототипа устройства, сходного с конструкцией из финального тура, также с подзадачами для конструктора и для электронщика/программиста.

Курс на Stepik "ОНТИ 18/19. Передовые производственные технологии. Этап 2." кроме (отыгранных) заданий, содержит полезные ссылки, тренировочные задания и материалы для хакатонов. 

\section{Работа в САПР: общее понимание и базовое моделирование}

\subimport{2nd_tour/course_3D/task_01/}{statement}
\subimport{2nd_tour/course_3D/task_02/}{statement}
\subimport{2nd_tour/course_3D/task_03/}{statement}
\subimport{2nd_tour/course_3D/task_04/}{statement}
\subimport{2nd_tour/course_3D/task_05/}{statement}

\section{Основы электроники}

\subimport{2nd_tour/course_electro/task_01/}{statement}

\section{Программируем Arduino: простые схемы и задачи}

\subimport{2nd_tour/course_electro/task_02/}{statement}
\subimport{2nd_tour/course_electro/task_02/}{solution}
\subimport{2nd_tour/course_electro/task_03/}{statement}
\subimport{2nd_tour/course_electro/task_04/}{statement}

\section{Работа в САПР: Сборки}

\subimport{2nd_tour/course_3D/task_10/}{statement}
\subimport{2nd_tour/course_3D/task_11/}{statement}
\subimport{2nd_tour/course_3D/task_12/}{statement}
\subimport{2nd_tour/course_3D/task_13/}{statement}

\section{Arduino: АЦП и делители напряжения}

\subimport{2nd_tour/course_electro/task_05/}{statement}

\subimport{2nd_tour/course_electro/task_06/}{statement}
\subimport{2nd_tour/course_electro/task_07/}{statement}

\subimport{2nd_tour/course_electro/task_08/}{statement}

\section{Шаговые двигатели и координатные устройства}

\subimport{2nd_tour/course_electro/task_09/}{statement}
\section{Механика}

\subimport{2nd_tour/course_3D/task_14/}{statement}
\subimport{2nd_tour/course_3D/task_15/}{statement}
\subimport{2nd_tour/course_3D/task_16/}{statement}

\section{Практическое конструирование}

\subimport{2nd_tour/ppt/task_23/}{statement}

\section{Практическая электроника (и еще немного конструирования)}

\subimport{2nd_tour/ppt/task_24/}{statement}

\section{Программирование}

\subimport{2nd_tour/ppt/task_25/}{statement}
