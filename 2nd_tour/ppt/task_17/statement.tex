\assignementTitle{Аналоговые кнопки}{4}

Если нужно подключить к устройству несколько кнопок, а пины контроллера приходится экономить, то подключают цепь из нескольких кнопок и резисторов к одному аналоговому входу, так что при разных комбинациях нажатий получаются разные сопротивления цепи, и, соответственно, разные напряжения на  аналоговом входе.  

И вот, имеется такая схема:

\putImgWOCaption{10cm}{1}

Напишите программу, которая сразу после включения определяет нажатые кнопки и передает через последовательный порт одно десятичное число, 
соответствующее их комбинации.  Каждой из кнопок S1..S3 соответствует один бит в полученном числе.  Например, нажатые кнопки S1 и S2 должны 
давать число 3  (2+1).     Ожидаемая скорость передачи - 115200 бод.  Программа должна учитывать,  что сопротивления резисторов могут 
несколько (до $2.5\%$ в каждую сторону) отклоняться от  указанных на схеме значений.

В этом примере, мини-симулятор будет выполнять функцию loop() однократно для каждой комбинации нажатых кнопок.  Никаких  задержек 
и ожиданий в коде писать не надо.  Используются функции analogRead(), pinMode() и упрощенный Serial, как в предыдущих заданиях.

\begin{myverbbox}[\small]{\vinput}
    557
\end{myverbbox}
\begin{myverbbox}[\small]{\voutput}
    2
\end{myverbbox}
\inputoutputTable

%\includeSolutionIfExistsByPath{2nd_tour/bas/task_01}