\assignementTitle{Простой blink}{0}{}

Как устроен мини-симулятор?  Тот же blink, но без магии!

Это опять та же самая программа BLINK и ответ для предыдущей задачи подходит и для этой.   

ЗА ЭТОТ ШАГ БАЛЛЫ НЕ НАЧИСЛЯЮТСЯ, зато вы можете познакомиться со внутренним устройством мини-симулятора 
Arduino, так как в этом шаге его код не скрыт.  Во всех остальных задачах вы его не увидите, но работает он примерно так же. 
Понимая хотя бы примерно, как устроен симулятор, вы, возможно, будете увереннее решать следующие задачки.
 
С точки зрения платформы Stepik, запуск программ Ардуино на мини-симуляторе не отличается от других заданий на программирование:  
программа считывает данные из входного потока и записывает результаты в выходной, где они затем проверяются.  В начале работы, в 
функции main(), мини-симулятор сам считывает из входного потока данные и, в соответствии с ними, устанавливает параметры или имитирует 
внешние воздействия:  например, "напряжение на аналоговом входе" (т.е. результат функции analogRead()) может быть разным в разные 
моменты времени, или в определенные моменты включаются/выключаются определенные пины, приходят данные через последовательный порт и т.п.   

В этой задаче, например, из входного потока вводится номер "мигательного" пина и задержки во включенном и выключенном состоянии в переменные 
LED\_PIN, ON\_PERIOD и OFF\_PERIOD.   Поскольку программа всегда проверяется на нескольких тестах, каждый со своими данными, 
она будет правильно работать на всех тестах только если вы везде, где это нужно по смыслу, используете эти переменные.

Когда из-за действий вашей программы в мини-симуляторе происходят какие-то события, он отправляет сообщения в выходной поток.  
Например, функция \linebreak digitalWrite() при переключении пина посылает сообщения вида "000700DW03=1" ("в момент времени 700мс, пин номер 3 
был установлен в 1").  При заданных входных параметрах, последовательность таких событий будет всегда одинакова, так что мы можем 
проверить правильность поведения программы.

В реальном микроконтроллере, функция loop() выполняется "пока не сядут батарейки".  Мини-симулятор же "крутит loop()" 
только до тех пор, пока не получит достаточно выходных данных для проверки.  Например, в этой задаче функция main() вызывает loop() 
ровно три раза, что должно дать 6 изменений уровня сигнала на пине и, соответственно, 6 строк результата для проверки.  И 
действительно, в тесте-примере вы видите 6 строк.   И в каждой из остальных задач есть какой-то критерий, когда пора прекращать 
выполнение цикла.  Вы можете об этом не задумываться, а просто писать код, как для настоящего Arduino.

\begin{myverbbox}[\small]{\vinput}
    3 300 400
\end{myverbbox}
\begin{myverbbox}[\small]{\voutput}
    000000DW03=1
    000300DW03=0
    000700DW03=1
    001000DW03=0
    001400DW03=1
    001700DW03=0
\end{myverbbox}
\inputoutputTable

%\includeSolutionIfExistsByPath{2nd_tour/bas/task_01}