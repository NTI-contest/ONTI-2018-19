\assignementTitle{Простейший омметр}{2}{}

Вам нужно реализовать на Arduino простейший омметр (измеритель сопротивления), как показано на схеме. 

\putImgWOCaption{9cm}{1}

Для этого к пину A0 подключен делитель напряжения, состоящий из прецизионного резистора 10 кОм "сверху" и измеряемого резистора "снизу".  После того, как очередной измеряемый резистор подключен к схеме, контроллер включается, программа на Arduino производит измерение и через последовательный порт выводит в виде числа значение сопротивления, в кОм, с округлением до целого и с переводом строки на конце.   

Для упрощения,  мы принимаем, что измеряемое сопротивление всегда целое число килоом и отсутствуют любые ошибки измерения. Все значения измеряемых резисторов находятся в диапазоне от 10 кОм до 90 кОм (т.е. допускающем их сравнительно точное измерение при 10кОм подтягивающем резисторе).


Напишите программу для мини-симулятора Arduino, которая выполняет измерения.

Данные для измерения поступают из входного потока и расшифровываются мини-симулятором, а вычисленные вашей программой и отправленные на Serial.println() значения передаются в выходной поток для проверки.  Мини-симулятор однократно вызывает функцию loop() для выполнения каждого измерения. 

Разрешается пользоваться функциями pinMode(), analogRead() ,  а также упрощенным классом Serial c методами begin(), print() и println().  Передача данных должна происходить на скорости 9600 бод.

\begin{myverbbox}[\small]{\vinput}
    511
\end{myverbbox}
\begin{myverbbox}[\small]{\voutput}
    10
\end{myverbbox}
\inputoutputTable

%\includeSolutionIfExistsByPath{2nd_tour/bas/task_01}