\assignementTitle{Складской робот}{5}{}

Имеется склад, представляющий собой прямоугольный массив ячеек. Размер массива задается входными данными. В ячейках могут размещаться грузы, опознаваемые по номерам. Все грузы уникальны (имеют разные номера), Имеется робот, который может перемещаться над ячейками, брать и переносить грузы. Робот умеет выполнять следующие команды:  "G"\ (grab) - взять груз, "Mx,y"\ (move) - переместиться к ячейке с заданными индексами, "P"\ (put) - положить груз в ячейку.  Индексы считаются от 0.

Напишите программу, расставляющую грузы на складе заданным образом.

Для упрощения примем, что ячейка, где должен находиться данный груз, не занята другим грузом (т.е. задача гарантированно решается за один проход, без перекладывания грузов в промежуточные ячейки). Оптимальность перемещений робота не учитывается, но избыточное число перемещений считается ошибкой.

\inputfmtSection
На входе (по строкам):  

N M - размеры склада

K  - число грузов (всегда меньше числа ячеек N$\cdot$M)

X Y Т  - К строк, описывающих начальное расположение грузов, где X,Y - координаты ячейки, T - числовой код груза

X Y Т - еще К строк, описывающих требуемое конечное расположение грузов.

\outputfmtSection
Последовательность строк с командами для робота (G, P и Mx,y), по одной команде на строку. Полученная последовательность интерпретируется программой автопроверки. Ошибками считаются: неверная команда, попытка взятия груза из пустой ячейки, помещение груза в уже заполненную ячейку, попытка взять груз, когда робот уже несет груз, попытка выгрузить, когда груз не был взят, избыточное число команд, неверное размещение грузов после выполнения программы.

\begin{myverbbox}[\small]{\vinput}
    3 2
    2
    0 0 2
    2 1 1
    0 1 1
    2 0 2
\end{myverbbox}
\begin{myverbbox}[\small]{\voutput}
    M2,1
    G
    M0,1
    P
    M0,0
    G
    M2,0
    P
\end{myverbbox}
\inputoutputTable

\solutionSection

Публикуемое ниже решение - одно из многих решений, присланных участниками 2-го тура.  Решение прошло тесты и зачтено как правильное, хотя у автора задания есть подозрения, что правильность работы этой программы не гарантируется при изменении порядка следования записей (X Y T) во входных данных. Добавлены только комментарии:

\includeSolutionIfExistsByPath{2nd_tour/ppt/task_25}