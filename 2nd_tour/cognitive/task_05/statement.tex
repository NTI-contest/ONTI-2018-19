\assignementTitle{}{2}{}

В парадигме ИМК Дончина и Фарвелла, позволяющей печатать при помощи «силы мысли», используются:

\begin{enumerate}
    \item показатели компонента P300
    \item показатели компонента N100
    \item показатели ЭАК и ЭЭГ
    \item показатели дельта-волны
    \item показатели SSVEP
\end{enumerate}

\explanationSection

Именно на показателях P300 основана парадигма интерфейса мозг-компьютер Дончина и Фарвелла. Индивиду предъявляется таблица размером шесть на шесть состоящая из букв и цифр. Задача испытуемого посчитать сколько раз подсветили букву, которую он хотел напечатать. В связанном событии потенциале мозга выделяются позитивные-негативные отклонения, связанные с различными когнитивными процессами. Компонент P300 развивается примерно через 300 миллисекунд после начала предъявления стимула. После нескольких повторений вспышек становится возможным по показателям Р300 понять, в какой строке или столбце содержится буква, которую пользователь хочет написать. 

\answerMath{1.}