\assignementTitle{}{3}{}

Google-таблица по адресу [\url{https://goo.gl/JKjHgh}] хранит в себе последовательность байт в виде разбитой на ячейки зашифрованной шестнадцатеричной строки, по две цифры в строке на байт. Шифр основан на сдвиге, таком, что цифрам 0..F соответствуют символы ‘a’..‘p’ (первые шестнадцать букв латинского алфавита). Эта шестнадцатеричная строка представляет собой двоичный файл, внутри которого хранится изображение в некотором известном формате. Изображение, в свою очередь, хранит закодированный сигнал в виде последовательности бит. Пиксели от начала рисунка, слева направо и сверху вниз, содержат эту информацию побитно в виде модуля разности между красным и синим каналом каждого пикселя (в модели RGB). Если разница между красным и синим каналом по модулю больше или равна основанию двоичной системы счисления (превышает ее размерность), это сигнализирует о конце последовательности бит. Представленный сигнал сильно зашумлен, однако, значимые, осмысленные сочетания бит, сгруппированные побайтно, преобладают по частоте появления. Выявите такие, наиболее часто повторяющиеся, сегменты в последовательности. 

Возвращаемое значение: перечисление повторяющихся сигналов по порядку их первого появления в последовательности через запятую без пробелов (упорядочивание осуществлять для случая с шагом в 1 бит, а не по байтовой границе).

Пример ответа (здесь и далее – ответ вводить без кавычек): 

"11110101,11001100,01011111,00010000"

%\includeSolutionIfExistsByPath{2nd_tour/bas/task_01}