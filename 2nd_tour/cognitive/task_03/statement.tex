\assignementTitle{}{2}{}

Парадигма ИМК, базирующаяся на показателях компонента N400, использует

\begin{enumerate}
    \item воображение движения правой и левой рукой
    \item изменения в показателях альфа-ритма
    \item показатели степени отклонения предъявляемого стимула от ожиданий индивида
    \item показатели скорости чтения
\end{enumerate}

\explanationSection

Правильный ответ: показатели степени отклонения предъявляемого стимула от ожиданий индивида, так как амплитуда N400 изменяется в зависимости от того, является ли событие ожидаемым или нет.

\answerMath{3.}