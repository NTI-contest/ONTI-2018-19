\assignementTitle{}{3}{}

Программой обрабатывается строка - последовательность символов из входного потока, которая, в результате обработки, преобразуется в новую строку - битовую последовательность.
Преобразование происходит по следующим законам:

\begin{itemize}
    \item Код каждого символа рассматривается либо как 1 байт (с учетом разрядов), либо как 1 бит.
    \item Символы 0 и 1 рассматриваются как бит (занимают 1 разряд) с соответствующим значением, остальные символы - как байт (занимают 8 разрядов) и переводятся в двоичную систему с учетом ведущих нулей.
    \item Символы 2..v и 2..V рассматриваются как цифры в 32-ричной системе счисления (по аналогии с 16-ричной).
    \item Остальные символы интерпретируются как значение их кода в таблице символов.
    \item Для символов, код которых превышает разрядность байта, все биты устанавливаются в 1.
    \item Если число бит в прочитанной последовательности не кратно 8, результирующая последовательность дополняется слева ведущими нулями до состояния соответствующей кратности.
    \item В случае, если результирующая последовательность пуста, она дополняется 8 нулевыми битовыми разрядами.
\end{itemize}

В полученной битовой последовательности программа находит все наиболее часто встречающиеся комбинации бит длиной, кратной 1 байту. Найденные комбинации выстраиваются сначала по длине (от наиболее длинных к коротким), а затем по порядку их появления в последовательности слева направо (от первых к последним).

Возвращаемая в выходной поток строка должна иметь формат:
ПОСЛЕДОВАТЕЛЬНОСТЬ;КОМБИНАЦИЯ,КОМБИНАЦИЯ,<...>,КОМБИНАЦИЯ.

где: ПОСЛЕДОВАТЕЛЬНОСТЬ - битовая последовательность; КОМБИНАЦИЯ - одна из найденных комбинаций. Запятые и точка - по образцу.

%\includeSolutionIfExistsByPath{2nd_tour/bas/task_01}