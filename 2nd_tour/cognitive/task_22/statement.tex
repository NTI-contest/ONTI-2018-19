\assignementTitle{}{2}{}

Какая проблема этического характера не возникает при использовании интерфейсов «мозг-компьютер»

\begin{enumerate}
    \item получение согласия на начало использования интерфейса «мозг-компьютер» у людей, испытывающих сложности в общении;
    \item проблема ответственности за ошибочные движения, совершенные нейропротезом;
    \item «чтение мыслей» индивида посредством анализа различных показателей работы его мозга и вопросы конфиденциальности;
    \item размывание границ между человеком и техническим устройством;
    \item проблема использования подобных технологий при допросе;
    \item проблема замены человека компьютерными устройствами.
\end{enumerate}

\explanationSection

При разработке интерфейсов «мозг-компьютер» всплывает множество этических вопросов. Как мы можем узнать, согласны они или нет, если мы ещё не начали применять интерфейс (комментарий к варианту 1). Кто несет ответственность за совершаемое нейропротезом действие: то ли сам человек, который дал команду этому движению, то ли плохо работающее устройство (комментарий к варианту 2).  Получается, что эти системы позволяют получать какую-то информацию и, возможно, человек не согласен на то, что он становится открытой книгой (комментарий к варианту 3). Размывание границ между человеком и техническим устройством – это тоже этическая проблема использования интерфейса «мозг компьютер». Подобные технологии позволяют узнать информацию, которую человек предпочел бы скрыть и поэтому это также проблема этического характера (комментарий к варианту 5).  Проблема замены человека компьютерными устройствами не является этической проблемой интерфейсов «мозг-компьютер».

\answerMath{6.}