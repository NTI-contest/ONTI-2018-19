\assignementTitle{}{3}{}

Программа получает на вход строку, представляющую собой последовательность из 64 шестнадцатеричных цифр, а также 2 шестнадцатеричные цифры-координаты – координаты X и Y точки начального расположения агента в лабиринте 16x16 [в диапазоне 0..F] – все три компонента строки разделены точкой с запятой. Если каждую 16-ричную цифру преобразовать в ее двоичный аналог, а полученную строку вновь соединить, получится двоичное представление лабиринта (0 – свободная клетка, 1 – занятая, «стена») слева направо и сверху вниз. Если свободная клетка находится на границе карты, и от нее можно построить маршрут до начального местоположения агента, последовательно оставаясь на месте либо перемещаясь вправо, влево, вверх и/или вниз, то такая клетка называется «выходом». Если агент изначально расположен на занятой клетке, то считается, что агент «вмурован» в стену, и выходов не существует. Свободная клетка, из которой невозможно осуществить движение ни вправо, ни влево, ни вверх, ни вниз (из-за границы карты или занятой смежной клетки), называется «тупиком». Программа возвращает суммарное количество всех выходов и тупиков в лабиринте.

\begin{myverbbox}[\small]{\vinput}
    5555aaaa5555aaaaffffffffffffffffffffffffffffffff5555aaaa5555aaaa;2;0
\end{myverbbox}
\begin{myverbbox}[\small]{\voutput}
    65
\end{myverbbox}
\inputoutputTable

Cпецифический случай: 64 тупика, при этом один из них - выход (под агентом, вторая клетка с левого края в верхнем углу), 64+1=65.

%\includeSolutionIfExistsByPath{2nd_tour/bas/task_01}