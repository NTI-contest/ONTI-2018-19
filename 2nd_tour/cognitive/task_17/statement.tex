\assignementTitle{}{3}

По адресу [\url{https://goo.gl/5YtqzN}] находится Google-таблица, зашифрованная по 
аналогии с первым заданием. Она содержит 64-разрядный файл в формате “Shared Object” (расширение .so). 
По сути, это динамическая библиотека. Найдите способ подключить эту библиотеку (например, с помощью 
PythonAnywhere) и получите сообщение, вызвав функцию Start(), объявленную в библиотеке.

Библиотека содержит внутри себя виртуальное пространство, которое представляет собой лабиринт 16x16. 
Управляемый агент находится в лабиринте. Для перемещения по лабиринту используются команды “вправо”, 
“влево”, “вверх” и “вниз”. Команды зашифрованы в виде комбинаций обнаруженных в прошлом задании четырех 
сигналов, от двух до четырех различных сигналов в комбинации. Определите соответствие комбинаций сигналов 
и команд, а также составьте последовательность сигналов, приводящую к выходу из лабиринта, в виде строки.
 Функция GetX() динамической библиотеки возвращает координату агента по X, функция GetY() - по Y (где {X=0, Y=0} – левый верхний угол). Функция Move(signal) принимает на вход сигнал в виде целого числа в диапазоне одного байта, а возвращает текущее состояние агента: 0 – в лабиринте, 1 – покинул лабиринт. Дополнительно, функция Move изменяет координаты местоположения агента, если переданный в составе комбинации сигнал имеет смысл, а движение в соответствующем направлении не заблокировано.

Второе возвращаемое значение: перечисление переданных комбинаций сигналов в порядке, согласующемся с соответствующим каждому из них направлением, по часовой стрелке (“вверх”, “вправо”, “вниз”, “влево”) через запятую без пробелов. Сигналы представить в двоичной форме. Отдельные сигналы (8 бит) в комбинации пишутся слитно.

Пример ответа: “1111010111001100, 1100110001011111, 010111110001000011110101, 0001000001011111”

%\includeSolutionIfExistsByPath{2nd_tour/bas/task_01}