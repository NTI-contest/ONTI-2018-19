\assignementTitle{}{3}

Библиотека, загруженная в предыдущем задании, содержит еще одно пространство, замкнутый лабиринт 16x16, составленный и действующий по тому же принципу, что и лабиринт в предыдущем задании. Для переключения в это пространство достаточно вызвать функцию Task3() (в самом начале) из библиотеки. По лабиринту случайным образом перемещаются, вслед за игроком, существа-телепорты. При соприкосновении с ними игрок телепортируется в случайную точку лабиринта; при этом само существо застывает на месте и больше не перемещается.

Напишите алгоритм полного обхода лабиринта. Составьте рисунок в формате png, 16x16, на котором будет отображено полностью строение всего лабиринта (под каждую клетку лабиринта отводится пиксель), причем синим цветом (RGB:0,0,255) отображаются свободные клетки, а красным (RGB:255,0,0) – “занятые” (стены). Телепорты на карте не отображаются.

Сохраните рисунок в файле под названием “map.png”, располагающемся в том же каталоге, что библиотека и вызывающая ее программа; вызовите функцию Check(). Если приведенная иллюстрация соответствует внутреннему представлению пространства, то будет сгенерировано сообщение №1.

Перезагрузите программу. Вызовите функции Task3() и Check() последовательно. Убедитесь, что теперь агент не совершил ни одного движения. Используйте функцию Ghost(x,y) для поиска телепортов. Функция опрашивает указанную часть пространства и возвращает 0, если телепорт отсутствует в указанных координатах x и y, 1 – если существует. Осуществите вызов для каждой “свободной” (синей) клетки, кроме той, на которой находится агент, но не более, чем необходимо (только один раз, а “занятые” клетки не опрашивать). Передайте количество (count) телепортов в функцию Last(count). Если все вызовы Ghost были осуществлены корректно, функция отобразит сообщение №2.

Третье возвращаемое значение: сообщение №2.

%\includeSolutionIfExistsByPath{2nd_tour/bas/task_01}