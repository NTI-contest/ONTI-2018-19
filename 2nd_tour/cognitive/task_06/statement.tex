\assignementTitle{}{10}{}

Оператор, от которого требуется высокий и стабильный уровень внимания, оказался не в состоянии достаточно хорошо и стабильно фокусировать своё внимание, дополнительно Вы обнаружили у него повышенную частоту сердечных сокращений и повышенный тонус скелетных мышц. Какие мероприятия следует применить для скорейшего восстановление рабочих возможностей оператора?

\begin{enumerate}
    \item приём пищи;
    \item дыхание с акцентом на фазе вдоха;
    \item дыхание с акцентом на фазе выдоха;
    \item 30 минут на велотренажёре со средней интенсивностью;
    \item 10 минут на велотренажёре с высокой интенсивностью;
    \item разминка в высокоинтенсивном темпе с быстрыми амплитудными движениями;
    \item разминка в низкоинтенсивном темпе с медленными амплитудными движениями;
    \item быстрый массаж с похлопываниями по спине;
    \item медленный массаж без похлопываний по спине.
\end{enumerate}

\explanationSection

Исходя из условия задачи, определяем, что оператор находится в состоянии симпатикотонии  – повышенный тонус симпатического отдела вегетативной нервной системы (ВНС), на это указывает повышенная частота сердечных сокращений и повышенный мышечный тонус. Кроме того, одним из следствий симпатикотонии является рассеянность внимания. Следовательно, для приведения оператора в сбалансированное состояние необходимо применить мероприятия, снижающие тонус симпатического отдела ВНС либо повышающие тонус парасимпатического отдела ВНС. К таким мероприятиям относятся пункты 1, 3,4, 7, 9 из предложенных вариантов ответа.

\putImgWOCaption{16cm}{1}

\answerMath{1, 3, 4, 7, 9.}