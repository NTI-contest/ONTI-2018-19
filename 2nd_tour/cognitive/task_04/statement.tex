\assignementTitle{}{2}{}

SSVEP — это

\begin{enumerate}
    \item колебательная активность ЭЭГ, частота которой синхронизируется с частотой звука
    \item колебательная активность ЭЭГ, частота которой синхронизируется с частотой мерцающего визуального стимула, на который пользователь обращает внимание
    \item колебательная активность ЭЭГ, наблюдающаяся во время воображения движения
    \item метод стимуляции мозга, позволяющий управлять движениями другого человека
\end{enumerate}

\explanationSection

Правильный ответ: колебательная активность ЭЭГ, частота которой синхронизируется с частотой мерцающего визуального стимула, на который индивид обращает внимание. В исследованиях с SSVEP используют несколько стимулов, каждый из которых имеет конкретную частоту мерцания. По показателям SSVEP можно понять, на каком стимуле индивид сосредоточил свое внимание. SSVEP в ответ на стимул, на который индивид обратил внимание, может быть переведен в конкретную команду для интерфейса «мозг-компьютер». Такой интерфейс мозг-компьютер был использован для управления движением влево и право в плоскости в симуляторе полета.

\answerMath{2.}