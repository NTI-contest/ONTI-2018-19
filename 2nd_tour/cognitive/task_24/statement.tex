\assignementTitle{}{2}{}

Показателем стресса при решении математических задач в сигналах ЭЭГ является

\begin{enumerate}
    \item Уменьшение мощности $\beta-$ и $\alpha-$ритмов
    \item Уменьшение мощности $\delta-$, $\alpha-$ и $\theta-$ ритмов
    \item Увеличение мощности $\delta-$ритма
    \item Уменьшение мощности $\alpha-$ритма
    \item Увеличение мощности $\alpha-$ритма
\end{enumerate}

\explanationSection

Когда человек расслабляется в записях электроэнцефалограммы появляется альфа-ритм. Уменьшение мощности альфа-ритма может служить показателем стресса при решении математических задач. Показателей этого ритма может оказаться достаточно, чтобы определить, испытывает ли человек стресс.

\answerMath{3.}