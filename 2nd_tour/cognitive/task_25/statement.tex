\assignementTitle{}{2}{}

Какая аббревиатура не имеет отношение к парадигмам интерфейсов «мозг-компьютер»

\begin{enumerate}
    \item DMI
    \item DNI
    \item BMI
    \item MMI
    \item BCI
    \item NCI
\end{enumerate}

\explanationSection

Давайте посмотрим, как расшифровываются эти аббревиатуры:
\begin{itemize}
    \item BCI, Brain-computer interface;
    \item NCI, Neural-control interface;
    \item MMI, Mind-machine interface;
    \item DNI, Direct neural interface;
    \item BMI, Brain-machine interface;
    \item DMI - это случайное сочетание букв
\end{itemize}

Правильный ответ: DMI. Все остальные варианты используются для обозначения интерфейсов «мозг-компьютер».

\answerMath{1.}