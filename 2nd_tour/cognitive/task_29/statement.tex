\assignementTitle{}{3}

На ввод подается иллюстрация лабиринта 16x16, красные точки \linebreak (RGB=FF0000) – стены, синие точки (RGB=0000FF) – свободное пространство, точки другого цвета – пирожные (свободное пространство, на котором располагается “пирожное”, которое агент “съедает” в случае присутствия в соответствующей клетке, делая ее “синей” и увеличивая счетчик съеденных пирожных на 1). Агент может перемещаться строго вправо, влево, вверх или вниз (но не по диагонали). Границы лабиринта – стены, кроме двух синих точек – выходов из лабиринта. На вывод подается число пирожных, которые может съесть агент при прохождении максимально короткого маршрута (в случае, если маршрутов несколько, выбирается маршрут с наибольшим количеством пирожных) от одного выхода до другого. Ввод данных задается строкой в представлении base64, совместимом с нотацией адресной строки браузера; рисунок формата PNG. Вывод данных – целочисленное значение (количество съеденных пирожных).

\begin{myverbbox}[\small]{\vinput}
    data:image/png;base64,iVBORw0KGgoAAAANSUhEUgAAABAAAAAQCAYAAAAf8/9hAA
    AAV0lEQVQ4T+2TQQoAIAgEx+/2oL5rBHkRLaFb5E3BUZdVFBQXgqKIL4e5lAAN6DGvBN
    htlAL8vOykEGDNFS0+AF7WYHrAYvcXuQbL/yczpYD7DYDT9HniAPuYQ/GPPzARAAAAAE
    lFTkSuQmCC
\end{myverbbox}
\begin{myverbbox}[\small]{\voutput}
    1
\end{myverbbox}
\inputoutputTable

%\includeSolutionIfExistsByPath{2nd_tour/bas/task_01}