\assignementTitle{}{10}{}

Одна из ключевых функций нервной системы – хранение информации, т.е. формирование памяти. Какие особенности нейронов лежат в основе механизмов памяти?

\begin{enumerate}
    \item способность нейрона синтезировать нейромедиаторы
    \item способность нейрона осуществлять сальтаторное проведение нервных импульсов
    \item способность нейрона изменять проводимость своих синапсов
    \item способность нейрона находиться в состоянии потенциала покоя
    \item способность нейрона находиться в состоянии потенциала действия
\end{enumerate}

\explanationSection

Синаптическая пластичность – способность синапсов под влиянием свой деятельности изменять свою структуру и характеристики, т.е. способность перестраиваться под нагрузкой. Именно благодаря этому свойству опыт (предыдущая деятельность) может закрепляться в структуре и свойствах нервных сетей. Ключевой «рабочей» характеристикой отдельно взятого синапса является его проводимость.

\answerMath{3.}