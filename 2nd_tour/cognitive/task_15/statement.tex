\assignementTitle{}{2}

Примером реактивного интерфейса «мозг-компьютер» будет

\begin{enumerate}
    \item система, которая извлекает соответствующие показатели из непрозвольных измерений текущей активности центральных и периферических отделов нервной системы, и использует эти показатели для мониторинга и адаптации схемы взаимодействия между компьютером и человеком
    \item система, в которой выявляются необходимые показатели активности мозга, которые переводятся в команды внешнему устройству. Примером может служить система, в которой индивид активно воображает движение руки, чтобы изменить свой сенсомоторный ритм
    \item система, в которой задача индивида заключается в том, чтобы сосредоточить внимание на определенном экзогенном стимуле, который он хочет выбрать из перечня других стимулов, что ведет к появлению определенных маркеров в связанном с событием потенциале мозга, наблюдаемых при предъявлении этого стимула. Примером будет служить система, позволяющая печатать при помощи силы мысли, когда индивид считает количество мельканий нужной буквы
\end{enumerate}