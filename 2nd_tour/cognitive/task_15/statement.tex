\assignementTitle{}{2}{}

Примером реактивного интерфейса «мозг-компьютер» будет

\begin{enumerate}
    \item Система, которая извлекает соответствующие показатели из непроизвольных измерений текущей активности центральных и периферических отделов нервной системы, и использует эти показатели для мониторинга и адаптации схемы взаимодействия между компьютером и человеком. И это так называемый пассивный интерфейс «мозг-компьютер»;
    \item Система, в которой выявляются необходимые показатели активности мозга, которые переводятся в команды внешнему устройству. Примером может служить система, в которой индивид активно воображает движение руки, чтобы изменить свой сенсомоторный ритм. Это пример активного интерфейса «мозг-компьютер»;
    \item Система, в которой задача индивида заключается в том, чтобы сосредоточить внимание на определенном экзогенном стимуле, который он хочет выбрать из перечня других стимулов, что ведет к появлению определенных маркеров в связанном с событием потенциале мозга, наблюдаемых при предъявлении этого стимула. Примером будет служить система, позволяющая печатать при помощи силы мысли, когда индивид считает количество мельканий нужной буквы.
\end{enumerate}

\explanationSection

1 вариант – это пример пассивного интерфейса «мозг-компьютер». 2 вариант – это пример активного интерфейса «мозг-компьютер».

\answerMath{3.}