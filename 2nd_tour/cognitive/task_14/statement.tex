\assignementTitle{}{2}{}

Сенсомоторный ритм, показатели которого необходимы для реализации одного из видов интерфейсов мозг-компьютер, регистрируется в сенсомоторных областях коры головного мозга. Для этого необходимы следующие отведения:

\begin{enumerate}
    \item C3, C4, Cz
    \item T3, T5, O1
    \item F3, F4, F8, F7, Fp1, Fp2
    \item P3, P4, Pz
\end{enumerate}

\explanationSection

Может быть вы уже прочитали, что есть специальная система постановки электродов, где точки куда мы ставим электроды отмечаются латинскими буквами и соответствующими числами. Нам потребуются именно отведения C3, C4, Cz, потому что они располагаются над необходимыми областями сенсомоторной области коры. Эти отведения подойдут нам для того, чтобы зарегистрировать сенсомоторный ритм, который также называется мю-ритмом.

\answerMath{1.}