\assignementTitle{}{10}{}

Мужчины, в отличие от женщин, в большей степени склонны проявлять агрессивное поведение. Это проявляется в большей частоте правонарушений со стороны мужчин, в большем числе заключённых мужского пола. Кроме того, мужчины чаще проявляют инициативные, лидерские качества для организации работы коллективов или самостоятельной предпринимательской деятельности. Какие особенности мужчин обуславливают вышеуказанные межполовые различия:

\begin{enumerate}
    \item наличие Y-хромосомы
    \item отсутствие второй Х-хромосомы
    \item усиленное развитие костно-мышечной системы
    \item наличие развитой бороды и/или усов
    \item пониженный уровень эстрогенов
    \item повышенный уровень тестосторонов
\end{enumerate}

\explanationSection

Ключевые фенотипические отличия мужчин и женщин формируются на основе различий в работе гормональной системы. Само по себе наличие Y-хромосомы не приводит к формированию характерной костно-мышечной структуры, оволосения тела или поведенческих особенностей. В то же время, если у женщины случается нарушение работы гормональной системы и начинается чрезмерный синтез тестостеронов, то её фенотип начинает тяготеть к мужскому как в плане строения тела, так и в плане поведения. Усиленное развитие костно-мышечной системы и наличие развитой бороды и/или усов (варианты ответов 3 и 4) являются следствием повышенного уровня тестостеронов. Поэтому правильный ответ – 6.

\putImgWOCaption{7cm}{1}

\answerMath{6.}