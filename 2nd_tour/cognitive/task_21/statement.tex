\assignementTitle{}{2}{}

При грубых нарушениях сознания, когда пациент не способен дать двигательный ответ на вопросы, предлагается использовать интерфейсы «мозг-компьютер», основанные

\begin{enumerate}
    \item На реакциях мозга на зрительно предъявленные стимулы
    \item На изменении диаметра зрачка
    \item На движениях глаз и морганиях
    \item На вибротактильной стимуляции
\end{enumerate}

\explanationSection

Очевидно, что если человек лежит с закрытыми глазами, то мы не можем использовать показатели движений глаз и моргания, измерять диаметр зрачка, а также предъявлять зрительные стимулы.
Следовательно, мы будем использовать вибротактильную стимуляцию.

\answerMath{4.}