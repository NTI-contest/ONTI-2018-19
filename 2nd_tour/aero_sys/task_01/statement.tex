\assignementTitle{}{30}

Задача проверяет базовые навыки работы с Robot Operating System, и системой управления колесных роботов.  
Решением задачи является присланный python скрипт. От одной команды принимается только 2 решения (две попытки)!

Выполнение задания:

\begin{enumerate}
    \item Установить ROS (ros-melodic-desktop-full) в используя пакеты на дистрибутив Ubuntu. Инструкция \url{http://wiki.ros.org/melodic/Installation/Ubuntu}
    \item Запустить и настроить ROS, изучить базовые принципы (Книжка "Введение в ROS" \url{http://docs.voltbro.ru/starting-ros/})
    \item Установить изучить пакет turtlesim \url{http://wiki.ros.org/turtlesim}
    \item Написать python скрипт, который перемещает черепашку по квадрату с со сторонами 3 метра, с линейной скоростью 0.5 м/с, и поворачивает со скорость 0.5 рад/с. Черепаха после перемещения по всем вершинам квадрата должны вернуться в исходное положение (вершина, направление)
    \begin{enumerate}
        \item Старт робота должен происходить из “нижней-левой” вершины квадрата, из координат переданных эмулятором после старта работы (сбрасывать положение черепахи не надо).
        \item Алгоритм движения робота: Движение по прямой, остановка, поворот налево, остановка, движение и тд.
        \item В каждой вершине квадрата, программа должна вывести в консоль координаты (включая стартовую и конечную вершины)
        \item После завершения движения черепахи, программа должна завершиться выйти
        \item Допускается погрешность работы с координатами не более $4\%$ относительно предыдущей и новой точки.
    \end{enumerate}
    \item Запущенная программа должны обрабатывать значения одометрии, подготовленные симулятором, а не работать по времени.
\end{enumerate}

Требования к решению и алгоритм проверки

\begin{enumerate}
    \item Прислать исполняемый код для Python 2.7. Весь исполняемый код должен быть в одном файле. Название файла должно быть латинскими символами, соответствовать названию команды и через дефис номеру попытки (Пример:  iskra-1.py ).
    \item На проверочной машине будут установлен пакет ros-melodic-desktop-full.
    \item Оператор проверяет наличие кода обработки данных одометрии робота.
    \item Оператор запускает roscore и turtlesim\_node
    \item Оператор запускает исполняемый файл python из командной строки: python filename.py.
    \item Оператор проверяет движение робота в симуляторе.
    \item Оператор ожидает завершения работы программы.
\end{enumerate}