Наш трек, предполагает довольно обширные знания в области разработки программного обеспечения. При этом важно отметить что именно "программирование" занимает тут не очень большую часть, а основная роль — это комплексное обслуживание и администрирование различных систем.

Одна из важных компетенций для нашего трека — это работа с операционными системами семейства Linux и мета операционной системой Robot Operation System (ROS).

Robot Operating System (ROS) - это гибкая платформа (или другими словами фреймворк) для разработки программного обеспечения специально для роботов. ROS включает в себя множество разнообразных инструментов, библиотек и определенных правил, целью которых является упрощение и унификация задач разработки ПО роботов.

ROS была создана чтобы стимулировать совместную разработку программного обеспечения робототехники. Каждая отдельная команда может работать над одной конкретной задачей, но использование единой платформы, позволяет всему сообществу получить и использовать эти результаты для своих проектов.

В текущий момент все что связано с ROS, работает только под управлением Linux. Таким образом для решения задач нам необходимо установить Linux.

\subsubsection*{Установка Linux}

Один из самых распространенных дистрибутивов Linux это дистрибутив Ubuntu. Именно этот дистрибутив мы и рассмотрим. Выбирайте версию \textbf{Ubuntu 18.04.1 LTS}.

Вы можете установить Linux несколькими способами.
\begin{enumerate}
    \item Второй операционной системой, при загрузке вашего компьютера вы выберите какую ОС загружать. 
    \begin{enumerate}
        \item \url{https://www.ubuntu.com/download/desktop} Скачать, записать на CD или flash и далее действовать по инструкциям инсталлятора. Это один из самых сложных и довольно радикальных способов. Подробная инструкция \url{https://help.ubuntu.ru/wiki/ubuntu_install}
    \end{enumerate}
    \item Запустить вторую ОС через виртуализацию. Для windows и MacOS это может быть программа \textbf{VirtualBox}.
    \item Для MacOS можно использовать бесплатную версию Parallels Desktop Lite (\url{https://itunes.apple.com/ru/app/parallels-desktop-lite/id1085114709?mt=12}). При установке только Linux версия не требует оплаты.
    \item Для Windows 10, есть возможность запустить Ubuntu встроенными средствами, через магазин приложений Microsoft Store.
\end{enumerate}

Самый простой способ начать работать с Linux, это установить программу виртуализации \textbf{VirtualBox}. Это программа позволит запустить "гостевую" ОС на уже работающую Windows.

Более подробно и с картинками можно прочитать по ссылке \url{https://guides.hexlet.io/ubuntu-linux-in-windows/} Часть для VirtualBox.

Краткое описание примера установки Linux на VirtualBox
\begin{enumerate}
    \item Устанавливаем VirtualBox
    \item Скачиваем дистрибутив (*.ico) файл.
    \item Запускаем Виртуальную машину, указываем диск и следуем инструкциям инсталлятора.
    \item Важно выполнить настройку единого буфера обмена, иначе работать будет не так удобно.
\end{enumerate}

Если вы установили Linux впервые, очень рекомендуем пройти курс по работе с командной строкой от компании Хакслет \url{https://ru.hexlet.io/courses/cli-basics}. Работа с командной строкой может вам показаться довольно "архаичным" занятием, но это научит вас работать с любым Линуксом, при этом вне зависимости — установлен он у вас, находиться на удаленном компьютере или даже в космосе.

\subsubsection*{Установка ROS}

Далее вам необходимо установить ROS. Для дистрибутива Ubuntu, есть довольно простой способ это сделать — это установить все из пакетов (Для других ОС, например Raspbian, которая по умолчанию ставить на Raspberry вам предаться компилировать все пакеты, такая установка занимает около 3 часов)

Официальная страница с инструкцией \url{http://wiki.ros.org/melodic/Installation/Ubuntu}

Введение в ROS на русском языке \url{http://docs.voltbro.ru/starting-ros/}

Запустим терминал. (Ctrl-Alt-T) или найдем его в меню запуска.

Пакеты ROS не включены в список пакетов доступных для установи в "стандартном" дистрибутиве Ubuntu. Поэтому нам надо добавить и настроить новый репозиторий.
\begin{enumerate}
    \item Добавим репозиторий пакетов ROS в список (выполним команду в консоли)\\
    sudo sh -c 'echo "deb http://packages.ros.org/ros/ubuntu \$(lsb\_release -sc) main" > /etc/apt/sources.list.d/ros-latest.list'
    \item Добавим ключи нового репозитория\\
    sudo apt-key adv --keyserver hkp://ha.pool.sks-keyservers.net:80 --recv-key 421C365BD9FF1F717815A3895523BAEEB01FA116
    \item Обновим список пакетов
    sudo apt update
    \item Установим все пакеты одной командой (это может занять несколько минут)\\
    sudo apt install ros-melodic-desktop-full
\end{enumerate}

Инсталяция законченна, осталось произвести последние настройки

Добавим все необходимые директории в пути для доступа\\
echo "source /opt/ros/melodic/setup.bash" >> \~/.bashrc\\
source \~/.bashrc

Чтобы изменения конфигурации начали работать, перезапустим терминал.

Запустим roscore чтобы убедиться что все работает.\\
roscore

Настройка завершена. У нас есть Линукс, есть ROS можем переходить к разбору задач.
Все ссылки есть в описании к ролику.

Ссылки

\url{https://www.virtualbox.org/wiki/Downloads}

\url{https://www.ubuntu.com/download/desktop}

\section{Управление роботом в симуляторе turtlesim}

\subimport{2nd_tour/aero_sys/task_01/}{statement}

\section{Подключение периферии через ROS}

\subimport{2nd_tour/aero_sys/task_02/}{statement}