\assignementTitle{}{30}

Задача проверяет базовые навыки способов подключения и управления периферией ROS на примере Arduino и сервопривода. От одной команды принимается только 2 решения (две попытки)!

Выполнение задания:

\begin{enumerate}
    \item Установить ROS (ros-melodic-desktop-full) в используя пакеты на дистрибутив Ubuntu. Инструкция \url{http://wiki.ros.org/melodic/Installation/Ubuntu}
    \item Запустить и настроить ROS, изучить базовые принципы (Книжка "Введение в ROS" \url{http://docs.voltbro.ru/starting-ros/})
    \item Установить и изучить документацию пакета rosserial \url{http://wiki.ros.org/rosserial}
    \item Сгенерировать клиентскую библиотеку для ардуино (пакет rosserial\_arduino)
    \item Для Arduino микроконтроллера написать программу которая:
    \begin{enumerate}
        \item Создаст Подписчика (subscriber) на топик /servo\_cmd c типом сообщения std\_msgs/Int32 работающего через Serial порт Arduino
        \item При получения сообщения, 0-180 переведет положения серво-машинки в установленный угол
        \item Создаст Издателя (publisher) для топика /servo\_cmd\_echo с типом сообщения std\_msgs/Int32
        \item При получения сообщения в /servo\_cmd издатель повторяет сообщение в топик servo\_cmd\_echo 
        \item При включении питания Arduino, серво машинка должны переключить положение в 90
    \end{enumerate}
    \item Сигнальный провод сервопривода подключен к 9 пину Arduino
\end{enumerate}

Алгоритм проверки

\begin{enumerate}
    \item Проверочный стенд использует Raspberry PI и Arduino Mega (ATmega 2560)
    \item Плата Arduino подключена к плате RaspberryPi через Serial и к персональному компьютеру через  USB (Serial) для загрузки прошивки.
    \item Оператор компилирует и загружает прошивку на Arduino, библиотека ros\_lib установлена на компьютере. Файл с программой для Arduino должен назватся латинскими символами соответствовать названию команды и через дефис номеру попытки (Пример:  iskra-1.ino ).
    \item Оператор подключает roscore и rosserial\_python
    \item Оператор из консоли отправляет сообщения в топик servo\_cmd и контролирует
\end{enumerate}

\begin{enumerate}
    \item[a)] Угол поворота сервопривода
    \item[б)] Публикацию ответного сообщения в топике servo\_cmd\_echo 
\end{enumerate}

\subsubsection*{Примечание}
При использовании Arudino Nano может не хватать оперативной памяти (RAM). В таком случае в Aruino IDE 
будут появляться сообщения, типа:

Глобальные переменные используют 1837 байт (89\%) динамической памяти, оставляя 211 байт для локальных 
переменных. Максимум: 2048 байт.

Недостаточно памяти, программа может работать нестабильно.

Можно сократить использование оперативной памяти уменьшив размер выделяемых 
буферов для передачи и приема сообщений. Для этого в самое начало программы следует поместить строку:

\#define \_\_AVR\_ATmega168\_\_ 1

Можно уменьшить количество занятой памяти еще сильнее, если вручную настроить количество publisher'ов и subscriber'ов, а также размеры буферов памяти, выделяемой для сообщений, например:

\#include <ros.h>
// ...
typedef ros::NodeHandle\_<ArduinoHardware, 3, 3, 100, 100> NodeHandle;
// ...
NodeHandle nh;