\assignementTitle{}{8}{}

Во втором задании вам предстоит проанализировать результаты эксперимента по определению микрофлоры почвы.

Для работы вам предлагается таблица, в которую внесены найденные организмы и их количество в исследуемом образце:

Таблица 1 – Количество микроорганизмов (общее микробное число) контрольном образце.

\putImgWOCaption{14cm}{1}

Высевы осуществлялись на питательные среды:

\begin{itemize}
    \item МПА (мясопептонный агар);
    \item среда Чапека;
    \item крахмал-казеиновая среда.
\end{itemize}

Известно, что в одной из проб на 7 сутки присутствует положительная реакция на цинк-йод-крахмал в кислой среде. На наличие каких микроорганизмов (из представленных в таблице) это указывает?

В ответ занесите количество клеток этих микроорганизмов на 21 день эксперимента, если известно, что динамика изменения их количества останется той же.