\assignementTitle{}{8}{}

В задании представлен текст об элементах круговорота азота в природе. Вставьте 
недостающие слова из предложенных.

Автотрофные, азот, азота, азотфиксация, амилаза, аммонификация, ассимиляция, водород, 
водорода, второй, выделяется, гетеротрофные, гниение, две, денитрификация, закись азота, 
кислород, кислорода, неорганические, нитрификация, окисление,  окись азота, органические, 
первой, поглощается, разложение, сахароза, $CO_2$, третьей, три

Атмосферный воздух на $78\%$ состоит из азота. Организмы и большинство зеленых растений 
нуждаются в различных химических формах азота. Благодаря процессам жизнедеятельности растений, 
водорослей и бактерий, осуществляется так называемый азотный цикл. Процессы, из которых 
складывается сложный круговорот азота - это ассимиляция, аммонификация, нитрификация, 
денитрификация, азотфиксация, разложение, выщелачивание, вынос, выпадение с осадками и т.д. 
Органические вещества, попадающие в почвы и воды подвергаются разложению, в ходе которого 
образуется аммиак. Под действием микроорганизмов проходит ряд дальнейших реакций и процессов. 
Процесс превращения аммиака в нитрат называется \makebox[2cm]{\hrulefill}$^1$. Он проходит в
\makebox[2cm]{\hrulefill}$^2$ стадии. Возбудителями \makebox[2cm]{\hrulefill}$^3$ стадии являются 
бактерии рода Nitrobacter. Они осуществляют превращение \makebox[2cm]{\hrulefill}$^4$ до
\makebox[2cm]{\hrulefill}$^5$. Возбудителями \makebox[2cm]{\hrulefill}$^6$ стадии являются бактерии 
рода Nitrosomonas, \linebreak Nitrosocystis, Nitrosolobus и Nitrosospira. Они окисляют \makebox[2cm]{\hrulefill}$^7$
до \makebox[2cm]{\hrulefill}$^8$. В ходе всех преобразований активно \makebox[2cm]{\hrulefill}$^9$ 
энергия. Процесс превращения нитратов в газообразные оксиды и молекулярный азот называется \makebox[2cm]{\hrulefill}$^{10}$
. Этот процесс происходят в среде, лишенной \makebox[2cm]{\hrulefill}$^{11}$. Т.е. процесс является 
анаэробным. В процессе преобразования исходного вещества (нитрат) в конечное 
(газообразный азот) последовательно появляются три промежуточных продукта: \makebox[2cm]{\hrulefill}$^{12}$ > 
\makebox[2cm]{\hrulefill}$^{13}$ > \makebox[2cm]{\hrulefill}$^{14}$. Нитрификация производится \makebox[2cm]{\hrulefill}$^{15}$ 
бактериями. Это означает, что они получают углерод, необходимый для роста, из \makebox[2cm]{\hrulefill}$^{16}$ веществ. 
Денитрифицирующие бактерии являются \makebox[2cm]{\hrulefill}$^{17}$, т.е. получают углерод из \makebox[2cm]{\hrulefill}$^{18}$ 
веществ, таких как \makebox[2cm]{\hrulefill}$^{19}$.

