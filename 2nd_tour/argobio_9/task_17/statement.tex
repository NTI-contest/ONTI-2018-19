\assignementTitle{}{6}{}

Одним из важнейших компонентов аквапонной системы является биофильтр, в котором происходит превращение вырабатываемого рыбами ядовитого аммиака в нитраты и нитриты, которые потом потребляются растениями в гидропонном модуле.

Для правильной работы системы необходимо соблюсти баланс рыбы, частоты кормления,  бактерий и растений.

В биофильтре бактерии закрепляются и размножаются на специальном наполнителе, в качестве которого может применяться гравий, специальная керамика и некоторые другие виды.

При выборе наполнителя наиболее важным параметром является площадь его поверхности 
(специфическая площадь поверхности, specific surface area, удельная площадь поверхности), 
которая измеряется в $m^2/m^3$.

Расположите предложенные варианты наполнителя в порядке возрастания этого параметра:

\begin{enumerate}
    \item Пластиковый наполнитель биофильтра (Bio Ball)
    \item Речной камень (диаметр - 25мм)
    \item Крупный песок (диаметр - 3мм)
    \item Гравий (диаметр - 15м) 
    \item Керамический пористый наполнитель
\end{enumerate}