\assignementTitle{}{4}

На графике изображена зависимость интенсивности фотосинтеза от интенсивности падающего на растение света.

\putImgWOCaption{11cm}{1}

Ось х – интенсивность света Вт$\cdot$м$^{-2}$

Ось у – интенсивность работы фотосинтетической системы растения

С чем может быть связана такая форма графика?

\begin{enumerate}
    \item Увеличение освещённости выше определённого значения приводит к нарушению процесса биосинтеза ферментов фотосинтетической системы и угнетению системы фотосинтеза
    \item Увеличение освещённости выше определённого значения приводит к фрагментарному некрозу листовой пластинки, что, в свою очередь, снижает интенсивность фотосинтеза
    \item Увеличение освещённости выше определённого значения приводит к снижению работы системы фотосинтеза по механизму обратной связи (много продуктов фотосинтеза угнетает сам фотосинтез из-за превышения процесса синтеза над процессом транспорта в организме растения)
    \item Увеличение освещённости выше определённого значения приводит к компенсаторным механизмам усиления пигментации поверхности листовой пластинки и к изменению проницаемости кутикулы, что, в свою очередь, снижает интенсивность работы фотосинтетических систем, сохраняя общий выход продуктов фотосинтеза оптимальным
\end{enumerate}