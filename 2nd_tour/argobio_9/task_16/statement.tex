\assignementTitle{}{8}

Используя промежуточные расчеты и условия из прошлой задачи, рассчитайте потребление азота (мг) одним корнеплодом в день.

Дополнительно известно, что:

\begin{itemize}
    \item Содержание белка в рыбьем корме составляет $48\%$
    \item Средний молекулярный вес аминокислотного остатка (в составе белка) составляет 110г/моль
    \item В среднем (при расчете всех возможных остатков), в аминокислотном остатке содержится 1,5 атома азота
    \item Считаем, что белок в корме - единственный источник азота в системе.
\end{itemize}

В процессе решения промежуточные значения округляйте до сотых.

В ходе решения массу рассчитывайте в граммах.

Ответ (в миллиграммах) округлите до десятых.