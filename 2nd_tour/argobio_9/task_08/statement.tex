\assignementTitle{}{4}{}

Что из перечисленных функций (процессов) осуществляется или относится к бактериям в аквапонной системе?  

\begin{enumerate}
    \item Не требуют освещения для жизнедеятельности
    \item Поглощают нитриты и нитраты, удаляя их из воды
    \item Не выводят нитраты из системы
    \item Требуют увеличения поверхности для эффективной работы фильтра
    \item Требуют кислород для дыхания
    \item Требуют освещения для жизнедеятельности
    \item Фильтрация 24 часа в сутки
    \item Преобразовывают аммиак в нитриты и, далее, в нитраты
    \item Предоставляют поверхность для колоний бактерий
    \item Выделяют кислород при фотосинтезе
\end{enumerate}

\explanationSection

Нитрифицирующие бактерии играют ключевую роль в азотном цикле и в аквапонных установках переводят выделяемый рыбами аммиак в нитраты и нитриты

Бактерии существуют в бескислородной среде (в природе- в почве) и работают 24 часа в сутки вне зависимости от освещения.  

Одним из ключевых параметров эффективности превращение аммиака в нитраты и нитриты является объем популяции бактерий, которые закрепляются в биофильтре аквапонной системы. В связи с этим, им необходима поверхность для закрепления. В аквариумистике используют специальный наполнитель.

\answerMath{1, 3, 4, 5, 7, 8.}