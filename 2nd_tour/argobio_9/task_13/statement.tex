\assignementTitle{}{6}{}

Аквапонная установка используется для проведения  исследование активности фотосинтеза в условиях 
изменения влажности, концентрации $CO_2$ в воздухе и интенсивности освещенности. Активность фотосинтеза проверяется измерением потребляемого $CO_2$.

В рамках одного эксперимента производится изменение только двух из трех упомянутых выше параметров.  

На рисунке приведено наложение трех графиков, полученных в ходе эксперимента. По оси X приведено время эксперимента. По оси Y - значения одного исследуемого (поглощение $CO_2$) и двух изменяемых параметров (влажность + интенсивность освещения ; или интенсивность освещения + изменение концентрации $CO_2$; или влажность + изменение концентрации $CO_2$). Каждый параметр приведен в своих единицах измерения.

Для проведения эксперимента в системе установлено следующее оборудование:
$\male \female$
\begin{itemize}
    \item Баллон с углекислым газом и система плавного повышения его концентрации в установке.
    \item Лампы освещения с плавной автоматической регулировкой интенсивности от 50 до 900 Вт$\cdot$м$^{-2}$
    \item Система увлажнения. Увлажнение осуществляется дискретно (не плавное, а резкое изменение параметра).
\end{itemize}

Продумайте, как связаны между собой углекислотное насыщение фотосинтеза, освещённость, влажность и концентрация углекислого газа в воздухе, и выберете верное утверждение описывающее поведение параметров на графике.

\putImgWOCaption{11cm}{1}

\begin{enumerate}
    \item Потребление углекислого газа не может изменяться после достижения предела углекислотного насыщения фотосинтеза, т.к., в состоянии насыщения субстратом, ферменты просто не успевают перерабатывать излишки веществ, поступающих в систему.
    \item Потребление углекислого газа изменяется в результате превышения порогового значения его концентрации для фотосинтетической системы 2. Как следствие, фотосинтетические системы получают возможность более эффективно потреблять воду.
    \item Ведущую роль в усилении фотосинтеза при увеличении концентрации углекислого газа играет влажность почвы. При постоянной влажности почвы обогащение газовой смеси углекислым газом не влияет на кривую поглощения углекислого газа в процессе фотосинтеза.
    \item Рост эффективности работы фотосинтетических систем в результате повышения влажности почвы возможен только при увеличении интенсивности освещённости, что не указано на представленном графике.
\end{enumerate}