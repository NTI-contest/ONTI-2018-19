\assignementTitle{}{4}{}

Вам предлагается дополнить существующую аквапонную систему, состоящую из гидропонного, аквакультурного и бактериального фильтрационного модуля дополнительной ёмкостью с:

\begin{itemize}
    \item Нитчатыми водорослями;
    \item Высшими водными растениями (роголистник, рдест кучерявый, телорез, водокрас лягушачий).
\end{itemize}

Для успешной интеграции вам надо просчитать оптимальную последовательность подключения исходя из потока и перехода воды из одного модуль в следующий.

Из предложенных вариантов выберите один или несколько подходящих вариантов последовательности подключения модулей. Отсчет начинается с Гидропонного блока. После последнего блока вода поступает вновь в первый-гидропонный.

Подключаемый модуль заполнен только высшими водными растениями

В ответах используются следующие индексы для обозначения модулей:

\begin{enumerate}
    \item Гидропонный
    \item Аквакультурный
    \item Растительноводный (дополнительный модуль)
    \item Бактериальный фильтрационный
\end{enumerate}

\begin{enumerate}
    \item[а.] 1-2-4-3
    \item[б.] 1-3-2-4
    \item[в.] 1-2-3-4
    \item[г.] 1-4-2-3
    \item[д.] 1-3-4-2
    \item[е.] 1-4-3-2
\end{enumerate}