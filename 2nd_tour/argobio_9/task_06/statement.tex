\assignementTitle{}{8}{}

Подберите оптимальный видовой состав дополнительного модуля, выбрав растения из списка.

Модуль будет устанавливаться в аквапонную систему, специализирующеюся на выращивании карпа Кои и салатной зелени

Температура воды при выращивании карпа не должна превышать $24^{\circ}C$

\begin{enumerate}
    \item Ряска
    \item Валиснерия гигантская
    \item Роголистник обыкновенный
    \item Риччия плавучая
    \item Нитчатка обыкновенная
    \item Элодея канадская
    \item Водяной мох
    \item Телорез
    \item Водокрас лягушачий
    \item Рдестр кучерявый
\end{enumerate}

\explanationSection

При ответе на вопрос нужно разобрать взаимодействие и оптимальные условия для жизни и развития: растений гидропонного модуля, аквакультуры (в данном случае это Карпы коя) и растений растительно-фильтрационного модуля.

Сопоставление будем проводить по параметрам азотного обмена и по дополнитлеьным питательным веществам из растительно-водного модуля для животных аквакультурного модуля.

Так же важно учитывать параметр температуры среды. В задании сказано, что температура среды не должны превышать 24 градусов цельсия. Так же обратим внимание на требования инженерной системы к нормальному функционированию.

Из представленного списка растений, можно выделить две группы: растения плавающие на поверхности воды (ряска, водокрас лягушачий, риччия плавучая и телорез) и прикреплённые к грунту (роголистник, элодея канадская, рдестр кучерявый, водяной мох, валиснерия гигантская и нитчатка обыкновенная). Последняя может свободно развиваться и как свободноплавающая, но склонна к образованию прикреплённых колоний.

По требованиям к температурному режиму мы можем сразу исключить риччию плавучую и валиснерию гигантскую. Карп может выдержат температуру в достаточно широком диапазоне (15-30 град.С), но рост салатной зелени не позволяет повышать температуру , удерживая её в в диапазоне 20-22 градуса С. Такая температура не является оптимальной для Риччии плавучей (22-26 градусов С) и погранична для Валиснерии (снижение ниже 20 град С не желательно). Можем отказаться от применения этих видов растений.

Водяной мох (или, иное название ручьевой мох) – холодноводное аквариумного растение. Выдерживает т-ру до 28 градусов цельсия, но с понижением температуры в зимний период до нуля. Без сезонных колебаний температуры  данное растение гибнет. Следовательно, в условиях аквапонной системы применение данного растения не эффективно. К тому же, в условиях УЗВ (установки замкнутого водоснабжения) Водяной мох, обитающий в природе в условиях быстрой проточной воды ручьёв и малых рек, будет заметно угнетение данного вида растений. В целом, использование данного вида растений для растительно-водного модуля не желательно.

Ряски. Применение рясок в данной системе имеет инженерные ограничения для применения. Мелкие растения, перемещающиеся с током воды по системе рециркуляции способны забивать патрубки и механические фильтры, приводя к аварийным состояниям и уменьшая срок эксплуатации механических фильтров. Применение данной группы растений не желательно. 

Водокрас лягушачий хорошо себя чувствует при температуре 20-28 градусов С. Однако, оптимальным является как - раз заявленный диапазон 20-24 градуса. 

Телорез обыкновенный – растение, в определённый период жизни (цветение) пребывающее на поверхности воды, но прикрепляющееся корнями к субстрату. Не прихотливое к температурному режиму и нормально произрастает в заданных температурных условиях.

Телорез и водокрас могут быть применены в данном модуле.

Рдестр кучерявый – растение распространено в не тропической зоне, достаточно лабильно по отношению к температуре, прекрасно себя чувствует в означенных выше условиях. Может расти как в прикреплённом, так и плавающем состоянии. Применение в системе возможно.

Элодея канадская не прихотливое растение с её скоростью роста может поспорить только рдестр. При разрастании и при отсутствии света является сильным конкурентом гидробионтов за кислород. К тому же, при повреждении стеблей Элодеи, в воду выделяется сок, который может быть токсичен для мальков рыб. Как следствие, от данного растения легче отказаться.

Роголистник – обладает способностью эффективно поглощать нитраты из системы, снижает жёсткость воды. Не прихотлив к условиям произрастания. Оптимален для данного модуля системы.

Нитчатка. В д. сл. возможно применение в системе, т.к. она является дополнительным источником питания для растительноядных рыб.

Т.о., в смешанном растительно-водном блоке аквапонной системы для данных условия остаются: рдерстр, роголистник, телорез, водокрас, нитчатка.

\answerMath{3, 5, 8, 9, 10.}