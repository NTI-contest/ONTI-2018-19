\assignementTitle{}{4}{}

Что из перечисленных функций (процессов) осуществляется или относится к растениям в аквапонной системе?  

\begin{enumerate}
    \item Предоставляют поверхность для колоний бактерий
    \item Поглощают нитриты и нитраты, удаляя их из воды
    \item Требуют кислород для дыхания
    \item Фильтрация 24 часа в сутки
    \item Не выводят нитраты из системы
    \item Преобразовывают аммиак в нитриты и, далее, в нитраты
    \item Выделяют кислород при фотосинтезе
    \item Не требуют освещения для жизнедеятельности
    \item Требуют увеличения поверхности для эффективной работы фильтра
    \item Требуют освещения для жизнедеятельности
\end{enumerate}

\explanationSection

Выделение кислорода в процессе жизнедеятельности растений дает жизнь всем организмам на нашей планете, которым требуется кислород. Процесс фотосинтеза, в результате которого происходит выделение кислорода может проходить в растениях только в условиях доступности света. В темное время суток растения, как и многие другие живые организмы потребляют кислород в процессе дыхания. 

Еще одной отличительной особенностью растений является возможность синтеза аминокислот из неорганических соединений. Для этого им необходим азот, который они получают при поглощении нитратов и нитритов с помощью азот-фиксирующих бактерий.

\answerMath{1, 2, 3, 7, 10.}