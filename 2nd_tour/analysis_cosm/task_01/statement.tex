\assignementTitle{Базовые навыки работы с ГИС. Геообработка и вычисление площадей}{3}

Для выполнения задач второго этапа Олимпиады вам понадобятся базовые навыки работы в ГИС. Если вы используете QGIS, мы рекомендуем вам сперва изучить следующие пособия на русском языке:

\begin{itemize}
    \item «Плавное введение в ГИС»\\ \url{http://gis-lab.info/qa/gentle-intro-gis.html};
    \item «Документация на NextGIS QGIS» \\ \url{http://docs.nextgis.ru/_downloads/NextGISQGIS.pdf}.
\end{itemize}

Малонарушенные лесные территории (\url{http://borealforestplatform.org/ru/intact_forest_landscapes/}) (МЛТ, Intact forest landscapes; \url{https://en.wikipedia.org/wiki/Intact_forest_landscape}) - это крупные природные ландшафты в 
пределах лесной зоны, в наименьшей степени нарушенные хозяйственной деятельностью человека. С 2000 года ученые и экологические организации нескольких стран ведут картографирование этих территорий с помощью космических снимков по всему миру. По их данным, площадь МЛТ практически повсеместно сокращается по тем или иным причинам, связанным почти всегда с деятельностью человека. 

С помощью QGIS (или другого используемого вами программного обеспечения) определите, какой процент МЛТ острова Калимантан охраняется в пределах различных особо охраняемых природных территорий (ООПТ) - национальных парков, природных резерватов и пр.

Сгрузите набор векторных пространственных данных с границами МЛТ по состоянию на 2016 год с сайта 
Intactforests.org \url{http://www.intactforests.org/data.ifl.html}.

В качестве источника исходных векторных данных по границам ООПТ используйте Всемирную базу 
данных по особо охраняемым территориям (World Database on Protected Areas - WDPA; \url{https://www.unep-wcmc.org/resources-and-data/wdpa}) 
от Всемирного центра природоохранного мониторинга (World Conservation Monitoring Centre; \url{https://www.unep-wcmc.org/}) Программы ООН по окружающей среде. Эти данные не всегда точные и актуальные (так, по территории России они явно неполны), но, в среднем по миру, это лучший из доступных источников информации. При геообработке и расчетах используйте только границы существующих (то есть уже созданных, а не планируемых) ООПТ. Набор векторных данных на весь мир очень велик по объёму - мы рекомендуем вам сгрузить данные только на нужную вам территорию.

Для получения контуров МЛТ в пределах ООПТ используйте в QGIS  инструменты геообработки (geoprocessing) векторных данных, такие как "Пересечение", "Обрезка" или "Объединение". Для вычисления площадей добавьте числовые поля в атрибутивные таблицы соответствующих слоев и используйте Калькулятор полей.

Помните, что результаты вычисления площадей по картам зависят от проекции, в которой производятся вычисления. Для данного задания используйте проекцию UTM (Universal Transverse Mercator), зона 50N на WGS84 (EPSG:32650)

Ответ представьте в процентах с округлением до десятых долей процента.