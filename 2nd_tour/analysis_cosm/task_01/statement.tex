\assignementTitle{Базовые навыки работы с ГИС. Геообработка и вычисление площадей}{3}{}

Для выполнения задач второго этапа Олимпиады вам понадобятся базовые навыки работы в ГИС. Если вы используете QGIS, мы рекомендуем вам сперва изучить следующие пособия на русском языке:

\begin{itemize}
    \item «Плавное введение в ГИС»\\ \url{http://gis-lab.info/qa/gentle-intro-gis.html};
    \item «Документация на NextGIS QGIS» \\ \url{http://docs.nextgis.ru/_downloads/NextGISQGIS.pdf}.
\end{itemize}

Малонарушенные лесные территории (\url{http://borealforestplatform.org/ru/intact_forest_landscapes/}) (МЛТ, Intact forest landscapes; \url{https://en.wikipedia.org/wiki/Intact_forest_landscape}) - это крупные природные ландшафты в пределах лесной зоны, в наименьшей степени нарушенные хозяйственной деятельностью человека. С 2000 года ученые и экологические организации нескольких стран ведут картографирование этих территорий с помощью космических снимков по всему миру. По их данным, площадь МЛТ практически повсеместно сокращается по тем или иным причинам, связанным почти всегда с деятельностью человека. 

С помощью QGIS (или другого используемого вами программного обеспечения) определите, какой процент МЛТ острова Калимантан охраняется в пределах различных особо охраняемых природных территорий (ООПТ) - национальных парков, природных резерватов и пр.

Сгрузите набор векторных пространственных данных с границами МЛТ по состоянию на 2016 год с сайта 
Intactforests.org \url{http://www.intactforests.org/data.ifl.html}.

В качестве источника исходных векторных данных по границам ООПТ используйте Всемирную базу 
данных по особо охраняемым территориям (World Database on Protected Areas - WDPA; \url{https://www.unep-wcmc.org/resources-and-data/wdpa}) 
от Всемирного центра природоохранного мониторинга (World Conservation Monitoring Centre; \url{https://www.unep-wcmc.org/}) Программы ООН по окружающей среде. Эти данные не всегда точные и актуальные (так, по территории России они явно неполны), но, в среднем по миру, это лучший из доступных источников информации. При геообработке и расчетах используйте только границы существующих (то есть уже созданных, а не планируемых) ООПТ. Набор векторных данных на весь мир очень велик по объёму - мы рекомендуем вам сгрузить данные только на нужную вам территорию.

Для получения контуров МЛТ в пределах ООПТ используйте в QGIS  инструменты геообработки (geoprocessing) векторных данных, такие как "Пересечение"{}, "Обрезка"\ или "Объединение". Для вычисления площадей добавьте числовые поля в атрибутивные таблицы соответствующих слоев и используйте Калькулятор полей.

Помните, что результаты вычисления площадей по картам зависят от проекции, в которой производятся вычисления. Для данного задания используйте проекцию UTM (Universal Transverse Mercator), зона 50N на WGS84 (EPSG:32650).

Ответ представьте в процентах с округлением до десятых долей процента.

\explanationSection

Для решения этой задачи необходимо скачать набор пространственных данных с границами малонарушенных лесов по приведённой в условиях ссылке: \url{http://www.intactforests.org/data.ifl.html}.

В соответствии с требованиями условия, необходимо выбрать и скачать набор данных за 2016 год в формате ESRI SHAPE: \url{http://www.intactforests.org/shp/IFL_2016.zip}.

Скачанный архив необходимо разархивировать любой программой-архиватором. Например, с помощью программы 7-Zip (\url{https://www.7-zip.org/}), которая распространяется абсолютно бесплатно. Полученный набор векторных пространственных данных (он называется ifl2016) загрузить в QGIS или другую ГИС-программу, которой Вы ползуетесь.

Для удобства работы лучше сразу выбрать из набора данных на весь мир контуры малонарушенных лесных территорий только на остров Калимантан. Самый простой путь сделать это – визуально выделить нужные контуры и сохранить в отдельный набор данных. Для визуального выделения необходимо добавить в ГИС-проект любую базовую карту из внешних источников, например, карту Google Maps или Яндекс Карты. В QGIS для добавления карт-подложек из внешних источников можно воспользоваться модулями QuickMapServices или OpenLayers Plugin.

Используя подгруженную базовую карту, найти на ней остров Калимантан (Борнео) и, используя инструменты выделения, пометить все объекты (контуры массивов малонарушенных лесов), попадающие на его территорию. Выделенные объекты сохранить как отдельный набор данных используя функцию «Сохранить как…» (в QGIS доступна в контекстном меню каждого слоя).

При этом лучше сразу сохранить этот набор данных в проекции UTM (Universal Transverse Mercator), зона 50N на WGS84 (EPSG:32650), в котором будут производиться расчеты площадей. Исходный набор данных находится в так называемой географической проекции (в десятичных градусах) в системе координат WGS 84 (EPSG:4326). Расчет площади в данной системе координат даст ошибочное значение. Выбрать систему координат сохраняемого набора данных можно в окне диалога функции «Сохранить как...».

Другой набор исходных данных необходимо скачать с указанного в условии задачи сайта используйте Всемирную базу данных по особо охраняемым территориям (World Database on Protected Areas – WDPA) от Всемирного центра природоохранного мониторинга (World Conservation Monitoring Centre): \url{https://www.unep-wcmc.org/resources-and-data/wdpa}.

Несмотря на то, что большая часть информации на сайте на английском языке, с указанной страницы также есть ссылка на документ «Всемирная база данных по охраняемым природным территориям. Руководство пользователя 1.5» на русском языке. В руководстве описывается структура базы данных и способы работы с ней.

С той же страницы по нажатию кнопки «Downlaod dataset» («Сгрузить набор данных») пользователь попадает на картографический сервис «Protected Planet»: \url{https://protectedplanet.net/}. В верхнем левом углу карты, расположенной на странице находится окошко поиска («Search a protected area»), с помощью которого можно выбрать из общей базы данных только на определённые территории. В частности, с помощью него можно выбрать наборы данных по особо охраняемым природным территориям (ООПТ) на страны, полностью или частично расположенные на территории острова Калимантан. Калимантан – единственный в мире остров, на котором расположены территории трёх стран: Индонезии, Малайзии и Брунея. Необходимо найти и загрузить векторные наборы данных с границами ООПТ на все три страны, поскольку малонарушенные лесные территории расположены на Калимантане в границах всех трёх стран.

Загрузка набора данных производится с помощью зелёной надписи «Download this dataset» с соответствующим значком, которая появляется при выборе соответствующей страны из списка результатов поиска. После нажатия также надо выбрать подходящий для обработки в настольной ГИС формат данных, в данном случае – .SHP. Не рекомендуется загружать набор данных по ООПТ сразу на весь мир, не выбрав данных на отдельные страны. Полученный набор данных будет огромен и потребует для своей обработки существенных компьютерных мощностей. Загруженные данные также потребуют распаковки с помощью архиватора.

Каждый сгруженный архив содержит по два набора векторных данных – точечный (для охраняемых территорий небольшого размера, не выражающихся в масштабе карты) и полигональный (названия файлов заканчиваются словом «polygons»), то есть содержащий замкнутые контуры границ соответствующих территорий. Для дальнейшей работы и расчётов площадей имеют значения только полигональные слои.

Для удобства работы данные по ООПТ трёх стран лучше объединить в один набор данных. Сделать это в QGIS проще всего с помощью инструментов из меню «Вектор» – «Объединение shape-файлов...» (раздел «Управление данными») или «Объединить векторные слои» (из раздела «Data Management Tools»). Оба инструмента реализуют аналог команды «Append» из ArcInfo GIS и объединяют пространственные базы данных с одинаковой структурой без изменения их геометрии.

После объединения полученный набор данных также необходимо перевести в проекцию UTM (Universal Transverse Mercator), зона 50N в системе координат WGS84 (EPSG:32650), как это было сделано для набора данных по малонарушенным лесам. Без этой операции не получится ни корректно рассчитать площади, ни воспользоваться инструментами геообработки. Рекомендуется, как и в случае с данными по малонарушенным лесам, совместить перевод в другую проекцию с выборкой данных только на территорию острова Калимантан из всей базы данных по трём странам. Это облегчит работу на недостаточно мощных компьютерах.

По условию задачи, при вычислениях необходимо использовать границы только существующих (то есть уже созданных, а не планируемых) ООПТ. Соответствующая информация хранится в поле «STATUS» атрибутивной таблицы полученного набора векторных данных, о чем можно прочесть в русскоязычном «Руководстве пользователя» (стр. 56 и 57, Приложение 1). Статусу существующих ООПТ для Калимантана соответствуют только значения «Designated» и «Inscribed».

С помощью инструмента QGIS «Выбрать по выражению...» необходимо пометить в полученном наборе данных все объекты, имеющие соответствующие значения в поле «STATUS» атрибутивной таблицы. Выбранные объекты лучше ещё раз сохранить как отдельный набор данных. (Хотя можно производить операции геообработки и только с помеченными объектами из общего набора данных.)

Для расчета площади малонарушенных лесов Калимантана, сохраняемых в пределах ООПТ, нам необходимо получит границы частей этих лесных массивов, находящихся внутри существующих («Designated», «Inscribed») природных резерватов трёх стран. Выделить эти границы можно с помощью так называемых инструментов геообработки, которые имеются в составе любой современной ГИС-системы. В частности, в QGIS эти инструменты сосредоточены в меню «Вектор» в разделах «Геообработка»/«Geoprocessing Tools».

Исходными наборами данных для данной операции являются векторные наборы данных с границами малонарушенных лесов Калимантана и с границами ООПТ на этот остров, отобранных по признаку их официального создания (значения \linebreak «Designated», «Inscribed» в поле «STATUS»). Важно, что оба набора данных должны находиться в одной и той же проекции (системе коорлинат). Лучше всего воспользоваться UTM (Universal Transverse Mercator), зона 50N WGS84 (EPSG:32650), в которой необходимо и рассчитывать площади. Однако, можно произвести данную операцию и между двумя слоями в десятичных градусах (в «географической» проекции), а в проекцию Universal Transverse Mercator перевести позднее уже только полученный результат.

Для получения границ малонарушенных лесов в пределах существующих ООПТ можно воспользоваться инструментами «Пересечение» или «Обрезка» из меню «Вектор». Результат геообработки надо сохранить как отдельный набор данных. Его необходимо перевести в проекцию Universal Transverse Mercator, если это не было сделано ранее для исходных слоёв.

Наконец, необходимо произвести расчёт суммарной площади всех полигонов (замкнутых контуров), входящих в состав двух наборов данных: (1) малонарушенных лесов острова Калимантан и (2) описанного выше результата пересечения этих границ малонарушенных лесов с объединёнными по трём странам границами существующих ООПТ. Для расчёта площадей в QGIS необходимо воспользоваться инструментом «Калькулятор полей». Результаты расчётов по каждому полигону будут записываться в соответствующее поле (колонку) атрибутивной таблицы, которое можно создать непосредственно в «Калькуляторе полей». (Но можно такое поле создать и заранее.) Для вычисления площади в «Калькуляторе полей» необходимо ввести в поле для выражений функцию \$area (набрать с клавиатуры или вставить двойным щелчком из группы «Геометрия»). При нажатии кнопки «ОК» «Калькулятор полей» вычислит площади для каждого полигона соответствующего набора данных. Следите только, чтобы в используемом слое наборе данных не было помеченных объектов (полигонов) – в этом случае расчеты будут произведены только для них.

Далее необходимо суммировать площади всех полигонов в этих двух наборах данных. Для этого в QGIS можно воспользоваться инструментом «Показать сводку статистики» или просто открыть атрибутивную таблицу соответствующего набора векторных данных (файл с расширением *.dbf) с помощью программы обработки электронных таблиц – например, MS Excel или аналогичной. В открытой таблице суммировать все значения в соответствующей колонке, в которую Вы записали значения площади полигонов с помощью «Калькулятора полей».

Полученную площадь по частям малонарушенных лесов в составе существующих ООПТ разделить на общую площадь малонарушенных лесов острова. Ответ представить в процентах до десятых.

Наши собственные расчеты дают значения площади 2 251 тыс. гектаров для малонарушенных лесов, охраняемых в составе существующих ООПТ, и 9 492 тыс. гектаров для всех малонарушенных лесов острова Калимантан. Использованы наборы данных по границам малонарушенных лесных территорий за 2016 года и по границам ООПТ версии за декабрь 2018 года.

\answerMath{29.0\%. Как правильный принимался ответ с погрешностью $\pm$2.0\%.}
