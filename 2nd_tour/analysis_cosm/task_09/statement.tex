\assignementTitle{Вычисление площадей. Концессии на разведку и добычу полезных ископаемых в Бразилии}{1}{}

Посчитайте суммарную площадь всех концессий, переданных/передаваемых компаниям для разведки (Exploration) и/или добычи (Exploitation) полезных ископаемых в Бразилии. Определите именно суммарную площадь выданных лицензий: если территории двух концессий перекрываются (например, лицензия выдаётся на добычу разных видов полезных ископаемых на одной территории), посчитайте такую площадь дважды.

Данные о границах всех участков, выделенных / планируемых для добычи или разведки полезных ископаемых 
(включая уже переданные компаниям), по Бразилии (Brazil mining concessions) загрузите с портала Всемирной 
лесной вахты (Global Forest Watch; \url{https://www.globalforestwatch.org/}).

Вычисления площадей производите в равновеликой конической проекции Альберса для Южной Америки (South America Albers Equal Area Conic) с центральным меридианом 60 градусов западной долготы и главными параллелями 5 и 42 градуса южной широты в Южно-американской системе координат 1969 года (EPSG:102033). 

Ответ представьте в гектарах с округлением до десятков.

\explanationSection

Задачи 1-4 данного блока во многом являются аналогом Задачи А первого блока. Здесь также, в конечном итоге, требуется найти площадь пересекающихся частей двух наборов векторных пространственных данных. Однако, в данном случае решение задачи осложнено низким качеством исходных данных, в которых требуется исправить ошибки топологии, препятствующие выполнению некоторых операций. Вся последовательность действий разбита на этапы и представлена в виде четырех отдельных задач.

Задача 1 требует выполнения уже встречавшейся участниками операции – вычисления площадей всех, входящих в состав векторного набора данных замкнутых контуров (полигонов). Предварительно участники должны были найти на портале Всемирной лесной вахты (\url{https://www.globalforestwatch.org/}) необходимый набор данных (легко находится по ключевым словам названия, приведённого в условии задачи), скачать его к себе на компьютер и распаковать.

Как и в Задаче А первого блока, из скачанного набора данных необходимо было выбрать часть объектов (полигонов) по информации в атрибутивной таблице, воспользовавшись для этого инструментом QGIS «Выбрать по выражению...» (или аналогичным в другой ГИС-системе). Выбранные полигоны лучше всего сохранить в отдельный набор векторных данных.

Также для корректного подсчета площадей полученный набор данных было необходимо перевести в указанную в условиях задачи географическую проекцию.

\answerMath{96 550 000. В качестве правильного принимался ответ с погрешностью $\pm$700 000.}