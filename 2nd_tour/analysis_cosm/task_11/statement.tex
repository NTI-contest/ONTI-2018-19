\assignementTitle{Геообработка и пространственный анализ. Добыча полезных ископаемых в охраняемых природных территориях Амазонии}{3}{}

Определите площадь, переданную/передаваемую компаниям для разведки \linebreak (Exploration) 
и/или добычи (Exploitation) полезных ископаемых, в пределах Бразильской Амазонии, 
которые попадают в границы существующих (Designated, \linebreak Inscribed) или планируемых (Proposed) 
особо охраняемых природных территорий (ООПТ; \url{https://en.wikipedia.org/wiki/Protected_areas_of_Brazil}) или 
в границы индейских резерваций (\url{https://en.wikipedia.org/wiki/Indigenous_territo}\linebreak\url{ry_(Brazil)}).

Данные о границах участков, переданных для добычи или разведки полезных ископаемых, по Бразилии 
(Brazil mining concessions) загрузите с портала Всемирной лесной вахты (Global Forest Watch; \url{https://www.globalforestwatch.org/}).

В качестве источника исходных векторных данных по границам ООПТ и границам индейских резерваций используйте 
Всемирную базу данных по особо охраняемым территориям (World Database on Protected Areas - WDPA; \url{https://www.unep-wcmc.org/resources-and-data/wdpa}) 
от Всемирного центра природоохранного мониторинга (World Conservation Monitoring Centre; \url{https://www.unep-wcmc.org/}) Программы ООН по окружающей среде. Набор векторных данных на весь мир очень велик по объёму - мы рекомендуем вам сгрузить данные только на нужную вам территорию.

Помните, что концессионные участки могут перекрываться между собой. Аналогично и ООПТ разного статуса также могут перекрываться. Одну и ту же площадь не надо считать дважды.

Используйте границы Амазонии, как они определены на карте сухопутных биомов Бразилии (Brazil biomes), 
созданной в результате сотрудничества Министерства окружающей среды Бразилии (MMA) и Бразильского института 
географии и статистики (IBGE). Соответствующий набор пространственных данных также доступен через портал 
Всемирной лесной вахты (\url{https://www.globalforestwatch.org/}).

Используйте инструменты геообработки (geoprocessing) векторных данных в \linebreak QGIS или в другом используемом вами программном обеспечении ГИС. Вычисления площадей производите в равновеликой конической проекции Альберса для Южной Америки (South America Albers Equal Area Conic) с центральным меридианом 60 градусов западной долготы и главными параллелями 5 и 42 градуса южной широты в южно-американской системе координат 1969 года (EPSG:102033).

Ответ представьте в гектарах с округлением до десятков.

\explanationSection

В данной задаче необходимо использовать полученный при решении предыдущей Задачи Б набор векторных данных с границами выданных концессий на добычу полезных ископаемых. В нём уже исправлены топологические ошибки, и все полигоны выданных концессий объединены в один или несколько непересекающихся контуров.

Для решения данной задачи этот набор данных необходимо дважды последовательно обрезать двумя другими наборами векторных данных, чтобы получить финальный набор данных, по которому произвести расчет площадей.

Один из этих наборов данных – границы Амазонии, которые также скачиваются с уже известного участникам портала Всемирной лесной вахты (\url{https://www.globalforestwatch.org/}).

Другой набор векторных данных – уже знакомыми участникам по первому блоку задач границами особо охраняемых природных территорий (ООПТ). (В случае Бразилии он содержит ещё и границы индейских резерваций.) При скачивании этих данных так же, как и в случае с Калимантаном, нужно ограничиться только границами ООПТ на Бразилию.

Набор векторных данных с границами ООПТ, как и границы лицензий на добычу полезных ископаемых, тоже содержит перекрывающиеся полигоны и, вероятно, топологические ошибки. Однако, поскольку они используется только для обрезки другого слоя, устранять в нём перекрывающиеся полигоны и топологические ошибки не нужно. Если мы производим обрезку одного слоя границами другого, достаточно проверить топологию только в обрезаемом слое.

Разумеется, чтобы обрезать в QGIS границы выданных лицензий на добычу полезных ископаемых указанными выше границами Амазонии и ООПТ, все три слоя должны находиться в одной и той же системе координат. В данном случае лучше использовать систему координат, указанную в условиях задачи: исходные данные по лицензиям уже в ней находятся, и площади, по условиям задачи, также требуется рассчитать именно в ней.

После обрезки суммарная площадь полигонов полученного слоя рассчитывается так же, как это делалось в предыдущих задачах.

\answerMath{6 850 000. В качестве правильного принимался ответ с погрешностью $\pm$70 000.}