\assignementTitle{Геообработка и пространственный анализ. Добыча полезных ископаемых в охраняемых природных территориях Амазонии}{3}{}

Определите площадь, переданную/передаваемую компаниям для разведки \linebreak (Exploration) 
и/или добычи (Exploitation) полезных ископаемых, в пределах Бразильской Амазонии, 
которые попадают в границы существующих (Designated, Inscribed) или планируемых (Proposed) 
особо охраняемых природных территорий (ООПТ; \url{https://en.wikipedia.org/wiki/Protected_areas_of_Brazil}) или 
в границы индейских резерваций (\url{https://en.wikipedia.org/wiki/Indigenous_territory_(Brazil)}).

Данные о границах участков, переданных для добычи или разведки полезных ископаемых, по Бразилии 
(Brazil mining concessions) загрузите с портала Всемирной лесной вахты (Global Forest Watch; \url{https://www.globalforestwatch.org/}).

В качестве источника исходных векторных данных по границам ООПТ и границам индейских резерваций используйте 
Всемирную базу данных по особо охраняемым территориям (World Database on Protected Areas - WDPA; \url{https://www.unep-wcmc.org/resources-and-data/wdpa}) 
от Всемирного центра природоохранного мониторинга (World Conservation Monitoring Centre; \url{https://www.unep-wcmc.org/}) Программы ООН по окружающей среде. Набор векторных данных на весь мир очень велик по объёму - мы рекомендуем вам сгрузить данные только на нужную вам территорию.

Помните, что концессионные участки могут перекрываться между собой. Аналогично и ООПТ разного статуса также могут перекрываться. Одну и ту же площадь не надо считать дважды.

Используйте границы Амазонии, как они определены на карте сухопутных биомов Бразилии (Brazil biomes), 
созданной в результате сотрудничества Министерства окружающей среды Бразилии (MMA) и Бразильского института 
географии и статистики (IBGE). Соответствующий набор пространственных данных также доступен через портал 
Всемирной лесной вахты (\url{https://www.globalforestwatch.org/}).

Используйте инструменты геообработки (geoprocessing) векторных данных в QGIS или в другом используемом вами программном обеспечении ГИС. Вычисления площадей производите в равновеликой конической проекции Альберса для Южной Америки (South America Albers Equal Area Conic) с центральным меридианом 60 градусов западной долготы и главными параллелями 5 и 42 градуса южной широты в южно-американской системе координат 1969 года (EPSG:102033).

Ответ представьте в гектарах с округлением до десятков.