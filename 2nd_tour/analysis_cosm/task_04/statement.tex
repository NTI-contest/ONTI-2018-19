\assignementTitle{Автоматизация отбора и загрузки снимков \\ Landsat}{3}{}

Если вам необходимо скачивать большое число снимков Landsat для значительной территории или за 
длительный временной период, отбирать их вручную через интерфейс портала EarthExplorer (\url{https://earthexplorer.usgs.gov/}) не всегда удобно. Напишите программу или скрипт, который(ая) бы автоматически отбирал(а) и скачивал(а) бы все доступные снимки Landsat 8 Collection 1 Level-1 для прямоугольной области с заданными координатами за указанный период времени с заданной облачностью.

Для выполнения данной задачи нужно воспользоваться API сервиса. Документацию можно найти по ссылке:\\
\url{https://earthexplorer.usgs.gov/inventory/documentation/json-api}

Для обращения к основным методам API необходим ключ (API key), который выдаётся сервером после авторизации через метод Login.

\inputfmtSection

Географические координаты области поиска:\\
60 05’ 53” N 105 25’ 49” E\\
56 40’ 30” N 110 10’ 00” E\\

Временной интервал поиска:\\
с 01.06.2017 по 30.09.2018

Максимально допустимая степень облачности:

не более 10\%

\outputfmtSection

Представляет собой файл формата JSON, выданный сервером в результате запроса, в котором приводятся данные о
том, сколько снимков было найдено, сопроводительная информация по каждому
снимку, в т.ч. координаты, временные рамки, ссылки на метаданные, скачивание
снимков и пр., а также информация о версии API и времени выполнении запроса. 

В качестве ответа загрузите полученный файл JSON с помощью кнопки ниже.

\solutionSection

Написанный участниками скрипт должен генерировать конечный запрос на поиск данных следующего вида:

\begin{minted}[fontsize=\footnotesize, linenos]{js}
https://earthexplorer.usgs.gov/inventory/json/v/1.4.0/search?jsonRequest={
  "apiKey":"3d3945c6720e4e88ab4adfd3c8578c4b",
  "datasetName":"LANDSAT_8_C1",
  "spatialFilter": {
    "filterType": "mbr",
    "lowerLeft": {
      "latitude": 56.675,
      "longitude": 105.4303
        },
    "upperRight": {
      "latitude": 60.0981,
      "longitude": 110.1667
        }
    },
  "temporalFilter": {
    "startDate": "2017-06-01",
    "endDate": "2018-09-30"
  },
  "maxCloudCover":10, "maxResults":50000
}
\end{minted}

В результате посылки такого запроса будет получен ответ в формате jsonсодержащий корректный набор данных.

Значение apiKey будет индивидуальным для каждого пользователя. Скрипт участника также должен получать его значение через метод Login.

\answerMath{Пример содержания файл формата JSON, полученного в результате правильно составленного запроса приведён ниже. Если входные данные одинаковые, то и структура правльного запроса одинаковая. Одинаковым будет и полученный в результате запроса файл. Единственное допустимое отличие – дата и время выполнения запроса. При проверке данная часть содержимого введённого файла игнорировалась.}

\inputminted[fontsize=\footnotesize]{console}{2nd_tour/analysis_cosm/task_04/source.json}