\assignementTitle{Автоматизация отбора и загрузки снимков Landsat.}{3}

Если вам необходимо скачивать большое число снимков Landsat для значительной территории или за 
длительный временной период, отбирать их вручную через интерфейс портала EarthExplorer (\url{https://earthexplorer.usgs.gov/}) не всегда удобно. Напишите программу или скрипт, который(ая) бы автоматически отбирал(а) и скачивал(а) бы все доступные снимки Landsat 8 Collection 1 Level-1 для прямоугольной области с заданными координатами за указанный период времени с заданной облачностью.

Для выполнения данной задачи нужно воспользоваться API сервиса. Документацию можно найти по ссылке:\\
\url{https://earthexplorer.usgs.gov/inventory/documentation/json-api}

Для обращения к основным методам API необходим ключ (API key), который выдаётся сервером после авторизации через метод Login.

\inputfmtSection

Географические координаты области поиска:

60 05’ 53” N 105 25’ 49” E

56 40’ 30” N 110 10’ 00” E

Временной интервал поиска:

с 01.06.2017 по 30.09.2018

Максимально допустимая степень облачности:

не более 10\%

\outputfmtSection

Представляет собой файл формата JSON, выданный сервером в результате запроса, в котором приводятся данные о
том, сколько снимков было найдено, сопроводительная информация по каждому
снимку, в т.ч. координаты, временные рамки, ссылки на метаданные, скачивание
снимков и пр., а также информация о версии API и времени выполнении запроса. 

В качестве ответа загрузите полученный файл JSON с помощью кнопки ниже.