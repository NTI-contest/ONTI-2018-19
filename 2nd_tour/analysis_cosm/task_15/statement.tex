\assignementTitle{Картографирование влажных тропических лесов.}{3}

С помощью портала EarthExplorer (\url{https://earthexplorer.usgs.gov/}) найдите доступные снимки со спутника 
Landsat 8 для точки с географическими координатами 18$^{\circ}$56'21'' южной широты и 48$^{\circ}$25'47'' восточной долготы за октябрь 2018 года.

С помощью DTclassifier \url{http://gis-lab.info/qa/dtclassifier.html} (или с помощью другого доступного Вам алгоритма/инструмента) выделите все покрытые лесом территории в пределах данной сцены Landsat 8, которые попадают при этом также в границы существующих или планируемых особо охраняемых природных территорий (ООПТ) всех категорий. Считать лесом любую древесную растительность с сомкнутостью полога 10\% и, предположительно, высотой хотя бы 5 метров.

В качестве источника векторных данных по границам ООПТ используйте Всемирную базу 
данных по особо охраняемым территориям (World Database on Protected Areas - WDPA; \url{https://www.unep-wcmc.org/resources-and-data/wdpa}) от 
Всемирного центра природоохранного мониторинга (World Conservation Monitoring Centre; \url{https://www.unep-wcmc.org/}) Программы ООН по окружающей среде. Набор векторных данных на весь мир очень велик по объёму - мы рекомендуем вам сгрузить данные только на нужную вам территорию. 

Измерьте общую площадь всех покрытых лесом территорий в границах ООПТ в пределах данной сцены Landsat 8, переведите в гектары, округлите до десятков и внесите в поле ниже. Для расчета площади используйте ту же проекцию, в которой находятся скачанные Вами сцены(а) Landsat 8.