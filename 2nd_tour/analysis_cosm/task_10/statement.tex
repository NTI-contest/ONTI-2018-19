\assignementTitle{Вычисление общей площади перекрывающихся полигонов. Концессии на разведку и добычу полезных ископаемых в Бразилии}{3}{}

Общие суммарные цифры площадей, переданных в концессию для разведки и добычи полезных ископаемых, хорошо смотрятся в правительственных отчётах. Однако, в реальности - для оценки, скажем, экологического ущерба от такой деятельности - часто важнее знать общую реальную площадь, которая будет затронута геологической деятельностью. Для этого надо исключить двойной учёт перекрывающихся участков концессий.

Посчитайте общую площадь, попадающую в пределы концессионных участков, переданных/передаваемых компаниям 
для разведки (Exploration) и/или добычи \linebreak (Exploitation) полезных ископаемых в Бразилии, исключив повторный подсчёт площадей перекрывающихся территорий.

Вы можете использовать инструмент "Объединение по признаку" ("Dissolve" - "растворение границ"), чтобы слить вместе все полигоны (контуры) с определёнными значениями полей в атрибутивной таблице или вообще все полигоны вместе (в том числе перекрывающиеся). Вы можете также объединять пересекающиеся полигоны в режиме редактирования векторного слоя, используя инструмент "Объединение объектов" (Merge).

Однако, автоматическая обработка всех полигонов в таком объёмном наборе данных как границы концессионных участков по всей Бразилии может не сработать (особенно при использовании свободного программного обеспечения, такого как QGIS), если исходный набор данных содержит топологические ошибки. А в большой официальной базе данных, пополняемой разными людьми, топологические ошибки - обычное дело. Будьте готовы потратить существенное время на объединение полигонов по частям, в "ручном" режиме, а также на поиск топологических ошибок и их исправление с помощью, например, инструмента "Редактирование узлов" и других средств.

Только убедившись, что в Вашем слое теперь отсутствуют перекрывающиеся полигоны, приступайте к вычислениям площадей. Как и в предыдущей задаче, вычисления площадей производите в равновеликой конической проекции Альберса для Южной Америки (South America Albers Equal Area Conic) с центральным меридианом 60 градусов западной долготы и главными параллелями 5 и 42 градуса южной широты в Южно-американской системе координат 1969 года (EPSG:102033).

Ответ представьте в гектарах с округлением до десятков.
