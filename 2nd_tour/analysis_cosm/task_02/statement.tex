\assignementTitle{Поиск и загрузка космических снимков Landsat}{1}{}

Снимки земли с американских спутников серии Landsat сегодня бесплатно доступны для всех желающих. 
Воспользуйтесь порталом EarthExplorer \linebreak (\url{https://earthexplorer.usgs.gov/}) Геологической службы 
США (US Geological \linebreak Survey) и найдите доступные снимки со спутника Landsat 8 для точки 
с географическими координатами $60^{\circ}05'53''$ северной широты и $105^{\circ}25'49''$ восточной долготы 
на сентябрь 2018 года. Выберите только снимки с облачностью, занимающей менее 10\% сцены. 
(Если Вы всё сделаете правильно, будет найден только один такой снимок.)

Описание работы с порталом на русском языке можно прочитать здесь (\url{http://gis-lab.info/qa/earthexplorer-work.html}). 
Поскольку статья была написана уже несколько лет назад, она описывает немного устаревшую версию портала. Однако, практически никаких существенных изменений не произошло. Единственным исключением является название набора данных, который Вам необходим. Для решения практических задач в рамках Олимпиады потребуются снимки с уровнем обработки Landsat Collection 1 Level-1. В остальном можно полагаться на описание из указанной статьи.

Скачайте найденную сцену (снимок с уникальным идентификатором) на Ваш компьютер. 
(Это потребует регистрации и заполнения анкеты на английском языке. Рекомендуем по возможности 
добросовестно ответить на вопросы анкеты.) Полученный архивный файл необходимо распаковать с 
помощью любой доступной Вам программы-архиватора. В результате распаковки Вы должны получить 14 файлов, 
12 из которых представляют из себя изображения в отдельных спектральных каналах в формате GeoTIFF. 
Описание каналов Landsat можно найти, например, здесь \url{https://landsat.usgs.gov/what-are-band-designations-landsat-satellites} или 
здесь \url{https://ru.wikipedia.org/wiki/Landsat-8#Operational_Land_Imager_(OLI)}. (Вы можете загрузить каждый их этих файлов по отдельности в Ваш проект в QGIS с помощью инструмента "Добавить растровый слой" и рассмотреть их как изображения в оттенках серого.)

Введите в поле ниже размер в байтах (целое число) файла, содержащего панхроматический канал.

\explanationSection

С помощью упомянутого в условии задачи портала EarthExplorer (\url{https://earth} \\ \url{explorer.usgs.gov/}) вы можете найти и подобрать снимки Landsat (а также множество других данных) на интересующую вас территорию или точку, за определённый период времени. Вы также можете отобрать снимки по степени облачности и другим параметрам, предварительно посмотреть уменьшенный вид снимка.

Хотя интерфейс портала англоязычный, существуют подробные инструкции по работе с ним на русском языке, в частности, статья, непосредственно упомянутая в условии задачи: \url{http://gis-lab.info/qa/earthexplorer-work.html}. Поскольку статья была написана уже несколько лет назад, она описывает немного устаревшую версию портала. Однако, существенных изменений немного. Важным исключением является название набора данных, который Вам необходим. Для решения практических задач в рамках Олимпиады потребуются снимки с уровнем обработки Landsat Collection 1 Level-1. В остальном можно, в целом, полагаться на описание из указанной статьи. (Всё это также прямо написано в условии.)

Скачивание снимков с портала требует регистрации и заполнения анкеты на английском языке. В целом, для её заполнения достаточно знаний английского в программе школьного курса. В сложных случаях следует воспользоваться «Гугл Переводчиком». В упомянутой выше статье есть некоторые рекомендации по заполнению анкеты. Но не обязательно просто слепо следовать им.

Выбрав необходимый снимок (на профессиональном языке обычно говорят «сцену»), его можно загрузить его на свой компьютер в нескольких разных формах. Выбор («Download Options») появляется в самом конце процесса при нажатии кнопки «Download». Однако, почти все их них являются малопригодными для серьёзной работы картинками, сжатыми до небольших размеров с потерей информации (LandsatLook). Для профессиональной работы необходимо выбирать последнюю в списке опцию – «Level-1 GeoTIFF Data Product». Она существенно превышает все прочие по размеру: как правило, размер сгружаемого файла приближается к 1 гигабайту.

Скачанный файл имеет расширение *.gz и является архивным (сжатым файлом). Его необходимо распаковать с помощью любой доступной программы-архиватор,а например, уже упомянутого в Задаче 1 «7-Zip». (Среди них достаточно и других, распространяемых свободно, или бесплатных для тестирования в течение определённого срока.) Большинство современных программ такого типа понимают данный формат архива и легко справятся с распаковкой. Необходимо помнить, что после распаковки извлечённые из архива файлы займут ещё около двух гигабайт на каждую сцену – необходимо иметь достаточно свободного места на диске. Мы рекомендуем распаковывать каждый архив в отдельную папку (директорию).

В результате распаковки скачанного снимка (сцены) Landsat 8 участники должны получить 14 отдельных файлов, из которых два представляют собой текстовые документы с технической документацией, а остальные 12 – графические изображения в формате GeoTIFF. Размер текстовых файлов с документацией – мизерный, по сравнению с графическими файлами.

Большинство оставшихся графических файлов имеют одинаковый размер, в данном случае - 129825 килобайт. Они представляют собой изображения в отдельных спектральных каналах в формате GeoTIFF. Описание каналов Landsat можно найти на английском языке, например, здесь: \url{https://landsat.usgs.gov/what-are-band-designations-landsat-satellites} или, на русском языке, здесь: \url{https://ru.wikipedia.org/wiki/Landsat-8#Operational_Land_Imager_(OLI)}.

Все графические файлы названы по одной схеме:
\begin{itemize}
    \item первые четыре символа – «LC08» – обозначают спутник, с которого производилась съёмка, – Landsat 8;
    \item ещё четыре символа после разделителя в виде подчёркивания – «L1TP» – уровень обработки данных – Level 1;
    \item потом после разделителя идут 6 символов, обозначающие номер витка (136) и ряда (номер сцены) в витке (018) – Path и Row, они всегда стандартные для спутников Landsat;
    \item потом – восемь символов, обозначающие дату съёмки в формате «год-месяц-день», в данном случае – это 10 сентября 2018 года;
    \item потом – ещё весь символов, дата съёмки;
    \item большинство файлов заканчиваются буквой «B» (от английского «band» – канал, полоса) с номером, который и является номером спектрального канала съёмочной аппаратуры спутника Landsat.
\end{itemize}

Файл, название которого оканчивается на «BQA» содержит данные технического канала, куда помещаются данные о наличии облачности.

И только один файл, содержащий 8-й канал, имеет размер, существенно превышающий остальные файлы каналов. В нашем случае его размер – 519106 килобайт. Это так называемый панхроматический канал. Его большой размер, как мы увидим в дальнейшем, связан с тем, что он имеет вдвое большее пространственное разрешение, чем другие каналы Landsat.

Участники должны продемонстрировать понимание  того, какой именно канал является панхроматическим, и умение измерить точный размер файл в своей операционной системе. Например, в большинстве систем Windows по умолчанию файлы в папке отображаются в виде уменьшенных изображений. Во многих случаях удобнее отображать файлы в виде списка с отображением всех важных характеристик. Для чего нужно изменить соответствующие настройки.

Также точный размер файла можно посмотреть в его свойствах. При этом, место, которое файл занимает на диске, может отличаться в разных операционных и файловых системах, но размер самого файла от этого не зависит – он будет одинаковый и в Windows любой версии, и в MacOS.

Точный размер графического файла, содержащего данные панхроматического канала, – 531563798 байт.

\answerMath{531563798. В качестве правильного принимался только абсолютно точный, до единиц байтов, ответ.}

