\assignementTitle{Поиск и загрузка космических снимков Landsat}{1}

Снимки земли с американских спутников серии Landsat сегодня бесплатно доступны для всех желающих. 
Воспользуйтесь порталом EarthExplorer \linebreak (\url{https://earthexplorer.usgs.gov/}) Геологической службы 
США (US Geological \linebreak Survey) и найдите доступные снимки со спутника Landsat 8 для точки 
с географическими координатами $60^{\circ}05'53''$ северной широты и $105^{\circ}25'49''$ восточной долготы 
на сентябрь 2018 года. Выберите только снимки с облачностью, занимающей менее 10\% сцены. 
(Если Вы всё сделаете правильно, будет найден только один такой снимок.)

Описание работы с порталом на русском языке можно прочитать здесь (\url{http://gis-lab.info/qa/earthexplorer-work.html}). 
Поскольку статья была написана уже несколько лет назад, она описывает немного устаревшую версию портала. Однако, практически никаких существенных изменений не произошло. Единственным исключением является название набора данных, который Вам необходим. Для решения практических задач в рамках Олимпиады потребуются снимки с уровнем обработки Landsat Collection 1 Level-1. В остальном можно полагаться на описание из указанной статьи.

Скачайте найденную сцену (снимок с уникальным идентификатором) на Ваш компьютер. 
(Это потребует регистрации и заполнения анкеты на английском языке. Рекомендуем по возможности 
добросовестно ответить на вопросы анкеты.) Полученный архивный файл необходимо распаковать с 
помощью любой доступной Вам программы-архиватора. В результате распаковки Вы должны получить 14 файлов, 
12 из которых представляют из себя изображения в отдельных спектральных каналах в формате GeoTIFF. 
Описание каналов Landsat можно найти, например, здесь \url{https://landsat.usgs.gov/what-are-band-designations-landsat-satellites} или 
здесь \url{https://ru.wikipedia.org/wiki/Landsat-8#Operational_Land_Imager_(OLI)}. (Вы можете загрузить каждый их этих файлов по отдельности в Ваш проект в QGIS с помощью инструмента "Добавить растровый слой" и рассмотреть их как изображения в оттенках серого.)

Введите в поле ниже размер в байтах (целое число) файла, содержащего панхроматический канал.