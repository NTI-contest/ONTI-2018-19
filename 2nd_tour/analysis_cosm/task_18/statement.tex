\assignementTitle{Анализ изменений (сhange detection) лесного покрова в тропических лесах}{5}{}

Для территории из задачи 1 данного раздела (территория в пределах той же сцены Landsat 8, которая попадает при этом в границы существующих или планируемых особо охраняемых природных территорий всех категорий) выделите все участки, которые за период с августа-сентября 2017 года по октябрь 2018 года лишились своего лесного покрова, и измерьте их площадь. Для расчета площади используйте ту же проекцию, в которой находится скачанная Вами сцена Landsat 8 на данный участок.

Для этого Вам будет необходимо скачать ещё один или несколько снимков Landsat и/или Sentinel-2 (\url{https://stepik.org/lesson/189306/step/3?unit=164850}) на эту же территорию, снятых в июле-сентябре 2017 года. Вы можете выделить территории, где произошли изменения с помощью DTclassifier (\url{http://gis-lab.info/qa/dtclassifier.html}), использовать иной алгоритм change detection или даже обвести их вручную.

Результат измерений переведите в гектары, округлите до десятков.

\explanationSection

Эта задача во многом аналогична Задаче 6 первого блока. Однако, в данном случае нельзя просто использовать информацию о пожарах текущего года, так как потеря лесного покрова может происходить и без пожара.

\answerMath{10 000. В качестве правильного принимался ответ с погрешностью $\pm$3 000 (30\%).}