\assignementTitle{Геообработка и пространственный анализ. Формальная охрана малонарушенных лесов Амазонии}{3}{}

Малонарушенные лесные территории (МЛТ; \url{http://borealforestplatform.org/ru/intact_forest_landscapes/}, 
Intact forest landscapes; \url{https://en.wikipedia.org/wiki/Intact_forest_landscape}) - это крупные природные ландшафты в пределах лесной зоны, в наименьшей степени нарушенные хозяйственной деятельностью человека. С 2000 года ученые и экологические организации нескольких стран ведут картографирование этих территорий с помощью космических снимков по всему миру. По их данным, площадь МЛТ практически повсеместно сокращается по тем или иным причинам, связанным почти всегда с деятельностью человека. Самый большие площади МЛТ тропиков находятся в Бразилии. 

Определите, какая доля (в процентах) общей площади малонарушенных лесных территорий 
(МЛТ, Intact forest landscapes) Бразилии охраняется в пределах различных особо охраняемых природных территорий 
(ООПТ; \url{https://en.wikipedia.org/wiki/Protected_areas_of_Brazil}) и/или индейских резерваций (\url{https://en.wikipedia.org/wiki/Indigenous_territory_(Brazil)}), в которых использование \linebreak природных ресурсов также ограничено.

Используйте информацию о границах МЛТ по состоянию на 2016 год из набора векторных пространственных данных с 
сайта Intactforests.org\\ (\url{http://www.intactforests.org/data.ifl.html}).

В качестве источника исходных векторных данных по границам ООПТ и индейским резервациям используйте 
Всемирную базу данных по особо охраняемым территориям (World Database on Protected Areas - WDPA; \url{https://www.unep-wcmc.org/resources-and-data/wdpa}) 
от Всемирного центра природоохранного мониторинга (World Conservation Monitoring Centre; \url{https://www.unep-wcmc.org/}) Программы ООН по окружающей среде. Набор векторных данных на весь мир очень велик по объёму - мы рекомендуем вам сгрузить данные только на нужную вам территорию.

При геообработке и расчетах используйте только границы существующих \linebreak (Designated), то есть уже созданных, а не планируемых ООПТ. Территории, включенные (Inscribed) в список Всемирного природного наследия ЮНЕСКО (World Heritage List) считайте существующими ООПТ: международный статус обычно гарантирует некую охрану. В реальности такие территории частично или полностью пересекаются с действующими ООПТ.

Используйте инструменты геообработки (geoprocessing) векторных данных в \linebreak QGIS или в другом используемом вами пространственном обеспечении ГИС.  Мы рекомендуем вам производить все операции геообработки с данными в десятичных градусах и перевести в нужную проекцию уже на конечном этапе - это снизит вероятность того, что дадут о себе знать возможные топологические ошибки исходных данных.

Вычисления площадей производите в равновеликой конической проекции Альберса для Южной Америки (South America Albers Equal Area Conic) с центральным меридианом 60 градусов западной долготы и главными параллелями 5 и 42 градуса южной широты в южно-американской системе координат 1969 года (EPSG:102033).

Ответ представьте в процентах с округлением до десятых долей процента.

\explanationSection

Данная задача практически полностью аналогична Задаче А первого блока. Здесь участвуют те же наборы векторных данных – границы малонарушенных лесных территорий и границы особо охраняемых природных территорий. Как и в задаче первого блока, речь идёт только о существующих ООПТ и тоже необходимо посчитать процентное соотношение.

Единственной дополнительной сложностью здесь является необходимость обрезки исходного набора данных по малонарушенным лесам государственными границами Бразилии. Пожалуй, простейший способ сделать это – воспользоваться границами сухопутных экорегионов Бразилии, которая уже использовалась для получения границ Амазонии в предыдущих задачах.

Однако, границы стран в виде наборов векторных ГИС можно найти и во многих других местах. В частности, они есть на портале OpenStreetMap (\url{https://www.openstreetmap.org}) откуда их можно экспортировать через различные сторонние сайты
(\url{https://wiki.openstreetmap.org/wiki/Processed_data_providers}).

\answerMath{71.7\%. В качестве правильного принимался ответ с погрешностью $\pm$0.2\%.}
