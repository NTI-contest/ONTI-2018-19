\assignementTitle{Склейка каналов и загрузка снимков в виде мультиспектральных изображений. Визуальное выделение объектов по космическим снимкам\\}{1}{}

Вокруг точки с географическими координатами $59^{\circ}31'14''$ северной широты и $105^{\circ}36'10''$ 
восточной долготы на скачанном вами при решении предыдущей задачи снимке хорошо видны следы лесного пожара, 
произошедшего в 2018 году.

Однако, хорошо они видны не на одноканальных изображениях, которые вы получили при распаковке архива, 
а на цветном многоканальном изображении. Чтобы получить такое изображение, вам необходимо склеить, 
по крайней мере, три канала в единый файл. В QIS для этого можно использовать функцию "Объединение" из меню 
работы с растрами. Обратитесь к документации на QGIS и/или к этой статье(\url{http://gis-lab.info/qa/qgis-landsat-merge.html}) (или к документации иного используемого вами программного обеспечения). Для уменьшения размера файла мы не рекомендуем вам склеивать панхроматический канал вместе с другими. Вообще, не обязательно склеивать все каналы - всё равно одновременно вы сможете рассматривать в QGIS только любые три из них.

Полученный в результате склейки каналов файл вы можете загрузить в ваш проект в QGIS с помощью инструмента "Добавить 
растровый слой". По умолчанию в оттенках красного цвета отображается первый из склеенных каналов, в оттенках зелёного - второй и в оттенках синего - третий. В QGIS это можно поменять в свойствах слоя в закладке "Стиль". Там же можно настроить более контрастное изображение. Обратитесь к документации QGIS (или другого используемого вами ПО) за подробностями.

Подберите комбинацию каналов, при которой вы будете хорошо различать свежую гарь вокруг точки с 
указанными координатами от окружающей растительности. Про использование различных комбинаций каналов 
можно прочесть здесь \url{http://gis-lab.info/qa/landsat-bandcomb.html} и здесь \url{https://habr.com/post/183416/}. Помните, что у разных спутников серии Landsat набор и нумерация каналов отличаются. Мы рекомендуем использовать комбинацию красного и ближних инфракрасных каналов (каналы 4, 5 и 6 для Landsat 8), в которых хорошо видны различия в растительности (отображая при этом 6-й канал в оттенках красного, а 4-й - в оттенках синего). Но вы можете выбрать и иную комбинацию, в которой гарь для вас выглядит наиболее ясно.

Для того, чтобы загруженные изображения выглядели контрастно, обычно также необходимо подстроить гистограммы распределения яркостей каждого спектрального канала под диапазон яркостей соответствующего цвета на вашем экране. В QGIS самый простой способ это сделать - воспользоваться панелью "Инструменты работы с растровыми данными". (Более сложный - через закладку "Гистограмма" в свойствах слоя.)

Измерьте как можно точнее площадь гари по внешнему контуру с помощью инструментов QGIS или другой используемой вами ГИС-программы. (Для того, чтобы сохранить ваш контур, вы также можете создать полигональный набор векторных данных в формате шейп-файла и провести расчет площади для него.) "Островки" несгоревшего леса, частично погибшего древостоя и болота, попавшие в пределы контура, включите в общую площадь. Но не включайте в контур болота и несгоревшие участки, если они прилегают к его внешнему краю. Используйте для измерений проекцию, в которой находится скачанный космический снимок.

Ответ выразите в гектарах, округлите до десятков и впишите в поле ниже.

\explanationSection

Что «видеть» нашими глазами информацию сразу с нескольких спектральных каналов Landsat, нем необходимо собрать их в многоканальное изображение. В QGIS это можно сделать с помощью утилиты «Объединение», которая находится в меню в разделе «Растр» --> «Прочее». Пошагово процедура объединения каналов подробно описана в статье, ссылка на которую даётся в условиях задачи: \url{http://gis-lab.info/qa/qgis-landsat-merge.html}. Также, можно обратиться к документации QGIS (\url{https://docs.nextgis.ru/docs_ngqgis/source/raster_op.html#id21}).

Для актуальной версии QGIS мы приводим пошаговые инструкции по склеиванию каналов здесь: \url{https://stepik.org/lesson/211332/step/4?unit=187728}. Ряд типичных ошибок разобраны здесь: \url{https://stepik.org/lesson/211332/step/5?uni} \\ \url{t=187728}. Однако, такие подробные инструкции не были доступны участникам – им было необходимо разобраться самостоятельно.

В статье, ссылка на которую даётся в условиях задачи (\url{http://gis-lab.info/qa/qgis-landsat-merge.html}), также описывается, как настроить в QGIS полученное в результате объединения растровое изображение для отображения различных вариантов синтеза спектральных каналов. Непосредственно в условии задачи также даны рекомендации по использованию определённых каналов для дешифрирования растительности и ссылки на соответствующие публикации. Пошаговые инструкции можно найти здесь: \url{https://stepik.org/lesson/211332/step/8?unit=187728}, и здесь: \url{https://stepik.org/lesson/211332/step/9?unit=187728}.

Для того, чтобы загруженные изображения выглядели контрастно, также необходимо подстроить гистограммы распределения яркостей каждого спектрального канала под диапазон яркостей соответствующего цвета на экране. Непосредственно в условии задачи содержится ссылка на самый простой способ это сделать в QGIS – воспользоваться панелью «Инструменты работы с растровыми данными». См. пошаговое описание здесь: \url{https://stepik.org/lesson/211332/step/7?unit=187728}.

При выполнении перечисленных выше условий, гарь по указанным координатам видна совершенно отчётливо – как красно-фиолетовое пятно на преимущественно зелёном фоне.

\putImgWOCaption{8.5cm}{1}

Поверх такого контрастного изображения уже достаточно легко провести контур гари, соблюдая требования из условий задачи. Наиболее корректный способ сделать это – создать набор векторных пространственных данных в полигональной топологии и воспользоваться инструментами редактирования в настольной ГИС.

Приёмы редактирования геометрии векторных данных в QGIS, в частности, подробно описаны в руководстве пользователя QGIS, которое указывалось как одно из основных справочных пособий для участников Олимпиады: \url{https://docs.nextgis.ru/docs_ngqgis/source/editing.html#id13}.

Данная задача, в значительной мере, оценивала навыки ручного редактирования векторных данных, умение участников выделять объекты на космических снимках, а также точность и аккуратность проведения контуров. Поэтому в качестве правильных ответов принимался достаточно узкий диапазон значений площади контура, который может быть получен только при аккуратном проведении границы гари и точном следовании условиям задачи.

Примеры точного (слева) и неточного (справа) проведения контура границ гари:

\putTwoImg{8cm}{2}{8cm}{3}

Методы подсчёта площадей также описаны в руководстве пользователя QGIS: \url{https://docs.nextgis.ru/docs_ngqgis/source/editing.html#id10}.

\answerMath{10 000. Как правильный принимался ответ с погрешностью $\pm$150.}