\assignementTitle{Выделение однородных объектов. Классификация с помощью деревьев решений}{5}{}

Точка с географическими координатами 59$^{\circ}31'14''$ северной широты и 105$^{\circ}36'10''$ восточной долготы 
находится не только в пределах гари от пожара 2018 года, но и, вместе с самой гарью, в пределах границ 
крупной малонарушенной лесной территории (МЛТ; \url{http://www.intactforests.org/}). Скачайте с уже знакомого 
вам сайта Intactforests.org (\url{http://www.intactforests.org/data.ifl.html}) границы МЛТ по состоянию на 2016 год и выберите контур (полигон), в пределах которого находится указанная точка и гарь.

С помощью DTclassifier (\url{http://gis-lab.info/qa/dtclassifier.html}) выделите все недавно пройденные огнём территории (те, на которых сигнал от грунта и углей преобладает над сигналом от живой древесной растительности) в пределах данного контура МЛТ и (одновременно) в пределах скачанного Вами ранее снимка Landsat 8. Обратите внимание, что при таком подходе в пределах уже известной вам гари выделится меньшая площадь, чем в Задаче 3, где вы включали в общий контур гари болота и другие участки с сохранившейся растительностью.

Контур данной МЛТ выходит за пределы снимка - используйте только часть контура, попадающего в пределы снимка, иначе говоря, - пересечение снимка с контуром МЛТ. В пределы снимка также попадают другие МЛТ, расположенные по соседству, - игнорируйте их. Ваша задача - выделить все гари ТОЛЬКО в пределах данного контура МЛТ, в части, покрываемой данным снимком Landsat.

Переведите результат классификации в векторный полигональный слой. В QGIS это можно сделать с помощью инструмента "Создание полигонов (растр в вектор)". Добавьте числовое поле в атрибутивную таблицу полученного векторного слоя и рассчитайте площадь каждого контура (полигона) средствами QGIS в той же проекции, в которой находится космический снимок. Запаситесь терпением, временем и компьютерными мощностями. Результат автоматической классификации обычно имеет довольно сложную конфигурацию границ, и их векторизация и геометрические расчеты занимают немало времени. Для облегчения задачи вы можете применить различные инструменты сглаживания, обрабатывать полученный растр по частям или рассчитать площадь по растровым данным (предварительно убедившись, что это не вносит существенных искажений). Ручная "доводка" результатов автоматического дешифрирования также является допустимой практикой.

Переведите результат в гектары и округлите до десятков.

\explanationSection

Как и все задачи на дешифрирование космических снимков, данная задача не имеет единственного «правильного» пути решения, который ведёт к корректному результату. Выделить сгоревшие территории можно десятками различных способов, включая выделение вручную.

Использование предлагаемого программного обеспечения, DTclassifier, реализующего один из популярных алгоритмов классификации, «деревья решений», достаточно исчерпывающе описано по ссылка, приведённым непосредственно в условии задачи.

В гораздо большей степени, чем от выбора алгоритма и от реализующего его программного обеспечения, результат зависит от обучающих выборок, на которых происходит обучение большинства алгоритмов. Обучающие выборки (трейнинги) участники должны были нарисовать вручную в виде векторных контуров, охватывающих примеры характерных участков, относящихся и не относящихся к гарям. Примеры таких трейнингов приведены на рисунке ниже. Жёлтым цветом показаны участки гарей, синим – примеры других типов растительного покрова.

\putImgWOCaption{11cm}{1}

Результаты классификации, которые обычно получаются в виде растровых наборов данных, можно перевести в векторный формат (оцифровать границы выделенных участков гарей) для дальнейших операций и подсчета площадей либо оставить в растровом формате. В последнем случае подсчет площадей можно произвести с помощью подсчёта количества пикселей, которым присвоено значение, соответствующее наличию выделяемых объектов – гарей. (В случае работы с DTclassifier’ом таким пикселям присваивается значение «1».)

Для подсчета количества пикселей в пределах векторного контура малонарушенной лесной территории, в QGIS можно воспользоваться инструментом «Зональная статистика». Зная размер пикселя (пространственное разрешение) снимка Landsat (30х30 метров) можно посчитать общую площадь выделенных гарей как произведение 900 кв.м. на количество пикселей, определённых в качестве гари в пределах данного векторного контура.

Можно пойти и по другому пути анализа результатов дешифрирования, использую навыки геообработки (пересечения) векторных наборов данных и подсчета площадей полигонов, полученных в результате решения предыдущих задач. Для этого необходимо перевести результат классификации в векторный полигональный слой. В QGIS это можно сделать с помощью инструмента «Создание полигонов (растр в вектор)». Данная функция описана в руководстве пользователя: \utl{https://docs.nextgis.ru/docs_ngqgis/source/raster_op.html#id7}.

В QGIS такая операция может занять достаточно много времени, так как результат автоматической классификации обычно имеет довольно сложную конфигурацию границ. Облегчить векторизацию можно, используя инструменты сглаживания полученного в результате классификации растра. Один из алгоритмов сглаживания реализован непосредственно в DTclassifier.

Также работа утилиты, переводящей растровые данные в векторные, существенно облегчится, если сразу обрезать результаты классификации космического снимка векторными границами малонарушенной лесной территории. Сделать это в QGIS можно с помощью инструмента «Обрезка» из меню «Растр». Работа с ним также описана в руководстве пользователя: \url{https://docs.nextgis.ru/docs_ngqgis/source/raster_op.html#id12}. В этом случае результат векторизации уже не потребуется повторно обрезать границами малонарушенной лесной территории.

Пример вида исходного снимка Landsat и результата выделения на нём свежих гарей приведены на рисунках ниже. Красным показан контур малонаушенной лесной территории, в пределах которой необходимо произвести подсчёты.

\putImgWOCaption{11cm}{2}

Выделенные в результате классификации участки свежих гарей показана сиреневым цветом:

\putImgWOCaption{11cm}{3}

\answerMath{170 000. В силу того, что выделение свежих гарей на данной территории Восточной Сибири является достаточно сложной задачей, ответы для этой задачи принимались с достаточно большой погрешностью. Правильным считался ответ с погрешностью $\pm$80 000.}