\assignementTitle{Выделение однородных объектов. Классификация с помощью деревьев решений}{5}{}

Точка с географическими координатами 59$^{\circ}31'14''$ северной широты и 105$^{\circ}36'10''$ восточной долготы 
находится не только в пределах гари от пожара 2018 года, но и, вместе с самой гарью, в пределах границ 
крупной малонарушенной лесной территории (МЛТ; \url{http://www.intactforests.org/}). Скачайте с уже знакомого 
вам сайта Intactforests.org (\url{http://www.intactforests.org/data.ifl.html}) границы МЛТ по состоянию на 2016 год и выберите контур (полигон), в пределах которого находится указанная точка и гарь.

С помощью DTclassifier (\url{http://gis-lab.info/qa/dtclassifier.html}) выделите все недавно пройденные огнём территории (те, на которых сигнал от грунта и углей преобладает над сигналом от живой древесной растительности) в пределах данного контура МЛТ и (одновременно) в пределах скачанного Вами ранее снимка Landsat 8. Обратите внимание, что при таком подходе в пределах уже известной вам гари выделится меньшая площадь, чем в Задаче В, где вы включали в общий контур гари болота и другие участки с сохранившейся растительностью.

Контур данной МЛТ выходит за пределы снимка - используйте только часть контура, попадающего в пределы снимка, иначе говоря, - пересечение снимка с контуром МЛТ. В пределы снимка также попадают другие МЛТ, расположенные по соседству, - игнорируйте их. Ваша задача - выделить все гари ТОЛЬКО в пределах данного контура МЛТ, в части, покрываемой данным снимком Landsat.

Переведите результат классификации в векторный полигональный слой. В QGIS это можно сделать с помощью инструмента "Создание полигонов (растр в вектор)". Добавьте числовое поле в атрибутивную таблицу полученного векторного слоя и рассчитайте площадь каждого контура (полигона) средствами QGIS в той же проекции, в которой находится космический снимок. Запаситесь терпением, временем и компьютерными мощностями. Результат автоматической классификации обычно имеет довольно сложную конфигурацию границ, и их векторизация и геометрические расчеты занимают немало времени. Для облегчения задачи вы можете применить различные инструменты сглаживания, обрабатывать полученный растр по частям или рассчитать площадь по растровым данным (предварительно убедившись, что это не вносит существенных искажений). Ручная "доводка" результатов автоматического дешифрирования также является допустимой практикой.

Переведите результат в гектары и округлите до десятков.