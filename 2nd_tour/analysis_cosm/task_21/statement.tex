\assignementTitle{Валидация результатов дешифрирования. Расчет матрицы ошибок}{3}{}

С помощью инструментов пространственной выборки в QGIS или в другой ГИС-системе, которую вы используете, определите количество пробных площадей, которые совпадают и не совпадают с каждым из ваших двух классов ("гарь" и "не гарь"), на которые вы разделили территорию в результате классификации снимка Landsat.

Составьте матрицу ошибок, рассчитав соответствующие цифры в ней пропорционально доле верно и неверно классифицированных вами пробных площадей каждого класса и доле площади каждого класса от всей территории исследования. Подробнее о таких расчетах можно прочитать, например, в этой статье\\ \url{http://reddcr.go.cr/sites/default/files/centro-de-documentacion/olofsson_et_al._2014_-_good_practices_for_estimating_area_and_assessing_accuracy_of_land_change.pdf}.

На основании полученной матрицы ошибок, рассчитайте ошибку производителя для вашего класса "гари". Выразите её в процентах, округлите до десятых долей.