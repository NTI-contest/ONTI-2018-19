\assignementTitle{Подсчет запаса древесины на плантациях на \\ Суматре}{24}{}

Вокруг точки с географическими координатами 0$^{\circ}$07'49'' северной широты и \linebreak 102$^{\circ}$48'28'' 
восточной долготы расположен огромный массив плантаций быстрорастущих деревьев (эвкалиптов и/или акаций), 
выращиваемых на целлюлозу. По сведениям портала Всемирной лесной вахты (\url{https://www.globalforestwatch.org/map}), древесина с этих плантаций поставляется на целлюлозно-бумажный комбинат одной из крупнейших мировых компаний на этом рынке -  
Asia Pulp \& Paper.

Скачайте с указанного портала векторный набор данных с границами концессионных участков, предоставленных компаниям для выращивания древесины на целлюлозу (Wood Fiber Concessions). Выберите три непосредственно граничащих друг с другом участка, ближайших к выбранной точке. Обратите внимание, что в данном векторном слое некоторые объекты состоят из нескольких полигонов (могут объединять несколько не граничащих друг с другом участков, принадлежащих одной компании). Для разделения таких полигонов в QGIS воспользуйтесь инструментом "Разбить составные объекты".

Оцените объём древесины на корню в пределах этих трёх выделенных концессионных участков по состоянию на середину сентября 2018 года. Для этого вам нужно выделить одновозрастные участки плантаций и определить возраст и площадь каждого из них. Доступные сегодня космические снимки позволяют это сделать.

С помощью уже известного Вам портала EarthExplorer (\url{https://earthexplorer.usgs.gov/}) подберите и 
скачайте необходимые Вам для этого космические снимки за текущий и прошедшие год. 
С помощью визуального дешифрирования или с помощью освоенных вами инструментов 
автоматической классификации (DTclassifier \url{http://gis-lab.info/qa/dtclassifier.html} или другого алгоритма/инструмента) выделите вырубки (очищенные от деревьев участки) на разных снимках.

На таких плантациях посадка следующего поколения быстрорастущих деревьев обычно происходит вскоре после вырубки предыдущего поколения. Полог вновь посаженных деревьев смыкается уже через несколько месяцев. Таким образом, вся вырубка, как правило, становится участком с деревьями одного возраста, который можно примерно отсчитывать от момента вырубки. (Исключение представляют случаи гибели насаждений или забрасывания участков. Как правило, это также можно видеть на космических снимках.)

Таким образом, для каждого одновозрастного участка плантаций Вам необходимо определить год, когда на данном участке ПОСЛЕДНИЙ ПО ВРЕМЕНИ раз был полностью сведён древесный покров (вырублено и отправлено на переработку предыдущее поколение быстрорастущих деревьев). 

Вам придётся обрабатывать значительное число снимков. Написанные Вами программы для автоматического отбора, скачивания и классификации снимков могут существенно ускорить получение результата. Но не забывайте проверять результат работы Ваших алгоритмов визуально!

Если, из-за многомесячной облачности, Вы сомневаетесь в том, не пропустили ли Вы смену поколений деревьев 
(ситуация, когда один снимок сделан до вырубки, а следующий - через несколько месяцев, уже после 
вырубки и смыкания крон молодых деревьев), воспользуйтесь снимками более низкого разрешения, чем Landsat и 
Sentinel, которые снимаются на данный участок практически ежедневно. Например, 
снимки Modis также доступны на портале EarthExplorer (\url{https://earthexplorer.usgs.gov/}). Несмотря на их низкое пространственное разрешение, крупные вырубки будут на них видны.

Если границы одновозрастного участка Вами уже определены, можно не скачивать к себе на компьютер все 
необходимые снимки, а просмотреть их онлайн, воспользовавшись доступными порталами открытых данных. 
Например, порталами LandLook Viewer Геологической службы США (USGS; \url{https://landlook.usgs.gov/viewer.html}) и WorldView (\url{https://worldview.earthdata.nasa.gov/}) американского космического агентства НАСА.

Зная возраст каждого участка плантации, Вы сможете оценить общий доступный запас древесины. Для целей данной задачи считать ежегодный средний прирост - 10 куб.м. на гектар в год. Все расчеты площадей производить для той проекции, в которой находятся скачанные Вами космические снимки Landsat. Выразите Вашу оценку запаса древесины в тысячах куб.м., округлите до целого числа.

\explanationSection

Задача фактически сводится к определению для каждого конкретного участка плантаций момента, когда на этом месте последний раз наблюдалась расчистка (вырубка, безлесая территория). Лишённые растительности участки, пока не произойдёт смыкания полога посаженных на них деревьев, прекрасно видны на космических снимках и хорошо отличаются от древесного полога. Для выявления такого момента необходимо просмотреть все доступные снимки ландсат и сентинель из архива за последние 5-10 лет. Сделать это можно не только скачивая снимки на компьютер участников, но и с помощью онлайновых сервисов типа LandLook Viewer (\url{https://landlook.usgs.gov/viewer.html}), на который дана ссылка в условиях задачи.

Все обнаруженные участки, лишённые растительности, необходимо оконтуривать, присваивая полученному полигону атрибутивную информация от дате последнего наблюдения данной территории в состоянии без растительности. Делать это для ограниченной территории эффективнее всего «вручную» – путем визуального дешифрирования. При этом наиболее аккуратной техникой является разделение единого полигона плантаций на участки с использованием инструмента «Разбить объекты» в режиме редактирования в QGIS. Важно внимательно просматривать все доступные космические снимки, чтобы не пропустить момент вырубки.

Для полученного набора векторных данных останется рассчитать площадь каждого участка, а затем вычислить запас древесины, умножив его на возраст участка (количество лет с момента наблюдения на снимках последней вырубки) и приведённый в условиях задачи среднегодовой прирост.

\answerMath{1 022 $\pm$ 200.}
