\assignementTitle{Обновление карты плантаций масличной пальмы на Калимантане}{18}{}

На портале Всемирной лесной вахты (Global Forest Watch) найдите 
и скачайте векторный набор пространственных данных по древесным плантациям (Tree \linebreak Plantations). Он 
содержит границы разных типов древесных плантаций для нескольких тропических стран (включая Индонезию) 
по состоянию на 2013-2014 гг. Подробнее про этот проект можно прочитать здесь (\url{http://www.wri.org/sites/default/files/Mapping_Tree_Plantations_with_Multispectral_Imagery_-_Preliminary_}\linebreak \url{Results_for_Seven_Tropical_Countries.pdf}).

Открыв данный слой в QGIS в одном проекте со скачанными на указанную точку снимками Landsat 8, Вы увидите, что вокруг указанной точки находится крупный массив плантаций масличной пальмы. Однако, территория плантаций с 2014 года заметно расширилась. Изучите, как выглядит данный тип плантаций на снимках Landsat и Sentinel-2. Обратите внимание, что на плоской приречной (по видимому, заболоченной) равнине и в гористой местности плантации пальмы выглядят по-разному.

Используйте инструмент QGIS "Send to Google Earth" (модуль Send2GE должен быть установлен и включен) 
совместно с заранее установленной программой Google Earth Pro (\url{https://www.google.com/intl/ru/earth/desktop/}), чтобы посмотреть, как те или иные участки местности выглядят на снимках сверхвысокого разрешения. Помните, что такие снимки в программе Google Earth, в отличии от снимков Landsat и Sentinel, как правило, не слишком свежие и собраны из кусков, снятых в разное время. У них также потеряна бОлшая часть исходной спектральной информации, а цвета искусственно подобраны так, чтобы походить на естественные.

Используя метод визуального дешифрирования или автоматическую классификацию с помощью DTclassifier \url{http://gis-lab.info/qa/dtclassifier.html} (или с помощью другого доступного Вам алгоритма/инструмента) выделите все новые плантации масличной пальмы в пределах данной сцены Landsat 8, которые появились здесь с момента картографирования в 2013-2014 гг. по настоящее время. Площади, вырубленные и расчищенные под плантации, но ещё не засаженные (незаросшие) также относить к плантациям. Однако, имейте в виду, что участки могут расчищаться и для других целей, например, для добычи полезных ископаемых.

Измерьте площадь новых плантаций, выразите её в гектарах, округлите до десятков.

\explanationSection

Задача аналогичная Задаче 5 первого блока и Задачам 1 и 2 второго блока. Однако, объект довольно сложный и мог потребовать в ряде случаев использование визуального дешифрирования вместо автоматического.

\answerMath{43 300 $\pm$ 8 600.}