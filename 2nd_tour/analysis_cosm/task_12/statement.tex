\assignementTitle{Геообработка и пространственный анализ. Масштабы потенциальной угрозы охраняемым природным территориям Амазонии от добычи полезных ископаемых}{2}{}

В условиях предыдущей задачи определите, какая доля от общей площади территории в пределах всех существующих 
(Designated, Inscribed) ООПТ Амазонии окажется под угрозой, если разработка полезных ископаемых будет 
вестись на территории всех концессий, границы которых сегодня определены и входят в скачанный вами набор 
пространственных данных.

Ответ выразите в процентах, округлите до десятых долей процента.

\explanationSection

В данной задаче требуется подсчитать не абсолютную величину площади, а процент. Таким образом, необходимо найти как общую площадь заданных категорий ООПТ в пределах Амазонии, так и площадь всех ООПТ, попадающих в пределы границ лицензий на добычу полезных ископаемых.

В отличие от предыдущей задачи, здесь нельзя пренебрегать ни фактом наличия перекрывающихся ООПТ, ни топологическими ошибками в данном наборе данных. К счастью, он включает в себя гораздо меньшее число полигонов и не содержит существенных топологических ошибок. Обе проблемы решаются применением к данному набору векторных полигонов в QGIS инструмента "Объединение по признаку" ("Dissolve"). Можно слить все полигоны, выставив соответствующую опцию. Однако, эту операцию надо выполнить уже после того, как из исходного набора данных с границами ООПТ будут выбраны только границы существующих резерватов (Designated, Inscribed), как это требуется в условиях данной задачи. На любом из этапов надо также провести обрезку набора данных границами Амазонии, чтобы получить только границы ООПТ в пределах этого экорегиона.

Также полученный набор векторных данных (границы только существующих ООПТ в пределах Амазонии) необходимо обрезать границами лицензионных участков на разведку и добычу полезных ископаемых. В этом случае нам необходимо использовать уже не только уже переданные лицензионные участки (Exploration, Exploitation), но и все остальные. Поэтому воспользоваться уже подготовленным и очищенным от перекрытий топологических ошибок набором данных из предыдущих задач уже не получится.

В этом и нет необходимости, поскольку исходным набором данных для обрезки будет служить очищенный от перекрытий и топологических ошибок с помощью инструмента "Объединение по признаку" ("Dissolve") набор данных по границам существующих ООПТ Амазонии. Топологические ошибки в обрезающем наборе данных не имеют значения для подсчета площадей в итоговом наборе данных.

Разумеется, при выполнении всех упомянутых операций обрезки участвующие наборы данных должны находиться в одинаковой системе координат. Теоретически, это не обязательно должна быть использовавшаяся в предыдущих задачах равновеликая проекция Альберса: поскольку вычисляются соотношения площадей, а не сами площади, результаты расчета в любой системе координат не должны существенно отличаться. Однако, приведённый ниже ответ получен при расчете в той же системе координат, что и в прошлых задачах.

\answerMath{21.5\%. В качестве правильного принимался ответ с погрешностью $\pm$0.5.}
