\assignementTitle{Анализ изменений}{3}{}

С помощью DTclassifier (\url{http://gis-lab.info/qa/dtclassifier.html}) или иным способом, выделите в 
пределах той же территории, 
что и в Задаче 5 (часть контура малонарушенной лесной территории (\url{http://www.intactforests.org/}), 
попадающей в пределы скачанного вами первого снимка Landsat), все гари только от пожаров 2018 года. Для этого Вам будет необходимо скачать ещё один или несколько снимков Landsat на ту же территорию, снятых после окончания пожарного сезона 2017 года, но до соответствующих пожаров 2018 года. Возможно, что из-за облачности некоторые гари будут видны на одних снимках и закрыты облаками на других.

Чтобы расширить диапазон используемых вами снимков, с помощью того же EarthExplorer (\url{https://earthexplorer.usgs.gov/}) 

Вы можете скачать на ту же территорию также снимки со спутника Landsat 7 (\url{https://ru.wikipedia.org/wiki/Landsat-7}). 

Этот аппарат, запущенный ещё в 1999 году, продолжает работать до сих пор. И хотя с 2003, в результате поломки (\url{https://ru.wikipedia.org/wiki/Landsat-7#%D0%A1%D0%B1%D0%BE%D0%B9_Scan_Line_Corrector}), на его снимках теряется около четверти информации (визуально это выглядит как тёмные полосы на большей части изображения), оставшаяся информация может быть полезной (например, для решения о времени пожара).

Помните, что спектральные каналы Landsat 7 отличаются от Landsat 8 (\url{https://landsat.usgs.gov/what-are-band-designations-landsat-satellites}), а спектрально близкие каналы имеют другую нумерацию \linebreak (\url{http://gis-lab.info/qa/landsat-bandcomb.html}). Так, каналам 4, 5 и 6 Landsat 8 соответствуют каналы 3, 4 и 5 Landsat 7. 

Вы можете применить DTclassifier (\url{http://gis-lab.info/qa/dtclassifier.html}) или иную технологию для сравнения двух снимков (change detection). Также, если вы успешно справились с Задачей 5, вы можете отсортировать вручную те гари, которые произошли в 2018 году. Кроме того, свежие гари отличаются от прошлогодних и по спектральным характеристикам - вы можете попытаться выделить их с помощь классификации одного снимка.

Если из-за высокой облачности вам не хватает снимков Landsat, чтобы решить, в каком году произошёл 
тот или иной конкретный пожар, обратитесь к другим снимкам среднего и низкого разрешения и продуктам 
на их основе, доступным онлайн. Например, воспользуйтесь геопорталом "Карта пожаров" (\url{http://fires.ru/}) 
российской компании "СКАНЭКС", Worldview (\url{https://worldview.earthdata.nasa.gov/}) американского 
космического агентства НАСА и/или LandLook Viewer (\url{https://landlook.usgs.gov/viewer.html}) Геологической службы США (USGS). 

Результат также переведите в гектары и округлите до десятков.

\explanationSection

Данную задачу можно решать различными способами. Самый очевидный (хотя и не самый точный) метод состоит в том, чтобы найти и скачать на ту же территорию доступные космические снимки конца 2017 / начала 2018 года и произвести по ним аналогичным методом выделение гарей, существовавших в конце пожарного 2017 года. Разница в площади даст искомый ответ.

Другой подход состоит в выделении гарей 2018 года сразу в результате совместной классификации двух космических снимков на одну и ту же территорию. Пример такого подхода с использованием DTclassifier’а оказан на видео, доступном по этой ссылке (приводимой в условии задачи): \url{http://nextgis.ru/blog/dtclassifier-is-back/}. Такой метод должен давать более точные результаты, чем описанный выше.

Но, пожалуй, самый простой подход может состоять в том, чтобы выбрать из контуров гарей, полученных в результате решения предыдущей Задачи 5, те, которые появились в 2018 году. Сделать это можно, обведя гари 2018 года по другим источникам. Сделать это можно, в большинстве случаев, достаточно грубо. Для этого достаточно воспользоваться геопорталом \url{http://fires.ru/}, знакомым участникам по задачам первого этапа Олимпиады. Выбрав соответствующий временной интервал, охватывающий пожарный сезон 2018 года, можно вручную обвести примерными контурами все группы зафиксированных в 2018 году термоточек. Данный слой можно сохранить в качестве набора векторных данных в формате SHAPE (правда, делать это надо не на \url{http://fires.ru/}, а на «материнском» геопортале \url{http://kosmosnimki.ru/}, где также можно вывести точки пожаров).

Более точно границу придётся провести только в случаях, когда пожар 2018 года произошёл частично на территории гарей прошлых лет. В данном случае необходимо воспользоваться тем же снимков Landsat, который использовался в Задаче 5. Границы самых свежих и более старых гарей различаются на нём по своим спектральным характеристикам.

Полученные грубые векторные границы можно использовать для обрезки более точных границ гарей, полученных в результате решения Задачи 5.

\answerMath{21 000. В качестве правильного принимался ответ с погрешностью $\pm$4 000.}
