\assignementTitle{Анализ изменений}{3}{}

С помощью DTclassifier (\url{http://gis-lab.info/qa/dtclassifier.html}) или иным способом, выделите в 
пределах той же территории, 
что и в Задаче 5 (часть контура малонарушенной лесной территории (\url{http://www.intactforests.org/}), 
попадающей в пределы скачанного вами первого снимка Landsat), все гари только от пожаров 2018 года. Для этого Вам будет необходимо скачать ещё один или несколько снимков Landsat на ту же территорию, снятых после окончания пожарного сезона 2017 года, но до соответствующих пожаров 2018 года. Возможно, что из-за облачности некоторые гари будут видны на одних снимках и закрыты облаками на других.

Чтобы расширить диапазон используемых вами снимков, с помощью того же EarthExplorer (\url{https://earthexplorer.usgs.gov/}) 
Вы можете скачать на ту же территорию также снимки со спутника Landsat 7 (\url{https://ru.wikipedia.org/wiki/Landsat-7}). 
Этот аппарат, запущенный ещё в 1999 году, продолжает работать до сих пор. И хотя с 2003, в результате поломки (\url{https://ru.wikipedia.org/wiki/Landsat-7#%D0%A1%D0%B1%D0%BE%D0%B9_Scan_Line_Corrector}), 
на его снимках теряется около четверти информации (визуально это выглядит как тёмные полосы на большей 
части изображения), оставшаяся информация может быть полезной (например, для решения о времени пожара). 
Помните, что спектральные каналы Landsat 7 отличаются от Landsat 8 (\url{https://landsat.usgs.gov/what-are-band-designations-landsat-satellites}), 
а спектрально близкие каналы имеют другую нумерацию (\url{http://gis-lab.info/qa/landsat-bandcomb.html}). Так, каналам 4, 5 и 6 Landsat 8 соответствуют каналы 3, 4 и 5 Landsat 7. 

Вы можете применить DTclassifier (\url{http://gis-lab.info/qa/dtclassifier.html}) или иную технологию для сравнения двух снимков (change detection). Также, если вы успешно справились с Задачей Д, вы можете отсортировать вручную те гари, которые произошли в 2018 году. Кроме того, свежие гари отличаются от прошлогодних и по спектральным характеристикам - вы можете попытаться выделить их с помощь классификации одного снимка.

Если из-за высокой облачности вам не хватает снимков Landsat, чтобы решить, в каком году произошёл 
тот или иной конкретный пожар, обратитесь к другим снимкам среднего и низкого разрешения и продуктам 
на их основе, доступным онлайн. Например, воспользуйтесь геопорталом "Карта пожаров" (\url{http://fires.ru/}) 
российской компании "СКАНЭКС", Worldview (\url{https://worldview.earthdata.nasa.gov/}) американского 
космического агентства НАСА и/или LandLook Viewer (\url{https://landlook.usgs.gov/viewer.html}) Геологической службы США (USGS). 

Результат также переведите в гектары и округлите до десятков.