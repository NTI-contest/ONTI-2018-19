\assignementTitle{Геообработка и пространственный анализ. Более-менее надёжная охрана малонарушенных лесов Амазонии}{2}{}

В условиях предыдущей Задачи определите, какая доля (в процентах) общей площади малонарушенных лесных территорий 
(МЛТ; \url{http://borealforestplatform.org/ru/intact_forest_landscapes/}, Intact forest landscapes; \url{https://en.wikipedia.org/wiki/Intact_forest_landscape}) 
Бразилии охраняется в пределах существующих особо охраняемых природных территорий (ООПТ; \url{https://en.wikipedia.org/wiki/Protected_areas_of_Brazil}), если учитывать ООПТ ТОЛЬКО тех категорий, которые имеют (хотя бы де юре) комплексный режим охраны, направленный на сохранение природных экосистем без вмешательства человека (proteção integral). Не учитывайте при этом категории ООПТ, которые направлены на так называемое "устойчивое использование" лесов и других природных ресурсов, а также индейские резервации (Terras Indígenas).

Территории, включенные в список Всемирного природного наследия ЮНЕСКО (World heritage sites), и водно-болотные угодья международного значения (Wetlands of international Importance), включенные в список Рамсарской конвенции, также не учитывайте в качестве территорий с комплексной охраной: их статус на национальном уровне всё равно обеспечивается конкретными ООПТ, территории которых они в себя включают.

Ответ представьте в процентах с округлением до десятых долей процента. 