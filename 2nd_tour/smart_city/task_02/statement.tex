\assignementTitle{}{5}{}

Текущая дата на городских табло представляется в виде последовательности из 4 цифр. Однако табло 
производят две разные фирмы, каждая из которых использует свой формат даты: ДДММ или ММДД, где ММ - 
номер месяца (от 01 до 12), \linebreak а ДД - номер дня месяца (от 01 до 31) с ведущими нулями. Гарантируется, что дата хотя бы в одном формате указана верно.

От вас требуется определить, можно ли по заданной последовательности из 4 цифр однозначно определить сегодняшнюю дату.

\inputfmtSection

В единственной строке дано четырёхзначное число, представляющее дату в одном из описанных форматов.

\outputfmtSection

YES, если дату можно определить однозначно, и NO в противном случае.

\sampleTitle{1}

\begin{myverbbox}[\small]{\vinput}
    3112
\end{myverbbox}
\begin{myverbbox}[\small]{\voutput}
    YES
\end{myverbbox}
\inputoutputTable

\sampleTitle{2}

\begin{myverbbox}[\small]{\vinput}
    1011
\end{myverbbox}
\begin{myverbbox}[\small]{\voutput}
    NO
\end{myverbbox}
\inputoutputTable

\includeSolutionIfExistsByPath{2nd_tour/smart_city/task_01}