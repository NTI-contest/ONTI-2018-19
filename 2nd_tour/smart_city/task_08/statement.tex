\assignementTitle{}{15}

При первоначальной разработке топологии сети передачи данных между различными системами умного города архитектор очень сильно болел, вследствие чего сеть была спроектирована не оптимально. Со временем, когда к сети было подключено большое количество управляющих систем, недостатки проектировки стали сильно влиять на скорость передачи данных. Для полной реорганизации сети потребовалось бы огромное количество ресурсов, поэтому администрацией было принято решение исправлять то, что уже есть.

В ходе долгих обсуждений эксперты пришли к выводу, что каждый участок сети должен быть организован таким образом, чтобы в нём содержалось ровно N «треугольных связей». Под «треугольными связями» эксперты подразумевают произвольную тройку узлов, между каждой парой из которых имеется связь. Например, на рисунке таких треугольных связей 3.

\putImgWOCaption{7cm}{1}

В результате некоторых исследований различных вариантов эксплуатации треугольных связей была выявлена закономерность, что когда количество узлов, внутри которых устанавливается ровно N треугольных связей, минимально, сеть работает с наибольшей пропускной способностью. При этом вся сеть должна быть связной.

Ваша задача по точно заданному числу треугольных связей N определить минимальное количество узлов M, необходимых для обеспечения этого количества связей.

\inputfmtSection

В единственной
строке дано целое число $ N $
($ 1 \le N \le 10^5 $).

\outputfmtSection

Выведите
единственное число $ M $.

\begin{myverbbox}[\small]{\vinput}
    1
\end{myverbbox}
\begin{myverbbox}[\small]{\voutput}
    3
\end{myverbbox}
\inputoutputTable

\begin{myverbbox}[\small]{\vinput}
    3
\end{myverbbox}
\begin{myverbbox}[\small]{\voutput}
    5
\end{myverbbox}
\inputoutputTable

%\includeSolutionIfExistsByPath{2nd_tour/smart_city/task_03}