\assignementTitle{}{15}

При первоначальной разработке топологии сети передачи данных между различными системами умного города архитектор очень сильно болел, вследствие чего сеть была спроектирована не оптимально. Со временем, когда к сети было подключено большое количество управляющих систем, недостатки проектировки стали сильно влиять на скорость передачи данных. Для полной реорганизации сети потребовалось бы огромное количество ресурсов, поэтому администрацией было принято решение исправлять то, что уже есть.

В ходе долгих обсуждений эксперты пришли к выводу, что каждый участок сети должен быть организован таким образом, чтобы в нём содержалось ровно N «треугольных связей». Под «треугольными связями» эксперты подразумевают произвольную тройку узлов, между каждой парой из которых имеется связь. Например, на рисунке таких треугольных связей 3.

\putImgWOCaption{7cm}{1}

В результате некоторых исследований различных вариантов эксплуатации треугольных связей была выявлена закономерность, что когда количество узлов, внутри которых устанавливается ровно N треугольных связей, минимально, сеть работает с наибольшей пропускной способностью. При этом вся сеть должна быть связной.

Ваша задача по точно заданному числу треугольных связей N определить минимальное количество узлов M, необходимых для обеспечения этого количества связей.

\inputfmtSection

В единственной
строке дано целое число $ N $
($ 1 \le N \le 10^5 $).

\outputfmtSection

Выведите
единственное число $ M $.

\sampleTitle{1}

\begin{myverbbox}[\small]{\vinput}
    1
\end{myverbbox}
\begin{myverbbox}[\small]{\voutput}
    3
\end{myverbbox}
\inputoutputTable

\sampleTitle{2}

\begin{myverbbox}[\small]{\vinput}
    3
\end{myverbbox}
\begin{myverbbox}[\small]{\voutput}
    5
\end{myverbbox}
\inputoutputTable

\solutionSection

Минимальное число узлов, которое может обеспечить хотя бы одну треугольную связь, – три. На трёх узлах можно построить ровно одну треугольную связь, соединив каждый узел с остальными двумя. При добавлении четвёртого узла и соединении его с двумя из уже имеющихся, образуется ещё одна треугольная связь (итого 2). Протянув последнюю недостающую связь между узлами, получаем ещё 2 новых треугольных связи (итого 4). Обратим внимание, что ровно три треугольных связи на четырёх узлах построить невозможно, несмотря на то, что можно построить 4 таких связи.

В общем случае при добавлении каждого последующего узла и поочередного соединения вновь добавленного узла с остальными каждая каждое очередное соединение (начиная со второго) будет создавать на одну треугольную связь больше, чем предыдущее (т.е. добавив первые два соединения создаётся одна треугольная связь, три соединения – ещё плюс две связи, четыре соединения – ещё плюс три связи и т.д.).

Таким образом задачу можно решить простым итерированием, постепенным добавлением по одному узлу и по одному ребру до тех пор, пока очередное ребро не создаст слишком много треугольных связей. В этот момент добавляется ещё один узел, для которого тоже начинаем создавать соединения.

\includeSolutionIfExistsByPath{2nd_tour/smart_city/task_08}