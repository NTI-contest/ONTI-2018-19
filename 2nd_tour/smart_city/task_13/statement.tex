\assignementTitle{}{20}{}

Егор – программист, настраивающий систему освещения умным городом. Иногда Егору бывает скучно на работе, и он придумывает себе различные алгоритмические задачки. Однажды Егор выписал на лист бумаги N строчек, каждая из которых состояла из M символов «a» и «b». Символы чередовались в произвольном порядке. Егор заметил, что в полученной табличке не было ни одного столбца, полностью состоящего из букв «a». Ему стало интересно, какое минимальное суммарное количество циклических сдвигов строк потребуется, чтобы в таблице появился хотя бы один такой столбец. Каждая строка сдвигается независимо от всех остальных.

Под циклическим сдвигом понимается смещение всех символов строки на одну позицию вправо или влево, при этом при сдвиге вправо последний символ исходной строки переносится на место первого, а при сдвиге влево – первый на место последнего.

Сдвиг вправо «bbbaba» $ \rightarrow $ «abbbab».

Сдвиг влево «bbbaba» $ \rightarrow $ «bbabab».

\inputfmtSection

В первой строке
даны два числа $ N $ и $ M $
($ 1 \le N \le 100; 1 \le M \le 10^4 $). Следующие
$ N $ строк
содержат строки длиной $ M $ состоящие из символов «a» и «b».

\outputfmtSection

Выведите
единственное число $ K $ – минимальное число циклических сдвигов, при котором хотя бы
один из столбцов таблицы будет состоять только из символов «a». Если такое состояние получить невозможно,
выведите «-1».

\begin{myverbbox}[\small]{\vinput}
    3 6
    ababab
    bbbabb
    babbbb
\end{myverbbox}
\begin{myverbbox}[\small]{\voutput}
    2
\end{myverbbox}
\inputoutputTable

\begin{myverbbox}[\small]{\vinput}
    3 3
    aba
    bbb
    bba
\end{myverbbox}
\begin{myverbbox}[\small]{\voutput}
    -1
\end{myverbbox}
\inputoutputTable

%\includeSolutionIfExistsByPath{2nd_tour/smart_city/task_03}