\assignementTitle{}{10}{}

Электрон влетает в область пространства с однородным магнитным полем (смотри рисунок) перпендикулярно линиям индукции магнитного поля.

\putImgWOCaption{7cm}{1}

Индукция магнитного поля равна $ 10^{-3} $ Тл, скорость электрона $ V=3\cdot10^6 $м/с. 
Расстояние $ L =10 $ см. Определите угол (в градусах) между векторами скорости влёта и вылета из 
области пространства с магнитным полем если начальная скорость была перпендикулярна границе этой области.

\putImgWOCaption{9cm}{2}

\answerMath{180.}