\assignementTitle{}{10}{}

Электрон влетает в область пространства с однородным магнитным полем (смотри рисунок) перпендикулярно линиям индукции магнитного поля.

\putImgWOCaption{7cm}{1}

Индукция магнитного поля равна $ 10^{-3} $ Тл, скорость электрона $ V=3\cdot10^6 $м/с. 
Расстояние $ L =10 $ см. Найдите величину изменение импульса электрона за время нахождения в магнитном поле (ответ представить в кг$\cdot$м/с на $10^{-24}$ , округлить до десятых).
\putImgWOCaption{9cm}{2}

\solutionSection

Изменение импульса электрона:

$$\Delta \overrightarrow{p} = \overrightarrow{p}_2 - \overrightarrow{p}_1 $$

В проекциях на направление оси:  
$$ \Delta p = -p_2 - p_1$$

Сила Лоренца перпендикулярна вектору скорости и не меняет 
его модуль, поэтому:

$$p_2 = p_1$$

$$|\Delta p| = 2p=2mV$$

$$\Delta p = 2mV$$

$$\Delta p = 5.466 \cdot 10^{-24} \: \text{кг} \cdot \text{м/с} \: \text{на} \: 10^{-24}$$
 
\answerMath{5.5 кг $\cdot$ м/с на $10^{-24}$}
