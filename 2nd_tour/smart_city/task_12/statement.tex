\assignementTitle{}{5}{}

Для украшения интерьера и снятия стресса вам подарили механическую систему, 
представляющую собой стальные шарики одинакового радиуса, подвешенные так, что 
находятся на одной прямой, касаясь друг друга (смотри рисунок).

\putImgWOCaption{7cm}{1}

В инструкции сообщается, что массы шариков разные: начиная с первого, каждый следующий имеет 
половину массы предыдущего (240г, 120г, 60г, 30г, 15г). Вы отклонили первый шарик в сторону так, 
что нить натянута и образует 30 градусов с вертикалью. Затем вы отпустили шарик с 
нулевой начальной скоростью.

Найдите кинетическую энергию (в джоулях) последнего шарика после удара. Удары считать абсолютно упругими. Длина нити 50см. 
Ответы представить с точностью до десятых. Ускорение свободного падения считать равным 10~м/с.

\answerMath{0.6.}