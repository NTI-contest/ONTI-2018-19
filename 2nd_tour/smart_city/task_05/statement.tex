\assignementTitle{}{15}{}

Злоумышленник сунулся в ваш умный дом, но сработала сигнализация, и человеку пришлось спешно ретироваться. Ультразвуковой датчик над дверью измеряет скорость удаляющегося субъекта. Частота сигнала датчика 40.00 КГц. Определите скорость субъекта, если отражённый от него сигнал приходит на частоте 39.60 КГц. Скорость звука равна 340 м/с.

Ответ дать в м/с с точностью до десятых.

\solutionSection

Датчик работает на эффекте Доплера. Частота отражённого 
сигнала $v$:
$$v=\nu_0 \frac{v-u2_x}{v-u1_x} \: \: (1)$$

Здесь $\nu_0$ – частота исходного сигнала, $v$ – скорость звука, 
$u1_x$ – скорость источника по отношению к среде 
распространения сигнала, $u2_x$ – скорость движения приёмника. Сначала источник покоится, 
приёмник удаляется. После отражения уже источник 
(отражённый сигнал) удаляется, приёмник покоится. 
Применяя эту формулу два раза (1) два раза, найдём частоту 
сигнала, фиксируемого приёмником ультразвукового датчика:
$$v=\nu_0 \frac{v-u}{v+u}$$

Выразим отсюда скорость u объекта:
$$u=v \frac{\nu_0-v}{v+\nu_0}$$

И подставим значения частот и скорости звука. Получим с 
точностью до десятых долей:

$$u=1.7 \: \text{м/с}.$$

\answerMath{1.7 м/с.}