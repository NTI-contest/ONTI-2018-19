\assignementTitle{}{10}{}

Электрическая схема приведена на рисунке. Приборы считать идеальными. Определите показания вольтметра, 
если на выводы цепи подано напряжение $ 9 $ В, \linebreak а $ r=90 $ Ом.

\putImgWOCaption{7cm}{1}

\putImgWOCaption{9cm}{2}

\solutionSection

Вольтметр идеальный. Следовательно, ток через вольтметр равен нулю. Амперметр идеальный – 
является перемычкой в сбалансированном мосте. Значит через амперметр ток не течёт, показания амперметра $I_A=0$.

Сопротивление верхней цепи $18r$, нижней $6r$. Отличаются в три раза, следовательно, и токи в верхней и 
нижней ветвях отличаются в три раза.

Общее сопротивление равно

$$R=\frac{18r \cdot 6r}{18r+6r}=\frac{9}{2} r.$$

Полный ток по закону Ома:

$$I=\frac{U}{R}=\frac{2U}{9r}=\frac{3U}{18r}+\frac{1U}{18r}.$$

Падения напряжения на резисторе $5r$ равно (ток равен одной четвёртой от общего):

$$U_5=\frac{1}{4} I \cdot 5r=\frac{U}{18 r} 5r=\frac{5U}{18}=2,5 \: \text{В}.$$

Падения напряжения на резисторе $r$ равно (ток равен трём четвёртым от общего):

$$U_5=\frac{3}{4} I \cdot r=\frac{3U}{18}=1.5 \: \text{В}.$$

Значит показание вольтметра

$$U_V=U_5-U_1=1 \text{В}.$$

\answerMath{1.}