\assignementTitle{}{5}{}

Навигационный отдел умного города тестируют новые маршруты общественного транспорта. 
Однако в базе данных имеется информация только о количестве минут, необходимых на проезд 
от текущей остановки до следующей. Отделу необходимо знать время проезда от одной до любой 
другой произвольной остановки. Для этого они посылают в базу данных запрос, представляющий 
собой пару чисел $ A $ и $ B $ – номера остановок одного маршрута, время в пути между которыми 
они хотят знать.
Ваша задача – написать программу, которая будет отвечать на такие запросы.

\inputfmtSection

В первой строке дано два целых числа $ N $ и $ M $ – количество остановок исследуемого 
маршрута и количество запросов соответственно ($2 \le N \le 1000000,$ \linebreak $1 \le M \le 1000000 $).

Во второй строке дано $ N-1 $ целых чисел $ t_i $ – время в пути между остановками с 
номерами $ i $ и $ i+1 $ ($ 0 \le t_i \le 2^{31}-1 $).

Остановки нумеруются по порядку с $ 1 $ до $ N $.

В следующих $ M $ строках даны пары целых чисел $ A$ и $ B $ – номера остановок, 
между которыми необходимо определить время пути маршрутного транспорта \linebreak ($1\le A, B \le N $). 

Гарантируется, что полное время, затраченное на проезд с начальной до конечной остановок, не превышает $ 2^{31}-1 $.

\outputfmtSection

В качестве ответа выведите M строк, содержащих ответы на запросы.

\begin{myverbbox}[\small]{\vinput}
    5 4
    3 4 1 3
    1 2
    1 3
    1 5
    2 4
\end{myverbbox}
\begin{myverbbox}[\small]{\voutput}
    3
    7
    11
    5
\end{myverbbox}
\inputoutputTable

%\includeSolutionIfExistsByPath{2nd_tour/smart_city/task_03}