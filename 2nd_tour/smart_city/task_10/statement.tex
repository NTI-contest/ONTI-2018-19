\assignementTitle{}{10}{}

Для украшения интерьера и снятия стресса вам подарили механическую систему, 
представляющую собой стальные шарики одинакового радиуса, подвешенные так, что 
находятся на одной прямой, касаясь друг друга (смотри рисунок).

\putImgWOCaption{7cm}{1}

В инструкции сообщается, что массы шариков разные: начиная с первого, каждый следующий имеет 
половину массы предыдущего (240~г, 120~г, 60~г, 30~г, 15~г). Вы отклонили первый шарик в сторону так, 
что нить натянута и образует 30~градусов с вертикалью. Затем вы отпустили шарик с 
нулевой начальной скоростью.

Определите скорость (в~м/с) первого шарика в нижней точке траектории. Удары считать абсолютно упругими. 
Длина нити 50~см. Ответ представить с точностью до десятых. Ускорение свободного падения считать равным 10~м/с.

\solutionSection

По закону сохранения полной механической энергии для первого шарика:
$$\frac{mv^2}{2} = mgh = mgl (1 - \cos \alpha).$$

Выразим скорость:

$$v \rightarrow \sqrt{2} \sqrt{g l (1 - \cos \alpha)} = 1.157 \: \text{м/с} \approx 1.2 \: \text{м/с}.$$

\answerMath{1.2 м/с.}