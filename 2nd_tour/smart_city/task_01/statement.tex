\assignementTitle{}{5}{}

Камеры
видеонаблюдения в умном городе записывают видео в чересстрочном (интерлейсном)
формате, при чём, для экономии занимаемого объёма памяти цветовая составляющая
каждого кадра отбрасывается, и каждое изображение записывается на носитель в градациях
серого с глубиной 8 бит. Таким образом, каждый пиксель видеопоследовательности
представляется одним числом в диапазоне от 0 (полностью чёрный) до 255
(полностью белый). Каждый кадр видеопоследовательности представляет собой
матрицу пикселей с шириной $ W $ и высотой $ H $
пикселей, при чём высота изображения $ H $ является чётной.

Интерлейсный
формат представления видео подразумевает, что каждый кадр размера $ W\times H $ на самом деле хранит два
последовательно идущих кадра таким образом, что чётные строки содержат значения
яркостей пикселей чётных строк кадра с номером $ 2K $, а нечётные содержат значения яркостей
пикселей нечётных строк кадра с номером $ 2K+1 $ ($ K $ – целое число).

Для
воспроизведения чересстрочного видео плеер должен уметь выполнить процесс
деинтерлейса, то есть получить из одного интерлейсного кадра размера $  W\times H  $ 
два кадра размера $ W\times H $. В простейшем случае этот
процесс можно разбить на следующие этапы

\begin{itemize}
    \item разбиение исходного чересстрочного кадра на два
    полукадра размером \linebreak $ W\times(\frac{H}{2}) $, каждый из которых
    содержит только пиксели одного из двух исходных кадров;
    \item восстановление недостающих строк путём
    усреднения значений двух соседних (сверху и снизу) пикселей. Если у
    восстанавливаемого пикселя отсутствует верхний или нижний сосед, значение этого
    пикселя копируется из имеющегося соседа. Так как все значения пикселей являются
    целыми числами, при усреднении дробная часть отбрасывается.
\end{itemize}

Ваша задача –
написать описанный деинтерлейсер.

\inputfmtSection

В первой строке
дано два целых числа $ W $ и $H$  ($16 \le W \le 1920, 16 \le H \le 1080$, $H$ -  чётное).

В следующих $ H $ строках содержится по $ W $ чисел $ p_{i,j}$ ($ 0 \le p_{i,j} \le 255 $), каждое из которых представляет значение
$j$-го пикселя $ i $-ой строки.

\outputfmtSection

Выведите $ H $ строк по $ W $ значений в каждой – значения пикселей
восстановленных  строк. Первые $ \frac{H}{2} $ строк должны представлять значения пикселей
первого кадра, вторые $ \frac{H}{2} $ строк должны представлять значения пикселей
второго кадра.

\solutionSection

Для решения задачи необходимо сделать ровно то, что сказано в условиях:

Разбить исходный чересстрочный кадра на два полукадра размером $W \times (H/2)$, каждый из которых содержит 
только пиксели одного из двух исходных кадров; Восстановить недостающие строки путём усреднения 
значений двух соседних (сверху и снизу) пикселей. Если у восстанавливаемого пикселя отсутствует верхний 
или нижний сосед, значение этого пикселя копируется из имеющегося соседа. Так как все значения пикселей 
являются целыми числами, при усреднении дробная часть отбрасывается.

\includeSolutionIfExistsByPath{2nd_tour/smart_city/task_01}