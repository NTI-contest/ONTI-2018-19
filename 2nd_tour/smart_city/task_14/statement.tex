\assignementTitle{}{30}{}

Данная задача является исследовательской, поэтому ограничения на количество попыток нет.

Мы взяли картинку размером $1920\times1080$, некоторым алгоритмом уменьшили её в два раза по ширине 
и в два раза по высоте, получили картинку размером $ 960\times540 $. 

Ваша задача растянуть картинку $ 960\times540 $ обратно до $ 1920\times1080 $ таким образом, 
чтобы разница между оригинальной картинкой и вашей растянутой была минимальной. Под разницей 
будем понимать средний квадрат разности $ D=\dfrac{\sum\limits_{i=1}^{w\times h} (p_{org}[i] - p_{rec}[i])^2}{w\times h},$ 
где $ p_{org}[i]$ - значения пикселей оригинальной картинки, $ p_{rec}[i] $ - значения пикселей восстановленной вами картинки, $ w $ и $ h $
- ширина и высота изображения (1920 и 1080 в данном случае).

\inputfmtSection

В первой строке выведите два числа $ w $ и $ h $ - ширину и высоту полученного вами изображения 
(1920 и 1080). В следующей строке выведите закодированное в Base64 изображение в градациях серого.

\outputfmtSection

Аналогично формату входных данных. В первой строке выведите два числа $ w $ и $ h $ - ширину и высоту 
полученного вами изображения (1920 и 1080). В следующей строке выведите закодированное в 
Base64 (\url{https://ru.wikipedia.org/wiki/Base64}) изображение в градациях серого (\url{https://ru.wikipedia.org/wiki/%D0%9E%D1%82%D1%82%D0%B5%D0%BD%D0%BA%D0%B8_%D1%81%D0%B5%D1%80%D0%BE%D0%B3%D0%BE}).

Примечание: В качестве примера входных и выходных данных предлагаем вам скачать файлы. sample\_input.txt  - 
входной файл\\
(\url{https://drive.google.com/file/d/1dloLs5pcLKvVybrCFEG5y6E_ahI5_J_J/view}), \\
sample\_output.txt - выходной\\ (\url{https://drive.google.com/file/d/107sLUq9s_kAQqZfuz6W_JKOOFCjwRqcL/view}) \\
Обращаем ваше внимание, что в примере размер большой картинки $720 \times 480$, 
а уменьшенной $360\times 240$ соответственно. Кроме того, что sample\_output.txt является 
корректным примером выхода вашей программы, он так же является оригиналом картинки уменьшенной до $360\times 240$ тем же алгоритмом, 
что и изображение из задания.

Входное изображение во всех тестах одинаковое. Каждый тест засчитывается, если величина $D$, подсчитанная на основании разности 
выхода вашей программы и оригинального изображения не превосходит некоторую величину погрешности $\epsilon$. Тесты упорядочены в порядке 
убывания $\varepsilon$. Величина $\varepsilon$ в первом тесте соответствует величине $D$ для метода ближайшего соседа.

\solutionSection

Задача является исследовательской и идеального решения не имеет. Для получения большего количества баллов за задачу необходимо реализовать хороший метод двумерной интерполяции. 