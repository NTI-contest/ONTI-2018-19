\assignementTitle{}{5}{}

Для украшения интерьера и снятия стресса вам подарили механическую систему, 
представляющую собой стальные шарики одинакового радиуса, подвешенные так, что 
находятся на одной прямой, касаясь друг друга (смотри рисунок).

\putImgWOCaption{7cm}{1}

В инструкции сообщается, что массы шариков разные: начиная с первого, каждый следующий имеет 
половину массы предыдущего (240~г, 120~г, 60~г, 30~г, 15~г). Вы отклонили первый шарик в сторону так, 
что нить натянута и образует 30 градусов с вертикалью. Затем вы отпустили шарик с 
нулевой начальной скоростью.

Найдите скорость (в~м/с) последнего шарика после удара. Удары считать абсолютно упругими. Длина нити 50~см. 
Ответы представить с точностью до десятых. Ускорение свободного падения считать равным 10~м/с.

\solutionSection

По закону сохранения полной механической энергии для первого шарика:
$$\frac{mv^2}{2} = mgh = mgl (1 - \cos \alpha).$$

Выразим скорость:

$$v \rightarrow \sqrt{2} \sqrt{g l (1 - \cos \alpha)} = 1.157 \: \text{м/с} \approx 1.2 \: \text{м/с}.$$

Масса второго шара $m_2=m_1⁄2$, $n$-ого – $m_n=m_1⁄2^{n-1}$. Из законов сохранения импульса и 
полной механической энергии для абсолютно упругого удара двух шаров, скорость $\overrightarrow{u}_n $ 
каждого из них после удара равна $\overrightarrow{u}_n=2 \overrightarrow{v}_c- \overrightarrow{v}_n$. 
Скорость $\overrightarrow{v}_n$ – скорость до удара. Поскольку исходно вся цепочка шаров покоилась, 
то скорость $v_n=0$ и
$$u_n=2v_c=2 \frac{m_{n-1} v_{n-1}}{m_{n-1}+m_n}=2 \frac{v_{n-1}}{1+m_n⁄m_{n-1}}=\frac{2}{1+1⁄2} v_{n-1}=\frac{4}{3} v_{n-1}.$$
Таким образом скорость второго шарика после взаимодействия с первым $u_2=\frac{4}{3} v_1$, где $v_1=1.157$ м/с. 
Cкорость третьего шарика после взаимодействия со вторым 
$$u_3=\frac{4}{3} u_2=\left(\frac{4}{3}\right)^2 v_1,$$
для n-ого шарика
$$u_n=\left(\frac{4}{3}\right)^{n-1}v_1.$$
Искомая скорость пятого шарика
$$u_5=\left(\frac{4}{3}\right)^4 v_1=3.658 \: \text{м/с} \approx 3.7 \: \text{м/с}.$$

Ответ 3.8 тоже считать верным, но давать меньше баллов. Дело в том, что если здесь подставить округлённое число скорости первого шарика 1.2, то ответ для скорости пятого будет отличаться на одну десятую. 

То есть ошибка – если округляют до вычислений.

\answerMath{3.7 м/с.}