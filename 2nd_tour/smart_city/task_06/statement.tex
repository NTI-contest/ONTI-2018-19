\assignementTitle{}{10}{}

Для обеспечения
безопасности все данные между любыми устройствами умного города передаются в
зашифрованном виде. Мы не будем вдаваться в подробности используемого метода
шифрования, скажем лишь, что для успешного дешифрования сообщения необходимо
сформировать минимальное возможное число, удовлетворяющее следующим условиям:

\begin{itemize}
    \item в десятичной записи числа есть только цифры 3 и 9;
    \item в десятичной записи числа есть ровно $A$  цифр 3;
    \item в десятичной записи числа есть ровно $B$  цифр 9;
    \item в десятичной записи числа есть ровно $C$  вхождений подстроки «39»;
    \item в десятичной записи числа есть ровно $D$  вхождений подстроки «93».
\end{itemize}

Ваша задача
написать программу, которая по заданным значениям $ A $, $ B $, $ C $ и $ D $ находит такое число.

\inputfmtSection

В единственной
строке даны 4 целых числа $ A $,
$ B $,
$ C $ и $ D $ ($ 1 \le A, B, C, D \le 10^6 $).

\outputfmtSection

В единственной
строке выведите минимальное число, удовлетворяющее заданным условиям. Если
такого числа не существует, выведите «$ -1 $».

\begin{myverbbox}[\small]{\vinput}
    3 2 2 1
\end{myverbbox}
\begin{myverbbox}[\small]{\voutput}
    33939
\end{myverbbox}
\inputoutputTable

\begin{myverbbox}[\small]{\vinput}
    5 4 4 1
\end{myverbbox}
\begin{myverbbox}[\small]{\voutput}
    -1
\end{myverbbox}
\inputoutputTable

%\includeSolutionIfExistsByPath{2nd_tour/smart_city/task_03}