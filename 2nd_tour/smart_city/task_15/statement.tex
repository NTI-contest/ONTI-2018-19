\assignementTitle{}{10}{}

Электрон влетает в область пространства с однородным магнитным полем (смотри рисунок) перпендикулярно линиям индукции магнитного поля.

\putImgWOCaption{7cm}{1}

Индукция магнитного поля равна $ 10^{-3} $ Тл, скорость электрона $ V=3\cdot10^6 $м/с. 
Расстояние $ L =10 $ см. Определите угол (в градусах) между векторами скорости влёта и вылета из 
области пространства с магнитным полем если начальная скорость была перпендикулярна границе этой области.

\putImgWOCaption{9cm}{2}

\solutionSection

На каждый движущийся заряд со стороны магнитного поля действует сила Лоренца:

$$F = q \cdot V \cdot B \cdot \sin (\overrightarrow{V} \overrightarrow{B})$$

Радиус найдем из уравнения Ньютона:

$F = ma$

$F = qVB$ - сила Лоренца, $a$ – центростремительное ускорение
 
$$qVB = m \frac{V^2}{R}$$

$$R = \frac{mV}{aB} = 0.0171 \: \text{м}$$

\noindent это меньше чем ширина области магнитного поля, следовательно электрон не выйдет за пределы поля с противоположной 
стороны от точки влёта, будет двигаться по полуокружности в м.п. и выйдет в обратную сторону, также под углом $90$ градусов к границе 
области м.п.

Следовательно, угол между скоростями влёта и вылета: $\alpha = 180^\circ$.

\answerMath{$180^\circ$.}