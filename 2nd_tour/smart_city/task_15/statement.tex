\assignementTitle{}{10}{}

Электрон влетает в область пространства с однородным магнитным полем (смотри рисунок) перпендикулярно линиям индукции магнитного поля.

\putImgWOCaption{7cm}{1}

Индукция магнитного поля равна $ 10^{-3} $ Тл, скорость электрона $ V=3\cdot10^6 $м/с. 
Расстояние $ L =10 $ см. Найдите величину изменение импульса электрона за время нахождения в магнитном поле (ответ представить в кг*м/с на $10^{-24}$ , округлить до десятых).
\putImgWOCaption{9cm}{2}

\answerMath{180.}