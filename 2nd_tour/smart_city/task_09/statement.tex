\assignementTitle{}{15}

Вы ремонтируете некое устройство в вашем доме. Видите, что оно представляет собой схему, изображенную на рисунке.

\putImgWOCaption{7cm}{1}

Вам необходимо узнать, какой заряд протечёт через резистор $ R $ после одновременного замыкания ключей 
$ K_1 $ и $ K_2 $. До замыкания конденсаторы $ C_1 = 1\text{мкФ}$, $ C_3 = 3\text{мкФ}$ не заряжены, 
а конденсатор $ C_2 = 2\text{мкФ}$ заряжен до разности потенциалов $ U_0 = 1\text{В}$. ЭДС батареи $ 6\text{В} $. 

Ответ дать в микрокулонах с точностью до десятых.

\putImgWOCaption{7cm}{2}

\solutionSection

После замыкания ключей конденсатор $C_2$ разряжается через $R$, а $C_1$ 
и $C_3$ заряжаются зарядом $q_1$:

$$\frac{q_1}{C_1} + \frac{q_1}{C_3} = \epsilon \rightarrow q_1 = \frac{\mathcal{E}C_1C_3}{C_1+C_3} +C_2U_0$$
 
После подстановки численных данных величина заряда, 
прошедшего через резистор будет равна $6.5$ мк Кулон.


\answerMath{6.5 мкКл.}