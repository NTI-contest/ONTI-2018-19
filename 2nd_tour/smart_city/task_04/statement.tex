\assignementTitle{}{15}{}

Элемент поршневого соединения в механизме погрузки-разгрузки имеет вид как на рисунке. 
Площади сечения цилиндрических труб $  S_1 = 25 \text{см}^2 $ и $ S_2 = 5 \text{см}^2 $. 
Поршни соединены жёстким стержнем. Вначале давление во всех частях системы равно атмосферному 
$ p_0 = 101 \text{кПа} $, а объём между поршнями равен $ V_0=0.5\text{м}^3 $. 
На сколько процентов нужно поднять давление в цилиндре меньшего диаметра 
(правая часть системы), чтобы поршни сместились на $ 5\text{см} $. Давление в левой 
части системы остаётся прежним. Трения нет, температура в системе постоянна.

\putImgWOCaption{7cm}{1}

Ответ дать в процентах с точностью до сотых.

\solutionSection

После смещения поршня влево, внутренний объём изменится на
$$\Delta V=xS_1-xS_2$$

Поскольку температура и масса газа между поршнями неизменна, 
то 

$$p_0 V_0=p_2 (V_0+\Delta V)=p_2 (V_0+x(S_1-S_2)), \: (2.1)$$
где $p_2$ – новое давление между поршнями. Из условия равновесия поршня:
$$p_0 S_1-p_2 S_1-px S_2+p_2 S_2=0 \:(2.2)$$

Решая совместно уравнения (2.1, 2.2) относительно $px$ и $p_2$ получим:
$$p_2 \rightarrow \frac{p_0 V_0}{xS_1-xS_2+V_0}, \: px \rightarrow \frac{p_0 (-xS_1^2+xS_1 S_2-S_2 V_0)}{S_2 (-xS_1+xS_2-V_0)}$$
$$px\rightarrow 101080.8,$$
$$p_2\rightarrow 100979.8$$
$$100 \frac{p_0-px}{p_0} =0.08\%$$

\answerMath{0.08 \%.}