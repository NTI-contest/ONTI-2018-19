\assignementTitle{}{15}{}

Элемент поршневого соединения в механизме погрузки-разгрузки имеет вид как на рисунке. 
Площади сечения цилиндрических труб $  S_1 = 25 \text{см}^2 $ и $ S_2 = 5 \text{см}^2 $. 
Поршни соединены жёстким стержнем. Вначале давление во всех частях системы равно атмосферному 
$ p_0 = 101 \text{кПа} $, а объём между поршнями равен $ V_0=0.5\text{м}^3 $. 
На сколько процентов нужно поднять давление в цилиндре меньшего диаметра 
(правая часть системы), чтобы поршни сместились на $ 5\text{см} $. Давление в левой 
части системы остаётся прежним. Трения нет, температура в системе постоянна.

\putImgWOCaption{7cm}{1}

Ответ дать в процентах с точностью до сотых.

\answerMath{0.08.}