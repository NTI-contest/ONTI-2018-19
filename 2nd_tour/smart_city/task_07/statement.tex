\assignementTitle{}{10}{}

Хороший робот-пылесос должен уметь убираться во всем доме. Пылесос 
Ёмкий Трудолюбивый Роботизированный (П.Ё.Т.Р.) как раз из таких. 
Однако он намного умнее своих собратьев: он умеет заряжаться до такого состояния, 
чтобы его заряда хватило на уборку всей квартиры. Для этого ему необходимо знать площадь, 
на которой требуется провести уборку. Благо, на этот случай у него есть план квартиры и 
место нахождения самого П.Ё.Т.Р.а. Помогите П.Ё.Т.Р.у найти площадь на которой он может навести порядок.

\inputfmtSection

В первой строке даны два целых числа $ W $ и $ H $ – размеры квартиры

В следующих $ H $ строках содержится $ W $ символов, описывающих квартиру: 

'\#' - стена

'.' - пол

'@' - пылесос П.Ё.Т.Р. 

\outputfmtSection

Число клеток, которые П.Ё.Т.Р. может убрать в квартире.

Пылесос в начальной точке всегда стоит на полу, и эта клетка тоже считается необходимой для уборки.

Гарантируется, что пылесос всегда находится внутри квартиры.

Квартира всегда является замкнутым контуром.

Пылесос не умеет ходить по диагонали.

\sampleTitle{1}

\begin{myverbbox}[\small]{\vinput}
    5 5
    #####
    #...#
    #.@.#
    #...#
    #####
\end{myverbbox}
\begin{myverbbox}[\small]{\voutput}
    9
\end{myverbbox}
\inputoutputTable

\sampleTitle{2}

\begin{myverbbox}[\small]{\vinput}
    8 11
    ........
    ........
    .######.
    .#....#.
    .###.##.
    .#....#.
    .##.@.#.
    .#....#.
    .######.
    ....#.#.
    ....###.
\end{myverbbox}
\begin{myverbbox}[\small]{\voutput}
    16
\end{myverbbox}
\inputoutputTable

\solutionSection

\includeSolutionIfExistsByPath{2nd_tour/smart_city/task_07}