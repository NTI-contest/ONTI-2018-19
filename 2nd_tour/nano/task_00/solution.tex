\subsubsection*{Пример успешного прохождения Этапа 1}

\begin{enumerate}
    \item Переходим во вкладку «Проекты», начинаем с моделирования дисплея.
    \item Дисплей на основе квантовых точек -> «Начать новый проект»
    \item В графе «Анализ требований» по умолчанию выставлено значение «Эффективности конверсии» - 20\%. Итоговый прототип дисплея должен иметь значение не меньше установленного.
    \item В графе «Параметры продукта» начинаем вводить параметры квантовых точек, которые необходимо будет получить:
    
    \begin{table}[H]
        \center
        \begin{tabular}{|p{3.5cm}|p{2.8cm}|p{2.8cm}|p{2.8cm}|}
            \hline
            &Квант. точки 1	&Квант. точки 2	&Квант. точки 3 \\
            \hline
            Длина волны люминесценции (нм)	&460 (синие)	&550 (зеленые)	&640 (красные) \\
            \hline
            Масса квантовых точек (г)	&3.1	&3.1	&3.1 \\
            \hline
            Квантовый выход, не менее (\%)	&35&	35&	35 \\
            \hline
            Стабильность, не менее (\%)&	80&	80&	80 \\
            \hline
            Планируемый состав квантовых точек&	CdS&CdSe&	CdTe \\
            \hline
            Токсичность, не более (\%)	&95	&95&	95 \\
            \hline
        \end{tabular}
    \end{table}

    Так как стабильность и токсичность не влияют на качество дисплеев, то удобно использовать соединения с ионом кадмия. Сульфид имеет наибольшую ширину запрещенной зоны, потому для него проще получить наночастицы с синей люминесценцией, а CdSe и CdTe отлично пойду т на роль источников зеленой и красной люминесценции. Все три соединения имеют достаточно высокий квантовый выход – по 40\%. Стабильность определяется соотношением реагентов, 80\% достичь не сложно. Токсичность выбрана высокой, чтобы удалось достичь этого значения в синтезе.
    \item В графе «Выбор параметров опытного образца» случайным образом предлагаются «Данные для конструирования», в нашем случае: Диагональ дисплея (дюйм) – 19, Соотношение сторон – 16:9, Количество дисплеев в партии (шт.)~– 60. Они позволяют вычислить массу квантовых точек для партии дисплеев.
    
    Рассчитаем площадь дисплея: $(16x)^2+(9x)^2 = (19)^2$, тогда $x = 1.07$ дюйма, тогда площадь равна: $S = 16x \cdot 9x = 165.25$ дюйма$^2$ или 1032.75 см$^2$
    
    Оптимальная высота слоя квантовых точек – 500 нм или $5 \cdot 10^{-5}$ см

    Объем слоя наночастиц: 0.052 см$^3$

    Масса слоя наночастиц: 0.155 г

    Масса на партию из 60 дисплеев: 9.3 г

    Известно, что дисплей состоит из наночастиц трех цветов в равном соотношении по массе, т.е. масса наночастиц с синей люминесценцией, равна с зеленой, равна с синей, равна 3.1 г.

    \item Нажимаем «Создать опытный образец», тратим 10 000 \textpeso
    \item Итог: «Успешно создан прототип дисплея», эффективность конверсии~– 58.1\%.
    
    Потери в качестве произошли из-за недостаточно высокого квантового выхода наночастиц (меньше максимально возможного 60\%), получается коэффициент $35/60 = 0.58$, который определил качество дисплеев

    \item Нажимаем «Отправить в производство» и начинаем синтезировать квантовые точки с заявленными параметрами. Начнем с синих, нажимаем в графе «Синтез квантовых точек 1» кнопку «Создать компонент с указанными требованиями» и попадаем в проект «Синтез квантовых точек».
    \item В «Целевых параметрах» автоматически вносятся заявленные ранее параметры квантовых точек. Дальнейшая задача – синтезировать квантовые точки с данными параметрами.
    \item Введенные параметры синтеза:
    
    \begin{longtable}{|p{8cm}|p{6cm}|}
        \hline
        Исходное вещество для введения аниона. Класс соединения:	& простые вещества \\
        \hline
        Выбор соединения	& сера-порошок \\
        \hline
        Масса (г.) & 	0.6 \\
        \hline
        Исходное вещество для введения катиона. Класс соединения:	& органическая соль \\
        \hline
        Выбор соединения:	& ацетат кадмия двухводный \\
        \hline
        Масса (г.)	& 5.32 \\
        \hline
        Добавление вещества для введения второго катиона	& нет \\
        \hline
        Выбор растворителя. Класс соединения:	& Органические соединения \\
        \hline
        Выбор соединения:	& 1-октадецен \\
        \hline
        Объём (мл.)& 	50 \\
        \hline
        Выбор поверхностно-активного вещества. Класс соединения:	& Органические соединения \\
        \hline
        Выбор соединения:	Олеиновая кислота
        Объём (мл.)	& 24.6 \\
        \hline
        Добавление дополнительного соединения: &	Да \\
        \hline
        Класс соединения:	& Органические соединения \\
        \hline
        Выбор соединения: & Три-н-октилфосфин \\
        \hline
        Объём (мл.)	& 12 \\
        \hline
        Температура синтеза ($^\circ C$) & 150 \\
        \hline
        Время синтеза (сек.) & 220 \\
        \hline
    \end{longtable}

    Количество вещества серы должно быть на 10\% меньше, чем кадмия для создания катион-избыточной поверхности наночастиц. Массы выбраны таким образом, чтобы масса полученных квантовых точек CdS была не менее 3.1 г. Объем растворителя был рассчитан исходя из оптимальной концентрации раствора – 0.4 М. Олеиновая кислота берется в четырехкратном избытке по молям по отношению к кадмию, чтобы получить олеат кадмия (нужен двукратный избыток) и нарастить оболочку нанчоастиц (также двукратный). Количество три-н-октилфосфина берется в таком количестве, чтобы можно было растворить серу в нем и получить соединение TOPS (три-н-октилфосфин сульфид) с концентрацией 1 М.

    Время синтеза и температура подбираются для получения наночастиц с оптимальной длиной волны люминесценции.

    \item В итоге были получены наночастицы со следующими параметрами:
    
    \begin{table}[H]
        \begin{center}
            \begin{tabular}{|p{9cm}|p{4cm}|}
                \hline
                Результат & Успешный синтез \\
                \hline
                Длина волны излучения КТ (нм) & 459 \\
                \hline
                Квантовый выход (\%) & 38.5 \\
                \hline
                При отправке продукта с заданными  параметрами в производство будет  получено квантовых точек (г)&	3.44 \\
                \hline
                При отправке продукта с заданными  параметрами в производство стоимость реагентов составит (руб)&	4560 \\
                \hline
                Состав наночастиц	&CdS \\
                \hline
                Стабильность (\%)&	90 \\
                \hline
                Токсичность (\%)&	70 \\
                \hline
            \end{tabular}
        \end{center}
    \end{table}

    \item Далее нажимаем на кнопку «Отправить в производство» и «Запустить в производство»
    \item Качество продукта составляет 100\%, что связано с полным соответствием исходным параметрам, продукт помещается на склад и уже приносит 1 балл к победным очкам.
    \item Заходим во вкладку «проекты», ищем синтезированные квантовые точки в разделе «Синтез квантовых точек» и нажимаем на кнопку «Привлечь инвестиции», выбираем «квантовые точки с синей люминесценцией» и получаем 100 000 \textpeso, так как квантовые точки имеют максимальное качество.
    \item Теперь нажимаем на кнопку «Производство» в разделе «Дисплей на основе квантовых точек» и нажимаем «Создать компонент с указанными требованиями в «Синтез квантовых точек 2».
    \item Далее все происходит аналогично описанному выше, мы просто приведем введенные значения:
    
    \begin{table}[H]
        \begin{center}
            \begin{tabular}{|p{9cm}|p{4cm}|}
                \hline
                Исходное вещество для введения аниона. Класс соединения:	&простые вещества\\
                \hline
                Выбор соединения	&селен-порошок\\
                \hline
                Масса (г.)	&1.4\\
                \hline
                Исходное вещество для введения катиона. Класс соединения:	&органическая соль\\
                \hline
                Выбор соединения:	&ацетат кадмия двухводный\\
                \hline
                Масса (г.)	&5.32\\
                \hline
                Добавление вещества для введения второго катиона	&нет\\
                \hline
                Выбор растворителя. Класс соединения:&	Органические соединения\\
                \hline
                Выбор соединения:	&1-октадецен\\
                \hline
                Объём (мл.)	&50\\
                \hline
                Выбор поверхностно-активного вещества. Класс соединения:	&Органические соединения \\
                \hline
                Выбор соединения:	&Олеиновая кислота \\
                \hline
                Объём (мл.)	&24.6 \\
                \hline
                Добавление дополнительного соединения:	&Да \\
                \hline
                Класс соединения:	&Органические соединения \\
                \hline
                Выбор соединения:	&Три-н-октилфосфин \\
                \hline
                Объём (мл.)	&12 \\
                \hline
                Температура синтеза ($^\circ C$)	&170 \\
                \hline
                Время синтеза (сек.)	&240 \\
                \hline
            \end{tabular}
        \end{center}
    \end{table}

    и результат:

    \begin{table}[H]
        \begin{center}
            \begin{tabular}{|p{9cm}|p{4cm}|}
                \hline
                Результат	&Успешный синтез \\
                \hline
                Длина волны излучения КТ (нм)	&549 \\
                \hline
                Квантовый выход (\%)	&38 \\
                \hline
                При отправке продукта с заданными параметрами в производство будет получено квантовых точек (г)&	4.57 \\
                \hline
                При отправке продукта с заданными параметрами в производство стоимость реагентов составит (руб)&	4764 \\
                \hline
                Состав наночастиц	&CdSe \\
                \hline
                Стабильность (\%)&	90 \\
                \hline
                Токсичность (\%)&	80 \\
                \hline
            \end{tabular}
        \end{center}
    \end{table}

    \item После получения инвестиций можно приступать к синтезу квантовых точек с красной люминесценцией:
    
    \begin{table}[H]
        \begin{center}
            \begin{tabular}{|p{9cm}|p{4cm}|}
                \hline
                Исходное вещество для введения аниона. Класс соединения:	&простые вещества \\
                \hline
                Выбор соединения	&Теллур-порошок \\
                \hline
                Масса (г.)&	2.3 \\
                \hline
                Исходное вещество для введения катиона. Класс соединения:&	органическая соль \\
                \hline
                Выбор соединения:	&ацетат кадмия двухводный \\
                \hline
                Масса (г.)&	5.32 \\
                \hline
                Добавление вещества для введения второго катиона	&нет \\
                \hline
                Выбор растворителя. Класс соединения:&	Органические соединения \\
                \hline
                Выбор соединения:	&1-октадецен \\
                \hline
                Объём (мл.)&	50 \\
                \hline
                Выбор поверхностно-активного вещества. Класс соединения:	&Органические соединения \\
                \hline
                Выбор соединения:	&Олеиновая кислота \\
                \hline
                Объём (мл.)	&24.6 \\
                \hline
                Добавление дополнительного соединения:&	Да \\
                \hline
                Класс соединения:	&Органические соединения \\
                \hline
                Выбор соединения:	&Три-н-октилфосфин \\
                \hline
                Объём (мл.)	12 \\
                \hline
                Температура синтеза ($^\circ C$)	&110 \\
                \hline
                Время синтеза (сек.)&	190 \\
                \hline
            \end{tabular}
        \end{center}
    \end{table}
    
    и результат:

    \begin{table}[H]
        \begin{center}
            \begin{tabular}{|p{9cm}|p{4cm}|}
                \hline
                Результат	&Успешный синтез \\
                \hline
                Длина волны излучения КТ (нм)&	639 \\
                \hline
                Квантовый выход (\%)	&38.2 \\
                \hline
                При отправке продукта с заданными параметрами в производство будет получено квантовых точек (г)	&5.7 \\
                \hline
                При отправке продукта с заданными параметрами в производство стоимость реагентов составит (руб)&	4619 \\
                \hline
                Состав наночастиц&	CdSe \\
                \hline
                Стабильность (\%)	&90 \\
                \hline
                Токсичность (\%)&	90 \\
                \hline
            \end{tabular}
        \end{center}
    \end{table}

    \item В результате все наночастицы имеют качество 100\%. Проверяем, что инвестиции получены на все цвета квантовых точек. Переходим в «Проекты», «Дисплей на основе квантовых точек», нажимаем на кнопку «Производство», выбираем из выпадающих меню синтезированные квантовые точки и нажимаем «Запустить в производство».
    \item Видим «Произведена партия продукта» с качеством 58.1\%. Большее значение качества можно было получить с использованием перовскитных квантовых точек состава $CsPbHal_3$, так как они имеют рекордное значение максимального квантового выхода – 60\%.
    \item Рассчитаем итоговое количество баллов:
    
    Были изготовлены продукты 1 уровня – три цвета качеством по 100\%. Используя эти продукты в качестве компонентов дисплеев, был создан продукт 2 уровня качеством 58.1\%, тогда

    Команда получает $1 \cdot 1.0 + 1 \cdot 1.0 + 1 \cdot 1.0 + 2 \cdot 0.58 = 4.16$ победных очка (без учета баллов за количество оставшихся денег).
\end{enumerate}

\subsubsection*{Пример успешного прохождения Этапа 2}

\begin{enumerate}
    \item Переходим во вкладку «Проекты», начинаем с моделирования солнечной батареи.
    \item Солнечные батареи -> «Начать новый проект»
    \item В графе «Анализ требований» по умолчанию выставлено значение «Эффективности конверсии солнечного излучения, не менее (\%)» 50 \%, а «Эффективности разделения зарядов, не менее (\%)» 40\%. Итоговый прототип солнечной батареи должен иметь значения не меньше установленных.
    \item В графе «Параметры продукта» начинаем выбирать слои для конструирования солнечной батареи, для чего используем энергетическую диаграмму:
    
    \putImgWOCaption{10cm}{12}

    \item Катод – всегда ITO в нашей задаче. Квантовые точки должны иметь ширину запрещенной зоны, позволяющей излучать на длине волны более 600 нм, подходят $CdTe$, $InP$, $PbS$ и $PbSe$. Однако анализ энергетических диаграмм показывает, что лучше всего для применения в солнечных батареях подходит $InP$.
    
    У $InP$ не очень высокое расположение границы запрещенной зоны в связи с чем удобно осуществлять транспорт носителей к ITO, а проводящий полимер PEDOT:~PSS позволит провести блокировку электронов и транспорт носителей. В то же время следует выбрать медь в качестве материала анода из-за небольшой энергетической ступеньки.
    
    $SnO_2$ подходит в качестве слоя для транспорта электронов к катоду.
    
    Масса квантовых точек определяется параметрами солнечной ячейки. В нашем случае значение высоты и ширины солнечной батареи составили 100 и 80 см соответственно. С учетом оптимальной толщины слоя наночастиц – 500 нм получим массу квантовых точек:

    $\text{Объем слоя} = 100\cdot 80\cdot 5 \cdot 10^{-5} = 0.4 \: \text{см}^3$. С учетом плотности наночастиц 3~г/см$^3$, получим: $m = 0.4 \: \text{см}^3 \cdot 3 \: \text{г/см}^3 = 1.2 г$

    \item В итоге, в окне «Выбор параметров опытного образца» были введены следующие параметры:
    
    \begin{table}[H]
        \begin{center}
            \begin{tabular}{|p{9cm}|p{4cm}|}
                \hline
                \multicolumn{2}{|c|}{Параметры продукта} \\
                \hline
                Материал катодного слоя	&ITO \\
                \hline
                Материал ETL	&SnO2 \\
                \hline
                Объем раствора для нанесения ETL (мл)&	10.15 \\
                \hline
                Материал HTL	&PEDOT:PSS \\
                \hline
                Объем раствора для нанесения HTL (мл)&	20 \\
                \hline
                Материал анодного слоя&	Cu \\
                \hline
                \multicolumn{2}{|c|}{Синтез квантовых точек 1} \\
                \hline
                Длина волны люминесценции (нм)&	750 \\
                \hline
                Масса квантовых точек (г)	&1.2 \\
                \hline
                Квантовый выход, не менее (\%)&	10 \\
                \hline
                Стабильность, не менее (\%)&	90 \\
                \hline
                Планируемый состав квантовых точек	&InP \\
                \hline
                Токсичность, не более (\%)&	90 \\
                \hline
            \end{tabular}
        \end{center}
    \end{table}

    Итог:

    \begin{table}[H]
        \begin{center}
            \begin{tabular}{|p{9cm}|p{6cm}|}
                \hline
                \multicolumn{2}{|c|}{Результаты опытных испытаний} \\
                \hline
                Результат	&Успешно создан прототип солнечной батареи. \\
                \hline
                Эффект конверсии солнечного излучения (\%)	&84.4 \\
                \hline
                Эффективность разделения зарядов (\%)&	80 \\
                \hline
            \end{tabular}
        \end{center}
    \end{table}

    Потеря в эффективности разделения ввиду высоких энергетических барьеров между слоями.
    \item Нажимаем «Отправить в производство» и «Создать компонент с указанными требованиями»
    \item В разделе «Целевые параметры» автоматически вводятся исходные параметры квантовых точек, в «Параметры продукта» вводим следующее:
    
    \begin{table}[H]
        \begin{center}
            \begin{tabular}{|p{9cm}|p{5cm}|}
                \hline
                Исходное вещество для введения аниона. Класс соединения:&	Органические соединения \\
                \hline
                Выбор соединения	&Трис(триметилсилил) фосфин \\
                \hline
                Масса (г.)&	1.8 \\
                \hline
                Исходное вещество для введения катиона. Класс соединения:&	Неорганические соли и оксиды \\
                \hline
                Выбор соединения:&	Хлорид индия \\
                \hline
                Масса (г.)	&1.76 \\
                \hline
                Добавление вещества для введения второго катиона	&нет \\
                \hline
                Выбор растворителя. Класс соединения:	&Органические соединения \\
                \hline
                Выбор соединения:&	1-октадецен \\
                \hline
                Объём (мл.)	&15 \\
                \hline
                Выбор поверхностно-активного вещества. Класс соединения:	&Органические соединения \\
                \hline
                Выбор соединения:	&Олеиновая кислота \\
                \hline
                Объём (мл.)	&20 \\
                \hline
                Добавление дополнительного соединения:	&Да \\
                \hline
                Класс соединения:	&Органические соединения \\
                \hline
                Выбор соединения: &Три-н-октилфосфин \\
                \hline
                Объём (мл.)	&5 \\
                \hline
                Температура синтеза ($^\circ C$)&	300 \\
                \hline
                Время синтеза (сек.)	&2000 \\
                \hline
            \end{tabular}
        \end{center}
    \end{table}
    
    Итог:
    \begin{table}[H]
        \begin{center}
            \begin{tabular}{|p{9cm}|p{4cm}|}
                \hline
                Результат	&Успешный синтез \\
                \hline
                Длина волны излучения КТ (нм)	&750 \\
                \hline
                Квантовый выход (\%)&	19.7 \\
                \hline
                При отправке продукта с заданными параметрами в производство будет получено квантовых точек (г)&	1.4 \\
                \hline
                При отправке продукта с заданными параметрами в производство стоимость реагентов составит (руб)&	8146 \\
                \hline
                Состав наночастиц&	InP \\
                \hline
                Стабильность (\%)&	90 \\
                \hline
                Токсичность (\%)	&40 \\
                \hline
            \end{tabular}
        \end{center}
    \end{table}

    \item Нажимаем «Отправить в производство» и «Запустить в производство»
    \item Заходим в производство солнечной батареи, выбираем в выпадающем меню произведенные квантовые точки и нажимаем «Запустить в производство»
    
    В итоге имеем Солнечную батарею с качеством 67.5\%.
    \item Рассчитаем итоговое количество баллов:
    
    Был изготовлен продукт 1 уровня с качеством по 100\%. Используя этот продукт в качестве компонента солнечной батареи, был создан продукт 2 уровня качеством 67.5\%, тогда

    Команда получает $1 \cdot 1.0 + 2 \cdot 0.68 (+4.16) = 6.52$ победных очка (без учета баллов за количество оставшихся денег, с учетом баллов за 1 Этап).
\end{enumerate}

\subsubsection*{Пример успешного прохождения Этапа 3}

\begin{enumerate}
    \item Переходим во вкладку «Проекты», начинаем с моделирования биометки.
    \item Биометки -> «Начать новый проект»
    \item В графе «Анализ требований» по умолчанию выставлено значение «Биосовместимость, не менее (\%)» 40 \%, «Относительная яркость излучения, не менее (\%)» 20 \%, а «Селективность диагностики, не менее (\%)» 70 \%. Итоговый прототип биометки должен иметь значения не меньше установленных.
    \item В разделе «Выбор параметров опытного образца» необходимо выбрать строительные блоки антитела. Известно, что наличие двух ветвей селективно воздействующих на различные мишени опухоли повышает селективность диагностики, для этого необходимо иметь как вариабильные, так и константные домены, осуществляющие сшивку двух цепей.
    \item Выбранные параметры при конструировании антитела:
    
    \begin{table}[H]
        \begin{center}
            \begin{tabular}{|p{9cm}|p{4cm}|}
                \hline
                \multicolumn{2}{|l|}{Параметры продукта} \\
                \hline
                \multicolumn{2}{|c|}{Ветвь 1} \\
                \hline
                Добавление легкой цепи 1&да \\
                \hline
                Мишень для 1VL&CSPG4 \\
                \hline
                $1CL_1$	&да \\
                \hline
                Добавление тяжелой цепи 1&да\\
                \hline
                Мишень для 1VH&CSPG4 \\
                \hline
                $1CH_1$ & да \\
                \hline
                $1CH_2$ & да \\
                \hline
                $1CH_3$ & нет \\
                \hline
                \multicolumn{2}{|c|}{Ветвь 2} \\
                \hline
                Добавление легкой цепи 1&	да \\
                \hline
                Мишень для 1VL	&MART-1 \\
                \hline
                $1CL_1$&	да \\
                \hline
                Добавление тяжелой цепи 1	&да \\
                \hline
                Мишень для 1VH	&MART-1 \\
                \hline
                $1CH_1$	&да \\
                \hline
                $1CH_2$	&да \\
                \hline
                $1CH_3$	&нет \\
                \hline
                Время до проведения люминесцентного анализа (мин.)	&150 \\
                \hline
                \multicolumn{2}{|c|}{Синтез квантовых точек 1} \\
                \hline
                Длина волны люминесценции (нм)&	1300 \\
                \hline
                Масса квантовых точек (г)&	0.68 \\
                \hline
                Квантовый выход, не менее (\%)&	25 \\
                \hline
                Стабильность, не менее (\%)	&90 \\
                \hline
                Планируемый состав квантовых точек	&PbS \\
                \hline
                Токсичность, не более (\%)&	50 \\
                \hline
            \end{tabular}
        \end{center}
    \end{table}
    
    Итог:

    \begin{table}[H]
        \begin{center}
            \begin{tabular}{|p{5cm}|p{8cm}|}
                \hline
                Результат	& Успешно создан прототип биомаркера. \\
                \hline
                Биосовместимость (\%)	& 67 \\
                \hline
                Относительная яркость излучения (\%)	& 86.2 \\
                \hline
                Селективность диагностики (\%)	& 100 \\
                \hline
            \end{tabular}
        \end{center}
    \end{table}

    \item Нажимаем «Отправить в производство» и «Создать компонент с указанными требованиями»
    \item В разделе «Целевые параметры» автоматически вводятся исходные параметры квантовых точек, в «Параметры продукта» вводим следующее:
    
    \begin{table}[H]
        \begin{center}
            \begin{tabular}{|p{9cm}|p{5cm}|}
                \hline
                Исходное вещество для введения аниона. Класс соединения:	&простые вещества \\
                \hline
                Выбор соединения	&сера-порошок \\
                \hline
                Масса (г.)&	0.6 \\
                \hline
                Исходное вещество для введения катиона. Класс соединения:&	органическая соль \\
                \hline
                Выбор соединения:	&ацетат свинца трехводный \\
                \hline
                Масса (г.)&	7.6 \\
                \hline
                Добавление вещества для введения второго катиона	&нет \\
                \hline
                Выбор растворителя. Класс соединения:	&Органические соединения \\
                \hline
                Выбор соединения:	&1-октадецен \\
                \hline
                Объём (мл.)	& 50 \\
                \hline
                Выбор поверхностно-активного вещества. Класс соединения:	&Органические соединения \\
                \hline
                Выбор соединения:&	Олеиновая кислота \\
                \hline
                Объём (мл.)&	24.6 \\
                \hline
                Добавление дополнительного соединения:	&Да \\
                \hline
                Класс соединения:	&Органические соединения \\
                \hline
                Выбор соединения:	&Три-н-октилфосфин \\
                \hline
                Объём (мл.)&	12 \\
                \hline
                Температура синтеза (°С)	&200 \\
                \hline
                Время синтеза (сек.)&	220 \\
                \hline
            \end{tabular}
        \end{center}
    \end{table}

Итог:

    \begin{longtable}{|p{9cm}|p{4cm}|}
        \hline
        Результат	&Успешный синтез \\
        \hline
        Длина волны излучения КТ (нм)&	1296 \\
        \hline
        Квантовый выход (\%)&	28.9 \\
        \hline
        При отправке продукта с заданными параметрами в производство будет получено квантовых точек (г)&	5.7 \\
        \hline
        При отправке продукта с заданными параметрами в производство стоимость реагентов составит (руб)&	4637 \\
        \hline
        Состав наночастиц&	PbS \\
        \hline
        Стабильность (\%)&	90 \\
        \hline
        Токсичность (\%)&	50 \\
        \hline
    \end{longtable}

    \item Нажимаем «Отправить в производство» и «Запустить в производство»
    \item Заходим в производство биометок, выбираем в выпадающем меню произведенные квантовые точки и нажимаем «Запустить в производство»
    
    В итоге имеем Биометки с качеством 52\%.
    \item Рассчитаем итоговое количество баллов:
    
    Был изготовлен продукт 1 уровня с качеством по 100\%. Используя этот продукт в качестве компонента солнечной батареи, был создан продукт 2 уровня качеством 52\%, тогда

    Команда получает $1 \cdot 1.0 + 2 \cdot 0.52 (+6.52) = 8.56$ победных очка (без учета баллов за количество оставшихся денег, с учетом баллов за 1 и 2 Этапы).
\end{enumerate}