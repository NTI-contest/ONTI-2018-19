\subsubsection*{Пояснительная записка по синтезу квантовых точек}

\underline{Составы успешных синтезов}

В таблице приведены наборы реактивов для синтеза квантовых точек заданного состава. Под каждым реагентом написана цифра, отражающая соотношение реагентов. Например, для соединения CdS соотношение S:Cd:OLA:TOP = 0.9:1:4:2 по молям.

\begin{longtable}{|p{2cm}|p{2.5cm}|p{3cm}|p{3cm}|p{1.5cm}|p{2cm}|}
    \hline
    Состав КТ & Исходное вещество для введения аниона & Исходное вещество для введения катиона & Исходное вещество для введения второго катиона & ПАВ & Доп. соединение \\ 
    \hline
     & \cellcolor[HTML]{C0C0C0}S & \cellcolor[HTML]{C0C0C0}$Cd(Ac)_2 \cdot 2H_2O$ & \cellcolor[HTML]{C0C0C0} & \cellcolor[HTML]{C0C0C0}OLA & \cellcolor[HTML]{C0C0C0}TOP \\ \cline{2-3} \cline{5-6} 
     & 0.9 & 1 & \multirow{-2}{*}{\cellcolor[HTML]{C0C0C0}-} & 4 & 2 \\ 
     \cline{2-6} 
     & \cellcolor[HTML]{C0C0C0}S & \cellcolor[HTML]{C0C0C0}$CdCl_2$ & \cellcolor[HTML]{C0C0C0} & \cellcolor[HTML]{C0C0C0}OLA & \cellcolor[HTML]{C0C0C0}TOP \\ 
     \cline{2-3} \cline{5-6} 
    \multirow{-4}{*}{CdS} & 0.9 & 1 & \multirow{-2}{*}{\cellcolor[HTML]{C0C0C0}-} & 2 & 4 \\ 
    \hline

    & \cellcolor[HTML]{C0C0C0}Se & \cellcolor[HTML]{C0C0C0}$Cd(Ac)_2 \cdot 2H_2O$ & \cellcolor[HTML]{C0C0C0} & \cellcolor[HTML]{C0C0C0}OLA & \cellcolor[HTML]{C0C0C0}TOP \\ \cline{2-3} \cline{5-6} 
     & 0.9 & 1 & \multirow{-2}{*}{\cellcolor[HTML]{C0C0C0}-} & 4 & 2 \\ 
     \cline{2-6} 
     & \cellcolor[HTML]{C0C0C0}Se & \cellcolor[HTML]{C0C0C0}$CdCl_2$ & \cellcolor[HTML]{C0C0C0} & \cellcolor[HTML]{C0C0C0}OLA & \cellcolor[HTML]{C0C0C0}TOP \\ 
     \cline{2-3} \cline{5-6} 
    \multirow{-4}{*}{CdSe} & 0.9 & 1 & \multirow{-2}{*}{\cellcolor[HTML]{C0C0C0}-} & 2 & 4 \\ 
    \hline

    & \cellcolor[HTML]{C0C0C0}Te & \cellcolor[HTML]{C0C0C0}$Cd(Ac)_2 \cdot 2H_2O$ & \cellcolor[HTML]{C0C0C0} & \cellcolor[HTML]{C0C0C0}OLA & \cellcolor[HTML]{C0C0C0}TOP \\ \cline{2-3} \cline{5-6} 
     & 0.9 & 1 & \multirow{-2}{*}{\cellcolor[HTML]{C0C0C0}-} & 4 & 2 \\ 
     \cline{2-6} 
     & \cellcolor[HTML]{C0C0C0}Te & \cellcolor[HTML]{C0C0C0}$CdCl_2$ & \cellcolor[HTML]{C0C0C0} & \cellcolor[HTML]{C0C0C0}OLA & \cellcolor[HTML]{C0C0C0}TOP \\ 
     \cline{2-3} \cline{5-6} 
    \multirow{-4}{*}{CdTe} & 0.9 & 1 & \multirow{-2}{*}{\cellcolor[HTML]{C0C0C0}-} & 2 & 4 \\ 
    \hline

    & \cellcolor[HTML]{C0C0C0}S & \cellcolor[HTML]{C0C0C0}$Pb(Ac)_2 \cdot 3H_2O$ & \cellcolor[HTML]{C0C0C0} & \cellcolor[HTML]{C0C0C0}OLA & \cellcolor[HTML]{C0C0C0}TOP \\ \cline{2-3} \cline{5-6} 
    \multirow{-2}{*}{PbS} & 0.9 & 1 & \multirow{-2}{*}{\cellcolor[HTML]{C0C0C0}-} & 4 & 2 \\ \hline

    & \cellcolor[HTML]{C0C0C0}S & \cellcolor[HTML]{C0C0C0}$Pb(Ac)_2 \cdot 3H_2O$ & \cellcolor[HTML]{C0C0C0} & \cellcolor[HTML]{C0C0C0}OLA & \cellcolor[HTML]{C0C0C0}TOP \\ \cline{2-3} \cline{5-6} 
    \multirow{-2}{*}{PbSe} & 0.9 & 1 & \multirow{-2}{*}{\cellcolor[HTML]{C0C0C0}-} & 4 & 2 \\ \hline

    & \cellcolor[HTML]{C0C0C0}Se & \cellcolor[HTML]{C0C0C0}$Zn(Ac)_2 \cdot 2H_2O$ & \cellcolor[HTML]{C0C0C0} & \cellcolor[HTML]{C0C0C0}OLA & \cellcolor[HTML]{C0C0C0}TOP \\ \cline{2-3} \cline{5-6} 
     & 0.9 & 1 & \multirow{-2}{*}{\cellcolor[HTML]{C0C0C0}-} & 4 & 2 \\ 
     \cline{2-6} 
     & \cellcolor[HTML]{C0C0C0}Se & \cellcolor[HTML]{C0C0C0}$ZnCl_2$ & \cellcolor[HTML]{C0C0C0} & \cellcolor[HTML]{C0C0C0}OLA & \cellcolor[HTML]{C0C0C0}TOP \\ 
     \cline{2-3} \cline{5-6} 
    \multirow{-4}{*}{ZnSe} & 0.9 & 1 & \multirow{-2}{*}{\cellcolor[HTML]{C0C0C0}-} & 2 & 4 \\ 
    \hline

    & \cellcolor[HTML]{C0C0C0}$(TMS)_3P$ & \cellcolor[HTML]{C0C0C0}$In(Ac)_3 \cdot 2H_2O$ & \cellcolor[HTML]{C0C0C0} & \cellcolor[HTML]{C0C0C0}OLA & \cellcolor[HTML]{C0C0C0}TOP \\ \cline{2-3} \cline{5-6} 
     & 0.9 & 1 & \multirow{-2}{*}{\cellcolor[HTML]{C0C0C0}-} & 4 & 2 \\ 
     \cline{2-6} 
     & \cellcolor[HTML]{C0C0C0}$(TMS)_3P$ & \cellcolor[HTML]{C0C0C0}$InCl_3$ & \cellcolor[HTML]{C0C0C0} & \cellcolor[HTML]{C0C0C0}OLA & \cellcolor[HTML]{C0C0C0}TOP \\ 
     \cline{2-3} \cline{5-6} 
    \multirow{-4}{*}{InP} & 0.9 & 1 & \multirow{-2}{*}{\cellcolor[HTML]{C0C0C0}-} & 2 & 4 \\ 
    \hline

    & \cellcolor[HTML]{C0C0C0}S & \cellcolor[HTML]{C0C0C0}$Cu(Ac)_2$ & \cellcolor[HTML]{C0C0C0}$In(Ac)_3 \cdot 2H_2O$ & \cellcolor[HTML]{C0C0C0}OLA & \cellcolor[HTML]{C0C0C0}TOP \\ \cline{2-6} 
    & 1.8 & 1 & 1 & 8 & 4 \\ \cline{2-6} 
    & \cellcolor[HTML]{C0C0C0}S & \cellcolor[HTML]{C0C0C0}$Cu(Ac)_2$ & \cellcolor[HTML]{C0C0C0}$InCl_3$ & \cellcolor[HTML]{C0C0C0}OLA & \cellcolor[HTML]{C0C0C0}TOP \\ \cline{2-6} \multirow{-4}{*}{$CuInS_2$} & 1.8 & 1 & 1 & 6 & 6 \\ \hline

    & \cellcolor[HTML]{C0C0C0}$NH_4Cl$ & \cellcolor[HTML]{C0C0C0}$Cs_2CO_3$ & \cellcolor[HTML]{C0C0C0}$PbO$ & \cellcolor[HTML]{C0C0C0}OLA & \cellcolor[HTML]{C0C0C0}TOP \\ \cline{2-6} \multirow{-2}{*}{$CsPbCl_3$} & 2.7 & 0.5 & 1 & 7.5 & 10 \\ \hline

    & \cellcolor[HTML]{C0C0C0}$NH_4Br$ & \cellcolor[HTML]{C0C0C0}$Cs_2CO_3$ & \cellcolor[HTML]{C0C0C0}$PbO$ & \cellcolor[HTML]{C0C0C0}OLA & \cellcolor[HTML]{C0C0C0}TOP \\ \cline{2-6} \multirow{-2}{*}{$CsPbBr_3$} & 2.7 & 0.5 & 1 & 7.5 & 10 \\ \hline

    & \cellcolor[HTML]{C0C0C0}$NH_4I$ & \cellcolor[HTML]{C0C0C0}$Cs_2CO_3$ & \cellcolor[HTML]{C0C0C0}$PbO$ & \cellcolor[HTML]{C0C0C0}OLA & \cellcolor[HTML]{C0C0C0}TOP \\ \cline{2-6} \multirow{-2}{*}{$CsPbI_3$} & 2.7 & 0.5 & 1 & 7.5 & 10 \\ \hline
\end{longtable}

Обратите внимание, что количество веществ для введения анионов во всех синтезах меньше на 10\%, чем это требуется для достижения необходимого соотношения. Это связано с необходимостью создания катион-избыточной поверхности наночастиц для осуществления успешной стабилизации ионами ПАВ.

\begin{longtable}{|p{3cm}|p{4.5cm}|p{4.5cm}|p{3cm}|}
    \hline
    \textbf{Соединение} & \textbf{Название на русском} & \textbf{Название на английском} & \textbf{Сокращение} \\
    \hline
    $Cd(CH_3COO)_2 \cdot 2H_2O$	& Ацетат кадмия двухводный	& Cadmium acetate dihydrate	& $Cd(Ac)_2 \cdot 2H_2O$ \\
    \hline
    $Zn(CH_3COO)_2 \cdot 2H_2O$	& Ацетат цинка двухводный	& Zinc acetate dihydrate	& $Zn(Ac)_2 \cdot 2H_2O$ \\
    \hline
    $(CH_3COO)_3In \cdot 2H_2O$ & Ацетат индия двухводный & Indium acetate dihydrate & $In(Ac)_3 \cdot 2H_2O$ \\
    \hline
    $Pb(CH_3COO)_2 \cdot 3H_2O$ & Ацетат свинца (II) трехводный	 & Lead(II) acetate trihydrate & $Pb(Ac)_2 \cdot 3H_2O$ \\
    \hline
    $Cu(CH_3COO_3)_2$ &	Ацетат меди (II) & Copper(II) acetate & $Cu(Ac)_2$ \\
    \hline
    $CdCl_2$	&Хлорид кадмия	&Cadmium chloride&	--- \\
    \hline
    $ZnCl_2$ &	Хлорид цинка&	Zinc chloride&	--- \\
    \hline
    $InCl_3$	&Хлорид индия&	Indium chloride	&--- \\
    \hline
   $ CuCl$	&Хлорид меди (I)&	Copper(I) chloride&	--- \\
    \hline
    $PbO$&	Окисд свинца&	Lead(II) oxide&	--- \\
    \hline
    $Cs_2CO_3$& 	Карбонат цезия&	Cesium carbonate&	--- \\
    \hline
    $C_{17}H_{33}COOH$&	Олеиновая кислота&	Oleic acid&	OLA \\
    \hline
    $(C_8H_{17})_3P$	&Три-н-октилфосфин	&Tri-n-octylphosphine&	TOP \\
    \hline
    $C_{18}H_{36}$ &	Октадецен-1	&Octadecene&	ODE \\
    \hline
    $C_{18}H_{35}NH_2$&	Олеиламин&	Oleylamine&	OLAm \\
    \hline
    $Se$&	Сера	&Sulfur&	--- \\
    \hline
    $S$&	Селен&	Selenium&	--- \\
    \hline
    $Te$&	Теллур&	Tellurim&	--- \\
    \hline
    $[(CH_3)_3Si]_3P$&	Трис(триметилсилил) фосфин&	tris(trimethylsilyl) phosphane	&$(TMS)_3P$ \\
    \hline
    $NH_4Cl$	&хлорид аммония&	Ammonium chloride&	--- \\
    \hline
    $NH_4Br$	&бромид аммония&	Ammonium bromide&	--- \\
    \hline
    $NH_4I$	&иодид аммония&	Ammonium iodide&	--- \\
    \hline
\end{longtable}

\underline{Определение количества растворителя:}

В качестве растворителя во всех синтезах применяется октадецен (ODE). Концентрация раствора должна быть равна 0.4 моль/л и требуется для расчета количества растворителя. Концентрация раствора рассчитывается по формуле:

$$0.4 \: \frac{\text{моль}}{\text{л}}= \frac{\sum_{i=1}^N \nu_{\text{кат}i)}}{V_\text{растворителя}}$$

Числитель дроби равен сумме по молям всех ($N$ штук) катионов в синтезе.