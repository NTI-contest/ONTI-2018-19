\assignementTitle{Производство биометок}{}

Ваша новая задача -- создать биометки на основе квантовых точек, способные детектировать одну из многих разновидностей раковых заболеваний под названием меланома.

Любой медицинский препарат перед началом массового производства и использования пациентами должен пройти длительную серию строгих клинических испытаний. В процессе создания прототипов биометок вам предстоит контроллировать все параметры квантовых точек, длину волны люминесценции, квантовый выход, стабильность и токсичность частиц. Несоответствие хотя бы одного из параметров предъявляемым критериям может стать причиной отказа от внедрения биометок в массовое производство. По условию данного этапа в зачет идут биометки с биосовместимостью не менее 40\%, относительной яркостью излучения не менее 20\% и селективностью диагностики не менее 70\%. 

Изучите подробнее прилагаемые материалы по данной теме. 

Напоминаем, что все продукты обладают уникальным набором параметров, характеризующий конкретный тип продукта, а также качеством. Параметр качества меняется в диапазоне от 0 до 100\% и показывается насколько точно полученный продукт соответствует требованиям задания. Чем выше качество продукта, тем больше рейтинговых баллов вы получите за его создание. В зачет идут баллы за продукт наивысшего качества из всех созданных командой продуктов в ходе выполнения этапа.

Предлагаемые параметры продукта изображены на следующем экране:
\putImgWOCaption{17cm}{1}