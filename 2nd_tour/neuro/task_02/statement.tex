\assignementTitle{}{10}

Как и в предыдущей задаче, входными данными в этой задаче являются сигналы ЭКГ из архива PhysioBank. Сигнал на входе представляет собой последовательность действительных чисел, разделенных пробелом и лежащих в диапазоне от -0.5 до 2.0. Сигнал был оцифрован с частотой 500 Гц. Частота сердечных сокращений в тестовых данных не превышает 200 ударов в минуту. Высота R-зубца не меньше 0.3.

Вам предлагается рассчитать временной интервал между вершинами соседних R-зубцов. Если R-зубец имеет на вершине несколько точек с максимальным значением, для расчёта интервала берём первую из этих точек. Ответом является последовательность временных интервалов между вершинами соседних R-зубцов в миллисекундах, разделённых пробелами.
Пример входных данных и ответ для данного примера доступен по ссылке:

\begin{itemize}
    \item пример входных данных (\url{https://www.dropbox.com/s/5qe3mk9zsb11l9h/2.txt?dl=1})
    \item ответ для данного примера (\url{https://www.dropbox.com/s/pu72nkzsf847phz/2_ans.txt?dl=1})
\end{itemize}

Для проверки корректности решения используется отрывок из примера с двумя первыми сокращениями сердца:

\putImgWOCaption{16cm}{1}

\newpage

\sampleTitle{1}

\begin{myverbbox}[\small]{\vinput}
    -0.03 -0.035 -0.045 -0.05 -0.06 -0.065 -0.075 -0.08 -0.085 -0.09
    -0.095 -0.095 -0.095 -0.095 -0.095 -0.095 -0.09 -0.085 -0.085
    ...
\end{myverbbox}
\begin{myverbbox}[\small]{\voutput}
   902
\end{myverbbox}
\inputoutputTable

%\includeSolutionIfExistsByPath{2nd_tour/neuro/task_02}