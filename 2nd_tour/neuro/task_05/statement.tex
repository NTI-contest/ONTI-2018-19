\assignementTitle{}{12}{}

Сигналы ЭЭГ, которые предстоит проанализировать в данной задаче, взяты из архива PhysioBank. 
Значения сигнала разделены пробелом и представляют собой последовательность целых чисел, 
лежащих в диапазоне от -500 до 600. Частота оцифровки сигнала составляет 256 Гц.

К некоторым участкам оригинального сигнала были добавлены гармоники, лежащие в диапазоне 
частот альфа-ритма. Длительность изменённых участков составляет от 10 до 130 секунд, 
интервал между участками - от 30 до 210 секунд. Амплитуды и частоты добавленных гармоник 
одинаковы для всех тестовых сигналов.

Определите число участков с увеличенной амплитудой колебаний в диапазоне от 8 до 13 Гц. 
Пример тестового сигнала с 19 участками вы можете скачать по ссылке (\url{https://www.dropbox.com/s/9a7m34r2jff8mnf/2.txt?dl=1}). Данный сигнал подаётся на вход в 
качестве второго теста.

Примечание: излишние колебания некой величины ($x$) может помочь сгладить комплементарный фильтр и введение величины y, на t-ом шаге описываемой формулой:

$y[t] = (1-k)\cdot y[t-1] + k\cdot x[t]$

\sampleTitle{1}

\begin{myverbbox}[\small]{\vinput}
    -47 -38 -36 -14 -3 -9 0 4 23 35 68 52 27 13 -8 
    -25 -32 -33 -33 -19 8 21 22 10 0 0 25 18 20 16 
    10 4 0 -4 -18 -6 15 31 34 54 57 44
    ... 
\end{myverbbox}
\begin{myverbbox}[\small]{\voutput}
   1
\end{myverbbox}
\inputoutputTable

\includeSolutionIfExistsByPath{2nd_tour/neuro/task_05}