\assignementTitle{}{12}{}

Финальная задача данного блока позволит вам познакомиться с очисткой сигнала от шумов. Мы продолжаем работать с сигналом ЭКГ. В данном примере на сигнал электрокардиограммы был наложен шум. Шум состоит из конечного числа гармонических колебаний заданных частот, одинаковых для всех тестовых данных. Подсчитайте число сердечных сокращений записанных в подаваемом на вход сигнале. Обратите внимание, что после фильтрации вид сигнала изменится, и величина максимумов может уменьшиться.

Пример входного сигнала с двадцатью тремя сокращениями сердца доступен по ссылке (\url{https://www.dropbox.com/s/ev1k9o0300mzvlz/3.txt?dl=0}).

Для проверки корректности решения используется отрывок из примера с двумя сокращениями сердца:

\putImgWOCaption{16cm}{1}

\sampleTitle{1}

\begin{myverbbox}[\small]{\vinput}
    -0.11821723252011541 1.135312534894712 -0.21982218822262528 
    -1.1057721743502749 -0.29103421006186825 1.4524076224589373 
    -0.30214821283080107 -0.4971958489822036 0.6206745987216689 
    -0.7334067500591215 0.5065983543323755 0.5871562045492225
    ...
\end{myverbbox}
\begin{myverbbox}[\small]{\voutput}
   2
\end{myverbbox}
\inputoutputTable

\includeSolutionIfExistsByPath{2nd_tour/neuro/task_03}