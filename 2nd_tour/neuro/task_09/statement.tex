\assignementTitle{}{8}{}

В архиве по данной ссылке находятся фотографии 200 разных людей. 
Для каждого человека есть две фотографии: на одной он улыбается, а на другой - нет. Названия фотографий 
с улыбкой оканчиваются 'b', без улыбки - 'a'. Вам требуется написать класс Solver обладающий следующими свойствами:

\begin{itemize}
    \item Конструктор класса должен принимать один аргумент - путь к папке, содержащей фотографии для обучения модели;
    \item Класс должен иметь метод predict, получающий в качестве аргумента путь к фотографии человека, и возвращающий False, если человек на фотографии не улыбается, и True, если улыбается.
\end{itemize}

При проверке файл с вашим кодом импортируется и создаётся экземпляр класса Solver. В качестве аргумента 
конструктор класса получает путь к директории, содержащей 200 случайных фотографий из архива. Названия 
фотографий будут изменены (только нумерация, буквы 'a' и 'b' останутся на месте), поэтому ознакомьтесь с 
работой os.listdir. Среди этих фотографии одинаковое количество изображений с улыбкой и без. Затем в цикле 
вызывается метод predict, которому в качестве аргумента последовательно подаётся путь к каждой из оставшихся 
фотографий. Названия этих фотографий не будут содержать букв  'a' и 'b'. Количество баллов за данное 
задание вычисляется по формуле:

$$16 \cdot \frac{score - 50}{50}$$

где score - процент изображений, для которых верно было определено наличие или отсутствие улыбки.
Пример решения представлен ниже (\url{https://www.dropbox.com/s/dcgqwcp7lzws6sj/solution.py?dl=1}).

\inputminted[fontsize=\footnotesize, linenos]{python}{2nd_tour/neuro/task_09/part_01.py}

В конструкторе класса создаётся поле predictor, в которое загружается предобученная модель. Затем в цикле последовательно перебираются все фотографии из директории, путь к которой был подан конструктору в качестве аргумента. Фотографии по очереди загружаются в переменную image, больше с ними ничего не происходит.
В метод predict всегда возвращает False.

К вашему коду предъявляются следующие требования:

\begin{itemize}
    \item Путь к изображениям должен формироваться с использованием os.path.join. Запрещена конкатенация строк вида
   
    \inputminted[fontsize=\footnotesize, linenos]{python}{2nd_tour/neuro/task_09/part_02.py}
    \item Файл c предобученной моделью для определения ключевых точек хранится в той же директории, в которой будет находится ваш код, т.е. к нему можно обращаться напрямую shape\_predictor\_68\_face\_landmarks.dat. Файл тот же, что и в предыдущей задаче.
    \item Все изображения, которые будут использованы при оценке точности решения, присутствуют в архиве. Может возникнуть соблазн создать структуру вида ключ - значение, где в качестве ключа используется один или несколько параметров изображения, однозначно его идентифицирующих, а в качестве значения - ответ, к какому классу данное изображение относится, и использовать данную структуру в методе predict. Все подобные решения будут оценены на 0 баллов.
    \item Можно использовать любые стандартные библиотеки Python, библиотеки OpenCV, Dlib, Imutils, Scikit-Learn, Numpy, Scipy.
\end{itemize}

Проверить, что ваше решение будет протестировано корректно можно с помощью следующего кода (\url{https://www.dropbox.com/s/msy9dnwe1kg0rk5/test.py?dl=1}):

\inputminted[fontsize=\footnotesize, linenos]{python}{2nd_tour/neuro/task_09/part_03.py}

Предполагается, что файл с вашим кодом называется solution.py и находится с test.py в одной директории. Также в этой директории находится папка с изображениями smile и она содержит изображение 1a.jpg. Если test.py выполняется без ошибок и выводит True или False (в зависимости от того, как хорошо работает ваша программа), то решение будет протестировано корректно.
Ограничение по времени на работу конструктора 5 минут, на работу метода predict - 10 секунд. Решение будет запускаться на CPU Core i5-M450 с 4 Гб оперативной памяти под управлением Ubuntu 16.04.5.

Решение скопируйте в текстовое поле, после чего оно будет протестировано разработчиками Олимпиады. Баллы будут выставлены после проверки.

\includeSolutionIfExistsByPath{2nd_tour/neuro/task_09}