\solutionSection

В данной блока требуется определить какой из двух сигналов был записан при закрытых глазах. Для ответа на этот вопрос сравним спектры данных сигналов. При закрытии глаз у большинства людей на энцефалограмме увеличивается амплитуда гармоник в диапазоне от 8 до 13 Гц.

Спектры исходных сигналов:

\putImgWOCaption{13cm}{2}

Просуммируем амплитуду колебаний в этом диапазоне. Будем считать, что закрытым глазам соответствует сигнал, у которого данная сумма больше.

\subsection*{Приступим к написанию кода}

\begin{minted}[fontsize=\footnotesize, linenos]{python}
    import numpy as np

    def get_spectrum(y):
        Fs = 160.0
        Ts = 1.0/Fs
        n = len(y)
        k = np.arange(n)
        T = n/Fs
        frq = k/T
        frq = frq[range(n//2)]
        Y = np.fft.fft(y)/n
        Y = Y[range(n//2)]
        return frq, abs(Y)

    def get_alpha(y):
        frq, Y = get_spectrum(y)
        alpha = 0
        for freq, ampl in zip(frq, Y):
            if freq > 8 and freq < 13:
                alpha += ampl
        return alpha

    st = [s.split(',') for s in input().split()]
    sig1, sig2 = zip(*st)
    sig1 = [int(s) for s in sig1]
    sig2 = [int(s) for s in sig2]
    alpha1 = get_alpha(sig1)
    alpha2 = get_alpha(sig2)
    print('1' if alpha1 > alpha2 else '2')
\end{minted}

Импортируем библиотеку numpy для вычисления спектра сигнала

Напишем функцию get\_spectrum, которая будет получать в качестве аргумента сигнал, для которого нужно вычислить его спектр.

По условию частота оцифровки сигнала составляет 160 Гц, запишем её в переменную Fs. Период оцифровки запишем в переменную Ts.

В переменную n запишем число значений в обрабатываемом сигнале.

В переменную k запишем массив значений от 0 до n-1. Он нам потребуется для вычисления частоты наших гармоник.

Зная частоту оцифровки, мы можем записать в массив frq частоты нашего сигнала

И затем выполнить дискретное преобразование Фурье. На 10 и 12 строках мы берём только половину значений массивов в силу симметричности получаемого после преобразования спектра.

Затем мы возвращаем два массива - массив частот frq и массив амплитуд Y, причём последний по модулю, так как после преобразования в нём хранятся комплексные числа.

Напишем функцию get\_alpha, которая будет вычислять сумму амплитуд всех частот в спектре сигнала от 8 до 13 Гц. В качестве аргумента данная функция получает сигнал.

В массивы frq и Y сохраним результат работы функции get\_spectrum.

Сумму амплитуд интересующих нас гармоник будем хранить в переменной alpha.

Будем последовательно перебирать элементы массивов frq и Y. На i-й позиции массива frq хранится частота i-й гармоники, а на i-й позиции массива Y хранится её амплитуда. Соответственно, если значение элемента первого массива лежит в диапазоне от 8 до 13, мы будем прибавлять соответствующее ему значение массива амплитуд в переменную alpha.

Таким образом функция будет возвращать амплитуду альфа-ритма.

Теперь остаётся реализовать считывание подаваемых на вход сигналов в разные списки и вычислить амплитуду альфа-ритма для каждого из этих сигналов. В выводим номер сигнала, у которого амплитуда альфа-ритма больше.




