\assignementTitle{}{10}{}

Для проведения различных исследований мозга иногда используют сигнал ЭЭГ. C помощью анализа данного сигнала можно выявить характерные состояния, например, закрытие глаз. При закрытии глаз у большинства людей на ЭЭГ усиливаются колебания в диапазоне от 8 до 13 Гц, называемые альфа-ритмом. Данные ЭЭГ, которые предстоит анализировать в этой задаче, взяты из архива PhysioBank. Они представляют собой последовательность целых чисел, лежащих в диапазоне от -440 до 300, разделённых запятыми и пробелами. Число перед запятой относится к первому каналу, число после - ко второму. Например, последовательность

1,2 3,4 5,6 7,8

следует рассматривать как два сигнала: чила 1, 3, 5, 7 относятся к первому каналу, 2, 4, 6, 8 - ко второму. В тестовых данных требуется по одному каналу передаётся сигнал ЭЭГ, записанный с человека с открытыми глазами, по другому - с закрытыми. Сигнал был оцифрован с частотой 160 Гц.

Требуется определить номер канала, сигнал которого соответствует закрытым глазам.

Пример входных данных, в которых ЭЭГ по второму каналу соответствует закрытым глазам доступен по ссылке (\url{https://www.dropbox.com/s/jow8pa0z348oa34/1.txt?dl=1}).

Для проверки корректности решения в качестве первого теста выступает последовательность чисел, у которых первому каналу соответствует синусоида с частотой 10 Гц, а второму - белый шум.

\putImgWOCaption{16cm}{1}

\sampleTitle{1}

\begin{myverbbox}[\small]{\vinput}
    0,49 114,-13 212,64 277,152 300,-23 277,-23 212,157 
    114,76 0,-46 -114,54 -212,-46 -277,-46 -300,24
    ...
\end{myverbbox}
\begin{myverbbox}[\small]{\voutput}
   1
\end{myverbbox}
\inputoutputTable

\includeSolutionIfExistsByPath{2nd_tour/neuro/task_04}