\assignementTitle{}{6}{}

В первой задаче вам предстоит работать с данными по раку молочной железы Университета Висконсин. 
Они могут быть получены вызовом функции \linebreak load\_breast\_cancer из модуля datasets библиотеки Scikit-Learn. 
В данных записаны клинические измерения опухолей молочной железы и указано является ли опухоль 
доброкачественной или злокачественной.

Ваша задача состоит в построении трёх моделей классификации опухолей с помощью библиотеки Scikit-Learn, 
которые на основании измерений опухоли будут прогнозировать её тип с точностью не менее $90\%$. 
К моделям предъявляются следующие требования:

\begin{itemize}
    \item Модели должны храниться в списке models.
    \item Модели должны быть разных типов. Например, нельзя дважды использовать метод k ближайших соседей.
    \item Нельзя использовать классификатор опорных векторов (SVC).
\end{itemize}

Пример решения вы можете скачать по ссылке (\url{https://www.dropbox.com/s/0ll6ybhg5ynsdsk/solution.py?dl=1}).  Данное решение получило бы 0 баллов, так как все три модели этого решения - это классификатор опорных векторов, но оно демонстрирует, как именно должен выглядеть ответ.

Решение будет тестироваться с помощью этого (\url{https://www.dropbox.com/s/xntz9qhx697assc/test.py?dl=1}) кода (random\_state будет другим).

Решение скопируйте в текстовое поле, после чего оно будет протестировано разработчиками Олимпиады НТИ. Оценки будут выставлены после проверки. За каждую модель, определяющую тип опухоли с точностью не ниже 90\% и удовлетворяющую условиям, вы можете получите по 2 балла. Всего за эту задачу можно получить 6 баллов. 

Обратите внимание, что у вас есть только одна попытка для ввода ответа.

\includeSolutionIfExistsByPath{2nd_tour/neuro/task_07}