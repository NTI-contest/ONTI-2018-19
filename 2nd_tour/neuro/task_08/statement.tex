\assignementTitle{}{8}{}

Для выделения ключевых точек лица с фотографии применяют библиотеку Dlib. Для того, чтобы выделить 68 
ключевых точек, следует воспользоваться предобученной моделью хранящейся по ссылке (\url{https://www.dropbox.com/s/9x6codhj1omkmd2/shape_predictor_68_face_landmarks.dat?dl=1}). Ключевые точки лица нумеруются следующим образом (см. рис. ниже):

\putImgWOCaption{8cm}{1}

Ширину открытия глаз будем считать как усреднённое расстояние между следующими парами точек: между №38 и №42, между №39 и №41, №44 и №48, №45 и №47, поделённое на расстояние межу точками №9 и №28. В данной задаче от вас требуется найти в наборе из 50 фотографий те фотографии, на которых ширина открытия глаз максимальная и минимальная. 

Замечание. При определении координат ключевых точек лица может возникнуть неоднозначность, поэтому:

\begin{enumerate}
    \item При поиске лица используйте детектор лиц со вторым аргументом равным 1, как в примере (\url{http://dlib.net/face_detector.py.html}).
    \item Ищите лицо и ключевые точки на изображении в градациях серого, т.е. если image - считанное 
    изображение, то применяйте детекторы лиц и ключевых точек к изображению gray = cv2.cvtColor(image, cv2.COLOR\_BGR2GRAY).
\end{enumerate}

Вам предстоит работать с изображениями из набора данных, собранного Маркусом Вебером в Калифорнийском 
технологическом институте. Архив с изображениями необходимо скачать по ссылке (\url{https://www.dropbox.com/s/h3tjg55hdm65cth/faces.zip?dl=1}). Изображения в архиве носят названия N.jpg - где N - натуральное число от 1 до 431 включительно. После нажатия кнопки "Нажмите чтобы начать решать" для вас будет сгенерирован файл, содержащий 50 натуральных чисел, разделённых пробелом. Числа в данном файле - номера фотографий, среди которых вы должны будете за 5 минут найти две фотографии с максимальной шириной открытия глаз и минимальной шириной открытия глаз. Ответ напишите в появившемся текстовом поле в виде "N1 N2" (без кавычек), где N1 - название фото (без .jpg), на котором ширина открытия глаз максимальная, а N2 - минимальная. Образец ввода и вывода дан ниже.

Внимание! У вас будет всего одна попытка. Прежде чем нажимать "Нажмите чтобы начать решать" убедитесь, что вы уверенно можете в течение 5 минут найти из произвольного набора 50 фотографий две требуемые.

\sampleTitle{1}

\begin{myverbbox}[\small]{\vinput}
    276 345 7 75 93 157 379 247 188 387 89 18 376 149 283 218 254 368 338 
    291 299 272 381 163 405 50 70 343 104 60 142 401 35 132 192 417 85 
    367 403 130 309 239 144 125 305 212 385 183 194 235 
\end{myverbbox}
\begin{myverbbox}[\small]{\voutput}
    183 163
\end{myverbbox}
\inputoutputTable

\includeSolutionIfExistsByPath{2nd_tour/neuro/task_08}