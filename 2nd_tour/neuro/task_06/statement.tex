\assignementTitle{}{8}

Как и в предыдущей задаче, к сигналу ЭЭГ были добавлены гармоники, лежащие в диапазоне частот альфа-ритма. Кроме того, некоторые участки сигнала были заменены белым шумом значительной большей амплитуды.

Определите число участков с увеличенной амплитудой колебаний в диапазоне от 8 до 13 Гц, при этом не являющихся белым шумом.

Временные интервалы между изменёнными участками сигнала (в том числе и заменёнными шумом) и их продолжительность 
такие же, как и в предыдущей задаче. Частота оцифровки сигнала равна 256 Гц. Пример тестового сигнала с 5 
промежутками с увеличенной амплитудой альфа-ритма, при этом не являющихся шумом, вы можете скачать по ссылке (\url{https://www.dropbox.com/s/ldvthnxm3m8685x/3.txt?dl=1}). Данный сигнал подаётся на вход в качестве второго теста.

\sampleTitle{1}

\begin{myverbbox}[\small]{\vinput}
    -47 -38 -36 -14 -3 -9 0 4 23 35 68 52 27 13 -8 -25 -32 
    -33 -33 -19 8 21 22 10 0 0 25 18 20 16 10 4 0 -4 -18 -6 
    ... 
\end{myverbbox}
\begin{myverbbox}[\small]{\voutput}
    1
\end{myverbbox}
\inputoutputTable

%\includeSolutionIfExistsByPath{2nd_tour/neuro/task_06}