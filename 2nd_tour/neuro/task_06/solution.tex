\solutionSection

В третьей задаче второго блока на записи электроэнцефалограммы требуется определить число участков с увеличенной амплитудой альфа-ритма, при этом игнорируя участки с белым шумом.

Построим график уровня альфа-ритма для тестового сигнала. На нём мы видим 18 пиков, но в условии сказано, что сигнал содержит только 5 участков к которым были добавлены гармоники от 8 до 13 Гц, остальные пики альфа-ритма соответствуют участкам с белым шумом. Построим график сигнала

\putImgWOCaption{13cm}{1}

Построим график сигнала:

\putImgWOCaption{13cm}{2}

На нём мы отчётливо видим участки с белым шумом. Будем считать, что если амплитуда сигнала больше 1300, то мы имеем дело с шумом.
Решение этой задачи, за исключением пары строчек кода, полностью совпадает с решением предыдущей.
В цикле мы дополнительно вводим переменную chunk. В ней будет хранится срез сигнала, по которому вычислялось значение alpha. Если амплитуда сигнала на этом отрезке превышает 1300, то мы не увеличиваем переменную beats, тем самым игнорируя участок с белым шумом.

\begin{minted}[fontsize=\footnotesize, linenos]{python}
    import numpy as np

    def get_spectrum(y):
        Fs = 256.0
        Ts = 1.0/Fs
        n = len(y)
        k = np.arange(n)
        T = n/Fs
        frq = k/T
        frq = frq[range(n//2)]
        Y = np.fft.fft(y)/n
        Y = Y[range(n//2)]
        return frq, abs(Y)

    def get_alpha(y):
        frq, Y = get_spectrum(y)
        alpha = 0
        for freq, ampl in zip(frq, Y):
            if freq > 8 and freq < 13:
                alpha += ampl
        return alpha

    data = np.array([int(s) for s in input().split()])
    alphas = []
    chunk_size = 256
    for start in range(0,len(data)-chunk_size,chunk_size):
        alphas.append(get_alpha(data[start:start+chunk_size]))

    for i in range(1,len(alphas)):
        alphas[i] = 0.8 * alphas[i-1] + 0.2 * alphas[i]

    level = (max(alphas)+min(alphas))/2
    peaks = 0
    for i in range(1,len(alphas)):
        if alphas[i-1]<=level and alphas[i]>level:
            chunk = data[i*chunk_size:(i+1)*chunk_size]
            if max(chunk)-min(chunk) < 1300:
                peaks += 1
    print(peaks)
\end{minted}
