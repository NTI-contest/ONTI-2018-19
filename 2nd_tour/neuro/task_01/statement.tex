\assignementTitle{}{8}

В данной задаче вам предлагается определить количество сокращений сердца на записи сигнала 
электрокардиограммы (ЭКГ) — электрического сигнала, возникающего при работе сердца человека. 
Данные, представленные в этой задаче, взяты из архива PhysioBank.

Сигнал на входе представляет собой последовательность действительных чисел, разделенных пробелом и лежащих в диапазоне от -0.5 до 2.0. Сигнал был оцифрован с частотой 500 Гц. Частота сердечных сокращений в тестовых данных не превышает 200 ударов в минуту. Высота R-зубца не меньше 0.3.

Пример входного сигнала с двадцатью тремя сокращениями сердца доступен по ссылке (\url{http://www.dropbox.com/s/cww1mhh6ge2ewxm/1.txt?dl=1}).

Приведём ниже визуализацию участка примера:

\putImgWOCaption{16cm}{1}

Как вы можете убедиться, на изображенном участке было зарегистрировано два сокращения сердца. Ваши 
решения будут тестироваться на более длительном сигнале, для проверки 
корректности их работы вы можете использовать данный образец.

\newpage

\sampleTitle{1}

\begin{myverbbox}[\small]{\vinput}
    -0.03 -0.035 -0.045 -0.05 -0.06 -0.065 -0.075 -0.08 -0.085 
    -0.09 -0.095 -0.095 -0.095 -0.095 -0.095 -0.095 -0.09 -0.085
    ...
\end{myverbbox}
\begin{myverbbox}[\small]{\voutput}
   2
\end{myverbbox}
\inputoutputTable

%\includeSolutionIfExistsByPath{2nd_tour/neuro/task_01}