\solutionSection

Визуализируем сигнал из данной задачи.

На кардиограмме чётко видны R-зубцы, число которых соответствует сокращениям сердца.

\putImgWOCaption{16cm}{2}

Рассмотрим фрагмент сигнала, содержащий R-зубец (рис. 2.3). Из рисунка видно, что ширина зубца приблизительно составляет 20 точек. Согласно условию, его высота не менее 0.3. Для поиска R-зубцов будем последовательно рассматривать участки сигнала, содержащие 20 точек. Если середина участка больше каждого из его концов на 0.3, то отрезок содержит R-зубец.

Фрагмент сигнала, содержащий R-зубец.

\putImgWOCaption{16cm}{3}

Фрагменты сигналов, содержащий и не содержащие R-зубец.

\putImgWOCaption{16cm}{4}

По условию задачи сигнал был оцифрован с частотой 500 Гц, а частота сердечных сокращений в тестовых данных не превышает 200 ударов в минуту
Следовательно, между вершинами последовательных R-зубцов лежит не менее 150 точек:

\putImgWOCaption{16cm}{5}

С учетом особенностей, выявленных выше возможное решение на языке Python 3 может иметь следующий вид:

\begin{minted}[fontsize=\footnotesize, linenos]{python}
    data = list(map(float, input().split()))
    i = 0
    beats = 0
    while i < len(data) - 20:
        chunk = data[i:i+20]
        if chunk[10]-chunk[0] > 0.3 and chunk[10]-chunk[-1] > 0.3: 
            beats += 1            
            i += 150          
        else:
            i += 1
    print(beats)
\end{minted}

Пояснения к данному решению:

Считаем сигнал в список data.

Введём переменные i и beats. Переменная i - индекс начала участка сигнала, в котором мы будем искать R-зубец, а переменная beats - число обнаруженных R-зубцов.

Значения переменной i будем перебирать в цикле while. Так как переменная i - индекс начала участка сигнала, содержащего 20 точек, то её максимальное значение должно быть на 20 меньше числа значений всего сигнала.

В переменной chunk будем хранить срез сигнала, в котором ищем R-зубец.

Если значение из середины среза больше значений первого и последнего элементов на 0.3, то отрезок содержит R-зубец и мы увеличиваем переменную beats на единицу. При этом чтобы исключить возможность дважды посчитать один и тот же пик, увеличим индекс начала среза на 150, то есть на минимальное число точек между R-зубцами.

Если условие наличия R-зубца не было удовлетворено, увеличиваем переменную i на 1.

В конце выведем получившееся число зубцов функцией print.


