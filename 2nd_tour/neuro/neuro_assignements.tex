\section{Блок заданий 1}

Справочные материалы к данному блоку задач представлены в Приложении 1. В скобках возле номера задачи указан максимальный балл за данную задачу.

\subimport{2nd_tour/neuro/task_01/}{statement}
\subimport{2nd_tour/neuro/task_02/}{statement}
\subimport{2nd_tour/neuro/task_03/}{statement}

\section{Блок заданий 2}

\subimport{2nd_tour/neuro/task_04/}{statement}
\subimport{2nd_tour/neuro/task_05/}{statement}
\subimport{2nd_tour/neuro/task_06/}{statement}

\section{Блок заданий 3}

В данном блоке потребуется работать с библиотеками OpenCV, Scikit-Learn, Dlib. Так как решение подготавливается на собственных персональных компьютерах, данные библиотеки вы можете установить с помощью pip.

Полезная литература и ссылки:

Андреас Мюллер, Сара Гвидо, Введение в машинное обучение с помощью Python (2017)

\url{www.pyimagesearch.com/2017/04/03/facial-landmarks-dlib-opencv-python/}

\url{www.stackoverflow.com/questions/41912372/dlib-installation-on-windows-10}

\url{www.pythonworld.ru/osnovy/pip.html}

\url{www.cmake.org/install} — может потребоваться установка cmake.

\subimport{2nd_tour/neuro/task_07/}{statement}
\subimport{2nd_tour/neuro/task_08/}{statement}
\subimport{2nd_tour/neuro/task_09/}{statement}

\section{Блок заданий 4}

\subimport{2nd_tour/neuro/task_10/}{statement}

\section{Приложение. Справочные материалы}

\subimport{2nd_tour/neuro/}{addition}