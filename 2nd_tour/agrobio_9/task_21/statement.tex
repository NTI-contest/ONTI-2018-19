\assignementTitle{}{5}{}

Выберите из ответов все варианты компенсации данного эффекта (из предыдушего задания).

\begin{enumerate}
    \item Отдать корм на проверку токсичности. На время проверки перейти на кормление другим кормом
    \item Осуществить полную замену воды минуя стадию водоподготовки (спасать рыбу надо скорее)
    \item Оптимизировать световой режим
    \item Повысить концентрацию хлорида натрия до 0.3 промилле
    \item Починить оксигенатор
\end{enumerate}

\explanationSection

Починка оксигенатора не соответствует условиям задачи. Если кислорода необходимо и достаточно, значит оксигенатор либо не нужен, либо стправляется с поставденной задачей.

Частичная замена воды может снизить эффект блокировки нитрит-ионом, но, более уместным будет увелчичение концентрации хлорида натрия до 0.3 промилле. Такая концентрация не оказывает влияние на рыб и допустима для гидропонных установок.

\answerMath{4.}