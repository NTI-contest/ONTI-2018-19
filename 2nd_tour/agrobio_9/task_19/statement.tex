\assignementTitle{}{8}{}

При наблюдении за Аквапонной системой в течении недели выявлены следующая динамика факторов:

\begin{itemize}
    \item Увеличение мутности (от прозрачной воды до воды с опалесценцией). При частичной замене воды опалесценция снижается.
    \item Увеличение концентрации  иона аммония. При частичной замене воды падения концентрации иона аммония не наблюдается.
    \item Значение рН в воде изменяется с 7.05 до 8.5. При частичной замене воды изменение рН с 8.5 не происходит
    \item При увеличении температуры воды в системе с $10^{\circ}C$ до $25^{\circ}C$ опалесценция исчезает. Без замены части воды значение  рН опускается до 7.05. снижается концентрация иона аммония.
\end{itemize}

Выберите из предлагаемых ответов все, объясняющие данную ситуацию.

\begin{enumerate}
    \item Аммиак при значении рН = 8.5 может находиться как в виде ионов, так и в виде растворённого в воде газа
    \item При частичной дегазации буферная ёмкость системы аммоний-аммиак падает, что приводит к изменению значания рН
    \item Частичная замена воды приводит к изменению равновесия между концентрациями растворённого иона аммония и газообразного
    \item Аммоний и аммиак образуют буферную систему, стабилизирующую рН при частичной замене воды
    \item При повышении температуры воды усиливается метаболическая активность бактерий биофильтра
    \item Увеличение температуры приводит к снижению растворимости газов и частичному удалению аммиака из раствора
\end{enumerate}

\explanationSection

В данной задачи важно отметить, в условии, нормальное насыщение кислородом системы. Как следствие, ответы, объясняющие поведение рыб через аварийное состояние системы контроля насыщения кислородом или аварийного состояния оксигенатора будут не верны по условию задачи.

Отравление рыб некачественным кормом сопровождается угнетенным состоянием, рыбы располагаются у дна, либо наблюдается вздутие брюшка. Но рыба не захватывает воздух с поверхности воды. 

Такое поведение характерно для рыб, страдающих бранхиомикозом – грибковым поражением сосудов жабр, однако такой ответ не предусмотрен. 

Сходное поведение проявляется у рыб в случае блокировки газообмена между средой и кровью капилляров жабр за счёт нитрит-иона.

\answerMath{1, 2, 3, 4, 5, 6.}