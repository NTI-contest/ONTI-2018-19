\assignementTitle{}{4}{}

На графике изображена зависимость интенсивности фотосинтеза от интенсивности падающего на растение света.

\putImgWOCaption{11cm}{1}

Ось х – интенсивность света Вт$\cdot$м$^{-2}$

Ось у – интенсивность работы фотосинтетической системы растения

С чем может быть связана такая форма графика?

\begin{enumerate}
    \item Увеличение освещённости выше определённого значения приводит к нарушению процесса биосинтеза ферментов фотосинтетической системы и угнетению системы фотосинтеза
    \item Увеличение освещённости выше определённого значения приводит к фрагментарному некрозу листовой пластинки, что, в свою очередь, снижает интенсивность фотосинтеза
    \item Увеличение освещённости выше определённого значения приводит к снижению работы системы фотосинтеза по механизму обратной связи (много продуктов фотосинтеза угнетает сам фотосинтез из-за превышения процесса синтеза над процессом транспорта в организме растения)
    \item Увеличение освещённости выше определённого значения приводит к компенсаторным механизмам усиления пигментации поверхности листовой пластинки и к изменению проницаемости кутикулы, что, в свою очередь, снижает интенсивность работы фотосинтетических систем, сохраняя общий выход продуктов фотосинтеза оптимальным
\end{enumerate}

\explanationSection

Зависимость скорости фотосинтеза от интенсивности света имеет форму логарифмической кривой. Прямая зависимость скорости процесса от притока энергии наблюдается только при низкой интенсивности света. 

При высоких значениях светового потока или превышении времени освещённости наблюдается фотодеструкция хлорофилла. Если растение не имеет компенсаторного эволюционного механизма (как растения, обитающие за поляроным кругом, подвергающиеся высокой освещённостью в течении полярного дня).

В принципе, фотосинтез начинается при очень слабом освещении. Впервые это было показано на установке искусственного освещения. Света керосиновой лампы оказалось достаточно для начала фотосинтеза и образования крахмала в растительных клетках. 

У многих светолюбивых растений максимальная (100\%) интенсивность фотосинтеза наблюдается при освещённости, достигающей половины от полной солнечной, которая, таким образом, является насыщающей. 

Дальнейшее возрастание освещённости не увеличивает фотосинтез, а затем и снижает его. Чрезмерно высокое освещение резко нарушает процесс биосинтеза пигментов (редукция пигментного аппарата) и фотосинтетические реакции. 

Так же избыточная освещённость вызывает образование активных форм кислорода, которые превышают компенсаторные антиокислительные механизмы клеток растений. В результате происходит деградация хлорофилла. 

\answerMath{1.}