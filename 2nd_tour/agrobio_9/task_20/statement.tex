\assignementTitle{}{5}{}

Рыбы в аквариумном блоке аквапонной системы часто всплывают на поверхность, заглатывают воздух.

Поведение рыб не спокойное: рыба «плещется», жаберные крышки приподнимаются чрезмерно, частота движений жаберных крышек выше нормальной.

При этом, в системе работает оксигенатор и концентрация кислорода в воде оптимальна для данного вида. Повышение концентрации кислорода до предельной равновесной не приводит к изменению поведения рыбы.

Выберите все варианты, которые могут объяснить подобное поведение.

\begin{enumerate}
    \item Такое поведение может быть вызвано стрессом из-за недавней пересадкой в новый аквариум с другими параметрами воды
    \item Нитриты в воде блокируют поступление кислорода в организм рыб
    \item Наблюдаемое явление могло быть спровоцировано резкой сменой периодичности светового режима и резким выключением света
    \item Данное поведение характерно для отравления рыб некачественным кормом
    \item Концентрация кислорода в воде не достаточна. Сломан оксигенатор
    \item Подобное поведение рыб может быть связано с повышенным значением Ph воды
    \item Концентрация кислорода в воде не достаточна. Сломан датчик растворённого в воде кислорода
\end{enumerate}

\answerMath{2, 6.}