\assignementTitle{}{4}{}

На рисунке приведено 2 графика, отражающие изменение некоторых параметров в аквапонной системе во времени.

По оси X приведено время в сутках, по оси Y- значения двух параметров (каждый параметр приведен в своих единицах измерения). Из приведенных вариантов выберете пару параметров, изменения которых могут носить такой характер.

\putImgWOCaption{11cm}{1}

*Действительные значения параметров для решения задания не принципиальны. Важен характер изменения.

Что представлено на графике зеленым цветом (Показатель-2)?

\begin{enumerate}
    \item Изменение концентрации углекислого газа в воде в течение суток
    \item Изменение концентрации общего азота в системе в течение суток
    \item Динамика популяции коловраток в системе
    \item Изменение концентрации кислорода в воде в течение суток
    \item Изменение суточной активности фермента каталазы в клетках гидробионтов
\end{enumerate}

\explanationSection

Из графика видно, что экстремумы линии 2 совпадают с моментами включения (максимум) и выключения (минимум) освещения аквапонной системы. 

Как будет изменяться в течении суток концентрация кислорода в воде? Если линия 2 – изменение концентрации кислорода в воде, то, во время светового дня концентрация кислорода в воде должна снижаться, а во время световой ночи концентрация кислорода должна увеличиваться. Подобное изменение концентрации кислорода в зависимости от времени суток ничем не обосновано.

Отмечу, что, т.к. в условиях задачи не сказано о работе аэратора или оксигенатора аквапонной системы, способных сохранять концентрацию кислорода в воде на постоянной уровне в течении суток, считаем, что аквапонная система работает без принудительного насыщения кислородом.

Изменение суточной активности фермента каталазы в клетках гидробионтов, как и в любых живых клетках не должно носить суточной периодичности, т.к. данный фермент защищает клетки организма от разрушения под действием перекисного окисления веществ. Титр каталазы в клетках и тканях изменяется в случае интенсификации процессов перекисного окисления липидов, снижении антиоксидантной активности (снижение концентрации витамина А, С, Е в клетках). Корреляцию между освещённостью и каталазной активностью можно предположить только в ситуации УФ или гамма-облучения организма, однако зависимость будет носить не суточный характер.

Изменение концентрации общего азота в системе в течение суток представляется возможным при переизбытке азота в системе и отсутствии компенсации поступающего азота растениями гидропонного модуля, но давайте продолжим анализировать предлагаемые варианты…

Динамика популяции коловраток в системе от времени суток – несколько надуманный, абсурдный вариант ответа. Сезонные изменения численности в популяции или изменения численности популяции в зависимости от концентрации отравляющих веществ в системе можно было бы рассмотреть, но суточные колебания в данном случае не соответствуют реальности.

Наиболее достоверным из перечисленных ответов для линии 2 будет изменение концентрации углекислого газа в воде в зависимости от освещённости. 

Т.к., по условию задачи, достоверным должен быть один ответ, принимаем за верный именно этот вариант ответа.

\answerMath{1.}