\assignementTitle{}{3}{}

Какой порядок светового дня при использовании модуля с водными растениями в проектируемой установке можно считать оптимальным с точки зрения азотного обмена?

\begin{enumerate}
    \item Любой, т.к. от чередования светового дня не зависит способность растений гидропонного модуля наращивать биомассу
    \item Попеременно: световой день для гидропонного модуля и световая ночь для дополнительного модуля фильтрации, либо световой день для дополнительного модуля фильтрации и световая ночь для гидропонного модуля
    \item Одновременно для гидропонного модуля и дополнительного модуля фильтрации в соответствии с суточным ритмом освещённости
\end{enumerate}

\explanationSection

Оптимальной настройкой светового дня будет попеременно либо световой день для гидропонного модуля и световая ночь для дополнительного модуля фильтрации, либо световой день для дополнительного модуля фильтрации и световая ночь для гидропонного модуля. Этот ответ верный, поскольку в данном случае актуален не прирост биомассы, а азотный обмен в аквапонной системе.

\answerMath{2.}