\assignementTitle{}{6}{}

Одним из важнейших компонентов аквапонной системы является биофильтр, в котором происходит превращение вырабатываемого рыбами ядовитого аммиака в нитраты и нитриты, которые потом потребляются растениями в гидропонном модуле.

Для правильной работы системы необходимо соблюсти баланс рыбы, частоты кормления,  бактерий и растений.

В биофильтре бактерии закрепляются и размножаются на специальном наполнителе, в качестве которого может применяться гравий, специальная керамика и некоторые другие виды.

При выборе наполнителя наиболее важным параметром является площадь его поверхности 
(специфическая площадь поверхности, specific surface area, удельная площадь поверхности), 
которая измеряется в $m^2/m^3$.

Расположите предложенные варианты наполнителя в порядке возрастания этого параметра:

\begin{enumerate}
    \item Пластиковый наполнитель биофильтра (Bio Ball)
    \item Речной камень (диаметр - 25 мм)
    \item Крупный песок (диаметр - 3 мм)
    \item Гравий (диаметр - 15 м) 
    \item Керамический пористый наполнитель
\end{enumerate}

\explanationSection

Для решения задания необходимо воспользоваться поиском в интернете.

Так, в условии было дано название одно из наполнителей - Bio Ball. Перейдя на один из сайтов мы сможем найти, 
что его специфическая площадь поверхности составляет 350-800 m$^2$/m$^3$. Там же есть сравнение с керамическим 
наполнителем, площадь которого составляет  1200-3000 m$^2$/m$^3$.

Природные наполнители (песок, гравий и камень) имеют меньшую площадь. И исходя из логических рассуждений, 
учитывая плотность, размер и шереховатость поверхности песчинки, камня и гравия, можно понять, что по возрастанию их можно выстроить как камень > гравий > песок.

\answerMath{2, 4, 1, 3, 5.}
