\assignementTitle{}{8}{}

Для вашей проектируемой системы, с учетом приведенных ранее данных и ваших произведенных расчетов,  
рассчитайте, сколько надо использовать наполнителя для биофильтра в системе, если известно, что 
планируется использовать наполнитель с площадью поверхности 1400 $\text{м}^2/\text{м}^3$.

Для обеспечения оптимального азотного баланса в системе (в расчете на общий объем воды) 
рекомендуется использовать 0.245 М$^2$/л наполнителя.

Ответ приведите в литрах и округлите до целых.


\solutionSection

Мы знаем, что объем системы должен составлять 800 л.

При этом, наполнителя должно быть (тут имеется в виду именно площадь, на которой смогут закрепляться бактерии): 
$$800 \: \text{л} \cdot 0.245 \: \text{м}^2 = 196 \: \text{м}^2$$  

Для рассчта объема этого наполнителя надо использовать его удельную площадь: $196/1400 = 0.14$ м$^2$ или 140 л.

\answerMath{140.}