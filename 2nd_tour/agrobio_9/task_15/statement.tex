\assignementTitle{}{8}{}

Вам необходимо спроектировать аквапонную установку для выращивания редиса сорта  «Суперред».

Известно, что этому сорту (гибриду первого поколения) требуется  22 дня  для достижения товарного размера и качества от посева.  (\url{http://www.ponics.ru/2009/04/agrotrip3/})

Аквапонная система должна быть спроектирована для одновременной работы 22 гидропонных стеллажей, рассчитанных на выращивание редиса в кассетах на 64 посадочных места в каждой. При этом, на один гидропонный стеллаж умещается 72 кассеты редиса.

В системе должен быть реализован конвейерный метод получения товарной продукции – по одному стеллажу товарной продукции в неделю.

Дополнительно известно:

\begin{itemize}
    \item Каждая посадочная кассета редиса требует внесения  1.5 г рыбьего корма для обеспечения минеральными веществами для роста и развития.
    \item Используемая в системе рыба может потреблять корма, в количестве $3\%$ в день от своей биомассы.
\end{itemize}

Какой объём воды должен быть в аквапонной установке, если на один кг рыбы необходимо 10 литров воды?

Ответ округлите сотен литров.

В ходе решения округление осуществляйте до 0.001

\solutionSection

\begin{enumerate}
    \item рассчитаем количество кассет в гидропонном модуле: \\
    22 стелажа $\cdot$ 72 кассеты = 1584 кассеты всего в гидропонном модуле;
    \item Рассчитаем, сколько надо вносить корма для того, чтобы обеспечить растения азотом в нужном количестве:
    1584 кассеты $\cdot$  0.0015 кг корма на одну кассету = 2.376 кг корма в день;
    \item Рассчитаем массу рыбы, которая сможет потребить вносимый корм:
    2.376 кг это 3\% от массы рыбы, сл-но: 2.376 $\cdot$ 100/3 = 79.2 кг
    \item Рассчитаем объем воды в системе, при рассчете 10 л на кг рыбы: \\
    (79.2 $\cdot$ 10) л = 792, примерно равно 800 л.
\end{enumerate}
\answerMath{800.}