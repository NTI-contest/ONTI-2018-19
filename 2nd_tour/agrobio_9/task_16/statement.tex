\assignementTitle{}{8}{}

Используя промежуточные расчеты и условия из прошлой задачи, рассчитайте потребление азота (мг) одним корнеплодом в день.

Дополнительно известно, что:

\begin{itemize}
    \item Содержание белка в рыбьем корме составляет $48\%$
    \item Средний молекулярный вес аминокислотного остатка (в составе белка) составляет 110 г/моль
    \item В среднем (при расчете всех возможных остатков), в аминокислотном остатке содержится 1.5 атома азота
    \item Считаем, что белок в корме - единственный источник азота в системе.
\end{itemize}

В процессе решения промежуточные значения округляйте до сотых.

В ходе решения массу рассчитывайте в граммах.

Ответ (в миллиграммах) округлите до десятых.

\explanationSection

\begin{enumerate}
    \item Используя рассчеты прошлой задачи рассчитаем количество растений в модуле:
    1584 кассеты $\cdot$ 64 посадочных места = 101376 корнеплодов в гидропонном модуле всего
    \item зная состав корма и его массу определим массу чистого белка: 
    2.376 кг корма в сутки $\cdot$  48\% белка/100\% = 1.14048 кг белка в сутки  (1140.48 г)
    \item Теперь рассчитаем массу азота в белке и его количество на 1 корнеплод:
    
    Сожержание азота в белке= 1.5 $\cdot$ 14/110 = 0.19  (19.09\%)
    
    1140.48 $\cdot$ 0.19 = 216.69г азота в сутки на все корнеплоды (217.72 - если считаем через 19.09\%)
    
    0.2166912/101376 = 2.137 мг азота на 1 корнеплод (или 2.1476 при расчете через 19.09)
\end{enumerate}

\answerMath{2.1.}