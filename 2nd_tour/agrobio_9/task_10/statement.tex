\assignementTitle{}{4}{}

Для оптимального поддержания жизнедеятельности гидробионтов в модельной системе требуется поддерживать временную жёсткость воды в пределах 8-12 ммоль/л. У вас есть в наличии:

\begin{enumerate}
    \item Низовой торф
    \item Кокосовое волокно
    \item Керамзит
    \item Вермикулит
    \item Щебень осадочных пород. 
    \item Перлит 
\end{enumerate}

Общая жёсткость воды, используемая для запуска системы – до 1.5 ммоль/л.

В какие участки на предложенном графике следует разместить предложенные субстраты по соотношению оказываемого влияния на жесткости воды и Ph?

\putImgWOCaption{8cm}{1}

\explanationSection

\putImgWOCaption{8cm}{2}

Для решения данной задачи нужно узнать химические свойства субстратов в воде.

Органические субстраты, такие как низовой торф и кокосовое волокно смещают рН и уменьшают жёскость воды.

Керамзит, вермикулит и перлит являются нейтральными субстратами по отношению к жёсткости и значению рН

Щебень осадочных пород увеличивает жёсткость и смещает значение рН в щелочную область.

\answerMath{Низовой торф - 3; Кокосовое волокно - 3; Керамзит - 2; Вермикулит - 2; Щебень осадочных пород - 1; Перлит - 2.}