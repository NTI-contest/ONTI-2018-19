\assignementTitle{}{6}{}

Аквапонная установка используется для проведения  исследование активности фотосинтеза в условиях 
изменения влажности, концентрации $CO_2$ в воздухе и интенсивности освещенности. Активность фотосинтеза проверяется измерением потребляемого $CO_2$.

В рамках одного эксперимента производится изменение только двух из трех упомянутых выше параметров.  

На рисунке приведено наложение трех графиков, полученных в ходе эксперимента. По оси X приведено время эксперимента. По оси Y - значения одного исследуемого (поглощение $CO_2$) и двух изменяемых параметров (влажность + интенсивность освещения ; или интенсивность освещения + изменение концентрации $CO_2$; или влажность + изменение концентрации $CO_2$). Каждый параметр приведен в своих единицах измерения.

Для проведения эксперимента в системе установлено следующее оборудование:

\begin{itemize}
    \item Баллон с углекислым газом и система плавного повышения его концентрации в установке.
    \item Лампы освещения с плавной автоматической регулировкой интенсивности от 50 до 900 Вт$\cdot$м$^{-2}$.
    \item Система увлажнения. Увлажнение осуществляется дискретно (не плавное, а резкое изменение параметра).
\end{itemize}

Продумайте, как связаны между собой углекислотное насыщение фотосинтеза, освещённость, влажность и концентрация углекислого газа в воздухе, и выберете верное утверждение описывающее поведение параметров на графике.

\putImgWOCaption{12cm}{1}

\begin{enumerate}
    \item Ведущую роль в усилении фотосинтеза при увеличении концентрации углекислого газа играет влажность почвы. При постоянной влажности почвы обогащение газовой смеси углекислым газом не влияет на кривую поглощения углекислого газа в процессе фотосинтеза.
    \item Потребление углекислого газа изменяется в результате превышения порогового значения его концентрации для фотосинтетической системы 2. Как следствие, фотосинтетические системы получают возможность более эффективно потреблять воду.
    \item Рост эффективности работы фотосинтетических систем в результате повышения влажности почвы возможен только при увеличении интенсивности освещённости, что не указано на представленном графике.
    \item Потребление углекислого газа не может изменяться после достижения предела углекислотного насыщения фотосинтеза, т.к., в состоянии насыщения субстратом, ферменты просто не успевают перерабатывать излишки веществ, поступающих в систему
\end{enumerate}

\explanationSection

Чтение графика нам позволяет сделать вывод о том, что изменение параметра 2 (линия 2) происходит однократно, практически в 2 раза по отношению к начальному значению параметра; 

Кривая изменения параметра 1 (линия 1) в точке мгновенного изменения параметра 2 (линия 2) имеет точку перегиба. Дальнейшая кривая повторяет полностью своё поведение до точки перегиба;

Кривая 3 на всём протяжении графика возрастает линейно.

Поскольку для линии 1 и линии 2 точка мгновенного изменения параметра совпадает, а линия 3 не изменяется за всё время регистрации, закономерно допустить корреляцию значений параметра 1  и параметра 2. 

Известно, что при постоянной влажности кривая поглощения углекислого газа в процессе фотосинтеза достигает своего насыщения вне зависимости от увеличения концентрации углекислого газа в воздухе. 

В то же время, при изменении влажности, способность поглощать углекислый газ вновь возрастает. Этому варианту ответа соответствует поведение кривых на представленном графике. (вариант ответа 1)

Однако, давайте рассмотрим остальные варианты ответов…

Предположим, что параметр 1 достигает порогового значения для параметра 3, при котором происходит «более эффективное потребление воды фотосинтетической системой» – параметр 2. Для того, что бы рассматривать этот ответ как правильный, необходимо увидеть график изменения освещённости и определить так называемую «точку компенсации», так же зависимость от температуры среды. Поскольку данные параметры в задачи не указаны, информации для подтверждения данного ответа не достаточно.

Точно так же не может быть рассмотрен следующий ответ. Рост эффективности работы фотосинтетических систем (поглощение большего количества углекислого газа) в результате повышения влажности почвы возможен только при увеличении интенсивности освещённости, что не указано на представленном графике.

Ответ 4 -  Параметр 3 не может в принципе изменяться после достижения предела углекислотного насыщения фотосинтеза, т.к., в состоянии насыщения субстратом, ферменты просто не успевают перерабатывать излишки веществ, поступающих в систему. Содержит допущение о не достоверности представленного варианта изменения графика, однако, в случае анализа информации из сети, становится понятным его необоснованность.

\answerMath{1.}