\assignementTitle{}{5}{}

Какие действия со сбалансированной аквапонной системой могут приводить к нарушению установившегося равновесия:

\begin{enumerate}
    \item Замена 1/2 объёма воды на отстоявшуюся мягкую
    \item Частичная замена биопродукции гидропонного модуля на рассаду тех же видов с увеличением количества растений
    \item Частичная подмена воды (до 1/10 объёма) на отстоявшуюся мягкую
    \item Замена 1/2 объёма воды на жёсткую хлорированную воду из под крана
    \item Подключение новых модулей к аквапонной системе
    \item Полная замена гидропонных растений на другие виды
    \item Частичная замена гидробионтов одного вида на молодь гидробионтов другого вида при различных требованиях к системе жизнеобеспечения и равных пищевых предпочтениях
    \item Частичная замена гидробионтов одного вида на молодь другого вида при равных требованиях по системе жизнеобеспечения и различных пищевых предпочтениях
    \item Частичная замена гидробионтов одного вида на молодь того же вида при смешанном возрастном составе аквакультуры
    \item Увеличение количества гидробионтов одного вида
\end{enumerate}

\explanationSection

Ключевой момент – стабильность аквапонной системы в целом. Резкое изменение параметров среды или дополнительная манипуляция с элементами системы приводит к нарушению динамического равновесия всей системы. Чем сильнее вносимые изменения или чем больше манипуляций с системой за единицу времени, тем выше вероятность её разбалансировки.

Какие параметры в системе мы можем контролировать? 
\begin{itemize}
    \item Формы азота в системе (соотношение аммиака, нитритов и нитратов в стабильной системе есть величина постоянная\footnote{В данном случае под термином «постоянная величина» имеется в виду среднее значение с небольшим доверительным интервалом значений, определяющий динамическое равновесие в системе.})
    \item Жёсткость (общая и кальциевая) так же является величиной постоянной
    \item Концентрация кислорода и углекислого газа в системе
    \item Скорость течения воды в УЗВ
    \item Видовой состав гидробионтов и количество задаваемого корма
    \item Возрастной состав гидробионтов
\end{itemize}

Смена 1/2 объёма воды в системе на отстоявшуюся мягкую приводит к резкому изменению этих параметров (в 2 раза). Тот же эффект (резкое изменение значений), плюс негативное влияние хлора на биоту системы даёт замена 1/2 объёма воды в системе на жёсткую хлорированноую воду.

При частичной подмене воды в 1/10 объёма изменения не будут превышать возможность быстрого восстановления параметров системы. Именно этот приём используется в аквариумистике и может быть применён для аквапоники.

Увеличение количества гидробионтов одного вида приводит к дефициту кислорода, увеличению концентрации углекислого газа и аммиака в системе. Это приведёт к изменению соотношения видового состава в бактериальном фильтре. Система выйдет из состояния динамического равновесия.

Последовательный отлов из системы товарной продукции при содержании разновозрастной группы одного вида возможна. Важно только контролировать потребление корма и скорость обмена веществ в разновозрастной популяции с окружающей средой так, что бы изменения при отлове товарной продукции и запуске молоди в систему были минимальны.

В случае замещения особей одного вида на тоже количество особей другого вида, требующих сходные условия существования, но отличающихся по типу питания формирует условия разведения смешанной аквакультуры. Этот приём используется для увеличения эффективности системы выращивания без увеличения объёма системы. 

Обратная ситуация, когда требования к содержанию и разведению у видов различно, а пищевые рационы одинаковы, мы увидим пищевую конкуренцию, плюс невозможность содержания при одних условиях среды. В данном случае система не может существовать.

Полная замена гидропонных растений на растения другого вида смещает динамическое равновесие соединений азота в системе.

Если же из гидропонного модуля производится отъём товарной продукции с заменой на более молодые растения того же вида с увеличением численности растений (занимают меньше площадь, хуже развита корневая система), мы приходим к варианту аквапонной системы с конвейерным способом выращивания зеленных культур. Изменения в системе не приведут к её дестабилизации.

При сохранении численности растений гидропонного модуля и частичной замене товарной продукции на более молодые растения, у нас может изменяться динамическое равновесие соединений азота в системе. На разных стадиях роста и развития у растений различные требования к потреблению веществ из среды. Не учитывая этот фактор можно сильно сместить равновесие в системе, приводя к её дестабилизации.

Подключение новых модулей к аквапонной системе приводит к изменению объёма и, как следствие всех концентрационных постоянных.

\answerMath{1, 4, 5, 6, 7, 10.}