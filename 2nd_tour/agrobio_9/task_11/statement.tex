\assignementTitle{}{4}{}

На рисунке приведено 2 графика, отражающие изменение некоторых параметров в аквапонной системе во времени.

По оси X приведено время в сутках, по оси Y- значения двух параметров (каждый параметр приведен в своих единицах измерения). 

Из приведенных вариантов выберете пару параметров, изменения которых могут носить такой характер.

\putImgWOCaption{12cm}{1}

*Действительные значения параметров для решения задания не принципиальны. Важен характер изменения.

Что представлено на графике фиолетовым цветом (Показатель-1)?

\begin{enumerate}
    \item Интенсивность работы плунжерной системы насоса подачи воды в аквакультурный модуль
    \item Изменение редокс - потенциала в воде в результате кормления рыб в аквакультурном модуле
    \item Освещённость гидропонного модуля
    \item Активность фотосинтетической системы 2 в течение суток
    \item Активность фотосинтетической системы 1 в течение суток
\end{enumerate}

\explanationSection

Работа плунжерного насоса сопровождается обратно-поступательным движением и чередованием заполнения рабочего объёма жидкостью с выталкиванием жидкости из цилиндра. Если линия 1 отображает работу плунжерного насоса, то линия 2 отображает синусоидальное изменение скорости течения в УЗВ. Однако, плунжерные насосы не находят применение в системе циркуляции воды УЗВ, т.к. циркуляция воды в системе должна имитировать равномерный ток воды в естественных водоёмах без колебаний скорости потока и других пульсаций.

Изменение редокс-потенциала в результате кормления рыб. Предположим, что это так. Тогда, согласно линии 1, окислительно-восстановительный потенциал среды меняется мгновенно от некоторой постоянной величины до нуля и обратно, что противоречит смылсу данного показателя.

Активность фотосинтетической системы в течение времени может меняться в зависимости от освещённости и накопленной системой энергии, однако, мгновенное изменение скорости ферментативных реакций представляется невероятным.

Освещённость гидропонного модуля может изменяться как плавно, так и резко, в зависимости от способа отключения осветительных приборов. Данный случай (линия~1) характерен для выключения системы освещения по таймеру.

\answerMath{3.}