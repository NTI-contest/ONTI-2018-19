\assignementTitle{}{4}{}

Вам предлагается дополнить существующую аквапонную систему, состоящую из гидропонного, аквакультурного и бактериального фильтрационного модуля дополнительной ёмкостью с:

\begin{itemize}
    \item Нитчатыми водорослями;
    \item Высшими водными растениями (роголистник, рдест кучерявый, телорез, водокрас лягушачий).
\end{itemize}

Для успешной интеграции вам надо просчитать оптимальную последовательность подключения исходя из потока и перехода воды из одного модуль в следующий.

Из предложенных вариантов выберите один или несколько подходящих вариантов последовательности подключения модулей. Отсчет начинается с Гидропонного блока. После последнего блока вода поступает вновь в первый-гидропонный.

Подключаемый модуль заполнен только высшими водными растениями

В ответах используются следующие индексы для обозначения модулей:

\begin{enumerate}
    \item Гидропонный
    \item Аквакультурный
    \item Растительноводный (дополнительный модуль)
    \item Бактериальный фильтрационный
\end{enumerate}

\begin{enumerate}
    \item[а.] 1-2-4-3
    \item[б.] 1-3-2-4
    \item[в.] 1-2-3-4
    \item[г.] 1-4-2-3
    \item[д.] 1-3-4-2
    \item[е.] 1-4-3-2
\end{enumerate}

\explanationSection

Восстановим последовательноть циркулирования воды в аквапонной установке. Из гидропонного модуля вода попадает в аквакультурный, далее из аквакультурного в блок механической и бактериальной фильтрации и возвращается обратно в гидропонный. Это стандартная схема циркулирования воды в аквапонной установке. 

Восстановим функции блоков. Гидропонный модуль «забирает» из системы нитраты, которые используются растениями. Аквакультурный модуль – модуль получения основной биопродукции животного происхождения. Именно сюда происходит выделение катаболитов, в т.ч. продуктов белкового обмена (в д.сд. аммиак). Бактериальный фильтр преобразует продукты аммиак в нитриты и нитраты.

Каковы условия функционирования каждого из модулей. 
\begin{enumerate}
    \item Непрерывность циркуляции воды в системе; 
    \item Освещение (прежде всего для гидропонного блока). При этом, длительность светового дня определяется требованиям выращиваемой культуры, но не 24 часа в сутки. Т.е., для нормальной физиологии растений должен чередоваться световой день и световая ночь. Цикличность в чередовании светового дня и ночи определяет периодичность работы фотосинтезирующей системы растений. Поскольку, у автотрофных организмов, получение энергии для синтетического обмена зависит от освещенности,  скорость синтетических реакции с участием соединений азота в растениях так же зависит от продолжительности светового дня. 
Нитраты могут накапливаться в растениях не только от избытка этих соединений в растворе, но и в следствие недостаточной освещённости растений, т.к. при низком уровне световой энергии процессы синтеза в клетках растений затормаживаются. 
    \item Концентрация нитратов в системе влияет на способность бактерий бактериального фильтра трансформировать аммиак в нитриты и далее в нитраты; 
    \item Повышение концентрации соединений азота в системе для аквакультуры, может быть вызвано «насыщением» потребления нитратов  растениями, снижением анаболической активности растений при стабильном выделении катаболитов гидробионтами.
    \item Т.к. в условиях задачи про влияние молибдена и фосфора на азотный обмен в организме растений не говориться, концентрациями этих веществ в системе пренебрегаем. 
\end{enumerate}

Рассмотрим как оптимально встроить в аквапонную систему модуль с нитчатыми водорослями.

Данный модуль может быть использован в противофазе светового дня после гидропонного модуля, для стабилизации суточной концентрации соединений азота в системе. 

Использовать водорослевый модуль как биопродуктивный крайне сложно, однако, он может служить дополнительным источником питания для растительноядных рыб. Следовательно, установка этого модуля перед аквакультурным снизит вероятность распространения водорослей в системе за счёт поедания водорослей при случайном забросе через систему рециркуляции воды.

Исходя из вышесказанного, последовательность подключения модулей аквапонной системы с водорослевым фильтром будет выглядеть так: гидропонный модуль, водорослевый фильтр, аквакультурный модуль, модуль механической и бактериальной фильтрации и далее, гидропонный модуль. Цикл замкнулся.

Теперь рассмотрим как оптимально встроить в аквапонную систему модуль с высшими водными растениями.

Данный модуль может быть так же использован в противофазе светового дня после гидропонного модуля, для стабилизации суточной концентрации соединений азота в системе или как дополнительный модуль стабилизации при резком увеличении концентрации форм азота в системе. 

Место подключения такое же как у водорослевого фильтра, однако второй вариант - подключения после механического и бактериального фильтра, перед гидропонным модулем.

\answerMath{a, б.}