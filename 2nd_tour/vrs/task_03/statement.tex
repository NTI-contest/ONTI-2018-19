\assignementTitle{}{30}{}

\subsubsection*{Конфигурация третьей сцены}

\putImgWOCaption{5cm}{1}

Для успешного прохождения задания аппарат в автономном режиме должен:

\begin{itemize}
    \item Пройти двойные ворота – 7 (возможен штраф в 7 баллов).
    \item Пройти по полоске – 3 балла.
    \item Пройти по полоске – 3 балла.
    \item Сбросить маркер в корзину – 7 (возможен штраф в 7 баллов).
    \item Пройти по полоске – 3 балла.
    \item Выстрелить в мишень – 17 баллов.
    \item Пройти по полоске – 3 балла.
    \item Всплыть в обруче – 7 баллов (возможен штраф в 3 балла).
\end{itemize}

\subsubsection*{Двойные ворота}

\putImgWOCaption{5cm}{2}

Пройти в зеленую часть ворот. Ворота разбиты на два сектора – красный и зеленый. Верхний край ворот совпадает с поверхностью воды. Расстояние от старта до ворот варьируется в зависимости от тестовой сцены. За прохождение красной части ворот начисляются 7 штрафных баллов, за прохождение зеленой части ворот начисляется 7 баллов.

\begin{itemize}
    \item Взаимное расположение красной и зеленой части ворот может меняться. 
\end{itemize}

\subsubsection*{Полоска}

\putImgWOCaption{5cm}{3}

Пройти по полоске. Цвет полоски оранжевый. Полоска служит для определения направление на следующие задание. Направление полоски может варьироваться от   -80 градусов до 80 градусов относительно начального курса аппарата. Расстояние от задания и направление полоски варьируется в зависимости от сцены. Полоска расположена на дне бассейна.

\begin{itemize}
    \item После ворот расположено две полоски, после красной и после зеленой части ворот, баллы начисляются только за одну полоску.
    \item После корзин расположено две полоски, после красной и после зеленой корзины, Баллы начисляются только за прохождение одной полоски.
\end{itemize}

\subsubsection*{Корзины}

\putImgWOCaption{5cm}{4}

Сбросить маркеры в зеленую корзину. Необходимо сбросить маркер синего цвета в зеленую корзину. Корзина имеет белые бортики и цветное дно. Корзины две – одна зеленого цвета, а вторая красного. Корзины расположены недалеко друг от друга. За сброс маркера в зеленую корзину начисляется 7 баллов. За сброс маркеров в красную корзину начисляется 7 штрафных баллов.  

\begin{itemize}
    \item Взаимное расположение корзин может меняться.
    \item Линия перед корзинами указывает на «центр» между корзинами.
    \item Бортики корзин соприкасаются.
\end{itemize}

\subsubsection*{Мишень}

\putImgWOCaption{5cm}{5}

Выстрелить в мишень. Необходимо выстрелить маркером в отверстие мишени. Цвет мишени красный. Отверстие квадратной формы имеет желтое обрамление. Отверстие может находится в любом из четырех углов красного прямоугольника. За попадание в отверстие начисляется 17 баллов. Штрафные Баллы в этом задании не начисляются.  Перед мишенью находится полоска, указывающая на треугольник и обруч.  Позиция  треугольника и обруча может взаимно меняться с позицией мишени. Высота  мишени относительно дна может меняться.

\subsubsection*{Тругольник и обруч}

\putImgWOCaption{5cm}{6}

Всплыть над треугольником в обруче. Центр треугольника совпадает с центром обруча. Треугольник размещен в квадрате, цвет треугольника белый, квадрата черный. Цвет обруча синий. За всплытие в обруче начисляется 7 баллов. За касание обруча начисляется 3 штрафных балла. Позиция  треугольника и обруча может взаимно меняться с позицией мишени.

\markSection

Двойные ворота: 
\begin{itemize}
    \item Баллы: 7
    \item Штрафные баллы: 7
    \item Описание штрафа: проход сквозь красную часть ворот
\end{itemize}

Полоска:
\begin{itemize}
    \item Баллы: 3
    \item Штрафные баллы: -
    \item Описание штрафа: -
\end{itemize}

Полоска: 
\begin{itemize}
    \item Баллы: 3
    \item Штрафные баллы: -
    \item Описание штрафа: -
\end{itemize}

Корзины:
\begin{itemize}
    \item Баллы: 7
    \item Штрафные баллы: 7
    \item Описание штрафа: сброс в красную корзину
\end{itemize}

Полоска:
\begin{itemize}
    \item Баллы: 3
    \item Штрафные баллы: -
    \item Описание штрафа: -
\end{itemize}

Мишень:
\begin{itemize}
    \item Баллы: 17
    \item Штрафные баллы: -
    \item Описание штрафа: -
\end{itemize}

Полоска:
\begin{itemize}
    \item Баллы: 3
    \item Штрафные баллы: -
    \item Описание штрафа: -
\end{itemize}

Всплыть в обруче:
\begin{itemize}
    \item Баллы: 7
    \item Штрафные баллы: 3
    \item Описание штрафа: Касание обруча
\end{itemize}