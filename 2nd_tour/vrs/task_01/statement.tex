\assignementTitle{}{20}

\subsubsection*{Конфигурация первой сцены}

\putImgWOCaption{5cm}{1}

Для успешного прохождения задания аппарат в автономном режиме
должен:

\begin{itemize}
    \item Пройти по полоске – 3 балла. 
    \item Пройти в ворота – 7 баллов.
    \item Пройти по полоске – 3 балла. 
    \item Всплыть в обруче – 7 баллов (возможен штраф в 3 балла).
\end{itemize}

\subsubsection*{Полоска}

\putImgWOCaption{5cm}{2}

Пройти по полоске. Цвет полоски оранжевый. Полоска
служит для определения направление на следующие задание. Направление полоски
может варьироваться от   -80 градусов до
80 градусов относительно начального курса аппарата. Расстояние от задания и
направление полоски варьируется в зависимости от сцены. Полоска расположена на
дне бассейна.

\subsubsection*{Ворота}

\putImgWOCaption{5cm}{3}

Пройти сквозь ворота. Цвет столбов и перекладины –
красный. Верхний край ворот совпадает с поверхностью воды. Расстояние от старта
до ворот варьируется в зависимости от тестовой сцены. За выполнение этого
задания начисляется 7 баллов. 

\subsubsection*{Треугольник и обруч}

\putImgWOCaption{5cm}{4}

Всплыть над треугольником в кольцо. Центр треугольника
совпадает с центром обруча. Треугольник размещен в квадрате, цвет треугольника
белый, квадрата черный. Цвет обруча синий. За всплытие в обруче начисляется 7.
За касание обруча начисляется 3 штрафных балла.

\markSection

Полоска: 
\begin{itemize}
    \item Баллы: 3
    \item Штрафные баллы: -
    \item Описание штрафа: -
\end{itemize}

Ворота:
\begin{itemize}
    \item Баллы: 7
    \item Штрафные баллы: -
    \item Описание штрафа: -
\end{itemize}

Полоска: 
\begin{itemize}
    \item Баллы: 3
    \item Штрафные баллы: -
    \item Описание штрафа: -
\end{itemize}

Всплыть в обруче:
\begin{itemize}
    \item Баллы: 7
    \item Штрафные баллы: 3
    \item Описание штрафа: Касание обруча
\end{itemize}