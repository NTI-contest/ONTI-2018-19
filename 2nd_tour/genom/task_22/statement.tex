\assignementTitle{Методы инженерной биологии}{6}{}

Генетическая инженерия позволяет выделять гены из одних организмов и вводить их в геном других организмов, получая при в результате рекомбинантные ДНК и трансгенные организмы. Методы генной инженерии [2] и редактирования генома [3] сегодня широко используются в синтетической биологии. 

Выберите наиболее корректные варианты пропущенных фраз в тексте.

Методы генетической инженерии широко используют различные (белки / гены / молекулы / ферменты / фрагменты) 
метаболизма ДНК. Среди прочих особенно следует отметить эндонуклеазы рестрикции, которые были открыты в начале 1970х годов. Данные ферменты используются 
(бактериями / бактериофагами / вирусами / грибами / растениями / эукариотами) 
для защиты от чужеродной ДНК. Система рестрикции-модификации распознает молекулы ДНК, которые неметилированы, и вносит в них 
(двуцепочечный разрыв / делецию / изменения / одноцепочечный разрыв / сайт рестрикции). Если молекула ДНК содержит метилированные нуклеотиды только в одной цепи, ферменты системы 
(амплификации / деметилирования / модификации / репарации / рестрикции) 
метилируют ее по второй цепи. Таким образом, клетка хозяина разрушает чужеродную ДНК и сохраняет свою. Открытие эндонуклеаз рестрикции привело к развитию методов генной инженерии. 

Сходным образом клетки (архей и бактерий / всех живых организмов / грибов и животных / прокариот и эукариот / растений и животных / растений и грибов) 
используют повторяющиеся последовательности, разделенные уникальными последовательностями, заимствованными из 
(гомологичных последовательностей ДНК / окружающей среды / фрагментов собственного генома / чужеродной ДНК). Такие короткие палиндромные повторы, регулярно расположенные группами, называют CRISPR (от англ. - clustered regularly interspaced short palindromic repeats). Для практического применения данной системы требуется также специальный белок Cas9, который вносит двуцепочечный разрыв в 
(гена / ДНК / РНК / эндонуклеазы) в области формирования гетеродуплекса с (ДНК / РНК), синтезированной на матрице CRISPR. Таким образом, две природные системы защиты клеток от чужеродной ДНК широко используются в генетической инженерии для переноса генов между организмами и в методе редактирования генома для коррекции имеющихся последовательностей ДНК. 

\subsubsection*{Рекомендуемая литература}

\begin{enumerate}
    \item Генетическая инженерия \url{https://biomolecula.ru/articles/12-metodov-v-kartinkakh-gennaia-inzheneriia-chast-i-istoricheskaia}
    \item "Синтетическая биология" Статья в журнале "Наука из первых рук" \url{https://scfh.ru/papers/sinteticheskaya-biologiya/}
    \item CRISPR \url{https://ru.wikipedia.org/wiki/CRISPR}
\end{enumerate}

\explanationSection

В некоторых случаях возможно несколько правильных ответов.

\answerMath{ферменты; бактериями; двуцепочечный разрыв; модификации; 
архей и бактерий; чужеродной ДНК; ДНК; РНК.}