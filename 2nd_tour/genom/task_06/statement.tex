\assignementTitle{}{4}{}

Искусственные плазмиды активно используют в генетической инженерии в качестве векторов, в которые клонируют гены, кодирующие белки, представляющие интерес для исследований. Для работы с плазмидами существует несколько программ для ПК, например, бесплатная SnapGene Viewer [1].
Молекулярные биологи активно используют плазмиду pBluescript для отбора успешно трансформированных клонов [2]. Карту этой плазмиды для работы в программе SnapGene Viewer можно скачать по ссылке [3]. 

Для решения задач в области генетической инженерии широко используют эндонуклеазы рестрикции - ферменты, позволяющие "разрезать" двуцепочечную ДНК [4]. Рестриктазы позволяют "вырезать" гены из одного источника и далее клонировать полученные последовательности в различные векторы, в том числе, плазмидные. Информация о специфичности эндонуклеаз рестрикции, а также сайты рестрикции имеется в программе SnapGene Viewer.

Используя программу SnapGene Viewer (или аналогичную) [1], карту плазмиды pBluescript [3], соотнесите эндонуклеазы рестрикции (PvuII, BstBAI, RsaI, EcoICRI, SspI) и фрагменты, которые будут получены при их действии на данную плазмиду. 

\begin{multicols}{2}
    {
        \begin{enumerate}
            \item PvuII + BstBAI
            \item RsaI + EcoICRI
            \item SspI + RsaI
            \item PvuII + SspI
            \item EcoICRI + BstBAI
        \end{enumerate}
    }
    {
        \begin{enumerate}
            \item[a.] 130 + 448 + 510 + 1873
            \item[б.] 302 + 448 + 2211
            \item[в.] 530 + 2431
            \item[г.] 130 + 324 + 636 + 1871
            \item[д.] 102 + 1090 + 1769
        \end{enumerate}
    }
    
\end{multicols}

\subsubsection*{Рекомендуемая литература}

\begin{enumerate}
    \item Программа SnapGene Viewer \url{http://www.snapgene.com/products/snapgene_viewer/} При установке рекомендуем выбрать Sibenzyme верхней строчкой поставщика эндонуклеаз рестрикции. 
    \item Информация о плазмиде pBluescript\\ \url{https://en.wikipedia.org/wiki/PBluescript}
    \item Ссылка для скачивания файла pBluescript II SK (+) для работы в программе SnapGene Viewer \url{https://www.snapgene.com/resources/plasmid_files/basic_cloning_vectors/pBluescript_II_SK(+)/}
    \item Статья о рестриктазах в Википедии \url{https://ru.wikipedia.org/wiki/%D0%AD%D0%BD%D0%B4%D0%BE%D0%BD%D1%83%D0%BA%D0%BB%D0%B5%D0%B0%D0%B7%D1%8}\linebreak \url{B_%D1%80%D0%B5%D1%81%D1%82%D1%80%D0%B8%D0%BA%D1%86%D0%B8%D0%B8}
\end{enumerate}

\explanationSection

Следует определить сайты рестрикции, длины фрагментов ДНК, получающихся в результате рестрикции, соотнести длины фрагментов и комбинации рестриктаз.

\answerMath{1 - б, 2 - д, 3 - г, 4 - а, 5 - в.}