\assignementTitle{Закономерности полимеразной цепной реакции}{6}{}

Для успешной специфичной наработки фрагмента ДНК в ходе ПЦР важны все компоненты реакционной смеси. В литературе подробно описаны принципы подбора праймеров, закономерности специфичности ПЦР, а также многие другие особенности [1, 2, 3].

Выберите корректные суждения о полимеразной цепной реакцииДНК на матрице мРНК играет важную роль в биотехнологии, для экспрессии определенных генов, в частности, в бактериальных клетках. 

\begin{enumerate}
    \item увеличение концентрации ионов Mg2+ приводит к снижению специфичности ПЦР
    \item ДНК-полимераза может использовать АТР в качестве субстрата при синтезе дочерней цепи
    \item для увеличения специфичности ПЦР в пробирки иногда добавляют минеральное масло
    \item проведение более 50 циклов ПЦР невозможно, так как снижается процессивность ДНК-полимеразы и/или заканчиваются субстраты ДНК-полимеразы
    \item ДНК-полимераза добавляет нуклеотиды к 5'-концу прямого праймера
    \item укорочение праймера приводит к снижению температуры отжига
    \item стандартная Taq-полимераза эффективно амплифицирует протяженных фрагменты ДНК длиной более 10 тысяч пар нуклеотидов
    \item увеличение длины праймера приводит к повышению специфичности ПЦР
    \item отсутствие спаривания на 5'-конце праймера не приводит к значительному снижению уровня наработки продукта ПЦР
\end{enumerate}

\subsubsection*{Рекомендуемая литература}

\begin{enumerate}
    \item Стратегия подбора праймеров \url{http://www.rfbr.ru/rffi/ru/books/o_1781847}
    \item Стратегия подбора праймеров \url{http://www.rfbr.ru/rffi/ru/books/o_1781847}
    \item ПЦР \url{https://cyberpedia.su/2x6e17.html}
\end{enumerate}

\answerMath{1, 4, 6, 8, 9.}