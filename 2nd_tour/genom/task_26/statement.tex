\assignementTitle{Ферменты метаболизма нуклеиновых кислот}{6}{}

Ферменты метаболизма ДНК используют в генетической инженерии для получения двуцепочечных молекул ДНК из одноцепочечных, а также для синтеза ДНК на матрице РНК.

Выберите наиболее корректные варианты пропущенных фраз в тексте.

К ферментам матричного синтеза нуклеиновых кислот относят ДНК-зависимые ДНК-полимеразы: это ДНК-полимераза I из E. coli, ее фрагмент, так называемый 
(фрагмент синоним / фрагмент Кленова / фрагмент Оказаки / фрагмент 67 кДа), ДНК-полимеразу фага Т4, Taq-полимеразу (из 
(Taq-зонда / Taq-фрагмента / Thermus aquaticus / Thermos / Thermo Scientific Fisher)). Все эти ферменты в присутствии ионов 
(натрия / калия / кальция / магния / железа) осуществляют синтез ДНК, комплементарной матричной цепи ДНК и для функционирования требуют наличия затравки (праймера) со свободным 
(2' / 3' / 4' / 5')-ОН-концом, комплементарного (соответствующей / матричной / синтезируемой / участку) ДНК. Фермент, синтезирующий ДНК на матрице РНК, называют РНК-зависимой ДНК-полимеразой, или 
(интегразой / ревертазой / обратной транскриптазой / ДНКазой / РНКазой / ДНК-РНКазой). Так же, как и обычные ДНК-полимеразы, РНК-зависимые ДНК-полимеразы функционируют только при наличии 
(матричной РНК / праймера / завтрака / затравки /олиго-ДНК), комплементарной РНК-матрице. Эти ферменты находят применение в синтезе двуцепочечных ДНК, комплементарных мРНК, так называемых 
(кРНК / мДНК / рРНК / кДНК /дРНК). Процесс синтеза ДНК на матрице мРНК играет важную роль в биотехнологии, для экспрессии определенных генов, в частности, в бактериальных клетках. 

\subsubsection*{Рекомендуемая литература}

\begin{enumerate}
    \item Генетическая инженерия \url{https://biomolecula.ru/articles/12-metodov-v-kartinkakh-gennaia-inzheneriia-chast-i-istoricheskaia}
    \item "Синтетическая биология" Статья в журнале "Наука из первых рук" \url{https://scfh.ru/papers/sinteticheskaya-biologiya/}
    \item CRISPR \url{https://ru.wikipedia.org/wiki/CRISPR}
\end{enumerate}