\assignementTitle{Определение массы фрагмента ДНК}{3}{}

В результате разрезания плазмиды pBR322 рестриктазами HindII и BstBAI образовалось несколько фрагментов.

Для работы с плазмидами существует несколько программ для ПК, например, бесплатная SnapGene Viewer [1].

Карту плазмиды pBR322 для работы в программе SnapGene Viewer можно скачать по ссылке [2].

Определите массу самого длинного фрагмента плазмиды pBR322, образовавшегося после гидролиза рестриктазами HindII и BstBAI, если известно, что масса исходной плазмиды составляла 2019 нг.

Ответ округлите до целого числа. 


\subsubsection*{Рекомендуемая литература}

\begin{enumerate}
    \item Ссылка для скачивания программы SnapGene Viewer \url{https://www.snapgene.com/products/snapgene_viewer/}
    \item Ссылка для скачивания файла pBR322 для работы в программе SnapGene Viewer \url{http://www.snapgene.com/resources/plasmid_files/basic_cloning_vectors/pBR322/}
\end{enumerate}