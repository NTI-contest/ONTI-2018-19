\assignementTitle{Секвенирование по Сэнгеру}{4}{}

Севенирование позволяет «побуквенно» прочитать нуклеотидную  последовательность ДНК. Наиболее распространенный метод секвенирования, который используется в рутинной лабораторной практике, был изобретен Фредериком Сэнгером [1, 2, 3]. Данный метод также называется методом терминирующих оснований.

Ключевым моментом является использование дидезоксинуклеозидтрифосфатов (ddNTPs), которые не имеют 3’-ОН группы для образования связи со следующей фосфатной группой. Поэтому в результате включения подобного дигидроксинуклеотида синтез комплементарной цепи ДНК терминируется. При проведении анализа для каждого образца ДНК готовится 4 реакционных смеси, которые содержат смесь четырех dNTP, ДНК-полимеразу и один из терминирующих ddNTP.

Результаты реакции визуализируют с помощью гель-электрофореза и по набору полос восстанавливают исходную последовательность. 

"Прочитать" результаты гель-электрофореза и определить последовательность нуклеотидов в анализируемом образце ДНК.

\putImgWOCaption{5cm}{1}

Ответ привести в виде последовательности нуклеотидов, в направлении от 5'- к 3'-концу, латинскими буквами, без обозначений 3'-, 5'-, пробелов, например: TATTCTA

\subsubsection*{Рекомендуемая литература}

\begin{enumerate}
    \item Статья о секвенировании по Сэнгеру в википедии \url{https://ru.wikipedia.org/wiki/%D0%9C%D0%B5%D1%82%D0%BE%D0%B4_%D0%A1%D1%8D%D0%BD%D0%B3%D0%B5%D1%80%D0%B0}
    \item Видеоуроки о секвенировании по Сэнгеру на Степике \url{https://stepik.org/lesson/13696/step/7?unit=3835}
    \item Статья о методе Сэнгера на сайте "Биомолекула" \url{ https://biomolecula.ru/articles/metody-v-kartinkakh-sekvenirovanie-nukleinovykh-kislot}
\end{enumerate}

\explanationSection

\answerMath{}