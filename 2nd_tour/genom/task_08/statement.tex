\assignementTitle{Определение массы фрагмента ДНК}{2}{}

В результате разрезания плазмиды pBluescript (длина 2961 п.н.) рестриктазой SspI образовались два фрагмента.

Для работы с плазмидами существует несколько программ для ПК, например, бесплатная SnapGene Viewer [1].

Карту плазмиды pBluescript для работы в программе SnapGene Viewer можно скачать по ссылке [2].

Определите массу более длинного фрагмента плазмиды pBluescirpt, образовавшегося после рестрикции SspI, если известно, что масса исходной плазмиды составляла 2018 нг.

Ответ округлите до целого числа. 

\subsubsection*{Рекомендуемая литература}

\begin{enumerate}
    \item Ссылка для скачивания программы SnapGene Viewer \url{https://www.snapgene.com/products/snapgene_viewer/}
    \item Ссылка для скачивания файла pBluescript II SK (+) для работы в программе SnapGene Viewer \url{https://www.snapgene.com/resources/plasmid_files/basic_cloning_vectors/pBluescript_II_SK(+)/}
\end{enumerate}

\explanationSection

Следует определить сайты рестрикции, длины фрагментов ДНК, получающихся в результате рестрикции, соотнести длину фрагментов и массу исходной плазмиды.

\answerMath{1929.}