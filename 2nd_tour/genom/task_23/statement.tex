\assignementTitle{Поиск PAM}{6}{}

Механизм действия системы редактирования CRISPR/Cas9 включает несколько этапов:

\begin{enumerate}
    \item поиск и узнавание последовательности PAM (от англ. protospacer adjacent motif — мотив, смежный с протоспейсером)
    \item формирование дуплекса между направляющей РНК и целевым участком ДНК
    \item разрезание цепи ДНК нуклеазой Cas9 с образованием двуцепочечного разрыва
    \item репарация ДНК по механизму гомологичной рекомбинации или негомологичного соединения концов
\end{enumerate}

Найдите в выбранном участке гена все возможные варианты последовательности PAM. Представлена кодирующая цепь:

5'-AAGTTGTCTGATTTTTAAAACACTGATGCAGCTGGCCTCA-3'

Выберите корректные последовательности PAM из предложенных ниже.

\begin{enumerate}
    \item 5'-TGA
    \item 5'-TGG
    \item 5'-GTC
    \item 5'-GGG
    \item 5'-AGG
    \item 5'-CCG
    \item 5'-CGG
\end{enumerate}

\subsubsection*{Рекомендуемая литература}

\begin{enumerate}
    \item Статья в Википедии о методе CRISPR-Cas \url{https://ru.wikipedia.org/wiki/CRISPR}
    \item НЕМУДРЫЙ А.А., ВАЛЕТДИНОВА К.Р., МЕДВЕДЕВ С.П., ЗАКИЯН С.М. СИСТЕМЫ РЕДАКТИРОВАНИЯ ГЕНОМОВ TALEN И CRISPR/CAS - ИНСТРУМЕНТЫ ОТКРЫТИЙ // Acta Naturae. - 2014. - Т. 6. - С. 20.  \url{https://yadi.sk/i/0epvvqFx2SrMoQ}
    \item Статья на сайте Addgene \url{https://www.addgene.org/crispr/guide/}
\end{enumerate}

\explanationSection

Необходимо выбрать последовательности PAM, встречающиеся в представленной и комплементарной цепи ДНК.

\answerMath{2, 5.}