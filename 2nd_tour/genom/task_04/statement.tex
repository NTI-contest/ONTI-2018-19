\assignementTitle{}{5}

Данный текст сравнивает методы селекции и генной инженерии.

Выберите корректные по смыслу варианты. В некоторых местах возможно несколько правильных вариантов, выберите любой из них.

Метод селекции основан на (гибридизации / получении / разнообразии / скрещивании / элиминировании) 
особей с интересными для исследователя признаками и дальнейшем (изучении / размножении / отборе / элиминировании) 
полученных потомков. На следующем этапе обычно требуется повысить (гетерозиготность / гомозиготность / единообразие / продуктивность / разнообразие) 
полученных форм. Для этого у растений широко используют (вегетативное размножение / обработку колхицином / отбор наиболее приспособленных форм / самоопыление / эффект гетерозиса), 
а у животных (аутбридинг / возвратное скрещивание / инбридинг / отбор наиболее приспособленных форм / партеногенез). 
Таким образом, селекция позволяет оперировать исключительно генофондом (вида / одной популяции / организма / особи / человека). 
В отличие от селекции, методы генетической инженерии позволяют переносить гены между 
(ДНК разных видов / митохондриями и ядром / организмами разных видов / разными клетками / ядром и цитоплазмой). 
Считается, что генетическая инженерия появилась благодаря открытию в 1971 году 
(генетического кода / строения генома эукариот / структуры ДНК / ферментов-рестриктаз / универсальности генетического кода). 
При помощи генетической инженерии были созданы генно-модифицированные (породы пчел / сорта арбузов без косточек / устойчивые к сорнякам сорта сои / флуоресцентные аквариумные рыбки / штаммы сибирской язвы).