\assignementTitle{Анализ секвенограмм}{6}

Результаты секвенирования ДНК по методу Сэнгера (с использованием флуоресцентно меченых ддНТФ) могут быть представлены в виде "секвенограммы" - результата разделения фрагментов ДНК капиллярным электрофорезом. Для анализа этих файлов можно использовать бесплатные программы, рекомендованные производителем оборудования [1], а также универсальные программы, например SnapGene Viewer [2].

Используя программу для анализа результатов секвенирования, определите ген человека, последовательность которого была амплифицирована при помощи ПЦР и далее в реакции Сэнгера. Ответ представьте в виде четырех символов, соответствующих краткому обозначению гена, используйте латинские буквы (при необходимости, цифры).

\subsubsection*{Рекомендуемая литература}

\begin{enumerate}
    \item Бесплатная программа Chromas\\ \url{http://www.technelysium.com.au/Chromas265Setup.exe}
    \item SnapGene Viewer \url{http://www.snapgene.com/products/snapgene_viewer/}
    \item Ссылка для скачивания файла с секвенограммой с Яндекс-диска  \url{https://yadi.sk/d/9kOZjICHS2zVKw} 
\end{enumerate}