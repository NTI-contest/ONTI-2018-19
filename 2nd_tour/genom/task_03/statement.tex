\assignementTitle{}{5}{}

Смесь для проведения ПЦР состоит из нескольких компонентов. Перед началом эксперимента часто нужно сначала приготовить рабочий раствор. Обычно в лаборатории имеются стоковые (исходные) растворы компонентов, необходимых для проведения ПЦР.

Даны концентрации стоковых растворов

\begin{itemize}
    \item ДНК-полимераза, 1.5 ед/мкл
    \item смесь дезоксинуклеозидтрифосфатов, 10 мМ каждого
    \item прямой праймер, 6 мкМ
    \item обратный праймер, 3.75 мкМ
    \item матрица ДНК, 50 нг/мкл
    \item хлорид магния, 30 мМ
    \item Tween 20, $1.25\%$
    \item Трис рН 8.5, 0.3 М
    \item хлорид калия, 0.25 М
\end{itemize}

Определите, какой объем стоковых растворов и воды следует добавить в реакционную смесь, если известно, что конечный объем реакционной смеси 25 мкл.

Сопоставьте компоненты реакционной смеси и их финальные концентрации и объем данного компонента, который следует добавить в реакционную смесь для достижения заданной концентрации.

\begin{multicols}{2}
    {
        \begin{enumerate}
            \item вода деионизованная
            \item ДНК-полимераза, 0.03 ед/мкл
            \item нуклеозидтрифосфаты, 0.4 мМ каждого
            \item прямой праймер, 300 нМ
            \item обратный праймер, 300 нМ
            \item матрица ДНК, 4.5 нг/мкл
            \item хлорид магния, 3 мМ
            \item Tween 20, $0.15\%$
            \item Трис рН 8.5, 45 мМ
            \item хлорид калия, 40 мМ
        \end{enumerate}
    {
        \begin{enumerate}
            \item[а.] 4 мкл
            \item[б.] 3.75 мкл
            \item[в.] 2.0 мкл
            \item[г.] 1.25 мкл
            \item[д.] 1.0 мкл
            \item[е.] 3 мкл
            \item[ж.] 0.5 мкл
            \item[з.] 2.5 мкл
            \item[и.] 2.25 мкл
            \item[л.] 4.75 мкл
        \end{enumerate}
    }
    
\end{multicols}

\explanationSection

Следует вычислить разведение стокового раствора до финальной концентрации, определить объем добавляемого стокового раствора, соотнести объем и компонент.

\answerMath{1 - л, 2 - ж, 3 - д, 4 - г, 5 - в, 6 - и, 7 - з, 8 - е, 9 - б, 10 - а.}