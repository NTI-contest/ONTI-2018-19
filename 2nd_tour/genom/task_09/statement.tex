\assignementTitle{Электрофорез ДНК}{2}{}

Для визуализации ДНК-фрагментов а также их разделения в зависимости от длины использует гель-электрофорез [1].

Для определения длины полученных ДНК фрагментов используются коммерческие растворы ДНК, которые содержат фрагменты ДНК молекул строго определенных длин. Такие растворы называется «маркерами длин ДНК-фрагментов» («DNA ladder», «линейка», «маркеры ДНК») [2].

На иллюстрации приведена фотография геля, на который был нанесен маркер ДНК (слева) и образец ДНК (справа), и расшифровка длин ДНК фрагментов маркера.

\putImgWOCaption{8cm}{1}

Необходимо определить примерную длину каждого из трех фрагментов ДНК. Соотнесите фрагменты и их длину в п. н.

\begin{multicols}{2}
    {
        \begin{enumerate}
            \item Фрагмент С
            \item Фрагмент B
            \item Фрагмент A
        \end{enumerate}
    }
    {
        \begin{enumerate}
            \item[a.] 3000-4000 п.н.
            \item[б.] 750-1000 п.н.
            \item[в.] 1500-2000 п.н.  
        \end{enumerate}
    }
    
\end{multicols}


\subsubsection*{Рекомендуемая литература}

\begin{enumerate}
    \item Подробнее о методе \url{https://en.wikipedia.org/wiki/Gel_electrophoresis_of_nucleic_acids}
    \item Статья о маркерах молекулярной массы \url{https://en.wikipedia.org/wiki/Molecular-weight_size_marker}
\end{enumerate}

\explanationSection

Следует соотнести длины полученных фрагментов ДНК и длины фрагментов ДНК маркера.

\answerMath{1 - б, 2 - в, 3 - а.}