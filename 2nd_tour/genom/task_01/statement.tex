\assignementTitle{}{1}{}

Лаборант-исследователь подготовил реакционную смесь для полимеразной цепной реакции (ПЦР), добавил в пробирку следующие компоненты: 

\begin{itemize}
    \item Двухкратный буфер для ПЦР (с Mg2+)
    \item ДНК-матрица
    \item Прямой праймер
\end{itemize}

Затем лаборант отвлекся на смс-сообщение, а когда вернулся к протоколу, задумался, каких компонентов не хватает в реакционной смеси.

Определите, какие компоненты нужно добавить в реакционную смесь

\begin{enumerate}
    \item дезоксигуанозинтрифосфат
    \item РНК-матрица
    \item РНК-зависимая ДНК-полимераза
    \item дезокситимидинтрифосфат
    \item дезоксиаденозинтрифосфат
    \item дезоксицитидинтрифосфат
    \item ДНК-зависимая РНК-полимераза
    \item ДНК-зависимая ДНК-полимераза
    \item обратный праймер
    \item дезоксиуридинтрифосфат
\end{enumerate}

\subsubsection*{Редомендуемая литература}

\begin{enumerate}
    \item Рекомендации по постановке ПЦР \url{http://evrogen.ru/kit-user-manuals/Evrogen-PCR-recommendation.pdf}
    \item 12 методов в картинках: полимеразная цепная реакция \url{https://biomolecula.ru/articles/metody-v-kartinkakh-polimeraznaia-tsepnaia-reaktsiia}
\end{enumerate}

\explanationSection

Необходимо выбрать компоненты, которые являются субстратами ДНК-полимеразы и принимают участие в репликации ДНК.

\answerMath{1, 4, 5, 6, 8.}