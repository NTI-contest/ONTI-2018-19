\assignementTitle{Дизайн праймеров}{4}{}

Полимеразная цепная реакция является исключительно важным современным методом молекулярной биологии. 
Принцип метода изложен в работах [1], [2]. Для амплификации участка ДНК методом ПЦР требуется заказать 
прямой и обратный праймер [3].

Ген PAX6 относят к семейству генов PAX [4], которые кодируют тканеспецифичные факторы транскрипции. 
Мутации в данном гене приводят к нарушениям строения органов зрения. Ортолог (гомолог) данного гена 
у дрозофилы называется eyless. Последовательность данного гена в базе данных GeneBank имеет идентификатор
 NG\_008679.1 [5]

Последовательность праймеров принято записывать от 5'-конца к 3'-концу. Определите последовательность прямого праймера длиной 20 нуклеотидов, если в качестве обратного праймера был использован следующий \linebreak олигонуклеотид 5'- CCTAGGCCGCCGAGAGGGCT-3', и известно, что длина ПЦР-фрагмента равна 210 пар нуклеотидов.

Введите последовательность прямого праймера латинскими буквами, без знаков 5'-, 3'-, и пробелов.

\subsubsection*{Рекомендуемая литература}

\begin{enumerate}
    \item Рекомендации по постановке ПЦР \url{http://evrogen.ru/kit-user-manuals/Evrogen-PCR-recommendation.pdf}
    \item 12 методов в картинках: полимеразная цепная реакция \url{https://biomolecula.ru/articles/metody-v-kartinkakh-polimeraznaia-tsepnaia-reaktsiia}
    \item Статья о праймерах в английской Википедии. Внимательно изучите иллюстрацию \url{https://en.wikipedia.org/wiki/Primer_(molecular_biology)}
    \item Статья о генах PAX в Википедии \url{https://ru.wikipedia.org/wiki/%D0%93%D0%B5%D0%BD%D1%8B_Pax}
    \item Интерфейс для поиска \url{https://www.ncbi.nlm.nih.gov/nuccore}
\end{enumerate}