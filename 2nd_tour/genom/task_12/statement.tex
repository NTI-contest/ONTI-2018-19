\assignementTitle{Продукты рестрикции плазмидной ДНК}{4}{}

Искусственные плазмиды активно используют в генетической инженерии в качестве векторов, в которые клонируют гены, кодирующие белки, представляющие интерес для исследований. Для работы с плазмидами существует несколько программ для ПК, например, бесплатная SnapGene Viewer [1].
Плазмиду pBR322 активно использовали в 1980-е годы. Карту этой плазмиды для работы в программе SnapGene Viewer можно скачать по ссылке [2]. 

Для решения задач в области генетической инженерии широко используют эндонуклеазы рестрикции - ферменты, позволяющие "разрезать" двуцепочечную ДНК. Рестриктазы позволяют "вырезать" гены из одного источника и далее клонировать полученные последовательности в различные векторы, в том числе, плазмидные. Информация о специфичности эндонуклеаз рестрикции, а также сайты рестрикции плазмид имеются в программе SnapGene Viewer.

Используя программу SnapGene Viewer (или аналогичную) [1], карту плазмиды pBR322 [2], соотнесите эндонуклеазы рестрикции (EcoRI, NruI и PstI) и фрагменты, которые будут получены при их действии на данную плазмиду. Фотография геля с продуктами рестрикции приведена на рисунке. Числа около фрагментов на дорожках В-Ж соответствуют расчетным длинам продуктов рестрикции.

\putImgWOCaption{10cm}{1}

        \begin{enumerate}
            \item A
            \item Б
            \item В
            \item Г
            \item Д
            \item Е
            \item Ж
        \end{enumerate}

        \begin{enumerate}
            \item[a.] pBR322
            \item[б.] Маркер молекулярной массы ДНК
            \item[в.] pBR322 + PstI + EcoRI
            \item[г.] pBR322 + PstI + NruI
            \item[д.] pBR322 + EcoRI + NruI
            \item[е.] pBR322 + PstI + NruI + EcoRI
            \item[ж.] pBR322 + PstI
        \end{enumerate}

\subsubsection*{Рекомендуемая литература}

\begin{enumerate}
    \item Программа SnapGene Viewer \url{http://www.snapgene.com/products/snapgene_viewer/} При установке рекомендуем выбрать Sibenzyme верхней строчкой поставщика эндонуклеаз рестрикции. 
    \item Ссылка для скачивания файла pBR322 для работы в программе SnapGene Viewer \url{http://www.snapgene.com/resources/plasmid_files/basic_cloning_vectors/pBR322/}
\end{enumerate}

\explanationSection

Следует определить сайты рестрикции, длины фрагментов ДНК, получающихся в результате рестрикции, соотнести длины фрагментов и комбинации рестриктаз.

\answerMath{1 - б, 2 - а, 3 - ж, 4 - д, 5 - г, 6 - в, 7 - е.}