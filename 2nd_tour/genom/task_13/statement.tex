\assignementTitle{}{4}

Данный текст посвящен методу ПЦР.

Выберите варианты ответа, максимально соответствующие по смыслу.

Для проведения полимеразной цепной реакции (ПЦР) необходимо, в первую очередь, выбрать участок 
(антигена / гена / генетического кода / молекулы / рибосомы) и сконструировать 
(азотистые основания / молекулы / нуклеотиды / олигонуклеотиды / праймеры), комплементарные участкам выбранного гена. Праймеры обычно выбирают длиной около 
(2 / 12 / 20 / 200 / 2000) (азотистых оснований / нуклеотидов / олигонуклеотидов / пар нуклеотидов / пар олигонуклеотидов). Положение праймеров задаёт 
(длину / массу / пространство / температуру плавления / ширину) конечного продукта. Затем нужно подготовить матрицу ДНК. Далее в пробирке смешивают все необходимые компоненты: ДНК-матрицу, праймеры, реакционный буфер, содержащий 
(ионы калия / ионы кальция / ионы магния / ионы натрия), (дезоксинуклеозидтрифосфаты / дезокситрифосфаты / нуклеозидтрифосфаты / трифосфаты), фермент ДНК-зависимую ДНК-полимеразу. На приборе, который называется 
(амплификатор / термостат / хроматограф / центрифуга) задают протокол ПЦР, который обычно состоит из 
(10-15 / 15-25 / 30-40 / 40-60 / 60-80)  циклов, каждый из которых содержит последовательно стадии 
(денатурации / отжига праймеров / плавления / поглощения / ренатурации / репликации / терминации / элонгации), 
(денатурации / достройки / плавления / отжига праймеров / удлинения / элонгации) и 
(денатурации / комплементации / поглощения / ориентации / отжига праймеров / элонгации).
\subsubsection*{Рекомендуемая литература}

\begin{enumerate}
    \item Рекомендации по постановке ПЦР \url{http://evrogen.ru/kit-user-manuals/Evrogen-PCR-recommendation.pdf}
    \item 12 методов в картинках: полимеразная цепная реакция \url{https://biomolecula.ru/articles/metody-v-kartinkakh-polimeraznaia-tsepnaia-reaktsiia}
\end{enumerate}