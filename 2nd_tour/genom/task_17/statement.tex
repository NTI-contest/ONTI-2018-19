\assignementTitle{Определение концентрации компонентов в смеси}{2}{}

Для постановки секвенирующей реакции Сэнгера с флуоресцентно мечеными дидезоксинуклеозидтрифосфатами часто используют продукт, наработанный в ПЦР и один из праймеров - прямой или обратный. Перед началом эксперимента необходимо приготовить рабочий раствор. Лаборант приготовил реакционную смесь из стоковых (исходных) растворы компонентов.

Концентрации стоковых растворов

\begin{itemize}
    \item прямой праймер, 6 мкМ
    \item хлорид магния, 0,1 М
    \item Трис рН 8.5, 1 М
    \item хлорид калия, 1 М
    \item матрица ДНК, 50 пМ
\end{itemize}

Определите конечную концентрацию компонентов в реакционной смеси, если известно, что к реакционной смеси добавили 5 мкл 10-кратного раствора смеси дезоксинуклеозидтрифосфатов с флуоресцентно мечеными дезоксинуклеозидтрифосфатами.

Сопоставьте компоненты реакционной смеси, добавленный объем (слева) и их финальные концентрации (справа).

\begin{multicols}{2}
    {
        \begin{enumerate}
            \item прямой праймер, 6 мкМ
            \item хлорид магния, 0,1 М
            \item Трис рН 8.5, 1 М
            \item хлорид калия, 1 М
            \item матрица ДНК, 50 пМ
        \end{enumerate}
    }

    {
        \begin{enumerate}
            \item[а.] 0.15 мкМ
            \item[б.] 2 мМ
            \item[в.] 1 пМ
            \item[г.] 40 мМ
            \item[д.] 50 мМ
        \end{enumerate}
    }
    
\end{multicols}

\explanationSection

Следует вычислить финальную концентрацию каждого компонента исходя из стоковой концентрации и добавленного объема. Соотнести компонент и конечную концентрацию.

\answerMath{1 - а, 2 - в, 3 - б, 4 - д, 5 - г.}