\assignementTitle{}{4}

Приведенный ниже отрывок текста посвящен использованию метода редактирования генома CRISPR-Cas9 [1, 2].

Выберите корректные по смыслу варианты из выпадающего списка. В некоторых местах возможно несколько правильных вариантов, выберите любой из них.

Метод редактирования генома, использующий систему CRISPR/Cas9, может быть использован для 
(анализа родословной / выключения заранее выбранных генов / разработки штаммов бактерий, устойчивых к любым антибиотикам /
создания новых видов животных / увеличения продуктивности экосистем). 
Молекулярный процесс редактирования гена включает несколько стадий. На первой происходит связывание комплекса белка-нуклеазы Cas9 и направляющей РНК с
(антипараллельной / антисмысловой / идентичной / коллинеарной / комплементарной) целевой областью ДНК. Далее 
(белок / фермент / нуклеаза / рестриктаза / протеаза) Cas9 осуществляет 
(разрезание / расщепление / диссоциацию / стабилизацию / сопоставление) нуклеотидной последовательности ДНК в области непосредственного взаимодействия РНК-ДНК с образованием двуцепочечного разрыва. На последнем этапе ферменты 
(репарации / репликации / ретардации / лигирования / рестрикции) ДНК восстанавливают образующийся разрыв с формированием мутаций в виде делеций или инсерций.

\subsubsection*{Редомендуемая литература}

\begin{enumerate}
    \item Редактирование генома с CRISPR/Cas9 \url{https://postnauka.ru/faq/59807}
    \item Статья в Википедии о CRISPR \url{https://ru.wikipedia.org/wiki/CRISPR}
    \item Статья о возможностях редактирования генома \url{https://chrdk.ru/sci/ot_bezobidnoi_kartoshki_do_biologicheskogo_oruzhiya}
\end{enumerate}