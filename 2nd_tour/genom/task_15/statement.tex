\assignementTitle{Приготовление разведений растворов}{2}{}

Концентрацию компонентов в растворе обозначают различными способами. Широко используют 
количественные характеристики, например, г/л, моль/л (М), \% и другие [1]. 
Например, при приготовлении растворов для нанесения образцов на гель, или при 
расчете компонентов смеси для ПЦР часто используют кратные растворы (2х, 4х, 5х, 10х). 
Например, для приготовления 100 мл однократного раствора (1х), нужно взять 50 мл двукратного раствора 
(2х) и добавить 50 мл воды или другого раствора.

Сколько шестикратного (6x) буферного раствора для нанесения пробы на гель необходимо добавить в 25 мкл реакционной смеси для достижения в реакционной смеси однократной (1x) концентрации буферного раствора? Ответ введите виде натурального целого числа без "мкл". 

\subsubsection*{Рекомендуемая литература}

\begin{enumerate}
    \item Статья в википедии о концентрации смеси \url{https://ru.wikipedia.org/wiki/%D0%9A%D0%BE%D0%BD%D1%86%D0%B5%D0%BD%D1%82%D1%80%D0%B0%D1%86%D0%B8} \linebreak \url{%D1%8F_%D1%81%D0%BC%D0%B5%D1%81%D0%B8}
\end{enumerate}

\explanationSection

Необходимо определить объем 6х буферного раствора, который следует добавить до достижения 1х концентрации.

\answerMath{5.}