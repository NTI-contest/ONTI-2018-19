\assignementTitle{Продукты рестрикции плазмидной ДНК}{3}{}

Искусственные плазмиды активно используют в генетической инженерии в качестве векторов, в которые клонируют гены, кодирующие белки, представляющие интерес для исследований. Для работы с плазмидами существует несколько программ для ПК, например, бесплатная SnapGene Viewer [1].
Плазмиду pBluescript активно использовали в 1980-е годы. Карту этой плазмиды для работы в программе SnapGene Viewer можно скачать по ссылке [2]. 

Для решения задач в области генетической инженерии широко используют эндонуклеазы рестрикции - ферменты, позволяющие "разрезать" двуцепочечную ДНК. Рестриктазы позволяют "вырезать" гены из одного источника и далее клонировать полученные последовательности в различные векторы, в том числе, плазмидные. Информация о специфичности эндонуклеаз рестрикции, а также сайты рестрикции плазмид имеются в программе SnapGene Viewer.

Используя программу SnapGene Viewer (или аналогичную) [1], карту плазмиды pBluescript [2], соотнесите длины фрагментов плазмидной ДНК, которые будут получены при действии рестриктаз Acc16I, BstBAI, HindII, PvuII, SspI и ZrmI с пробирками, которые содержат реакционные смеси с разными рестриктазами.

\begin{multicols}{2}
    {
        \begin{enumerate}
            \item Acc16I + HindII
            \item HindII + SspI
            \item SspI + ZrmI
            \item ZrmI + PvuII
            \item PvuII + BstBAI
            \item BstBAI + Acc16I
        \end{enumerate}
    }

    {
        \begin{enumerate}
            \item[а.] 302, 448, 2211
            \item[б.] 448, 964, 1549
            \item[в.] 197, 1172, 1592
            \item[г.] 130, 324, 2507
            \item[д.] 130, 657, 2174
            \item[е.] 252, 920, 1789            
        \end{enumerate}
    }
    
\end{multicols}

\subsubsection*{Рекомендуемая литература}

\begin{enumerate}
    \item Программа SnapGene Viewer \url{http://www.snapgene.com/products/snapgene_viewer/} При установке рекомендуем выбрать Sibenzyme верхней строчкой поставщика эндонуклеаз рестрикции. 
    \item Ссылка для скачивания файла плазмиды pBluescript\\ \url{https://www.snapgene.com/resources/plasmid_files/basic_cloning_vectors/pBluescript_II_SK(+)/}
\end{enumerate}

\explanationSection

Следует определить сайты рестрикции, длины фрагментов ДНК, получающихся в результате рестрикции, соотнести длины фрагментов и комбинации рестриктаз.

\answerMath{1 - в, 2 - д, 3 - г, 4 - б, 5 - а, 6 - е.}