\assignementTitle{Определение последовательности ДНК по секвенограмме}{3}{}

Современный вариант исполнения метода Сэнгера [1, 2] предполагает использование автоматических капиллярных ДНК-анализаторов, которые определяют наличие флуоресцентно-меченых мономеров-терминаторов в составе продуктов реакции. Затем с помощью программного обеспечения прибора устанавливают соответствия между длиной продуктов и положением конкретного нуклеотида. В итоге получается «секвенограмма», аналогичной той, что представлена на рисунке:

Разные цвета обозначают положения различных нуклеотидов (четыре цвета – четыре нуклеотида):

\begin{enumerate}
    \item Зеленая линия – положения A
    \item Красная линия – положения Т 
    \item Черная линия – положения G
    \item Синяя линия –  положения С
\end{enumerate}

\putImgWOCaption{15cm}{1}

Определите последовательность и с помощью сервиса Blast [3] определите, какому гену она принадлежит.
Ответ должен содержать краткое обозначение гена в виде трех латинских букв и одной цифры без пробела, без дефиса или других знаков препинания (формат XYZ9)

\subsubsection*{Рекомендуемая литература}

\begin{enumerate}
    \item Биомолекула: 12 методов в картинках \url{https://biomolecula.ru/articles/metody-v-kartinkakh-sekvenirovanie-nukleinovykh-kislot}
    \item Видеолекция на степике о методе Сэнгера \url{https://stepik.org/lesson/13696/step/7?unit=3835}
    \item \url{https://blast.ncbi.nlm.nih.gov/Blast.cgi}, выбрать Nucleotide BLAST, ввести полученную последовательность нуклеотидов в желтое поле, нажать кнопку Blast. 
\end{enumerate}