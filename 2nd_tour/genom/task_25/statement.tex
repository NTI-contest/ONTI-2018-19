\assignementTitle{Дизайн праймеров}{6}{}

Полимеразная цепная реакция является исключительно важным современным методом молекулярной биологии. Принцип метода изложен в работах [1], [2]. Для амплификации участка ДНК методом ПЦР требуется заказать прямой и обратный праймер [3].

Ген SRY кодирует фактор транскрипции, который входит в семейство ДНК-связывающих белков HMG. 
Белок, кодируемый данным геном называют фактором развития семенников, данный белок определяет 
пол у мужчин. Мутации в данном гене приводят к формированию женских гениталий у лиц с генотипом 
XY (синдром Свайера) [5]. Транслокация данного участка Y-хромосомы на Х-хромосому приводит 
к мужскому фенотипу у лиц XX. Последовательность гена SRY человека в базе данных GeneBank 
имеет идентификатор NG\_011751.1 [6]

Последовательность праймеров принято записывать от 5'-конца к 3'-концу. Определите последовательность обратного праймера длиной 18 
нуклеотидов, использованного для амплификации фрагмента гена SRY человека, если в 
качестве прямого праймера был использован следующий олигонуклеотид\\ 5'- TGACATAAAAGGTCAATG-3', и известно, что длина ПЦР-фрагмента равна 218 пар нуклеотидов.

Введите последовательность прямого праймера латинскими буквами, без знаков 5'-, 3'-, и пробелов.

\subsubsection*{Рекомендуемая литература}

\begin{enumerate}
    \item Рекомендации по постановке ПЦР \url{http://evrogen.ru/kit-user-manuals/Evrogen-PCR-recommendation.pdf}
    \item 12 методов в картинках: полимеразная цепная реакция \url{https://biomolecula.ru/articles/metody-v-kartinkakh-polimeraznaia-tsepnaia-reaktsiia}
    \item Статья о праймерах в английской Википедии. Внимательно изучите иллюстрацию \url{https://en.wikipedia.org/wiki/Primer_(molecular_biology)}
    \item Статья о гене SRY в Википедии \url{https://ru.wikipedia.org/wiki/SRY}
    \item Синдром Свайера в Википедии \url{https://ru.wikipedia.org/wiki/%D0%A1%D0%B8%D0%BD%D0%B4%D1%80%D0%BE%D0%BC_%D0%A1%D0%B2%D0%B0%D0%B9%D0%B5%D1%80%D0%B0}
    \item Интерфейс для поиска \url{https://www.ncbi.nlm.nih.gov/nuccore}
\end{enumerate}