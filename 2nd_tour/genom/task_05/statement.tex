\assignementTitle{}{2}{}

Полимеразная цепная реакция является исключительно важным современным методом молекулярной биологии. Принцип метода изложен в работах [1], [2].
В честь дня рождения Томаса Ханта Моргана лаборант решил получить ПЦР-продукт гена white длиной 152 пары нуклеотидов. Ген white кодирует транспортер прекурсоров пигментов глаза дрозофилы, мутация в нем приводит к формирования белых глаз [3, 4]. Последовательность данного гена в базе данных Gene Bank имеет идентификатор X02974.2 [5]

Для амплификации участка ДНК методом ПЦР требуется заказать прямой и обратный праймер [6]. Последовательность праймеров принято записывать от 5'-конца к 3'-концу. 

Определите последовательность обратного праймера длиной 16 нуклеотидов, если в качестве прямого праймера был использован следующий олигонуклеотид 5'- CTCGCAACGGAAAACC-3'.

Введите последовательность обратного праймера латинскими буквами, без знаков 5'-, 3'-, и пробелов.

\subsubsection*{Рекомендуемая литература}

\begin{enumerate}
    \item Рекомендации по постановке ПЦР \url{http://evrogen.ru/kit-user-manuals/Evrogen-PCR-recommendation.pdf}
    \item 12 методов в картинках: полимеразная цепная реакция \url{https://biomolecula.ru/articles/metody-v-kartinkakh-polimeraznaia-tsepnaia-reaktsiia}
    \item Biochimica et Biophysica Acta (BBA) - Biomembranes. - 1999. - V. 1419. - P. 173-185\\ \url{https://www.sciencedirect.com/science/article/pii/S0005273699000644?via%3Dihub}
    \item Статья о мутации white в Википедии \url{https://ru.wikipedia.org/wiki/White_(%D0%BC%D1%83%D1%82%D0%B0%D1%86%D0%B8%D1%8F)}
    \item Интерфейс для поиска \url{https://www.ncbi.nlm.nih.gov/nuccore}
    \item Статья о праймерах в английской Википедии. Внимательно изучите иллюстрацию \url{https://en.wikipedia.org/wiki/Primer_(molecular_biology)}
\end{enumerate}

\explanationSection

Для решения задачи следует воспользоваться интерфейсом NCBI (ссылка 5). 

\answerMath{GGCTGTTGCTAATATT.}