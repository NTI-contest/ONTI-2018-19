\assignementTitle{}{10}{}

Из приведённых на рисунке плазмид выберете ту, которая удовлетворяет требованиям  вашего эксперимента:

\begin{enumerate}
    \item Вставка изучаемого гена в плазмиду будет проводиться с помощью рестрикции.
    \item Вы будете клонировать плазмиду в клетках E.coli. а затем проверять активность гена в эукариотических клетках, соответственно, плазмида должна содержать элементы, позволяющие отобрать клетки, получившие плазмиду, в обоих случаях.
    \item Репортёрный ген, входящий в состав плазмиды, позволяет сделать точный количественный анализ и провести отбор клеток по нему на сортере.
\end{enumerate}

\putImgWOCaption{14cm}{1}

\putImgWOCaption{13cm}{2}

\putImgWOCaption{14cm}{3}

\putImgWOCaption{14cm}{4}

\putImgWOCaption{14cm}{5}

\begin{enumerate}
    \item 1
    \item 2
    \item 3
    \item 4
    \item 5
\end{enumerate}

\explanationSection

Давайте разберёмся с условиями задачи поочерёдно. Начнём со второго условия: Вы будете клонировать плазмиду в клетках E.coli. а затем проверять активность гена в эукариотических клетках, соответственно, плазмида должна содержать элементы, позволяющие отобрать клетки, получившие плазмиду, в обоих случаях. После трансформации клеток плазмидой нужно отобрать только те клетки, которые получили эту плазмиду. Для отбора клеток в среду добавляется антибиотик, а сама плазмида должна содержать ген устойчивости к этому антибиотику. Для отбора бактериальных и эукариотических клеток используются разные антибиотика, соответственно, нужная нам плазмида должна содержать 2 разных гена устойчивости. 
Из предложенных вам плазмид плазмиды 1 и 4 содержат только один ген устойчивости AmpR~— устойчивость к ампицилину, который подходит только для отбора бактериальных клеток. Значит нам эти плазмиды не подходят.

Первое условие звучит так: вставка изучаемого гена в плазмиду будет проводиться с помощью рестрикции. Это значит, что плазмида должна содержать сайты рестрикции и эти сайты должны располагаться рядом с репортёрным геном. Теперь давайте снова посмотрим на плазмиды. Все они содержат сайты рестрикции. 

Прежде чем разберёмся с положением сайтов в плазмиде давайте проверим выполнение третьего условия: репортёрный ген, входящий в состав плазмиды, позволяет сделать точный количественный анализ и провести отбор клеток по нему на сортере. К репортёрным генам относятся гены флуоресцентных и люминесцентных белков. При этом точный количественный анализ с отбором на сортере позволяют сделать только флуоресцентные белки. Смотрим на оставшиеся плазмиды: 2,3,5. Плазмида 2 - содержит ген люциферазы, плазмида 3 - ген EGFP, плазмида 5 не содержит репортёрных генов. Люцифераза катализирует окисление люциферина, что приводит к испусканию света. Этот процесс называется люминесценцией и сделать точный количественный анализ с отбором люминесцирующих клеток на сортере не позволяет. EGFP - ген, кодирующий зелёный флуоресцентный белок. Значит, нам подходит одна плазмида под номером 3. Для нашего эксперимента, необходимо вставить изучаемый ген из генома азотфиксирующих бактерий перед репортёрным геном. Проверяем наличие сайтов рестрикции в плазмиде 3 сразу перед геном EGFP, они есть.

\answerMath{3.}