\assignementTitle{}{7}{}

Приведите последовательность одного из возможных обратных праймеров, удовлетворяющие условиям Здания 3.

\explanationSection

Обратный праймер должен быть комплементарен прямой цепи самого конца гена, вот этой части \dots \dots. GGTCGAACCGCTAGCGGGGACGGCGTGA 3’

Записывается праймер от 5’ к 3’. Поэтому составляем комплементарную последовательность заданному фрагменту и начинаем записывать с конца. По той же формуле, что и в предыдущем задании, вычисляем температуру плавления. Получаем вот эти праймеры:

TCACGCCGTCCCCGCTA (58, 16 нуклеотидов) $T=4 \cdot 12+2 \cdot 5=58$

TCACGCCGTCCCCGCTAС (62, 17 нуклеотидов) $T=4 \cdot 13+2 \cdot 5=62$

TCACGCCGTCCCCGCTAСC (66, 18 нуклеотидов) $T=4 \cdot 14+2 \cdot 5=66$

Короче праймеры не получатся, иначе нарушится условие про минимальную температуру в 58 градусов, а длиннее нельзя, т.к. температура превысит 68 градусов

Теперь давайте проверим праймеры из задания 16 и 17 на выполнение условия про максимальную разницу в температуре между ними 2 градуса. Видим, что у нас с вами получилось 3 возможные пары праймеров. В ответ вы могли записать по одному любому праймеру из тех, что мы перечислили.

\answerMath{TCACGCCGTCCCCGCTAC?C?.}