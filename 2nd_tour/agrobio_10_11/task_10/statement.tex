\assignementTitle{}{8}{}

Во втором задании вам предстоит проанализировать результаты эксперимента по определению микрофлоры почвы.

Для работы вам предлагается таблица, в которую внесены найденные организмы и их количество в исследуемом образце:

\begin{center}
Количество микроорганизмов (общее микробное число) контрольном образце.
\end{center}

\putImgWOCaption{14cm}{1}

Высевы осуществлялись на питательные среды:

\begin{itemize}
    \item МПА (мясопептонный агар);
    \item среда Чапека;
    \item крахмал-казеиновая среда.
\end{itemize}

Известно, что в одной из проб на 7 сутки присутствует положительная реакция на цинк-йод-крахмал в кислой среде. На наличие каких микроорганизмов (из представленных в таблице) это указывает?

В ответ занесите количество клеток этих микроорганизмов на 21 день эксперимента, если известно, что динамика изменения их количества останется той же.

\solutionSection

Раствор цинк-йод-крахмал используется для обнаружения азотистой кислоты. Следовательно, положительная реакция в кислой среде говорит о наличие микроорганизмов, которые образуют азотистую кислоту. К таким бактериям относятся нитрифицирующие бактерии первой стадии. 

Из приведённых в таблице микроорганизмов только $N$. Europaea относится к нитрифицирующие бактерииям и как раз окисляет нитраты до нитритов. 

Из таблицы видно, что каждые 7 дней количество микроорганизмов удваивается. 
Значит, на 21 дней мы получим $7.08 \cdot10 6  \cdot 2 = 14160000$ шт.

\answerMath{14160000.}