\assignementTitle{}{6}{}

Через некоторое время после рассеивания клеток, вы решили, сколько бактерий содержится в колониях, с которыми вы работаете. Для этого вы используете камеру Горяева. Среднее количество клеток микроорганизмов в малом квадрате камеры Горяева составляет 5 клеток. Определите количество микроорганизмов в 1 мл суспензии во втором десятикратном разведении.

Методические указания: 

Для расчетов используйте стандартные подходы, описанные в статье Государственной фармакопеи Российской Федерации - ОФС.1.7.2.0008.15 

В ходе решения промежуточные значения не округляйте. 

Общую информацию о камерах Горяева можно найти по следующим ссылкам:

\noindent\url{http://cldtest.ru/hdbk/chamber}

\noindent\url{https://opticalmarket.com.ua/kamera-gorjaeva-prakticheskoe-primenenie.html}

\noindent\url{https://ru.wikipedia.org/wiki/%D0%9A%D0%B0%D0%BC%D0%B5%D1%80%D0%B0_%D0%93%D0%BE%D1%80%D1%8F%D0%B5%D0%B2%D0%B0}

