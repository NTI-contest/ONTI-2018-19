\assignementTitle{}{6}{}

Через некоторое время после рассеивания клеток, вы решили, сколько бактерий содержится в колониях, с которыми вы работаете. Для этого вы используете камеру Горяева. Среднее количество клеток микроорганизмов в малом квадрате камеры Горяева составляет 5 клеток. Определите количество микроорганизмов в 1 мл суспензии во втором десятикратном разведении.

Методические указания: 

Для расчетов используйте стандартные подходы, описанные в статье Государственной фармакопеи Российской Федерации - ОФС.1.7.2.0008.15 

В ходе решения промежуточные значения не округляйте. 

Общую информацию о камерах Горяева можно найти по следующим ссылкам:

\noindent\url{http://cldtest.ru/hdbk/chamber}

\noindent\url{https://opticalmarket.com.ua/kamera-gorjaeva-prakticheskoe-primenenie.html}

\noindent\url{https://ru.wikipedia.org/wiki/%D0%9A%D0%B0%D0%BC%D0%B5%D1%80%D0%B0_%D0%93%D0%BE%D1%80%D1%8F%D0%B5%D0%B2%D0%B0}

\solutionSection

Давайте сначала разберём, что из себя представляет камера Горяева:

Это оптическое устройство для подсчета клеток в заданном объеме жидкости. Она состоит из толстого предметного стекла, имеющего прямоугольное углубление с нанесенной микроскопической сеткой и тонкого покровного стекла. 

Сетка камеры Горяева состоит из 225 больших квадратов, из которых 25 разделены на 16 малых квадратов.

\putImgWOCaption{8cm}{1}

\begin{center}
    Сетка камеры Горяева
\end{center}

\putImgWOCaption{8cm}{2}

\begin{center}
    Большой (1) и малый (2) квадраты сетки камеры Горяева
\end{center}

\putImgWOCaption{8cm}{3}

\begin{center}
    225 больших квадратов сетки камеры Горяева
\end{center}

\putImgWOCaption{8cm}{4}

\begin{center}
    Большой квадрат камеры Горяева разделенный на 16 малых квадратов
\end{center}

Технические характеристики камеры Горяева:

Размеры малого квадрата камеры Горяева $0.05 \times 0.05$ мм

Размеры большого квадрата камеры Горяева $0.2 \times 0.2$ мм

Глубина камеры 0.1 мм

Объем жидкости под 1 малым квадратом 0.00025 мм$^3$ (мкл) = 1/4000 мм$^3$ (мкл)

Объем жидкости под 1 большим квадратом 0.004 мм$^3$ (мкл) = 1/250 мм$^3$ (мкл)

Объем камеры Горяева 0.9 мм$^3$ (мкл)

Теперь давайте решим нашу задачу:

Из ОФС берем формулу:

$$x=(a/20) \cdot N \cdot k \cdot b$$

где $a$ – число клеток в 20 квадратах;

$N = 225$ –  число больших квадратов в камере Горяева;

$k$ – коэффициент, равный величине, обратной объему камеры Горяева \\($v=0.9 \: \text{мм}^3=0.9 \cdot 10^{-3}\: \text{мл}$);

$b$ – разведение исходной взвеси микроорганизма.

В условии указано «во втором десятикратном разведении», значит, $b = 10^{-2}$

Ищем стандартную малую клетку Горяева – в каждой большой клетке содержится 16 маленьких. Всего расчет ведется по 20 большим клеткам.

Тогда $$а=5 \cdot 16 \cdot 20=1600 \: \text{шт}$$
$$x= (1600/20) \cdot 225 \cdot (1/0.9 \cdot 10^{-3}) \cdot 0.01= 199984 \: \text{шт}$$

\answerMath{199984 (+-16).}
