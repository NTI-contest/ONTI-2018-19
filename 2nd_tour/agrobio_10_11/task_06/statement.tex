\assignementTitle{}{6}{}

Вам предстоит поработать с бактериями для проведения качественно- количественного учета микрофлоры 
почвы и дальнейшего подбора оптимальных условий  активности нитрифицирующих. В связи с этим,  
вам необходимо разработать протокол исследования и определить ход выполнения работы, начиная с 
подготовки образца.

Расставьте приведенные ниже лабораторные процедуры в правильном хронологическом порядке. 

При планировании эксперимента вам необходимо руководствоваться следующими правилами и рекомендациями:

\noindent\url{http://mibio.ru/docs/110/mr_fts4022_metodi_mikrobiologicheskogo_kontrolya_pochvi.pdf}

\noindent\url{http://window.edu.ru/resource/215/69215/files/bioprakt.pdf}

\noindent\url{http://biologo.ru/10909/10909.pdf}

\begin{enumerate}
    \item Посеять образцы на среды
    \item Маркировать чашки Петри
    \item Разлить среды по чашкам Петри
    \item Растереть до пастообразного состояния
    \item Сделать навеску почвы
    \item Оставить образцы для адсорбции при комнатной температуре
    \item Перенести в стерильную ступку
    \item Нанести маркером на дно чашки Петри сектора
    \item Подсчитать количество колоний
    \item Добавить стерильной воды
    \item Приготовить ряд 10-кратных разведений
    \item Дать суспензии отстояться
    \item Распределить инокулянт по питательной среде
    \item Поместить образцы в термостат
    \item Перевернуть чашки Петри
\end{enumerate}

\explanationSection

Сначала нужно получить собственно сам образец и полностью подготовить его для эксперимента. Затем нужно подготовить материалы для проведения эксперимента, в нашем случае чашки Петри, а затем уже провести само исследование.

Поэтому давайте расставим процедуры в правильном порядке:
\begin{itemize}
    \item Сделать навеску почвы
    \item Перенести в стерильную ступку
    \item Добавить стерильной воды
    \item Растереть до пастообразного состояния
    \item Дать суспензии отстояться
    \item Приготовить ряд 10-кратных разведений
    \item Маркировать чашки Петри
    \item Разлить среды по чашкам Петри
    \item Посеять образцы на среды
    \item Распределить инокулянт по питательной среде
    \item Оставить образцы для адсорбции при комнатной температуре
    \item Перевернуть чашки Петри
    \item Поместить образцы в термостат
    \item Нанести маркером на дно чашки Петри сектора
    \item Подсчитать количество колоний
\end{itemize}

\answerMath{5, 7, 10, 4, 12, 11, 2, 3, 1, 13, 6, 15, 14, 8, 9.}