\assignementTitle{}{6}{}

Проведя литературный поиск, вы поняли, что для оптимального рассеивания бактерий вам понадобится 
внести в среду раствор гидроортофосфата калия.

Определите, сколько грамм сухого гидроортофосфата калия нужно добавить к 127 г водного раствора соли $12\%$ 
концентрации для ее увеличения до $44\%$. При проведении расчетов полученные значения округлять до 
второго знака после запятой.

Ответ округлите, используя стандартные правила округления, и введите как целое число.

\solutionSection

Давайте распишем, что нам дано:

массовая доля 1 = 12\%,

$m_1$ = 127 г,

массовая доля 3 = 44\%.

К раствору добавляют сухой гидроортофосфат, это значит, что его массовую долю можно считать 100\%. Значит массовая доля 2 = 100\%. Необходимо найти $m_2$. 

Дальше задачу можно решать двумя способами: с применением правила креста, либо методом пропорций.

Сначала разберём вариант решения с применением правила креста (нарисовать на доске крест из массовых долей):

\putImgWOCaption{8cm}{1}

По этому правилу 
(массовая доля 2 - массовая доля 3) относится к (массовая доля 3 - массовая доля 1) как масса 1 к массе 2

Получаем $56/32 = 127/m_2$

Следовательно $m_2 = 127*32/56 = 72.57=73$ г

Теперь давайте решим эту же задачу методом пропорций:

$m_3 = m_1 + m_2 = 127 + x.$

Исходя из определения массовой доли вещества, процентная концентрация раствора показывает, сколько граммов растворенного вещества находится 
в 100 г раствора, то есть 

100 г 12\% раствора - это 12 г вещества

Значит 127 г 12\% раствора - это 127 $\cdot$ 12/100 = 15.24 г вещества

Для второго раствора составляем аналогичную пропорцию:

100 г 100\% раствора – 100 г вещества

x г 100\% раствора – x г вещества,

Следовательно, 127+x г нового раствора содержит 15.24+x г растворенного вещества.

Теперь, зная концентрацию нового раствора, можно определить x, то есть массу сухого добавленного вещества.

127+x г раствора – 15.24+x г вещества,

100 г раствора – 44 г вещества,

Получаем, что

(127+x)$\cdot$44 = (15.24+x)$\cdot$100

5588+44x=1524+100x

x=(5588-1524)/56 = 73 г

\answerMath{73.}