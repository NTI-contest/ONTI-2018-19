\assignementTitle{}{8}{}

В задании представлен текст об элементах круговорота азота в природе. Вставьте 
недостающие слова из предложенных.

Автотрофные, азот, азота, азотфиксация, амилаза, аммонификация, ассимиляция, водород, 
водорода, второй, выделяется, гетеротрофные, гниение, две, денитрификация, закись азота, 
кислород, кислорода, неорганические, нитрификация, окисление,  окись азота, органические, 
первой, поглощается, разложение, сахароза, $CO_2$, третьей, три

Атмосферный воздух на $78\%$ состоит из азота. Организмы и большинство зеленых растений 
нуждаются в различных химических формах азота. Благодаря процессам жизнедеятельности растений, 
водорослей и бактерий, осуществляется так называемый азотный цикл. Процессы, из которых 
складывается сложный круговорот азота - это ассимиляция, аммонификация, нитрификация, 
денитрификация, азотфиксация, разложение, выщелачивание, вынос, выпадение с осадками и т.д. 
Органические вещества, попадающие в почвы и воды подвергаются разложению, в ходе которого 
образуется аммиак. Под действием микроорганизмов проходит ряд дальнейших реакций и процессов. 
Процесс превращения аммиака в нитрат называется \makebox[2cm]{\hrulefill}$^1$. Он проходит в
\makebox[2cm]{\hrulefill}$^2$ стадии. Возбудителями \makebox[2cm]{\hrulefill}$^3$ стадии являются 
бактерии рода Nitrobacter. Они осуществляют превращение \makebox[2cm]{\hrulefill}$^4$ до
\makebox[2cm]{\hrulefill}$^5$. Возбудителями \makebox[2cm]{\hrulefill}$^6$ стадии являются бактерии 
рода Nitrosomonas, \linebreak Nitrosocystis, Nitrosolobus и Nitrosospira. Они окисляют \makebox[2cm]{\hrulefill}$^7$
до \makebox[2cm]{\hrulefill}$^8$. В ходе всех преобразований активно \makebox[2cm]{\hrulefill}$^9$ 
энергия. Процесс превращения нитратов в газообразные оксиды и молекулярный азот называется \makebox[2cm]{\hrulefill}$^{10}$
. Этот процесс происходят в среде, лишенной \makebox[2cm]{\hrulefill}$^{11}$. Т.е. процесс является 
анаэробным. В процессе преобразования исходного вещества (нитрат) в конечное 
(газообразный азот) последовательно появляются три промежуточных продукта: \makebox[2cm]{\hrulefill}$^{12}$ > 
\makebox[2cm]{\hrulefill}$^{13}$ > \makebox[2cm]{\hrulefill}$^{14}$. Нитрификация производится \makebox[2cm]{\hrulefill}$^{15}$ 
бактериями. Это означает, что они получают углерод, необходимый для роста, из \makebox[2cm]{\hrulefill}$^{16}$ веществ. 
Денитрифицирующие бактерии являются \makebox[2cm]{\hrulefill}$^{17}$, т.е. получают углерод из \makebox[2cm]{\hrulefill}$^{18}$ 
веществ, таких как \makebox[2cm]{\hrulefill}$^{19}$.

\explanationSection

Первое пропущенное слово - нитрификация. 
Из учебной литературы известно. что нитрификация проходит в две стадии:
Первая стадия описывается следующими реакциями:
\begin{enumerate}
    \item $NH_3 + O_2 + \text{НАДН}_2 \rightarrow NH_2OH + H_2O + \text{НАД}+$
    \item $NH_2OH + H_2O \rightarrow HNO_2 + 4H+ + 4e-$
    \item $1/2O_2 + 2H+ + 2e- \rightarrow H_2O$
\end{enumerate}

В результате в ходе первой стадии аммиак окисляется до нитрита. Осуществляют данный процесс нитрозные бактерии.

Вторая стадия:
$$NO_2- + H_2O \rightarrow NO_3- + 2H+ + 2e-$$

В результате которой нитрит окисляется до нитрата. Превращение второй стадии осуществляется нитратными бактериями.

На основании этого можно вставить пропущенные слова в следующий абзац:

“Он проходит в две стадии. Возбудителями второй стадии являются бактерии рода Nitrobacter. Они осуществляют превращение нитрита до нитрата. Возбудителями первой стадии являются бактерии рода Nitrosomonas, Nitrosocystis, Nitrosolobus и Nitrosospira. Они окисляют аммиак до нитрита. 
В ходе всех преобразований активно выделяется энергия.”

Теперь переходим к следующему. описанному в тексте процессу: 
“Процесс превращения нитратов в газообразные оксиды и молекулярный азот называется денитрификация.” 

Из учебной литературы известно, что денитрификация протекает в несколько стадий:
$$NO_3-\rightarrow NO_2- \rightarrow NO \rightarrow N_2O \rightarrow N_2$$
Также известно, что восстановление нитратов является анаэробным дыханием, 
“Этот процесс происходят в среде, лишенной кислорода, то есть анаэробной.”

Опираясь на эту информацию, можем заполнить недостающие слова в следующем абзаце:
“В процессе преобразования исходного вещества (нитрат) в конечное (газообразный азот) последовательно появляются три промежуточных продукта: нитрит, окись азота ($NO$) и закись азота ($N_2O$).”

Изучив более подробно информацию об о бактериях, которые участвуют в нитрификации и денитрификации, дополняем последнюю часть текста:

“Нитрификация производится автотрофными бактериями. Это означает, что они получают углерод, необходимый для роста, из неорганических веществ, таких как $СО_2$.

Денитрифицирующие бактерии являются гетеротрофными, то есть получают углерод из органических веществ, таких как сахароза.”

\answerMath{1.~нитрификация~/ нитрификацией; 2.~две; 3.~второй; 4.~нитрита; 5.~нитрата; 6.~первой; 7.~аммиак; 8.~нитрита; 9.~выделяется; 10.~денитрификация~/ денитрификацией; 11.~кислорода~/ кислород; 12.~нитрит; 13.~окись азота; 14.~закись азота; 15.~автотрофными; 16.~неорганических; 17.~гетеротрофными; 18.~органических; 19.~Сахароза.}