\assignementTitle{}{7}{}

Изучите методы определение активности денитрификации и азотфиксации в почве и выполните следующие задания.

Перечислите все газы, которые присутствуют в конечной реакционной смеси при проведении экспермента по определению активности денитрификации с использованием ацетилена. Ответ запишите формулами.

Формулы расположите в алфавитном порядке и в порядке увеличения индексов. Приоритет дается цифрам перед буквами (пример: H2, H2O, HCl, HOPO3, NaCl...).

\explanationSection

Суммарное уравнение денитрификации выглядит следующим образом:
$$C_6H_{12}O_6 + 4NO_3- = 6CO_2 + 6H_2O + 2N_2 + Q$$
Из него мы получаем газы $N_2$ и $CO_2$.

Теперь давайте разберём ацетиленовый метод. Принцип метода заключается в том, что при добавлении в систему ацетилена позволяет фиксировать количество образовавшейся закиси азота ($N_2O$). Накопление $N_2O$ прекращается. когда заканчиваются нитраты и нитриты. в то же время появляется этилен. 

Уравнение восстановления ацетилена:
$$C_2H_2+H_2=C_2H_4, $$

В системе может протекать также реакция:
$$CO_2+4H_2=CH_4+2H_2O.$$

Получаем ещё газы $C_2H_2$, $CH_4$ и $C_2H_4$

Поэтапная денитрификация выглядит так:
$$NO_3- \rightarrow NO_2-  \rightarrow NO  \rightarrow N_2O  \rightarrow N_2$$

Закись азота не полностью превращается в азот.

\answerMath{C2H2 C2H4 CH4 CO2 H2 N2 N2O.}