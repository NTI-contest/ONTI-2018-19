\assignementTitle{}{7}{}

Комплекс фермента из Задания 1 кодируется целым кластером генов.

Вы будете изучать один ген, входящий в состав этого кластера.

Для выделения изучаемого гена из генома Rhizobium mesoamericanum будет использован метод ПЦР (полимеразной цепной реакции).

Для работы вам предлагается нуклеотидная последовательность начала и конца гена:

5’ATGTCCACACCGATGATTTCGCTTGAGAGCCTGGCCAGCAGGACATCCT\linebreak
TGGATCAATTGCTGGCGACCTCGAAATCCGGTGGTTGCACATCCTCATCCTG\linebreak
CGGCGCCTCCACAAATCCGGACGACTTCGACCAGGCCAT

………………...…………………………………………………

GCGCCAAGATCGGGGAGTGCCCTAGGAATCAGCTCATGGAGGCTGGAGTC\linebreak
CGAGCAACGGACGCTTATGG

CTATGACTACATCGAGACCGCCATCGGCGCCCTTTACGCCGCCGAGTTTGG\linebreak
GGTCGAACCGCTAGCGGGGACGGCGTGA 3’

Изучив правила дизайна и написания праймеров для ПЦР, приведите одну из воможных последовательностей прямого праймера, удовлетворяющие следующим условиям:

\begin{enumerate}
    \item Ваша пара праймеров позволит реплицировать всю последовательность гена.
    \item Максимальная разница температур между прямым и обратным праймерами должна составлять не более $2^{\circ}C$
    \item Длина праймеров должна быть между 15 и 22  нуклеотидами
    \item Температура плавления праймеров должна лежать в интервале $58-68^{\circ}C$
    \item Для расчета температур плавления вам не надо пользоваться какими-либо программами или сервисами! Примените простейшую формулу
\end{enumerate}
