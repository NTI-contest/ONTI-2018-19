\assignementTitle{}{7}{}

Комплекс фермента из Задания 14 кодируется целым кластером генов.

Вы будете изучать один ген, входящий в состав этого кластера.

Для выделения изучаемого гена из генома Rhizobium mesoamericanum будет использован метод ПЦР (полимеразной цепной реакции).

Для работы вам предлагается нуклеотидная последовательность начала и конца гена:

5’ATGTCCACACCGATGATTTCGCTTGAGAGCCTGGCCAGCAGGACATCCT\linebreak
TGGATCAATTGCTGGCGACCTCGAAATCCGGTGGTTGCACATCCTCATCCTG\linebreak
CGGCGCCTCCACAAATCCGGACGACTTCGACCAGGCCAT

………………...…………………………………………………

GCGCCAAGATCGGGGAGTGCCCTAGGAATCAGCTCATGGAGGCTGGAGTC\linebreak
CGAGCAACGGACGCTTATGG

CTATGACTACATCGAGACCGCCATCGGCGCCCTTTACGCCGCCGAGTTTGG\linebreak
GGTCGAACCGCTAGCGGGGACGGCGTGA 3’

Изучив правила дизайна и написания праймеров для ПЦР, приведите одну из воможных последовательностей прямого праймера, удовлетворяющие следующим условиям:

\begin{enumerate}
    \item Ваша пара праймеров позволит реплицировать всю последовательность гена.
    \item Максимальная разница температур между прямым и обратным праймерами должна составлять не более $2^{\circ}C$
    \item Длина праймеров должна быть между 15 и 22  нуклеотидами
    \item Температура плавления праймеров должна лежать в интервале $58-68^{\circ}C$
    \item Для расчета температур плавления вам не надо пользоваться какими-либо программами или сервисами! Примените простейшую формулу
\end{enumerate}

\solutionSection

Для начала давайте разберём, что называется прямым праймером. Прямой праймер - это праймер к началу гена. Начало определяем по 5’ концу. Он комплементарен обратной цепи, соответственно, последовательность прямого праймера будет идентична началу прямой цепи гена. Вернёмся к последовательности и к условиям: нам нужно амплифицировать весь ген целиком, значит начало праймера должно совпадать с началом гена, то есть начинаться с 5’ATGT \dots. 

Простейшая формула для расчёта температуры плавления выглядит так:
$$T_\text{плавления} = 4 \cdot n(GC\text{пар})+2 \cdot n(AT\text{пар})$$ 
При этом у нас есть условие, что длина праймера должна быть в диапазоне 15-22 нуклеотида, а температура плавления лежит в интервале $58-68^\circ C$.
Давайте запишем сначала все возможные прямые праймеры, удовлетворяющие данным условиям. Помните. что все праймеры записываются от 5’ конца к 3’ концу. 

ATGTCCACACCGATGATTTC ($58^\circ C$, 20нуклеотидов) $T=4 \cdot 9+2 \cdot 11=58$

ATGTCCACACCGATGATTTCG ($62^\circ C$, 21 нуклеотид) $T=4 \cdot 10+2 \cdot 11=62$

ATGTCCACACCGATGATTTCGC ($66^\circ C$, 22 нуклеотида) $T=4 \cdot 11+2 \cdot 11=66$

Это все возможные праймеры, короче быть не могу, т.к. нарушим условие про температуру не ниже $58^\circ C$, а длиннее нарушится условие про 22 нуклеотида.

Остаётся проверить выполнение условия про разницу температуры между прямым и обратным праймером. Поэтому давайте перейдём пока к следующему заданию, а потом вернёмся к ответу на это.

\answerMath{ATGTCCACACCGATGATTTCG?C?.}