\assignementTitle{}{6}{}

Вам предложен следующий список питательных сред, которые в дальнейшем могут быть использованы для определения микробиологического состава биофильтра пресноводной системы:

\begin{enumerate}
    \item МПА (мясопептонный агар)
    \item Среда Эшби
    \item Раствор Люголя
    \item Бифидум-среда
    \item Среда Сабуро
    \item Картофеле-глюкозный бульон
    \item Среда Чапека-Докса
\end{enumerate}

Определите органический состав предложенных сред.  

\begin{enumerate}
    \item[a.] Сахароза
    \item[б.] Агар
    \item[в.] Глюкоза
    \item[г.] Бенгальский розовый
    \item[д.] Настой паслена клубненосного
    \item[е.] Пептонный бульон
    \item[ж.] Гидролизат казеина
    \item[з.] Томатный сок
    \item[и.] Дрожжевой экстракт
\end{enumerate}

\begin{table}[!h]
    \small
    \begin{tabular}{|c|c|c|c|c|c|}
        \hline 
                            & Сахароза & Агар & Глюкоза & Бенгальский & Настой паслена \\
                            &          &      &         & розовый     & клубненосного \\
        \hline
        МПА                 &          & +    &         &             &              \\
        (мясопептонный 
        агар)               &          &      &         &             &  \\
        \hline
        Среда 
        Эшби                & +        &      &         &             &  \\
        \hline
        Раствор 
        Люголя              &          &      &         &             & \\
        \hline
        Бифидум-среда       &          &      & +       &             & \\
        \hline
        Среда Сабуро        &          & +/-  & +       &             &\\
        \hline
        Картофеле-глюкозный &          & +    & +       & +           & +\\
        бульон & & & & & \\
        \hline
        Среда Чапека-Докса & +         &      &         &             &  \\
        \hline
    \end{tabular}
\end{table}

\begin{table}[!ht]
    \small
    \begin{tabular}{|c|c|c|c|c|}
        \hline 
                            & Пептонный & Гидролизат & Томатный & Дрожжевой \\
                            & бульон    & казеина    & сок      & экстракт \\
        \hline
        МПА                 &  +         &            & & \\
        (мясопептонный 
        агар) &  & & & \\
        \hline
        Среда Эшби          &           &            &           &  \\
        \hline
        Раствор Люголя      &           &            &          & \\
        \hline
        Бифидум-среда       & +/-          & +/-          & +         &+ \\
        \hline
        Среда Сабуро        &  +        & +          &          & \\
        \hline
        Картофеле-глюкозный &           &            &          & \\
        бульон & & & & \\
        \hline
        Среда Чапека-Докса  &           &             &           &  \\
        \hline
    \end{tabular}
\end{table}

\newpage

\explanationSection

Задание аналогично первому, только нужно определить органический состав тех же самых сред. Полный состав для всех сред мы уже с вами выписали, поэтому давайте просто выберем, какие компоненты являются органическими.

Для МПА (мясопептонного агара) отмечаем агар, пептонный бульон и мясной перевар.

Для среды Эшби - сахарозу.

Раствор Люголя встречается в двух видах - в виде водного раствора и в виде раствора в глицерине. В таблице есть глицерин, поэтому отмечаем его.

В Бифидум-среде из органических компонентов, приведённых в таблице, присутствуют глюкоза, дрожжевой экстракт, гидролизат казеина. 

В среде Сабуро есть пептонный бульон, агар, глюкоза.

В Картофеле-глюкозном бульоне - агар, глюкоза, бенгальский розовый, настой паслена клубненосного (аналог картофельного настоя).

И в последней среде Чапека-Докса из органических компонентов присутствует только сахароза.
