\assignementTitle{}{6}{}

Вам предложен следующий список питательных сред:

\begin{enumerate}
    \item МПА (мясопептонный агар)
    \item Среда Эшби
    \item Раствор Люголя
    \item Бифидум-среда
    \item Среда Сабуро
    \item Картофеле-глюкозный бульон
    \item Среда Чапека-Докса
\end{enumerate}

Определите, какое действие оказывают среды на  Rhodotorula glutinis?

\begin{enumerate}
    \item[a.] Стимулирующее
    \item[б.] Угнетающее
    \item[в.] Нейтральное
\end{enumerate}

\explanationSection

Организм Rhodotorula glutinis представяет собой розовые дрожжи и принадлежит к царству Грибов. 

Из литературы известно, что раствор Люголя оказывает бактерицидное действие и действует также на патогенные грибы и дрожжи. Следовательно, раствор Люголя оказывает угнетающее действие на данный организм. 

Среда для культивирования Rhodotorula glutinis должна содержать источники азота, углерода, минеральные соли. Все перечисленные среду, за исключением Бифидум -среды оказывают стимулирующее действие. Бифидум-среда - нейтральное.

\answerMath{a - 1, 2, 5, 6, 7; б - 3; в - 4.}