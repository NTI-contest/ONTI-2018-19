\assignementTitle{}{8}{}

При измерении активности азотфиксации аналогичным методом (с использованием ацетилена) с помощью калибровочной кривой рассчитали, что конечные количества ацетилена и этилена равны 0.15 мМ и 0.95 мМ, соответственно. Определите коэффициент активности азотфиксации в данной пробе (ответ округлите до десятых).

\solutionSection

Ацетиленовый метод определения активности азотфиксации основан на способности нитрогеназы восстанавливать не только азот до аммиака, но и ацетилен до этилена. Давайте распишем все реакции процесса. 

Уравнение азотфиксации: $N_2 + 8H+ + 8e - \rightarrow 2NH_2 $

Восстановление ацетилена: $C_2H_2 + 2H+ + 2e - \rightarrow C_2H_4$

Мы знаем, что конечные концентрации ацетилена и этилена равны 0.15 mM и 0.95 mM соответственно. Превращение ацетилена в этилен идёт 1:1. 

Коэффициент активности нитрогеназы в процессе превращения ацетилена в этилен считается по формуле: 

Кол-во образовавшегося этилена/кол-во оставшегося ацетилена = $0.95/0.15=6.3$

Коэффициент перевода этилена в аммиак равен 3. Поэтому величину, полученную для ацетилена мы делим на 3. 
$6.3/3=2.1$ 

\answerMath{2.1.}