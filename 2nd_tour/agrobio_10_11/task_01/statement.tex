\assignementTitle{}{8}{}

Вам предложен следующий список питательных сред, которые в дальнейшем могут быть использованы для определения микробиологического состава биофильтра пресноводной системы:

\begin{enumerate}
    \item МПА (мясопептонный агар)
    \item Среда Эшби
    \item Раствор Люголя
    \item Бифидум-среда
    \item Среда Сабуро
    \item Картофеле-глюкозный бульон
    \item Среда Чапека-Докса
\end{enumerate}

Определите неорганический состав предложенных сред.

\begin{enumerate}
    \item[a.] Вода
    \item[б.] $NaNO_3$
    \item[в.] $MgSO_4$
    \item[г.] $Fe_2(SO_4)_3$
    \item[д.] $KCl$
    \item[е.] $K_2HPO_4 / KH_2PO_4$
    \item[ж.] $KMnO_4$
    \item[з.] $Kl$
    \item[и.] $NaCl$
    \item[к.] $K_2SO_3$
    \item[л.] $CaCO_3$
\end{enumerate}

\begin{table}[!ht]
\small
\begin{tabular}{|c|c|c|c|c|c|c|c|c|c|c|c|c|}
    \hline 
    & Вода & Нитрат & Сульфат & Сульфат & Хлорид & Калий фосфорнокислый & Пермагнат & Калия & Хлорид & Калий фосфорнокислотный & Сульфит & Карбонат \\
    & натрия & магния & железа 3 & калия& & двузамещенный & калия & йодид & натрия & однозамещенный & калия & кальция \\
    МПА & & & & & & & & & & & & \\
    \hline
    Среда Эшби & & & & & & & & & & & & \\
    \hline
    Раствор Люголя & & & & & & & & & & & & \\
    \hline
    Бифидум-среда & & & & & & & & & & & & \\
    \hline
    Среда Сабуро & & & & & & & & & & & & \\
    \hline
    Картофеле-глюкозный& & & & & & & & & & & & \\
    бульон& & & & & & & & & & & & \\
    \hline
    Среда Чапека-Докса& & & & & & & & & & & & \\
    \hline
    & & & & & & & & & & & & \\

\end{tabular}
\end{table}