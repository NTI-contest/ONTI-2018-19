\assignementTitle{}{8}{}

Вам предложен следующий список питательных сред, которые в дальнейшем могут быть использованы для определения микробиологического состава биофильтра пресноводной системы:

\begin{enumerate}
    \item МПА (мясопептонный агар)
    \item Среда Эшби
    \item Раствор Люголя
    \item Бифидум-среда
    \item Среда Сабуро
    \item Картофеле-глюкозный бульон
    \item Среда Чапека-Докса
\end{enumerate}

Определите неорганический состав предложенных сред.

\begin{enumerate}
    \item[a.] Вода
    \item[б.] $NaNO_3$
    \item[в.] $MgSO_4$
    \item[г.] $Fe_2(SO_4)_3$
    \item[д.] $KCl$
    \item[е.] $K_2HPO_4 / KH_2PO_4$
    \item[ж.] $KMnO_4$
    \item[з.] $KI$
    \item[и.] $NaCl$
    \item[к.] $K_2SO_3$
    \item[л.] $CaCO_3$
\end{enumerate}

\newpage

\begin{table}[h]
\small
\begin{tabular}{|c|c|c|c|c|c|c|}
    \hline 
    & Вода & $NaNO_3$ & $MgSO_4$ & $Fe_2(SO_4)_3$ & $KCl$ & $K_2HPO_4 /$ \\
    & & & & & & $KH_2PO_4$ \\
    \hline
    МПА & + & & & & & \\
    (мясопептонный агар) &  & & & & & \\
    \hline
    Среда Эшби & + & & + & & & + \\
    \hline
    Раствор Люголя & + & & & & &\\
    \hline
    Бифидум-среда & + & & & & &\\
    \hline
    Среда Сабуро & + & & & & &\\
    \hline
    Картофеле-глюкозный & + & & & & &\\
    бульон & & & & & &\\
    \hline
    Среда Чапека-Докса & + & + & + & + & + & +\\
    \hline
\end{tabular}
\end{table}

\begin{table}[h]
    \small
    \begin{tabular}{|c|c|c|c|c|c|}
        \hline 
        & $KMnO_4$ & $KI$ & $NaCl$ & $K_2SO_3$ & $CaCO_3$ \\
        & & & & & \\
        \hline
        МПА & & & + & & \\
        (мясопептонный агар) & & & & & \\
        \hline
        Среда Эшби & & & + & + & + \\
        \hline
        Раствор Люголя & & + & & & \\
        \hline
        Бифидум-среда & & & & & \\
        \hline
        Среда Сабуро & & & & & \\
        \hline
        Картофеле-глюкозный & & & & & \\
        бульон & & & & & \\
        \hline
        Среда Чапека-Докса & & & & & \\
        \hline
    \end{tabular}
    \end{table}

\explanationSection

МПА или мясопетонный агар. Ищем состав среды в интернете. Например, Вам была дана ссылка на сайт himedialabs, 
на этом сайте приведён состав данной среды: 
\begin{itemize} 
    \item пептический перевар животной ткани
    \item настой говядины
    \item натрия хлорид
    \item агар-агар
\end{itemize}

Дальше смотрим на неорганические вещества в таблице и отмечаем нужные. Получается, что в состав МПА из неорганических соединений входит 
$NaCl$ и, конечно же, вода. 

Следующая среда Эшби. 
Снова ищем состав в литературе (например, \url{http://www.rcm.kz/ru/sw}):
\begin{itemize}
    \item вода дистиллированная
    \item сахароза или маннит
    \item калий фосфорнокислый однозамещенный ($K_2HPO_4$) 
    \item сульфат магния ($MgSO_4 \cdot 7H_2O$) 
    \item хлорид натрия ($NaCl$) 
    \item сульфит калия ($K_2SO_3$) 
    \item карбонат кальция ($CaCO_3$) 
\end{itemize}

Выбираем из этого списка неорганические соединения и отмечаем в таблице: вода, $K_2HPO_4$, $MgSO_4$, $NaCl$, $K_2SO_3$, $CaCO_3$.

Дальше дан раствор Люголя. Он представляет собой раствор йода в водном растворе иодида калия. Следовательно, отмечаем в таблице воду и $KI$.

Следующая среда - бифидум-среда. находим состав:
\begin{itemize}
    \item Панкреатический гидролизат казеина
    \item Дрожжевой экстракт 
    \item a-Д-лактоза 
    \item Д-глюкоза
    \item Цистеина гидрохлорид 
    \item Натрий хлористый  
    \item Магний сернокислый 
    \item Кислота аскорбиновая
    \item Натрий уксуснокислый
    \item Агар микробиологический
\end{itemize}

Из приведённых в таблице неорганических веществ отмечаем $NaCl$, $MgSO_4$ и воду

Дальше идёт среда Сабуро, снова ищем состав:
\begin{itemize}
    \item вода водопроводная 
    \item глюкоза
    \item пептон
    \item агар 
\end{itemize}

Видим, что из неорганических веществ в состав входит только вода, отмечаем в таблице.

Следующий - картофеле-глюкозный бульон. В его состав с сайта himedialabs входят:
\begin{itemize}
    \item картофельный настой
    \item глюкоза
    \item бенгальский розовый
    \item агар-агар
\end{itemize}

То есть из неорганических веществ отмечаем только воду.

И последняя среда -Среда Чапека-Докса. Её состав также есть на сайте himedialabs:
\begin{itemize}
    \item сахароза
    \item натрия нитрат
    \item калия гидрофосфат
    \item магния сульфат
    \item калия хлорид
    \item железа сульфат
    \item агар-агар
\end{itemize}

Теперь отмечаем в таблице только неорганические вещества: снова в первую очередь вода, $NaNO_3$, $K_2HPO_4$, $MgSO_4$, $KCl$, $Fe_2(SO_4)_3$.