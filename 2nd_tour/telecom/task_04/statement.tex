\assignementTitleT{Передача информации по нескольким каналам связи (22 балла)}

\textbf{Тренируемые навыки: }теория вероятности, условная вероятность, работа с массивами.

\subsubsection*{Условие задачи}

Необходимо принять с некоторого удаленного объекта сообщение длиной M единиц. Для этого важно понять хватит ли для этого наших возможностей. В нашем распоряжении 2 канала связи в каждом присутствует свой стационарный шум. Для анализа свойств шума в каждом канале была записана реализация шумового сигнала в $N = 1000$ отсчетов. Также известно, что вероятность передачи данных (полезного сигнала) с удаленного объекта составляет $50\%$, т.е. $50\%$ времени объект передает данные и $50\%$ времени «молчит». Амплитуда полезного сигнала не постоянна, но для каждого канала известно распределение вероятности его амплитуды. В качестве иллюстрации на рисунке приведены: верхний левый -- реализация шумового сигнала, нижний левый -- распределение вероятности амплитуды полезного сигнала, нижний правый -– пример реализации сигнала, соответствующий распределению на левом нижнем рисунке, правый верхний –- пример принятого сигнала $A_R$, представляющего сумму шума $A_N$ и полезного сигнала $A_S$:

$$A_R = A_N + A_S.$$

\putImgWOCaption{16.5cm}{1}

Верхний левый рисунок –- пример шумового сигнала, нижний левый рисунок~–- пример распределения вероятности амплитуды полезного сигнала, пример реализации сигнала, соответствующий распределению на левом нижнем рисунке, правый верхний –- пример принятого сигнала, представляющего сумму шума и полезного сигнала.

Из рисунка видно, что сигналы в каналах очень слабые, таким образом, рассматривается задача обнаружения сигнала (есть сигнал/нет сигнала). Решение о том, что в принятом сигнале присутствует полезный сигнал, принимается на основе сравнения амплитуды принятого сигнала с пороговым уровнем, если амплитуда больше или равна порогу, считаем, что полезный сигнал присутствует и отсутствует в противном случае. Пороговый уровень должен быть таким, чтобы сумма вероятности пропуска сигнала и вероятности ложных тревог (вероятность превышения шумом порога) была минимальной.

Если использование только одного из представленных каналов не позволяет решить задачу, тогда нужно задействовать несколько каналов и принимать решение о наличии сигнала голосованием – при условии, что, хотя бы в двух каналах сигнал выше порогового уровня, то принимаем решение — полезный сигнал присутствует. Для каждого канала известна потребляемая мощность. При использовании нескольких каналов суммируется и их энергопотребление.

Таким образом, необходимо проанализировать представленные шумовые сигналы и гистограммы распределения полезного сигнала, определить пороговый уровень и принять решение о том, какие каналы нужно задействовать, для приема сообщения длинной М=50 с наименьшими энергетическими затратами, так чтобы с вероятностью 99\% в сообщении было не более 2 ошибок.

Языки программирования — Python, C, C++, Java

Примеры работы программы, решающей предлагаемую задачу присоединены.

\inputfmtSection

2 –- кол-во каналов;

50 –- длина передаваемого сообщения;

2 –- допустимое количество ошибок;

0.6 -- вероятность передачи данных (полезного сигнала);

4.0296 5.0000 –- 2 числа -- затраты энергии на передачу сигнала в каждом канале;

15 –- количество уровней сигнала;

0 1 2 3 4 5 6 7 8 9 10 11 12 13 14 –- значения уровней сигнала;

Далее 2 строки -- Распределения уровня сигнала в каждом каналах (вероятность встретить сигнал с данным уровнем в данном канале, строка 1 –- распределение вероятности в 1 канале, 2-я строка –- распределение во втором канале)

1000 -- Длина шумовой реализации

Далее 2 строки -- Реализация шумового сигнала для каждого канала (строка 1 -- реализация, полученная в канале 1, строка 2 –- реализация, полученная в канале 2).


\outputfmtSection

1 –- необходимое кол-во каналов;

1 –- номера используемых каналов;

4 –- пороговый уровень для каждого из выбранных каналов;

0.996471 –- вероятность верного обнаружения единичного сигнала, при использовании всех выбранных каналов;

0.999239 -- Вероятность того что сообщение длины 50 будет получено с не более чем 2 ошибками;

5.000 -- Необходимая энергия для передачи единицы сообщения (суммарная энергия всех задействованных каналов)

\markSection

Максимальное число баллов — за полностью верный ответ, включающий:

\begin{itemize}
    \item Правильно определены номера каналов, использование которых позволит решить задачу с наименьшими энергетическими затратами; (2 балла)
    \item Правильно определены пороговые уровни выбранных каналов, перечисленных в пункте 1; (6 баллов)
    \item Правильно определена вероятность верного обнаружения единичного сигнала, при использовании всех выбранных каналов, перечисленных в пункте 1 (данную вероятность необходимо определить с точность до 4-го знака); (10 баллов)
    \item Правильно определена вероятность того, что с использованием выбранных каналов (пункт 1), сообщение длины M будет передано с не более чем 2 ошибками (данную вероятность необходимо определить с точность до 4-го знака); (3 балла)
    \item Правильно рассчитана энергия, необходимая для передачи единицы сообщения (суммарная энергия всех задействованных каналов). (1 балл)
\end{itemize}

Штрафы:
\begin{itemize}
    \item В пункте (3):
    \begin{itemize}
        \item за ошибку в четвертом знаке снимается 3 балла.
        \item за ошибку в третьем знаке снимается 10 баллов.
    \end{itemize}
    \item В пункте (4):
    \begin{itemize}
        \item за ошибку в четвертом знаке снимается 1 балл.
        \item за ошибку в третьем знаке снимается 3 балла.
    \end{itemize}
\end{itemize}

Скорость выполнения программы — не более 10 секунд

Программа должна читать исходный файл со стандартного потока $stdin$ и передавать решение на стандартный поток $stdout$.

Проверка проводилась в несколько этапов(попыток), проводимых в заданную дату (20.12.2018; 29.12.2018; 06.01.2019; 11.01.2019; 14.01.2019). В каждой попытке участники могли загрузить и проверить одно свое решение этой задачи.

Результирующая оценка за задачу выбиралась, как максимальная из попыток.

Загрузка решений проводилась на сайте \url{http://dep1.iszf.irk.ru/wlcomm} на котором было установлено специальное программное обеспечение для регистрации, разделения доступа и возможности загрузки решений участниками.

Тестирование решений проводилось на виртуальной машине с установленной ОС Kubuntu, проверку и начисление баллов за каждый тест осуществляли автоматические программы.

\subsection*{Алгоритм решения задачи}

\begin{enumerate}
    \item Для каждого канала строим гистограмму распределения шума. Интеграл, полученной гистограммы нормируем на единицу. Полученное распределение обозначим $P_{Ni} (u)$, $I$ – номер канала $i \ni [1, 2]$.
    \item Для каждого канала на основе распределений сигнала и шума строим распределение “сигнал+шум” $P_{SNi} (u)=\int_0^uP_{Si}(v) P_{Ni}(u-v)dv$. В дальнейшем пороговый уровень будет определяться на основе $P_{Ni} (u)$ и $P_{SNi} (u).$
    \item Используем т. Байесса для того чтобы для каждого канала найти пороговый уровень $u_{bi}$, при котором минимизируется вероятность ложных тревог и вероятность пропуска.
    $$q_i+n_i \rightarrow min,$$
    
    $q_i=\int_0^{u_{bi}}P_{SNi}(u)du$ – вероятность пропуска сигнала,

    $p_i=1- q_i$ – вероятность верного определения сигнала в канале при данном пороговом уровне $u_{bi}$.
    
    $n_i=\int_{u_bi}^\infty P_{Ni}(u)du$ – вероятность ложной тревоги.

	\item Находим вероятность правильного принятия решения о наличии сигнала, на основе, найденного порогового уровня.
    
    Вероятность принятия правильного решения о наличии сигнала в канале:  $P_i=\frac{p \cdot p_i}{q \cdot n_i+p \cdot p_i }$, где $p$ – вероятность присутствия сигнала, $q =1-p$.
    \item Для каждого канала рассчитываем вероятность принятия сообщения из M отсчетов с не более чем двумя ошибками:
    $$P_{Mi}=P_i^M+C_M^1 P_i^{M-1} (1-P_i )+C_M^2 P_i^{M-2} (1-P_i)^2$$
    \item Если для одного канала вероятность принятия сообщения из M отсчетов с не более чем двумя ошибками удовлетворяет требованиям задачи, тогда решение задачи – использование данного канала. В случае, если оба канала удовлетворяют требованиям задачи, выбирается канал с наименьшими затратами энергии.
    \item Если ни один из каналов не удовлетворяет требованиям задачи, тогда возвращаем сообщение о том, что невозможно принять сообщение длинной М с не более чем двумя ошибками.
\end{enumerate}