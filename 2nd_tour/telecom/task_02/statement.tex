\assignementTitle{Чет-нечет}{6}

Начинающий связист написал программу, реализующую алгоритм помехоустойчивого кодирования на основе кода Хемминга, например (7,4) и решил использовать его для любого канала связи, предварительно не исследуя, присутствующие в нем шумы и помехи. Однако через какое-то время выяснилось, что выбранный код не обеспечивает помехоустойчивую связь, т.е. в блоке длинной 7 бит появляются двукратные ошибки. Задача состоит в том, чтобы по частоте возникновения ошибки в зависимости от номера блока определить, как часто встречаются ошибки и какой код Хемминга нужно использовать для полного устранения ошибок. Необходимо выбрать такой код, который будет обеспечивать наименьшую избыточность и исправлять все ошибки.

\inputfmtSection

Файл с распределение вероятности ошибки в зависимости от номера принятого блока, следующего формата: первая строка, содержит N – кол-во блоков, в следующих N строках pi – вероятность встретить ошибку в блоке i.

\outputfmtSection

Файл, содержащий код Хемминга, который позволяет избежать ошибок.

\sampleTitle{1}

\begin{myverbbox}[\small]{\vinput}
    11
    0
    0.12345679
    0.148148148
    0.074074074
    0
    0.074074074
    0.148148148
    0.12345679
    0
    0
    0.12345679    
\end{myverbbox}
\begin{myverbbox}[\small]{\voutput}
    (9, 5)
\end{myverbbox}
\inputoutputTable