\assignementTitle{Чет-нечет}{6}{}

Начинающий связист написал программу, реализующую алгоритм помехоустойчивого кодирования на основе кода Хемминга, например (7,4) и решил использовать его для любого канала связи, предварительно не исследуя, присутствующие в нем шумы и помехи. Однако через какое-то время выяснилось, что выбранный код не обеспечивает помехоустойчивую связь, т.е. в блоке длинной 7 бит появляются двукратные ошибки. Задача состоит в том, чтобы по частоте возникновения ошибки в зависимости от номера блока определить, как часто встречаются ошибки и какой код Хемминга нужно использовать для полного устранения ошибок. Необходимо выбрать такой код, который будет обеспечивать наименьшую избыточность и исправлять все ошибки.

\inputfmtSection

Файл с распределение вероятности ошибки в зависимости от номера принятого блока, следующего формата: первая строка, содержит N – кол-во блоков, в следующих N строках pi – вероятность встретить ошибку в блоке i.

\outputfmtSection

Файл, содержащий код Хемминга, который позволяет избежать ошибок.

\sampleTitle{1}

\begin{myverbbox}[\small]{\vinput}
    11
    0
    0.12345679
    0.148148148
    0.074074074
    0
    0.074074074
    0.148148148
    0.12345679
    0
    0
    0.12345679    
\end{myverbbox}
\begin{myverbbox}[\small]{\voutput}
    (9, 5)
\end{myverbbox}
\inputoutputTable

\markSection

Программа по входному файлу создает выходной файл. Если программа получает верный результат и выполняется не более 1 минуты, команда получает максимальное количество баллов - 6.

\commentsSection

\underline{Вопрос:} 

Под частотами ошибок, которые подают на ввод, имеются ввиду частоты появления в принципе ошибки хотя бы в одном бите или частоты появления именно двукратной ошибки?

\underline{Ответ:}

Речь идет о частоте ошибок на выходе декодера, работающего на основе кода Хемминга (7,4). Сбой происходит, если на вход декодера приходит блок с двумя и большим количеством ошибок.

Проверка проводилась в несколько этапов(попыток), проводимых в заданную дату (20.12.2018; 29.12.2018; 06.01.2019; 11.01.2019; 14.01.2019). В каждой попытке участники могли загрузить и проверить одно свое решение этой задачи.
Результирующая оценка за задачу выбиралась, как максимальная из попыток.

Загрузка решений проводилась на сайте \url{http://dep1.iszf.irk.ru/wlcomm} на котором было установлено специальное программное обеспечение для регистрации, разделения доступа и возможности загрузки решений участниками.

Тестирование решений проводилось на виртуальной машине с установленной ОС Kubuntu, проверку и начисление баллов за каждый тест осуществляли автоматические программы.

\explanationSection

Причина появления ошибок в случае, когда частота ошибок реже, чем ожидается в канале. 

\putImgWOCaption{13cm}{1}

\begin{center}
    Рисунок 1. Причина появления 2 кратной ошибки.
\end{center}

Видно, что если однократная ошибка появляется в 9-битовом блоке, то двукратная ошибка в 7-битном блоке появляется в случае, когда 7-битный блок оказывается на стыке двух 9 битовых блоков. На рисунке также показано как рассчитывается вероятность двукратной ошибки в зависимости от номера 7-битного блока.

На рисунке ниже представлен график вероятности, появления двукратной ошибки в зависимости от номера 7-битного блока. На графике виден характерный период, соответствующий частоте появления ошибки, в данном случае видно, что однократная ошибка встречается в среднем каждые 9 бит. Таким образом, для решения задачи необходимо определить период в представленной зависимости вероятности появления двукратной ошибки в зависимости от номера 7-битного блока.

\putImgWOCaption{13cm}{2}

\begin{center}
    Рисунок 2. Вероятность двукратной ошибки.
\end{center}

Для поиска периодичности в функции можно воспользоваться автокорреляционным подходом. В данном подходе определяется энергия $f(i)$  произведения исходной функции $p(j)$ и сдвинутой $p(i+j)$ на некоторый интервал $i$.
$$f(i)=\sum_{j=1}^N p(j) \times p(i+j)$$

При сдвиге $i$ кратном периоду функции, сдвинутая функция совпадет с исходной и энергия, при таких сдвигах, даст максимальные значения. На рисунке 3 представлена автокорреляционная функция $f(i)$ рассчитанная для функции, приведенной на рисунке 2.

\putImgWOCaption{13cm}{3}

\begin{center}
    Рисунок 3. Вид автокорреляционной функции для функции, представленной на рисунке 2.
\end{center}

Далее, вычисляя расстояния между главными максимумами, полученной автокорреляционной функции $f(i)$ находим период функции $m$.

Осталось определить какой код Хэмминга использовать для того чтобы общая длина блока была равна $m = n+k$, где $n$~– количество информационных бит, $k$~– количество контрольных бит. Для этого нужно решить неравенство:

$$2^k \geq k+n+1=m+1,$$

тогда 

$$k=[log_2(m)]+1.$$

В выходной фал записываем \textbf{(m,m-k)}.
