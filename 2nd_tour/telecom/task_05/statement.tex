\assignementTitle{Веселый спутник}{26}

Спутник занимается мониторингом землетрясений на удаленной планете. Программное обеспечение спутника позволяет определять границы континентов и местоположение эпицентра землетрясения. Зная его ориентацию и орбиту программы спутника позволяют определять географические координаты землетрясения и передавать их на Землю.

В процессе функционирования блок, отвечающий за ориентацию спутника вышел из строя, и спутник начал вращаться по очень сложной орбите, которую нельзя скорректировать и блок расчета координат землетрясения перестал функционировать. Съемка границ континентов и привязка к ним эпицентра землетрясения продолжает работать и это информация предаётся на Землю в виде трехмерного образа планеты (а именно в виде последовательности нулей, единиц и двоек). Единицам соответствуют границы континентов, двойкам - местоположение эпицентра землетрясения, нулям - все остальные точки. Контуры континентов расположены на поверхности воображаемой сферы с неизвестным радиусом, центром и ориентацией (они стали неизвестны из-за сбоя орбиты спутника). Кроме того, в данных может присутствовать случайный шум из единиц, возникший из-за передачи большого объема данных.

Задача: Сделать программу, которая по полученному трехмерному образу планеты (заданному файлом в виде последовательности 0, 1 и 2)и двумерной карте материков (заданному файлом в виде последовательности координат каждой границы - широта и долгота) определяет географические координаты землетрясения с точностью не хуже удвоенного размера дискрета в первом файле.

\explaneSection

\putImgWOCaption{17cm}{1}
\putImgWOCaption{17cm}{2}

Красная точка [2] на верхнем рисунке -- эпицентр землетрясения, который надо определить.

Синие точки -- границы континентов и случайный шум, остальные точки -- нули.

Снизу -- карта неизвестной планеты.
 

Приложения к задаче:
\begin{enumerate}
    \item Файл с двумерной картой неизвестной планеты
    \item Несколько примеров трехмерных образов планеты с нанесенными на них землетрясениями и результат определения координат для каждого варианта.
\end{enumerate}

\markSection

Чем быстрее работает программа, тем выше оценка (при условии получения необходимой точности определения координат).

