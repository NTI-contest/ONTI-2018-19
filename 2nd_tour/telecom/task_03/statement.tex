\assignementTitle{Сложный рельеф}{16}{}

Оператор сотовой связи принял решение обеспечить связью край Хоббитания. Хоббитания обладает холмистым рельефом, но не содержит крутых гор и скал. Оператор так планирует расположить вышки так, чтобы обеспечить связью всю территорию и при этом использовать наименьшее количество антенн – ретрансляторов. 

Рельеф Хоббитании дан в виде двумерного массива. Задача состоит в том, чтобы определить минимально необходимое число ретрансляторов и их положения, которые обеспечивали бы устойчивую связь между любыми двумя точками на данной территории.

\inputfmtSection

N M (кол-во строк, кол-во столбцов)

Далее в каждой строке через пробел приводятся М дробных чисел - высота рельефа.

\outputfmtSection

Первая строка – K (кол-во вышек);

Далее в каждой строке приводятся координаты вышек – два числа: номер строки и номер столбца в массиве входного файла.

