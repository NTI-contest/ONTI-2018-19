\assignementTitle{Сложный рельеф}{16}{}

\textbf{Тренируемые навыки: }освоение работы с дискретными величинами, растровыми алгоритмами, расчет сторон и углов треугольника, работа с массивами, рациональная организация циклов и условий.

\subsubsection*{Условие задачи}

Оператор сотовой связи принял решение обеспечить связью край Хоббитания. Хоббитания обладает холмистым рельефом, но не содержит крутых гор и скал. Оператор так планирует расположить вышки так, чтобы обеспечить связью всю территорию и при этом использовать наименьшее количество антенн – ретрансляторов. 

Антенна устанавливается на высоте 1 от поверхности рельефа. Хобитания будет считаться обеспеченной связью, если любой ее житель находится в прямой видимости хотя бы одной антенны. Сами антенны будут связаны оператором кабелем, находящемся в земле. (см.рисунок). Желательно расположить антенны как можно ближе к друг другу (чтобы уменьшить затраты оператора на монтаж и обслуживание антенн и кабеля).

\putImgWOCaption{10cm}{1}

Задача состоит в том, чтобы написать программу, которая для любого рельефа (удовлетворяющего условиям задачи) определяет положения двух антенн (ретрансляторов), которые обеспечивают устойчивую связь между любыми двумя жителями Хоббитании.

\inputfmtSection

Входной файл содержит форму рельефа Хобитании.

Первая строка заголовок.

Каждая последующая строка содержит запись, представляющую собой 3 числа, разделенных табуляцией:
X(координату),Y(координату) и Z(высоту) точки рельефа.

Всего записей в файле $101 \cdot 101$.

\outputfmtSection

Первая строка – координаты первой антенны (X1, Y1);

Вторая строка – координаты второй антенны (X2, Y2);

Программа должна читать входной файл со стандартного потокв ввода, и выдавать выходной файл на стандратный поток вывода.

\markSection

Если задача решена успешно, и антенны обеспечивают устойчивую связь в Хобитании, то баллы определяется по формуле:
$$\text{Баллы}=16-8 \cdot (R-R_{min})/(R_{max}-R_{min})$$

$R_{max}$ — максимально возможное расстояние между антеннами;

$R_{min}$ — минимально возможное расстояние между антеннами;

$R$ — расстояние между антеннами, полученное вашей программой;

Таким образом, максимальное количество баллов заработает тот, кто поставит антенны, обеспечивающие связь в Хобитании, ближе друг к другу;

Программа должна работать не более 10 минут и использовать не более 100МБ памяти.

\commentsSection

\underline{Вопрос:} 

Для определения видимости вышки с произвольных точек рельефа нужно, чтобы рельеф был представлен непрерывной поверхностью. Но рельеф в задаче представлен набором дискретных. Как из них выразить непрерывную поверхность?

\underline{Ответ:} 

Рельеф представляет собой набор вокселей с единичной стороной (растр в трёхмерном пространстве или сплайн 0-й степени).

\underline{Вопрос:} 

Как считается расстояние при начисления баллов: не учитывая z точек или учитывая ее?

\underline{Ответ:} 

Расстояние рассчитывается с учетом z координат точек.

Проверка проводилась в несколько этапов(попыток), проводимых в заданную дату (20.12.2018; 29.12.2018; 06.01.2019; 11.01.2019; 14.01.2019). В каждой попытке участники могли загрузить и проверить одно свое решение этой задачи.
Результирующая оценка за задачу выбиралась, как максимальная из попыток.

Загрузка решений проводилась на сайте \url{http://dep1.iszf.irk.ru/wlcomm} на котором было установлено специальное программное обеспечение для регистрации, разделения доступа и возможности загрузки решений участниками.

Тестирование решений проводилось на виртуальной машине с установленной ОС Kubuntu, проверку и начисление баллов за каждый тест осуществляли автоматические программы.

\subsection*{Алгоритм решения задачи}

Для решения задачи нужно найти две ближайшие точки $P_1$ и $P_2$, из которых остальные точки Хобитании будут находиться в прямой видимости.

\begin{enumerate}
    \item \textbf{Проверки видимости точки.}
    Точка $p_{ij}$ видна из точки $P_k$ если ее угол места $\theta_ij$ больше угла места $\theta$ любой другой точки на прямой соединяющей точки $P_k$ и $p_{ij}$, см. рисунок 1 и 2. Угол места может быть рассчитан так:
    $$sin(\theta_{ij} )=\frac{z_{ij}-z_k}{((x_{ij}-x_k )^2+(y_{ij}-y_k )^2 ) )}$$

    \putImgWOCaption{13cm}{2}

    \begin{center}
        Рисунок 1. Вариант с отрицательыми углами места
    \end{center}

    \putImgWOCaption{13cm}{3}

    \begin{center}
        Рисунок 2. Вариант с положительными углами места.
    \end{center}

    \item \textbf{Построение прямой на растре.}
    Так как рельеф задан дискретно на матрице, для определения точек лежащих на прямой можно воспользоваться, например, растровым алгоритмом Брезенхема. Алгоритм основан на формуле между двумя точками. Для каждого $x$ в заданном диапазоне, значение $y$ получается округлением к целому следующего выражения: 
    $$y=\frac{y_1-y_0}{x_1-x_0} (x-x_0 )+y_0$$
    
    Рассмотрим вариант расположения точек, представленный на рисунке 3. На каждом шаге приращения $x$ на 1 $y$ изменяется на $\dfrac{y_1-y_0}{x_1-x_0}$, а так как \linebreak $|x_1-x_0 |>|y_1-y_0 |$, то $y$ сохраняет свое значение, либо уменьшается на 1. Данный алгоритм можно распространить на вычисление координат клетки, принадлежащей прямой, во всех направлениях. Это достигается за счет: зеркальных отражений, то есть заменой знака (шаг в 1 заменяется на -1), обменом переменных $x$ и $y$, обменом координат начала отрезка с координатами конца.

    \putImgWOCaption{13cm}{4}

    \begin{center}
        Рисунок 3. Растровый алгоритм построения прямой.
    \end{center}

    \item \textbf{Поиск оптимального расположения антенн.}
    \begin{enumerate}
        \item[3.1] Первую антенну располагаем последовательно во всех точках.
        \item[3.2] Для выбранного положения первой антенны, определяем массив невидимых относительно нее точек.
        \item[3.3] Располагаем вторую антенну во всех точках, в которой еще не было первой антенны.
        \item[3.4] Для каждого выбранного положения второй антенны проверяем массив точек невидимых относительно первой антенны. Если в массиве встречается точка, которая невидна также из текущего положения второй антенны, то выходим из цикла перебора положения второй антенны и ставим первую антенну на новую позицию (переходим к 3.1).
        \item[3.5] Если в 3.4 все невидимые точки относительно антенны 1 просматриваются из антенны 2, тогда рассчитываем расстояние между ними. Если найденное расстояние меньше найденного ранее, тогда запоминаем новые положения антенн и расстояние между ними, после переходим к пункту 3.1.
    \end{enumerate}
\end{enumerate}
