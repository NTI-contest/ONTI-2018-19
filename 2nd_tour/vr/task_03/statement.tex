\assignementTitle{RollerBall-1}{100}{}

Юный программист Вася решил поиграть в RollerBall. Цель игры — собрать как можно больше монет, которые находятся в лабиринте. Для того, чтобы взять монетку, её границы нужно пересечь мячом, которым Вася может управлять. У мяча есть энергия, которая расходуется на каждом шаге.

Вася хочет собрать очень много монет и для этого решил написать программу, которая бы управляла мячом и сама собирала все монеты в лабиринте. К сожалению, Вася очень плохо знает C\# и просит Вас ему помочь. Проект, который написал Вася, уже содержит игру и умеет вводить и выводить файлы нужных форматов. Вам осталось реализовать класс AutoBallControl для управления мячом.

Первый тест совпадает с примером, содержащимся в файле input.txt в репозитории проекта.

\inputfmtSection

Входной файл содержит 5 целых чисел $S$, $R$, $C$, $N$, $E$ —- номер (seed) лабиринта, количество строк и столбцов лабиринта, количество монет, количество энергии у мяча.

$0\leq S \leq 10^7$

$1\leq R \leq 600$

$1\leq C \leq 600$

$2\leq R\cdot C \leq 600$

$0\leq N \leq 9$

$0\leq E \leq 3\cdot 10^4$

\outputfmtSection

Файл с решением должен содержать реализацию класса AutoBallControl.

\includeSolutionIfExistsByPath{2nd_tour/vr/task_03}