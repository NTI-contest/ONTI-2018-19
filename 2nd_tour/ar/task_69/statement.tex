\assignementTitle{Часть 2}{60}{}

\begin{enumerate}
    \item Создайте AR-приложение для мобильного устройства, которое позволяет высвечивать линии градуировки сетки, расположенной между четырьмя маркерами при наведении экрана смартфона на них. Размер сетки: количество ячеек по длине и ширине задается в приложении, в соответствии с этим, автоматически, определяется размер одной ячейки. Работающее приложение оценивается в 30 баллов. Если в приложении реализована статическая сетка (не растягивается при изменении положения маркеров), то работа оценивается в 15 баллов.
    \item Реализуйте в приложении, созданного в предыдущем шаге, возможность визуализировать виртуальные объекты, появляющиеся над поверхностью эмуляции сетки, при вводе соответствующих координат в диалоговое окно или текстовые поля приложения (+15 баллов).
    \item Модифицируйте приложение таким образом, чтобы координаты добавляемого виртуального объекта на сетку задавались не в окне приложения, а касанием соответствующей виртуальной ячейке на экране смартфона, либо наведением фокуса приложения (крестика/прицела) на соответствующую клетку, над которой будет появляться виртуальный объект (+15 баллов).
\end{enumerate}

На рисунке изображена сетка и результат работы приложения.

\putImgWOCaption{15cm}{1}

Необходимо предоставить приложение, выполненное для мобильного устройства, работающего под управлением ОС Android, маркеры (используемые для обозначения углов сетки) и короткий видеоролик, демонстрирующий функциональные возможности AR-приложения в работе. 

В ответ на задание приложите ссылку на архив с приложением и остальными материалами. 

Оценка за приложение выставляется экспертом вручную на следующем шаге.

РЕКОМЕНДАЦИИ:

Есть несколько подходов к решению этой части задачи Института внеземных культур. Один из них совпадает с предыдущей задачей, только вместо маркеров aruco будут следует использовать обычные маркеры. Если вы используете библиотеку Vuforia, то она позволяет получить "в коде" описание положения маркера (угол наклона, координаты).

Отображать над сеткой в соответствующей координате можно, как плоские, так и трехмерные модели объектов, как с анимацией, так и без, как по одному объекту, так и все три: «пустышка» с содержимым, «пустышка» без содержимого, объект-призрак. Однако стоит учитывать, что все это влияет на формирование итогового балла по данному шагу задачи.

\solutionSection

Данную задачу можно свести к задаче генерации и масштабированию объектов. Для решения можно воспользоваться следующим подходом:

\begin{enumerate}
    \item Создать единицу сетки “клетку” из четырех кубов. И объединить все объекты с помощью одного EmptyObject, чтобы мы могли работать с ними как с одним объектом. Переименуем EmptyObject в Cell
    \item Создадим класс GridMaker. Он будет генерировать поле заданного размера клеток из объекта Cell.
    \inputminted[fontsize=\footnotesize, linenos]{csharp}{2nd_tour/ar/task_69/source_1.cs}

    Полученное сгенерированное поле помещается под единый родительский объект Grid (нужно создать заранее). Это делается для того, чтобы сетку было удобнее масштабировать под маркеры. 
    \item Далее нам нужно, чтобы поле автоматически натягивалось между двумя маркерами. Для этого будет использован класс GridTransform. Он получает координаты внутренних углов маркеров. Вычисляет позицию куда нужно переместить поле, а также из расстояния между маркерами определяет, как нужно растянуть сетку.  
    
    \inputminted[fontsize=\footnotesize, linenos]{csharp}{2nd_tour/ar/task_69/source_2.cs}
    Все это делается в методе Update для того чтобы сетка автоматически обновлялась и подстраивалась при перемещении маркеров.
    
    \item Для того, чтобы по нажатию на клетку появлялись объекты, можно в объект Cell  добавить элемент Cavas'а button. Нужно реализовать класс Changer в котором прописать метод change. Его добавить на нижнюю часть собранной клетки. В onClick у кнопки нужно указать ссылку на метод change нужного объекта.  В методе реализуется генерация нужного 3D объекта с помощью метода Instance
\end{enumerate}