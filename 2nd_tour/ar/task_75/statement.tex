\assignementTitle{}{30}{}

Теперь сталкеры работают в Зоне одновременно, но им нельзя вместе находится на одной локации, чтобы не раскрыть друг друга. Ваша задача как организатора - собрать все артефакты и провести каждого сталкера так, чтобы он не встретился с остальными. Критерием будет общее время нахождения сталкеров в Зоне. Сталкер может проходить по локациям, где есть "чужие" артефакты. Время старта каждого сталкера можно выбирать произвольно.

\inputfmtSection

В 1-й строке - число минут, имеющееся в вашем распоряжении, далее - три строки с номерами локаций с артефактами

\outputfmtSection

3 строки, где первое число в строке: момент старта (рекомендуется 0 для одного из сталкеров), третье и последующие числа в строке: путь обхода (номера локаций, начинается и заканчивается в локации 1). Если по истечении выделенного времени кто-то из сталкеров остался в Зоне, решение не принимается.

\begin{myverbbox}[\small]{\vinput}
    99
    2 8 10
    3 9 11
    5 7 13
\end{myverbbox}
\begin{myverbbox}[\small]{\voutput}
    2 1 2 3 4 7 8 7 10 7 4 3 2 1
    0 1 2 3 4 7 8 9 10 11 12 15 1
    1 1 6 5 7 10 11 13 15 1
\end{myverbbox}
\inputoutputTable

\explanationSection

Перед решением этой задачи необходимо решить предыдущие (№№ 3, 4, 5).

Аналогично решению задачи 5 необходимо построить маршрут через требуемые локации с артефактами. Важно во время обхода графа записывать не только длину пути, но и пройденные вершины (в список или стек). Используя перестановку порядка обхода локаций, находим оптимальный по длине маршрут для каждого сталкера.

Особенность этой задачи в том, что сталкеры не могут входить в Зону одновременно.

Определим для каждого из сталкеров паузу перед стартом (в минутах), например: 0, 1 и 2. Как и в случае с локациями, здесь 3! = 6 перестановок: (0 1 2), (0 2 1), (1 0 2), (1 2 0), (2 0 1), (2 1 0). Кроме того, нужно проверить, не окажутся ли два сталкера на одной локации одновременно. Для этого построим для пути каждого сталкера список из пар значений (время; локация). Проверим, нет ли совпадающих пар (столкновений на локациях). Если случилось столкновение, можно попробовать другие перестановки пауз или удлинить паузу того сталкера, длительность пути которого меньше.

Заметим, что оценивается общее время нахождения сталкеров на локациях, поэтому варьировать паузы на старте лучше у тех сталкеров, путь которых короче.