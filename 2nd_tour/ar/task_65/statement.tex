\assignementTitle{}{7}{}

Продолжаем знакомство с GeoJSON. Вам даны контуры нескольких стран в формате GeoJSON (архив по ссылке \url{https://github.com/ipetrushin/NTI-Contest/blob/master/8countries-geojson.zip}). 
С помощью онлайн-инструментов определите, какой из стран соответствует каждый из файлов и сопоставьте элементы списка ниже.

\begin{multicols}{2}
    {
        \begin{enumerate}
            \item CountryA
            \item CountryB
            \item CountryC
            \item CountryD
            \item CountryE
            \item CountryF
            \item CountryG
        \end{enumerate}
    }

    {
        \begin{enumerate}
            \item[а.] Мальдивы
            \item[б.] Италия
            \item[в.] Россия
            \item[г.] Мальта
            \item[д.] Перу
            \item[е.] Камбоджа
            \item[ж.] Ватикан
        \end{enumerate}
    }  
\end{multicols}

\explanationSection

Каждый из контуров  в файле можно наложить на реальную географическую карту с помощью онлайн-сервиса, например \url{http://geojson.io}, по положению и подписи легко определить страну.

\answerMath{1 - б, 2 - е, 3 - а, 4 - ж, 5 - в, 6 - д, 7 - г.}