Рэдрик Шухарт* возглавляет отдел по изучению Зон Посещения Института (\url{https://ru.wikipedia.org/wiki/%D0%9F%D0%B8%D0%BA%D0%BD%D0%B8%D0%BA_%D0%BD%D0%B0_%D0%BE%D0%B1%D0%BE%D1%87%D0%B8%D0%BD%D0%B5}) внеземных культур. Теперь подвергать людей риску нет никакой необходимости, вместо сталкеров на территорию контакта с внеземной цивилизацией отправляются мобильные роботы. Они передают в Институт всю необходимую информацию об аномалиях. Информация отображается на плоской цифровой сетке-табло, имитирующей поверхность одного из многочисленных квадратов Зоны. На краях сетки установлены маркеры, отмечающие границы изучаемой территории. Красным внутри сетки отмечается квадрат, в котором находится «пустышка»** с содержимым (сущностью внутри), синий квадрат – места нахождения пустышки без содержания, зеленый – отделившаяся от пустышек сущность - виртуальный объект, «призрак», плавающий над поверхностью Зоны. Камера фиксирует состояние табло в течение дня в виде серии снимков, которые передаются в отдел для анализа. 

Стажерам-исследователям, принятым в отдел, необходимо:

разработать программу, позволяющую распознать снимки и вывести в файл координаты перемещения объектов внутри сетки за день;
создать AR-приложение, позволяющее отобразить виртуальную сетку и объекты на ней в произвольном месте и масштабе
За полное и верное решение каждой части задачи участник получают 60 баллов. Кроме того, оценивается частичное решение каждой задачи: первой в зависимости от процента верно распознанных изображений, второй – от части реализованной демонстрационной модели. Описание частей – в следующих шагах курса.

Примечания:

*Рэдрик Шухарт – сталкер, главный герой фантастической повести Аркадия и Бориса Стругацких «Пикник на обочине» (\url{https://ru.wikipedia.org/wiki/%D0%9F%D0%B8%D0%BA%D0%BD%D0%B8%D0%BA_%D0%BD%D0%B0_%D0%BE%D0%B1%D0%BE%D1%87%D0%B8%D0%BD%D0%B5}).

** «Пустышка» представляет собой два диска, между которыми находится пространство, то есть механически они не связаны. При этом диски нельзя сдвинуть с места относительно друг друга. В наши дни стало известно, что внутри пустышки может находиться сущность – некий виртуальный объект, наблюдать который можно через специальные устройства


\assignementTitle{Часть 1}{60}{}
В папке dataset (архив на Google Диске \url{https://goo.gl/whS5jm}) находятся фотографии 
электронного табло, отображающего состояние одного из территориальных квадратов Зоны в течение дня. 
Все они имеют название cadrN.png", где N - номер полученного кадра.

\putImgWOCaption{11cm}{1}

Необходимо обработать все фотографии в соответствии с их очередностью и в результате получить файл, каждая строчка которого соответствует положению сначала красного, зеленого и желтого объекта на каждом отдельном фото. 

Для разделения кадров использовать "----------" (10 знаков "минус").

Т.е. для примера одному кадру, будет соответствовать файл: 

r - (7,5)

g - (4,3)

y - (7,6)

----------

А для двух фотографий, файл будет выглядеть так (и так далее):

r - (7,5)

g - (4,3)

y - (7,6)

----------

r - (8,6)

g - (3,3)

y - (3,0)

----------

Пример выходного файла для 1561 фотографии -  aruko-markers-output-test.txt (\url{https://github.com/ipetrushin/NTI-Contest/blob/master/aruko-markers-output-test.txt}).

РЕКОМЕНДАЦИИ: 

В файле random\_point.py (\url{https://github.com/ipetrushin/NTI-Contest/blob/master/random_point.py}) пример того, 
как формируется выходной файл на языке python.

Для упрощения работы возможно использование библиотеки OpenCV (\url{https://docs.opencv.org}) и конкретного 
модуля в ней cv2.aruco (\url{https://docs.opencv.org/3.1.0/d5/dae/tutorial_aruco_detection.html}).

В первую очередь нужно детектировать сетку и преобразовать ее к нормальной форме (к прямоугольнику) с 
помощью аффинных преобразований. Для распознавания цвета объектов рекомендуется использование цветовой модели HSV.

\solutionSection

\begin{enumerate}
    \item На фотографии находятся ARuko маркеры.
    \item По внутренним углам маркеров находится сетка.
    \item В зависимости от положения маркеров сетка разворачивается таким образом, чтобы начало координат было в нижнем левом углу.
    \item Получаются все цветные клетки сетки и определятся их цвета.
\end{enumerate}

Рассмотрим по отдельности каждый пункт.

Для того, чтобы не писать самим алгоритмы распознования нужно установить библиотеку OpenCV и отдельный модуль из нее для распознавания маркеров aruko.

Далее подключаем эти библиотеки.

\inputminted[fontsize=\footnotesize, linenos]{python}{2nd_tour/ar/task_68/source_1.py}

Главная функция программы

\inputminted[fontsize=\footnotesize, linenos]{python}{2nd_tour/ar/task_68/source_2.py}

Получение сетки между маркерами

\textit{Алгоритм работы данного блока:}

\begin{enumerate}
    \item Находим внутренние углы маркеров.
    \item Для этого нужно найти у каждого маркера те углы, которые ближе всего к центру. Далее сортируем их по правильному порядку следования Aruko маркеров, как на картинке.
\end{enumerate}

С помощью афинных преобразований преобразуем часть изображения между внутренними углами к прямому.

\putImgWOCaption{11cm}{2}

\textit{Получение внутренних углов}

\inputminted[fontsize=\footnotesize, linenos]{python}{2nd_tour/ar/task_68/source_3.py}

\textit{Код для афинных преобразований}

\inputminted[fontsize=\footnotesize, linenos]{python}{2nd_tour/ar/task_68/source_4.py}

\textit{Результат работы кода:}

Рисунок 1 - оригинальное фото

Рисунок 2 - вырезанная и преобразованная сетка

\putTwoImg{7cm}{1}{7cm}{2}

\textit{Получение объектов (цвеных клеток) на сетке}

Для начала нам нужно просто определить нахождение всех цветных квадратиков на сектке. Для этого можно бинаризировать изображение по яркости (объекты на сетке светлее, чем окружающие их клетки). Чтобы определить степень яркости переведем изображение из модели RGB в модель HSV.

Чтобы определить какого цвета каждый объект, отсортируем их по параметру H(цветовому тону). Соответсвенно, вначале списка тогда окажется красный, потом желтый затем зеленый.

\textit{Алгоритм работы данного блока}
\begin{enumerate}
    \item Переводим изображение в HSV
    \item Бинаризируем изображения по яркости и насыщенности. Применяем размытие гаусса, чтобы убрать лишние проблески.
    \item Для того, чтобы отделить объекты между собой, находим на изображении контуры.
    \item Находим центральный пиксель каждого контура, определяем его цвет в модели HSV. Формируем список по типу: [(значение пикселя в HSV, X, Y) , (значение пикселя в HSV, X, Y)]
    \item   Сортируем по параметру H.
    \item Переводим координаты объектов из координат на изображении к координатам на сетке.
\end{enumerate}

\textit{Получение координат объектов на изображении}

\inputminted[fontsize=\footnotesize, linenos]{python}{2nd_tour/ar/task_68/source_5.py}

\textit{Преобразование к координатам на сетке}

\inputminted[fontsize=\footnotesize, linenos]{python}{2nd_tour/ar/task_68/source_6.py}

\textit{Функция преобразования для одной клетки}

\inputminted[fontsize=\footnotesize, linenos]{python}{2nd_tour/ar/task_68/source_7.py}

\textit{Результат}

После применения главной функции detect, будет получен следующий результат:

\putImgWOCaption{7cm}{5}

Примечание: на данный момент для того, чтобы вырезать сетку нужно находить все 4 маркера. Однако на чати фотографий всего 3 маркера из 4х. В такой ситуации можно восстановить сетку по следующему алгоритму:

\textit{Алгоритм:}
\begin{enumerate}
    \item Находим диагональные маркеры (к примеру, верхний правый и нижний левый).
    \item Строим прямую по нижним координатам верхнего маркера и внутренним правым нижнего. Пересечение двух прямых это верхний угол сетки (См. рис1).
    \item Соответственные действия производим для нахождения нижнего угла сетки (См. рис2) .
    \item По полученным углам восстанавливаем сетку.
\end{enumerate}

\putTwoImg{7cm}{6}{7cm}{7}
