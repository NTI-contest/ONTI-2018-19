Рэдрик Шухарт* возглавляет отдел по изучению Зон Посещения Института (\url{https://ru.wikipedia.org/wiki/%D0%9F%D0%B8%D0%BA%D0%BD%D0%B8%D0%BA_%D0%BD%D0%B0_%D0%BE%D0%B1%D0%BE%D1%87%D0%B8%D0%BD%D0%B5}) внеземных культур. Теперь подвергать людей риску нет никакой необходимости, вместо сталкеров на территорию контакта с внеземной цивилизацией отправляются мобильные роботы. Они передают в Институт всю необходимую информацию об аномалиях. Информация отображается на плоской цифровой сетке-табло, имитирующей поверхность одного из многочисленных квадратов Зоны. На краях сетки установлены маркеры, отмечающие границы изучаемой территории. Красным внутри сетки отмечается квадрат, в котором находится «пустышка»** с содержимым (сущностью внутри), синий квадрат – места нахождения пустышки без содержания, зеленый – отделившаяся от пустышек сущность - виртуальный объект, «призрак», плавающий над поверхностью Зоны. Камера фиксирует состояние табло в течение дня в виде серии снимков, которые передаются в отдел для анализа. 

Стажерам-исследователям, принятым в отдел, необходимо:

разработать программу, позволяющую распознать снимки и вывести в файл координаты перемещения объектов внутри сетки за день;
создать AR-приложение, позволяющее отобразить виртуальную сетку и объекты на ней в произвольном месте и масштабе
За полное и верное решение каждой части задачи участник получают 60 баллов. Кроме того, оценивается частичное решение каждой задачи: первой в зависимости от процента верно распознанных изображений, второй – от части реализованной демонстрационной модели. Описание частей – в следующих шагах курса.

Примечания:

*Рэдрик Шухарт – сталкер, главный герой фантастической повести Аркадия и Бориса Стругацких «Пикник на обочине» (\url{https://ru.wikipedia.org/wiki/%D0%9F%D0%B8%D0%BA%D0%BD%D0%B8%D0%BA_%D0%BD%D0%B0_%D0%BE%D0%B1%D0%BE%D1%87%D0%B8%D0%BD%D0%B5}).

** «Пустышка» представляет собой два диска, между которыми находится пространство, то есть механически они не связаны. При этом диски нельзя сдвинуть с места относительно друг друга. В наши дни стало известно, что внутри пустышки может находиться сущность – некий виртуальный объект, наблюдать который можно через специальные устройства


\assignementTitle{Часть 1}{60}{}
В папке dataset (архив на Google Диске \url{https://goo.gl/whS5jm}) находятся фотографии 
электронного табло, отображающего состояние одного из территориальных квадратов Зоны в течение дня. 
Все они имеют название cadrN.png", где N - номер полученного кадра.

\putImgWOCaption{11cm}{1}

Необходимо обработать все фотографии в соответствии с их очередностью и в результате получить файл, каждая строчка которого соответствует положению сначала красного, зеленого и желтого объекта на каждом отдельном фото. 

Для разделения кадров использовать "----------" (10 знаков "минус").

Т.е. для примера одному кадру, будет соответствовать файл: 

r - (7,5)

g - (4,3)

y - (7,6)

----------

А для двух фотографий, файл будет выглядеть так (и так далее):

r - (7,5)

g - (4,3)

y - (7,6)

----------

r - (8,6)

g - (3,3)

y - (3,0)

----------

Пример выходного файла для 1561 фотографии -  aruko-markers-output-test.txt (\url{https://github.com/ipetrushin/NTI-Contest/blob/master/aruko-markers-output-test.txt}).

РЕКОМЕНДАЦИИ: 

В файле random\_point.py (\url{https://github.com/ipetrushin/NTI-Contest/blob/master/random_point.py}) пример того, 
как формируется выходной файл на языке python.

Для упрощения работы возможно использование библиотеки OpenCV (\url{https://docs.opencv.org}) и конкретного 
модуля в ней cv2.aruco (\url{https://docs.opencv.org/3.1.0/d5/dae/tutorial_aruco_detection.html}).

В первую очередь нужно детектировать сетку и преобразовать ее к нормальной форме (к прямоугольнику) с 
помощью аффинных преобразований. Для распознавания цвета объектов рекомендуется использование цветовой модели HSV.