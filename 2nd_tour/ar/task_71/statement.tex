\assignementTitle{}{6}{}

На рисунке изображена сетка 2х2 клетки. Если двигаться от левого верхнего угла сетки в правый нижний, перемещаясь только вниз или вправо, то существует ровно 6 различных маршрутов (см. рисунок). 

Определите число различных маршрутов для сетки размером 20х20 клеток. 

\putImgWOCaption{6cm}{1}

\solutionSection

Из условия задачи следует, что длина любого пути одинакова и равна $2N$, где $N$ - число клеток. Обозначим конкретный путь буквами В (вниз) и П (право), число ходов вниз и вправо одинаково. Например так выглядят обозначения путей, приведённых на рисунке: ППВВ, ПВПВ, ПВВП, ВППВ, ВПВП, ВВПП. Рассчитаем число возможных перестановок (всего $4!$), с учётом, что буквы В и П повторяются $frac{4!}{2!  \cdot  2!} = \frac{1 \cdot 2 \cdot 3 \cdot 4}{1 \cdot 2 \cdot 1 \cdot 2} = 2 \cdot 3 = 6$. Для поля $20 \cdot 20$, соответственно, число путей = $\frac{40!}{20!  \cdot  20!} = 137846528820$.

\answerMath{137846528820.}

