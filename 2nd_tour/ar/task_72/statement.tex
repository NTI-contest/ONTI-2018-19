\assignementTitle{}{10}{}

С самого своего появления Зона притягивала к себе авантюристов - сталкеров. В Зоне на некоторых локациях располагаются артефакты. Каждый сталкер (или команда) собирает только определённые артефакты, которые нужны заказчику.

Постепенно совместными усилиями сталкеры составили карту всех локаций в Зоне, пускай и неточную, 
    но сильно упростившую ориентацию на местности. Карта одинаковая для всех задач и представлена в виде графа
    (\url{https://ru.wikipedia.org/wiki/%D0%93%D1%80%D0%B0%D1%84_(%D0%BC%D0%B0%D1%82%D0%B5%D0%BC%D0%B0%D1%82%D0%B8%D0%BA%D0%B0)#%D0%A1%D0%BF%D0%B8%D1%81%D0%BE%D0%BA_%D1%80%D1%91%D0%B1%D0%B5%D1%80}), заданного числом вершин (от 1 до N) и набором рёбер (указывается также длина ребра - время пути между локациями в минутах).

Для удобства мы приводим схематическую иллюстрацию карты и её текстовое представление (в тексте ниже первая строка - число вершин и рёбер; далее следуют номера соединяемых вершин и вес каждого ребра)

Используйте эту карту для выполнения заданий в следующих шагах этого модуля.

Для поиска кратчайшего пути вам может потребоваться алгоритм Дейкстры или поиск по графу в ширину.

\putImgWOCaption{13cm}{1}

\begin{longtable}{l}
    15 21 \\
    1 2 10 \\
    1 15 14 \\
    1 6 16 \\ 
    2 3 6 \\
    3 6 9 \\
    3 4 6 \\
    4 5 7 \\
    4 7 10 \\
    5 6 11 \\
    5 7 13 \\
    7 8 3 \\
    7 10 10 \\
    8 9 12 \\
    9 10 5 \\
    10 11 7 \\
    11 12 7 \\
    11 13 11 \\
    11 14 21 \\
    12 15 16 \\
    13 14 20 \\
    13 15 14 \\
\end{longtable}

Для заданной карты необходимо проложить кратчайший маршрут от данной локации до всех остальных. Программа получает на вход одно число от 1 до 15. В качестве результата сообщите кратчайший путь до каждой локации (15 чисел в одну строку через пробел, путь до локации старта = 0).

\begin{myverbbox}[\small]{\vinput}
    1
\end{myverbbox}
\begin{myverbbox}[\small]{\voutput}
    0 10 16 22 27 16 32 35 47 42 37 30 28 48 14
\end{myverbbox}
\inputoutputTable

\explanationSection

Возможно несколько способов решения задачи, мы будем использовать рекурсивный алгоритм. Выберем начальную вершину (из входных данных), например 1-ю. Создадим линейный массив размером N (число вершин), в котором будем хранить длину кратчайшего пути от начальной вершины до всех остальных. В ячейку, соответствующую начальной вершине запишем число 0, остальные заполним числом -1, чтобы обозначить, что до остальных вершин длина пути неизвестна. Число -1 выбрано, так как длина не может быть отрицательной. 

По исходным данным (карта дана во введении) построим матрицу смежности (\url{https://ru.wikipedia.org/wiki/%D0%9C%D0%B0%D1%82%D1%80%D0%B8%D1%86%D0%B0_%D1%81%D0%BC%D0%B5%D0%B6%D0%BD%D0%BE%D1%81%D1%82%D0%B8}).

Используя рекурсию переходим от начальной вершины к тем, которые связаны с ней, отмечая для этих вершин длину пути = 1. Аналогичным образом делаем следующий шаг, но в массив записывается длина пути = 2. Важно на каждом шаге сравнивать длину пути до данной вершины, которую мы получили рекурсивным обходом в данный момент и длину, записанную в массиве. Стоит делать шаг только в том случае, если в массиве хранится длина -1 (то есть по этой вершине ещё не проходили) или бОльшее значение, чем получили рекурсивным обходом.

Когда все вызовы рекурсий завершатся, в массиве получим длины кратчайших путей от начальной вершины до всех остальных.