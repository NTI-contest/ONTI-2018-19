С самого своего появления Зона притягивала к себе авантюристов - сталкеров. В Зоне на некоторых локациях располагаются артефакты. Каждый сталкер (или команда) собирает только определённые артефакты, которые нужны заказчику.

Постепенно совместными усилиями сталкеры составили карту всех локаций в Зоне, пускай и неточную, 
    но сильно упростившую ориентацию на местности. Карта одинаковая для всех задач и представлена в виде графа
    (\url{https://ru.wikipedia.org/wiki/%D0%93%D1%80%D0%B0%D1%84_(%D0%BC%D0%B0%D1%82%D0%B5%D0%BC%D0%B0%D1%82%D0%B8%D0%BA%D0%B0)#%D0%A1%D0%BF%D0%B8%D1%81%D0%BE%D0%BA_%D1%80%D1%91%D0%B1%D0%B5%D1%80}), заданного числом вершин (от 1 до N) и набором рёбер (указывается также длина ребра - время пути между локациями в минутах).

Для удобства мы приводим схематическую иллюстрацию карты и её текстовое представление (в тексте ниже первая строка - число вершин и рёбер; далее следуют номера соединяемых вершин и вес каждого ребра)

Используйте эту карту для выполнения заданий в следующих шагах этого модуля.

Для поиска кратчайшего пути вам может потребоваться алгоритм Дейкстры или поиск по графу в ширину.

\putImgWOCaption{13cm}{1}

ВСТАВИТЬ ТАБЛИЦУ!!!

%10 11 711 12 711 13 1111 14 2112 15 1613 14 2013 15 14

\assignementTitle{}{10}{}

Для заданной на предыдущем шаге карты необходимо проложить кратчайший маршрут от данной 
локации до всех остальных. Программа получает на вход одно число от 1 до 15. 
В качестве результата сообщите кратчайший путь до каждой локации 
(15 чисел в одну строку через пробел, путь до локации старта = 0)

\begin{myverbbox}[\small]{\vinput}
    1
\end{myverbbox}
\begin{myverbbox}[\small]{\voutput}
    0 10 16 22 27 16 32 35 47 42 37 30 28 48 14
\end{myverbbox}
\inputoutputTable

%\includeSolutionIfExistsByPath{2nd_tour/ar/task_72}