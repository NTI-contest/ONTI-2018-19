\assignementTitle{}{12}{}

Для заданного графа (см. описание карты) каждому из трёх сталкеров известны номера локаций на карте, которые содержат нужные артефакты. Артефакты необходимо собирать в том порядке, каком они перечислены. Вход и выход из Зоны через локацию № 1.

Определите, какой из сталкеров сможет собрать артефакты за наименьшее время.

\inputfmtSection

Три строки с номерами локаций с артефактами

\outputfmtSection

Номер строки (одного из сталкеров) и общее время прохода его по маршруту

\begin{myverbbox}[\small]{\vinput}
    5 8 9
    2 9 13
    6 11 15
\end{myverbbox}
\begin{myverbbox}[\small]{\voutput}
    3 94
\end{myverbbox}
\inputoutputTable

\explanationSection

Используя алгоритм Дейкстры или описанный в решении задачи 3 рекурсивный обход графа, построим участки пути от точки старта (локации 1) через необходимые локации обратно до точки старта. Сложив длины каждого участка получим общую длину пути для каждого сталкера. Сравнив пути выбираем кратчайший и выводим на экран.

%\includeSolutionIfExistsByPath{2nd_tour/ar/task_73}