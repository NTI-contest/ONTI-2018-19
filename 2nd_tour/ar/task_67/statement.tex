\assignementTitle{}{17}{}

Шаг 1. Если вы выполнили предыдущее задание, пропустите этот шаг. Иначе, скачайте файл, содержащий описание зданий, являющихся достопримечательностями города Иркутска. По описанию отыщите эти здания в сети интернет и определите их географические координаты. Заполните таблицу с полями:

\begin{enumerate}
    \item Координаты lon, lat
    \item Описание
    \item Название
\end{enumerate}

Вам необходимо найти координаты ЦЕНТРА каждого здания, мы рекомендуем для этого воспользоваться сервисом 2ГИС (ВАЖНО!). Заполненная Вами таблица будет являться вспомогательным инструментом в следующих этапах задания.

Шаг 2. Правильное решение шага задачи оценивается в 17 баллов.  Вам необходимо определить координаты (lon, lat) расположения пользователя AR-браузера, находясь в котором он сможет видеть максимальное количество достопримечательностей из файла задания, при угле обзора 90 градусов и радиусе до 2 км.

Для решения этой задачи мы рекомендуем воспользоваться одним из браузеров дополненной реальности: создать гео-слой с отображением на нем необходимых POI (points of interest, координат достопримечательностей, определенных вами в предыдущем шаге). Урок по созданию простого гео-слоя 
расположен по ссылке (\url{https://www.layar.com/documentation/browser/tutorials-tools/create-simple-geo-location-layer/}). Также вам может пригодиться мобильное приложение для подмены геопозиции вашего мобильного устройства.

В качестве ответа необходимо прикрепить долготу и широту через пробел.

Пример:  37.531276 55.702651

\solutionSection

Для решения этой задачи можно взять за основу алгоритм перебора из предыдущей задачи. Но, в этом случае, необходимо также проверять каждую найденную точку - взять от окружности, центром которой является эта точка, сектор в 90 градусов и посчитать количество достопримечательностей, попадающих в этот сектор. После чего развернуть сектор на 1 градус и повторить подсчет достопримечательностей уже в новом секторе. Ответом будет являться та точка, которая является центром окружности, содержащей сектор с наибольшим количеством достопримечательностей в нем. Cтоит отметить, что тут приведено не оптимальное решение, занимающее значительное процессорное время. Это решение можно оптимизировать, используя различные алгоритмы.

\includeSolutionIfExistsByPath{2nd_tour/ar/task_67}