\assignementTitle{}{14}{}

Шаг 1. Скачайте файл, содержащий описание зданий, являющихся достопримечательностями города Иркутска. По описанию отыщите эти здания в сети интернет и определите их географические координаты. Заполните таблицу с полями:

\begin{enumerate}
    \item Координаты lon, lat
    \item Описание
    \item Название
\end{enumerate}

Вам необходимо найти координаты ЦЕНТРА каждого здания, мы рекомендуем для этого воспользоваться сервисом 2ГИС (ВАЖНО!). Заполненная Вами таблица будет являться вспомогательным инструментом в следующих этапах задания.

Шаг 2. Правильное решение этого шага задачи оценивается в 14 баллов. Найдите координаты (lon, lat) 
центра окружности радиусом 2 км, находясь в котором пользователь увидит на экране смартфона максимальное 
количество достопримечательностей, описание которых дано в файле задания.  

Для решения этой задачи мы рекомендуем воспользоваться одним из браузеров дополненной реальности: 
создать гео-слой с отображением на нем необходимых POI (points of interest, координат достопримечательностей, 
определенных вами в предыдущем шаге). Урок по созданию простого гео-слоя расположен по ссылке (\url{https://www.layar.com/documentation/browser/tutorials-tools/create-simple-geo-location-layer/}). Также вам может пригодиться мобильное приложение для подмены геопозиции вашего мобильного устройства.

В качестве ответа необходимо прикрепить долготу и широту через пробел.

Пример ответа:  37.531276 55.702651

\solutionSection

Есть несколько способов решения этой задачи. Самый простой - это перебор. Для этого можно при помощи урока (\url{https://www.layar.com/documentation/browser/tutorials-tools/create-simple-geo-location-layer/}) создать гео-слой (минус этого способа в том, что для этого потребуется хостинг) или вручную нанести на карту все найденные точки (если не хотите разбираться с хостингом, php и БД). Если вы используете Layar, то на сайте можно протестировать слой. Во время тестирования можно перетаскивать иконку телефона и смотреть, сколько точек попадают в окружность.

\putImgWOCaption{15cm}{1}

Второй способ - написать программу на любом языке программирования (например, Python), которая найдет необходимую точку. Для решения этим способом понадобится функция для расчета расстояния (в метрах) между двумя точками, заданными в формате географических координат (lon, lat). Для этой функции понадобится формула Гаверсинуса:

$$d = 2 r arcsin \left( \sqrt{sin^2\left(\frac{\phi_2 - \phi_1}{2}\right) + cos(\phi_1)cos(\phi_2)sin^2\left(\frac{\lambda_2 - \lambda_1}{2}\right)}\right)$$

Функция расчета этой формулы может выглядеть вот так:

\putImgWOCaption{13cm}{2}

Далее нужна функция, принимающая на вход координаты центра и массив точек. Эта функция будет возвращать количество точек, находящихся внутри окружности в заданном центре, радиусом 2000 м.

После необходимо обойти все возможные точки, начиная с минимальных lat и lon, в каждом шаге добавляя к ним по 0.001 (обход необходимо делать, используя вложенный цикл - для каждого lat каждый lon). Внутри циклов необходимо проверять, сколько точек попадает в окружность с центром в полученных lat и lon. Таким образом можно найти координату центра окружности, в которой будет максимальное количество точек.

\includeSolutionIfExistsByPath{2nd_tour/ar/task_66}