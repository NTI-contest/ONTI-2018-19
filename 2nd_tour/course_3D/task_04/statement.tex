\assignementTitle{}{1}{}

По размерам, имеющимся на чертеже (все размеры даны в мм), определите диаметр d:

\putImgWOCaption{10cm}{1}

Вычислите d, введите только число (целое).

\solutionSection

В задачах этого типа вас посылают в путешествие вокруг детали, начиная с какого-то известного размера, карабкаясь по цепочке размеров, добавляя или вычитая очередной размер.  Препятствия, которые вам могут встретиться в пути:
\begin{itemize}
    \item Размер поставлен не с той стороны, где вы его ожидаете увидеть,
    \item Вы "залезаете" или "съезжаете" по склону с наклоном 45 или 30 градусов, и вместо высоты склона, вам дана его ширина. Вычислять хитрые синусы/косинусы не понадобится, но кое-что про свойства треугольников надо знать,
    \item Вместо вертикальной стенки, вам могут встретиться скругления с заданным радиусом,
    \item Вы можете забыть в нужный момент перейти от радиуса к диаметру или обратно.
\end{itemize}

\putImgWOCaption{7cm}{1}

Как видно из рисунка, в данной задаче ответом будет число 36.

\answerMath{40.}