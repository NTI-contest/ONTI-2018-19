\assignementTitle{}{1}{}

На этом чертеже неправильно указан один из размеров:

\putImgWOCaption{10cm}{1}

Найдите неверный размер и введите для него правильное значение, указав букву размера и верное значение в формате "<буква размера>=<размер>" (например A=24).
Все буквы прописные.


\solutionSection

Назовем избыточным размер, который можно вычислить из других размеров, имеющихся на чертеже. Если такая цепочка размеров присутствует, то избыточным можно считать, на выбор, любой из входящих в нее размеров.  При этом совсем необязательно, чтобы избыточный размер оказался неправильным.  На самом деле, при генерации чертежей в САПР  по 3D-модели наставить лишних размеров очень легко и просто, но нужно специально постараться (вручную изменяя размеры), чтобы один из этих размеров стал неправильным.

В данном примере есть только одна избыточная цепочка - это размеры $A$, $E$ и $F$.  Эти размеры должны удовлетворять соотношению $2 \cdot F = (A-E)$,  однако на чертеже это соотношение не выполняется: $ 2 \cdot F = 10$, а  $A-E=24-18=6$.  Значит, действительно, один из размеров в цепочке задан неверно.   Очень хочется сразу решить, что неправильным (и лишним) является размер F и его настоящее значение должно быть $F=(A-E)/2 = 3$ (мм).

Однако это пока только гипотеза,  с тем же успехом ошибка может быть в размере $A$ или в размере $E$. Попробовать выяснить, который из этих трех размеров ошибочен можно, только сопоставляя их с другими размерами чертежа. Поэтому проверим каждый из размеров $A$, $E$, $F$,  учитывая, что по условию задачи неправильный размер есть только один:
\begin{itemize}
    \item Предположим, что ошибочен размер $E$. В этом случае его значение должно быть: $E = A - 2 \cdot F = 14$ мм, что не согласуется с чертежом: размер $D=16$ должен быть меньше $E$. 
    \item Предположим, что ошибочен размер $A$. Но на чертеже мы видим, что с показанными на чертеже значениями $A=2 \cdot B$, и это подтверждается визуально равенством отмеченных отрезков. Если принять что $A = E + 2 \cdot F = 28$, то пропорции чертежа существенно изменились бы.
\end{itemize}

\putImgWOCaption{7cm}{2}

Таким образом, "неправильным" размером действительно является F и ответ, ожидаемый Stepik'ом: $F=3$.

\answerMath{F=3.}