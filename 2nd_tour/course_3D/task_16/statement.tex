\assignementTitle{}{5}{}

\putImgWOCaption{7cm}{1}

Ось Х 3D-принтера приводится в действие шаговым двигателем NEMA17  с вот такими параметрами:

\putImgWOCaption{13cm}{2}

На валу двигателя смонтирован шкив с 20 зубьями под зубчатый ремень типа GT2 (с шагом зуба 2 мм),  перемещающий каретку. В драйвере мотора используется микрошаг $1/16$.

Сколько шагов мотора необходимо выполнить, чтобы переместить каретку на $A$ мм?

\solutionSection

В данной задаче не требуется работать с силами и моментами вращения, но нужны базовые знания о принципах работы шагового двигателя и ременной передачи.  

Из таблички с данными о моторе нам понадобится только одна величина: угол шага $1.8^\circ$.  Это означает, что 1 оборот вал мотора совершит за $360/1.8 = 200$ шагов, или, наоборот, за 1 шаг вал поворачивается на $1/200$ оборота.  Кроме того, драйвер шагового двигателя умеет работать с микрошагами, искусственными промежуточными положениями вала внутри одного шага. Например, микрошаг $1/16$ означает, что для поворота вала двигателя на один шаг нужно подать 16 управляющих импульсов. 

Нам не потребуется диаметр шкива и число $\pi$, чтобы вычислять длину окружности. Достаточно знать, что шкив рассчитан на ремень с шагом зуба 2 мм и имеет 20 зубцов. Диаметр шкива подобран так, чтобы именно столько зубцов укладывалось по окружности. Это означает, что за один оборот шкива ремень продвинется на $S = 20 \cdot 2 = 40$ мм. 

Таким образом,  чтобы переместить каретку на $X$ мм,  на драйвер мотора необходимо подать $N$ шаговых импульсов: 
$$N = \frac{X}{40} \cdot 200 \cdot 16 = 80 \cdot X \: \text{(микрошагов)}$$
В задаче в Stepik вместо переменной X каждый раз появляется случайное число, и вам нужно ввести числовой ответ. 

\answerMath{$80\cdot a$}