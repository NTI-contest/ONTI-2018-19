\assignementTitle{Пластина-2}{4}{}

Постройте 3D-модель той же самой пластины (если Вы еще этого не сделали).   Дополните модель детали недостающими элементами,  чтобы пластину можно было надеть на круглый вал радиусом 3 мм.

\putImgWOCaption{9cm}{1}

Определите массу получившейся детали в граммах (с точностью не ниже 0.1 г) при условии, что она выполнена из высокопрочной низколегированной стали с плотностью 7.85 г/см$^3$.

\solutionSection

Просто строим модель по чертежу.  Можно сильно упростить себе жизнь, если не пытаться делать сложные эскизы, а строить элементы модели последовательно, вот так:

\putImgWOCaption{15cm}{2}

Здесь применяются только выдавливания и круговые массивы. Очень важно полностью определять каждый эскиз, задавая размеры и привязки для всех его элементов.  Обратите внимание на скрытую в задании "засаду":  указан \textit{радиус} отверстия, в то время как нам нужен его \textit{диаметр}!  Всегда внимательно читайте задание!

Когда модель готова, средствами вашего САПРа получаем ее массу, для заданного материала. Например, в Autodesk Inventor это будет выглядеть так:

\putImgWOCaption{8cm}{3}

\answerMath{$12.7 \pm 0.1$}