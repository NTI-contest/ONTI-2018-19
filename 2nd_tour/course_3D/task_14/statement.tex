\assignementTitle{}{5}{}

Вам надо спроектировать узел линейного перемещения для автомата, который раскладывает грузы по 8 ячейкам.  Ячейки выстроены в одну линию,  расстояние между соседними ячейками 40 мм.  В качестве привода применен сервомотор с шестерней и зубчатой рейкой, как показано на рисунке:

\putImgWOCaption{10cm}{1}

Сервопривод имеет вращательный момент (stall torque) 0.2 H $\cdot$ м  и диапазон поворота вала 270 градусов.  Какое максимальное усилие можно получить на зубчатой рейке, при условии, что узел линейного перемещения должен обеспечивать позиционирование схвата над центром любой из ячеек?   

Введите максимальное усилие, в Ньютонах (вводить только число),  с точностью не ниже 0.1 Н.

\solutionSection

Чем больше диаметр зубчатого колеса, тем больше ход зубчатой рейки, но меньше усилие. Нам нужно максимальное усилие, а значит, минимальный диаметр зубчатого колеса, при котором ход рейки оказывается достаточным для перемещения от центра первой до центра последней ячейки. Для N ячеек число "перегонов" между ячейками составит $N-1$,  а необходимый ход рейки $S = 7 \cdot 40 = 280$ мм:

\putImgWOCaption{7cm}{2}

Такой ход рейки должен составлять $3/4$  окружности, поскольку угол поворота вала серво ограничен $270^\circ$.  Следовательно, радиус зубчатого колеса будет равен:
$$R = \frac{4}{3} \cdot S / (2 \pi)= \frac{2 S}{3 \pi} \approx 59.42 \: \text{мм} \approx 0.0594 \: \text{м}$$
Вращательный момент будет составлять:
$$T = \frac{Ts}{R} = \frac{0.2}{0.0594} = 3.37 \: \text{H}$$
Обратите внимание, что в реальности зубчатое колесо не может иметь произвольный диаметр. При конструировании задается модуль зуба $M$, а диаметр делительной окружности зубчатого колеса получается умножением модуля зуба на число зубьев. Так, например, если было решено использовать зубчатую передачу с $M=1.0$ мм, допустимые зубчатые колеса будут иметь целочисленный диаметр (а радиусы с шагом 0.5) и нам придется округлить результат до 60~мм.  

В этом случае $T = 3.33$ H.  И тот и другой ответ находится в диапазоне допустимых погрешностей для данной задачи в Stepik.

\answerMath{$3.33 \pm 0.1$}