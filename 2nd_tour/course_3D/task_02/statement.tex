\assignementTitle{Геометрия сечений}{2}{}

Имеется пирамида, некоторые вершины и середины ребер которой обозначены буквами.  Построено несколько плоскостей сечения, каждая из которых проходит, по крайней мере, через 3 точки и обозначается по этим точкам (например, "плоскость BLD").  Эти плоскости создали следующие сечения. Для каждого сечения укажите соответствующую секущую плоскость (еще 3 сечения этой же пирамиды Вы увидите на следующем шаге):

\putImgWOCaption{9cm}{1}

\begin{multicols}{2}
    {
        \begin{enumerate}
            \item Плоскость FGK
            \item Плоскость GHK
            \item Плоскость ABD
        \end{enumerate}
    }
    {
        \begin{enumerate}
            \item[a.] Сечение 1
            \item[б.] Сечение 3
            \item[в.] Сечение 2  
        \end{enumerate}
    }
\end{multicols}

\answerMath{
    \begin{multicols}{2}
        {
            \begin{enumerate}
                \item Плоскость FGK
                \item Плоскость GHK
                \item Плоскость ABD
            \end{enumerate}
        }
        {
            \begin{enumerate}
                \item[б.] Сечение 3
                \item[a.] Сечение 1
                \item[в.] Сечение 2  
            \end{enumerate}
        }
    \end{multicols}
}
