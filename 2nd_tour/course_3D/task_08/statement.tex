\assignementTitle{Ключ}{6}{}

Смоделируйте ключ по приведенному чертежу. Значение параметра H взять равным 48 (мм), параметра $R_1$ — равным 152 (мм), 
значение параметра В определяется равенством B=178 – $R_1$ (мм).   Радиусы всех сопряжений по углам внутреннего выреза одинаковы (R7).

\putImgWOCaption{9cm}{1}

Определите объем детали в мм$^3$  (число, с точностью до целых).

\explanationSection

Деталь почти плоская, всего лишь с двумя элементами разной толщины. Поскольку вид сбоку не показан, может быть не сразу понятно, откуда брать информацию о толщинах. Она задана выносками "s3" и "s7", т.е. основная пластина детали имеет толщину 3 мм, а два круглых утолщения - 7 мм.

Основная сложность при моделировании этого ключа - правильно построить первый эскиз, применив все показанные на чертеже размеры и догадавшись обо всех использованных привязках и симметриях.  Критерий правильности построения - полная определенность эскиза. Например, в Autodesk Inventor с обычной цветовой схемой, все линии эскиз должны стать темно-синими.  По готовности модели, получаем объем. Например, в Autodesk Inventor:

\putImgWOCaption{11cm}{2}

\answerMath{$82476\pm 20$}