\assignementTitle{Кулачковый механизм}{5}{}

Скачайте архив с деталями с формате STEP. 

(\url{https://drive.google.com/file/d/1NaBgymX6Suge4ll5Ia8cB15TWLGF4S6A/view?usp=sharing}) 

Соберите из этих деталей кулачковый механизм, как показано на рисунке:

\putImgWOCaption{6cm}{1}

Введите амплитуду движения поршня при вращении рукоятки (т.е. расстояние между крайними положениями поршня), в мм, только число, с точностью не хуже 0.1.

\explanationSection

В задачах этого типа вам предлагается собрать механизм и вычислить диапазон движений, либо найти секретный код, определяемый по движению механизма.  Требуется умение применять в сборках динамические зависимости - зубчатые передачи, кулачковые механизмы.  Возможны задачи, в которых нужно изменить или добавить детали, чтобы механизм работал.  В этом случае спрашивается какой-либо ключевой параметр измененной/добавленной детали.
Полный процесс сборки кулачкового механизма (Вариант A) в САПР Autodesk Inventor показан в видеоуроке по адресу: \url{https://bit.ly/2Jih7gs}.

\answerMath{$10.14 \pm 0.1$}