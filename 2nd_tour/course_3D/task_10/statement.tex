\assignementTitle{}{2}{}

Соберите головоломку, используя приложенные файлы (в  формате STEP):

Деталь A

\url{https://drive.google.com/file/d/1JcXJwKiEi42S5JiICzoJdA31vLrjrEB-/view?usp=sharing}

Деталь B

\url{https://drive.google.com/file/d/1jTdQmNkdBCKJRBW_yd5K-0kuQv0URAFe/view?usp=sharing}
Деталь C

\url{https://drive.google.com/file/d/1prU86HaGt6WiuD1ZvGuyYfNLL1i98_tx/view?usp=sharing}

\putImgWOCaption{6cm}{1}

Какова площадь (в мм$^2$) треугольника, построенного по точкам ABC, указанным на рисунке?

\solutionSection

Создаем сборочную модель, загружаем в нее 3 детали в формате STEP, скачанные по ссылкам.  Крутим каждую деталь (в Autodesk Inventor - команда "свободный поворот"), пока она не займет положение, узнаваемое по верхнему рисунку.

\putImgWOCaption{6cm}{2}

Соединяем детали сборочными зависимостями по граням, ребрам или точкам.  Обычно требуется 3 зависимости для соединения каждых двух деталей.

\putImgWOCaption{6cm}{3}

Когда головоломка собрана, строим плоскость по 3-м точкам, показанным в задании.  Учитывая, что работа идет в сборке, дальнейшее будет различаться в разных САПР.  

Так, в Autodesk Inventor придется создать новую деталь в контексте сборки. После этого В плоскости строим эскиз.  а в эскизе - треугольник, построенный по проекциям этих 3х точек. Слегка выдавливаем контур, чтобы измерить его площадь.

\putTwoImg{5.5cm}{4}{8cm}{5}


\answerMath{$3316.625 \pm 1$}