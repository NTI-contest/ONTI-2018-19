\assignementTitle{}{2}{}

Каждое из показанных справа тел было получено из эскиза слева, вращением плоской фигуры вокруг одной из осей:  

\putImgWOCaption{9cm}{1}

\begin{multicols}{2}
    {
        \begin{enumerate}
            \item Тело A
            \item Тело B
            \item Тело C
            \item Тело D
            \item Тело E
            \item Тело F
            \item Не используется
        \end{enumerate}
    }
    {
        \begin{enumerate}
            \item[a.] Ось 5
            \item[б.] Ось 3
            \item[в.] Ось 6
            \item[г.] Ось 7
            \item[д.] Ось 2
            \item[е.] Ось 4
            \item[ж.] Ось 1      
        \end{enumerate}
    }
\end{multicols}

\solutionSection

По эскизу определяем характерные элементы, соответствие с которыми нужно искать в телах вращения. Для каждого из показанных тел определяем ось, если надо - проверяем несколько осей, исключая неподходящие.  Например, объект C имеет следующие характерные признаки:
\begin{enumerate}
    \item сглаженный край верхнего "кратера"{}, которому должен соответствовать скругленный угол в нижней части контура
\end{enumerate}

\putImgWOCaption{9cm}{2}

Эти признаки однозначно определяют \textbf{ось 4} как нужную нам ось. Аналогично определяются оси для каждого из остальных тел.


\answerMath{
    \begin{multicols}{2}
    {
        \begin{enumerate}
            \item Тело A
            \item Тело B
            \item Тело C
            \item Тело D
            \item Тело E
            \item Тело F
            \item Не используется
        \end{enumerate}
    }
    {
        \begin{enumerate}
            \item[б.] Ось 3
            \item[г.] Ось 7
            \item[е.] Ось 4
            \item[a.] Ось 5
            \item[д.] Ось 2
            \item[ж.] Ось 1  
            \item[в.] Ось 6     
        \end{enumerate}
    }
    
\end{multicols}}