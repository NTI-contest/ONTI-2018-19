\documentclass[a4paper,12pt,oneside]{book}

\usepackage{localfmt}
\usepackage{localshortcuts}

\def\withsolution{1}

\def\withgrading{1}

\begin{document}

\titlePageB{Интеллектуальные робототехнические системы}

\setcounter{tocdepth}{1}

\tableofcontents

%\newpage 
%Профиль ''Интеллектуальные робототехнические системы'' посвящен
%решению классических задач робототехники: автономной локализации
%мобильного наземного робототехнического устройства, планирования и
%построения маршрута мобильного робототехнического устройства или его
%частей, навигации, распознавания графической информации, межагентному 
%взаимодействию.
%Профиль включает в себя задачи по двум школьным предметам:
%\textbf{математика} и \textbf{информатика}.

\chapter{Введение}

Профиль ''Интеллектуальные робототехнические системы'', проводимый в
рамках Олимпиады НТИ 2018-2019 учебного года,  был посвящен изучению
алгоритмов компьютерного зрения и межагентрому взаимодейстивию.
Необходимость высокого уровня подготовки участников для решения данных
задач диктовала логику проведения отборочных  этапов: необходимо было
не только выявить школьников, заинтересованных в решении сложной
финальной задачи, но и дать необходимые знания для ее решения.    

Первый отборочный дистанционный этап (индивидуальный) определял общий
уровень подготовки школьников по предметам математика и информатика. 
Решая задачи по программированию, школьники должны были
продемонстрировать простейшие навыки составления и отладки программ,
обрабатывающих  массивы данных, и понимание таких тем, как
комбинаторика, операции со строками, вычислительная геометрия, теория
графов. Задачи по математике проверяли у участников  знания по
алгебре, комбинаторике, геометрии. Количество попыток сдачи решения
задач не ограничивалось. Таким образом, задачи первого этапа выявляли
наличие у участников знаний необходимых не только для решения задач
следующего этапа, но и финальной задачи.

Задачи второго отборочного этапа были разработаны таким образом, что
их было бы сложно решить индивидуальному участнику, поэтому школьники 
должны были объединиться в команды для успешного прохождения в финал. 
Задачи  требовали от участников погружения  в такие робототехнические
темы, как кинематика и навигация робототехнических устройств,
планирование маршрутов перемещения, построение карты с использованием 
экстероцептивных сенсоров, компьютерное зрение, обработка сетевой
информации. Для получения дополнительной информации, необходимой для
решения задач  второго этапа, командам были предложены образовательные
материалы, разработанные в Университете Иннополис.

Команды, прошедшие в финал профиля, приглашались на очный хакатон
(учебно-тренировочные сборы), на котором они могли познакомиться с
аппаратными  особенностями платформы, на которой предстояло решать
задачу финала. В течение сборов команды оттачивали умение управлять
наземным мобильным роботом,  изучали специфику работы с цифровыми
датчиками, реализовывали простейшие алгоритмы обработки изображения,
захваченного с камеры роботототехнического устройства, а также
организовывали передачу информации между несколькими роботами по
беспроводному каналу данных.

Таким образом при решении финальной задачи в очном заключительном
этапе участники могли использовать все знания и наработки, которые они
сделали во  время участия во втором туре и учебно-тренировочных
сборах. Несмотря на то, что задача финала была заранее неизвестна, ее
элементы были рассмотрены на  предварительных этапах, что значительно
упрощало реализацию алгоритма управления робототехническим устройством
в течение 3.5 соревновательных дней.  Дополнительной частью
заключительного этапа являлся индивидуальный тур, в ходе которого
участники решали задачи по математике и информатике.  Задачи по
математике покрывали следующие области математики: оптимизация,
комбинаторика, алгебра и геометрия. А темы задач по информатике
перекликались с классическими темами всероссийской олимпиады
школьников.

\part{Первый этап}
\newpage
Первый отборочный тур проводится индивидуально в сети Интернет,
работы оцениваются автоматически средствами системы
онлайн-тестирования.
Для каждой из параллелей (9 класс или 10-11
класс)
предлагается свой набор задач по математике, задачи по информатике общие
для всех участников. Решение задач по информатике предполагало
написание программ. Участники не были ограничены в выборе языка программирования для
решения задач. На решение
задач каждого предмета первого отборочного этапа участникам давалось 2
дня. У участников было три временных слота по 2 дня каждый, когда они
могли решать задачи по предмету. Решение каждой задачи дает
определенное количество баллов.

Участники получают оценку за решение задач
в совокупности по всем предметам данного профиля (математика и
информатика) --- суммарно от 0 до 200 баллов.

\chapter{Задачи первого этапа. Математика.}

\section{Первая попытка. Задачи 9 класса.}

\subimport{1st_tour/math/try_1/}{math_try_1_9.tex}

\section{Первая попытка. Задачи 10-11 класса.}

\subimport{1st_tour/math/try_1/}{math_try_1_10_11.tex}

\section{Вторая попытка. Задачи 9 класса.}

\subimport{1st_tour/math/try_2/}{math_try_2_9.tex}

\section{Вторая попытка. Задачи 10-11 класса.}

\subimport{1st_tour/math/try_2/}{math_try_2_10_11.tex}

\section{Третья попытка. Задачи 9 класса.}

\subimport{1st_tour/math/try_3/}{math_try_3_9.tex}

\section{Третья попытка. Задачи 10-11 класса.}

\subimport{1st_tour/math/try_3/}{math_try_3_10_11.tex}

\chapter{Задачи первого этапа. Углубленная информатика.}

\section{Первая попытка.}

\subimport{1st_tour/inf2/try_1/}{inf2_try_1.tex}

\section{Вторая попытка.}

\subimport{1st_tour/inf2/try_2/}{inf2_try_2.tex}

\section{Третья попытка.}

\subimport{1st_tour/inf2/try_3/}{inf2_try_3.tex}

\part{Второй этап}
\newpage

Второй отборочный этап проводится в командном формате в сети
интернет, работы оцениваются автоматически средствами системы
онлайн-тестирования.
Продолжительность второго этапа составляет 52 дня. Задачи
по информатике носят междисциплинарный характер и помогают отработать те
навыки, которые потребуются для решения командной задачи заключительного этапа.

Участники не были ограничены в выборе языка программирования для
решения задач.

Объем и сложность задач этого этапа подобраны таким образом,
чтобы решение всех задач одним человеком было маловероятно. Это
призвано обеспечить включение командной работы и распределения
обязанностей. Решение каждой задачи дает определенное количество
баллов. Баллы зачисляются в полном объеме за
правильное решение задачи.
Также существуют задачи, где допускается частичное решение.
В данном этапе можно получить суммарно от
0 до 138 баллов.

Задачи по программированию выкладывались
тремя партиями: в начале второго этапа, через три недели после начала
и через шесть недель после начала.
Команды могут выполнять задачи в любом порядке. Задачи допускают
неограниченное число попыток сдать решение.

\chapter{Задачи второго этапа}

\assignementTitle{Запуск всенаправленной тележки}{5}

Робот, моторы которого расположены под углом в $120^{\circ}$ друг к другу,
движется в некотором направлении, пока на моторы подается мощность.
Каждый мотор работает некоторое время: $t_1$,~$t_2$,~$t_3$, соответственно.
На валах моторов закреплены омниколёса (\url{https://en.wikipedia.org/wiki/Omni_wheel}).
С некоторой кинематической моделью робота можно познакомиться по ссылке:
\url{https://bharat-robotics.github.io/blog/kinematic-analysis-of-holonomic-robot/}.

Необходимо определить новые координаты центра робота, если в момент начала движения он находился в точке
$(0,0)$ и один из двигателей находился на оси $Y$, в направлении положительной части (см. рис. \ref{fig:sampleOnceStarted}).
Расположение моторов является постоянным и соответствует рисунку \ref{fig:sampleOnceStarted}.

Считать, что мощность на все моторы подаётся одновременно и достигается мгновенно.
Также считать что происходит движение без поворотов,
иначе говоря робот двигается только прямолинейно в любом направлении.
Колёса вращаются без проскальзывания.

Данные в тестах подобраны таким образом, что нет вариата движения, когда робот движется вокруг какой-то точки.


\putImgForRef{8cm}{2nd_tour/irs/01_Tribot/tribot_configuration}
{Расположение моторов робота относительно центра глобальной системы отсчета в начальный момент времени}{fig:sampleOnceStarted}


\inputfmtSection

Одна строчка, состоящая из 7ми чисел: $d,$~$w_1,$~$t_1,$~$w_2,$~$t_2,$~$w_3,$~$t_3$, разделёнными пробелами, где
\begin{itemize}
    \item $d$ --- диаметр колёс в мм ($30 \leq d \leq 100$);
    \item $w_1, w_2, w_3$ --- скорости вращения моторов в рад/с ($-2$ $\leq w_1, w_2, w_3 \leq 2$);
    \item $t_1, t_2, t_3$ --- время работы каждого мотора в с ($10 \leq t_1, t_2, t_3 \leq 1000$).
\end{itemize}

Диаметр колёс и время движения --- целые числа, скорости вращения --- вещественные.

\commentsSection

Дополнительные наборы входных данных доступны по \url{http://bit.ly/2RdizA3}{ссылке}.

\outputfmtSection


Одна строка, содержащая два целых числа  через пробел  --- координаты центра робота в мм,
где он закончил своё движение.
Допускается погрешность в 1 мм по каждой из координат.

\exampleSection

\sampleTitle{1}


\begin{myverbbox}[\small]{\vinput}
    35 1 15 0 0 -1 15
\end{myverbbox}
\begin{myverbbox}[\small]{\voutput}
    196.875 -113.666
\end{myverbbox}
\inputoutputTable

\sampleTitle{2}

\begin{myverbbox}[\small]{\vinput}
    40 1.4 100 -1.4 100 1.4 100
\end{myverbbox}
\begin{myverbbox}[\small]{\voutput}
    2800 2424.87
\end{myverbbox}
\inputoutputTable


\includeSolutionIfExistsByPath{2nd_tour_progr/01_Tribot/01_move_once_started/solution}
\assignementTitle{Управление всенаправленной тележкой}{10}

Робот, моторы которого расположены под углом в $120^{\circ}$ друг к другу,
движется в заданном направлении некоторое время $t$.
На валах моторов закреплены омниколёса (\url{https://en.wikipedia.org/wiki/Omni_wheel}).
С некоторой кинематической моделью робота можно познакомиться по ссылке:
\url{https://bharat-robotics.github.io/blog/kinematic-analysis-of-holonomic-robot/}.


Необходимо определить новые координаты центра робота,  если в момент начала движения он находился в точке
$(0,0)$ и один из двигателей находился на оси $Y$, в направлении положительной части (см. рис. \ref{fig:sampleMulSstarted}).
Расположение моторов является постоянным и соответствует рисунку \ref{fig:sampleMulSstarted}.

\putImgForRef{8cm}{2nd_tour/irs/01_Tribot/tribot_configuration}
{Расположение моторов робота относительно центра глобальной системы отсчета в начальный момент времени}{fig:sampleMulSstarted}

Считать, что мощность достигается мгновенно и может подаваться на все моторы одновременно.
Колёса вращаются без проскальзывания.
В случае когда на моторы ничего не подаётся, их скорость равна $0$.


\inputfmtSection
Первая строчка содержит четыре целых числа через пробел -- $d,$~$p,$~$t,$~$N$, где:

\begin{itemize}
    \item $d$ --- диаметр колёс в мм ($30 \leq d \leq 100$);
    \item $p$ --- длина оси~(осевой балки) от центра робота до колеса в мм ($ 50 \leq p \leq 125$);
    \item $t$ --- общее время работы моторов в с ($10 \leq t \leq 1000$);
    \item $N$ --- количество измерений ($1 \leq N \leq 1000$).
\end{itemize}


Далее идут $3$ строки --- для первого, второго и третьего моторов, соотвественно.

В каждой строке находится $N$ вещественных чисел через пробел ---  подаваемая на мотор скорость $w_i$, через равные промежутки
времени. ($-2$ рад/с $\leq w_i \leq 2$ рад/с)


\outputfmtSection

Одна строка, содержащая два целых числа  через пробел  --- координаты центра робота в мм,
где он закончил своё движение.
Допускается погрешность в 1 мм по каждой из координат.


\exampleSection

\sampleTitle{1}


\begin{myverbbox}[\small]{\vinput}
    30 50 12 4
    0 0 2 2
    -2 -2 0 0
    2 2 2 2
\end{myverbbox}
\begin{myverbbox}[\small]{\voutput}
    77.486 167.373
\end{myverbbox}
\inputoutputTable

\sampleTitle{2}

\begin{myverbbox}[\small]{\vinput}
    35 55 16 8
    -0.8 -0.8 -0.8 0.6 0.6 0.6 0 0
    -0.8 -0.8 -0.8 0 0 0 0 0
    0.8 0.8 0.8 0.6 0.6 0.6 0 0
\end{myverbbox}
\begin{myverbbox}[\small]{\voutput}
    1.504 130.544
\end{myverbbox}
\inputoutputTable


\includeSolutionIfExistsByPath{2nd_tour/irs/01_Tribot/02_move_multiple_starting/solution}

\assignementTitle{Определение размера препятствия}{15}

Робот, собранный по дифференциальной схеме, передвигается по полигону.
Он оснащён приёмником и дальномером.
Приёмник позволяет измерять расстояния до установленных на поле и находящихся в прямой видимости маяков.
Дальномер направлен влево по ходу движения робота и позволяет получать информацию
о находящихся препятствиях в данном направлении.
На полигоне находятся препятствие и $K$ маяков, координаты которых передаются через входной файл.
Гарантируется, что маяки не мешают роботу перемещаться и не попадают в поле зрения дальномера.
Пример полигона представлен на рис. \ref{fig:beaconFieldExample}.

Необходимо найти площадь препятствия, расположенного на полигоне.
Известно что, приёмник возвращает значения расстояний до препятствия в мм, считая от оси вращения робота,
находящейся в центре между колёсами. Дальномер расположен в том же месте, что и приёмник.

\putImgForRef{8cm}{2nd_tour_progr/02_Beacons/01_beacon_searching/beacons}
{Пример соревновательного полигона}{fig:beaconFieldExample}

\inputfmtSection

Первая строка содержит 3 целых числа: $K$ и $N$ и $dT$:
\begin{itemize}
    \item $K$ --- количество маяков ($10 \leq K \leq 100$);
    \item $N$ --- количество замеров ($10 \leq N \leq 10~000$);
    \item $dT$ --- пауза между измерениями в мс ($0 \leq dT \leq 10~000$).
\end{itemize}

Далее идёт $K$ строк. Каждая строка имеет следующую структуру:
$i$,~ $x_i$,~ $y_i$, в которой все числа вещественные и разделены пробелами, где:
\begin{itemize}
    \item $i$ --- номер маяка по порядку ($0 \leq i \leq 10~000$);
    \item $x_i$ --- координата $x$ $i$-того маяка ($-10~000 \leq x_i \leq 10~000$);
    \item $y_i$ --- координата $y$ $i$-того маяка ($-10~000 \leq y_i \leq 10~000$).
\end{itemize}

Далее идёт $N$ строк. Каждая строка содержит:
\begin{itemize}
    \item $d$ --- целое число, показание дальномера в мм ($0 \leq d \leq 1~000$), в случае если предмет
    находится вне поля видимости дальномера, то его значение будет равно $1~000$;
    \item $Value_{1} ~Value_2 ~\dots ~Value_k$ --- вещественные числа, показания, получаемые приёмником с каждого маяка.
    В случае, если маяк находится вне зоны видимости, то соответствующее значение будет равно $-1$.
\end{itemize}

Все числа указаны через пробел.

\outputfmtSection

Одна строка, содержащая одно целове число -- площадь препятствия в мм$^2$.
Допускается погрешность в $\pm 1$ м$^2$.

\exampleSection

Примеры входных данных и ответов к ним можно найти по \href{http://bit.ly/2Re5l9N}{данной} сслыке.




\includeSolutionIfExistsByPath{2nd_tour_progr/02_Beacons/01_beacon_searching/solution}
\assignementTitle{Определение цветных цилиндров}{15}


Робот, собранный по дифференциальный схеме и оснащенный дальномером, направленным прямо по ходу движения,
перемещается по робототехническому полигону.
На данном полигоне установлены цилиндры различного цвета. Робот также является цилиндром.

В процессе своего передвижения по полигону (см рис. \ref{fig:colourCylindersExample})
робототехническое устройство измерило расстояние до возникающих прямо препятствий
и их цвет в шестнадцатиричном формате вида $RRGGBB$, где $RR$ - 16тиричное число $R$ составляющей данного элемента
матрицы, $GG$ и $BB$ - 16тиричные числа $G$ и $B$ составляющих, соответственно.

\putImgForRef{8cm}{2nd_tour/irs/02_Beacons/02_colour_cylinders/beacons2}
{Пример соревновательного полигона}{fig:colourCylindersExample}

Необходимо определить между какими цилиндрами робот не сможет проехать, если известен маршрут робота и показания
датчика в процессе движения.
Размер цилиндров и робота: $20$ см в диаметре.
Дальномер возвращает расстояние до цилиндра, считая от оси вращения робота,
находящейся в центре между колёсами.
Также гарантируется, что робот измерил каждый цилиндр не менее трех раз, и разница между этими измерениями составляет
минимум $5$ дуговых градусов (при измерении по дуге цилиндра).

\inputfmtSection

Первая строка содержит 2 целых числа: $N$ и $dT$:
\begin{itemize}
    \item $N$ --- количество замеров ($10 \leq N \leq 10000$);
    \item $dT$ --- пауза между измерениями в мс ($0 \leq dT \leq 10000$).
\end{itemize}

Далее идёт $N$ строк. Каждая строка имеет следующую структуру:
$Vel_{linear}$, $Vel_{angular}$, $Distance$, $Color$, в которой все числа разделены пробелами, где:
\begin{itemize}
    \item $Vel_{linear}$ --- линейная скорость в см/с, с которой движется робот в данный момент ($-100 \leq Vel_{linear} \leq 100$);
    \item $Vel_{angular}$ --- угловая скорость в рад/мс, с которой движется робот в данный момент ($-10 \leq Vel_{angular} \leq 10$);
    \item $Distance$ --- показания датчика расстояния в см в диапазоне ``5--255'' см, если предмет вне зоны видимости, то
    будет выведено $255$;
    \item $Color$ --- цвет обнаруженного циллиндра в шестнадцатиричном формате вида $RRGGBB$.
    В случае если цилиндр не обнаружен, будет выведено ``000000''.
\end{itemize}

Значения скоростей и показания датчика являются вещественными числами. Моторы достигают скоростей мгновенно. Робот движется без проскальзывания.
Значения цвета --- целые числа.


\outputfmtSection

Одна строка, в которой указана пара цветов цилиндров в шестнадцатиричной записи через пробел, в порядке возрастания чисел.
В случае если таких пар несколько, то каждую пару следует выводить в отдельной строке.
Несколько строк следует выводить в порядке возрастания первых чисел, а в случае их равенства, в порядке возрастания вторых.


\exampleSection

Примеры входных данных и ответов к ним можно найти по сслыке \url{http://bit.ly/2UoC22Z}.



\includeSolutionIfExistsByPath{2nd_tour/irs/02_Beacons/02_colour_cylinders/solution}

\assignementTitle{Определения расстояния до препятствия}{5}

Камера статично установлена на высоте $h$ мм и отклонена на $\alpha^\circ$ от горизонта.
Она имеет разрешение $height\times width$ и угол обзора $\beta^\circ$.
Поясняющая картинка представлена на рисунке \ref{fig:cameraExample}.

\putImgForRef{8cm}{2nd_tour/irs/03_Camera/01_get_distance/camera_example}
{Схема установки камеры}{fig:cameraExample}


Необходимо найти расстояние до предмета, установленного на полигоне.
Гарантируется, что предмет находится в поле зрения камеры, является однотонным.
Его минимальная площадь на изображении равна $s \%$.
Также известно, что помимо требуемого предмета
в камеру попадает поверхность полигона и(или) борта.
Поверхность полигона не однородная, а содержащая различные цвета в хаотическом порядке.

Пример изображения представлен на рис.  \ref{fig:getDistanceImageExample},

\putImgForRef{8cm}{2nd_tour/irs/03_Camera/01_get_distance/field_example}
{Пример изображения с камеры}{fig:getDistanceImageExample}


\inputfmtSection

Первая строка входного файла содержит 6 чисел: $h$,~ $\alpha$,~ $\beta$,~ $height$,~ $width$,~$s$:
\begin{itemize}
    \item $h$ --- высота установки камеры в мм ($10 \leq h \leq 100$);
    \item $\alpha$ --- угол в градусах,  под которым расположена камера относительно горизонта ($0^\circ \leq \alpha \leq 45^\circ$);
    \item $\beta$ --- угол обзора камеры в градусах ($30^\circ \leq \alpha \leq 180^\circ$);
    \item $height$ --- разрешение изображения по высоте ($10 \leq height \leq 2000$);
    \item $width$ --- разрешение изображения по ширине ($10 \leq width \leq 2000$);
    \item $s$ --- минимальная площадь предмета в $\%$ от всего изображения, которое измеряется в пикселях ($1 \leq s \leq 100$).
\end{itemize}

На следующих $height$ строках расположено изображение размером $height\times width$ в виде шестнадцатиричных чисел.
Данные числа имеют следующий вид: $RRGGBB$, где $RR$ - 16тиричное число $R$ составляющей данного элемента
матрицы, $GG$ и $BB$ - 16тиричные числа $G$ и $B$ составляющих, соответственно.

Все числа являются целыми.


\outputfmtSection

Одна строка, в которой указано число --- расстояние до предмета в мм. Допускается погрешность в 10 мм.


\exampleSection

Примеры входных данных и ответов к ним можно найти по сслыке \url{http://bit.ly/2FCbgAU}.


\includeSolutionIfExistsByPath{2nd_tour/irs/03_Camera/01_get_distance/solution}
\assignementTitle{Определение вектора перемещения робота в заданном виде}{20}


Робот, собранный по дифференциальной схеме, оборудован камерой.
Камера установлена так, что смотрит вниз перпендикулярно поверхности, по которой движется робот.
Известны характеристики камеры такие, как разрешение ($height\times width$),
угол обзора $\alpha$, высота расположения над плоскостью движения.
Верхний край кадра находится дальше от робота, чем нижний край кадра.

Есть последовательность кадров, сделанных камерой через равные промежутки времени.
Всего кадров было сделано $N$.

\putImg{8cm}{2nd_tour/irs/03_Camera/02_find_velocity/robot_example}
{Вид робота. Поясняющая картинка}


Зная, что робот движется с постоянной линейной $(v)$ и постоянной угловой $(\omega)$ скоростью,
необходимо определить перемещение по $X$ и по $Y$ в мм.

\inputfmtSection

Входной файл состоит из нескольких строк.

Первая строка содержит 5 целых чисел: $N$,~ $h$,~ $\alpha$,~ $height$,~ $width$:
\begin{itemize}
    \item $N$ --- количество замеров камерой ($2 \leq N \leq 16$);
    \item $h$ --- высота установки камеры в мм ($10 \leq h \leq 2000$);
    \item $\alpha$ --- угол обзора камеры в градусах ($15^\circ \leq \alpha \leq 175^\circ$);
    \item $height$ --- разрешение изображения по высоте ($5 \leq height \leq 16$);
    \item $width$ --- разрешение изображения по ширине ($5 \leq width \leq 16$).
\end{itemize}

Далее расположено $N$ строк, на каждой из которых расположено изображение размером
$height\times width$ в виде шестнадцатиричных чисел слева направо, сверху вниз.
Данные числа имеют следующий вид: $RRGGBB$, где $RR$ - 16тиричное число $R$ составляющей данного элемента
матрицы, $GG$ и $BB$ --- 16тиричные числа $G$ и $B$ составляющих соответственно.

Все числа являются целыми.


\outputfmtSection

Вывести пару вещественных чисел с точностью до целых --- перемещение по $X$ и по $Y$  в мм в данном порядке через пробел.


\exampleSection

Примеры входных данных и ответов к ним можно найти по сслыке \url{http://bit.ly/2EP1c5g}.





\includeSolutionIfExistsByPath{2nd_tour/irs/03_Camera/02_find_velocity/solution}
\assignementTitle{Распознавание ARTag маркера}{10}

Для определения своего местоположения квадрокоптер использует камеру, снимающую поверхность,
над которой перемещается робот. Изображение с камеры приходит в управляющую
программу в виде набора $height \times width$ точек, где каждая точка закодирована в RGB-формате.


\putImgForRef{8cm}{2nd_tour/irs/03_Camera/03_ArTag_detecting/arTag_description}
{Нумерация битов в маркера}{fig:arTagDescription}

На поверхность нанесены ARTag маркеры (\url{https://inside.mines.edu/~whoff/courses/EENG512/lectures/other/ARTag.pdf}).
Элементы маркера, расположенные по его границе - всегда черные. Четыре элемента,
находящиеся в углах внутреннего $3\times 3$ квадрата определяют ориентацию маркера таким
образом, что только элемент в нижнем правом углу квадрата - белый. Центральный элемент
квадрата используется для проверки четности (parity check): если количество единичных бит в двоичной
записи закодированного в маркере числа нечетное, то он черный. Оставшиеся $4$ элемента маркера
кодируют число по следующему правилу: если элемент черный, то в он обозначает $1$,
если белый, то $0$ при этом самый первый элемент --- старший бит закодированного числа.
Элементы пронумерованы сверху вниз, слева направо (см. рис. \ref{fig:arTagDescription}).

Например, на маркере с рис. \ref{fig:arTagExample} закодировано число $0011_2$, что
эквивалентно $3_{10}$.

\putImgForRef{8cm}{2nd_tour/irs/03_Camera/03_ArTag_detecting/arTag_example}
{Маркер с закодированным значением - $0011_2$}{fig:arTagExample}

Поскольку квадрокоптер не перемещается постоянно параллельно поверхности, то изображения
ARTag маркера, получаемые с камеры, получаются в виде неправильного выпуклого
четырехугольника, а непостоянные условия освещенности изменяют фокус, тон и добавляют блики на изображение.
Также квадрокоптеру не всегда удается полностью захватить маркер, и необходимо обрабатывать данные с нескольких снимков.


Поскольку направление запуска квадрокоптера заранее неизвестно, то ориентация маркеров
заранее неизвестна, но его изображение таково, что оно по каждой из осей $X, Y, Z$
относительно оптической оси камеры не превышает $25$ градусов.

Напишите программу для определения закодированного в маркере числа.

\inputfmtSection

Входной файл состоит из нескольких строк.

Первая строка содержит 3 целых числа: $N$,~ $height$,~ $width$:
\begin{itemize}
    \item $N$ --- количество замеров камерой ($1 \leq N \leq 100$);
    \item $height$ --- разрешение изображения по высоте ($10 \leq height \leq 1000$);
    \item $width$ --- разрешение изображения по ширине ($10 \leq width \leq 1000$).
\end{itemize}

Далее расположено $N$ строк на каждой расположено изображение размером
$height\times width$ ввиде шестнадцатиричных чисел слева направо сверху вниз.
Данные числа имеют следующий вид: $RRGGBB$, где $RR$ - 16тиричное число $R$ составляющей данного элемента
матрицы, $GG$ и $BB$ --- эт0 16ти-ричные числа $G$ и $B$ составляющих соответственно.

Все числа являются целыми.

\outputfmtSection
Вывести одно целое число в десятичной системе счисления, которое было закодированное на маркере.
В случае если невозможно определить число, следует вывести $-1$.


\exampleSection


Примеры входных данных и ответов к ним можно найти по сслыке \url{http://bit.ly/2DHww5w}.




\includeSolutionIfExistsByPath{2nd_tour/irs/03_Camera/03_ArTag_detecting/solution}

\assignementTitle{Определение лабиринта}{10}

Роботы в количестве $N$, собранные по дифференциальной схеме, оснащёны тремя дальномерами, направленными налево, прямо и
направо относительно направления движения робота.
Роботы движутся одновременно  из заранее известных секторов по неизвестному лабиринту, размером $K \times M$ секторов.

Между секторами могут быть стены, нахождение препятствия, внутри сектора недопустимо.
Отсчёт координат секторов начинается с $(0;0)$ в левом верхнем углу, $X$  возрастает вправо, $Y$ --вниз.

В процессе своих перемещений они получают значения с дальномеров, позволяющих узнать структуру лабиринта.
В результате полученных данных необходимо понять в какие сектора полигона может попасть первый робот, без
учёта нахождения на поле других роботов. В случае если данное значение невозможно определить,
следует вывести количество секторов, посещённых в процессе движения данным роботом.

\putImg{8cm}{2nd_tour/irs/04_Multiple_agents/01_maze_structure_detecting/field_example}
{Пример соревновательного полигона}

\putImg{5cm}{2nd_tour/irs/04_Multiple_agents/01_maze_structure_detecting/sensors_location}
{Расположение датчиков}


\inputfmtSection

Первая строка содержит 4 целых числа: $N$, $K$, $M$ и $i$:
\begin{itemize}
    \item $N$ --- количество роботов на поле $(2 \leq N \leq 5)$;
    \item $K$ --- ширина (по оси $X$) лабиринта в секторах $(4 \leq K \leq 20)$;
    \item $M$ --- длина (по оси $Y$) лабиринта в секторах $(4 \leq M \leq 20)$;
    \item $i$ --- количество показаний каждого робота.
\end{itemize}

Далее идет $N$  строк, содержащие координаты старта каждого робота, а также направление робота при старте,
т.е. одна строка имеет следующую структуру: $x_s$, $y_s$, $dir$, где:
\begin{itemize}
    \item $x_s$ --- координаты старта данного робота по оси $X$;
    \item $y_s$ --- координаты старта данного робота по оси $Y$;
    \item $dir$ --- направление робота при старте:
    \begin{itemize}
        \item $U$ --- робот направлен вверх;
        \item $L$ --- робот направлен влево;
        \item $D$ --- робот направлен вниз;
        \item $R$ --- робот направлен вправо;
    \end{itemize}
\end{itemize}

Далее идет $N$ блоков по $i$ строк.
На каждой строке находится действие робота и показания всех датчиков после этого действия через пробел:
$Movement, ~ D_l,~ D_f, ~D_r$:
\begin{itemize}
    \item $Movement$ --- выполненное действие робота:
    \begin{itemize}
        \item $F$ --- проезд робота в следующий по ходу движения сектор;
        \item $L$ --- поворот робота в данном секторе налево;
        \item $R$ --- поворот робота в данном секторе направо;
    \end{itemize}
    \item $D_l$ --- показания датчика расстояния, направленного влево:
    \begin{itemize}
        \item $1$ --- присутствует препятствие;
        \item $0$ --- отсутствует препятствие;
    \end{itemize}
    \item $D_f$ --- показания датчика расстояния, направленного вперёд:
    \begin{itemize}
        \item $1$ --- присутствует препятствие;
        \item $0$ --- отсутствует препятствие;
    \end{itemize}
    \item $D_r$ --- показания датчика расстояния, направленного вправо:
    \begin{itemize}
        \item $1$ --- присутствует препятствие;
        \item $0$ --- отсутствует препятствие.
    \end{itemize}
\end{itemize}



\outputfmtSection

Одно число - количество секторов, которое возможно посетить. Если невозможно определить, то вывести число
секторов посещённых первым роботом.


\exampleSection

Примеры входных данных и ответов к ним можно найти по сслыке \url{http://bit.ly/2LK5FrH}.





\includeSolutionIfExistsByPath{2nd_tour/irs/04_Multiple_agents/01_maze_structure_detecting/solution}
\assignementTitle{Планирование движения в лабиринте}{15}

Роботы в количестве $N$, собранные по дифференциальной схеме, движутся по заранее известному лабиринту и
представленному на рис. \ref{fig:mazeStructureExample}.

Необходимо доехать из начального в конечный сектор каждым роботом, если они
движутся одновременно и параллельно и имеют следующие команды:
\begin{itemize}
    \item $F$ --- проезд робота в следующий по ходу движения сектор;
    \item $L$ --- поворот робота в данном секторе налево и проезд в следующий по ходу движения сектор;
    \item $R$ --- поворот робота в данном секторе направо и проезд в следующий по ходу движения сектор.
\end{itemize}
Считать что роботы поворачиваются мгновенно, а после одновременно начинают движение вперёд с 
одинаковой скоростью.Робот является цилиндром и занимает 2/3 площади клетки. Каждое перемещение он начинает и 
заканчивает в центре клетки.

Также гарантируется, что данных команд будет достаточно для выполнения задачи. При движении столкновение не 
допускается и перемещение необходимо осуществлять так, чтобы роботы получили наименьшее число команд. Иначе 
говоря, необходимо, чтобы сумма длин строк, содержащих команды передаваемые на робота в хронологическом порядке, 
без пробелов и запятых, была минимально возможной.

\putImgForRef{8cm}{2nd_tour/irs/04_Multiple_agents/02_maze_movement/field_example}
{Структура лабиринта для решения задачи}{fig:mazeStructureExample}

\inputfmtSection

Первая строка содержит 1 целое число: $N$ --- количество роботов на поле \linebreak $(1 \leq N \leq 3)$.

Далее идет $N$ строк, содержащие координаты старта и финиша каждого робота, а также
направление робота при старте, т.е. одна строка имеет следующую структуру:
$x_s$,~ $y_s$,~$dir$,~ $x_f$,~ $y_f$, где:
\begin{itemize}
    \item $x_s$ --- координаты старта данного робота по оси $X$;
    \item $y_s$ --- координаты старта данного робота по оси $Y$;
    \item $dir$ --- направление робота при старте:
        \begin{itemize}
            \item $U$ -- робот направлен вверх;
            \item $L$ -- робот направлен влево;
            \item $D$ -- робот направлен вниз;
            \item $R$ -- робот направлен вправо;
        \end{itemize}
    \item $x_f$ --- координаты финиша данного робота по оси $X$;
    \item $y_f$ --- координаты финиша данного робота по оси $Y$;
\end{itemize}

Все данные указаны через пробел, числа являются целыми.


\outputfmtSection

$N$ строк, каждая строка содержит команды, передаваемые на данного робота в хронологическом порядке
(самая первая команда расположена в начале строки), без пробелов и запятых.


\commentsSection

Дополнительные наборы входных данных доступны по ссылке \url{http://bit.ly/2KtN4z9}.

\exampleSection

\sampleTitle{1}

\begin{myverbbox}[\small]{\vinput}
    2
    6 7 U 0 2
    7 7 U 1 2
\end{myverbbox}
\begin{myverbbox}[\small]{\voutput}
    FLLRFFFFRFFFF
    FLRFFFLLRFFRL
\end{myverbbox}
\inputoutputTable

\sampleTitle{2}

\begin{myverbbox}[\small]{\vinput}
    2
    0 0 D 2 5
    0 1 R 2 1
\end{myverbbox}
\begin{myverbbox}[\small]{\voutput}
    FFFFFFLFL
    RFFFFFLFFLFFLRFRLFLL
\end{myverbbox}
\inputoutputTable




\includeSolutionIfExistsByPath{2nd_tour/irs/04_Multiple_agents/02_maze_movement/solution}

\assignementTitle{Определение пересекающихся траекторий \\}{10}

Имеется набор байт --- трафик, собранный на концентраторе, во время общения в локальной сети нескольких  устройств,
включая робототехнические устройства.

Известно, что робототехнические устройства общались по протоколу, построенному поверх UDP (\url{https://ru.wikipedia.org/wiki/UDP}) и
реализованному следующим образом: три целых числа -- $t_i$,~ $X_i$, ~$Y_i$ -- каждое из которых занимает $4$ байта,
где $t_i$ --- время, в которое были сняты данные показатели координат, а $X_i$ ~и~ $Y_i$ --- координаты местоположения
робототехнической тележки в данный момент времени.
При записи чисел использовалась нотация BigEndian (\url{https://en.wikipedia.org/wiki/Endianness}).

Необходимо определить IP-адреса (\url{https://ru.wikipedia.org/wiki/IPv4}) \linebreak устройств, чьи траектории пересекались.


Считать, что между изменениями робототехнические тележки перемещались прямо.
Гарантируется, что через данный концентратор проходит лишь необходимый трафик, т.е. отсутствуют пакеты,
не относящиеся к данной задаче.

\inputfmtSection


Первая строка содержит 1 целое число: $N$ --- количество переданых пакетов через концентратор  $(4 \leq N \leq 10^3)$.

Далее идут $N$ строк, каждая из которых содержит один пакет, переданный через концентратор в побитовом формате, который содержит
$t_i$,~ $X_i$, ~$Y_i$, где:
\begin{itemize}
    \item $t_i$ --- время в мс, в которое были сняты данные показатели координат~-- $4$~байта $(0 \leq t_i < 2^{32})$;
    \item $X_i$ --- координаты по оси $X$~ местоположения робототехнической тележки в данный момент времени --- $4$ байтa $(0 \leq X_i < 2^{32})$;
    \item $Y_i$ --- координаты по оси $Y$~ местоположения робототехнической тележки в данный момент времени --- $4$ байтa $(0 \leq Y_i < 2^{32})$.
\end{itemize}

\outputfmtSection

Необходимо вывести два IP-адресса в десятичном формате через пробел в порядке их возрастания -- адреса устройств,
чьи траектории пересекались.

В случае если пересекались несколько пар роботов, эти пары следуют в порядке первых пересечений траекторий
робототехнических устройств.

В случае если таких пересечений нет, следует вывести $-1$.

\exampleSection

Примеры входных данных и ответов к ним можно найти по сслыке \url{http://bit.ly/2QkcNzv}.


\includeSolutionIfExistsByPath{2nd_tour_progr/05_Communication/01_Found_intersects/solution}

\assignementTitle{Определение собственных координат}{10}

Робот собранный по дифференциальной схеме оснащен двумя инфракрасными датчиками расстояния. Один из датчиков расстояния установлен так, что показывает расстояние до препятствий прямо по курсу робота. Второй датчик расстояния направлен влево.

\putImgForRef{8cm}{2nd_tour/irs/06_in_simulatior/04_discrete_odometry/maze-002}
{Пример начального расположения робота}{fig:discreteOdometryStartPos}

Для решения задачи робот запускается на поле, состоящим из 8х8 квадратных секторов. На поле между некоторыми секторами установлены препятствия для ограничения перемещения робота.

Роботу необходимо проехать некоторое количество секторов, заданное через входной файл \textit{input.txt}, по правилу ''левой руки'', после остановиться и вывести на экран относительные координаты, т.е. свои координаты относительно точки старта, в формате $(X, Y)$, где начало отсчета - точка старта, направление оси $Y$ совпадает с первоначальным направлением движения, ось $X$ направлена вправо перпендикулярно оси $Y$.

\begin{center}
\noindent
\textbf{Конфигурация робота}
\end{center}

\twoitems{Подключение моторов}{Левый мотор - порт M3;}{Правый мотор - порт M4.}
\twoitems{Подключение датчиков}{Датчик расстояния, направленный вперед - порт A1}
{Датчик расстояния, направленный влево - порт A2}

\inputfmtSection

Входной файл содержит только одну строчку. В строке - целое число $N$ \linebreak $(5 \leq N \leq 40)$, определяющее количество секторов, которое необходимо проехать. Сектор старта не учитывается в подсчёте.

\begin{center}
\noindent
\textbf{Ограничения}
\end{center}

Робот не должен выполнять задание дольше 3 минут.

\exampleSection

Для лабиринта, представленного на рис. \ref{fig:discreteOdometryStartPos}, и при числе $7$ во входных данных, робот должен остановиться в соответствии с рис. \ref{fig:discreteOdometryFinishPos} и вывести на экран $(4,1)$.

\putImgForRef{8cm}{2nd_tour/irs/06_in_simulatior/04_discrete_odometry/maze-002-finish}
{Расположение робота, стартовавшего из позиции на рис. \ref{fig:discreteOdometryStartPos} и проехавшего $7$ секторов}{fig:discreteOdometryFinishPos}

\includeSolutionIfExistsByPath{2nd_tour/irs/06_in_simulatior/04_discrete_odometry/solution}

\part{Заключительный этап}

\clearpage
\chapter{Предметный тур}

На индивидуальное решение задач дается по 2 часа на один предмет.
Для каждой из параллелей (9 класс или 10-11 класс) предлагается свой
набор задач по математике, задачи по информатике - общие для всех
участников.

Решение каждой задачи по математике дает определенное количество
баллов (см. критерии оценки). При этом некоторые задачи делятся на
подзадачи. За каждую подзадачу можно получить от 0 до указанного
количества баллов.

Решение задач по информатике предполагало написание программ.
Ограничения по используемым языкам программирования не было.
Проверочные тесты для каждой задачи по информатике делились на
несколько групп. Прохождение всех тестов в группе тестов дает
определенное количество баллов за решение задачи.

Участники получают оценку за решение задач в совокупности по
всем предметам данного профиля (математика и информатика) ---
суммарно от 0 до 200 баллов.

\begin{itemize}
    \item {\bf Математика 9 класс} количество набранных баллов
    (от 0 до 100);
    \item {\bf Математика 10-11 класс} количество набранных баллов
    (от 0 до 100);
    \item {\bf Информатика} количество набранных баллов (от 0 до
    300) делится на коэффициент 3.
\end{itemize}

\section{Математика. 9 класс}

\subimport{final/subject_tour/math0903_irs_9/task_01/}{statement}
\subimport{final/subject_tour/math0903_irs_9/task_01/}{solution}

\subimport{final/subject_tour/math0903_irs_9/task_02/}{statement}
\subimport{final/subject_tour/math0903_irs_9/task_02/}{solution}

\subimport{final/subject_tour/math0903_irs_9/task_03/}{statement}
\subimport{final/subject_tour/math0903_irs_9/task_03/}{solution}


\section{Математика. 10-11 класс}

\subimport{final/subject_tour/math0903_irs_10/task_01/}{statement}
\subimport{final/subject_tour/math0903_irs_10/task_01/}{solution}

\subimport{final/subject_tour/math0903_irs_10/task_02/}{statement}
\subimport{final/subject_tour/math0903_irs_10/task_02/}{solution}

\subimport{final/subject_tour/math0903_irs_10/task_03/}{statement}
\subimport{final/subject_tour/math0903_irs_10/task_03/}{solution}


\section{Информатика}

\subimport{final/subject_tour/inf0903_irs/task_01/}{statement.tex}
\subimport{final/subject_tour/inf0903_irs/task_02/}{statement.tex}
\subimport{final/subject_tour/inf0903_irs/task_03/}{statement.tex}


\chapter{Командный тур}

\subimport{final/command_tour/irs/}{content.tex}

\part{Критерии}

\chapter{Критерии определения победителей и призеров заключительного этапа}
 
\section{Первый отборочный этап}
 
В первом отборочном этапе участники решали задачи по двум предметам - математика и информатика, в каждом предмете максимально можно было набрать 100 баллов. Для того, чтобы пройти во второй этап участники должны были набрать в сумме по обоим предметам:
\begin{itemize}
    \item 9 классы и младше - не менее 80 баллов.
    \item 10-11 класс - не менее 100 баллов.
\end{itemize}

\section{Второй отборочный этап}

Количество баллов, набранных при решении всех задач, суммируется. Призерам второго отборочного этапа было необходимо набрать не менее 40 баллов.

Победители второго отборочного этапа являются финалистами олимпиады и должны были набрать 46 баллов и выше независимо от уровня.

\section{Заключительный этап}

В заключительном этапе олимпиады баллы участника складываются из двух частей: 
\begin{enumerate}
    \item[1 -] участник получает баллы за индивидуальное решение задач по предметам (математика и информатика)
    \item[2 -] участник получает баллы за командное решение практической задачи.
\end{enumerate} 

Итоговый личный балл участника рассчитывается по формуле:
$$S = Si + Sm + St, \: \text{где:}$$
 
$Si$ - сумма баллов по информатике, набранная в рамках индивидуальной части заключительного этапа (максимум 300 баллов), нормированная на 100;

$Sm$ - сумма баллов по математике, набранная в рамках индивидуальной части заключительного этапа (максимум 100 баллов); 

$St$ - количество баллов, набранное в рамках командной части заключительного этапа (максимум 400 баллов).

Критерий определения победителей и призеров:
\begin{center}
    \begin{tabular}{|l|l|}
        \hline
        Категория&Количество баллов\\
        \hline
        Победители&160 баллов и выше\\
        \hline
        Призеры&От 140 до 159 баллов\\
        \hline
    \end{tabular}
\end{center}

\end{document}