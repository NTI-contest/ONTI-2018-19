\documentclass[a4paper,12pt,oneside]{book}

\usepackage{localfmt}
\usepackage{localshortcuts}

\def\withsolution{1}

\def\withgrading{1}

\begin{document}

\titlePage{Интеллектуальные робототехнические системы}

\setcounter{tocdepth}{1}

\tableofcontents

\part{Первый этап}

\chapter{Задачи первого этапа. Математика.}

\section{Первая попытка. Задачи 9 класса.}

\subimport{1st_tour/math/try_1/}{math_try_1_9.tex}

\section{Первая попытка. Задачи 10-11 класса.}

\subimport{1st_tour/math/try_1/}{math_try_1_10_11.tex}

\section{Вторая попытка. Задачи 9 класса.}

\subimport{1st_tour/math/try_2/}{math_try_2_9.tex}

\section{Вторая попытка. Задачи 10-11 класса.}

\subimport{1st_tour/math/try_2/}{math_try_2_10_11.tex}

\section{Третья попытка. Задачи 9 класса.}

\subimport{1st_tour/math/try_3/}{math_try_3_9.tex}

\section{Третья попытка. Задачи 10-11 класса.}

\subimport{1st_tour/math/try_3/}{math_try_3_10_11.tex}

\chapter{Задачи первого этапа. Углубленная информатика.}

\section{Первая попытка.}

\subimport{1st_tour/inf2/try_1/}{inf2_try_1.tex}

\section{Вторая попытка.}

\subimport{1st_tour/inf2/try_2/}{inf2_try_2.tex}

\section{Третья попытка.}

\subimport{1st_tour/inf2/try_3/}{inf2_try_3.tex}

\part{Второй этап}
\clearpage
\chapter{Задачи второго этапа}

\section{Блок заданий 1}

\subimport{2nd_tour/irs/task_01/}{statement}
\subimport{2nd_tour/irs/task_02/}{statement}
\subimport{2nd_tour/irs/task_03/}{statement}
\subimport{2nd_tour/irs/task_04/}{statement}
\subimport{2nd_tour/irs/task_05/}{statement}

\section{Блок заданий 2}

\subimport{2nd_tour/irs/task_06/}{statement}
\subimport{2nd_tour/irs/task_07/}{statement}
\subimport{2nd_tour/irs/task_08/}{statement}
\subimport{2nd_tour/irs/task_09/}{statement}

\section{Блок заданий 3}

\subimport{2nd_tour/irs/task_10/}{statement}
\subimport{2nd_tour/irs/task_11/}{statement}
\subimport{2nd_tour/irs/task_12/}{statement}
\subimport{2nd_tour/irs/task_13/}{statement}

\part{Заключительный этап}

\clearpage
\chapter{Предметный тур}

На индивидуальное решение задач дается по 2 часа на один предмет.
Для каждой из параллелей (9 класс или 10-11 класс) предлагается свой
набор задач по математике, задачи по информатике - общие для всех
участников.

Решение каждой задачи по математике дает определенное количество
баллов (см. критерии оценки). При этом некоторые задачи делятся на
подзадачи. За каждую подзадачу можно получить от 0 до указанного
количества баллов.

Решение задач по информатике предполагало написание программ.
Ограничения по используемым языкам программирования не было.
Проверочные тесты для каждой задачи по информатике делились на
несколько групп. Прохождение всех тестов в группе тестов дает
определенное количество баллов за решение задачи.

Участники получают оценку за решение задач в совокупности по
всем предметам данного профиля (математика и информатика) ---
суммарно от 0 до 200 баллов.

\begin{itemize}
    \item {\bf Математика 9 класс} количество набранных баллов
    (от 0 до 100);
    \item {\bf Математика 10-11 класс} количество набранных баллов
    (от 0 до 100);
    \item {\bf Информатика} количество набранных баллов (от 0 до
    300) делится на коэффициент 3.
\end{itemize}

\section{Математика. 9 класс}

\subimport{final/subject_tour/math0903_irs_9/task_01/}{statement}
\subimport{final/subject_tour/math0903_irs_9/task_01/}{solution}

\subimport{final/subject_tour/math0903_irs_9/task_02/}{statement}
\subimport{final/subject_tour/math0903_irs_9/task_02/}{solution}

\subimport{final/subject_tour/math0903_irs_9/task_03/}{statement}
\subimport{final/subject_tour/math0903_irs_9/task_03/}{solution}


\section{Математика. 10-11 класс}

\subimport{final/subject_tour/math0903_irs_10/task_01/}{statement}
\subimport{final/subject_tour/math0903_irs_10/task_01/}{solution}

\subimport{final/subject_tour/math0903_irs_10/task_02/}{statement}
\subimport{final/subject_tour/math0903_irs_10/task_02/}{solution}

\subimport{final/subject_tour/math0903_irs_10/task_03/}{statement}
\subimport{final/subject_tour/math0903_irs_10/task_03/}{solution}


\section{Информатика}

\subimport{final/subject_tour/inf0903_irs/task_01/}{statement.tex}
\subimport{final/subject_tour/inf0903_irs/task_02/}{statement.tex}
\subimport{final/subject_tour/inf0903_irs/task_03/}{statement.tex}


\chapter{Командный тур}

\subimport{final/command_tour/irs/}{content.tex}

\end{document}