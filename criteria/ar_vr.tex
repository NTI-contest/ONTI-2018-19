\part{Критерии}
\chapter{Критерии определения победителей и призеров заключительного этапа}

\section{Первый отборочный этап}
 
Количество баллов, набранных при решении всех задач, суммируется. Задания различались для учащихся 9 (и младше, при желании участвовать) и 10-11 классов. Победителями считаются участники, набравшие необходимое количество баллов и допущенные во второй этап: 
\begin{itemize}
    \item \textbf{по направлению «Виртуальная реальность»}: 
    \begin{itemize}
        \item 9 класс и младше: 50 баллов и выше
        \item 10-11 классы: 60 баллов и выше 
    \end{itemize}
    \item \textbf{по направлению «Дополненная реальность»}: 20 баллов, независимо от уровня.
\end{itemize}

\section{Второй отборочный этап}

\textbf{Направление «Виртуальная реальность»}

Второй этап отбора проводится дистанционно с помощью системы автоматической проверки задач CATS, разработанной Дальневосточным федеральным университетом.

Участникам предлагаются как классические олимпиадные задачи с тематикой ВР, так и задачи на разработку модулей в приложениях на фреймворке Unity. Количество баллов, набранных при решении всех задач, суммируется.

Победителями объявлялись участники, набравшие не менее 210 баллов, которые допущены к участию в финале Олимпиады НТИ 2018/19 по направлению «Виртуальная реальность».

\textbf{Направление «Дополненная реальность»}

Количество баллов, набранных при решении всех задач, суммируется.

Победителями объявлялись участники, набравшие не менее 131 балла, которые допущены к участию в финале Олимпиады НТИ 2018/19 по направлению «Дополненная реальность».

\section{Заключительный этап}

В заключительном этапе олимпиады баллы участника складываются из двух частей: он получает баллы за индивидуальное решение задач по предметам (информатика, математика) и за командное решение практической задачи по разработке приложения для виртуальной реальности. Итоговая оценка участника олимпиады получается по следующей формуле:
$$P = P_1/S_1 \cdot 20 + P_2/S_2 \cdot 20 + P_3/S_3 \cdot 60, \: \text{где}$$
$P_1$ — количество баллов, набранное в рамках индивидуальной части заключительного этапа по математике,
$S_1$ — максимально возможное количество баллов за индивидуальную часть заключительного этапа по математике.
$P_2$ — количество баллов, набранное в рамках индивидуальной части заключительного этапа по информатике,
$S_2$ — максимально возможное количество баллов за индивидуальную часть заключительного этапа по информатике.
$P_3$ — количество баллов, набранное в рамках командной части заключительного этапа,
$S_3$ — максимально возможное количество баллов за командную часть заключительного этапа.

Критерий определения победителей и призеров (независимо от класса):
\begin{center}
    \begin{tabular}{|l|l|l|}
        \hline
        \multirow{2}{*}{Категория}& \multicolumn{2}{|c|}{Количество баллов} \\
        \cline{2-3}
        &Дополненная реальность& Виртуальная реальность \\
        \hline
        Победители&65 и выше& 74.91 и выше\\
        \hline
        Призеры&От 54 до 63 баллов&От 67.37 до 73.47 баллов \\
        \hline
    \end{tabular}
\end{center}

