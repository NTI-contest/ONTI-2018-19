\part{Критерии}

\chapter{Критерии определения победителей и призеров заключительного этапа}

\section{Первый отборочный этап}

В первом отборочном этапе участники решали задачи по двум предметам: физика и информатика, в каждом предмете максимально можно было набрать 100 баллов. Для того, чтобы пройти во второй этап участники  должны были набрать в сумме по обоим предметам не менее 15 баллов, независимо от уровня.

\section{Второй отборочный этап}

Максимальный балл, который можно было набрать за решение задач второго этапа – 100 баллов на команду. Был составлен общий рейтинг, и с начала рейтинга (с максимальных результатов) были выбраны победители и призеры (по командам). На заключительный (финальный) этап прошло 6 команд. Команда победитель набрала 70,1 балл.   

\section{Заключительный этап}

В заключительном этапе олимпиады баллы участника складываются из двух частей: баллы за индивидуальное решение задач по предметам (математика, информатика) и баллы за командное решение практических задач в области технологий беспроводной связи.

Итоговый балл определяется по формуле: $$S = 0.2 \cdot (S_1+S_2) + 0.6 \cdot S_3, \: \text{где}$$

$S_1$ – балл первой части заключительного этапа по математике в стобалльной системе ($S_1$ макс = 100);

$S_2$ – балл первой части заключительного этапа по информатике в стобалльной системе ($S_2$ макс = 100);

$S_3$ – итоговый балл командного тура в стобалльной системе ($S_3$ макс = 100).

Итого максимально возможный балл по условиям общего рейтинга $0.2 \cdot (100+100) + 0.6 \cdot 100 = 100$ баллов.

Индивидуальный предметный тур:

Математика – максимально возможный балл за все задачи - 100 баллов; 

Информатика – максимально возможный балл за все задачи - 100 баллов; 

Командный финальный тур:

Команды, прошедшие в финал, получали за командный тур от 0 до 100 баллов: команда, набравшая максимальное число очков, получала 100 баллов. И становилась командой-победителем. Остальные команды, получали баллы, нормированные на этот результат по формуле $S = 100 \cdot x/MAX$, где $x$ — число внутриигровых очков, набранных командой, $MAX$ — число очков, набранное в финале командой-победителем. Таким образом, победитель финала получает всегда 100 баллов, остальные — от 0 до 100, пропорционально показанным ими результатам.

Критерий определения победителей и призеров (независимо от класса): 

С начала рейтинга были выбраны 2 победителя и 4 призера (первые 25\% участников рейтинга становятся победителями или призерами – первые 8\% участников рейтинга становятся победителями).

\begin{center}
    \begin{tabular}{|l|l|}
        \hline
        Категория&Количество баллов\\
        \hline
        Победители&70.4 и выше\\
        \hline
        Призеры&От 64.57 до 70\\
        \hline
    \end{tabular}
\end{center}