\part{Критерии}

\chapter{Критерии определения победителей и призеров заключительного этапа}

\section{Первый отборочный этап}

Участникам было представлено 40 задач, за каждую из которых можно было набрать не более 1000 баллов, в зависимости от количества участников ранее решивших каждую конкретную задачу. Балл за задачу вычисляется по формуле $$\frac{1000}{log_3(3^3 + n) - 2},$$ где $n$ — количество прочих участников, решивших задачу. Количество баллов, набранных при решении всех задач, суммируется.

Возможный максимум — 40000 баллов.

Победители отборочного этапа являются финалистами и должны были набрать 6550 баллов и выше вне зависимости от уровня.

\section{Заключительный этап}

В заключительном этапе участникам представляются 11 задач. Максимальный возможный балл равен 11000. Балл за задачу вычисляется по формуле $$\frac{1000}{log_2(2^4 + n) - 3},$$ где $n$ — количество прочих участников, решивших задачу. Количество баллов, набранных при решении всех задач, суммируется.

Критерий определения победителей и призеров:
\begin{itemize} 
    \item Победителями становились не более 8\% лучших финалистов.
    \item Призерами становились не более 25\% лучших финалистов (за исключением победителей).
\end{itemize} 

\begin{center} 
    \begin{tabular}{|l|l|} 
        \hline
        Категория & Количество баллов \\
        \hline
        Победители & 3000 баллов и выше \\
        \hline
        Призеры & От 1586 до 2999 баллов \\
        \hline
    \end{tabular} 
\end{center} 