\part{Критерии}

\chapter{Критерии определения победителей и призеров заключительного этапа}

\section{Первый отборочный этап}

В первом отборочном этапе участники решали задачи по двум предметам: физика и информатика, в каждом предмете максимально можно было набрать 100 баллов. Для того, чтобы пройти во второй этап участники должны были набрать в сумме по обоим предметам не менее 20 баллов, независимо от уровня.

\section{Второй отборочный этап}

Максимально возможное число баллов за этап — 34,6. Баллы за каждую задачу автоматически подсчитываются на платформе Stepik. При решении задачи одним из участников команды баллы начислялись всем членам команды. Был составлен общий рейтинг, и с начала рейтинга (с максимальных результатов) были выбраны победители и призеры. В заключительный этап прошли команды, набравшие не менее 5,6 баллов (победители и призеры). Команда победитель набрала 11,2 балла.   

\section{Заключительный этап}

В заключительном этапе олимпиады баллы участника складываются из двух частей: баллы за индивидуальное решение задач по предметам (математика, информатика) и баллы за командное решение практических задач в области интеллектуальных энергетических систем.

Итоговый балл определяется по формуле: $S = 0.2 \cdot  (S1+S2) + 0.6 \cdot S3$, где:

S1 – балл первой части заключительного этапа по математике в стобалльной системе (S1 макс = 100);

S2 – балл первой части заключительного этапа по информатике в стобалльной системе (S2 макс = 100);

S3 – итоговый балл командного тура в стобалльной системе (S3 макс = 100).

Итого максимально возможный балл по условиям общего рейтинга $0.2 \cdot (100+100) + 0.6 \cdot 100 = 100$ баллов.

Индивидуальный предметный тур:

Математика – максимально возможный балл за все задачи — 100 баллов;

Информатика – максимально возможный балл за все задачи — 100 баллов;

Командный тур заключительного этапа:

Командный тур заключительного этапа был распределенный и проходил в двух регионах — в Москве и Иркутске. В каждом регионе игры проводились на идентичных стендах. Связь между стендами и главным сервером игры осуществлялась через сеть Интернет. Возможности стенда позволяли одновременно играть 4-м командам, итого в одной игре одновременно могло принимать участие до 8 команд.

Командный тур проводился в два шага: сначала проводились два полуфинала (в каждом участвовали: на стенде в Москве – 4 команды, на стенде в Иркутске – 3 команды), затем финальная игра. По регламенту соревнований, с учетом обеспечения равенства возможностей на двух площадках, с каждого стенда с каждого полуфинала в финальную игру проходило по 2 команды. Затем 8 команд-финалистов участвовали в финальной игре.

Командам, не прошедшим в финальную игру, за командный тур начислялось 0 баллов.

Команды, прошедшие в финальную игру, получали за командный тур от 0 до 100 баллов: команда, набравшая максимальное число очков, получала 100 баллов (и становилась командой-победителем), команда, набравшая минимальное число очков — 0 баллов. Остальные команды, получали баллы, нормированные на эти два результата по формуле S = 100×(x–MIN)/(MAX–MIN), где x — число внутриигровых очков, набранных командой, MAX — число очков, набранное в финале командой-победителем, MIN — число очков, набранное командой, занявшее в финальной игре последнее место. Таким образом, команда-победитель финальной игры получает всегда 100 баллов, команда, занявшая в финальной игре последнее место — всегда 0 баллов, остальные — от 0 до 100, пропорционально показанным ими результатам.

Критерий определения победителей и призеров (независимо от класса):

С начала рейтинга были выбраны 4 победителя и 9 призеров (первые 25\% участников рейтинга становятся победителями или призерами, из них первые 8\% участников рейтинга становятся победителями).

\begin{center}
    \begin{tabular}{|l|l|}
        \hline
        Категория&Количество баллов\\
        \hline
        Победители&61.1 и выше\\
        \hline
        Призеры&От 51.6 до 61\\
        \hline
    \end{tabular}
\end{center}