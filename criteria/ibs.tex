\part{Критерии}

\chapter{Критерии определения победителей и призеров заключительного этапа}
 
\section{Первый отборочный этап}

В первом отборочном этапе участники решали задачи по двум предметам: биология и химия, в каждом предмете максимально можно было набрать 100 баллов. Для того, чтобы пройти во второй этап, участники должны были набрать в сумме по обоим предметам необходимое количество баллов:
\begin{itemize}
    \item Направление «Агробиотехнологии»
    \begin{itemize}
        \item 9 классы и младше - 120 баллов и выше;
        \item 10-11 классы - 130 баллов и выше.
    \end{itemize}
    \item Направление «Геномное редактирование»
    \begin{itemize}
        \item только учащиеся 10-11 классов - 90 баллов и выше.
    \end{itemize}
\end{itemize}

\section{Второй отборочный этап}

Количество баллов, набранных при решении всех задач, суммируется. 

Победители второго отборочного этапа должны были набрать:
\begin{itemize}
    \item Направление «Агробиотехнологии»
    \begin{itemize}
        \item 9 классы и младше - 57 баллов и выше;
        \item 10-11 классы - 28,1 балл и выше.
    \end{itemize}
    \item Направление «Геномное редактирование» 
    \begin{itemize}
        \item только учащиеся 10-11 классов - 44 балла и выше.
    \end{itemize}
\end{itemize}

\section{Заключительный этап}

Направление «Агробиотехнологии»

Определение команд-победителей в финале проводилось при помощи сложения баллов за каждое выполненное или невыполненное задание командой. Команда, набравшая наибольшее количество баллов, становилась победителем.

Определение победителей и призеров (индивидуальный зачет) проводилось путем сложения нормированных баллов участника за каждый предмет индивидуального тура (химия и биология) с учетом их вклада и командных баллов с учетом соответствующего вклада.

В финале 2018/19 гг. использовались следующие вклады индивидуального и командного зачета:\\
20\% - биология (индивидуальный тур); \\
20\% - химия (индивидуальный тур); \\
60\%- командные задания.

Таким образом общий балл участника рассчитывался по формуле:
$$Score = Sn_\text{био} \cdot 0.2 + Sn_\text{хим} \cdot 0.2 + Sn_\text{ком} \cdot 0.6, \: \text{где:}$$\\
$Sn_\text{био}$ - нормированный балл по биологии индивидуального тура;\\
$Sn_\text{хим}$ - нормированный балл по химии индивидуального тура;\\
$Sn_\text{ком}$ - нормированный балл командного тура.

Нормировка баллов индивидуальных и командных туров проводилась по формуле:

$$Sn = (S / S_{max}) \cdot 100, \: \text{где:}$$

$S$ - сумма баллов за соответствующий предметный или командный тур;

$S_{max}$ - максимум баллов за предметный или командный тур.

По направлению «Агробиотехнологии» были следующие максимальные баллы за предметный или командный тур:
\begin{itemize}
    \item 9 классы:
    \begin{itemize}
        \item Химия (индивидуальный тур) - 100 
        \item Биология (индивидуальный тур) - 46
        \item Командный тур - 90
    \end{itemize}
    \item 10-11 классы:
    \begin{itemize}
        \item Химия (индивидуальный тур) - 100
        \item Биология (индивидуальный тур) - 51
        \item Командный тур - 100
    \end{itemize}
\end{itemize}

Зачет победителей и призеров общий для 9 и 10-11 классов по нормированным баллам.

Направление «Геномное редактирование» 

В заключительном этапе олимпиады баллы участника складываются из двух частей: он получает баллы за индивидуальное решение задач по предметам (химия, биология) и за командное решение практической задачи на анализ результатов редактирования клеточной линии технологией CRISPR/Cas.

Оценки первой части заключительного этапа:
\begin{itemize}
    \item Химия  - 50 баллов (максимальный возможный балл за задачи);
    \item Биология  - 50 баллов (максимальный возможный балл за задачи).
\end{itemize}

Командный тур - максимум 100 баллов (максимальный возможный балл за задачи командного тура).

Итоговый балл определяется по формуле: $$S = S1 \cdot 0.3 + S2 \cdot 0.3 + S3 \cdot 0.7, \: \text{где:}$$ \\
$S1$ – балл первой части заключительного этапа по химии ($S1$ макс = 50);\\
$S2$ – балл первой части заключительного этапа по биологии ($S2$ макс = 50);\\
$S3$ – итоговый балл командного тура ($S3$ макс = 100).

Критерий определения победителей и призеров (независимо от уровня):
\begin{center}
    \begin{tabular}{|l|l|l|}
        \hline
        Категория& \multicolumn{2}{|c|}{Количество баллов} \\
        \hline
        &«Агробиотехнологии»&«Геномное редактирование»\\
        \hline
        Победители&65 и выше&75.05 и выше\\
        \hline
        Призеры&От 55 до 64 баллов&От 67.8 до 74.55 баллов\\
        \hline
    \end{tabular}{|l|l|}
\end{center}