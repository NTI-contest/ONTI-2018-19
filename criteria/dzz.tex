\part{Критерии}

\chapter{Критерии определения победителей и призеров заключительного этапа}

\section{Первый отборочный этап}

В первом отборочном этапе участники решали задачи по двум предметам: физика и информатика, в каждом предмете максимально можно было набрать 100 баллов. Для того, чтобы пройти во второй этап участники  должны были набрать в сумме по обоим предметам не менее 10 баллов независимо от уровня

\section{Второй отборочный этап}

Количество баллов, набранных при решении всех задач, суммируется. 

Победители второго отборочного этапа должны были набрать 44 балла и выше - для 9 классов и младше, и 75 баллов и выше -  для 10-11 классов. 

\section{Заключительный этап}

Общий критерий оценки за командный тур вычисляется по формуле:
$$Nc = (Nd1 + Nd2 + (N1 + N2 + N3)/3)/4$$

Максимальный балл за командный тур: 400

$Nd1$ - балл, набранный командой за первый оцениваемый день;

$Nd2$ - балл, набранный командой за второй оцениваемый день;

$N1$, $N2$, $N3$ - балл, выставленный команде 1-ым, 2-ым и 3-им членом комиссии жюри на финальном зачете;

$(N1 + N2 + N3)/3$ - усредненный балл команды за финальный зачет.

Сумма баллов за командный тур нормирована к 100 баллам.

Победителем становится команда, набравшая по итогу максимальный балл. Для 11-ых и 9-ых классов команда-победитель определяется отдельно.

Общий балл определяется по формуле:
$$N = 0.7Nc +0.15(Nf + Ni)$$

$Nf$ - балл, набранный участником в предметном туре по физике;

$Ni$ - балл, набранный участником в предметном туре по информатике;

$0.15$ - коэффициент, дающий вес предметному туру в финальном зачете в 30\%;

$0.7$ - коэффициент, дающий вес командному туру в финальном зачете в 70\%.

Критерий определения победителей и призеров (независимо от уровня):
\begin{center}
    \begin{tabular}{|l|l|}
        \hline
        Категория&Количество баллов\\
        \hline
        Победители&74.5 и выше\\
        \hline
        Призеры&От 66.2 до 72.6 баллов\\
        \hline
    \end{tabular}
\end{center}
