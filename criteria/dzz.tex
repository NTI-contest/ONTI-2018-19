\part{Критерии}

\chapter{Критерии определения победителей и призеров заключительного этапа}

\section{Первый отборочный этап}

В первом отборочном этапе участники решали задачи по двум предметам: физика и информатика, в каждом предмете максимально можно было набрать 100 баллов. Для того, чтобы пройти во второй этап участники  должны были набрать в сумме по обоим предметам не менее 10 баллов независимо от уровня

\section{Второй отборочный этап}

Количество баллов, набранных при решении всех задач, суммируется. 

Победители второго отборочного этапа должны были набрать 44 балла и выше - для 9 классов и младше, и 75 баллов и выше -  для 10-11 классов. 

\section{Заключительный этап}

Общий критерий оценки за командный тур вычисляется по формуле:
$$N_c = (N_{d1} + N_{d2} + (N_1 + N_2 + N_3)/3)/4$$

Максимальный балл за командный тур: 400

$N_{d1}$ — балл, набранный командой за первый оцениваемый день;

$N_{d2}$ — балл, набранный командой за второй оцениваемый день;

$N_1$, $N_2$, $N_3$ — балл, выставленный команде 1-ым, 2-ым и 3-им членом комиссии жюри на финальном зачете;

$(N_1 + N_2 + N_3)/3$ — усредненный балл команды за финальный зачет.

Сумма баллов за командный тур нормирована к 100 баллам.

Победителем становится команда, набравшая по итогу максимальный балл. Для 11-ых и 9-ых классов команда-победитель определяется отдельно.

Общий балл определяется по формуле:

$$N = 0.7N_c +0.15(N_f + N_i)$$

$N_f$ — балл, набранный участником в предметном туре по физике (90 баллов нормированы до 100);

$N_i$ — балл, набранный участником в предметном туре по информатике (100 баллов);

$0.15$ — коэффициент, дающий вес предметному туру в финальном зачете в 30\%;

$0.7$ — коэффициент, дающий вес командному туру в финальном зачете в 70\%.

Критерий определения победителей и призеров (независимо от уровня):
\begin{center}
    \begin{tabular}{|l|l|}
        \hline
        Категория&Количество баллов\\
        \hline
        Победители&74.5 и выше\\
        \hline
        Призеры&От 66.2 до 72.6 баллов\\
        \hline
    \end{tabular}
\end{center}
