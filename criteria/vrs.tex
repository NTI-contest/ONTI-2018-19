\part{Критерии}

\chapter{Критерии определения победителей и призеров заключительного этапа}

\section{Первый отборочный этап}

В первом отборочном этапе участники решали задачи по двум предметам: физика и информатика, в каждом предмете максимально можно было набрать 100 баллов. Для того, чтобы пройти во второй этап участники  должны были набрать в сумме по обоим предметам не менее:
\begin{itemize}
    \item 20 баллов - учащиеся 9 классов и младше;
    \item 40 баллов - учащиеся 10-11 классов.
\end{itemize}

\section{Второй отборочный этап}

Количество баллов, набранных при решении всех задач, суммируется. 

Победители второго отборочного этапа должны были набрать 100 баллов и выше, независимо от уровня.

\section{Заключительный этап}

Заключительный этап состоит из командного и индивидуального тура.
\begin{itemize}
    \item Командный тур имеет вес 0.6;
    \item Индивидуальный тур имеет вес 0.4.
\end{itemize}
 
Командный тур состоит из трех задач:
\begin{itemize}
    \item Задача 1. Разработка устройства – 30 баллов;
    \item Задача 2. Миссия в симуляторе – 30 баллов;
    \item Задача 3. Миссия в бассейне – 40 баллов.
\end{itemize}

Итого за командный тур можно заработать 100 баллов. Каждый член команды получает столько баллов, сколько заработала команда.
 
Индивидуальный тур состоит из решения задач по физике и информатике:
\begin{itemize}
    \item На физике можно было набрать 100 баллов;
    \item На информатике 300 баллов.
\end{itemize}
 
Чтобы вычислить итоговые баллы, использовалась следующая формула:
$$P = K \cdot 0.6 + (\text{Ф}/2+\text{И}/6) \cdot 0.4$$
где $P$ – итоговые баллы, $K$ – баллы за командный тур, $\text{Ф}$ – баллы за физику, $\text{И}$ –
баллы за информатику.

Победителями становились лучшие 8\% финалистов (с округлением вниз).

Призерами становились лучшие 25\% финалистов (с округлением вниз без учёта победителей).

Критерий определения победителей и призеров (независимо от уровня):
\begin{center}
    \begin{tabular}{|l|l|}
        \hline
        Категория&Количество баллов\\
        \hline
        Победители&63.6 и выше\\
        \hline
        Призеры&От 60.86 до 63 баллов\\
        \hline
    \end{tabular}
\end{center}
