\part{Критерии}

\chapter{Критерии определения победителей и призеров заключительного этапа}
 
\section{Первый отборочный этап}

В первом отборочном этапе участники решали задачи по двум предметам - математика и информатика, в каждом предмете максимально можно было набрать 100 баллов. Для того, чтобы пройти во второй этап участники должны были набрать в сумме по обоим предметам 40 баллов независимо от уровня.

\section{Второй отборочный этап}

Количество баллов, набранных при решении всех задач, суммируется. Победители второго отборочного этапа являются финалистами и должны были набрать 155 баллов и выше вне зависимости от уровня.

\section{Заключительный этап}

В заключительном этапе олимпиады баллы участника складываются из двух частей: 1 часть - баллы за индивидуальное решение задач по предметам (физика, информатика);

2 часть - баллы за командное решение практической задачи в области проектирования и реализации систем умного города.

Оценки первой части заключительного этапа:

Физика - 100 баллов (максимальный возможный балл за задачи);

Информатика - 100 баллов (максимальный возможный балл за задачи).

Командный тур максимум 142 балла, нормировано на 100 баллов (максимальный возможный балл за задачи командного тура).

Итоговый балл определяется по формуле: 
$$S = S_1 \cdot 0.2 + S_2 \cdot 0.2 + S_3 \cdot 0.6, где$$
$S_1$ – балл первой части заключительного этапа по физике ($S_1$ макс = 100);
$S_2$ – балл первой части заключительного этапа по информатике ($S_2$ макс = 100);
$S_3$ – итоговый балл командного тура в стобалльной системе ($S_3$ макс = 100).

Критерий определения победителей и призеров (независимо от уровня):
\begin{center}
    \begin{tabular}{|l|l|}
        \hline
        Категория&Кол-во баллов\\
        \hline
        Победители&50.63 балла  и выше\\
        \hline
        Призеры&От 44.73 до 49.95 баллов\\
        \hline
    \end{tabular}
\end{center}
