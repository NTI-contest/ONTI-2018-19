\part{Критерии}

\chapter{Критерии определения победителей и призеров заключительного этапа}
 
\section{Первый отборочный этап}

В первом отборочном этапе участники решали задачи по двум предметам - математика и информатика, в каждом предмете максимально можно было набрать 100 баллов. Для того, чтобы пройти во второй этап участники должны были набрать в сумме по обоим предметам 75 баллов независимо от уровня.

\section{Второй отборочный этап}

Количество баллов, набранных при решении всех задач, суммируется. Участники второго отборочного этапа, набравшие 20 баллов и больше вне зависимости от уровня,  являются участниками заключительного этапа или финалистами олимпиады.

\section{Заключительный этап}

В заключительном этапе баллы участника S складываются из оценки за индивидуальную часть S1 (максимум – 100 баллов) и оценки за командную задачу S2 (максимум – 100 баллов) по профилю «Нейротехнологии».  

Оценка за индивидуальную часть S1 складывается из оценки предметного тура по биологии b (максимум – 100 баллов) данного профиля и оценки предметного тура по информатике i (максимум 100 баллов) данного профиля, нормированные на 100 баллов. 

Таким образом, оценка за индивидуальную часть: $S1 = (b + i)/2$.

Итоговая оценка участника (S) считается по формуле: $S = 0.3 \cdot S1 + 0.7 \cdot S2$ (Максимальное значение S – 100 баллов).
 
Критерии определения победителей и призеров (независимо от класса):
\begin{center}
    \begin{tabular}{|l|p{7cm}|}
        \hline
        Категория&Количество баллов\\
        \hline
        Победители&от 72 баллов \\
        \hline
        Призеры&от 60 баллов до 72 (не включая)\\
        \hline
    \end{tabular}
\end{center}