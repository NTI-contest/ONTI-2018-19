\part{Критерии}

\chapter{Критерии определения победителей и призеров заключительного этапа}

\section{Первый отборочный этап}
 
В первом отборочном этапе участники решали задачи по двум предметам - физика и химия, в каждом предмете максимально можно было набрать 100 баллов. Для того, чтобы пройти во второй этап участники должны были набрать в сумме по обоим предметам не менее 5 баллов вне зависимости от уровня.

\section{Второй отборочный этап}

Количество баллов, набранных при решении всех задач, суммируется. Возможный максимум - 100 баллов.
 
Победители второго отборочного этапа являются финалистами и должны были набрать 30 баллов и выше вне зависимости от уровня.

\section{Заключительный этап}

В заключительном этапе олимпиады баллы участника складываются из двух частей:
\begin{itemize}
    \item 1 часть: баллы за индивидуальное решение задач по предметам (физика, химия);
    \item 2 часть: баллы за командное решение практической задачи в области проектирования и изготовления сегмента крыла самолёта из композиционного материала.
\end{itemize}

Оценки первой части заключительного этапа:
\begin{itemize}
    \item Физика - 100 баллов (максимальный возможный балл за задачи);
    \item Химия - 100 баллов (максимальный возможный балл за задачи).
\end{itemize}

Командный тур - 100 баллов (максимальный возможный балл за задачи командного
тура).

Итоговый балл определяется по формуле: $S = 0.15 \cdot S1 + 0.15 \cdot S2 + 0.7 \cdot S3$, где\\
$S1$ – балл первой части заключительного этапа по физике ($S1$ макс = 100);\\
$S2$ – балл первой части заключительного этапа по химии ($S2$ макс = 100);\\
$S3$ – итоговый балл командного тура ($S3$ макс = 100).

Критерий определения победителей и призеров независимо от уровня:
\begin{center}
    \begin{tabular}{|l|l|}
        \hline
        Категория&Количество баллов\\
        \hline
        Победители&73 балла и выше\\
        \hline
        Призеры&От 59 до 72 баллов\\
        \hline
    \end{tabular}
\end{center}