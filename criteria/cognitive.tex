\part{Критерии}

\chapter{Критерии определения победителей и призеров заключительного этапа}

\section{Первый отборочный этап}

В первом отборочном этапе участники решали задачи по двум предметам: информатика и биология.  По каждому предмету максимальное количество баллов составляло – 100.  Для прохождения на второй этап олимпиады участнику необходимо было набрать в сумме по обоим предметам не менее 5 баллов.

\section{Второй отборочный этап}

Количество баллов, набранных при решении  всех задач, суммируется. Максимальное количество – 100 баллов.

Победители второго отборочного этапа являются финалистами. Победители второго отборочного этапа должны были набрать  60 баллов и выше.

\section{Заключительный этап}

На заключительном этапе олимпиады баллы участников складываются из двух частей:
\begin{itemize} 
\item Часть I:  баллы за индивидуальное решение задач по предметам (биология, информатика)
\item Часть II: баллы за командное решение практической задачи в области когнитивных технологий.
\end{itemize} 

Оценки первой части заключительного этапа:
\begin{itemize} 
\item Биология: 100 баллов (максимально возможный балл за задания).
\item Информатика: 100 баллов (максимально возможный балл за задания).
\end{itemize} 

Командный тур: 60 баллов (максимально возможный балл за задания командного тура).

Итоговый балл определяется по формуле:
$$I =0. 6 \cdot I1 + 0.2 \cdot I2 + 0.2 \cdot I3, \: \text{где}$$\\
$I1$ – итоговый балл командного тура ($I1$  макс=100)\\
$I2$ –  балл первой части заключительного этапа по биологии  ($I2$ макс=100)\\
$I3$ –  балл первой части заключительного этапа по информатике  ($I3$ макс=100)

Критерии определения победителей и призеров (независимо от уровня):
\begin{center} 
    \begin{tabular}{|l|l|}
        \hline
        Категория & Количество баллов \\
        \hline
        Победители & 60.4 балла и выше \\
        \hline
        Призеры & 54.4 балла и выше \\
        \hline
    \end{tabular} 
\end{center} 