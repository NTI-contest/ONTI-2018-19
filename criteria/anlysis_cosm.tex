\part{Критерии}
\chapter{Критерии определения победителей и призеров заключительного этапа}

\section{Первый отборочный этап}
 
В первом отборочном этапе участники решали задачи по двум предметам: физика и информатика, в каждом предмете максимально можно было набрать 100 баллов. Для того, чтобы пройти во второй этап участники  должны были набрать в сумме по обоим предметам не менее 10 баллов независимо от уровня.

\section{Второй отборочный этап}

Количество баллов, набранных при решении всех задач, суммируется.

Победители второго отборочного этапа должны были набрать 12 баллов и выше независимо от уровня.

\section{Заключительный этап}

На заключительном этапе Олимпиады итоговые баллы участников складываются из трёх частей:
\begin{itemize}
    \item 1 часть – баллы за индивидуальное решение задач предметного тура по информатике;
    \item 2 часть – баллы за индивидуальное решение задач предметного тура по географии;
    \item 3 часть – баллы за решение командных задач.
\end{itemize}

Баллы за решение предметных задач определялись как сумма баллов за каждую задачу по данному предмету. Максимальный возможный балл за решение предметных задач индивидуального этапа по географии – 100 баллов, по информатике – 100 баллов.

Система оценки решений задач командного тура строилась следующим образом: общая максимальная оценка за решение  трёх задач  А, Б и В и за продемонстрированные навыки составляла 100 баллов. Их распределение было следующее.
\begin{itemize}
    \item До 50 баллов – за решение основной Задачи А (результат оценивался путем сравнения с набором пробных площадей, интерпретированных экспертами).
    \item До 20 баллов – за базовые навыки и знания, продемонстрированные при решении Задачи А (результат оценивался по итогам изучения лабораторного журнала, а также ежедневных выходных данных).
    \item До 20 баллов – за решение Задачи Б (результат оценивался путём сравнения результата участников с результатом экспертов).
    \item До 10 баллов – за решение Задачи В (требовались решения как Задачи А, так и Задачи Б, результат оценивался по совпадению результатов расчётов самих участников с результатами проверочного расчёта по данным, переданным командами как результаты задач А и Б.)
\end{itemize}

Методика расчёта итоговых баллов за решение задач заключительного этапа олимпиады описана ниже.

Итоговый балл определяется по формуле: $$S = S1 \cdot 0.2 + S2 \cdot 0.2 + S3 \cdot 0.6, $$где:\\
$S1$ – балл за индивидуальное решение предметных задач по географии ($S1$ макс = 100);\\
$S2$ – балл за индивидуальное решение предметных задач по информатике ($S2$ макс = 100);\\
$S3$ – итоговый балл командного тура ($S3$ макс = 100).

Критерий определения победителей и призеров: призеры – 25\% от числа участников финала с максимальным количеством баллов, из них победители – 8\% от числа участников финала с максимальным количеством баллов.

Критерий определения победителей и призеров (независимо от уровня):
\begin{center}
    \begin{tabular}{|l|l|}
        \hline
        Категория& Количество баллов\\
        \hline
        Победители&38.6 и выше\\
        \hline
        Призеры&От 31 до 35.4 баллов\\
        \hline
    \end{tabular}
\end{center}