\part{Критерии}
\chapter{Критерии определения победителей и призеров заключительного этапа}

\section{Первый отборочный этап}
 
В первом отборочном этапе участники решали задачи по двум предметам - физике и информатике, в каждом предмете максимально можно было набрать 100 баллов. Для того, чтобы пройти во второй этап участники должны были набрать в сумме по обоим предметам 10 баллов независимо от уровня.

\section{Второй отборочный этап}

Задание 2-го этапа выполнялось онлайн, на платформе Stepik.org и содержало задачи, соответствующие профилю ППТ, а именно:
\begin{itemize}
    \item Моделирование в САПР (Stepik, автоматическое оценивание)
    \item Основы электроники и программирование микроконтроллерных устройств (Arduino) (Stepik, автоматическое оценивание)
    \item Практическое задание. Участникам нужно было спроектировать несложное устройство, изготовить его и запрограммировать его перемещение (Stepik, экспертное оценивание по снимкам и видеороликам).  Проектируемое устройство было сильно упрощенным подобием финальной задачи (модель мостового крана с перемещаемой кареткой).
\end{itemize}

Количество баллов, набранных при решении каждой из задач, суммируется. Максимальное количество баллов 100  (103  балла в сезоне 2018-2019), проходной балл определяется отсечкой по максимально возможному числу команд на финале. В сезоне 2018-2019 на финал было допущено 16 команд, с проходным баллом 83.  Все эти команды удовлетворительно выполнили практическое задание 2-го тура.


\section{Заключительный этап}

В заключительном этапе баллы участника $S$ складываются из оценки за индивидуальную часть $S1$ (максимум – 100 баллов) и оценки за командную задачу $S2$ (максимум – 100 баллов) по профилю «Передовые производственные технологии».

Оценка за индивидуальную часть $S1$ складывается из оценки предметного тура по физике $Sf$ (максимум – 100 баллов) данного профиля и оценки предметного тура по информатике $Si$ (максимум 100 баллов) данного профиля, нормированные на 100 баллов.

Таким образом оценка за индивидуальную часть:$ S1 = \frac{Sf + Si}{2}$

Итоговая оценка участника ($S$) считается по формуле: $S = 0.3 \cdot S1 + 0.7 \cdot S2$

(Максимальное значение $S$ = 100 баллов).

На заключительном этапе баллы за командное задание начислялись участникам в течение всех 4-х дней соревнования, по мере сдачи тестов-испытаний отдельных узлов или целиком собранного устройства. Начисленные баллы отмечались в большой таблице, размещенной на площадке и видимой всем участникам.   Баллы за прохождение промежуточных тестов уменьшались в каждый из последующих дней.  Наибольшее количество баллов начислялось за интегральные тесты всего устройства, проходящие в последний день.  Прохождение промежуточных тестов оценивалось "сдал/не сдал", для интегральных тестов подсчитывалось число перестановок, успешно выполненных устройством. Подсчет суммарных оценок производился немедленно по завершении соревнования, отдельная экспертная оценка изделий не требовалась.

\begin{center}
    Таблица оценок на конец соревнований
\end{center}

\putImgWOCaption{10cm}{criteria/ppt}

Критерий определения победителей и призеров (независимо от уровня):
\begin{center}
    \begin{tabular}{|l|l|l|}
        \hline
        Категория&Количество баллов&Число победителей/призеров\\
        \hline
        Победители & от 39.95 (включая) и выше & 2\\
        \hline
        Призёры & от 25 и выше & 8 \\
        \hline
    \end{tabular}
\end{center}