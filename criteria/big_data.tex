\part{Критерии}

\chapter{Критерии определения победителей и призеров заключительного этапа}

\section{Первый отборочный этап}
 
В первом отборочном этапе участники решали задачи по двум предметам - математика и информатика, в каждом предмете максимально можно было набрать 100 баллов. Для того, чтобы пройти во второй этап участники должны были набрать в сумме по обоим предметам не менее 90 баллов вне зависимости от уровня.

\section{Второй отборочный этап}

Количество баллов, набранных при решении всех задач, суммируется. 

Победители второго отборочного этапа являются финалистами и должны были набрать 85 баллов и выше вне зависимости от уровня.

\section{Заключительный этап}

Заключительный этап состоит из предметного тура по предметам математика и информатика и командного тура.\\
$Sm$ - баллы набранные по математике. Можно было набрать максимум 100 баллов.\\
$Si$ - баллы набранные по информатике. Можно было набрать максимум 100 баллов.\\
$Sc$ - баллы за командный тур. Можно было набрать максимум 100 баллов.

Балл за командный тур вычисляется следующим образом: команда, набравшая наибольший score (наименьшую ошибку),получает 100 баллов, все остальные команды получают балл, которые расчитываюися по формуле:

$$Sс = 100 \cdot \frac{score \: \text{команды}}{\text{максимально набранное} \: score}$$
 
Итоговая формула $$S = (Si + Sm) \cdot 0.2 + Sc \cdot 0.6$$

Критерий определения победителей и призеров независимо от уровня:
\begin{center}
    \begin{tabular}{|l|l|}
        \hline
        Категория&Количество баллов\\
        \hline
        Победители&69 баллов и выше\\
        \hline
        Призеры&От 24.98 до 63.55 баллов\\
        \hline
    \end{tabular}
\end{center}