\part{Критерии}
\chapter{Критерии определения победителей и призеров заключительного этапа}

\section{Первый отборочный этап}
 
В первом отборочном этапе участники решали задачи по двум предметам: физика и информатика, в каждом предмете максимально можно было набрать 100 баллов.

Для того, чтобы пройти во второй этап участники должны были набрать в сумме по обоим предметам не менее 10 баллов независимо от уровня.

\section{Второй отборочный этап}

Количество баллов, набранных при решении всех задач, суммируется.

Победители второго отборочного этапа должны были набрать 44 балла и выше - для 9 классов и младше, и 75 баллов и выше - для 10-11 классов.

\section{Заключительный этап}

Заключительный этап состоит из командного и индивидуального тура.
\begin{itemize}
    \item Командный тур имеет вес 0.6;
    \item Индивидуальный тур имеет вес 0.4.
\end{itemize}

Командный тур состоит из двух задач:
\begin{itemize}
    \item Задача 1. Наземные испытания – 35 баллов;
    \item Задача 2. Испытания на полигоне – 65 баллов;
\end{itemize}

Итого за командный тур можно заработать 100 баллов. Каждый член команды получает столько баллов, сколько заработала команда.

Индивидуальный тур состоит из решения задач по физике и информатике:
\begin{itemize}
    \item На физике можно было набрать 100 баллов;
    \item На информатике 100 баллов.
\end{itemize}

Чтобы вычислить итоговые баллы, использовалась следующая формула:
$$P = K \cdot 0.6 + (\text{Ф}/2+\text{И}/6) \cdot 0.4$$
где $P$ – итоговые баллы, $K$ – баллы за командный тур, $\text{Ф}$ – баллы за физику, $\text{И}$ –
баллы за информатику.

Победителями становились лучшие 8\% участников (с округлением вниз).

Призерами становились лучшие 17\% участников (с округлением вниз).

Критерий определения победителей и призеров (независимо от уровня):
\begin{center}
    \begin{tabular}{|l|l|}
        \hline
        Категория& Количество баллов\\
        \hline
        Победители&43 и выше\\
        \hline
        Призеры&От 37 до 43 баллов \\
        \hline
    \end{tabular}
\end{center}

