\part{Критерии}
\chapter{Критерии определения победителей и призеров заключительного этапа}

\section{Первый отборочный этап}
 
В первом отборочном этапе участники решали задачи по двум предметам – физика и информатика, в каждом предмете максимально можно было набрать 100 баллов.

Для того, чтобы пройти во второй этап участники 9 класса должны были набрать в сумме по обоим предметам не менее 15 баллов, а участники 10-11 класса должны были набрать в сумме по обоим предметам не менее 16 баллов.

\section{Второй отборочный этап}

Количество баллов, набранных при решении всех задач, суммируется. Возможный максимум - 100 баллов.

Победители второго отборочного этапа должны были набрать 35 баллов и выше.

\section{Заключительный этап}

Определение команд-победителей в финале проводилось при помощи сложения баллов за каждое выполненное или невыполненное задание командой. Команда, набравшая наибольшее количество баллов, становилась победителем. 

Определение победителей и призеров (индивидуальный зачет) проводилось путем сложения нормированных баллов участника за каждый предмет индивидуального тура (химия и биология) с учетом их вклада и командных баллов с учетом соответствующего вклада.

В финале 2018/19 гг. использовались следующие вклады индивидуального и командного зачета: 
\begin{itemize}
    \item 20\% - физика (индивидуальный тур);  
    \item 20\% - информатика (индивидуальный тур);  
    \item 60\%- командные задания. 
\end{itemize}

Таким образом общий балл участника рассчитывался по формуле: 
$$Score = Sn\text{физ} \cdot 0.2 + Sn\text{инф} \cdot 0.2 + Sn\text{ком} \cdot 0.6, \: \text{где}:$$\\
$Sn\text{физ}$ - нормированный балл по физике индивидуального тура;\\
$Sn\text{инф}$ - нормированный балл по информатике индивидуального тура;\\
$Sn\text{ком}$ - нормированный балл командного тура.

Нормировка баллов индивидуальных и командных туров проводилась по формуле: $$Sn = (S / Smax) \cdot 100, \text{где}:$$\\
$S$ - сумма баллов за соответствующий предметный или командный тур;\\
$Smax$ - максимум баллов за предметный или командный тур.

Критерий определения победителей и призеров (независимо от класса):
\begin{center}
    \begin{tabular}{|l|l|}
        \hline
        Категория & Количество баллов \\
        \hline
        Победители & 56.7 и выше \\
        \hline
        Призеры	& От 18.9 до 56.6 баллов \\
        \hline
    \end{tabular}
\end{center}