\part{Критерии}

\chapter{Критерии определения победителей и призеров заключительного этапа}

\section{Первый отборочный этап}
 
В первом отборочном этапе участники решали задачи по трем предметам - математика и информатика, в каждом предмете максимально можно было набрать 100 баллов. Для того, чтобы пройти во второй этап участники должны были набрать в сумме по обоим предметам не менее 50 баллов вне зависимости от уровня.

\section{Второй отборочный этап}

Второй этап представлял собой онлайн-симулятор естественнонаучного исследования и производства (синтез квантовых точек и продуктов на их основе). Задача имела три уровня. Продукты имеют параметр “качество”, максимальный балл можно набрать создавая продукты с качеством 100%. Продукт первого уровня мог максимально дать 1 балл, второго уровня 2 балла, 3 уровня — 4 балла. Еще один балл давался команде с минимальными расходами на производство и исследования

Количество баллов, набранных при решении всех задач, суммируется. Максимально возможное количество баллов за этап — 15.

Победители второго отборочного этапа являются финалистами и должны были набрать 1,72 балла и выше вне зависимости от уровня.

\section{Заключительный этап}

В заключительном этапе олимпиады баллы участника складываются из двух частей: 1 часть: баллы за индивидуальное решение задач по предметам (физика, химия, биология);

2 часть: баллы за командное решение практической задачи в области синтеза квантовых точек (наноразмерных полупроводников) и создания на их основе прототипа дисплея.

Оценки первой части заключительного этапа:
\begin{itemize}
    \item Физика — 100 баллов (максимальный возможный балл за задачи);
    \item Химия — 100 баллов (максимальный возможный балл за задачи).
    \item Биология — 100 баллов (максимальный возможный балл за задачи).
\end{itemize}
 
Командный тур — 300 баллов (максимальный возможный балл за задачи командного тура).

Итоговый балл определяется по формуле: $S = S1 + S2 + S3 + S4$, где

S1 – балл первой части заключительного этапа по физике (S1 макс = 100);

S2 – балл первой части заключительного этапа по химии (S2 макс = 100);

S3 – балл первой части заключительного этапа по биологии (S3 макс = 100);

S4 – итоговый балл командного тура (S3 макс = 300).

Критерий определения победителей и призеров (независимо от уровня):
\begin{center}
    \begin{tabular}{|l|l|}
        \hline
        Категория&Количество баллов\\
        \hline
        Победители&311 баллов и выше\\
        \hline
        Призеры&От 249 до 310 баллов\\
        \hline
    \end{tabular}
\end{center}