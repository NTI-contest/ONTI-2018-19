\part{Критерии}

\chapter{Критерии определения победителей и призеров заключительного этапа}
\section{Первый отборочный этап}

В первом отборочном этапе участники решали задачи по двум предметам - физика и информатика, в каждом предмете максимально можно было набрать 100 баллов. Для того, чтобы пройти во второй этап участники вне зависимости от уровня должны были набрать в сумме по обоим предметам не менее 20 баллов.

\section{Второй отборочный этап}

Количество баллов, набранных при решении всех задач, суммируется. Возможный максимум - 100 баллов. 
Победители второго отборочного этапа должны были набрать 75 баллов и выше.

\section{Заключительный этап}

В заключительном этапе олимпиады баллы участника складываются из двух частей: 1 часть - участник получает баллы за индивидуальное решение задач по предметам (физика, информатика)
2 часть - участник получает баллы за командное решение практической задачи в области проектирования систем управления беспилотными летательными аппаратами (БПЛА).

Оценки первой части заключительного этапа:
Физика - 100 баллов (максимальный возможный балл за задачи);
Информатика - 100 баллов (максимальный возможный балл за задачи).
Командный тур  - максимум 47 баллов, нормировано на 100 баллов (максимальный возможный балл за задачи командного тура).

Итоговый балл определяется по формуле: $S = S_1 \cdot 0.15 + S_2 \cdot 0.15 + S_3 \cdot 0.$7, где:

$S_1$ – балл первой части заключительного этапа по физике ($S_1$ макс = 100);
$S_2$ – балл первой части заключительного этапа по информатике ($S_2$ макс = 100);
$S_3$ – итоговый балл командного тура в стобалльной системе ($S_3$ макс = 100).

Критерий определения победителей и призеров (независимо от класса):

\begin{center}
    \begin{tabular}{|l|l|}
        \hline
        Категория&Количество баллов\\
        \hline
        Победители&74.91 и выше\\
        \hline
        Призеры&От 67.37 до 73.47 баллов\\
        \hline
    \end{tabular}
\end{center}