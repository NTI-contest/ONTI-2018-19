\part{Критерии}

\chapter{Критерии определения победителей и призеров заключительного этапа}
 
\section{Первый отборочный этап}
 
В первом отборочном этапе участники решали задачи по двум предметам - математика и информатика, в каждом предмете максимально можно было набрать 100 баллов. Для того, чтобы пройти во второй этап участники должны были набрать в сумме по обоим предметам:
\begin{itemize}
    \item 9 классы и младше — не менее 80 баллов.
    \item 10-11 класс — не менее 100 баллов.
\end{itemize}

\section{Второй отборочный этап}

Количество баллов, набранных при решении всех задач, суммируется. Призерам второго отборочного этапа было необходимо набрать не менее 40 баллов.

Победители второго отборочного этапа являются финалистами олимпиады и должны были набрать 46 баллов и выше независимо от уровня.

\section{Заключительный этап}

В заключительном этапе олимпиады баллы участника складываются из двух частей: 
\begin{enumerate}
    \item[1 -] участник получает баллы за индивидуальное решение задач по предметам (математика и информатика)
    \item[2 -] участник получает баллы за командное решение практической задачи.
\end{enumerate} 

Итоговый личный балл участника рассчитывается по формуле:
$$S = Si + Sm + St, \: \text{где:}$$
 
$Si$ — сумма баллов по информатике, набранная в рамках индивидуальной части заключительного этапа (максимум 300 баллов), нормированная на 100;

$Sm$ — сумма баллов по математике, набранная в рамках индивидуальной части заключительного этапа (максимум 100 баллов); 

$St$ — количество баллов, набранное в рамках командной части заключительного этапа (максимум 400 баллов).

Критерий определения победителей и призеров:
\begin{center}
    \begin{tabular}{|l|l|}
        \hline
        Категория&Количество баллов\\
        \hline
        Победители&160 баллов и выше\\
        \hline
        Призеры&От 140 до 159 баллов\\
        \hline
    \end{tabular}
\end{center}