\assignementTitle{Brain Damage}{1000}{}

Друг скинул мне этот текст вместе с улыбающимся смайликом. Хорошо зная своего друга, я предполагаю, что для раскрытия загадки этого сообщения нужно напрячь мозги на полную катушку. И хотя я уже потратил много времени на поиски решения, ничего больше странных символов и некоторой структуры в их расположении мне увидеть не удалось. Мне очень бы помогло, если бы вы хотя бы сказали мне, что представляет из себя этот текст...

Файл из задания доступен по ссылке: \url{https://cyberchallenge.rt.ru/files/85b80336a3cd3a1ee35739dab1601ba9/prog.txt}

\solutionSection

Исходный файл – программа на языке Brainfuck. Для запуска программы, можно воспользоваться любым подходящим онлайн сервисом. Видим, что программа выводит фразу «haha no flag here» и зацикливается.

Наиболее простой способ понять, почему же зацикливается программа, это перевести ее в код на другом, более читаемом языке, например, на Python. Для этого можно воспользоваться скриптом \url{https://www.nayuki.io/res/optimizing-brainfuck-compiler/bfc.py}.

В полученной программе на языке Python, бесконечный цикл будет начинаться в строке №68:

\begin{minted}[fontsize=\footnotesize, linenos]{python}
while mem[i] != 0:
    mem[i] = (mem[i] + 2) & 0xFF
    mem[i + 1] = (mem[i + 1] + 0) & 0xFF
    mem[i] = (mem[i] + 254) & 0xFF
\end{minted}

Легко заметить, что в данном цикле, на каждой итерации значение mem[i] меняться не будет, а значит и условие выхода из цикла никогда не выполнится. Все, что нам остается, это убрать этот бесконечный цикл и весь код до него. После запуска, такая программа выведет искомый флаг.

\answerMath{CC\{Br4inf0ck\_ta5k\_num\_57123849\}.}

