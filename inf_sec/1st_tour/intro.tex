Первый отборочный этап проводится индивидуально в сети Интернет, работы оцениваются автоматически средствами системы онлайн-тестирования. Для всех \linebreak участников предлагается единый набор задач в формате CTF (Capture The Flag).

Решение задач предполагает нахождение специальной последовательности символов – флага. Задания распределены по категориям. Задания каждой категории для решения требуют определенных знаний и навыков таких как: понимание  работы операционных систем, веб-сервисов, систем и сетей связи, знание различных форматов файлов, знание алгоритмов и умение реализовывать их на одном из языков программирования, знание низкоуровневых языков программирования, знание основ криптографии и криптоанализа.

Задачи формата CTF не предполагают подробного описание условия задачи, т. к. цель, фактически, всегда одна – найти верный флаг. В описании дается то, где его искать, например, файл или ссылка на веб-сервис. Любое другое описание может либо содержать дополнительную информацию, необходимую для решения задачи, либо служить подсказкой, чтобы задать правильное направление ходу мыслей решающего. Помимо этого, задание относится к одной из категорий, что позволяет понять, какие знания и умения потребуются для его решения. Таким образом, достаточными исходными данными являются категория задачи и данные, в которых необходимо найти флаг.

Участники не были ограничены в выборе языка программирования и программного обеспечения для решения задач. На решение задач первого отборочного этапа участникам давалось 5 дней. Использовалась динамическая система оценивания — количество баллов за задание зависит от числа участников, которые его решили. Таким образом, чем больше участников решило задание, тем меньшей становится его стоимость. При этом время решения заданий учитывается только при ранжировании участников, набравших одинаковое число баллов. Максимальное количество баллов за задание – 1000.