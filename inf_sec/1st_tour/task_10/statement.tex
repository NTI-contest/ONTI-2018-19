\assignementTitle{Image Box}{1000}{}

Стартап ImageBox объявил о запуске нового облачного сервиса для загрузки и хранения изображений. А чтобы загружать было не так скучно, разработчики еще и спрятали флаг на своем сервере. Попробуйте извлечь его.

Ссылка на сервис: \url{http://imagebox.2018.cyberchallenge.ru}.

\solutionSection

Скриншот главной страницы сервиса:

\putImgWOCaption{14cm}{1st_tour/task_10/1}

Заметим, что сервис передает HTTP-заголовок “X-Server” с указанием используемой версией PHP. Предположим, что директория, в которую загружаются пользовательские файлы, из-за некорректных настроек веб-сервера позволяет исполнять также и загруженные сценарии на языке PHP. Попробуем загрузить простейший веб-шелл, позволяющий выполнять произвольные команды оболочки операционной системы на сервере.

\putImgWOCaption{14cm}{1st_tour/task_10/2}

Сохраним его в файле с названием shell.php. При попытке его загрузить, пользователь получит ошибку: “Разрешены только файлы содержащие расширения .jpg или .png!”.  Эту проверку можно попробовать обойти, добавив в название файла второе расширение, например shell.png.php.

Загрузив такой файл, и передав в качестве параметра командной строки cmd команду cat /etc/flag.txt, как указано в описании, получаем искомый флаг.


\putImgWOCaption{14cm}{1st_tour/task_10/3}

\answerMath{CC\{upl0ad\_wis3ly\}.}