\assignementTitle{Porter}{}{}

Веб-сервис расположен на каком-то из портов в диапазоне 1000-2000 на сервере 2018.cyberchallenge.ru. Сможешь ли ты найти его?

\solutionSection

Для того чтобы узнать, какой порт находится в режиме прослушивания, воспользуемся утилитой nmap. Скачать её можно с сайта \url{https://nmap.org/}. 

Воспользовавшись утилитой можно найти сервис на порту 1337.

\putImgWOCaption{14cm}{1st_tour/task_04/1}

Если сделать HTTP-запрос к сервису (например с помощью утилиты httpie \url{https://httpie.org/}), получим следующую страницу:

\putImgWOCaption{14cm}{1st_tour/task_04/2}

Видно, что флаг находится на странице flag.html. Сделав HTTP-запрос на эту страницу, участники получат флаг. Стоит отметить, что использование обычного веб-браузера для этих целей приводит к выполнению JavaScript-кода на странице, который выполняет перенаправление через 500 мс на главную страницу, не позволяя получить таким образом флаг.

\putImgWOCaption{14cm}{1st_tour/task_04/3}

\answerMath{CC\{ra8Zb53uJeA\_this\_was\_a\_trick\}.}