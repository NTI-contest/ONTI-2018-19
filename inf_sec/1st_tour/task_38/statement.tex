\assignementTitle{Brutality}{1000}{}

Мы посчитали MD5-хеш от не очень длинного флага. Получилось 2FBBDC6A5FC\\F7B96B0B1BE4DD33F94A7.

Обратную операцию произвести тоже легко, не правда ли?

\solutionSection

В данном задании предлагается применить брутфорс атаку и подобрать простым перебором флаг, MD5-хеш которого известен. Для этого можно написать либо свою программу, либо воспользоваться уже готовыми утилитами. Одна из них – hashcat. Чтобы подобрать флаг, нужно запустить его со следующими параметрами:
\begin{minted}[fontsize=\footnotesize]{console}
hashcat64.exe -m 0 -a 3 -1 ?l?d_{} --increment 2FBBDC6A5FCF7B96B0B1BE4DD33F94A7 
    CC{?1?1?1?1?1?1?1?1} --force
\end{minted}

Hashcat позволяет задавать маску искомого значения, а также ограничивать множество перебираемых символов. В данном примере мы ограничились набором символов [a-z0-9\_\{\}], а флаг искали по маске: CC\{[максимум 8 символов]\}.

\answerMath{CC\{f0rc3\}.}