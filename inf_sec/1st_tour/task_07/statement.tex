\assignementTitle{JSt Do It}{1000}{}

Не дай своим мечтам остаться мечтами.

Каждый день ты говоришь: «Завтра».

Просто сделай это!

Претвори свои мечты (про кавычки) в жизнь.

Просто сделай это!


Ссылка на сервис: \url{http://jstdoit.2018.cyberchallenge.ru}.

\solutionSection

Сложность данного задания состоит в том, что при открытии главной страницы, пользователь перенаправляется на сервис youtube.com. Чтобы избежать выполнения кода, которые осуществляет это перенаправление, воспользуемся утилитой httpie (\url{https://httpie.org/}).

\putImgWOCaption{16cm}{1st_tour/task_07/1}

На странице действительно исполняется JavaScript код, но он обфусцирован. Поиски данного формата JavaScript приведут к варианту обфускации, основанному на использовании всего 6 символов: ()[]+!

Базовая идея метода основана на использовании следующих “строительных блоков”:

\begin{itemize}
    \item false       =>  ![]
    \item true        =>  !![]
    \item undefined   =>  [][[]]
    \item NaN         =>  +[![]]
    \item 0           =>  +[]
    \item 1           =>  +!+[]
    \item 2           =>  !+[]+!+[]
    \item 10          =>  [+!+[]]+[+[]]
    \item Array       =>  []
    \item Number      =>  +[]
    \item String      =>  []+[]
    \item Boolean     =>  ![]
    \item Function    =>  []["filter"]
    \item eval        =>  []["filter"]["constructor"]( CODE )()
    \item window      =>  []["filter"]["constructor"]("return this")()
\end{itemize}

Прочие, более сложные конструкции, формируются по аналогии.

После ручной или автоматической (с помощью доступных в сети Интернет утилит) деобфускации получаем исходный скрипт.

Первая часть скрипта осуществляет перенаправление:

\putImgWOCaption{16cm}{1st_tour/task_07/2}

Вторая часть осуществляет вывод флага:

\putImgWOCaption{16cm}{1st_tour/task_07/3}

\answerMath{CC\{yTQhjFDMQGo\_never\_gonna\_give\_you\_up\}.}