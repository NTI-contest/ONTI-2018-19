\assignementTitle{Octocat}{1000}{}

Ссылка на сервис: \url{http://octocat.2018.cyberchallenge.ru}.

\solutionSection

Из названия сервиса можно догадаться, что речь идет о системе контроля версий git. Проверить, доступна ли она для скачивания, можно попробовав скачать конфигурационный файл. Воспользуемся для этого утилитой httpie (\url{https://httpie.org}).

\putImgWOCaption{14cm}{1st_tour/task_13/1}

Видно, что конфигурационные файлы действительно присутствуют. Скачать и восстановить всю директорию можно с помочью утилиты \url{https://github.com/internetwache/GitTools}.

\putImgWOCaption{14cm}{1st_tour/task_13/2}

В директории octocat нет никаких файлов, но можно посмотреть историю обновлений:

\putImgWOCaption{14cm}{1st_tour/task_13/3}

В последнем изменении были удалены какие-то важные данные. Чтобы их посмотреть, нужно воспользоваться командой git show.

\putImgWOCaption{14cm}{1st_tour/task_13/4}

\answerMath{CC\{CseK6Eo762c\_octocat\_is\_fuuuuuuuun\}.}