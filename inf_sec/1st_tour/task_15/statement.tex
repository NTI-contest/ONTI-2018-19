\assignementTitle{«The End»}{1000}{}

— Когда человек счастлив, смысл жизни и прочие вечные темы его редко интересуют. Ими следует задаваться в конце жизни.

— А когда наступит этот конец — мы же не знаем, вот и торопимся.

— А ты не торопись — самые счастливые люди те, кто никогда не задавался этими проклятыми вопросами.

«Солярис», 1972

Может быть, конец — не то, чем кажется?

Файл из задания доступен по ссылке: \url{https://cyberchallenge.rt.ru/files/de5a262059ad928737f89913f43af994/EnD.jpeg}.

Данный файл является изображением в формате JPEG. Так как описание задания намекает нам заглянуть в конец файла, для начала, проверим, нет ли там каких-либо дополнительных данных. Для этого можно воспользоваться любым шестнадцатеричным редактором, например, Hiew. Согласно спецификации формата JPEG, данный тип файлов должен заканчиваться байтами FF D9. Данные байты можно обнаружить в файле по смещению 0x190C1. И, как видно на рисунке 1, следом за сигнатурой конца изображения начинается сигнатура RAR-архива (52 61 72 21 1A 07 01 00).

\putImgWOCaption{16cm}{1st_tour/task_15/1}

\begin{center}
    Рис. 1. Конец файла EnD.jpeg
\end{center}

Вырежем RAR-архив из JPEG-файла и поместим его в отдельный файл с расширением «.rar». Далее, откроем его с помощью программы WinRar. Видим, что архив содержит файл flag.txt, но он защищен паролем. Пароль можно увидеть в левом верхнем углу исходного изображения. Подставим пароль, распакуем flag.txt и откроем его любым текстовым редактором, чтобы получить флаг.

\putImgWOCaption{10cm}{1st_tour/task_15/2}

\answerMath{CC\{1\_h4v3\_n3v3r\_533n\_r4r\_jp36\_b3f0r3\}.}