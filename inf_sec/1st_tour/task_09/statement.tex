\assignementTitle{Another Day}{1000}{}

Поиск уязвимостей в языках программирования является очень интересной темой для исследований в области информационной безопасности. Позвольте мне показать вам один пример...

Ссылка на сервис: \url{http://anotherday.2018.cyberchallenge.ru}.

\solutionSection

Перебор стандартных комбинаций логин-пароль в этом сервисе ничего не даст. SQL-инъекции также не дадут результата. Также можно попробовать NoSQL-инъекцию. Для этого нужно перехватить HTTP-запрос с помощью утилиты BurpSuite (\url{https://portswigger.net/burp}). Инструкции по настройке можно найти на сайте.

Перехваченный запрос может выглядеть так:

\putImgWOCaption{14cm}{1st_tour/task_09/1}

Для того чтобы иметь возможность менять параметры, воспользуемся утилитой Repeater, которая входит в состав программы BurpSuite. Для этого нужно нажать правой клавишей мыши в любой части запроса и выбрать пункт меню “Send to Intruder”.

\putImgWOCaption{16cm}{1st_tour/task_09/2}

Для того, чтобы осуществить NoSQL инъекцию, нужно модифицировать поле password. Нужно изменить его на password[\$ne]. Сделать это можно прямо в редакторе запроса BurpSuite.

В ответ на модифицированный запрос сервис отправит флаг:

\putImgWOCaption{16cm}{1st_tour/task_09/3}

\answerMath{CC\{Am\_I\_great\_bug\_or\_what?\}.}