\assignementTitle{Cookie Monster}{1000}{}

При посещении вами данного веб-сайта сбор информации может осуществляться посредством cookie-файлов и других технологий. Используя данный веб-сайт, вы даете свое согласие на использование нами cookie-файлов в соответствии с данными Условиями...

Ссылка на сервис: \url{http://cookiemonster.2018.cyberchallenge.ru}

\solutionSection

На главной странице сайта нет ничего, с чем можно взаимодействовать, значит, клиент взаимодействует с сервером как-то иначе. Из названия задания можно понять, что нужно обратить внимание на cookie файлы. Чтобы их посмотреть, нужно зайти в меню разработчика браузера.

Нажать правой кнопкой мыши в любом месте страницы, выбрать пункт меню “Inspect”, выбрать вкладку “Application”, в этой вкладке найти пункт меню “Cookies”. Видим, что для данного сайта сервер выставляет cookie “role” со значением “user”.

\putImgWOCaption{16cm}{1st_tour/task_03/1}

Возможно, контроль доступа осуществляется с использованием cookie файлов. Попробуем перебрать другие возможные роли: “admin”, “administrator”, “root”. Сделать это можно двойным нажатием по значению cookie файла, переписав данное значение и перезагрузив страницу.

При использовании роли “admin”, видим на главной странице флаг.

\putImgWOCaption{8cm}{1st_tour/task_03/2}

\answerMath{CC\{y4dC3raadSI\_yay\_i\_like\_cookies\}.}