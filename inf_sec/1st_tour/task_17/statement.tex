\assignementTitle{Beauty \& Beast}{1000}{}

На первый взгляд, это самое прекрасное из того, что создало человечество. Но я уверен — внутри таится чудовище.

Файл из задания доступен по ссылке: \url{https://cyberchallenge.rt.ru/files/006f462c76a5b8c5d99df663487fe8df/so_pretty.wav}

\solutionSection

Исходный файл представляет из себя аудиозапись песни Рика Эстли “Never Gonna Give You Up” в формате WAV. Так как никаких подсказок описание не содержит, приходится перебирать все варианты от простого к сложному. К счастью, не так много известных стеганографических программ умеют работать с файлами в формате WAV. Самая популярная из них – Steghide.

Steghide использует пароль, введенный пользователем, чтобы сгенерировать из него ключ шифрования для AES-128, с помощью которого зашифровываются данные перед их сокрытием в WAV-файле. Пароль не сложно угадать, вспомнив, что за композиция содержится в WAV-файле. Итак, пароль – rickroll.

\putImgWOCaption{14cm}{1st_tour/task_17/1}

Таким образом, используя steghide и угаданный пароль, можно посмотреть, что за данные сокрыты в WAV-файле. Оказывается, что это изображение в формате PNG. Достать изображение из аудиофайла можно командой:\\
steghide extract –p rickroll –sf so\_pretty.wav

Ни на изображении, ни в его метаданных не видно никаких подсказок к дальнейшим действиям. Однако у PNG-файлов есть интересная особенность. Размер изображения указан в заголовке файла, и все программы-просмотрщики будут четко следовать написанному там. Получается, что так можно спрятать часть изображения, просто изменив его размер в заголовке. Проверим, нет ли чего-то за пределами отображаемого изображения в нашем случае. Для этого в PNG-файле по смещению 0x16 изменим значение высоты изображения (0x2d0) на, например, в 2 раза большее. Открыв измененный файл, увидим внизу странную надпись: 
Ao(mg+D,P4+EV:2F!,R5F)*fZ6U\\Q2X2`!?KDD5F71cKH\#0K!NZ1gb!?2)elS0QLNG0Qhd

По используемому алфавиту, можно догадаться, что это данные в кодировке Base85. Декодировать данные можно, например, онлайн сервисом CyberChef, получив в результате фразу:
\mint{console}|flag for this task: CC{57364n0_m47ry05hk4_ju57_f0r_y0u}|

\answerMath{CC\{57364n0\_m47ry05hk4\_ju57\_f0r\_y0u\}.}
