\assignementTitle{Time Loop}{1000}{}

Если вы сможете остановить время, то я отдам вам флаг.
Обратите внимание, что данный исполняемый файл предназначен для запуска под операционной системой семейства GNU/Linux.

Файл из задания доступен по ссылке: \url{https://cyberchallenge.rt.ru/files/c76babe6e494a29236bcc21432b020b6/timeloop}.

\solutionSection

Нам дан исполняемый файл в формате ELF. После запуска программа выводит 2 строки:

«Hi! I will give you the flag if you show me that you can stop the time! but you can't...»

Дизассемблировав файл, увидим в функции main следующее условие, невыполнение которого приводит к появлению фразы «but you can’t»:

\putImgWOCaption{8cm}{1st_tour/task_22/1}

Данная проверка осуществляется 9 раз в цикле. Если все 9 раз результат вызова системной функции time был равен 123456, то программа расшифровывает флаг, используя алгоритм RC4 и заданный в теле программы ключ.

Есть несколько вариантов решения задания. Вот самые очевидные из них:

\begin{enumerate}
        \item Подменить системную функцию time своей собственной, используя переменную окружения LD\_PRELOAD;
        \item Разобрать в дизассемблированном коде, как формируется ключ шифрования и написать программу расшифровывающую флаг;
        \item Используя отладчик, обойти проверку и получить флаг. Тут важно учесть, что переменная, в которую кладется возвращаемое функцией time значение, используется в формировании флага. Поэтому, после обхода проверки, нужно установить её значение равным 123456.
\end{enumerate}

\answerMath{CC\{Can\_you\_really\_stop\_the\_time?\}.}