Проход морских судов по акватории Арктики играет важную роль в экономике приарктических государств, в т.ч. России, Норвегии, Швеции, Канады и США. Оперативное наблюдение ледовой обстановки не только способствует безопасности судоходства, но и позволяет решать задачи точного планирования отгрузок.

\putImgWOCaption{10cm}{1}

Контроль движения судов и мониторинг ледовой обстановки осуществляется, в том числе, следующими инструментами:
\begin{itemize}
    \item AIS (Аutomatic Identification System) - международная система определения положения судов, состоящая из бортовых станций на кораблях, наземных опорных станций и станций космического базирования.
    \item Данными космической съемки спутниками ДЗЗ, в т.ч.
    \begin{itemize}
        \item Данные радарной съемки (в т.ч. например канадского RadarSat-2)
        \item Обзорные данные с оптических мультиспектральных камер низкого разрешения (в т. ч. например аппаратов Aqua и Terra)
        \item Данные с оптических камер (в первую очередь видимого и инфракрасного диапазона) высокого разрешения.
    \end{itemize}
\end{itemize}

Участникам предлагаются компоненты задачи разработки проекта созвездия  спутников ДЗЗ с оптическими и мультиспектральными камерами, предназначенным для мониторинга ледовой обстановки Северного Морского Пути.


\section{Категория Web}

Задания данной категории предполагают поиск уязвимостей веб-приложения (веб-сайта) и дальнейшую их эксплуатацию.

Для этого необходимо знать основы разработки веб-приложений, понимать базовые принципы их проектирования, иметь представление о том, как работает тот или иной веб-фреймворк или CMS. Кроме того, важно понимать природу возникновения веб-уязвимостей, понимать и уметь эксплуатировать типовые уязвимости в веб-приложениях. Список наиболее распространенных веб-уязвимостей периодически публикуется в рамках проекта OWASP Top 10.

\assignementTitle{Печеньки}{10}

У Ильнара есть три коробки с печеньем, в которых $a$, $b$, $c$ печенек соответственно. К Ильнару в гости пришло $(n - 1)$ человек. Ильнар хочет, чтобы всем гостям и ему досталось одинаковое количество печенек. Поэтому он хочет узнать сколько печенек достанется каждому, если он откроет несколько (возможно ни одной, возможно все три) коробок с печеньем, при этом все печеньки из открытых коробок должны быть розданы поровну (сами печенья нельзя ломать на части).

Определите максимальное количество печенек, которое получит каждый гость и Ильнар.

\inputfmtSection

В первой строке вводится $4$ целых числа $a, b, c, n$ $(1 \le a, b, c, n \le 10^8)$ .

\outputfmtSection

Выведите единственное число -- ответ на задачу.

\exampleSection

\sampleTitle{1}

\begin{myverbbox}[\small]{\vinput}
1 2 3 4
\end{myverbbox}
\begin{myverbbox}[\small]{\voutput}
1
\end{myverbbox}
\inputoutputTable

\sampleTitle{2}

\begin{myverbbox}[\small]{\vinput}
3 4 5 19
\end{myverbbox}
\begin{myverbbox}[\small]{\voutput}
0
\end{myverbbox}
\inputoutputTable

\includeSolutionIfExistsByPath{1st_tour_progr/task_031}
\assignementTitle{Печеньки}{10}

У Ильнара есть три коробки с печеньем, в которых $a$, $b$, $c$ печенек соответственно. К Ильнару в гости пришло $(n - 1)$ человек. Ильнар хочет, чтобы всем гостям и ему досталось одинаковое количество печенек. Поэтому он хочет узнать сколько печенек достанется каждому, если он откроет несколько (возможно ни одной, возможно все три) коробок с печеньем, при этом все печеньки из открытых коробок должны быть розданы поровну (сами печенья нельзя ломать на части).

Определите максимальное количество печенек, которое получит каждый гость и Ильнар.

\inputfmtSection

В первой строке вводится $4$ целых числа $a, b, c, n$ $(1 \le a, b, c, n \le 10^8)$ .

\outputfmtSection

Выведите единственное число -- ответ на задачу.

\exampleSection

\sampleTitle{1}

\begin{myverbbox}[\small]{\vinput}
1 2 3 4
\end{myverbbox}
\begin{myverbbox}[\small]{\voutput}
1
\end{myverbbox}
\inputoutputTable

\sampleTitle{2}

\begin{myverbbox}[\small]{\vinput}
3 4 5 19
\end{myverbbox}
\begin{myverbbox}[\small]{\voutput}
0
\end{myverbbox}
\inputoutputTable

\includeSolutionIfExistsByPath{1st_tour_progr/task_031}
\assignementTitle{Печеньки}{10}

У Ильнара есть три коробки с печеньем, в которых $a$, $b$, $c$ печенек соответственно. К Ильнару в гости пришло $(n - 1)$ человек. Ильнар хочет, чтобы всем гостям и ему досталось одинаковое количество печенек. Поэтому он хочет узнать сколько печенек достанется каждому, если он откроет несколько (возможно ни одной, возможно все три) коробок с печеньем, при этом все печеньки из открытых коробок должны быть розданы поровну (сами печенья нельзя ломать на части).

Определите максимальное количество печенек, которое получит каждый гость и Ильнар.

\inputfmtSection

В первой строке вводится $4$ целых числа $a, b, c, n$ $(1 \le a, b, c, n \le 10^8)$ .

\outputfmtSection

Выведите единственное число -- ответ на задачу.

\exampleSection

\sampleTitle{1}

\begin{myverbbox}[\small]{\vinput}
1 2 3 4
\end{myverbbox}
\begin{myverbbox}[\small]{\voutput}
1
\end{myverbbox}
\inputoutputTable

\sampleTitle{2}

\begin{myverbbox}[\small]{\vinput}
3 4 5 19
\end{myverbbox}
\begin{myverbbox}[\small]{\voutput}
0
\end{myverbbox}
\inputoutputTable

\includeSolutionIfExistsByPath{1st_tour_progr/task_031}
\assignementTitle{Печеньки}{10}

У Ильнара есть три коробки с печеньем, в которых $a$, $b$, $c$ печенек соответственно. К Ильнару в гости пришло $(n - 1)$ человек. Ильнар хочет, чтобы всем гостям и ему досталось одинаковое количество печенек. Поэтому он хочет узнать сколько печенек достанется каждому, если он откроет несколько (возможно ни одной, возможно все три) коробок с печеньем, при этом все печеньки из открытых коробок должны быть розданы поровну (сами печенья нельзя ломать на части).

Определите максимальное количество печенек, которое получит каждый гость и Ильнар.

\inputfmtSection

В первой строке вводится $4$ целых числа $a, b, c, n$ $(1 \le a, b, c, n \le 10^8)$ .

\outputfmtSection

Выведите единственное число -- ответ на задачу.

\exampleSection

\sampleTitle{1}

\begin{myverbbox}[\small]{\vinput}
1 2 3 4
\end{myverbbox}
\begin{myverbbox}[\small]{\voutput}
1
\end{myverbbox}
\inputoutputTable

\sampleTitle{2}

\begin{myverbbox}[\small]{\vinput}
3 4 5 19
\end{myverbbox}
\begin{myverbbox}[\small]{\voutput}
0
\end{myverbbox}
\inputoutputTable

\includeSolutionIfExistsByPath{1st_tour_progr/task_031}
\assignementTitle{Печеньки}{10}

У Ильнара есть три коробки с печеньем, в которых $a$, $b$, $c$ печенек соответственно. К Ильнару в гости пришло $(n - 1)$ человек. Ильнар хочет, чтобы всем гостям и ему досталось одинаковое количество печенек. Поэтому он хочет узнать сколько печенек достанется каждому, если он откроет несколько (возможно ни одной, возможно все три) коробок с печеньем, при этом все печеньки из открытых коробок должны быть розданы поровну (сами печенья нельзя ломать на части).

Определите максимальное количество печенек, которое получит каждый гость и Ильнар.

\inputfmtSection

В первой строке вводится $4$ целых числа $a, b, c, n$ $(1 \le a, b, c, n \le 10^8)$ .

\outputfmtSection

Выведите единственное число -- ответ на задачу.

\exampleSection

\sampleTitle{1}

\begin{myverbbox}[\small]{\vinput}
1 2 3 4
\end{myverbbox}
\begin{myverbbox}[\small]{\voutput}
1
\end{myverbbox}
\inputoutputTable

\sampleTitle{2}

\begin{myverbbox}[\small]{\vinput}
3 4 5 19
\end{myverbbox}
\begin{myverbbox}[\small]{\voutput}
0
\end{myverbbox}
\inputoutputTable

\includeSolutionIfExistsByPath{1st_tour_progr/task_031}
\assignementTitle{Печеньки}{10}

У Ильнара есть три коробки с печеньем, в которых $a$, $b$, $c$ печенек соответственно. К Ильнару в гости пришло $(n - 1)$ человек. Ильнар хочет, чтобы всем гостям и ему досталось одинаковое количество печенек. Поэтому он хочет узнать сколько печенек достанется каждому, если он откроет несколько (возможно ни одной, возможно все три) коробок с печеньем, при этом все печеньки из открытых коробок должны быть розданы поровну (сами печенья нельзя ломать на части).

Определите максимальное количество печенек, которое получит каждый гость и Ильнар.

\inputfmtSection

В первой строке вводится $4$ целых числа $a, b, c, n$ $(1 \le a, b, c, n \le 10^8)$ .

\outputfmtSection

Выведите единственное число -- ответ на задачу.

\exampleSection

\sampleTitle{1}

\begin{myverbbox}[\small]{\vinput}
1 2 3 4
\end{myverbbox}
\begin{myverbbox}[\small]{\voutput}
1
\end{myverbbox}
\inputoutputTable

\sampleTitle{2}

\begin{myverbbox}[\small]{\vinput}
3 4 5 19
\end{myverbbox}
\begin{myverbbox}[\small]{\voutput}
0
\end{myverbbox}
\inputoutputTable

\includeSolutionIfExistsByPath{1st_tour_progr/task_031}
\assignementTitle{Печеньки}{10}

У Ильнара есть три коробки с печеньем, в которых $a$, $b$, $c$ печенек соответственно. К Ильнару в гости пришло $(n - 1)$ человек. Ильнар хочет, чтобы всем гостям и ему досталось одинаковое количество печенек. Поэтому он хочет узнать сколько печенек достанется каждому, если он откроет несколько (возможно ни одной, возможно все три) коробок с печеньем, при этом все печеньки из открытых коробок должны быть розданы поровну (сами печенья нельзя ломать на части).

Определите максимальное количество печенек, которое получит каждый гость и Ильнар.

\inputfmtSection

В первой строке вводится $4$ целых числа $a, b, c, n$ $(1 \le a, b, c, n \le 10^8)$ .

\outputfmtSection

Выведите единственное число -- ответ на задачу.

\exampleSection

\sampleTitle{1}

\begin{myverbbox}[\small]{\vinput}
1 2 3 4
\end{myverbbox}
\begin{myverbbox}[\small]{\voutput}
1
\end{myverbbox}
\inputoutputTable

\sampleTitle{2}

\begin{myverbbox}[\small]{\vinput}
3 4 5 19
\end{myverbbox}
\begin{myverbbox}[\small]{\voutput}
0
\end{myverbbox}
\inputoutputTable

\includeSolutionIfExistsByPath{1st_tour_progr/task_031}
\assignementTitle{Печеньки}{10}

У Ильнара есть три коробки с печеньем, в которых $a$, $b$, $c$ печенек соответственно. К Ильнару в гости пришло $(n - 1)$ человек. Ильнар хочет, чтобы всем гостям и ему досталось одинаковое количество печенек. Поэтому он хочет узнать сколько печенек достанется каждому, если он откроет несколько (возможно ни одной, возможно все три) коробок с печеньем, при этом все печеньки из открытых коробок должны быть розданы поровну (сами печенья нельзя ломать на части).

Определите максимальное количество печенек, которое получит каждый гость и Ильнар.

\inputfmtSection

В первой строке вводится $4$ целых числа $a, b, c, n$ $(1 \le a, b, c, n \le 10^8)$ .

\outputfmtSection

Выведите единственное число -- ответ на задачу.

\exampleSection

\sampleTitle{1}

\begin{myverbbox}[\small]{\vinput}
1 2 3 4
\end{myverbbox}
\begin{myverbbox}[\small]{\voutput}
1
\end{myverbbox}
\inputoutputTable

\sampleTitle{2}

\begin{myverbbox}[\small]{\vinput}
3 4 5 19
\end{myverbbox}
\begin{myverbbox}[\small]{\voutput}
0
\end{myverbbox}
\inputoutputTable

\includeSolutionIfExistsByPath{1st_tour_progr/task_031}
\assignementTitle{Печеньки}{10}

У Ильнара есть три коробки с печеньем, в которых $a$, $b$, $c$ печенек соответственно. К Ильнару в гости пришло $(n - 1)$ человек. Ильнар хочет, чтобы всем гостям и ему досталось одинаковое количество печенек. Поэтому он хочет узнать сколько печенек достанется каждому, если он откроет несколько (возможно ни одной, возможно все три) коробок с печеньем, при этом все печеньки из открытых коробок должны быть розданы поровну (сами печенья нельзя ломать на части).

Определите максимальное количество печенек, которое получит каждый гость и Ильнар.

\inputfmtSection

В первой строке вводится $4$ целых числа $a, b, c, n$ $(1 \le a, b, c, n \le 10^8)$ .

\outputfmtSection

Выведите единственное число -- ответ на задачу.

\exampleSection

\sampleTitle{1}

\begin{myverbbox}[\small]{\vinput}
1 2 3 4
\end{myverbbox}
\begin{myverbbox}[\small]{\voutput}
1
\end{myverbbox}
\inputoutputTable

\sampleTitle{2}

\begin{myverbbox}[\small]{\vinput}
3 4 5 19
\end{myverbbox}
\begin{myverbbox}[\small]{\voutput}
0
\end{myverbbox}
\inputoutputTable

\includeSolutionIfExistsByPath{1st_tour_progr/task_031}
\assignementTitle{Печеньки}{10}

У Ильнара есть три коробки с печеньем, в которых $a$, $b$, $c$ печенек соответственно. К Ильнару в гости пришло $(n - 1)$ человек. Ильнар хочет, чтобы всем гостям и ему досталось одинаковое количество печенек. Поэтому он хочет узнать сколько печенек достанется каждому, если он откроет несколько (возможно ни одной, возможно все три) коробок с печеньем, при этом все печеньки из открытых коробок должны быть розданы поровну (сами печенья нельзя ломать на части).

Определите максимальное количество печенек, которое получит каждый гость и Ильнар.

\inputfmtSection

В первой строке вводится $4$ целых числа $a, b, c, n$ $(1 \le a, b, c, n \le 10^8)$ .

\outputfmtSection

Выведите единственное число -- ответ на задачу.

\exampleSection

\sampleTitle{1}

\begin{myverbbox}[\small]{\vinput}
1 2 3 4
\end{myverbbox}
\begin{myverbbox}[\small]{\voutput}
1
\end{myverbbox}
\inputoutputTable

\sampleTitle{2}

\begin{myverbbox}[\small]{\vinput}
3 4 5 19
\end{myverbbox}
\begin{myverbbox}[\small]{\voutput}
0
\end{myverbbox}
\inputoutputTable

\includeSolutionIfExistsByPath{1st_tour_progr/task_031}
\assignementTitle{Печеньки}{10}

У Ильнара есть три коробки с печеньем, в которых $a$, $b$, $c$ печенек соответственно. К Ильнару в гости пришло $(n - 1)$ человек. Ильнар хочет, чтобы всем гостям и ему досталось одинаковое количество печенек. Поэтому он хочет узнать сколько печенек достанется каждому, если он откроет несколько (возможно ни одной, возможно все три) коробок с печеньем, при этом все печеньки из открытых коробок должны быть розданы поровну (сами печенья нельзя ломать на части).

Определите максимальное количество печенек, которое получит каждый гость и Ильнар.

\inputfmtSection

В первой строке вводится $4$ целых числа $a, b, c, n$ $(1 \le a, b, c, n \le 10^8)$ .

\outputfmtSection

Выведите единственное число -- ответ на задачу.

\exampleSection

\sampleTitle{1}

\begin{myverbbox}[\small]{\vinput}
1 2 3 4
\end{myverbbox}
\begin{myverbbox}[\small]{\voutput}
1
\end{myverbbox}
\inputoutputTable

\sampleTitle{2}

\begin{myverbbox}[\small]{\vinput}
3 4 5 19
\end{myverbbox}
\begin{myverbbox}[\small]{\voutput}
0
\end{myverbbox}
\inputoutputTable

\includeSolutionIfExistsByPath{1st_tour_progr/task_031}
\assignementTitle{Печеньки}{10}

У Ильнара есть три коробки с печеньем, в которых $a$, $b$, $c$ печенек соответственно. К Ильнару в гости пришло $(n - 1)$ человек. Ильнар хочет, чтобы всем гостям и ему досталось одинаковое количество печенек. Поэтому он хочет узнать сколько печенек достанется каждому, если он откроет несколько (возможно ни одной, возможно все три) коробок с печеньем, при этом все печеньки из открытых коробок должны быть розданы поровну (сами печенья нельзя ломать на части).

Определите максимальное количество печенек, которое получит каждый гость и Ильнар.

\inputfmtSection

В первой строке вводится $4$ целых числа $a, b, c, n$ $(1 \le a, b, c, n \le 10^8)$ .

\outputfmtSection

Выведите единственное число -- ответ на задачу.

\exampleSection

\sampleTitle{1}

\begin{myverbbox}[\small]{\vinput}
1 2 3 4
\end{myverbbox}
\begin{myverbbox}[\small]{\voutput}
1
\end{myverbbox}
\inputoutputTable

\sampleTitle{2}

\begin{myverbbox}[\small]{\vinput}
3 4 5 19
\end{myverbbox}
\begin{myverbbox}[\small]{\voutput}
0
\end{myverbbox}
\inputoutputTable

\includeSolutionIfExistsByPath{1st_tour_progr/task_031}
\assignementTitle{Печеньки}{10}

У Ильнара есть три коробки с печеньем, в которых $a$, $b$, $c$ печенек соответственно. К Ильнару в гости пришло $(n - 1)$ человек. Ильнар хочет, чтобы всем гостям и ему досталось одинаковое количество печенек. Поэтому он хочет узнать сколько печенек достанется каждому, если он откроет несколько (возможно ни одной, возможно все три) коробок с печеньем, при этом все печеньки из открытых коробок должны быть розданы поровну (сами печенья нельзя ломать на части).

Определите максимальное количество печенек, которое получит каждый гость и Ильнар.

\inputfmtSection

В первой строке вводится $4$ целых числа $a, b, c, n$ $(1 \le a, b, c, n \le 10^8)$ .

\outputfmtSection

Выведите единственное число -- ответ на задачу.

\exampleSection

\sampleTitle{1}

\begin{myverbbox}[\small]{\vinput}
1 2 3 4
\end{myverbbox}
\begin{myverbbox}[\small]{\voutput}
1
\end{myverbbox}
\inputoutputTable

\sampleTitle{2}

\begin{myverbbox}[\small]{\vinput}
3 4 5 19
\end{myverbbox}
\begin{myverbbox}[\small]{\voutput}
0
\end{myverbbox}
\inputoutputTable

\includeSolutionIfExistsByPath{1st_tour_progr/task_031}
\assignementTitle{Печеньки}{10}

У Ильнара есть три коробки с печеньем, в которых $a$, $b$, $c$ печенек соответственно. К Ильнару в гости пришло $(n - 1)$ человек. Ильнар хочет, чтобы всем гостям и ему досталось одинаковое количество печенек. Поэтому он хочет узнать сколько печенек достанется каждому, если он откроет несколько (возможно ни одной, возможно все три) коробок с печеньем, при этом все печеньки из открытых коробок должны быть розданы поровну (сами печенья нельзя ломать на части).

Определите максимальное количество печенек, которое получит каждый гость и Ильнар.

\inputfmtSection

В первой строке вводится $4$ целых числа $a, b, c, n$ $(1 \le a, b, c, n \le 10^8)$ .

\outputfmtSection

Выведите единственное число -- ответ на задачу.

\exampleSection

\sampleTitle{1}

\begin{myverbbox}[\small]{\vinput}
1 2 3 4
\end{myverbbox}
\begin{myverbbox}[\small]{\voutput}
1
\end{myverbbox}
\inputoutputTable

\sampleTitle{2}

\begin{myverbbox}[\small]{\vinput}
3 4 5 19
\end{myverbbox}
\begin{myverbbox}[\small]{\voutput}
0
\end{myverbbox}
\inputoutputTable

\includeSolutionIfExistsByPath{1st_tour_progr/task_031}

\section{Категория Stegano}

В заданиях данной категории, в некотором объеме данных (картинки, видео, аудиофайл и т.д.) предлагается найти информацию, спрятанную в нём таким образом, что на первый взгляд ничего особенного в данных файлах нет. Чтобы решить задание этой категории, необходимо понять и обнаружить, каким именно образом была спрятана информация.

\assignementTitle{Печеньки}{10}

У Ильнара есть три коробки с печеньем, в которых $a$, $b$, $c$ печенек соответственно. К Ильнару в гости пришло $(n - 1)$ человек. Ильнар хочет, чтобы всем гостям и ему досталось одинаковое количество печенек. Поэтому он хочет узнать сколько печенек достанется каждому, если он откроет несколько (возможно ни одной, возможно все три) коробок с печеньем, при этом все печеньки из открытых коробок должны быть розданы поровну (сами печенья нельзя ломать на части).

Определите максимальное количество печенек, которое получит каждый гость и Ильнар.

\inputfmtSection

В первой строке вводится $4$ целых числа $a, b, c, n$ $(1 \le a, b, c, n \le 10^8)$ .

\outputfmtSection

Выведите единственное число -- ответ на задачу.

\exampleSection

\sampleTitle{1}

\begin{myverbbox}[\small]{\vinput}
1 2 3 4
\end{myverbbox}
\begin{myverbbox}[\small]{\voutput}
1
\end{myverbbox}
\inputoutputTable

\sampleTitle{2}

\begin{myverbbox}[\small]{\vinput}
3 4 5 19
\end{myverbbox}
\begin{myverbbox}[\small]{\voutput}
0
\end{myverbbox}
\inputoutputTable

\includeSolutionIfExistsByPath{1st_tour_progr/task_031}
\assignementTitle{Печеньки}{10}

У Ильнара есть три коробки с печеньем, в которых $a$, $b$, $c$ печенек соответственно. К Ильнару в гости пришло $(n - 1)$ человек. Ильнар хочет, чтобы всем гостям и ему досталось одинаковое количество печенек. Поэтому он хочет узнать сколько печенек достанется каждому, если он откроет несколько (возможно ни одной, возможно все три) коробок с печеньем, при этом все печеньки из открытых коробок должны быть розданы поровну (сами печенья нельзя ломать на части).

Определите максимальное количество печенек, которое получит каждый гость и Ильнар.

\inputfmtSection

В первой строке вводится $4$ целых числа $a, b, c, n$ $(1 \le a, b, c, n \le 10^8)$ .

\outputfmtSection

Выведите единственное число -- ответ на задачу.

\exampleSection

\sampleTitle{1}

\begin{myverbbox}[\small]{\vinput}
1 2 3 4
\end{myverbbox}
\begin{myverbbox}[\small]{\voutput}
1
\end{myverbbox}
\inputoutputTable

\sampleTitle{2}

\begin{myverbbox}[\small]{\vinput}
3 4 5 19
\end{myverbbox}
\begin{myverbbox}[\small]{\voutput}
0
\end{myverbbox}
\inputoutputTable

\includeSolutionIfExistsByPath{1st_tour_progr/task_031}
\assignementTitle{Печеньки}{10}

У Ильнара есть три коробки с печеньем, в которых $a$, $b$, $c$ печенек соответственно. К Ильнару в гости пришло $(n - 1)$ человек. Ильнар хочет, чтобы всем гостям и ему досталось одинаковое количество печенек. Поэтому он хочет узнать сколько печенек достанется каждому, если он откроет несколько (возможно ни одной, возможно все три) коробок с печеньем, при этом все печеньки из открытых коробок должны быть розданы поровну (сами печенья нельзя ломать на части).

Определите максимальное количество печенек, которое получит каждый гость и Ильнар.

\inputfmtSection

В первой строке вводится $4$ целых числа $a, b, c, n$ $(1 \le a, b, c, n \le 10^8)$ .

\outputfmtSection

Выведите единственное число -- ответ на задачу.

\exampleSection

\sampleTitle{1}

\begin{myverbbox}[\small]{\vinput}
1 2 3 4
\end{myverbbox}
\begin{myverbbox}[\small]{\voutput}
1
\end{myverbbox}
\inputoutputTable

\sampleTitle{2}

\begin{myverbbox}[\small]{\vinput}
3 4 5 19
\end{myverbbox}
\begin{myverbbox}[\small]{\voutput}
0
\end{myverbbox}
\inputoutputTable

\includeSolutionIfExistsByPath{1st_tour_progr/task_031}

\section{Категория Reverse}

Задания, для решения которых необходимо уметь внимательно и аккуратно вникать в логику работы программы, чаще всего, не имея её исходного кода. Но случается, что участникам даётся обфусцированный исходный код, разобрать работу которого задача также не из самых простых. Форматы файлов могут быть самыми различными: PE, ELF, Mach-O, APK или даже просто байт-код программы для некоей виртуальной машины, спецификация которой дается участникам.

В заданиях этой категории обычно флаг спрятан где-то внутри программы в зашифрованном виде. И либо необходимо восстановить и переписать алгоритм шифрования, чтобы расшифровать флаг, либо выполнить некоторые действия (например, записать определенное значение в реестр), чтобы программа в дальнейшем сама расшифровала и вывела на экран флаг.

\assignementTitle{Печеньки}{10}

У Ильнара есть три коробки с печеньем, в которых $a$, $b$, $c$ печенек соответственно. К Ильнару в гости пришло $(n - 1)$ человек. Ильнар хочет, чтобы всем гостям и ему досталось одинаковое количество печенек. Поэтому он хочет узнать сколько печенек достанется каждому, если он откроет несколько (возможно ни одной, возможно все три) коробок с печеньем, при этом все печеньки из открытых коробок должны быть розданы поровну (сами печенья нельзя ломать на части).

Определите максимальное количество печенек, которое получит каждый гость и Ильнар.

\inputfmtSection

В первой строке вводится $4$ целых числа $a, b, c, n$ $(1 \le a, b, c, n \le 10^8)$ .

\outputfmtSection

Выведите единственное число -- ответ на задачу.

\exampleSection

\sampleTitle{1}

\begin{myverbbox}[\small]{\vinput}
1 2 3 4
\end{myverbbox}
\begin{myverbbox}[\small]{\voutput}
1
\end{myverbbox}
\inputoutputTable

\sampleTitle{2}

\begin{myverbbox}[\small]{\vinput}
3 4 5 19
\end{myverbbox}
\begin{myverbbox}[\small]{\voutput}
0
\end{myverbbox}
\inputoutputTable

\includeSolutionIfExistsByPath{1st_tour_progr/task_031}
\assignementTitle{Печеньки}{10}

У Ильнара есть три коробки с печеньем, в которых $a$, $b$, $c$ печенек соответственно. К Ильнару в гости пришло $(n - 1)$ человек. Ильнар хочет, чтобы всем гостям и ему досталось одинаковое количество печенек. Поэтому он хочет узнать сколько печенек достанется каждому, если он откроет несколько (возможно ни одной, возможно все три) коробок с печеньем, при этом все печеньки из открытых коробок должны быть розданы поровну (сами печенья нельзя ломать на части).

Определите максимальное количество печенек, которое получит каждый гость и Ильнар.

\inputfmtSection

В первой строке вводится $4$ целых числа $a, b, c, n$ $(1 \le a, b, c, n \le 10^8)$ .

\outputfmtSection

Выведите единственное число -- ответ на задачу.

\exampleSection

\sampleTitle{1}

\begin{myverbbox}[\small]{\vinput}
1 2 3 4
\end{myverbbox}
\begin{myverbbox}[\small]{\voutput}
1
\end{myverbbox}
\inputoutputTable

\sampleTitle{2}

\begin{myverbbox}[\small]{\vinput}
3 4 5 19
\end{myverbbox}
\begin{myverbbox}[\small]{\voutput}
0
\end{myverbbox}
\inputoutputTable

\includeSolutionIfExistsByPath{1st_tour_progr/task_031}
\assignementTitle{Печеньки}{10}

У Ильнара есть три коробки с печеньем, в которых $a$, $b$, $c$ печенек соответственно. К Ильнару в гости пришло $(n - 1)$ человек. Ильнар хочет, чтобы всем гостям и ему досталось одинаковое количество печенек. Поэтому он хочет узнать сколько печенек достанется каждому, если он откроет несколько (возможно ни одной, возможно все три) коробок с печеньем, при этом все печеньки из открытых коробок должны быть розданы поровну (сами печенья нельзя ломать на части).

Определите максимальное количество печенек, которое получит каждый гость и Ильнар.

\inputfmtSection

В первой строке вводится $4$ целых числа $a, b, c, n$ $(1 \le a, b, c, n \le 10^8)$ .

\outputfmtSection

Выведите единственное число -- ответ на задачу.

\exampleSection

\sampleTitle{1}

\begin{myverbbox}[\small]{\vinput}
1 2 3 4
\end{myverbbox}
\begin{myverbbox}[\small]{\voutput}
1
\end{myverbbox}
\inputoutputTable

\sampleTitle{2}

\begin{myverbbox}[\small]{\vinput}
3 4 5 19
\end{myverbbox}
\begin{myverbbox}[\small]{\voutput}
0
\end{myverbbox}
\inputoutputTable

\includeSolutionIfExistsByPath{1st_tour_progr/task_031}
\assignementTitle{Печеньки}{10}

У Ильнара есть три коробки с печеньем, в которых $a$, $b$, $c$ печенек соответственно. К Ильнару в гости пришло $(n - 1)$ человек. Ильнар хочет, чтобы всем гостям и ему досталось одинаковое количество печенек. Поэтому он хочет узнать сколько печенек достанется каждому, если он откроет несколько (возможно ни одной, возможно все три) коробок с печеньем, при этом все печеньки из открытых коробок должны быть розданы поровну (сами печенья нельзя ломать на части).

Определите максимальное количество печенек, которое получит каждый гость и Ильнар.

\inputfmtSection

В первой строке вводится $4$ целых числа $a, b, c, n$ $(1 \le a, b, c, n \le 10^8)$ .

\outputfmtSection

Выведите единственное число -- ответ на задачу.

\exampleSection

\sampleTitle{1}

\begin{myverbbox}[\small]{\vinput}
1 2 3 4
\end{myverbbox}
\begin{myverbbox}[\small]{\voutput}
1
\end{myverbbox}
\inputoutputTable

\sampleTitle{2}

\begin{myverbbox}[\small]{\vinput}
3 4 5 19
\end{myverbbox}
\begin{myverbbox}[\small]{\voutput}
0
\end{myverbbox}
\inputoutputTable

\includeSolutionIfExistsByPath{1st_tour_progr/task_031}
\assignementTitle{Печеньки}{10}

У Ильнара есть три коробки с печеньем, в которых $a$, $b$, $c$ печенек соответственно. К Ильнару в гости пришло $(n - 1)$ человек. Ильнар хочет, чтобы всем гостям и ему досталось одинаковое количество печенек. Поэтому он хочет узнать сколько печенек достанется каждому, если он откроет несколько (возможно ни одной, возможно все три) коробок с печеньем, при этом все печеньки из открытых коробок должны быть розданы поровну (сами печенья нельзя ломать на части).

Определите максимальное количество печенек, которое получит каждый гость и Ильнар.

\inputfmtSection

В первой строке вводится $4$ целых числа $a, b, c, n$ $(1 \le a, b, c, n \le 10^8)$ .

\outputfmtSection

Выведите единственное число -- ответ на задачу.

\exampleSection

\sampleTitle{1}

\begin{myverbbox}[\small]{\vinput}
1 2 3 4
\end{myverbbox}
\begin{myverbbox}[\small]{\voutput}
1
\end{myverbbox}
\inputoutputTable

\sampleTitle{2}

\begin{myverbbox}[\small]{\vinput}
3 4 5 19
\end{myverbbox}
\begin{myverbbox}[\small]{\voutput}
0
\end{myverbbox}
\inputoutputTable

\includeSolutionIfExistsByPath{1st_tour_progr/task_031}
\assignementTitle{Печеньки}{10}

У Ильнара есть три коробки с печеньем, в которых $a$, $b$, $c$ печенек соответственно. К Ильнару в гости пришло $(n - 1)$ человек. Ильнар хочет, чтобы всем гостям и ему досталось одинаковое количество печенек. Поэтому он хочет узнать сколько печенек достанется каждому, если он откроет несколько (возможно ни одной, возможно все три) коробок с печеньем, при этом все печеньки из открытых коробок должны быть розданы поровну (сами печенья нельзя ломать на части).

Определите максимальное количество печенек, которое получит каждый гость и Ильнар.

\inputfmtSection

В первой строке вводится $4$ целых числа $a, b, c, n$ $(1 \le a, b, c, n \le 10^8)$ .

\outputfmtSection

Выведите единственное число -- ответ на задачу.

\exampleSection

\sampleTitle{1}

\begin{myverbbox}[\small]{\vinput}
1 2 3 4
\end{myverbbox}
\begin{myverbbox}[\small]{\voutput}
1
\end{myverbbox}
\inputoutputTable

\sampleTitle{2}

\begin{myverbbox}[\small]{\vinput}
3 4 5 19
\end{myverbbox}
\begin{myverbbox}[\small]{\voutput}
0
\end{myverbbox}
\inputoutputTable

\includeSolutionIfExistsByPath{1st_tour_progr/task_031}
\assignementTitle{Печеньки}{10}

У Ильнара есть три коробки с печеньем, в которых $a$, $b$, $c$ печенек соответственно. К Ильнару в гости пришло $(n - 1)$ человек. Ильнар хочет, чтобы всем гостям и ему досталось одинаковое количество печенек. Поэтому он хочет узнать сколько печенек достанется каждому, если он откроет несколько (возможно ни одной, возможно все три) коробок с печеньем, при этом все печеньки из открытых коробок должны быть розданы поровну (сами печенья нельзя ломать на части).

Определите максимальное количество печенек, которое получит каждый гость и Ильнар.

\inputfmtSection

В первой строке вводится $4$ целых числа $a, b, c, n$ $(1 \le a, b, c, n \le 10^8)$ .

\outputfmtSection

Выведите единственное число -- ответ на задачу.

\exampleSection

\sampleTitle{1}

\begin{myverbbox}[\small]{\vinput}
1 2 3 4
\end{myverbbox}
\begin{myverbbox}[\small]{\voutput}
1
\end{myverbbox}
\inputoutputTable

\sampleTitle{2}

\begin{myverbbox}[\small]{\vinput}
3 4 5 19
\end{myverbbox}
\begin{myverbbox}[\small]{\voutput}
0
\end{myverbbox}
\inputoutputTable

\includeSolutionIfExistsByPath{1st_tour_progr/task_031}
\assignementTitle{Печеньки}{10}

У Ильнара есть три коробки с печеньем, в которых $a$, $b$, $c$ печенек соответственно. К Ильнару в гости пришло $(n - 1)$ человек. Ильнар хочет, чтобы всем гостям и ему досталось одинаковое количество печенек. Поэтому он хочет узнать сколько печенек достанется каждому, если он откроет несколько (возможно ни одной, возможно все три) коробок с печеньем, при этом все печеньки из открытых коробок должны быть розданы поровну (сами печенья нельзя ломать на части).

Определите максимальное количество печенек, которое получит каждый гость и Ильнар.

\inputfmtSection

В первой строке вводится $4$ целых числа $a, b, c, n$ $(1 \le a, b, c, n \le 10^8)$ .

\outputfmtSection

Выведите единственное число -- ответ на задачу.

\exampleSection

\sampleTitle{1}

\begin{myverbbox}[\small]{\vinput}
1 2 3 4
\end{myverbbox}
\begin{myverbbox}[\small]{\voutput}
1
\end{myverbbox}
\inputoutputTable

\sampleTitle{2}

\begin{myverbbox}[\small]{\vinput}
3 4 5 19
\end{myverbbox}
\begin{myverbbox}[\small]{\voutput}
0
\end{myverbbox}
\inputoutputTable

\includeSolutionIfExistsByPath{1st_tour_progr/task_031}

\section{Категория Forensic}

Компьютерная криминалистика – прикладная наука о поиске и исследовании доказательств совершения различных действий, связанных с компьютерной информацией.
Чтобы справляться с задачами компьютерной криминалистики, надо иметь представление об основах работы ОС, понимать строение файловой системы и взаимодействие различных процессов. В ходе работы необходимо анализировать образы дисков, дампы памяти, дампы сетевых пакетов, а также логи и т.п.

\assignementTitle{Печеньки}{10}

У Ильнара есть три коробки с печеньем, в которых $a$, $b$, $c$ печенек соответственно. К Ильнару в гости пришло $(n - 1)$ человек. Ильнар хочет, чтобы всем гостям и ему досталось одинаковое количество печенек. Поэтому он хочет узнать сколько печенек достанется каждому, если он откроет несколько (возможно ни одной, возможно все три) коробок с печеньем, при этом все печеньки из открытых коробок должны быть розданы поровну (сами печенья нельзя ломать на части).

Определите максимальное количество печенек, которое получит каждый гость и Ильнар.

\inputfmtSection

В первой строке вводится $4$ целых числа $a, b, c, n$ $(1 \le a, b, c, n \le 10^8)$ .

\outputfmtSection

Выведите единственное число -- ответ на задачу.

\exampleSection

\sampleTitle{1}

\begin{myverbbox}[\small]{\vinput}
1 2 3 4
\end{myverbbox}
\begin{myverbbox}[\small]{\voutput}
1
\end{myverbbox}
\inputoutputTable

\sampleTitle{2}

\begin{myverbbox}[\small]{\vinput}
3 4 5 19
\end{myverbbox}
\begin{myverbbox}[\small]{\voutput}
0
\end{myverbbox}
\inputoutputTable

\includeSolutionIfExistsByPath{1st_tour_progr/task_031}
\assignementTitle{Печеньки}{10}

У Ильнара есть три коробки с печеньем, в которых $a$, $b$, $c$ печенек соответственно. К Ильнару в гости пришло $(n - 1)$ человек. Ильнар хочет, чтобы всем гостям и ему досталось одинаковое количество печенек. Поэтому он хочет узнать сколько печенек достанется каждому, если он откроет несколько (возможно ни одной, возможно все три) коробок с печеньем, при этом все печеньки из открытых коробок должны быть розданы поровну (сами печенья нельзя ломать на части).

Определите максимальное количество печенек, которое получит каждый гость и Ильнар.

\inputfmtSection

В первой строке вводится $4$ целых числа $a, b, c, n$ $(1 \le a, b, c, n \le 10^8)$ .

\outputfmtSection

Выведите единственное число -- ответ на задачу.

\exampleSection

\sampleTitle{1}

\begin{myverbbox}[\small]{\vinput}
1 2 3 4
\end{myverbbox}
\begin{myverbbox}[\small]{\voutput}
1
\end{myverbbox}
\inputoutputTable

\sampleTitle{2}

\begin{myverbbox}[\small]{\vinput}
3 4 5 19
\end{myverbbox}
\begin{myverbbox}[\small]{\voutput}
0
\end{myverbbox}
\inputoutputTable

\includeSolutionIfExistsByPath{1st_tour_progr/task_031}
\assignementTitle{Печеньки}{10}

У Ильнара есть три коробки с печеньем, в которых $a$, $b$, $c$ печенек соответственно. К Ильнару в гости пришло $(n - 1)$ человек. Ильнар хочет, чтобы всем гостям и ему досталось одинаковое количество печенек. Поэтому он хочет узнать сколько печенек достанется каждому, если он откроет несколько (возможно ни одной, возможно все три) коробок с печеньем, при этом все печеньки из открытых коробок должны быть розданы поровну (сами печенья нельзя ломать на части).

Определите максимальное количество печенек, которое получит каждый гость и Ильнар.

\inputfmtSection

В первой строке вводится $4$ целых числа $a, b, c, n$ $(1 \le a, b, c, n \le 10^8)$ .

\outputfmtSection

Выведите единственное число -- ответ на задачу.

\exampleSection

\sampleTitle{1}

\begin{myverbbox}[\small]{\vinput}
1 2 3 4
\end{myverbbox}
\begin{myverbbox}[\small]{\voutput}
1
\end{myverbbox}
\inputoutputTable

\sampleTitle{2}

\begin{myverbbox}[\small]{\vinput}
3 4 5 19
\end{myverbbox}
\begin{myverbbox}[\small]{\voutput}
0
\end{myverbbox}
\inputoutputTable

\includeSolutionIfExistsByPath{1st_tour_progr/task_031}
\assignementTitle{Печеньки}{10}

У Ильнара есть три коробки с печеньем, в которых $a$, $b$, $c$ печенек соответственно. К Ильнару в гости пришло $(n - 1)$ человек. Ильнар хочет, чтобы всем гостям и ему досталось одинаковое количество печенек. Поэтому он хочет узнать сколько печенек достанется каждому, если он откроет несколько (возможно ни одной, возможно все три) коробок с печеньем, при этом все печеньки из открытых коробок должны быть розданы поровну (сами печенья нельзя ломать на части).

Определите максимальное количество печенек, которое получит каждый гость и Ильнар.

\inputfmtSection

В первой строке вводится $4$ целых числа $a, b, c, n$ $(1 \le a, b, c, n \le 10^8)$ .

\outputfmtSection

Выведите единственное число -- ответ на задачу.

\exampleSection

\sampleTitle{1}

\begin{myverbbox}[\small]{\vinput}
1 2 3 4
\end{myverbbox}
\begin{myverbbox}[\small]{\voutput}
1
\end{myverbbox}
\inputoutputTable

\sampleTitle{2}

\begin{myverbbox}[\small]{\vinput}
3 4 5 19
\end{myverbbox}
\begin{myverbbox}[\small]{\voutput}
0
\end{myverbbox}
\inputoutputTable

\includeSolutionIfExistsByPath{1st_tour_progr/task_031}
\assignementTitle{Печеньки}{10}

У Ильнара есть три коробки с печеньем, в которых $a$, $b$, $c$ печенек соответственно. К Ильнару в гости пришло $(n - 1)$ человек. Ильнар хочет, чтобы всем гостям и ему досталось одинаковое количество печенек. Поэтому он хочет узнать сколько печенек достанется каждому, если он откроет несколько (возможно ни одной, возможно все три) коробок с печеньем, при этом все печеньки из открытых коробок должны быть розданы поровну (сами печенья нельзя ломать на части).

Определите максимальное количество печенек, которое получит каждый гость и Ильнар.

\inputfmtSection

В первой строке вводится $4$ целых числа $a, b, c, n$ $(1 \le a, b, c, n \le 10^8)$ .

\outputfmtSection

Выведите единственное число -- ответ на задачу.

\exampleSection

\sampleTitle{1}

\begin{myverbbox}[\small]{\vinput}
1 2 3 4
\end{myverbbox}
\begin{myverbbox}[\small]{\voutput}
1
\end{myverbbox}
\inputoutputTable

\sampleTitle{2}

\begin{myverbbox}[\small]{\vinput}
3 4 5 19
\end{myverbbox}
\begin{myverbbox}[\small]{\voutput}
0
\end{myverbbox}
\inputoutputTable

\includeSolutionIfExistsByPath{1st_tour_progr/task_031}

\section{Категория Crypto}

Криптография – наука о том, как преобразовать исходные данные таким образом, чтобы обеспечить их защиту от посторонних, а также защитить от подмены или сделать её невозможной.

Задания данной категории предлагают применить свои знания в математике и криптоанализе для решения криптографических головоломок, будь то простой шифр замены или же некорректно использованный шифр RSA.

\assignementTitle{Печеньки}{10}

У Ильнара есть три коробки с печеньем, в которых $a$, $b$, $c$ печенек соответственно. К Ильнару в гости пришло $(n - 1)$ человек. Ильнар хочет, чтобы всем гостям и ему досталось одинаковое количество печенек. Поэтому он хочет узнать сколько печенек достанется каждому, если он откроет несколько (возможно ни одной, возможно все три) коробок с печеньем, при этом все печеньки из открытых коробок должны быть розданы поровну (сами печенья нельзя ломать на части).

Определите максимальное количество печенек, которое получит каждый гость и Ильнар.

\inputfmtSection

В первой строке вводится $4$ целых числа $a, b, c, n$ $(1 \le a, b, c, n \le 10^8)$ .

\outputfmtSection

Выведите единственное число -- ответ на задачу.

\exampleSection

\sampleTitle{1}

\begin{myverbbox}[\small]{\vinput}
1 2 3 4
\end{myverbbox}
\begin{myverbbox}[\small]{\voutput}
1
\end{myverbbox}
\inputoutputTable

\sampleTitle{2}

\begin{myverbbox}[\small]{\vinput}
3 4 5 19
\end{myverbbox}
\begin{myverbbox}[\small]{\voutput}
0
\end{myverbbox}
\inputoutputTable

\includeSolutionIfExistsByPath{1st_tour_progr/task_031}
\assignementTitle{Печеньки}{10}

У Ильнара есть три коробки с печеньем, в которых $a$, $b$, $c$ печенек соответственно. К Ильнару в гости пришло $(n - 1)$ человек. Ильнар хочет, чтобы всем гостям и ему досталось одинаковое количество печенек. Поэтому он хочет узнать сколько печенек достанется каждому, если он откроет несколько (возможно ни одной, возможно все три) коробок с печеньем, при этом все печеньки из открытых коробок должны быть розданы поровну (сами печенья нельзя ломать на части).

Определите максимальное количество печенек, которое получит каждый гость и Ильнар.

\inputfmtSection

В первой строке вводится $4$ целых числа $a, b, c, n$ $(1 \le a, b, c, n \le 10^8)$ .

\outputfmtSection

Выведите единственное число -- ответ на задачу.

\exampleSection

\sampleTitle{1}

\begin{myverbbox}[\small]{\vinput}
1 2 3 4
\end{myverbbox}
\begin{myverbbox}[\small]{\voutput}
1
\end{myverbbox}
\inputoutputTable

\sampleTitle{2}

\begin{myverbbox}[\small]{\vinput}
3 4 5 19
\end{myverbbox}
\begin{myverbbox}[\small]{\voutput}
0
\end{myverbbox}
\inputoutputTable

\includeSolutionIfExistsByPath{1st_tour_progr/task_031}
\assignementTitle{Печеньки}{10}

У Ильнара есть три коробки с печеньем, в которых $a$, $b$, $c$ печенек соответственно. К Ильнару в гости пришло $(n - 1)$ человек. Ильнар хочет, чтобы всем гостям и ему досталось одинаковое количество печенек. Поэтому он хочет узнать сколько печенек достанется каждому, если он откроет несколько (возможно ни одной, возможно все три) коробок с печеньем, при этом все печеньки из открытых коробок должны быть розданы поровну (сами печенья нельзя ломать на части).

Определите максимальное количество печенек, которое получит каждый гость и Ильнар.

\inputfmtSection

В первой строке вводится $4$ целых числа $a, b, c, n$ $(1 \le a, b, c, n \le 10^8)$ .

\outputfmtSection

Выведите единственное число -- ответ на задачу.

\exampleSection

\sampleTitle{1}

\begin{myverbbox}[\small]{\vinput}
1 2 3 4
\end{myverbbox}
\begin{myverbbox}[\small]{\voutput}
1
\end{myverbbox}
\inputoutputTable

\sampleTitle{2}

\begin{myverbbox}[\small]{\vinput}
3 4 5 19
\end{myverbbox}
\begin{myverbbox}[\small]{\voutput}
0
\end{myverbbox}
\inputoutputTable

\includeSolutionIfExistsByPath{1st_tour_progr/task_031}
\assignementTitle{Печеньки}{10}

У Ильнара есть три коробки с печеньем, в которых $a$, $b$, $c$ печенек соответственно. К Ильнару в гости пришло $(n - 1)$ человек. Ильнар хочет, чтобы всем гостям и ему досталось одинаковое количество печенек. Поэтому он хочет узнать сколько печенек достанется каждому, если он откроет несколько (возможно ни одной, возможно все три) коробок с печеньем, при этом все печеньки из открытых коробок должны быть розданы поровну (сами печенья нельзя ломать на части).

Определите максимальное количество печенек, которое получит каждый гость и Ильнар.

\inputfmtSection

В первой строке вводится $4$ целых числа $a, b, c, n$ $(1 \le a, b, c, n \le 10^8)$ .

\outputfmtSection

Выведите единственное число -- ответ на задачу.

\exampleSection

\sampleTitle{1}

\begin{myverbbox}[\small]{\vinput}
1 2 3 4
\end{myverbbox}
\begin{myverbbox}[\small]{\voutput}
1
\end{myverbbox}
\inputoutputTable

\sampleTitle{2}

\begin{myverbbox}[\small]{\vinput}
3 4 5 19
\end{myverbbox}
\begin{myverbbox}[\small]{\voutput}
0
\end{myverbbox}
\inputoutputTable

\includeSolutionIfExistsByPath{1st_tour_progr/task_031}
