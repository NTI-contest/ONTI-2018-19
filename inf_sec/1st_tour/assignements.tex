На этом этапе участникам необходимо было разработать части автономной транспортной системы: алгоритмы локального, глобального позиционирования и поиска пути для беспилотного автомобиля, алгоритмы навигации коптера внутри помещения с использованием технического зрения, алгоритмы передачи груза, а также смоделировать и изготовить механизм для захвата грузов. Для решения данной задачи участникам необходимо было использовать методы компьютерного зрения и машинного обучения, знания ЗД моделирования и электроники. 

Участникам был предложен ряд задач. Для их решения, необходимо было определить кто занимается коптером, а кто автомобилем. В получившихся парах обязательно должен быть программист и инженер. Инженер решает задачи 3Д моделирования и сборки устройства в оптимальной конфигурации, программист отвечает за управляющие алгоритмы устройства.

Максимальное количество баллов, которые могла заработать команда - 85.5 баллов. Каждый день открывались новые задания, их можно было выполнять в произвольном порядке. Задачи для автомобилей предыдущего дня можно было сдавать в любой момент без штрафов. Задания на день по 3D моделированию и программированию полета коптера каждый день выполнялись в зачет в обозначенное организаторами время. Задание текущего дня в последующие дни не засчитывалось.

В таблице ниже представлен список заданий по дням.

\begin{table}
    \begin{center}
        \begin{tabular}{|p{6.5cm}|p{2.5cm}|p{1.5cm}|}
            \hline
            ДЕНЬ 1 & Максимальный балл & Сумма \\
            \hline
            Создайте 3D модель и gcode универсального захватывающего устройства, способного захватывать и переносить грузы различной формы и сам груз & 4.7 & 6.5 \\
            \hline
            Нарисуйте принципиальную схему захватывающего устройства & 1
            Сохраните все файлы в форматах, пригодных для последующего редактирования и печати на 3D принтере
            0,8
            Запрограммируйте автономный взлёт коптера, сопровождаемый световой индикацией
            1,5
            8,5
            Напишите программу, позволяющую коптеру прилететь в заданную точку, зависнуть на 5 секунд и продемонстрировать достижение точки световой индикацией. Во время полета необходимо показывать, сколько осталось лететь до точки назначения 
            2,5
            Сделайте автономный пролет со световой индикацией над 3 контрольными точками
            4,5




            Научите автомобиль двигаться по прямому участку трассы, не вылетая за пределы своей полосы движения
            1
            15
            Научите автомобиль проходить повороты на трассе, не вылетая за пределы своей полосы движения
            2
            Нацчите автомобиль детектировать стоп-линию и остановиться перед перекрёстком.
            1
            Научите автомобиль пересекать перекрёсток совершая поворот направо
            2
            Научите автомобиль пересекать перекрёсток совершая поворот налево
            2
            Научите автомобиль пересекать перекрёсток не поворачивая
            2
            Научите автомобиль, не вылетая со своей полосы движения, проходить маршрут из точки А в точку Б
            5
            Сумма баллов:
            30
 

        \end{tabular}
    \end{center}
\end{table}

\begin{table}
    \begin{center}
        \begin{tabular}{|p{6.5cm}|p{2.5cm}|p{1.5cm}|}

        \end{tabular}
    \end{center}
\end{table}

\begin{table}
    \begin{center}
        \begin{tabular}{|p{6.5cm}|p{2.5cm}|p{1.5cm}|}

        \end{tabular}
    \end{center}
\end{table}

\section{Категория Web}

Задания данной категории предполагают поиск уязвимостей веб-приложения (веб-сайта) и дальнейшую их эксплуатацию.

Для этого необходимо знать основы разработки веб-приложений, понимать базовые принципы их проектирования, иметь представление о том, как работает тот или иной веб-фреймворк или CMS. Кроме того, важно понимать природу возникновения веб-уязвимостей, понимать и уметь эксплуатировать типовые уязвимости в веб-приложениях. Список наиболее распространенных веб-уязвимостей периодически публикуется в рамках проекта OWASP Top 10.

\assignementTitle{Brain Damage}{1000}{}

Друг скинул мне этот текст вместе с улыбающимся смайликом. Хорошо зная своего друга, я предполагаю, что для раскрытия загадки этого сообщения нужно напрячь мозги на полную катушку. И хотя я уже потратил много времени на поиски решения, ничего больше странных символов и некоторой структуры в их расположении мне увидеть не удалось. Мне очень бы помогло, если бы вы хотя бы сказали мне, что представляет из себя этот текст...

Файл из задания доступен по ссылке: \url{https://cyberchallenge.rt.ru/files/85b80336a3cd3a1ee35739dab1601ba9/prog.txt}

\solutionSection

Исходный файл – программа на языке Brainfuck. Для запуска программы, можно воспользоваться любым подходящим онлайн сервисом. Видим, что программа выводит фразу «haha no flag here» и зацикливается.

Наиболее простой способ понять, почему же зацикливается программа, это перевести ее в код на другом, более читаемом языке, например, на Python. Для этого можно воспользоваться скриптом \url{https://www.nayuki.io/res/optimizing-brainfuck-compiler/bfc.py}.

В полученной программе на языке Python, бесконечный цикл будет начинаться в строке №68:

\begin{minted}[fontsize=\footnotesize, linenos]{python}
while mem[i] != 0:
    mem[i] = (mem[i] + 2) & 0xFF
    mem[i + 1] = (mem[i + 1] + 0) & 0xFF
    mem[i] = (mem[i] + 254) & 0xFF
\end{minted}

Легко заметить, что в данном цикле, на каждой итерации значение mem[i] меняться не будет, а значит и условие выхода из цикла никогда не выполнится. Все, что нам остается, это убрать этот бесконечный цикл и весь код до него. После запуска, такая программа выведет искомый флаг.

\answerMath{CC\{Br4inf0ck\_ta5k\_num\_57123849\}.}


\assignementTitle{Brain Damage}{1000}{}

Друг скинул мне этот текст вместе с улыбающимся смайликом. Хорошо зная своего друга, я предполагаю, что для раскрытия загадки этого сообщения нужно напрячь мозги на полную катушку. И хотя я уже потратил много времени на поиски решения, ничего больше странных символов и некоторой структуры в их расположении мне увидеть не удалось. Мне очень бы помогло, если бы вы хотя бы сказали мне, что представляет из себя этот текст...

Файл из задания доступен по ссылке: \url{https://cyberchallenge.rt.ru/files/85b80336a3cd3a1ee35739dab1601ba9/prog.txt}

\solutionSection

Исходный файл – программа на языке Brainfuck. Для запуска программы, можно воспользоваться любым подходящим онлайн сервисом. Видим, что программа выводит фразу «haha no flag here» и зацикливается.

Наиболее простой способ понять, почему же зацикливается программа, это перевести ее в код на другом, более читаемом языке, например, на Python. Для этого можно воспользоваться скриптом \url{https://www.nayuki.io/res/optimizing-brainfuck-compiler/bfc.py}.

В полученной программе на языке Python, бесконечный цикл будет начинаться в строке №68:

\begin{minted}[fontsize=\footnotesize, linenos]{python}
while mem[i] != 0:
    mem[i] = (mem[i] + 2) & 0xFF
    mem[i + 1] = (mem[i + 1] + 0) & 0xFF
    mem[i] = (mem[i] + 254) & 0xFF
\end{minted}

Легко заметить, что в данном цикле, на каждой итерации значение mem[i] меняться не будет, а значит и условие выхода из цикла никогда не выполнится. Все, что нам остается, это убрать этот бесконечный цикл и весь код до него. После запуска, такая программа выведет искомый флаг.

\answerMath{CC\{Br4inf0ck\_ta5k\_num\_57123849\}.}


\assignementTitle{Brain Damage}{1000}{}

Друг скинул мне этот текст вместе с улыбающимся смайликом. Хорошо зная своего друга, я предполагаю, что для раскрытия загадки этого сообщения нужно напрячь мозги на полную катушку. И хотя я уже потратил много времени на поиски решения, ничего больше странных символов и некоторой структуры в их расположении мне увидеть не удалось. Мне очень бы помогло, если бы вы хотя бы сказали мне, что представляет из себя этот текст...

Файл из задания доступен по ссылке: \url{https://cyberchallenge.rt.ru/files/85b80336a3cd3a1ee35739dab1601ba9/prog.txt}

\solutionSection

Исходный файл – программа на языке Brainfuck. Для запуска программы, можно воспользоваться любым подходящим онлайн сервисом. Видим, что программа выводит фразу «haha no flag here» и зацикливается.

Наиболее простой способ понять, почему же зацикливается программа, это перевести ее в код на другом, более читаемом языке, например, на Python. Для этого можно воспользоваться скриптом \url{https://www.nayuki.io/res/optimizing-brainfuck-compiler/bfc.py}.

В полученной программе на языке Python, бесконечный цикл будет начинаться в строке №68:

\begin{minted}[fontsize=\footnotesize, linenos]{python}
while mem[i] != 0:
    mem[i] = (mem[i] + 2) & 0xFF
    mem[i + 1] = (mem[i + 1] + 0) & 0xFF
    mem[i] = (mem[i] + 254) & 0xFF
\end{minted}

Легко заметить, что в данном цикле, на каждой итерации значение mem[i] меняться не будет, а значит и условие выхода из цикла никогда не выполнится. Все, что нам остается, это убрать этот бесконечный цикл и весь код до него. После запуска, такая программа выведет искомый флаг.

\answerMath{CC\{Br4inf0ck\_ta5k\_num\_57123849\}.}


\assignementTitle{Brain Damage}{1000}{}

Друг скинул мне этот текст вместе с улыбающимся смайликом. Хорошо зная своего друга, я предполагаю, что для раскрытия загадки этого сообщения нужно напрячь мозги на полную катушку. И хотя я уже потратил много времени на поиски решения, ничего больше странных символов и некоторой структуры в их расположении мне увидеть не удалось. Мне очень бы помогло, если бы вы хотя бы сказали мне, что представляет из себя этот текст...

Файл из задания доступен по ссылке: \url{https://cyberchallenge.rt.ru/files/85b80336a3cd3a1ee35739dab1601ba9/prog.txt}

\solutionSection

Исходный файл – программа на языке Brainfuck. Для запуска программы, можно воспользоваться любым подходящим онлайн сервисом. Видим, что программа выводит фразу «haha no flag here» и зацикливается.

Наиболее простой способ понять, почему же зацикливается программа, это перевести ее в код на другом, более читаемом языке, например, на Python. Для этого можно воспользоваться скриптом \url{https://www.nayuki.io/res/optimizing-brainfuck-compiler/bfc.py}.

В полученной программе на языке Python, бесконечный цикл будет начинаться в строке №68:

\begin{minted}[fontsize=\footnotesize, linenos]{python}
while mem[i] != 0:
    mem[i] = (mem[i] + 2) & 0xFF
    mem[i + 1] = (mem[i + 1] + 0) & 0xFF
    mem[i] = (mem[i] + 254) & 0xFF
\end{minted}

Легко заметить, что в данном цикле, на каждой итерации значение mem[i] меняться не будет, а значит и условие выхода из цикла никогда не выполнится. Все, что нам остается, это убрать этот бесконечный цикл и весь код до него. После запуска, такая программа выведет искомый флаг.

\answerMath{CC\{Br4inf0ck\_ta5k\_num\_57123849\}.}


\assignementTitle{Brain Damage}{1000}{}

Друг скинул мне этот текст вместе с улыбающимся смайликом. Хорошо зная своего друга, я предполагаю, что для раскрытия загадки этого сообщения нужно напрячь мозги на полную катушку. И хотя я уже потратил много времени на поиски решения, ничего больше странных символов и некоторой структуры в их расположении мне увидеть не удалось. Мне очень бы помогло, если бы вы хотя бы сказали мне, что представляет из себя этот текст...

Файл из задания доступен по ссылке: \url{https://cyberchallenge.rt.ru/files/85b80336a3cd3a1ee35739dab1601ba9/prog.txt}

\solutionSection

Исходный файл – программа на языке Brainfuck. Для запуска программы, можно воспользоваться любым подходящим онлайн сервисом. Видим, что программа выводит фразу «haha no flag here» и зацикливается.

Наиболее простой способ понять, почему же зацикливается программа, это перевести ее в код на другом, более читаемом языке, например, на Python. Для этого можно воспользоваться скриптом \url{https://www.nayuki.io/res/optimizing-brainfuck-compiler/bfc.py}.

В полученной программе на языке Python, бесконечный цикл будет начинаться в строке №68:

\begin{minted}[fontsize=\footnotesize, linenos]{python}
while mem[i] != 0:
    mem[i] = (mem[i] + 2) & 0xFF
    mem[i + 1] = (mem[i + 1] + 0) & 0xFF
    mem[i] = (mem[i] + 254) & 0xFF
\end{minted}

Легко заметить, что в данном цикле, на каждой итерации значение mem[i] меняться не будет, а значит и условие выхода из цикла никогда не выполнится. Все, что нам остается, это убрать этот бесконечный цикл и весь код до него. После запуска, такая программа выведет искомый флаг.

\answerMath{CC\{Br4inf0ck\_ta5k\_num\_57123849\}.}


\assignementTitle{Brain Damage}{1000}{}

Друг скинул мне этот текст вместе с улыбающимся смайликом. Хорошо зная своего друга, я предполагаю, что для раскрытия загадки этого сообщения нужно напрячь мозги на полную катушку. И хотя я уже потратил много времени на поиски решения, ничего больше странных символов и некоторой структуры в их расположении мне увидеть не удалось. Мне очень бы помогло, если бы вы хотя бы сказали мне, что представляет из себя этот текст...

Файл из задания доступен по ссылке: \url{https://cyberchallenge.rt.ru/files/85b80336a3cd3a1ee35739dab1601ba9/prog.txt}

\solutionSection

Исходный файл – программа на языке Brainfuck. Для запуска программы, можно воспользоваться любым подходящим онлайн сервисом. Видим, что программа выводит фразу «haha no flag here» и зацикливается.

Наиболее простой способ понять, почему же зацикливается программа, это перевести ее в код на другом, более читаемом языке, например, на Python. Для этого можно воспользоваться скриптом \url{https://www.nayuki.io/res/optimizing-brainfuck-compiler/bfc.py}.

В полученной программе на языке Python, бесконечный цикл будет начинаться в строке №68:

\begin{minted}[fontsize=\footnotesize, linenos]{python}
while mem[i] != 0:
    mem[i] = (mem[i] + 2) & 0xFF
    mem[i + 1] = (mem[i + 1] + 0) & 0xFF
    mem[i] = (mem[i] + 254) & 0xFF
\end{minted}

Легко заметить, что в данном цикле, на каждой итерации значение mem[i] меняться не будет, а значит и условие выхода из цикла никогда не выполнится. Все, что нам остается, это убрать этот бесконечный цикл и весь код до него. После запуска, такая программа выведет искомый флаг.

\answerMath{CC\{Br4inf0ck\_ta5k\_num\_57123849\}.}


\assignementTitle{Brain Damage}{1000}{}

Друг скинул мне этот текст вместе с улыбающимся смайликом. Хорошо зная своего друга, я предполагаю, что для раскрытия загадки этого сообщения нужно напрячь мозги на полную катушку. И хотя я уже потратил много времени на поиски решения, ничего больше странных символов и некоторой структуры в их расположении мне увидеть не удалось. Мне очень бы помогло, если бы вы хотя бы сказали мне, что представляет из себя этот текст...

Файл из задания доступен по ссылке: \url{https://cyberchallenge.rt.ru/files/85b80336a3cd3a1ee35739dab1601ba9/prog.txt}

\solutionSection

Исходный файл – программа на языке Brainfuck. Для запуска программы, можно воспользоваться любым подходящим онлайн сервисом. Видим, что программа выводит фразу «haha no flag here» и зацикливается.

Наиболее простой способ понять, почему же зацикливается программа, это перевести ее в код на другом, более читаемом языке, например, на Python. Для этого можно воспользоваться скриптом \url{https://www.nayuki.io/res/optimizing-brainfuck-compiler/bfc.py}.

В полученной программе на языке Python, бесконечный цикл будет начинаться в строке №68:

\begin{minted}[fontsize=\footnotesize, linenos]{python}
while mem[i] != 0:
    mem[i] = (mem[i] + 2) & 0xFF
    mem[i + 1] = (mem[i + 1] + 0) & 0xFF
    mem[i] = (mem[i] + 254) & 0xFF
\end{minted}

Легко заметить, что в данном цикле, на каждой итерации значение mem[i] меняться не будет, а значит и условие выхода из цикла никогда не выполнится. Все, что нам остается, это убрать этот бесконечный цикл и весь код до него. После запуска, такая программа выведет искомый флаг.

\answerMath{CC\{Br4inf0ck\_ta5k\_num\_57123849\}.}


\assignementTitle{Brain Damage}{1000}{}

Друг скинул мне этот текст вместе с улыбающимся смайликом. Хорошо зная своего друга, я предполагаю, что для раскрытия загадки этого сообщения нужно напрячь мозги на полную катушку. И хотя я уже потратил много времени на поиски решения, ничего больше странных символов и некоторой структуры в их расположении мне увидеть не удалось. Мне очень бы помогло, если бы вы хотя бы сказали мне, что представляет из себя этот текст...

Файл из задания доступен по ссылке: \url{https://cyberchallenge.rt.ru/files/85b80336a3cd3a1ee35739dab1601ba9/prog.txt}

\solutionSection

Исходный файл – программа на языке Brainfuck. Для запуска программы, можно воспользоваться любым подходящим онлайн сервисом. Видим, что программа выводит фразу «haha no flag here» и зацикливается.

Наиболее простой способ понять, почему же зацикливается программа, это перевести ее в код на другом, более читаемом языке, например, на Python. Для этого можно воспользоваться скриптом \url{https://www.nayuki.io/res/optimizing-brainfuck-compiler/bfc.py}.

В полученной программе на языке Python, бесконечный цикл будет начинаться в строке №68:

\begin{minted}[fontsize=\footnotesize, linenos]{python}
while mem[i] != 0:
    mem[i] = (mem[i] + 2) & 0xFF
    mem[i + 1] = (mem[i + 1] + 0) & 0xFF
    mem[i] = (mem[i] + 254) & 0xFF
\end{minted}

Легко заметить, что в данном цикле, на каждой итерации значение mem[i] меняться не будет, а значит и условие выхода из цикла никогда не выполнится. Все, что нам остается, это убрать этот бесконечный цикл и весь код до него. После запуска, такая программа выведет искомый флаг.

\answerMath{CC\{Br4inf0ck\_ta5k\_num\_57123849\}.}


\assignementTitle{Brain Damage}{1000}{}

Друг скинул мне этот текст вместе с улыбающимся смайликом. Хорошо зная своего друга, я предполагаю, что для раскрытия загадки этого сообщения нужно напрячь мозги на полную катушку. И хотя я уже потратил много времени на поиски решения, ничего больше странных символов и некоторой структуры в их расположении мне увидеть не удалось. Мне очень бы помогло, если бы вы хотя бы сказали мне, что представляет из себя этот текст...

Файл из задания доступен по ссылке: \url{https://cyberchallenge.rt.ru/files/85b80336a3cd3a1ee35739dab1601ba9/prog.txt}

\solutionSection

Исходный файл – программа на языке Brainfuck. Для запуска программы, можно воспользоваться любым подходящим онлайн сервисом. Видим, что программа выводит фразу «haha no flag here» и зацикливается.

Наиболее простой способ понять, почему же зацикливается программа, это перевести ее в код на другом, более читаемом языке, например, на Python. Для этого можно воспользоваться скриптом \url{https://www.nayuki.io/res/optimizing-brainfuck-compiler/bfc.py}.

В полученной программе на языке Python, бесконечный цикл будет начинаться в строке №68:

\begin{minted}[fontsize=\footnotesize, linenos]{python}
while mem[i] != 0:
    mem[i] = (mem[i] + 2) & 0xFF
    mem[i + 1] = (mem[i + 1] + 0) & 0xFF
    mem[i] = (mem[i] + 254) & 0xFF
\end{minted}

Легко заметить, что в данном цикле, на каждой итерации значение mem[i] меняться не будет, а значит и условие выхода из цикла никогда не выполнится. Все, что нам остается, это убрать этот бесконечный цикл и весь код до него. После запуска, такая программа выведет искомый флаг.

\answerMath{CC\{Br4inf0ck\_ta5k\_num\_57123849\}.}


\assignementTitle{Brain Damage}{1000}{}

Друг скинул мне этот текст вместе с улыбающимся смайликом. Хорошо зная своего друга, я предполагаю, что для раскрытия загадки этого сообщения нужно напрячь мозги на полную катушку. И хотя я уже потратил много времени на поиски решения, ничего больше странных символов и некоторой структуры в их расположении мне увидеть не удалось. Мне очень бы помогло, если бы вы хотя бы сказали мне, что представляет из себя этот текст...

Файл из задания доступен по ссылке: \url{https://cyberchallenge.rt.ru/files/85b80336a3cd3a1ee35739dab1601ba9/prog.txt}

\solutionSection

Исходный файл – программа на языке Brainfuck. Для запуска программы, можно воспользоваться любым подходящим онлайн сервисом. Видим, что программа выводит фразу «haha no flag here» и зацикливается.

Наиболее простой способ понять, почему же зацикливается программа, это перевести ее в код на другом, более читаемом языке, например, на Python. Для этого можно воспользоваться скриптом \url{https://www.nayuki.io/res/optimizing-brainfuck-compiler/bfc.py}.

В полученной программе на языке Python, бесконечный цикл будет начинаться в строке №68:

\begin{minted}[fontsize=\footnotesize, linenos]{python}
while mem[i] != 0:
    mem[i] = (mem[i] + 2) & 0xFF
    mem[i + 1] = (mem[i + 1] + 0) & 0xFF
    mem[i] = (mem[i] + 254) & 0xFF
\end{minted}

Легко заметить, что в данном цикле, на каждой итерации значение mem[i] меняться не будет, а значит и условие выхода из цикла никогда не выполнится. Все, что нам остается, это убрать этот бесконечный цикл и весь код до него. После запуска, такая программа выведет искомый флаг.

\answerMath{CC\{Br4inf0ck\_ta5k\_num\_57123849\}.}


\assignementTitle{Brain Damage}{1000}{}

Друг скинул мне этот текст вместе с улыбающимся смайликом. Хорошо зная своего друга, я предполагаю, что для раскрытия загадки этого сообщения нужно напрячь мозги на полную катушку. И хотя я уже потратил много времени на поиски решения, ничего больше странных символов и некоторой структуры в их расположении мне увидеть не удалось. Мне очень бы помогло, если бы вы хотя бы сказали мне, что представляет из себя этот текст...

Файл из задания доступен по ссылке: \url{https://cyberchallenge.rt.ru/files/85b80336a3cd3a1ee35739dab1601ba9/prog.txt}

\solutionSection

Исходный файл – программа на языке Brainfuck. Для запуска программы, можно воспользоваться любым подходящим онлайн сервисом. Видим, что программа выводит фразу «haha no flag here» и зацикливается.

Наиболее простой способ понять, почему же зацикливается программа, это перевести ее в код на другом, более читаемом языке, например, на Python. Для этого можно воспользоваться скриптом \url{https://www.nayuki.io/res/optimizing-brainfuck-compiler/bfc.py}.

В полученной программе на языке Python, бесконечный цикл будет начинаться в строке №68:

\begin{minted}[fontsize=\footnotesize, linenos]{python}
while mem[i] != 0:
    mem[i] = (mem[i] + 2) & 0xFF
    mem[i + 1] = (mem[i + 1] + 0) & 0xFF
    mem[i] = (mem[i] + 254) & 0xFF
\end{minted}

Легко заметить, что в данном цикле, на каждой итерации значение mem[i] меняться не будет, а значит и условие выхода из цикла никогда не выполнится. Все, что нам остается, это убрать этот бесконечный цикл и весь код до него. После запуска, такая программа выведет искомый флаг.

\answerMath{CC\{Br4inf0ck\_ta5k\_num\_57123849\}.}


\assignementTitle{Brain Damage}{1000}{}

Друг скинул мне этот текст вместе с улыбающимся смайликом. Хорошо зная своего друга, я предполагаю, что для раскрытия загадки этого сообщения нужно напрячь мозги на полную катушку. И хотя я уже потратил много времени на поиски решения, ничего больше странных символов и некоторой структуры в их расположении мне увидеть не удалось. Мне очень бы помогло, если бы вы хотя бы сказали мне, что представляет из себя этот текст...

Файл из задания доступен по ссылке: \url{https://cyberchallenge.rt.ru/files/85b80336a3cd3a1ee35739dab1601ba9/prog.txt}

\solutionSection

Исходный файл – программа на языке Brainfuck. Для запуска программы, можно воспользоваться любым подходящим онлайн сервисом. Видим, что программа выводит фразу «haha no flag here» и зацикливается.

Наиболее простой способ понять, почему же зацикливается программа, это перевести ее в код на другом, более читаемом языке, например, на Python. Для этого можно воспользоваться скриптом \url{https://www.nayuki.io/res/optimizing-brainfuck-compiler/bfc.py}.

В полученной программе на языке Python, бесконечный цикл будет начинаться в строке №68:

\begin{minted}[fontsize=\footnotesize, linenos]{python}
while mem[i] != 0:
    mem[i] = (mem[i] + 2) & 0xFF
    mem[i + 1] = (mem[i + 1] + 0) & 0xFF
    mem[i] = (mem[i] + 254) & 0xFF
\end{minted}

Легко заметить, что в данном цикле, на каждой итерации значение mem[i] меняться не будет, а значит и условие выхода из цикла никогда не выполнится. Все, что нам остается, это убрать этот бесконечный цикл и весь код до него. После запуска, такая программа выведет искомый флаг.

\answerMath{CC\{Br4inf0ck\_ta5k\_num\_57123849\}.}


\assignementTitle{Brain Damage}{1000}{}

Друг скинул мне этот текст вместе с улыбающимся смайликом. Хорошо зная своего друга, я предполагаю, что для раскрытия загадки этого сообщения нужно напрячь мозги на полную катушку. И хотя я уже потратил много времени на поиски решения, ничего больше странных символов и некоторой структуры в их расположении мне увидеть не удалось. Мне очень бы помогло, если бы вы хотя бы сказали мне, что представляет из себя этот текст...

Файл из задания доступен по ссылке: \url{https://cyberchallenge.rt.ru/files/85b80336a3cd3a1ee35739dab1601ba9/prog.txt}

\solutionSection

Исходный файл – программа на языке Brainfuck. Для запуска программы, можно воспользоваться любым подходящим онлайн сервисом. Видим, что программа выводит фразу «haha no flag here» и зацикливается.

Наиболее простой способ понять, почему же зацикливается программа, это перевести ее в код на другом, более читаемом языке, например, на Python. Для этого можно воспользоваться скриптом \url{https://www.nayuki.io/res/optimizing-brainfuck-compiler/bfc.py}.

В полученной программе на языке Python, бесконечный цикл будет начинаться в строке №68:

\begin{minted}[fontsize=\footnotesize, linenos]{python}
while mem[i] != 0:
    mem[i] = (mem[i] + 2) & 0xFF
    mem[i + 1] = (mem[i + 1] + 0) & 0xFF
    mem[i] = (mem[i] + 254) & 0xFF
\end{minted}

Легко заметить, что в данном цикле, на каждой итерации значение mem[i] меняться не будет, а значит и условие выхода из цикла никогда не выполнится. Все, что нам остается, это убрать этот бесконечный цикл и весь код до него. После запуска, такая программа выведет искомый флаг.

\answerMath{CC\{Br4inf0ck\_ta5k\_num\_57123849\}.}


\assignementTitle{Brain Damage}{1000}{}

Друг скинул мне этот текст вместе с улыбающимся смайликом. Хорошо зная своего друга, я предполагаю, что для раскрытия загадки этого сообщения нужно напрячь мозги на полную катушку. И хотя я уже потратил много времени на поиски решения, ничего больше странных символов и некоторой структуры в их расположении мне увидеть не удалось. Мне очень бы помогло, если бы вы хотя бы сказали мне, что представляет из себя этот текст...

Файл из задания доступен по ссылке: \url{https://cyberchallenge.rt.ru/files/85b80336a3cd3a1ee35739dab1601ba9/prog.txt}

\solutionSection

Исходный файл – программа на языке Brainfuck. Для запуска программы, можно воспользоваться любым подходящим онлайн сервисом. Видим, что программа выводит фразу «haha no flag here» и зацикливается.

Наиболее простой способ понять, почему же зацикливается программа, это перевести ее в код на другом, более читаемом языке, например, на Python. Для этого можно воспользоваться скриптом \url{https://www.nayuki.io/res/optimizing-brainfuck-compiler/bfc.py}.

В полученной программе на языке Python, бесконечный цикл будет начинаться в строке №68:

\begin{minted}[fontsize=\footnotesize, linenos]{python}
while mem[i] != 0:
    mem[i] = (mem[i] + 2) & 0xFF
    mem[i + 1] = (mem[i + 1] + 0) & 0xFF
    mem[i] = (mem[i] + 254) & 0xFF
\end{minted}

Легко заметить, что в данном цикле, на каждой итерации значение mem[i] меняться не будет, а значит и условие выхода из цикла никогда не выполнится. Все, что нам остается, это убрать этот бесконечный цикл и весь код до него. После запуска, такая программа выведет искомый флаг.

\answerMath{CC\{Br4inf0ck\_ta5k\_num\_57123849\}.}



\section{Категория Stegano}

В заданиях данной категории, в некотором объеме данных (картинки, видео, аудиофайл и т.д.) предлагается найти информацию, спрятанную в нём таким образом, что на первый взгляд ничего особенного в данных файлах нет. Чтобы решить задание этой категории, необходимо понять и обнаружить, каким именно образом была спрятана информация.

\assignementTitle{Brain Damage}{1000}{}

Друг скинул мне этот текст вместе с улыбающимся смайликом. Хорошо зная своего друга, я предполагаю, что для раскрытия загадки этого сообщения нужно напрячь мозги на полную катушку. И хотя я уже потратил много времени на поиски решения, ничего больше странных символов и некоторой структуры в их расположении мне увидеть не удалось. Мне очень бы помогло, если бы вы хотя бы сказали мне, что представляет из себя этот текст...

Файл из задания доступен по ссылке: \url{https://cyberchallenge.rt.ru/files/85b80336a3cd3a1ee35739dab1601ba9/prog.txt}

\solutionSection

Исходный файл – программа на языке Brainfuck. Для запуска программы, можно воспользоваться любым подходящим онлайн сервисом. Видим, что программа выводит фразу «haha no flag here» и зацикливается.

Наиболее простой способ понять, почему же зацикливается программа, это перевести ее в код на другом, более читаемом языке, например, на Python. Для этого можно воспользоваться скриптом \url{https://www.nayuki.io/res/optimizing-brainfuck-compiler/bfc.py}.

В полученной программе на языке Python, бесконечный цикл будет начинаться в строке №68:

\begin{minted}[fontsize=\footnotesize, linenos]{python}
while mem[i] != 0:
    mem[i] = (mem[i] + 2) & 0xFF
    mem[i + 1] = (mem[i + 1] + 0) & 0xFF
    mem[i] = (mem[i] + 254) & 0xFF
\end{minted}

Легко заметить, что в данном цикле, на каждой итерации значение mem[i] меняться не будет, а значит и условие выхода из цикла никогда не выполнится. Все, что нам остается, это убрать этот бесконечный цикл и весь код до него. После запуска, такая программа выведет искомый флаг.

\answerMath{CC\{Br4inf0ck\_ta5k\_num\_57123849\}.}


\assignementTitle{Brain Damage}{1000}{}

Друг скинул мне этот текст вместе с улыбающимся смайликом. Хорошо зная своего друга, я предполагаю, что для раскрытия загадки этого сообщения нужно напрячь мозги на полную катушку. И хотя я уже потратил много времени на поиски решения, ничего больше странных символов и некоторой структуры в их расположении мне увидеть не удалось. Мне очень бы помогло, если бы вы хотя бы сказали мне, что представляет из себя этот текст...

Файл из задания доступен по ссылке: \url{https://cyberchallenge.rt.ru/files/85b80336a3cd3a1ee35739dab1601ba9/prog.txt}

\solutionSection

Исходный файл – программа на языке Brainfuck. Для запуска программы, можно воспользоваться любым подходящим онлайн сервисом. Видим, что программа выводит фразу «haha no flag here» и зацикливается.

Наиболее простой способ понять, почему же зацикливается программа, это перевести ее в код на другом, более читаемом языке, например, на Python. Для этого можно воспользоваться скриптом \url{https://www.nayuki.io/res/optimizing-brainfuck-compiler/bfc.py}.

В полученной программе на языке Python, бесконечный цикл будет начинаться в строке №68:

\begin{minted}[fontsize=\footnotesize, linenos]{python}
while mem[i] != 0:
    mem[i] = (mem[i] + 2) & 0xFF
    mem[i + 1] = (mem[i + 1] + 0) & 0xFF
    mem[i] = (mem[i] + 254) & 0xFF
\end{minted}

Легко заметить, что в данном цикле, на каждой итерации значение mem[i] меняться не будет, а значит и условие выхода из цикла никогда не выполнится. Все, что нам остается, это убрать этот бесконечный цикл и весь код до него. После запуска, такая программа выведет искомый флаг.

\answerMath{CC\{Br4inf0ck\_ta5k\_num\_57123849\}.}


\assignementTitle{Brain Damage}{1000}{}

Друг скинул мне этот текст вместе с улыбающимся смайликом. Хорошо зная своего друга, я предполагаю, что для раскрытия загадки этого сообщения нужно напрячь мозги на полную катушку. И хотя я уже потратил много времени на поиски решения, ничего больше странных символов и некоторой структуры в их расположении мне увидеть не удалось. Мне очень бы помогло, если бы вы хотя бы сказали мне, что представляет из себя этот текст...

Файл из задания доступен по ссылке: \url{https://cyberchallenge.rt.ru/files/85b80336a3cd3a1ee35739dab1601ba9/prog.txt}

\solutionSection

Исходный файл – программа на языке Brainfuck. Для запуска программы, можно воспользоваться любым подходящим онлайн сервисом. Видим, что программа выводит фразу «haha no flag here» и зацикливается.

Наиболее простой способ понять, почему же зацикливается программа, это перевести ее в код на другом, более читаемом языке, например, на Python. Для этого можно воспользоваться скриптом \url{https://www.nayuki.io/res/optimizing-brainfuck-compiler/bfc.py}.

В полученной программе на языке Python, бесконечный цикл будет начинаться в строке №68:

\begin{minted}[fontsize=\footnotesize, linenos]{python}
while mem[i] != 0:
    mem[i] = (mem[i] + 2) & 0xFF
    mem[i + 1] = (mem[i + 1] + 0) & 0xFF
    mem[i] = (mem[i] + 254) & 0xFF
\end{minted}

Легко заметить, что в данном цикле, на каждой итерации значение mem[i] меняться не будет, а значит и условие выхода из цикла никогда не выполнится. Все, что нам остается, это убрать этот бесконечный цикл и весь код до него. После запуска, такая программа выведет искомый флаг.

\answerMath{CC\{Br4inf0ck\_ta5k\_num\_57123849\}.}



\section{Категория Reverse}

Задания, для решения которых необходимо уметь внимательно и аккуратно вникать в логику работы программы, чаще всего, не имея её исходного кода. Но случается, что участникам даётся обфусцированный исходный код, разобрать работу которого задача также не из самых простых. Форматы файлов могут быть самыми различными: PE, ELF, Mach-O, APK или даже просто байт-код программы для некоей виртуальной машины, спецификация которой дается участникам.

В заданиях этой категории обычно флаг спрятан где-то внутри программы в зашифрованном виде. И либо необходимо восстановить и переписать алгоритм шифрования, чтобы расшифровать флаг, либо выполнить некоторые действия (например, записать определенное значение в реестр), чтобы программа в дальнейшем сама расшифровала и вывела на экран флаг.

\assignementTitle{Brain Damage}{1000}{}

Друг скинул мне этот текст вместе с улыбающимся смайликом. Хорошо зная своего друга, я предполагаю, что для раскрытия загадки этого сообщения нужно напрячь мозги на полную катушку. И хотя я уже потратил много времени на поиски решения, ничего больше странных символов и некоторой структуры в их расположении мне увидеть не удалось. Мне очень бы помогло, если бы вы хотя бы сказали мне, что представляет из себя этот текст...

Файл из задания доступен по ссылке: \url{https://cyberchallenge.rt.ru/files/85b80336a3cd3a1ee35739dab1601ba9/prog.txt}

\solutionSection

Исходный файл – программа на языке Brainfuck. Для запуска программы, можно воспользоваться любым подходящим онлайн сервисом. Видим, что программа выводит фразу «haha no flag here» и зацикливается.

Наиболее простой способ понять, почему же зацикливается программа, это перевести ее в код на другом, более читаемом языке, например, на Python. Для этого можно воспользоваться скриптом \url{https://www.nayuki.io/res/optimizing-brainfuck-compiler/bfc.py}.

В полученной программе на языке Python, бесконечный цикл будет начинаться в строке №68:

\begin{minted}[fontsize=\footnotesize, linenos]{python}
while mem[i] != 0:
    mem[i] = (mem[i] + 2) & 0xFF
    mem[i + 1] = (mem[i + 1] + 0) & 0xFF
    mem[i] = (mem[i] + 254) & 0xFF
\end{minted}

Легко заметить, что в данном цикле, на каждой итерации значение mem[i] меняться не будет, а значит и условие выхода из цикла никогда не выполнится. Все, что нам остается, это убрать этот бесконечный цикл и весь код до него. После запуска, такая программа выведет искомый флаг.

\answerMath{CC\{Br4inf0ck\_ta5k\_num\_57123849\}.}


\assignementTitle{Brain Damage}{1000}{}

Друг скинул мне этот текст вместе с улыбающимся смайликом. Хорошо зная своего друга, я предполагаю, что для раскрытия загадки этого сообщения нужно напрячь мозги на полную катушку. И хотя я уже потратил много времени на поиски решения, ничего больше странных символов и некоторой структуры в их расположении мне увидеть не удалось. Мне очень бы помогло, если бы вы хотя бы сказали мне, что представляет из себя этот текст...

Файл из задания доступен по ссылке: \url{https://cyberchallenge.rt.ru/files/85b80336a3cd3a1ee35739dab1601ba9/prog.txt}

\solutionSection

Исходный файл – программа на языке Brainfuck. Для запуска программы, можно воспользоваться любым подходящим онлайн сервисом. Видим, что программа выводит фразу «haha no flag here» и зацикливается.

Наиболее простой способ понять, почему же зацикливается программа, это перевести ее в код на другом, более читаемом языке, например, на Python. Для этого можно воспользоваться скриптом \url{https://www.nayuki.io/res/optimizing-brainfuck-compiler/bfc.py}.

В полученной программе на языке Python, бесконечный цикл будет начинаться в строке №68:

\begin{minted}[fontsize=\footnotesize, linenos]{python}
while mem[i] != 0:
    mem[i] = (mem[i] + 2) & 0xFF
    mem[i + 1] = (mem[i + 1] + 0) & 0xFF
    mem[i] = (mem[i] + 254) & 0xFF
\end{minted}

Легко заметить, что в данном цикле, на каждой итерации значение mem[i] меняться не будет, а значит и условие выхода из цикла никогда не выполнится. Все, что нам остается, это убрать этот бесконечный цикл и весь код до него. После запуска, такая программа выведет искомый флаг.

\answerMath{CC\{Br4inf0ck\_ta5k\_num\_57123849\}.}


\assignementTitle{Brain Damage}{1000}{}

Друг скинул мне этот текст вместе с улыбающимся смайликом. Хорошо зная своего друга, я предполагаю, что для раскрытия загадки этого сообщения нужно напрячь мозги на полную катушку. И хотя я уже потратил много времени на поиски решения, ничего больше странных символов и некоторой структуры в их расположении мне увидеть не удалось. Мне очень бы помогло, если бы вы хотя бы сказали мне, что представляет из себя этот текст...

Файл из задания доступен по ссылке: \url{https://cyberchallenge.rt.ru/files/85b80336a3cd3a1ee35739dab1601ba9/prog.txt}

\solutionSection

Исходный файл – программа на языке Brainfuck. Для запуска программы, можно воспользоваться любым подходящим онлайн сервисом. Видим, что программа выводит фразу «haha no flag here» и зацикливается.

Наиболее простой способ понять, почему же зацикливается программа, это перевести ее в код на другом, более читаемом языке, например, на Python. Для этого можно воспользоваться скриптом \url{https://www.nayuki.io/res/optimizing-brainfuck-compiler/bfc.py}.

В полученной программе на языке Python, бесконечный цикл будет начинаться в строке №68:

\begin{minted}[fontsize=\footnotesize, linenos]{python}
while mem[i] != 0:
    mem[i] = (mem[i] + 2) & 0xFF
    mem[i + 1] = (mem[i + 1] + 0) & 0xFF
    mem[i] = (mem[i] + 254) & 0xFF
\end{minted}

Легко заметить, что в данном цикле, на каждой итерации значение mem[i] меняться не будет, а значит и условие выхода из цикла никогда не выполнится. Все, что нам остается, это убрать этот бесконечный цикл и весь код до него. После запуска, такая программа выведет искомый флаг.

\answerMath{CC\{Br4inf0ck\_ta5k\_num\_57123849\}.}


\assignementTitle{Brain Damage}{1000}{}

Друг скинул мне этот текст вместе с улыбающимся смайликом. Хорошо зная своего друга, я предполагаю, что для раскрытия загадки этого сообщения нужно напрячь мозги на полную катушку. И хотя я уже потратил много времени на поиски решения, ничего больше странных символов и некоторой структуры в их расположении мне увидеть не удалось. Мне очень бы помогло, если бы вы хотя бы сказали мне, что представляет из себя этот текст...

Файл из задания доступен по ссылке: \url{https://cyberchallenge.rt.ru/files/85b80336a3cd3a1ee35739dab1601ba9/prog.txt}

\solutionSection

Исходный файл – программа на языке Brainfuck. Для запуска программы, можно воспользоваться любым подходящим онлайн сервисом. Видим, что программа выводит фразу «haha no flag here» и зацикливается.

Наиболее простой способ понять, почему же зацикливается программа, это перевести ее в код на другом, более читаемом языке, например, на Python. Для этого можно воспользоваться скриптом \url{https://www.nayuki.io/res/optimizing-brainfuck-compiler/bfc.py}.

В полученной программе на языке Python, бесконечный цикл будет начинаться в строке №68:

\begin{minted}[fontsize=\footnotesize, linenos]{python}
while mem[i] != 0:
    mem[i] = (mem[i] + 2) & 0xFF
    mem[i + 1] = (mem[i + 1] + 0) & 0xFF
    mem[i] = (mem[i] + 254) & 0xFF
\end{minted}

Легко заметить, что в данном цикле, на каждой итерации значение mem[i] меняться не будет, а значит и условие выхода из цикла никогда не выполнится. Все, что нам остается, это убрать этот бесконечный цикл и весь код до него. После запуска, такая программа выведет искомый флаг.

\answerMath{CC\{Br4inf0ck\_ta5k\_num\_57123849\}.}


\assignementTitle{Brain Damage}{1000}{}

Друг скинул мне этот текст вместе с улыбающимся смайликом. Хорошо зная своего друга, я предполагаю, что для раскрытия загадки этого сообщения нужно напрячь мозги на полную катушку. И хотя я уже потратил много времени на поиски решения, ничего больше странных символов и некоторой структуры в их расположении мне увидеть не удалось. Мне очень бы помогло, если бы вы хотя бы сказали мне, что представляет из себя этот текст...

Файл из задания доступен по ссылке: \url{https://cyberchallenge.rt.ru/files/85b80336a3cd3a1ee35739dab1601ba9/prog.txt}

\solutionSection

Исходный файл – программа на языке Brainfuck. Для запуска программы, можно воспользоваться любым подходящим онлайн сервисом. Видим, что программа выводит фразу «haha no flag here» и зацикливается.

Наиболее простой способ понять, почему же зацикливается программа, это перевести ее в код на другом, более читаемом языке, например, на Python. Для этого можно воспользоваться скриптом \url{https://www.nayuki.io/res/optimizing-brainfuck-compiler/bfc.py}.

В полученной программе на языке Python, бесконечный цикл будет начинаться в строке №68:

\begin{minted}[fontsize=\footnotesize, linenos]{python}
while mem[i] != 0:
    mem[i] = (mem[i] + 2) & 0xFF
    mem[i + 1] = (mem[i + 1] + 0) & 0xFF
    mem[i] = (mem[i] + 254) & 0xFF
\end{minted}

Легко заметить, что в данном цикле, на каждой итерации значение mem[i] меняться не будет, а значит и условие выхода из цикла никогда не выполнится. Все, что нам остается, это убрать этот бесконечный цикл и весь код до него. После запуска, такая программа выведет искомый флаг.

\answerMath{CC\{Br4inf0ck\_ta5k\_num\_57123849\}.}


\assignementTitle{Brain Damage}{1000}{}

Друг скинул мне этот текст вместе с улыбающимся смайликом. Хорошо зная своего друга, я предполагаю, что для раскрытия загадки этого сообщения нужно напрячь мозги на полную катушку. И хотя я уже потратил много времени на поиски решения, ничего больше странных символов и некоторой структуры в их расположении мне увидеть не удалось. Мне очень бы помогло, если бы вы хотя бы сказали мне, что представляет из себя этот текст...

Файл из задания доступен по ссылке: \url{https://cyberchallenge.rt.ru/files/85b80336a3cd3a1ee35739dab1601ba9/prog.txt}

\solutionSection

Исходный файл – программа на языке Brainfuck. Для запуска программы, можно воспользоваться любым подходящим онлайн сервисом. Видим, что программа выводит фразу «haha no flag here» и зацикливается.

Наиболее простой способ понять, почему же зацикливается программа, это перевести ее в код на другом, более читаемом языке, например, на Python. Для этого можно воспользоваться скриптом \url{https://www.nayuki.io/res/optimizing-brainfuck-compiler/bfc.py}.

В полученной программе на языке Python, бесконечный цикл будет начинаться в строке №68:

\begin{minted}[fontsize=\footnotesize, linenos]{python}
while mem[i] != 0:
    mem[i] = (mem[i] + 2) & 0xFF
    mem[i + 1] = (mem[i + 1] + 0) & 0xFF
    mem[i] = (mem[i] + 254) & 0xFF
\end{minted}

Легко заметить, что в данном цикле, на каждой итерации значение mem[i] меняться не будет, а значит и условие выхода из цикла никогда не выполнится. Все, что нам остается, это убрать этот бесконечный цикл и весь код до него. После запуска, такая программа выведет искомый флаг.

\answerMath{CC\{Br4inf0ck\_ta5k\_num\_57123849\}.}


\assignementTitle{Brain Damage}{1000}{}

Друг скинул мне этот текст вместе с улыбающимся смайликом. Хорошо зная своего друга, я предполагаю, что для раскрытия загадки этого сообщения нужно напрячь мозги на полную катушку. И хотя я уже потратил много времени на поиски решения, ничего больше странных символов и некоторой структуры в их расположении мне увидеть не удалось. Мне очень бы помогло, если бы вы хотя бы сказали мне, что представляет из себя этот текст...

Файл из задания доступен по ссылке: \url{https://cyberchallenge.rt.ru/files/85b80336a3cd3a1ee35739dab1601ba9/prog.txt}

\solutionSection

Исходный файл – программа на языке Brainfuck. Для запуска программы, можно воспользоваться любым подходящим онлайн сервисом. Видим, что программа выводит фразу «haha no flag here» и зацикливается.

Наиболее простой способ понять, почему же зацикливается программа, это перевести ее в код на другом, более читаемом языке, например, на Python. Для этого можно воспользоваться скриптом \url{https://www.nayuki.io/res/optimizing-brainfuck-compiler/bfc.py}.

В полученной программе на языке Python, бесконечный цикл будет начинаться в строке №68:

\begin{minted}[fontsize=\footnotesize, linenos]{python}
while mem[i] != 0:
    mem[i] = (mem[i] + 2) & 0xFF
    mem[i + 1] = (mem[i + 1] + 0) & 0xFF
    mem[i] = (mem[i] + 254) & 0xFF
\end{minted}

Легко заметить, что в данном цикле, на каждой итерации значение mem[i] меняться не будет, а значит и условие выхода из цикла никогда не выполнится. Все, что нам остается, это убрать этот бесконечный цикл и весь код до него. После запуска, такая программа выведет искомый флаг.

\answerMath{CC\{Br4inf0ck\_ta5k\_num\_57123849\}.}


\assignementTitle{Brain Damage}{1000}{}

Друг скинул мне этот текст вместе с улыбающимся смайликом. Хорошо зная своего друга, я предполагаю, что для раскрытия загадки этого сообщения нужно напрячь мозги на полную катушку. И хотя я уже потратил много времени на поиски решения, ничего больше странных символов и некоторой структуры в их расположении мне увидеть не удалось. Мне очень бы помогло, если бы вы хотя бы сказали мне, что представляет из себя этот текст...

Файл из задания доступен по ссылке: \url{https://cyberchallenge.rt.ru/files/85b80336a3cd3a1ee35739dab1601ba9/prog.txt}

\solutionSection

Исходный файл – программа на языке Brainfuck. Для запуска программы, можно воспользоваться любым подходящим онлайн сервисом. Видим, что программа выводит фразу «haha no flag here» и зацикливается.

Наиболее простой способ понять, почему же зацикливается программа, это перевести ее в код на другом, более читаемом языке, например, на Python. Для этого можно воспользоваться скриптом \url{https://www.nayuki.io/res/optimizing-brainfuck-compiler/bfc.py}.

В полученной программе на языке Python, бесконечный цикл будет начинаться в строке №68:

\begin{minted}[fontsize=\footnotesize, linenos]{python}
while mem[i] != 0:
    mem[i] = (mem[i] + 2) & 0xFF
    mem[i + 1] = (mem[i + 1] + 0) & 0xFF
    mem[i] = (mem[i] + 254) & 0xFF
\end{minted}

Легко заметить, что в данном цикле, на каждой итерации значение mem[i] меняться не будет, а значит и условие выхода из цикла никогда не выполнится. Все, что нам остается, это убрать этот бесконечный цикл и весь код до него. После запуска, такая программа выведет искомый флаг.

\answerMath{CC\{Br4inf0ck\_ta5k\_num\_57123849\}.}



\section{Категория Forensic}

Компьютерная криминалистика – прикладная наука о поиске и исследовании доказательств совершения различных действий, связанных с компьютерной информацией.
Чтобы справляться с задачами компьютерной криминалистики, надо иметь представление об основах работы ОС, понимать строение файловой системы и взаимодействие различных процессов. В ходе работы необходимо анализировать образы дисков, дампы памяти, дампы сетевых пакетов, а также логи и т.п.

\assignementTitle{Brain Damage}{1000}{}

Друг скинул мне этот текст вместе с улыбающимся смайликом. Хорошо зная своего друга, я предполагаю, что для раскрытия загадки этого сообщения нужно напрячь мозги на полную катушку. И хотя я уже потратил много времени на поиски решения, ничего больше странных символов и некоторой структуры в их расположении мне увидеть не удалось. Мне очень бы помогло, если бы вы хотя бы сказали мне, что представляет из себя этот текст...

Файл из задания доступен по ссылке: \url{https://cyberchallenge.rt.ru/files/85b80336a3cd3a1ee35739dab1601ba9/prog.txt}

\solutionSection

Исходный файл – программа на языке Brainfuck. Для запуска программы, можно воспользоваться любым подходящим онлайн сервисом. Видим, что программа выводит фразу «haha no flag here» и зацикливается.

Наиболее простой способ понять, почему же зацикливается программа, это перевести ее в код на другом, более читаемом языке, например, на Python. Для этого можно воспользоваться скриптом \url{https://www.nayuki.io/res/optimizing-brainfuck-compiler/bfc.py}.

В полученной программе на языке Python, бесконечный цикл будет начинаться в строке №68:

\begin{minted}[fontsize=\footnotesize, linenos]{python}
while mem[i] != 0:
    mem[i] = (mem[i] + 2) & 0xFF
    mem[i + 1] = (mem[i + 1] + 0) & 0xFF
    mem[i] = (mem[i] + 254) & 0xFF
\end{minted}

Легко заметить, что в данном цикле, на каждой итерации значение mem[i] меняться не будет, а значит и условие выхода из цикла никогда не выполнится. Все, что нам остается, это убрать этот бесконечный цикл и весь код до него. После запуска, такая программа выведет искомый флаг.

\answerMath{CC\{Br4inf0ck\_ta5k\_num\_57123849\}.}


\assignementTitle{Brain Damage}{1000}{}

Друг скинул мне этот текст вместе с улыбающимся смайликом. Хорошо зная своего друга, я предполагаю, что для раскрытия загадки этого сообщения нужно напрячь мозги на полную катушку. И хотя я уже потратил много времени на поиски решения, ничего больше странных символов и некоторой структуры в их расположении мне увидеть не удалось. Мне очень бы помогло, если бы вы хотя бы сказали мне, что представляет из себя этот текст...

Файл из задания доступен по ссылке: \url{https://cyberchallenge.rt.ru/files/85b80336a3cd3a1ee35739dab1601ba9/prog.txt}

\solutionSection

Исходный файл – программа на языке Brainfuck. Для запуска программы, можно воспользоваться любым подходящим онлайн сервисом. Видим, что программа выводит фразу «haha no flag here» и зацикливается.

Наиболее простой способ понять, почему же зацикливается программа, это перевести ее в код на другом, более читаемом языке, например, на Python. Для этого можно воспользоваться скриптом \url{https://www.nayuki.io/res/optimizing-brainfuck-compiler/bfc.py}.

В полученной программе на языке Python, бесконечный цикл будет начинаться в строке №68:

\begin{minted}[fontsize=\footnotesize, linenos]{python}
while mem[i] != 0:
    mem[i] = (mem[i] + 2) & 0xFF
    mem[i + 1] = (mem[i + 1] + 0) & 0xFF
    mem[i] = (mem[i] + 254) & 0xFF
\end{minted}

Легко заметить, что в данном цикле, на каждой итерации значение mem[i] меняться не будет, а значит и условие выхода из цикла никогда не выполнится. Все, что нам остается, это убрать этот бесконечный цикл и весь код до него. После запуска, такая программа выведет искомый флаг.

\answerMath{CC\{Br4inf0ck\_ta5k\_num\_57123849\}.}


\assignementTitle{Brain Damage}{1000}{}

Друг скинул мне этот текст вместе с улыбающимся смайликом. Хорошо зная своего друга, я предполагаю, что для раскрытия загадки этого сообщения нужно напрячь мозги на полную катушку. И хотя я уже потратил много времени на поиски решения, ничего больше странных символов и некоторой структуры в их расположении мне увидеть не удалось. Мне очень бы помогло, если бы вы хотя бы сказали мне, что представляет из себя этот текст...

Файл из задания доступен по ссылке: \url{https://cyberchallenge.rt.ru/files/85b80336a3cd3a1ee35739dab1601ba9/prog.txt}

\solutionSection

Исходный файл – программа на языке Brainfuck. Для запуска программы, можно воспользоваться любым подходящим онлайн сервисом. Видим, что программа выводит фразу «haha no flag here» и зацикливается.

Наиболее простой способ понять, почему же зацикливается программа, это перевести ее в код на другом, более читаемом языке, например, на Python. Для этого можно воспользоваться скриптом \url{https://www.nayuki.io/res/optimizing-brainfuck-compiler/bfc.py}.

В полученной программе на языке Python, бесконечный цикл будет начинаться в строке №68:

\begin{minted}[fontsize=\footnotesize, linenos]{python}
while mem[i] != 0:
    mem[i] = (mem[i] + 2) & 0xFF
    mem[i + 1] = (mem[i + 1] + 0) & 0xFF
    mem[i] = (mem[i] + 254) & 0xFF
\end{minted}

Легко заметить, что в данном цикле, на каждой итерации значение mem[i] меняться не будет, а значит и условие выхода из цикла никогда не выполнится. Все, что нам остается, это убрать этот бесконечный цикл и весь код до него. После запуска, такая программа выведет искомый флаг.

\answerMath{CC\{Br4inf0ck\_ta5k\_num\_57123849\}.}


\assignementTitle{Brain Damage}{1000}{}

Друг скинул мне этот текст вместе с улыбающимся смайликом. Хорошо зная своего друга, я предполагаю, что для раскрытия загадки этого сообщения нужно напрячь мозги на полную катушку. И хотя я уже потратил много времени на поиски решения, ничего больше странных символов и некоторой структуры в их расположении мне увидеть не удалось. Мне очень бы помогло, если бы вы хотя бы сказали мне, что представляет из себя этот текст...

Файл из задания доступен по ссылке: \url{https://cyberchallenge.rt.ru/files/85b80336a3cd3a1ee35739dab1601ba9/prog.txt}

\solutionSection

Исходный файл – программа на языке Brainfuck. Для запуска программы, можно воспользоваться любым подходящим онлайн сервисом. Видим, что программа выводит фразу «haha no flag here» и зацикливается.

Наиболее простой способ понять, почему же зацикливается программа, это перевести ее в код на другом, более читаемом языке, например, на Python. Для этого можно воспользоваться скриптом \url{https://www.nayuki.io/res/optimizing-brainfuck-compiler/bfc.py}.

В полученной программе на языке Python, бесконечный цикл будет начинаться в строке №68:

\begin{minted}[fontsize=\footnotesize, linenos]{python}
while mem[i] != 0:
    mem[i] = (mem[i] + 2) & 0xFF
    mem[i + 1] = (mem[i + 1] + 0) & 0xFF
    mem[i] = (mem[i] + 254) & 0xFF
\end{minted}

Легко заметить, что в данном цикле, на каждой итерации значение mem[i] меняться не будет, а значит и условие выхода из цикла никогда не выполнится. Все, что нам остается, это убрать этот бесконечный цикл и весь код до него. После запуска, такая программа выведет искомый флаг.

\answerMath{CC\{Br4inf0ck\_ta5k\_num\_57123849\}.}


\assignementTitle{Brain Damage}{1000}{}

Друг скинул мне этот текст вместе с улыбающимся смайликом. Хорошо зная своего друга, я предполагаю, что для раскрытия загадки этого сообщения нужно напрячь мозги на полную катушку. И хотя я уже потратил много времени на поиски решения, ничего больше странных символов и некоторой структуры в их расположении мне увидеть не удалось. Мне очень бы помогло, если бы вы хотя бы сказали мне, что представляет из себя этот текст...

Файл из задания доступен по ссылке: \url{https://cyberchallenge.rt.ru/files/85b80336a3cd3a1ee35739dab1601ba9/prog.txt}

\solutionSection

Исходный файл – программа на языке Brainfuck. Для запуска программы, можно воспользоваться любым подходящим онлайн сервисом. Видим, что программа выводит фразу «haha no flag here» и зацикливается.

Наиболее простой способ понять, почему же зацикливается программа, это перевести ее в код на другом, более читаемом языке, например, на Python. Для этого можно воспользоваться скриптом \url{https://www.nayuki.io/res/optimizing-brainfuck-compiler/bfc.py}.

В полученной программе на языке Python, бесконечный цикл будет начинаться в строке №68:

\begin{minted}[fontsize=\footnotesize, linenos]{python}
while mem[i] != 0:
    mem[i] = (mem[i] + 2) & 0xFF
    mem[i + 1] = (mem[i + 1] + 0) & 0xFF
    mem[i] = (mem[i] + 254) & 0xFF
\end{minted}

Легко заметить, что в данном цикле, на каждой итерации значение mem[i] меняться не будет, а значит и условие выхода из цикла никогда не выполнится. Все, что нам остается, это убрать этот бесконечный цикл и весь код до него. После запуска, такая программа выведет искомый флаг.

\answerMath{CC\{Br4inf0ck\_ta5k\_num\_57123849\}.}



\section{Категория Crypto}

Криптография – наука о том, как преобразовать исходные данные таким образом, чтобы обеспечить их защиту от посторонних, а также защитить от подмены или сделать её невозможной.

Задания данной категории предлагают применить свои знания в математике и криптоанализе для решения криптографических головоломок, будь то простой шифр замены или же некорректно использованный шифр RSA.

\assignementTitle{Brain Damage}{1000}{}

Друг скинул мне этот текст вместе с улыбающимся смайликом. Хорошо зная своего друга, я предполагаю, что для раскрытия загадки этого сообщения нужно напрячь мозги на полную катушку. И хотя я уже потратил много времени на поиски решения, ничего больше странных символов и некоторой структуры в их расположении мне увидеть не удалось. Мне очень бы помогло, если бы вы хотя бы сказали мне, что представляет из себя этот текст...

Файл из задания доступен по ссылке: \url{https://cyberchallenge.rt.ru/files/85b80336a3cd3a1ee35739dab1601ba9/prog.txt}

\solutionSection

Исходный файл – программа на языке Brainfuck. Для запуска программы, можно воспользоваться любым подходящим онлайн сервисом. Видим, что программа выводит фразу «haha no flag here» и зацикливается.

Наиболее простой способ понять, почему же зацикливается программа, это перевести ее в код на другом, более читаемом языке, например, на Python. Для этого можно воспользоваться скриптом \url{https://www.nayuki.io/res/optimizing-brainfuck-compiler/bfc.py}.

В полученной программе на языке Python, бесконечный цикл будет начинаться в строке №68:

\begin{minted}[fontsize=\footnotesize, linenos]{python}
while mem[i] != 0:
    mem[i] = (mem[i] + 2) & 0xFF
    mem[i + 1] = (mem[i + 1] + 0) & 0xFF
    mem[i] = (mem[i] + 254) & 0xFF
\end{minted}

Легко заметить, что в данном цикле, на каждой итерации значение mem[i] меняться не будет, а значит и условие выхода из цикла никогда не выполнится. Все, что нам остается, это убрать этот бесконечный цикл и весь код до него. После запуска, такая программа выведет искомый флаг.

\answerMath{CC\{Br4inf0ck\_ta5k\_num\_57123849\}.}


\assignementTitle{Brain Damage}{1000}{}

Друг скинул мне этот текст вместе с улыбающимся смайликом. Хорошо зная своего друга, я предполагаю, что для раскрытия загадки этого сообщения нужно напрячь мозги на полную катушку. И хотя я уже потратил много времени на поиски решения, ничего больше странных символов и некоторой структуры в их расположении мне увидеть не удалось. Мне очень бы помогло, если бы вы хотя бы сказали мне, что представляет из себя этот текст...

Файл из задания доступен по ссылке: \url{https://cyberchallenge.rt.ru/files/85b80336a3cd3a1ee35739dab1601ba9/prog.txt}

\solutionSection

Исходный файл – программа на языке Brainfuck. Для запуска программы, можно воспользоваться любым подходящим онлайн сервисом. Видим, что программа выводит фразу «haha no flag here» и зацикливается.

Наиболее простой способ понять, почему же зацикливается программа, это перевести ее в код на другом, более читаемом языке, например, на Python. Для этого можно воспользоваться скриптом \url{https://www.nayuki.io/res/optimizing-brainfuck-compiler/bfc.py}.

В полученной программе на языке Python, бесконечный цикл будет начинаться в строке №68:

\begin{minted}[fontsize=\footnotesize, linenos]{python}
while mem[i] != 0:
    mem[i] = (mem[i] + 2) & 0xFF
    mem[i + 1] = (mem[i + 1] + 0) & 0xFF
    mem[i] = (mem[i] + 254) & 0xFF
\end{minted}

Легко заметить, что в данном цикле, на каждой итерации значение mem[i] меняться не будет, а значит и условие выхода из цикла никогда не выполнится. Все, что нам остается, это убрать этот бесконечный цикл и весь код до него. После запуска, такая программа выведет искомый флаг.

\answerMath{CC\{Br4inf0ck\_ta5k\_num\_57123849\}.}


\assignementTitle{Brain Damage}{1000}{}

Друг скинул мне этот текст вместе с улыбающимся смайликом. Хорошо зная своего друга, я предполагаю, что для раскрытия загадки этого сообщения нужно напрячь мозги на полную катушку. И хотя я уже потратил много времени на поиски решения, ничего больше странных символов и некоторой структуры в их расположении мне увидеть не удалось. Мне очень бы помогло, если бы вы хотя бы сказали мне, что представляет из себя этот текст...

Файл из задания доступен по ссылке: \url{https://cyberchallenge.rt.ru/files/85b80336a3cd3a1ee35739dab1601ba9/prog.txt}

\solutionSection

Исходный файл – программа на языке Brainfuck. Для запуска программы, можно воспользоваться любым подходящим онлайн сервисом. Видим, что программа выводит фразу «haha no flag here» и зацикливается.

Наиболее простой способ понять, почему же зацикливается программа, это перевести ее в код на другом, более читаемом языке, например, на Python. Для этого можно воспользоваться скриптом \url{https://www.nayuki.io/res/optimizing-brainfuck-compiler/bfc.py}.

В полученной программе на языке Python, бесконечный цикл будет начинаться в строке №68:

\begin{minted}[fontsize=\footnotesize, linenos]{python}
while mem[i] != 0:
    mem[i] = (mem[i] + 2) & 0xFF
    mem[i + 1] = (mem[i + 1] + 0) & 0xFF
    mem[i] = (mem[i] + 254) & 0xFF
\end{minted}

Легко заметить, что в данном цикле, на каждой итерации значение mem[i] меняться не будет, а значит и условие выхода из цикла никогда не выполнится. Все, что нам остается, это убрать этот бесконечный цикл и весь код до него. После запуска, такая программа выведет искомый флаг.

\answerMath{CC\{Br4inf0ck\_ta5k\_num\_57123849\}.}


\assignementTitle{Brain Damage}{1000}{}

Друг скинул мне этот текст вместе с улыбающимся смайликом. Хорошо зная своего друга, я предполагаю, что для раскрытия загадки этого сообщения нужно напрячь мозги на полную катушку. И хотя я уже потратил много времени на поиски решения, ничего больше странных символов и некоторой структуры в их расположении мне увидеть не удалось. Мне очень бы помогло, если бы вы хотя бы сказали мне, что представляет из себя этот текст...

Файл из задания доступен по ссылке: \url{https://cyberchallenge.rt.ru/files/85b80336a3cd3a1ee35739dab1601ba9/prog.txt}

\solutionSection

Исходный файл – программа на языке Brainfuck. Для запуска программы, можно воспользоваться любым подходящим онлайн сервисом. Видим, что программа выводит фразу «haha no flag here» и зацикливается.

Наиболее простой способ понять, почему же зацикливается программа, это перевести ее в код на другом, более читаемом языке, например, на Python. Для этого можно воспользоваться скриптом \url{https://www.nayuki.io/res/optimizing-brainfuck-compiler/bfc.py}.

В полученной программе на языке Python, бесконечный цикл будет начинаться в строке №68:

\begin{minted}[fontsize=\footnotesize, linenos]{python}
while mem[i] != 0:
    mem[i] = (mem[i] + 2) & 0xFF
    mem[i + 1] = (mem[i + 1] + 0) & 0xFF
    mem[i] = (mem[i] + 254) & 0xFF
\end{minted}

Легко заметить, что в данном цикле, на каждой итерации значение mem[i] меняться не будет, а значит и условие выхода из цикла никогда не выполнится. Все, что нам остается, это убрать этот бесконечный цикл и весь код до него. После запуска, такая программа выведет искомый флаг.

\answerMath{CC\{Br4inf0ck\_ta5k\_num\_57123849\}.}


\assignementTitle{Brain Damage}{1000}{}

Друг скинул мне этот текст вместе с улыбающимся смайликом. Хорошо зная своего друга, я предполагаю, что для раскрытия загадки этого сообщения нужно напрячь мозги на полную катушку. И хотя я уже потратил много времени на поиски решения, ничего больше странных символов и некоторой структуры в их расположении мне увидеть не удалось. Мне очень бы помогло, если бы вы хотя бы сказали мне, что представляет из себя этот текст...

Файл из задания доступен по ссылке: \url{https://cyberchallenge.rt.ru/files/85b80336a3cd3a1ee35739dab1601ba9/prog.txt}

\solutionSection

Исходный файл – программа на языке Brainfuck. Для запуска программы, можно воспользоваться любым подходящим онлайн сервисом. Видим, что программа выводит фразу «haha no flag here» и зацикливается.

Наиболее простой способ понять, почему же зацикливается программа, это перевести ее в код на другом, более читаемом языке, например, на Python. Для этого можно воспользоваться скриптом \url{https://www.nayuki.io/res/optimizing-brainfuck-compiler/bfc.py}.

В полученной программе на языке Python, бесконечный цикл будет начинаться в строке №68:

\begin{minted}[fontsize=\footnotesize, linenos]{python}
while mem[i] != 0:
    mem[i] = (mem[i] + 2) & 0xFF
    mem[i + 1] = (mem[i + 1] + 0) & 0xFF
    mem[i] = (mem[i] + 254) & 0xFF
\end{minted}

Легко заметить, что в данном цикле, на каждой итерации значение mem[i] меняться не будет, а значит и условие выхода из цикла никогда не выполнится. Все, что нам остается, это убрать этот бесконечный цикл и весь код до него. После запуска, такая программа выведет искомый флаг.

\answerMath{CC\{Br4inf0ck\_ta5k\_num\_57123849\}.}



\section{Категория PPC}

Требуется применение навыков программирования и знания алгоритмов, например, чтобы восстановить некий исходный файл или написать бота для прохождения лабиринта.

\assignementTitle{Brain Damage}{1000}{}

Друг скинул мне этот текст вместе с улыбающимся смайликом. Хорошо зная своего друга, я предполагаю, что для раскрытия загадки этого сообщения нужно напрячь мозги на полную катушку. И хотя я уже потратил много времени на поиски решения, ничего больше странных символов и некоторой структуры в их расположении мне увидеть не удалось. Мне очень бы помогло, если бы вы хотя бы сказали мне, что представляет из себя этот текст...

Файл из задания доступен по ссылке: \url{https://cyberchallenge.rt.ru/files/85b80336a3cd3a1ee35739dab1601ba9/prog.txt}

\solutionSection

Исходный файл – программа на языке Brainfuck. Для запуска программы, можно воспользоваться любым подходящим онлайн сервисом. Видим, что программа выводит фразу «haha no flag here» и зацикливается.

Наиболее простой способ понять, почему же зацикливается программа, это перевести ее в код на другом, более читаемом языке, например, на Python. Для этого можно воспользоваться скриптом \url{https://www.nayuki.io/res/optimizing-brainfuck-compiler/bfc.py}.

В полученной программе на языке Python, бесконечный цикл будет начинаться в строке №68:

\begin{minted}[fontsize=\footnotesize, linenos]{python}
while mem[i] != 0:
    mem[i] = (mem[i] + 2) & 0xFF
    mem[i + 1] = (mem[i + 1] + 0) & 0xFF
    mem[i] = (mem[i] + 254) & 0xFF
\end{minted}

Легко заметить, что в данном цикле, на каждой итерации значение mem[i] меняться не будет, а значит и условие выхода из цикла никогда не выполнится. Все, что нам остается, это убрать этот бесконечный цикл и весь код до него. После запуска, такая программа выведет искомый флаг.

\answerMath{CC\{Br4inf0ck\_ta5k\_num\_57123849\}.}


\assignementTitle{Brain Damage}{1000}{}

Друг скинул мне этот текст вместе с улыбающимся смайликом. Хорошо зная своего друга, я предполагаю, что для раскрытия загадки этого сообщения нужно напрячь мозги на полную катушку. И хотя я уже потратил много времени на поиски решения, ничего больше странных символов и некоторой структуры в их расположении мне увидеть не удалось. Мне очень бы помогло, если бы вы хотя бы сказали мне, что представляет из себя этот текст...

Файл из задания доступен по ссылке: \url{https://cyberchallenge.rt.ru/files/85b80336a3cd3a1ee35739dab1601ba9/prog.txt}

\solutionSection

Исходный файл – программа на языке Brainfuck. Для запуска программы, можно воспользоваться любым подходящим онлайн сервисом. Видим, что программа выводит фразу «haha no flag here» и зацикливается.

Наиболее простой способ понять, почему же зацикливается программа, это перевести ее в код на другом, более читаемом языке, например, на Python. Для этого можно воспользоваться скриптом \url{https://www.nayuki.io/res/optimizing-brainfuck-compiler/bfc.py}.

В полученной программе на языке Python, бесконечный цикл будет начинаться в строке №68:

\begin{minted}[fontsize=\footnotesize, linenos]{python}
while mem[i] != 0:
    mem[i] = (mem[i] + 2) & 0xFF
    mem[i + 1] = (mem[i + 1] + 0) & 0xFF
    mem[i] = (mem[i] + 254) & 0xFF
\end{minted}

Легко заметить, что в данном цикле, на каждой итерации значение mem[i] меняться не будет, а значит и условие выхода из цикла никогда не выполнится. Все, что нам остается, это убрать этот бесконечный цикл и весь код до него. После запуска, такая программа выведет искомый флаг.

\answerMath{CC\{Br4inf0ck\_ta5k\_num\_57123849\}.}


\assignementTitle{Brain Damage}{1000}{}

Друг скинул мне этот текст вместе с улыбающимся смайликом. Хорошо зная своего друга, я предполагаю, что для раскрытия загадки этого сообщения нужно напрячь мозги на полную катушку. И хотя я уже потратил много времени на поиски решения, ничего больше странных символов и некоторой структуры в их расположении мне увидеть не удалось. Мне очень бы помогло, если бы вы хотя бы сказали мне, что представляет из себя этот текст...

Файл из задания доступен по ссылке: \url{https://cyberchallenge.rt.ru/files/85b80336a3cd3a1ee35739dab1601ba9/prog.txt}

\solutionSection

Исходный файл – программа на языке Brainfuck. Для запуска программы, можно воспользоваться любым подходящим онлайн сервисом. Видим, что программа выводит фразу «haha no flag here» и зацикливается.

Наиболее простой способ понять, почему же зацикливается программа, это перевести ее в код на другом, более читаемом языке, например, на Python. Для этого можно воспользоваться скриптом \url{https://www.nayuki.io/res/optimizing-brainfuck-compiler/bfc.py}.

В полученной программе на языке Python, бесконечный цикл будет начинаться в строке №68:

\begin{minted}[fontsize=\footnotesize, linenos]{python}
while mem[i] != 0:
    mem[i] = (mem[i] + 2) & 0xFF
    mem[i + 1] = (mem[i + 1] + 0) & 0xFF
    mem[i] = (mem[i] + 254) & 0xFF
\end{minted}

Легко заметить, что в данном цикле, на каждой итерации значение mem[i] меняться не будет, а значит и условие выхода из цикла никогда не выполнится. Все, что нам остается, это убрать этот бесконечный цикл и весь код до него. После запуска, такая программа выведет искомый флаг.

\answerMath{CC\{Br4inf0ck\_ta5k\_num\_57123849\}.}


\assignementTitle{Brain Damage}{1000}{}

Друг скинул мне этот текст вместе с улыбающимся смайликом. Хорошо зная своего друга, я предполагаю, что для раскрытия загадки этого сообщения нужно напрячь мозги на полную катушку. И хотя я уже потратил много времени на поиски решения, ничего больше странных символов и некоторой структуры в их расположении мне увидеть не удалось. Мне очень бы помогло, если бы вы хотя бы сказали мне, что представляет из себя этот текст...

Файл из задания доступен по ссылке: \url{https://cyberchallenge.rt.ru/files/85b80336a3cd3a1ee35739dab1601ba9/prog.txt}

\solutionSection

Исходный файл – программа на языке Brainfuck. Для запуска программы, можно воспользоваться любым подходящим онлайн сервисом. Видим, что программа выводит фразу «haha no flag here» и зацикливается.

Наиболее простой способ понять, почему же зацикливается программа, это перевести ее в код на другом, более читаемом языке, например, на Python. Для этого можно воспользоваться скриптом \url{https://www.nayuki.io/res/optimizing-brainfuck-compiler/bfc.py}.

В полученной программе на языке Python, бесконечный цикл будет начинаться в строке №68:

\begin{minted}[fontsize=\footnotesize, linenos]{python}
while mem[i] != 0:
    mem[i] = (mem[i] + 2) & 0xFF
    mem[i + 1] = (mem[i + 1] + 0) & 0xFF
    mem[i] = (mem[i] + 254) & 0xFF
\end{minted}

Легко заметить, что в данном цикле, на каждой итерации значение mem[i] меняться не будет, а значит и условие выхода из цикла никогда не выполнится. Все, что нам остается, это убрать этот бесконечный цикл и весь код до него. После запуска, такая программа выведет искомый флаг.

\answerMath{CC\{Br4inf0ck\_ta5k\_num\_57123849\}.}



\section{Категория Misc}

Различные задачи, не подходящие под другие категории.

\assignementTitle{Brain Damage}{1000}{}

Друг скинул мне этот текст вместе с улыбающимся смайликом. Хорошо зная своего друга, я предполагаю, что для раскрытия загадки этого сообщения нужно напрячь мозги на полную катушку. И хотя я уже потратил много времени на поиски решения, ничего больше странных символов и некоторой структуры в их расположении мне увидеть не удалось. Мне очень бы помогло, если бы вы хотя бы сказали мне, что представляет из себя этот текст...

Файл из задания доступен по ссылке: \url{https://cyberchallenge.rt.ru/files/85b80336a3cd3a1ee35739dab1601ba9/prog.txt}

\solutionSection

Исходный файл – программа на языке Brainfuck. Для запуска программы, можно воспользоваться любым подходящим онлайн сервисом. Видим, что программа выводит фразу «haha no flag here» и зацикливается.

Наиболее простой способ понять, почему же зацикливается программа, это перевести ее в код на другом, более читаемом языке, например, на Python. Для этого можно воспользоваться скриптом \url{https://www.nayuki.io/res/optimizing-brainfuck-compiler/bfc.py}.

В полученной программе на языке Python, бесконечный цикл будет начинаться в строке №68:

\begin{minted}[fontsize=\footnotesize, linenos]{python}
while mem[i] != 0:
    mem[i] = (mem[i] + 2) & 0xFF
    mem[i + 1] = (mem[i + 1] + 0) & 0xFF
    mem[i] = (mem[i] + 254) & 0xFF
\end{minted}

Легко заметить, что в данном цикле, на каждой итерации значение mem[i] меняться не будет, а значит и условие выхода из цикла никогда не выполнится. Все, что нам остается, это убрать этот бесконечный цикл и весь код до него. После запуска, такая программа выведет искомый флаг.

\answerMath{CC\{Br4inf0ck\_ta5k\_num\_57123849\}.}

