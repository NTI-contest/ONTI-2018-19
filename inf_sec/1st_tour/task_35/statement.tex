\assignementTitle{Alice}{1000}{}

Расшифруйте перехваченные сообщения, полученные с помощью скрипта alice.py

Файл из задания доступен по ссылке:	\url{https://cyberchallenge.rt.ru/files/6c15d633ba428aab36187d5acaa18f36/alice.zip}

\solutionSection

В архиве alice.zip хранится скрипт alice.py, 5 шифротекстов и 5 публичных ключей. Из alice.py становится понятно, что все шифротексты – это один и тот же флаг, зашифрованный алгоритмом RSA с различными публичными ключами.

Если внимательней посмотреть на открытые ключи, можно заметить, что открытая экспонента каждого из них равна 5. И, так как количество перехваченных шифротекстов равно открытой экспоненте, то можно применить атаку Хастада, основанную на использовании китайской теоремы об остатках. Реализацию атаки на языке Python можно легко найти на GitHub. 

\includeSolutionIfExistsByPath{1st_tour/task_35}