\assignementTitle{Intercepted}{1000}{}

Мы перехватили какой-то странный интернет-трафик. Нам кажется, эти люди что-то замышляют. Впрочем, люди ли?

Файл из задания доступен по ссылке: \url{https://cyberchallenge.rt.ru/files/c902c54cc105b9afb70ffc469b0686dc/intercepted.pcap}

\solutionSection

Исходный файл – дамп трафика в формате PCAP. Для его анализа целесообразно воспользоваться программой Wireshark. Поискав в содержимом пакетов строку «flag» можно обнаружить пакет, пришедший с адреса 159.69.59.245. Это листинг содержимого директории FTP-сервера, в которой лежит файл flag.txt.

\putImgWOCaption{16cm}{1st_tour/task_27/1}

Далее, нужно выяснить, с какими учетными данными можно авторизоваться на этом FTP-сервере, чтобы прочитать файл flag.txt. Для этого с помощью фильтра оставим в списке только FTP-пакеты. В получившемся списке легко видеть, что пользователь в качестве логина ввел anunak, а пароля – subdue\_the\_humanity.

\putImgWOCaption{16cm}{1st_tour/task_27/2}

Чтобы получить флаг, нужно повторить действия пользователя, с машины которого получен дамп трафика. Т. е. используя любой FTP-клиент подключиться к серверу по адресу 159.69.59.245. Затем, используя для авторизации вышеуказанные логин и пароль, скачать файл flag.txt и открыть его в любом текстовом редакторе.

\answerMath{CC\{1\_s33\_wh47\_y0u\_d1d\_7h3r3\}.}