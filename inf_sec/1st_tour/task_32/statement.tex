\assignementTitle{Big Text}{1000}{}

Я просто хотел рассказать тебе про один из самых простых шифров, но что-то пошло не так.

Файл из задания доступен по ссылке: \url{https://cyberchallenge.rt.ru/files/c3099ddfe58cb7c773d1b92d828044a4/text.txt}

\solutionSection

В данном шифротексте, слова все еще разделены пробелами, и используются только символы латинского алфавита. Можно предположить, что для шифрования использовался шифр перестановки или простой замены. Однако, если обратить внимание на частоту встречаемости символов в шифротексте, то вариант с перестановкой отпадет.

Если шифротекст достаточно большой, то можно использовать частотный анализ, чтобы заменить буквы шифротекста на буквы, имеющие такую же частоту встречаемости в открытом тексте на английском языке.  Для этого удобно воспользоваться программой Cryptool. Она также позволяет редактировать финальный вариант подстановки. Это бывает нужно, если шифротекст недостаточно большой, и некоторые буквы в нем имеют отличную от английского языка частоту встречаемости. 

\putImgWOCaption{10cm}{1st_tour/task_32/1}

В результате получаем открытый текст, в конце которого есть фраза:\\
flag for this task is cc curly bracket cryp seven four n four ly five one five underscore one five underscore five zero underscore c zero zero l curly bracket

Заменив слова на соответствующие символы, получаем флаг.

\answerMath{CC\{cryp74n4ly515\_15\_50\_c00l\}.}
