\assignementTitle{Difference}{1000}{}

bliss.bmp: 216d20df62af34d39089e066d1d4b3af

Файл из задания доступен по ссылке: \url{https://cyberchallenge.rt.ru/files/4b6b410010968becb8bce3621f1c7871/bliss_new.bmp}.

\solutionSection

Текст задания подсказывает нам название оригинального файла изображения и его MD5 хеш-сумму. Такое изображение легко находится в сети Интернет.

Название задания подсказывает нам, что в первую очередь стоит понять, чем отличаются изображения. Для этого, например, можно воспользоваться программой Stegsolve.

\putImgWOCaption{16cm}{1st_tour/task_16/1}

Как видно, у изображений не совпадают некоторые пиксели в начале 1-го ряда. Начиная с пикселя (0, 0) и двигаясь по ряду вправо, запишем вместо совпадающих пикселей 0, а вместо несовпадающих – 1. Получим следующую бинарную последовательность:\\

0100001101000011011110110110010000110001010001100100011001011111011011010011010000\\1101110011011100110011011100100011010101111101000010100...00

Разбив последовательность на байты (по 8 бит), заменим их на символы согласно кодировке ASCII и получим флаг.

\includeSolutionIfExistsByPath{1st_tour/inf_sec/1st_tour/task_16}
