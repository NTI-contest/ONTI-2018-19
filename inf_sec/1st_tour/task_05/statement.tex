\assignementTitle{Silmarill Store}{}{}

Мы открыли новый интернет-магазин. Достаточно ли у тебя денег, чтобы купить самый дорогой товар?

Ссылка на наш интернет-магазин: \url{http://silmarilstore.2018.cyberchallenge.ru/}

Внимание! В этом интернет-магазине можно обнаружить целых два флага. Другой флаг надо сдавать во вторую задачу.

\solutionSection

Данный сервис представляет из себя интернет магазин с двумя товарами. На счету пользователя изначально 100 единиц некоторой валюты. Самый дорогой предмет стоит 999999 единиц валюты, самый дешевый 1 единицу валюты. Путем логических рассуждений можно догадаться, что задача состоит в том, чтобы купить предмет, на который не хватает денег.

Посмотрев код главной страницы сайта (как в задаче No Comments), или перехватив HTTP-запрос при покупке (например, с помощью утилиты Burp Suite: \url{https://portswigger.net/burp}), можно увидеть, что цена предмета передается на сервер с клиентской части.

\putImgWOCaption{16cm}{1st_tour/task_05/1}

Поменять это значение также можно с помощью инструментов разработчика в браузере. В браузере Chrome для этого нужно нажать правой кнопкой мыши на любом месте экрана и выбрать раздел меню “Inspect”. В появившемся окне нужно найти данный тег input и поменять значение цены самого дорогого предмета. Сделать это можно двойным нажатием на атрибут тега.

\putImgWOCaption{16cm}{1st_tour/task_05/2}

Дальше требуется “купить” этот предмет, нажав на обычную кнопку “Купить”.

После покупки предмета пользователь попадает на страницу с флагом.

\putImgWOCaption{16cm}{1st_tour/task_05/3}

\answerMath{CC\{lembas\_elvish\_waybread\}.}