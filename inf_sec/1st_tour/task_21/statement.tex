\assignementTitle{Cobra}{1000}{}

В диких условиях продолжительность жизни кобр составляет в среднем 20 лет. Но у тебя есть всего 5 дней, чтобы решить эту задачу.

Файл из задания доступен по ссылке: \url{https://cyberchallenge.rt.ru/files/8a4e2531ce3c70165d58ec763b9be177/cobra.py}

\solutionSection

Исходный файл – скрипт на языке Python. Все, что он делает, это загружает и исполняет некий объект из base64-строки. Таким образом, чтобы понять, как получить флаг, нужно заглянут внутрь этого объекта.

Сначала нужно понять, где в этом объекте код. Сделать это можно немного модифицировав исходный скрипт:

\begin{minted}[fontsize=\footnotesize, linenos]{python}
s = '<строка из исходного файла>'
code = marshal.loads(base64.b64decode(s))
for item in code.co_consts: 
    print('%s: %r' % (type(item), item))
\end{minted}

Результат работы скрипта следующий:\\
<type 'int'>: -1\\
<type 'tuple'>: ('AES',)\\
<type 'NoneType'>: None\\
<type 'code'>: <code object check at 0000000005B0E830, file "<string>", line 6>\\
<type 'code'>: <code object generate\_flag at 0000000005B71330, file "<string>", line 13>\\
<type 'str'>: 'enter serial: '

Видим, что в объекте есть 2 объекта с кодом – функции check и generate\_flag. Чтобы посмотреть их код, удобно воспользоваться библиотекой uncompyle6. Дополним наш модифицированный скрипт следующими строками:

\begin{minted}[fontsize=\footnotesize, linenos]{python}
uncompyle6.main.decompile(2.7, code.co_consts[3], sys.stdout)
uncompyle6.main.decompile(2.7, code.co_consts[4], sys.stdout)
\end{minted}

Результат:
\begin{minted}[fontsize=\footnotesize, linenos]{python}
# uncompyle6 version 3.3.1
# Python bytecode 2.7
# Decompiled from: Python 2.7.16 (v2.7.16:413a49145e, Mar  4 2019, 01:37:19) [MSC v.1500 64 bit (AMD64)]
# Embedded file name: <string>
expressions = [
 '1790 + 1543', '1234 * 3', '9999 - 1337', '2048 // 2', '3 ** 8']
for index, value in enumerate(serial.split('-')):
    if eval(expressions[index]) != int(value):
        return False

return True

# uncompyle6 version 3.3.1
# Python bytecode 2.7
# Decompiled from: Python 2.7.16 (v2.7.16:413a49145e, Mar  4 2019, 01:37:19) [MSC v.1500 64 bit (AMD64)]
# Embedded file name: <string>
cipher = AES.new(serial, AES.MODE_ECB)
decoded = cipher.decrypt(base64.b64decode('0P8pV0G6WlqUxuuKNk+y4N5PTfamGAln9gDhXDxi5rM='))
return decoded.strip()
\end{minted}

Проанализировав эти 2 функции, несложно понять, что флаг можно получить, расшифровав base64-строку из функции generate\_flag. В качестве ключа для AES выступает строка, содержащая числа, полученные в результате вычисления арифметических выражений из списка expressions, объединенные через символ «-». Скрипт, расшифровывающий флаг, выглядит так:

\begin{minted}[fontsize=\footnotesize, linenos]{python}
from Crypto.Cipher import AES
import base64

expressions = ['1790 + 1543', '1234 * 3', '9999 - 1337', '2048 // 2', '3 ** 8']
serial = '-'.join([str(eval(x)) for x in expressions])

cipher = AES.new(serial, AES.MODE_ECB)
decoded = cipher.decrypt(base64.b64decode('0P8pV0G6WlqUxuuKNk+y4N5PTfamGAln9gDhXDxi5rM='))
print(decoded.strip())
\end{minted}

\answerMath{CC\{1\_60774\_5uch\_4\_l0n6\_5n4k3\}.}