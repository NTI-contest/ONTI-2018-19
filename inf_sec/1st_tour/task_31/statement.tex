\assignementTitle{Uroboros}{1000}{}

Наш скрипт, похоже, немного повредился. Но мы вспомнили, что в результате он выводил строку
\mint{console}|2E\x18fQ)61X@\x10j\x0bjJ+ <\x1fH\x0cuD/Ll|

Файл из задания доступен по ссылке: \url{https://cyberchallenge.rt.ru/files/97f4b4b3b9ca239d0708de73dd75b884/uroboros.py}

\solutionSection

Легко увидеть, что в скрипте используется рекурсивная функция, несколько раз выполняющая шифрование XOR с заданным ключом.

Так как $x \: xor \: y \: xor \: y = x$, чтобы получить исходный текст достаточно применить функцию шифрования к зашифрованному значению, указанному в описании задания.

\answerMath{CC\{w45\_17\_cryp70\_0r\_wh47?\}.}