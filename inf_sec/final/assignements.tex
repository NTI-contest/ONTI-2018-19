\section{Категория Web}

Задания данной категории предполагают поиск уязвимостей веб-приложения (веб-сайта) и дальнейшую их эксплуатацию.

Для этого необходимо знать основы разработки веб-приложений, понимать базовые принципы их проектирования, иметь представление о том, как работает тот или иной веб-фреймворк или CMS. Кроме того, важно понимать природу возникновения веб-уязвимостей, понимать и уметь эксплуатировать типовые уязвимости в веб-приложениях. Список наиболее распространенных веб-уязвимостей периодически публикуется в рамках проекта OWASP Top 10.

Примеры используемых в решениях задач инструментов:
\begin{itemize}
    \item Браузер и различные расширения для него
    \item Burp Suite
    \item nmap
    \item sqlmap
    \item httpie
\end{itemize}

Решения задач данной категории, как правило, начинаются с исследования данного веб-приложения: какие страницы есть на сайте, какой фреймворк или CMS были использованы и т. п. Далее нужно определить поверхность атаки, т. е. понять, как пользователь может влиять на работу веб-приложения. Иными словами, как он может передать данные и как они будут обработаны сайтом. Исходя из полученных данных уже можно определить возможные проблемные места и проверить их на типовые (и не только) уязвимости.

\assignementTitle{Brain Damage}{1000}{}

Друг скинул мне этот текст вместе с улыбающимся смайликом. Хорошо зная своего друга, я предполагаю, что для раскрытия загадки этого сообщения нужно напрячь мозги на полную катушку. И хотя я уже потратил много времени на поиски решения, ничего больше странных символов и некоторой структуры в их расположении мне увидеть не удалось. Мне очень бы помогло, если бы вы хотя бы сказали мне, что представляет из себя этот текст...

Файл из задания доступен по ссылке: \url{https://cyberchallenge.rt.ru/files/85b80336a3cd3a1ee35739dab1601ba9/prog.txt}

\solutionSection

Исходный файл – программа на языке Brainfuck. Для запуска программы, можно воспользоваться любым подходящим онлайн сервисом. Видим, что программа выводит фразу «haha no flag here» и зацикливается.

Наиболее простой способ понять, почему же зацикливается программа, это перевести ее в код на другом, более читаемом языке, например, на Python. Для этого можно воспользоваться скриптом \url{https://www.nayuki.io/res/optimizing-brainfuck-compiler/bfc.py}.

В полученной программе на языке Python, бесконечный цикл будет начинаться в строке №68:

\begin{minted}[fontsize=\footnotesize, linenos]{python}
while mem[i] != 0:
    mem[i] = (mem[i] + 2) & 0xFF
    mem[i + 1] = (mem[i + 1] + 0) & 0xFF
    mem[i] = (mem[i] + 254) & 0xFF
\end{minted}

Легко заметить, что в данном цикле, на каждой итерации значение mem[i] меняться не будет, а значит и условие выхода из цикла никогда не выполнится. Все, что нам остается, это убрать этот бесконечный цикл и весь код до него. После запуска, такая программа выведет искомый флаг.

\answerMath{CC\{Br4inf0ck\_ta5k\_num\_57123849\}.}


\assignementTitle{Brain Damage}{1000}{}

Друг скинул мне этот текст вместе с улыбающимся смайликом. Хорошо зная своего друга, я предполагаю, что для раскрытия загадки этого сообщения нужно напрячь мозги на полную катушку. И хотя я уже потратил много времени на поиски решения, ничего больше странных символов и некоторой структуры в их расположении мне увидеть не удалось. Мне очень бы помогло, если бы вы хотя бы сказали мне, что представляет из себя этот текст...

Файл из задания доступен по ссылке: \url{https://cyberchallenge.rt.ru/files/85b80336a3cd3a1ee35739dab1601ba9/prog.txt}

\solutionSection

Исходный файл – программа на языке Brainfuck. Для запуска программы, можно воспользоваться любым подходящим онлайн сервисом. Видим, что программа выводит фразу «haha no flag here» и зацикливается.

Наиболее простой способ понять, почему же зацикливается программа, это перевести ее в код на другом, более читаемом языке, например, на Python. Для этого можно воспользоваться скриптом \url{https://www.nayuki.io/res/optimizing-brainfuck-compiler/bfc.py}.

В полученной программе на языке Python, бесконечный цикл будет начинаться в строке №68:

\begin{minted}[fontsize=\footnotesize, linenos]{python}
while mem[i] != 0:
    mem[i] = (mem[i] + 2) & 0xFF
    mem[i + 1] = (mem[i + 1] + 0) & 0xFF
    mem[i] = (mem[i] + 254) & 0xFF
\end{minted}

Легко заметить, что в данном цикле, на каждой итерации значение mem[i] меняться не будет, а значит и условие выхода из цикла никогда не выполнится. Все, что нам остается, это убрать этот бесконечный цикл и весь код до него. После запуска, такая программа выведет искомый флаг.

\answerMath{CC\{Br4inf0ck\_ta5k\_num\_57123849\}.}


\assignementTitle{Brain Damage}{1000}{}

Друг скинул мне этот текст вместе с улыбающимся смайликом. Хорошо зная своего друга, я предполагаю, что для раскрытия загадки этого сообщения нужно напрячь мозги на полную катушку. И хотя я уже потратил много времени на поиски решения, ничего больше странных символов и некоторой структуры в их расположении мне увидеть не удалось. Мне очень бы помогло, если бы вы хотя бы сказали мне, что представляет из себя этот текст...

Файл из задания доступен по ссылке: \url{https://cyberchallenge.rt.ru/files/85b80336a3cd3a1ee35739dab1601ba9/prog.txt}

\solutionSection

Исходный файл – программа на языке Brainfuck. Для запуска программы, можно воспользоваться любым подходящим онлайн сервисом. Видим, что программа выводит фразу «haha no flag here» и зацикливается.

Наиболее простой способ понять, почему же зацикливается программа, это перевести ее в код на другом, более читаемом языке, например, на Python. Для этого можно воспользоваться скриптом \url{https://www.nayuki.io/res/optimizing-brainfuck-compiler/bfc.py}.

В полученной программе на языке Python, бесконечный цикл будет начинаться в строке №68:

\begin{minted}[fontsize=\footnotesize, linenos]{python}
while mem[i] != 0:
    mem[i] = (mem[i] + 2) & 0xFF
    mem[i + 1] = (mem[i + 1] + 0) & 0xFF
    mem[i] = (mem[i] + 254) & 0xFF
\end{minted}

Легко заметить, что в данном цикле, на каждой итерации значение mem[i] меняться не будет, а значит и условие выхода из цикла никогда не выполнится. Все, что нам остается, это убрать этот бесконечный цикл и весь код до него. После запуска, такая программа выведет искомый флаг.

\answerMath{CC\{Br4inf0ck\_ta5k\_num\_57123849\}.}



\section{Категория Crypto}

Криптография – наука о том, как преобразовать исходные данные таким образом, чтобы обеспечить их защиту от посторонних, а также защитить от подмены или сделать её невозможной.

Задания данной категории предлагают применить свои знания в математике и криптоанализе для решения криптографических головоломок, будь то простой шифр замены или же некорректно использованный шифр RSA.

Обычно алгоритм решения задач этой категории выглядит следующим образом:
\begin{enumerate}
    \item Понять, что за шифр использовался
    \item Узнать возможные атаки на этот шифр
    \item Выяснить, начальные условия для какой из возможных атак выполняются в данном случае
    \item Применить выбранную атаку
\end{enumerate}
Примеры используемых в решениях задач инструментов:
\begin{itemize}
    \item CrypTool;
    \item xortool;
    \item RsaCtfTool;
    \item Языки программирования (python, perl, js и т.д.).
\end{itemize}

\assignementTitle{Brain Damage}{1000}{}

Друг скинул мне этот текст вместе с улыбающимся смайликом. Хорошо зная своего друга, я предполагаю, что для раскрытия загадки этого сообщения нужно напрячь мозги на полную катушку. И хотя я уже потратил много времени на поиски решения, ничего больше странных символов и некоторой структуры в их расположении мне увидеть не удалось. Мне очень бы помогло, если бы вы хотя бы сказали мне, что представляет из себя этот текст...

Файл из задания доступен по ссылке: \url{https://cyberchallenge.rt.ru/files/85b80336a3cd3a1ee35739dab1601ba9/prog.txt}

\solutionSection

Исходный файл – программа на языке Brainfuck. Для запуска программы, можно воспользоваться любым подходящим онлайн сервисом. Видим, что программа выводит фразу «haha no flag here» и зацикливается.

Наиболее простой способ понять, почему же зацикливается программа, это перевести ее в код на другом, более читаемом языке, например, на Python. Для этого можно воспользоваться скриптом \url{https://www.nayuki.io/res/optimizing-brainfuck-compiler/bfc.py}.

В полученной программе на языке Python, бесконечный цикл будет начинаться в строке №68:

\begin{minted}[fontsize=\footnotesize, linenos]{python}
while mem[i] != 0:
    mem[i] = (mem[i] + 2) & 0xFF
    mem[i + 1] = (mem[i + 1] + 0) & 0xFF
    mem[i] = (mem[i] + 254) & 0xFF
\end{minted}

Легко заметить, что в данном цикле, на каждой итерации значение mem[i] меняться не будет, а значит и условие выхода из цикла никогда не выполнится. Все, что нам остается, это убрать этот бесконечный цикл и весь код до него. После запуска, такая программа выведет искомый флаг.

\answerMath{CC\{Br4inf0ck\_ta5k\_num\_57123849\}.}


\assignementTitle{Brain Damage}{1000}{}

Друг скинул мне этот текст вместе с улыбающимся смайликом. Хорошо зная своего друга, я предполагаю, что для раскрытия загадки этого сообщения нужно напрячь мозги на полную катушку. И хотя я уже потратил много времени на поиски решения, ничего больше странных символов и некоторой структуры в их расположении мне увидеть не удалось. Мне очень бы помогло, если бы вы хотя бы сказали мне, что представляет из себя этот текст...

Файл из задания доступен по ссылке: \url{https://cyberchallenge.rt.ru/files/85b80336a3cd3a1ee35739dab1601ba9/prog.txt}

\solutionSection

Исходный файл – программа на языке Brainfuck. Для запуска программы, можно воспользоваться любым подходящим онлайн сервисом. Видим, что программа выводит фразу «haha no flag here» и зацикливается.

Наиболее простой способ понять, почему же зацикливается программа, это перевести ее в код на другом, более читаемом языке, например, на Python. Для этого можно воспользоваться скриптом \url{https://www.nayuki.io/res/optimizing-brainfuck-compiler/bfc.py}.

В полученной программе на языке Python, бесконечный цикл будет начинаться в строке №68:

\begin{minted}[fontsize=\footnotesize, linenos]{python}
while mem[i] != 0:
    mem[i] = (mem[i] + 2) & 0xFF
    mem[i + 1] = (mem[i + 1] + 0) & 0xFF
    mem[i] = (mem[i] + 254) & 0xFF
\end{minted}

Легко заметить, что в данном цикле, на каждой итерации значение mem[i] меняться не будет, а значит и условие выхода из цикла никогда не выполнится. Все, что нам остается, это убрать этот бесконечный цикл и весь код до него. После запуска, такая программа выведет искомый флаг.

\answerMath{CC\{Br4inf0ck\_ta5k\_num\_57123849\}.}



\section{Категория Pwn}

Задания данной категории предполагают поиск и эксплуатацию бинарных уязвимостей, чаще всего в сетевых сервисах.

Для решения задач категории Pwn необходимо не только уметь разбираться, как работает программа, но и знать основные уязвимые места программ, а также, как их можно проэксплуатировать. Например, найти буфер ввода данных, уязвимый для переполнения, а затем внедрить свой код.

Универсального алгоритма решения, как такого, не существует. Все зависит от конкретного задания. Тем не менее, решение задания обычно начинается с обратной разработки данного в задании сервиса или приложения. Далее нужно найти потенциально проблемные места, например, копирование данных в массив на стеке. И, если копируемые данные управляются атакующим, а необходимых проверок ввода нет, то нужно приступать к написанию эксплоита (или использованию готового). Иначе, нужно продолжать поиски.

Примеры используемых в решениях задач инструментов:
\begin{itemize}
    \item Metasploit
    \item Pwntools
    \item Дизассемблер (IDA, Ghidra, radare2)
    \item Отладчик (gdb, lldb, WinDbg)
    \item Hex-редактор (Hiew, WinHex, Beye)
\end{itemize}

\assignementTitle{Brain Damage}{1000}{}

Друг скинул мне этот текст вместе с улыбающимся смайликом. Хорошо зная своего друга, я предполагаю, что для раскрытия загадки этого сообщения нужно напрячь мозги на полную катушку. И хотя я уже потратил много времени на поиски решения, ничего больше странных символов и некоторой структуры в их расположении мне увидеть не удалось. Мне очень бы помогло, если бы вы хотя бы сказали мне, что представляет из себя этот текст...

Файл из задания доступен по ссылке: \url{https://cyberchallenge.rt.ru/files/85b80336a3cd3a1ee35739dab1601ba9/prog.txt}

\solutionSection

Исходный файл – программа на языке Brainfuck. Для запуска программы, можно воспользоваться любым подходящим онлайн сервисом. Видим, что программа выводит фразу «haha no flag here» и зацикливается.

Наиболее простой способ понять, почему же зацикливается программа, это перевести ее в код на другом, более читаемом языке, например, на Python. Для этого можно воспользоваться скриптом \url{https://www.nayuki.io/res/optimizing-brainfuck-compiler/bfc.py}.

В полученной программе на языке Python, бесконечный цикл будет начинаться в строке №68:

\begin{minted}[fontsize=\footnotesize, linenos]{python}
while mem[i] != 0:
    mem[i] = (mem[i] + 2) & 0xFF
    mem[i + 1] = (mem[i + 1] + 0) & 0xFF
    mem[i] = (mem[i] + 254) & 0xFF
\end{minted}

Легко заметить, что в данном цикле, на каждой итерации значение mem[i] меняться не будет, а значит и условие выхода из цикла никогда не выполнится. Все, что нам остается, это убрать этот бесконечный цикл и весь код до него. После запуска, такая программа выведет искомый флаг.

\answerMath{CC\{Br4inf0ck\_ta5k\_num\_57123849\}.}


\assignementTitle{Brain Damage}{1000}{}

Друг скинул мне этот текст вместе с улыбающимся смайликом. Хорошо зная своего друга, я предполагаю, что для раскрытия загадки этого сообщения нужно напрячь мозги на полную катушку. И хотя я уже потратил много времени на поиски решения, ничего больше странных символов и некоторой структуры в их расположении мне увидеть не удалось. Мне очень бы помогло, если бы вы хотя бы сказали мне, что представляет из себя этот текст...

Файл из задания доступен по ссылке: \url{https://cyberchallenge.rt.ru/files/85b80336a3cd3a1ee35739dab1601ba9/prog.txt}

\solutionSection

Исходный файл – программа на языке Brainfuck. Для запуска программы, можно воспользоваться любым подходящим онлайн сервисом. Видим, что программа выводит фразу «haha no flag here» и зацикливается.

Наиболее простой способ понять, почему же зацикливается программа, это перевести ее в код на другом, более читаемом языке, например, на Python. Для этого можно воспользоваться скриптом \url{https://www.nayuki.io/res/optimizing-brainfuck-compiler/bfc.py}.

В полученной программе на языке Python, бесконечный цикл будет начинаться в строке №68:

\begin{minted}[fontsize=\footnotesize, linenos]{python}
while mem[i] != 0:
    mem[i] = (mem[i] + 2) & 0xFF
    mem[i + 1] = (mem[i + 1] + 0) & 0xFF
    mem[i] = (mem[i] + 254) & 0xFF
\end{minted}

Легко заметить, что в данном цикле, на каждой итерации значение mem[i] меняться не будет, а значит и условие выхода из цикла никогда не выполнится. Все, что нам остается, это убрать этот бесконечный цикл и весь код до него. После запуска, такая программа выведет искомый флаг.

\answerMath{CC\{Br4inf0ck\_ta5k\_num\_57123849\}.}



\section{Категория Reverse}

Задания, для решения которых необходимо уметь внимательно и аккуратно вникать в логику работы программы, чаще всего, не имея её исходного кода. Но случается, что участникам даётся обфусцированный исходный код, разобрать работу которого задача также не из самых простых. Форматы файлов могут быть самыми различными: PE, ELF, Mach-O, APK или даже просто байт-код программы для некой виртуальной машины, спецификация которой дается участникам.

В заданиях этой категории обычно флаг спрятан где-то внутри программы в зашифрованном виде. И, либо необходимо восстановить и переписать алгоритм шифрования, чтобы расшифровать флаг, либо выполнить какие-от действия (например, записать определенное значение в реестр), чтобы программа в дальнейшем сама расшифровала и вывела на экран флаг.

Предположим, что есть следующее задание: после запуска программа предлагает ввести флаг и отвечает верный он или нет. В данном случае алгоритм решения будет следующий:
\begin{enumerate}
    \item Найти место, где происходит сравнение, в результате которого выдается вердикт
    \item Тут введенные атакующим данные могут просто сравниваться с зашитым в теле программы верным флагом. Или же, верный флаг будет в некоем преобразованном виде (например, закодирован в base64)
    \item Чтобы сравнить вводимое значение с верным, программа сначала проделает все те же манипуляции, какие были проделаны изначально с верным флагом, с введенным флагом.
    \item Остается только применить преобразования в обратном порядке к зашитому в программе верному флагу в преобразованном виде.
\end{enumerate}
Примеры используемых в решениях задач инструментов:
\begin{itemize}
    \item Дизассемблер (IDA, Ghidra, radare2)
    \item Отладчик (gdb, lldb, WinDbg)
    \item Hex-редактор (Hiew, WinHex, Beye)
    \item Декомпиляторы (Hex-Rays, Ghidra, JADX, jd-gui, dotPeek)
    \item Инструменты для динамического анализа (Frida, qemu)
\end{itemize}

\assignementTitle{Brain Damage}{1000}{}

Друг скинул мне этот текст вместе с улыбающимся смайликом. Хорошо зная своего друга, я предполагаю, что для раскрытия загадки этого сообщения нужно напрячь мозги на полную катушку. И хотя я уже потратил много времени на поиски решения, ничего больше странных символов и некоторой структуры в их расположении мне увидеть не удалось. Мне очень бы помогло, если бы вы хотя бы сказали мне, что представляет из себя этот текст...

Файл из задания доступен по ссылке: \url{https://cyberchallenge.rt.ru/files/85b80336a3cd3a1ee35739dab1601ba9/prog.txt}

\solutionSection

Исходный файл – программа на языке Brainfuck. Для запуска программы, можно воспользоваться любым подходящим онлайн сервисом. Видим, что программа выводит фразу «haha no flag here» и зацикливается.

Наиболее простой способ понять, почему же зацикливается программа, это перевести ее в код на другом, более читаемом языке, например, на Python. Для этого можно воспользоваться скриптом \url{https://www.nayuki.io/res/optimizing-brainfuck-compiler/bfc.py}.

В полученной программе на языке Python, бесконечный цикл будет начинаться в строке №68:

\begin{minted}[fontsize=\footnotesize, linenos]{python}
while mem[i] != 0:
    mem[i] = (mem[i] + 2) & 0xFF
    mem[i + 1] = (mem[i + 1] + 0) & 0xFF
    mem[i] = (mem[i] + 254) & 0xFF
\end{minted}

Легко заметить, что в данном цикле, на каждой итерации значение mem[i] меняться не будет, а значит и условие выхода из цикла никогда не выполнится. Все, что нам остается, это убрать этот бесконечный цикл и весь код до него. После запуска, такая программа выведет искомый флаг.

\answerMath{CC\{Br4inf0ck\_ta5k\_num\_57123849\}.}


\assignementTitle{Brain Damage}{1000}{}

Друг скинул мне этот текст вместе с улыбающимся смайликом. Хорошо зная своего друга, я предполагаю, что для раскрытия загадки этого сообщения нужно напрячь мозги на полную катушку. И хотя я уже потратил много времени на поиски решения, ничего больше странных символов и некоторой структуры в их расположении мне увидеть не удалось. Мне очень бы помогло, если бы вы хотя бы сказали мне, что представляет из себя этот текст...

Файл из задания доступен по ссылке: \url{https://cyberchallenge.rt.ru/files/85b80336a3cd3a1ee35739dab1601ba9/prog.txt}

\solutionSection

Исходный файл – программа на языке Brainfuck. Для запуска программы, можно воспользоваться любым подходящим онлайн сервисом. Видим, что программа выводит фразу «haha no flag here» и зацикливается.

Наиболее простой способ понять, почему же зацикливается программа, это перевести ее в код на другом, более читаемом языке, например, на Python. Для этого можно воспользоваться скриптом \url{https://www.nayuki.io/res/optimizing-brainfuck-compiler/bfc.py}.

В полученной программе на языке Python, бесконечный цикл будет начинаться в строке №68:

\begin{minted}[fontsize=\footnotesize, linenos]{python}
while mem[i] != 0:
    mem[i] = (mem[i] + 2) & 0xFF
    mem[i + 1] = (mem[i + 1] + 0) & 0xFF
    mem[i] = (mem[i] + 254) & 0xFF
\end{minted}

Легко заметить, что в данном цикле, на каждой итерации значение mem[i] меняться не будет, а значит и условие выхода из цикла никогда не выполнится. Все, что нам остается, это убрать этот бесконечный цикл и весь код до него. После запуска, такая программа выведет искомый флаг.

\answerMath{CC\{Br4inf0ck\_ta5k\_num\_57123849\}.}



\section{Категория Stegano}

В заданиях данной категории, в некотором объеме данных (картинки, видео, аудиофайл и т.д.) предлагается найти информацию, спрятанную в нём таким образом, что на первый взгляд ничего особенного в данных файлах нет. Чтобы решить задание этой категории, необходимо понять и обнаружить, каким именно образом была спрятана информация.

Универсального алгоритма действий тут нет. Все зависит от того, какой тип файла был дан в задании. Далее, имея представление о возможных способах сокрытия информации в данном типе файлов, только и остается, что попробовать их все по очереди.

Примеры используемых в решениях задач инструментов:
\begin{itemize}
    \item StegSolve
    \item Binwalk
    \item file
    \item Аудиоредакторы
    \item Видеоредакторы
\end{itemize}

\assignementTitle{Brain Damage}{1000}{}

Друг скинул мне этот текст вместе с улыбающимся смайликом. Хорошо зная своего друга, я предполагаю, что для раскрытия загадки этого сообщения нужно напрячь мозги на полную катушку. И хотя я уже потратил много времени на поиски решения, ничего больше странных символов и некоторой структуры в их расположении мне увидеть не удалось. Мне очень бы помогло, если бы вы хотя бы сказали мне, что представляет из себя этот текст...

Файл из задания доступен по ссылке: \url{https://cyberchallenge.rt.ru/files/85b80336a3cd3a1ee35739dab1601ba9/prog.txt}

\solutionSection

Исходный файл – программа на языке Brainfuck. Для запуска программы, можно воспользоваться любым подходящим онлайн сервисом. Видим, что программа выводит фразу «haha no flag here» и зацикливается.

Наиболее простой способ понять, почему же зацикливается программа, это перевести ее в код на другом, более читаемом языке, например, на Python. Для этого можно воспользоваться скриптом \url{https://www.nayuki.io/res/optimizing-brainfuck-compiler/bfc.py}.

В полученной программе на языке Python, бесконечный цикл будет начинаться в строке №68:

\begin{minted}[fontsize=\footnotesize, linenos]{python}
while mem[i] != 0:
    mem[i] = (mem[i] + 2) & 0xFF
    mem[i + 1] = (mem[i + 1] + 0) & 0xFF
    mem[i] = (mem[i] + 254) & 0xFF
\end{minted}

Легко заметить, что в данном цикле, на каждой итерации значение mem[i] меняться не будет, а значит и условие выхода из цикла никогда не выполнится. Все, что нам остается, это убрать этот бесконечный цикл и весь код до него. После запуска, такая программа выведет искомый флаг.

\answerMath{CC\{Br4inf0ck\_ta5k\_num\_57123849\}.}



\section{Категория Forensics}

Компьютерная криминалистика – прикладная наука о поиске и исследовании доказательств совершения различных действий, связанных с компьютерной информацией.

Чтобы справляться с задачами компьютерной криминалистики, надо иметь представление об основах работы ОС, понимать строение файловой системы и взаимодействие различных процессов. В ходе работы необходимо анализировать образы дисков, дампы памяти, дампы сетевых пакетов, а также логи и т.п.

Как правило, в заданиях данной категории необходимо разобраться, что происходило на машине жертвы, с который был получен дамп или логи для задания. Любая обнаруженная аномалия может привести к флагу. Например, в дампе памяти обнаружилось, что в списке запущенных процессов есть графический редактор Paint, далее, нужно добраться до изображения, открытого в нем и увидеть на нем флаг. Универсального рецепта нет, с каждым типом дампа или другого файла нужно работать по разному.

Примеры используемых в решениях задач инструментов:
\begin{itemize}
    \item Виртуальная машина (VirtualBox, VMWare и т.п.);
    \item Volatility
    \item WireShark
    \item Binwalk
    \item foremost
    \item Hex-редакторы
    \item Дизассемблер
    \item Отладчик
\end{itemize}

\assignementTitle{Brain Damage}{1000}{}

Друг скинул мне этот текст вместе с улыбающимся смайликом. Хорошо зная своего друга, я предполагаю, что для раскрытия загадки этого сообщения нужно напрячь мозги на полную катушку. И хотя я уже потратил много времени на поиски решения, ничего больше странных символов и некоторой структуры в их расположении мне увидеть не удалось. Мне очень бы помогло, если бы вы хотя бы сказали мне, что представляет из себя этот текст...

Файл из задания доступен по ссылке: \url{https://cyberchallenge.rt.ru/files/85b80336a3cd3a1ee35739dab1601ba9/prog.txt}

\solutionSection

Исходный файл – программа на языке Brainfuck. Для запуска программы, можно воспользоваться любым подходящим онлайн сервисом. Видим, что программа выводит фразу «haha no flag here» и зацикливается.

Наиболее простой способ понять, почему же зацикливается программа, это перевести ее в код на другом, более читаемом языке, например, на Python. Для этого можно воспользоваться скриптом \url{https://www.nayuki.io/res/optimizing-brainfuck-compiler/bfc.py}.

В полученной программе на языке Python, бесконечный цикл будет начинаться в строке №68:

\begin{minted}[fontsize=\footnotesize, linenos]{python}
while mem[i] != 0:
    mem[i] = (mem[i] + 2) & 0xFF
    mem[i + 1] = (mem[i + 1] + 0) & 0xFF
    mem[i] = (mem[i] + 254) & 0xFF
\end{minted}

Легко заметить, что в данном цикле, на каждой итерации значение mem[i] меняться не будет, а значит и условие выхода из цикла никогда не выполнится. Все, что нам остается, это убрать этот бесконечный цикл и весь код до него. После запуска, такая программа выведет искомый флаг.

\answerMath{CC\{Br4inf0ck\_ta5k\_num\_57123849\}.}

