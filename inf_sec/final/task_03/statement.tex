\assignementTitle{Blogger}{1000}{}

После кибервызова я стану великим белым хакером! Сделаю блог (\url{http://blogger.2018.cyberchallenge.ru/}) про мои великие взломы...

(простая уязвимость не даст вам флаг)

\solutionSection

Сайт представляет из себя блог с одной единственной записью. Обратив внимание на параметры URL (\url{http://blogger.2018.cyberchallenge.ru/post?id=helloworld.html}), можно заметить, что файл, который нужно прочитать, передается в открытом виде. Попробуем прочитать любой другой файл на файловой системе.

\putImgWOCaption{14cm}{final/task_03/1}

Прочитав исходные коды сервиса, понимаем, что файл спрятан не в нем, а, скорее всего, на файловой системе сервиса. Если запросить несуществующий файл, приложение переходит в режим отладки:

\putImgWOCaption{14cm}{final/task_03/2}

Из сообщения можно понять, что сервис использует сервер werkzeug (\url{https://github.com/pallets/werkzeug}), отладочный режим которого может позволить исполнять код на сервере.

При попытке получит такой доступ, пользователь увидит сообщение с запрашиваемым пин-кодом 

\putImgWOCaption{14cm}{final/task_03/3}

Поиск в интернете позволяет найти инструкцию по получению этого пин кода \url{https://www.kingkk.com/2018/08/Flask-debug-pin%E5%AE%89%E5%85%A8%E9%97%AE%E9%A2%98/}.

С помощью инструкции и возможности чтения файлов, которую дает нам предыдущая уязвимость, генерируем пин код:

\putImgWOCaption{14cm}{final/task_03/4}

\putImgWOCaption{8cm}{final/task_03/5}

\putImgWOCaption{14cm}{final/task_03/6}

\answerMath{CC\{daaaaaamn\_everything\_is\_vulnerable\_even\_my\_static\_site\}.}