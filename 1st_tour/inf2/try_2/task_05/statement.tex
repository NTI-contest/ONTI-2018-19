\assignementTitle{Сжали}{30}

Checker Timeout Error - вердикт, который получают зависшие (то есть превысившие TL) на большинстве тестов решения. 
Внимательно рассмотрите случаи, на которых ваше решение будет выходить за TL, перед тем, как послать его на проверку.

Обратите внимание на систему оценки.

Строка $t$ является результатом сжатия строки $s$, если (должны выполняться все пункты):
\begin{itemize}
    \item Строка $t$ является подпоследовательностью строки $s$.
    \item Строка $t$ длины $k$ является подпоследовательностью строки $s$ длины $n$, если существует такой набор чисел $p$, что $i$-ый символ строки $t$ равен 
    $p_{i}$-ому символу строки $s$ и выполняется неравенство $1 \leq p_{1} < p_{2} < ... < p_{k-1} < p_{k} \leq n$.
    \item Первый символ строки $t$ равен первому символу строки $s$ (то есть $p_{1} = 1$).
    \item Последний символ строки $t$ равен последнему символу строки $s$ (то есть $p_{k} = n$).
    \item Если $i$-ый $(1 \leq i < k)$ символ строки $t$ является $m$-ой буквой латинского алфавита, то в удовлетворяющем строке $t$ наборе чисел $p$ между $p_i$-ым и 
    $p_{i+1}$-ым символами в строке $s$ не больше $bonus_m$ символов (то есть $p_{i+1}-p_{i} - 1 \leq bonus_m$).   
\end{itemize}

Стоимость строки определяется как сумма стоимостей всех ее символов. Стоимость $i$-ой буквы латинского алфавита равна $cost_{i}$.

Определите стоимость самой дешёвой из сжатых строк $s$.

\inputfmtSection
В первой строке входных данных одно целое число: 

$3 \le n \le 10^{6}$ - длина строки $s$.

Во второй строке - сама строка $s$, состоящая из строчных латинских букв.

В третьей строке - 26 целых чисел:

$0 \le cost_{i} \le 10^{9}$ - стоимость добавления в сжатую строку $i$-го $(1 \le i \le 26)$ символа латинского алфавита.

В четвертой строке - 26 целых чисел:

$0 \le bonus_{i} \le 10^{9}$ - сколько символов можно пропустить в строке $s$ после добавления в сжатую строку $i$-го 
$(1 \le i \le 26)$ символа латинского алфавита.

\outputfmtSection

Выведите одно неотрицательное число – стоимость самой дешёвой из полученных описанным сжатием строк.

\markSection

В задаче 100 тестов. Баллы за задачу будут начисляться пропорционально количеству успешно пройденных тестов.

Первый тест совпадает с тестом из условия.

В тестах $1-15$ все $n$ не превышают 20.

В тестах $16-31$ все $cost_{i} = i - 1$ и $bonus_{i} = i - 1$.

В тестах $1 - 43$ все $n$ не превышают 1000.

В тестах $44 - 58$ все $bonus_{i} = i - 1$.

В тестах $1 - 68$ все $n$ не превышают $10^{5}$ .

Пояснение к примеру

В примере самой дешёвой строкой является строка $abeha$.

\sampleTitle{1}

\begin{myverbbox}[\small]{\vinput}
    8
    abcdebha
    0 2 2 3 0 0 1 1 2 1 0 3 2 0 2 2 3 1 0 1 2 2 0 1 2 3
    1 2 0 2 1 1 2 0 1 1 1 1 0 1 2 0 0 1 2 2 3 0 1 3 1 2
\end{myverbbox}
\begin{myverbbox}[\small]{\voutput}
    3
\end{myverbbox}
\inputoutputTable

\includeSolutionIfExistsByPath{1st_tour/inf2/try_2/task_05}