\assignementTitle{Сжали}{30}

Строка $t$ является результатом сжатия строки $s$, если (должны выполняться все пункты):

\begin{itemize}
\item Строка $t$ является подпоследовательностью строки $s$ .

Строка $t$ длины $k$ является подпоследовательностью строки $s$ длины $n$, если существует такой набор чисел $p$, что $i$-ый символ строки $t$ равен $p_i$-ому символу строки $s$ и выполняется неравенство $1 \le p_{1} < p_{2} < \ldots < p_{k-1} < p_{k} \le n$ .


\item Первый символ строки $t$ равен первому символу строки $s$ (то есть $p_{1} = 1$).
\item Последний символ строки $t$ равен последнему символу строки $s$ (то есть $p_{k} = n$).
\item Если $i$-ый $(1 \le i < k)$ символ строки $t$ является $m$-ой буквой латинского алфавита, то в удовлетворяющем строке $t$ наборе чисел $p$ между $p_i$-ым и $p_{i+1}$-ым символами в строке $s$ не больше $bonus_m$ символов (то есть $p_{i+1}-p_{i} - 1 \le bonus_m$) .
\end{itemize}


Стоимость строки определяется как сумма стоимостей всех ее символов. Стоимость $i$-ой буквы латинского алфавита равна $cost_{i}$ .

Определите стоимость самой дешёвой из сжатых строк $s$ .

\inputfmtSection

В первой строке входных данных одно целое число:

$3 \le n \le 10^{6}$ -- длина строки $s$ .

Во второй строке -- сама строка $s$ , состоящая из строчных латинских букв.

В третьей строке -- $26$ целых чисел:

$0 \le cost_{i} \le 10^{9}$ -- стоимость добавления в сжатую строку $i$-го $(1 \le i \le 26)$ символа латинского алфавита.

В четвертой строке -- $26$ целых чисел:

$0 \le bonus_{i} \le 10^{9}$ -- сколько символов можно пропустить в строке $s$ после добавления в сжатую строку $i$-го $(1 \le i \le 26)$ символа латинского алфавита.

\outputfmtSection

Выведите одно неотрицательное число -- стоимость самой дешёвой из полученных описанным сжатием строк.

\exampleSection

\sampleTitle{1}

\begin{myverbbox}[\small]{\vinput}
8
abcdebha
0 2 2 3 0 0 1 1 2 1 0 3 2 0 2 2 3 1 0 1 2 2 0 1 2 3
1 2 0 2 1 1 2 0 1 1 1 1 0 1 2 0 0 1 2 2 3 0 1 3 1 2
\end{myverbbox}
\begin{myverbbox}[\small]{\voutput}
3
\end{myverbbox}
\inputoutputTable

\includeSolutionIfExistsByPath{1st_tour_progr/task_025}