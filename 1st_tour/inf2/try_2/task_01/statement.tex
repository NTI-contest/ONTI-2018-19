\assignementTitle{Призы}{15}

Организаторы олимпиады по робототехнике решили, что у них будет ровно $n$ участников финального этапа. 
Они хотят подарить каждому участнику батончики KitKat и уже нашли, что можно купить по низкой цене пачки, 
в которых будет ровно по $m$ батончиков.

Организаторы решили, что все участники должны получить поровну батончиков, иначе кто-то обидится. 
Также они решили, что не должно остаться лишних батончиков. Поэтому им нужно определить минимальное 
количество пачек, которые они должны купить.

\inputfmtSection

Вводится два целых числа $n$ и $m$ $(1 \leq n,m \leq 10^9)$ — количество участников и количество батончиков в пачке.

\outputfmtSection

Выведите единственное число — ответ на задачу.

\markSection

Баллы за задачу будут начисляться пропорционально количеству успешно пройденных тестов. 

Примерно в $33\%$ тестов $1 \leq n,m \leq 103$ 

Примерно в $60\%$ тестов $1 \leq n,m \leq 106$



\sampleTitle{1}

\begin{myverbbox}[\small]{\vinput}
    8 4
\end{myverbbox}
\begin{myverbbox}[\small]{\voutput}
    2
\end{myverbbox}
\inputoutputTable

\includeSolutionIfExistsByPath{1st_tour/inf2/try_2/task_01}