\assignementTitle{Цепь}{25}{}

Герои книги "Автостопом по галактике" в своем путешествии случайно попали в параллельную вселенную. В этой вселенной все по-другому. 
Можно перемещаться не только по пространству, но и во времени. Для этого они используют свой звездолет.

Законы физики и всех других наук здесь работают не так.

В этой вселенной есть $n$ планет, для каждой планеты известно ее расстояние от центра вселенной $r_i$ и коэффициент искажения пространства 
$k_i$.

Чтобы сделать эту вселенную более интересной для туристов из других вселенных, наши герои решили соединить все планеты с помощью пар звездных 
врат. Между звездными вратами можно перемещаться в обе стороны. При этом для каждой планеты герои выбрали год $y_i$, когда эта планета наиболее 
интересна туристам. Герои хотят, чтобы после этого построения звездных врат можно было посетить все планеты.

При этом для построения звездных врат нужно будет потратить часть энергии звездолета.

Чтобы построить портал между $i$-ой и $j$-ой планетами, нужно потратить

$3 \cdot (r_{i} - r_{j})^{2} + 2 \cdot |(2 \cdot k_{i} - 2 \cdot k_{j}) \cdot (2 \cdot k_{i} + 2 \cdot k_{j})| + 5 \cdot |y_{i} - y_{j}|$ единиц энергии.

Герои хотят минимизировать суммарное количество энергии, которое нужно потратить на построение звездных врат, чтобы как можно больше топлива осталось им для остальных путешествий. Помогите им это сделать.

\inputfmtSection

В первой строке вводится целое число $n$ ($1 \le n \le 10^4$) - количество планет.

В следующих $n$ строках вводятся по три целых числа $r_i, k_i, y_i$ $(0 \le r_i, k_i,y_i\le 10^6)$

\outputfmtSection

Выведите единственное число - минимальное суммарное количество энергии.

\markSection

Баллы за задачу будут начисляться пропорционально количеству успешно пройденных тестов.

Примерно в $50\%$  тестов $1 \le n \le 500$ 


\sampleTitle{1}

\begin{myverbbox}[\small]{\vinput}
    2
    1 1 1
    2 2 2
\end{myverbbox}
\begin{myverbbox}[\small]{\voutput}
    32
\end{myverbbox}
\inputoutputTable

\includeSolutionIfExistsByPath{1st_tour/inf2/try_2/task_04}