\assignementTitle{Шпионаж }{20}

Наше агентство осуществило перехват нескольких предположительно шпионских сообщений. Однако возникли проблемы при декодировании.

Нам удалось узнать, что:

\begin{itemize}
    \item каждый символ изначального сообщения закодировали последовательностью из нулей и единиц;
    \item длина каждой из этих последовательностей равна k ;
    \item каждому символу поставлена в соответствие ровно одна последовательность из k нулей и единиц;
    \item каждой последовательности из k нулей и единиц поставлен в соответствие ровно один символ;
    \item экземпляры таблицы декодирования испорчены и не подлежат восстановлению.
\end{itemize}

Большего вам знать не нужно.

Для первичного отделения шпионских сообщений от сообщений, попавших в рассмотрение случайно, нам нужна программа, подсчитывающая количество различных символов, используемых в сообщении, представленном в виде строки.

Берётесь за эту работу?

\inputfmtSection

В первой строке входных данных два целых числа:

$1 \le n \le 10^{5}$ - длина строки;

$1 \le k \le n$ - длина последовательностей, которыми были закодированы символы.

Во второй строке дано сообщение в виде строки $s$. 

Гарантируется, что число $n$ кратно $k$ и закодированная 
строка $s$ состоит из $n$ символов, каждый из которых равен 0 или 1.

\outputfmtSection

Выведите одно положительное число – количество различных символов в строке.

\markSection

Баллы начисляются только если ваша программа прошла все тесты.

\sampleTitle{1}

\begin{myverbbox}[\small]{\vinput}
    9 3
    001000100
\end{myverbbox}
\begin{myverbbox}[\small]{\voutput}
    3
\end{myverbbox}
\inputoutputTable

\includeSolutionIfExistsByPath{1st_tour/inf2/try_2/task_03}