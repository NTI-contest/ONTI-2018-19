\assignementTitle{k-экстремум}{15}

Дан массив $a$ из $n$ элементов.

Элемент массива $a_{i}$ является $k$-экстремумом, если он является $k$-максимумом или $k$-минимумом.

Элемент массива $a_{i}$ $(1 <  i <  n)$ является $k$-максимумом тогда и только тогда, 
когда $a_{i-1}+k \leq a_{i}$ и $a_{i+1} + k \leq a_i$.

Элемент массива $a_{i}$ $(1< i< n)$ является $k$-минимумом тогда и только тогда, когда 
$a_{i-1}-k \geq a_{i}$ и $a_{i+1}-k \geq a_{i}$.

Если $a_{i}$ — один из крайних ($i=1$ или $i=n$ элементов массива, то он не может быть $k$-экстремумом.

За одну единицу времени вы можете увеличить или уменьшить на единицу любой элемент массива.

Какое минимальное количество времени нужно потратить, чтобы в массиве появился хотя бы один $k$-экстремум?

\inputfmtSection

В первой строке вводятся два целых числа $n$ и $k$ ($3 \leq n \leq 10^5, 1 \leq k \leq 10^9$)

Во второй строке вводятся $n$ чисел $a_i$ ($1 \leq a_i \leq 10^9$)

\outputfmtSection

Выведите единственное число — ответ на задачу.

\markSection

Баллы за задачу будут начисляться пропорционально количеству успешно пройденных тестов. 

Примерно в $30\%$  тестов $1 \leq n, k \leq 100$ 

Примерно в $60\%$ тестов $1 \leq n \leq 10^3$

Пояснение к примеру

В примере достаточно уменьшить второй элемент на единицу.



\sampleTitle{1}

\begin{myverbbox}[\small]{\vinput}
    3 2
    4 3 4
\end{myverbbox}

\begin{myverbbox}[\small]{\voutput}
    1
\end{myverbbox}
\inputoutputTable

\includeSolutionIfExistsByPath{1st_tour/inf2/try_2/task_02}