\solutionSection

Заметим, что тратить время стоит только на увеличение или только на уменьшение одного конкретного элемента массива $a_{i}$ $(1 < i < n)$.

Давайте для каждого элемента посчитаем, сколько нужно потратить времени, чтобы сделать его $k$-экстремумом.

Чтобы сделать $i$-ый $(1 < i < n)$ элемент $k$-максимумом, его значение должно стать не менее $\max(a_{i - 1} + k, a_{i + 1} + k)$, а на это нужно потратить $\max(a_{i}, a_{i - 1} + k, a_{i + 1} + k) - a_{i}$ единиц времени.

Чтобы сделать $i$-ый $(1 < i < n)$ элемент $k$-минимумом, его значение должно стать не более $\min(a_{i - 1} - k, a_{i + 1} - k)$, а на это нужно потратить $a_{i} - \min(a_{i}, a_{i - 1} - k, a_{i + 1} - k)$ единиц времени.

Тогда время, которое необходимо потратить, чтобы сделать $i$-ый элемент $k$-экстремумом, будет равно минимуму из этих двух значений. А ответом на всю задачу -- минимальное значение по всем элементам.

\codeExample

%\inputPythonSource
%\inputJavaSource
\inputCPPSource