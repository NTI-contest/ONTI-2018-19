\assignementTitle{Робот}{25}

Ильнар решил научить робота перемещаться по плоскости.
Но у него оставалось мало времени и он решил немного схитрить.
Он научил робота перемещаться правильно только на дистанцию не более чем $\sqrt k$.
Чтобы робот смог при этом добраться от начальной точки до конечной,
он задал в конфигурации еще дополнительные точки на плоскости.
Теперь робот может перемещаться только между этими точками.

Чтобы добраться от точки $i$ до точки $j$, робот тратит $(x_i - x_j)^2 + (y_i - y_j)^2$ единиц времени.

\inputfmtSection

В первой строке заданы числа $n$ и $k$ $(1 \le n \le 5000$, $1 \le k \le 10^{9})$ .

В следующих $n$ строках заданы координаты точек $x_i$, $y_i$ $(1 \le x_{i}, y_{i} \le 20000)$ .

В последней строке заданы два числа $s$ и $t$  $(1 \le s, t \le n$, $s \ne t )$ -- начальная и конечная точки пути робота.

\outputfmtSection

Выведите единственное число -- минимальное время за которое робот доберется из $s$ в $t$ или $-1$, если робот не может добраться.

\exampleSection

\sampleTitle{1}

\begin{myverbbox}[\small]{\vinput}
2 32
1 1
5 5
1 2
\end{myverbbox}
\begin{myverbbox}[\small]{\voutput}
32
\end{myverbbox}
\inputoutputTable

\sampleTitle{2}

\begin{myverbbox}[\small]{\vinput}
2 31
1 1
5 5
1 2
\end{myverbbox}
\begin{myverbbox}[\small]{\voutput}
-1
\end{myverbbox}
\inputoutputTable

\includeSolutionIfExistsByPath{1st_tour_progr/task_034}