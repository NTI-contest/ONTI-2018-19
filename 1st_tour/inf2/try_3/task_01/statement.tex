\assignementTitle{Печеньки}{10}

У Ильнара есть три коробки с печеньем, в которых $a$, $b$, $c$ печенек соответственно. К Ильнару 
в гости пришло $(n - 1)$ человек. Ильнар хочет, чтобы всем гостям и ему досталось одинаковое 
количество печенек. Поэтому он хочет узнать сколько печенек достанется каждому, 
если он откроет несколько (возможно ни одной, возможно все три) коробок с печеньем, 
при этом все печеньки из открытых коробок должны быть розданы поровну 
(сами печенья нельзя ломать на части).  

Определите максимальное количество печенек, которое получит каждый гость и Ильнар.

\inputfmtSection

В первой строке вводится 4 целых числа $a$, $b$, $c$, $n$ $(1 \leq a, b, c, n \leq 10^8)$ 

\outputfmtSection

Выведите единственное число — ответ на задачу.

\markSection

Баллы начисляются только если ваша программа прошла все тесты.

\sampleTitle{1}

\begin{myverbbox}[\small]{\vinput}
    1 2 3 4
\end{myverbbox}
\begin{myverbbox}[\small]{\voutput}
    1
\end{myverbbox}
\inputoutputTable

\sampleTitle{2}

\begin{myverbbox}[\small]{\vinput}
    3 4 5 19
\end{myverbbox}
\begin{myverbbox}[\small]{\voutput}
    0
\end{myverbbox}
\inputoutputTable

\includeSolutionIfExistsByPath{1st_tour/inf2/try_3/task_01}