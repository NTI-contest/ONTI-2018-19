\assignementTitle{ЗАГС}{15}

В стране $M$ есть город $N$. А в этом городе есть ЗАГС. И сегодня там случилась поломка ПО.

Были потеряны данные о заявлениях на регистрацию брака. Однако остался журнал прихода и вызова.

В журнале в хронологическом порядке записаны события двух видов:

\begin{itemize}
    \item Приход человека, желающего подать заявление. В журнале записана фамилия этого человека.
    \item Объявление о вызове для утверждения заявления и проверки документов. В этом случае в журнале записано слово $next$.
\end{itemize}

В стране $M$ принято, что супруги должны иметь одинаковую фамилию и начиная с прихода в ЗАГС 
для подачи заявления на регистрацию брака должны называть фамилию, которую будут носить после 
вступления в брак.

В городе $N$ заявление на регистрацию брака принимается только в присутствии обоих будущих 
супругов.

После прихода в ЗАГС и внесения в журнал прихода и вызова соответствующей записи нужно 
встать в очередь. В случае, если один из будущих супругов пришёл раньше, то другой или другая 
присоединяются к будущей супруге или будущему супругу.

Когда вызывают следующую пару для утверждения заявления, может оказаться так, что первым 
в очереди стоит человек, будущая супруга или будущий супруг которого ещё не пришёл в ЗАГС. 
В этом случае на утверждение заявления идут те уже пришедшие в ЗАГС будущие супруги, которые 
находятся ближе всего к началу очереди. В случае, если в очереди нет ни одной пары, на вызов 
пойдёт первая появившаяся в очереди пара сразу после прихода второго супруга или супруги.

Ваша задача - восстановить по данной информации события, произошедшие в ЗАГСе города $N$ за сегодня.

\inputfmtSection

В первой строке дано число $n$ ($n \leq 105$) - количество записей в журнале.

В каждой из следующих $n$ строк находится либо слово $next$, обозначающее вызов следующей пары, 
либо фамилия, состоящая из латинских букв и длиной не менее одного и не более двадцати символов.

Каждая из фамилий встречается в записях журнала не менее одного и не более двух раз.

\outputfmtSection

В случае прихода человека, супруг или супруга которого ещё не пришёл или не пришла в ЗАГС, выведите перед фамилией этого человека $1st$. Таким образом мы опишем событие становления в конец очереди нового пришедшего.

В случае прихода человека, супруг или супруга которого уже в ЗАГСе, выведите перед фамилией этого человека $2nd$ без кавычек. Таким образом мы опишем событие появления в очереди пары на месте, которое занял пришедший ранее супруг или супруга.

В случае объявления о вызове, выведите фамилию будущих супругов, чьё заявление будет утверждаться следующим. В случае, если на момент вызова в очереди нет ни одной пары будущих супругов, выводить фамилию будущих супругов, чьё заявление будет утверждаться следующим, следует только после появления этой пары в очереди. Если же до конца дня в ЗАГС не придёт ни одной пары, ничего выводить не нужно.

Не выводите лишние пробелы в конце или начале строк - это будет считаться за ошибку.

Для лучшего понимания формата выходных данных ознакомьтесь с примерами ниже.

\markSection

Баллы начисляются только если ваша программа прошла все тесты.

\sampleTitle{1}

\begin{myverbbox}[\small]{\vinput}
    5
    Pit
    Wait
    Pit
    Wait
    next
\end{myverbbox}
\begin{myverbbox}[\small]{\voutput}
    1st Pit
    1st Wait
    2nd Pit
    2nd Wait
    Pit
\end{myverbbox}
\inputoutputTable

\sampleTitle{2}

\begin{myverbbox}[\small]{\vinput}
    5
    Pit
    Wait
    Wait
    Pit
    next
\end{myverbbox}
\begin{myverbbox}[\small]{\voutput}
    1st Pit
    1st Wait
    2nd Wait
    2nd Pit
    Pit
\end{myverbbox}
\inputoutputTable

\sampleTitle{3}

\begin{myverbbox}[\small]{\vinput}
    5
    next
    Pit
    Wait
    Wait
    Pit
\end{myverbbox}
\begin{myverbbox}[\small]{\voutput}
    1st Pit
    1st Wait
    2nd Wait
    Wait
    2nd Pit
\end{myverbbox}
\inputoutputTable

\includeSolutionIfExistsByPath{1st_tour/inf2/try_3/task_05}