\assignementTitle{Анна и красивая строка}{20}

Анна написала генератор красивых строк. Она считает строку красивой,
если она одинаково читается как слева направо, так и справа налево.
Например, $rrr$ и $anna$ -- красивые строки, а $abc$ и $adba$ -- нет.

Но она допустила ошибку в коде и генератор выводит не красивые строки, а строки,
которые можно сделать красивыми, если из каждой удалить ровно один символ.
По крайней мере, она так думает.

Она просит вас помочь определить, верно ли её предположение.

\inputfmtSection

В первой строке вводится строка $s$ состоящая из маленьких латинских букв
($4 \le |s| \le 10^{5}$, где $|s|$ -- длина строки).

\outputfmtSection

Выведите \textit{YES} , если можно удалить один символ из строки так,
чтобы она стала красивой; иначе -- \textit{NO} .

\exampleSection

\sampleTitle{1}

\begin{myverbbox}[\small]{\vinput}
abca
\end{myverbbox}
\begin{myverbbox}[\small]{\voutput}
YES
\end{myverbbox}
\inputoutputTable

\sampleTitle{2}

\begin{myverbbox}[\small]{\vinput}
abcd
\end{myverbbox}
\begin{myverbbox}[\small]{\voutput}
NO
\end{myverbbox}
\inputoutputTable

\includeSolutionIfExistsByPath{1st_tour_progr/task_033}