\solutionSection

В данной задаче нас просят определить, может ли строка $s$ длины $n$ стать палиндромом, если из неё удалить ровно один символ. Если может, то ответ \textit{YES} , иначе -- \textit{NO} .

Заметим, что если строка $s$ является палиндромом, то можно удалить из неё $\left( \left \lfloor \frac{n}{2} \right \rfloor + 1  \right)$-ый символ и строка $s$ останется палиндромом, то есть в качестве ответа нужно вывести \textit{YES} .

Теперь рассмотрим случай, когда строка $s$ не является палиндромом, то есть существует такое $k$ $(k \ge 1)$, что $s_1 = s_n$, $s_2 = s_{n - 1}$, $...$, $s_{k - 1} = s_{n - k + 2}$, $s_k \neq s_{n - k + 1}$ $(k < n - k + 1)$, где $s_i$ -- $i$-ый символ строки.

Докажем, что если $s_1 = s_n$, то ответ для строки $s$, из которой предварительно удалили первый и последний символы, будет такой же, как и для самой строки $s$.

Докажем от противного. Если утверждение не верно, то строка $s$ может стать палиндромом после удаления $i$-го символа $(1 \le i \le n)$, только если $i = 1$ или $i = n$. Но если после удаления первого символа получился палиндром, то получается, что $s_2 = s_n = s_1$, то есть не важно, удалим мы первый или второй символ, а если палиндром получился после удаления последнего символа, то $s_{n - 1} = s_1 = s_n$, то есть не важно, удалим мы последний или предпоследний символ. Пришли к противоречию.

Следовательно, если после удаления одного символа строка $s$ может стать палиндромом, то существует такое $i$, что соблюдается двойное неравенство $k \le i \le n - k + 1$ и строка $s$ станет палиндромом, если удалить $i$-ый символ. Также заметим, что после удаления такого $i$-го символа, что соблюдается двойное неравенство $k < i < n - k + 1$, строка $s$ точно не станет палиндромом, так как $s_k \neq s_{n - k + 1}$, что противоречит определению палиндрома.

Таким образом, для определения ответа достаточно посмотреть, станет ли строка $s$ палиндромом после удаления $k$-го символа и станет ли строка $s$ палиндромом после удаления $(n - k + 1)$-го символа. Если хотя бы в одном из этих двух случаев строка $s$ станет палиндромом, то ответ \textit{YES} , иначе -- \textit{NO} .

\codeExample

\inputPythonSource
%\inputJavaSource
%\inputCPPSource