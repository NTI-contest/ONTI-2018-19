\solutionSection

Поработаем с данным в условии задачи свойством. Какие числа находятся в $i$-ом ряду?

Если $i = 1$, то -- числа от $1$ до $s_{1}$.

Если $i > 1$, то -- числа от $s_{1} + s_{2} + \ldots + s_{i - 1} + 1$ до $s_{1} + s_{2} + \ldots + s_{i}$, так как числа от $1$ до $s_{1} + s_{2} + \ldots + s_{i - 1}$ записаны в первых $(i - 1)$ рядах.

Таким образом, для нахождения ответа для отдельно взятого набора рядов можно было, например, отсортировать числа в каждом ряду и проверить, что $j$-ое (после сортировки) число в $i$-ом ряду равно $s_{1} + s_{2} + \ldots + s_{i - 1} + j$ для всех возможных пар $(i, j)$.

\codeExample

\inputPythonSource
%\inputJavaSource
%\inputCPPSource