\assignementTitle{Классики}{15}

Шёл $2048$-ой год. Робот Вася шёл по улице и увидел играющих детей. Они прыгали то на одной ноге, то на двух. Вася подошёл и спросил, во что играют дети. Ему объяснили, что эта игра называется классики и дети прыгают по написанным на асфальте числам. Вася пропрыгал разок и пошёл дальше по своим делам.

Через несколько дней идя по улице он увидел написанные на асфальте ряды чисел и задался вопросом, классики ли это. Ещё через пару дней он снова увидел написанные на асфальте ряды чисел и вновь задался вопросом, классики ли это. Но ответить на этот вопрос он не мог.

Классиками в понимании Васи (ну, так ему объяснили) являются ряды, которые обладают следующим свойством:

Каждое целое число от $1$ до $s_{1} + s_{2} + \ldots + s_{i}$ встречается в первых $i$ рядах \emph{ровно один раз}, где $s_{i}$ -- количество чисел в $i$-ом ряду.

Напишите для Васи программу, которая будет помогать ему определять, являются ли те или иные ряды чисел классиками (в понимании Васи).

\inputfmtSection

В первой строке одно целое число:

$1 \le t \le 500$ -- количество раз, которое Вася встречал написанные на асфальте ряды чисел в последнее время.

Каждый набор рядов описан следующим образом:

В первой строке одно целое число:

$1 \le n \le 500$ -- количество рядов.

Во второй строке $n$ целых чисел:

$s_{i}$ -- количество чисел в $i$-ом ряду

$(1 \le i \le n; 1 \le s_{i} \le 500)$.

Далее идёт ещё $n$ строк. В $(i+2)$-ой строке $s_{i}$ целых чисел:

$a_{i,j}$ -- $j$-ое число в $i$-ом ряду

$(|a_{i,j}| \le 10^9)$.

Гарантируется, что суммарное количество чисел во входных данных не превышает $10^5$.

\outputfmtSection

Для каждого набора рядов в отдельной строке выведите: 

$Yes$, если данные ряды чисел являются классиками в понимании Васи, иначе -- $No$.

\exampleSection

\sampleTitle{1}

\begin{myverbbox}[\small]{\vinput}
3
5
1 2 1 2 1
1
2 3
4
5 6
7
5
1 2 1 2 1
1
3 2
4
5 6
7
5
1 2 1 2 1
1
3 2
4
6 6
7
\end{myverbbox}
\begin{myverbbox}[\small]{\voutput}
Yes
Yes
No
\end{myverbbox}
\inputoutputTable

\includeSolutionIfExistsByPath{1st_tour/inf2/try_1/task_02}