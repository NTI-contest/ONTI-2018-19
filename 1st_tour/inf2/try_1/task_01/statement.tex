\assignementTitle{Башня}{15}

Девочка Аня построила на пляже песчаный замок и хочет украсить главную башню замка лентой следующим образом:

Она хочет сделать $k$ витков вокруг башни, а оставшуюся часть ленты разделить на ленточки равной длины и прикрепить получившиеся ленточки к вершине башни. 

Аня посчитает башню красивой, если она сможет сделать желаемые kkk витков, затем разделить
оставшуюся часть ленты на ленточки равной длины, а потом прикрепить к вершине хотя бы одну 
ленточку с длиной, равной высоте башни.

Родители помогли измерить Ане высоту башни, длину одного витка и длину имеющейся у неё ленты. Но Аня ещё не научилась считать, поэтому просит вас помочь определить, сможет ли она сделать свою башню красивой, используя имеющуюся у неё ленту.

\inputfmtSection

В единственной строке входных данных четыре целых числа:

$1 \le n \le 1000$ — длина ленты;

$1 \le m \le 1000$ — длина одного витка;

$1 \le k \le 1000$ — количество витков;

$1 \le h \le 1000$ — высота башни.

\outputfmtSection

Выведите Yes, если лента подходит, иначе — No.

\markSection

Баллы начисляются только если ваша программа прошла все тесты.

\explaneSection

На витки тратится $6$ единиц длины, и остаётся ровно на одну ленточку с длиной, равной высоте башни.

\sampleTitle{1}

\begin{myverbbox}[\small]{\vinput}
    13 2 3 7
\end{myverbbox}
\begin{myverbbox}[\small]{\voutput}
    Yes
\end{myverbbox}
\inputoutputTable

\includeSolutionIfExistsByPath{1st_tour/inf2/try_1/task_01}