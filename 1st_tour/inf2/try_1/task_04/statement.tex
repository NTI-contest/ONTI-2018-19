\assignementTitle{R2D2}{25}

Робот R2D2 случайно оказался на Имперском корабле. Он хочет покинуть его как можно скорее. Для этого ему надо добраться до спасательной капсулы.

Для упрощения задачи корабль представляет собой прямоугольную таблицу высотой $n$ и шириной $m$. Ячейка может быть либо пустой, 
либо представлять собой препятствие. Помогите за минимальное время добраться R2D2 из своей начальной точки до спасательной капсулы.

При этом известно, что робот может передвигаться только в клетки, соседние по стороне. То есть двигаться только вверх, вниз, влево и вправо. 
Также у робота есть текущее направление.

Движение вперед занимает у робота 1 секунду и поворот на $90^{\circ}$ также занимает 1 секунду.

Зная начальное расположение робота и его направление. Выясните за какое минимальное время он сможет покинуть корабль. При этом, 
если робот оказался в ячейке со спасательной капсулой, его текущее направление не имеет значения.

Изначально робот всегда смотрит вниз.

\inputfmtSection

В первой строке вводятся два целых числа $n$ и $m$ $(1 \leq n,m \leq 1000)$ - высота и ширина.

В следующих $n$ строках вводятся $m$ символов $a_{i,j}$. Значения ячейки  $a_{i,j}$ могут быть \# - препятствие, . - пустая клетка, 
$s$ - начальная позиция робота, $f$ - спасательная капсула.

Гарантируется, что ровно одна клетка в таблице имеет значение $s$.

Гарантируется, что ровно одна клетка в таблице имеет значение $f$.

\outputfmtSection

Выведите минимальное количество секунд, нужное чтобы добраться роботу до спасательной капсулы или $-1$, если это сделать невозможно

\markSection

В задаче 25 тестов. Баллы за задачу будут начисляться пропорционально количеству успешно пройденных тестов.

Первые два теста совпадают с тестами из условия.

В тестах $1-7$  Следующие ограничения: $1 \leq n,m \leq 10$.

В тестах $1-14$  Следующие ограничения: $1 \leq n,m \leq 100$.

В тестах $1-17$  Следующие ограничения: $1 \leq n,m \leq 500$.

\sampleTitle{1}

\begin{myverbbox}[\small]{\vinput}
    3 3
    s..
    ...
    ..f
\end{myverbbox}
\begin{myverbbox}[\small]{\voutput}
    5
\end{myverbbox}
\inputoutputTable

\sampleTitle{2}

\begin{myverbbox}[\small]{\vinput}
    3 3
    s..
    ###
    ..f
\end{myverbbox}
\begin{myverbbox}[\small]{\voutput}
    -1
\end{myverbbox}
\inputoutputTable

\includeSolutionIfExistsByPath{1st_tour/inf2/try_1/task_04}