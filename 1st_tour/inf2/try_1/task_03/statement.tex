\assignementTitle{Треугольник}{20}

Плоскость задана тремя точками. Определите площадь треугольника, образованного этими тремя точками.

\inputfmtSection

Входные данные состоят из трёх строк. В каждой из трёх строк по три целых числа:

$x_i$ $y_i$ $z_i$ -- координаты $i$-ой точки

$(1 \le i \le 3; |x_{i}| \le 10^{9}; |y_{i}| \le 10^{9}; |z_{i}| \le 10^{9})$.

Гарантируется, что данные три точки не лежат на одной прямой.

\outputfmtSection

Одно положительное число -- площадь треугольника.

Ваш ответ будет засчитан, если его абсолютная или относительная ошибка не превосходит $10^{-6}$. Формально, пусть ваш ответ равен $a$, а ответ жюри равен $b$. Ваш ответ будет засчитан, если $\frac {|a - b|}{\max(1, b)} \le 10^{-6}$.

\exampleSection

\sampleTitle{1}

\begin{myverbbox}[\small]{\vinput}
0 4 0
0 0 0
3 0 0
\end{myverbbox}
\begin{myverbbox}[\small]{\voutput}
6.0000000000
\end{myverbbox}
\inputoutputTable

\includeSolutionIfExistsByPath{1st_tour_progr/task_013}