\assignementTitle{Робот и камушки}{25}

Ильнар в одной из комнат увидел странного робота. Во время выполнения алгоритма, он доставал из мешка разные камушки. 
Причем он никогда не доставал один и тот же камень два раза. И говорил сколько камушков в мешке такого же цвета.

Разработчик робота рассказал Ильнару, что из-за ошибки в коде робот ровно один раз всегда ошибается.

Теперь Ильнару интересно, а сколько минимально может быть камушков в мешке.

Примечание

Робот говорит количество камушков того же цвета, что в руке. При этом камень, который у него в руках, он тоже учитывает. После данной операции камень возвращается обратно в мешок.

\inputfmtSection

В первой строке содержится единственное целое число $n$ ($1 \leq n \leq 10^5$) — количество выбранных камушков.

Во второй строке находятся $n$ целых чисел $a_i$ ($1 \leq a_i \leq 10^9$) — значения, названные роботом.

\outputfmtSection

Выведите одно положительное целое число — минимальное возможное количество камушков в мешке.

\markSection

Баллы начисляются только если ваша программа прошла все тесты.

\sampleTitle{1}

\begin{myverbbox}[\small]{\vinput}
    4
    2 2 2 2
\end{myverbbox}
\begin{myverbbox}[\small]{\voutput}
    5
\end{myverbbox}
\inputoutputTable

\includeSolutionIfExistsByPath{1st_tour/inf2/try_1/task_05}