\assignementTitle{Посади деревце}{10}

Инженер-программист решил соорудить ограждение для деревца. Для этого он включил генератор случайных чисел и получил координаты точек, в которые он вобьет колышки и протянет между ними веревку. Сможет ли наш герой поместить в это сооружение деревце, чтобы оно лежало внутри ограждения? Для посадки деревца достаточно любой ненулевой площади.

\inputfmtSection
На вход программе в одной строке подаются координаты трёх точек $a_x$, $a_y$, $b_x$, $b_y$, $c_x$, $c_y$. 
Значения абсцисс и ординат целочисленные и не превышают по модулю $10^9$. 

\outputfmtSection
Проверьте, может ли инженер посадить в описанном ограждении дерево. Выведите "Yes" при положительном ответе или "No" в противном случае.

\markSection

Баллы за задачу будут начисляться пропорционально количеству успешно пройденных тестов.

\sampleTitle{1}

\begin{myverbbox}[\small]{\vinput}
    10 0 0 10 10 10
\end{myverbbox}

\begin{myverbbox}[\small]{\voutput}
    Yes
\end{myverbbox}
\inputoutputTable

\includeSolutionIfExistsByPath{1st_tour/inf/try_1/task_07}