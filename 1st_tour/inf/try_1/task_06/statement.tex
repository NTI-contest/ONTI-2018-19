\assignementTitle{Система передатчиков}{10}{}

Инженер разрабатывает оптимальный маршрут для передачи данных через систему передатчиков. 
Все передатчики располагаются на прямоугольной сетке. Каждый из передатчиков может 
передавать сообщение только ближайшим соседям справа и снизу, если они есть. 
Определите наибольшую пропускную способность от левого верхнего передатчика до правого нижнего, 
если Вам известны пропускные способности каждого из передатчиков.

\inputfmtSection
На вход программе подаются размеры прямоугольника: длина $n$ и ширина $m$ \linebreak 
$(1\leq n\leq 1000, 1\leq m \leq 1000)$. 

В следующих $n$ строках записываются $m$ целых чисел -- максимальная пропускная способность 
каждого из передатчиков $k (0\leq k\leq 1000000)$.

\outputfmtSection

Выведите целое число -- максимально возможную пропускную способность имеющейся системы.

\markSection

Баллы за задачу будут начисляться пропорционально количеству успешно пройденных тестов.

\sampleTitle{1}

\begin{myverbbox}[\small]{\vinput}
    5 6 
    8 7 6 5 0 0 
    4 3 1 8 1 8 
    4 4 4 5 0 4 
    7 8 9 3 6 0 
    1 4 2 2 4 6
\end{myverbbox}

\begin{myverbbox}[\small]{\voutput}
    3
\end{myverbbox}
\inputoutputTable

\solutionSection

Данная задача решается алгоритмом динамического программирования. Для каждого передатчика определяется наиболее оптимальный сосед, благодаря которому может быть достигнута наибольшая пропускная способность. 
\includeSolutionIfExistsByPath{1st_tour/inf/try_1/task_06}