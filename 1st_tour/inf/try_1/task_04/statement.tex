\assignementTitle{Градусы}{10}{}

Для различных картографических сервисов требуются различные форматы координат. Чаще всего 
координаты обозначаются в десятичных градусах (DD -- decimal degrees), например, так:
(55.7644871, 37.6602897)  

А для некоторых сервисов требуется перевод координат в формат $(A_1^{\circ}B_1'C_1''D_1,$ $A_2^{\circ}B_2'C_2''D_2)$,  
где $A$~—~градусы, $B$~—~минуты, $C$~—~секунды, и направление $D$ ($N$~—~северная широта, $S$~—~южная широта,
 $W$  — западная долгота, $E$  — восточная долгота).
 
Напишите программу-калькулятор, осуществляющую вышеописанный перевод координат. 

\inputfmtSection

На вход программе подается строка, в которой записаны два вещественных числа с 7-ю знаками после точки: широта 
$X(-90.0000000 \leq X \leq 90.0000000)$ и долгота $Y(-180.0000000 \leq Y \leq 180.0000000)$ некоторого объекта на карте.

\outputfmtSection

Выведите в отдельной строке координаты в требуемом формате. Обратите внимание, что значения $A , B , C$  в формате указываются с ведущими нулями до двух знаков. Округление осуществляйте по модулю в меньшую сторону.

\markSection

Баллы за задачу будут начисляться пропорционально количеству успешно пройденных тестов.

\sampleTitle{1}

\begin{myverbbox}[\small]{\vinput}
    55.7644871 37.6602897
\end{myverbbox}
\begin{myverbbox}[\small]{\voutput}
   55°45'52''N 37°39'37''E
\end{myverbbox}
\inputoutputTable

\sampleTitle{2}

\begin{myverbbox}[\small]{\vinput}
    0.0000000 0.0000000
\end{myverbbox}
\begin{myverbbox}[\small]{\voutput}
   00°00'00''N 00°00'00''E
\end{myverbbox}
\inputoutputTable

\solutionSection
Особенность данной задачи в целочисленной арифметике. Если оперировать с входными данными как с вещественными числами, то некоторые числа, не представимые конечным числом знаков в двоичной системе счисления, переводились неверно. Например, правильный результат для теста $(23.2000000 -34.8000000) \rightarrow (23^\circ12'00''N 34^\circ48'00''W)$, а не $23^\circ11'59''N 34^\circ47'59''W$.

\includeSolutionIfExistsByPath{1st_tour/inf/try_1/task_04}