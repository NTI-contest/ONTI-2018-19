\assignementTitle{Диета}{10}{}

Родители научили Иру, что много есть сладкого вредно. Поэтому она решила есть не больше 8 конфет в неделю. Бабушка Ире привезла мешок с конфетами, в котором 100 конфет Красная Шапочка и 100 конфет Мишка на Севере. Ира решила выбрать 8 конфет из мешка и разложить их по дням на неделю. Ещё Ира не хочет в любой из дней оставаться без конфет. Сколькими способами она может это сделать? Порядок употребления конфет в каждый из дней не важен.

\solutionSection

Очевидно, что девочка может выбирать любое количество конфет каждого типа, так как Красных шапочек не менее 8 и Мишек на Севере тоже. 

8 конфет на 7 дней означает, что 6 дней у нее обычные (по одной конфете), а один -- особенный (2 конфеты). Выбрать особенный день в неделе можно $7$ способами. Количество способов выбрать конфеты в обыденные дни равно $2^6$, а количество способов выбрать конфету в особенный день равно $3$ (две одного вида, или две другого вида, или две разные). Так как описанные процессы выбора независимы, то количество различных комбинаций равно $7 \cdot 2^6 \cdot 3 = 1344$.

\answerMath{1344.}