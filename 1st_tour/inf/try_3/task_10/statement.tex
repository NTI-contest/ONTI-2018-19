\assignementTitle{До красных стен}{10}{}

Робот находится в центре некоторого лабиринта. Лабиринт состоит из помещений и дверей:

\putImgWOCaption{8cm}{1}

На схеме изображены помещения, ограниченные радиальными и шестиугольными стенами. По краям лабиринта 
стены покрашены в красный цвет. В шестиугольных стенах смонтированы двери между помещениями. Радиальные 
стены соединены с красной стеной. Дверей в красной стене нет.

Каждое помещение имеет свой уникальный номер. Номер центрального помещения, из которого стартует робот, всегда равен 0. Ни один номер не повторяется дважды. Между помещениями установлены двери. Каждая дверь связывает некоторую пару помещений. Между двумя помещениями может быть не более одной двери. Нет ни одного помещения, в которое нельзя прийти из центра и из которого нельзя выйти к красной стене, двигаясь по направлению из центра. В радиальных стенах двери отсутствуют.

Задача робота добраться до красной стены. Робот в данном лабиринте ведет себя следующим образом: каждый раз он случайно (равновероятно) выбирает дверь, ведущую в сторону от центра. В каких помещениях робот будет заканчивать работу чаще? Определите вероятности попадания робота в помещения с красной стеной.

\inputfmtSection
Для каждого помещения с красной стеной в отдельной строке выведите вероятность попадания робота в данное помещение в следующем формате:

\begin{itemize}
    \item номера помещений запишите в порядке возрастания;
    \item после каждого номера поставьте двоеточие;
    \item затем через пробел укажите вероятность попадания в указанное помещение. Если вероятность является целым числом, 
    укажите это число. Если вероятность — дробное число, то запишите его в виде простой дроби $x/y$, \linebreak $\text{где НОД}(x,y)=1$.
\end{itemize}

\markSection

Баллы за задачу будут начислены, если все тесты будут пройдены успешно.

\sampleTitle{1}

\begin{myverbbox}[\small]{\vinput}
    23
    0 1
    0 2
    0 3
    0 9
    1 16
    1 17
    16 15
    15 14
    15 13
    17 21
    17 18
    18 19
    18 20
    2 4
    2 5
    4 6
    4 7
    5 8
    3 22
    9 23
    9 10
    10 11
    10 12
\end{myverbbox}

\begin{myverbbox}[\small]{\voutput}
    6: 1/16
    7: 1/16
    8: 1/8
    11: 1/16
    12: 1/16
    13: 1/16
    14: 1/16
    19: 1/32
    20: 1/32
    21: 1/16
    22: 1/4
    23: 1/8    
\end{myverbbox}
\inputoutputTable

\solutionSection

Из условия очевидно, что структура переходов между помещениями древовидная, вершина дерева помещение с номером 0, листья -- помещения у красной стены. Вероятность перехода из родительской вершины в конкретную из дочерних равна $1/n$, где $n$ -- количество детей у родительской вершины. Собственно, вероятность попадания в лист равно произведению вероятности попадания в его родителя, умноженное на количество детей у его родителя.

Дроби (сами вероятности) умножать при этом абсолютно неправильно, можно перемножать знаменатели и поделить в самом конце. Ограничения выставлены таким образом, что длинная арифметика не требуется. 

\includeSolutionIfExistsByPath{1st_tour/inf/try_3/task_10}