\assignementTitle{Reverse engineering}{10}{}

Витя написал программу, которая вычисляет функцию вида $$f(x)=(ax^2+bx+c)\: mod \: d,$$ где $mod$ — остаток от деления. 
Для $25$ различных значений аргументов из диапазона от $0$ до $24$ включительно он узнал, чему равна данная функция. Из этих 
данных он составил тесты к этой задаче, а саму функцию забыл. Автор задачи точно уверен в том, что значения $a$, $b$, $c$ и $d$ — 
неотрицательные целые числа, не превосходящие $10$.

Помогите Вите восстановить эту функцию. Напишите программу, которая проходит все тесты.

В данной задаче нет ограничение на количество посылок. Ваша задача "смайнить" тесты и узнать, что же за функция зашита в программе. 
Как это сделать? Учитесь правильно ошибаться! 

Для того, чтобы вы могли смотреть вердикты по всем тестам, вы обязательно должны проходить верно первый тест.

\inputfmtSection

Целое число $N\space(0\leq N\leq24)$  — аргумент функции.

\outputfmtSection

Единственное целое число — значение функции при заданном аргументе.

\markSection

Баллы за задачу будут начисляться пропорционально количеству успешно пройденных тестов.

\sampleTitle{1}

\begin{myverbbox}[\small]{\vinput}
    6
\end{myverbbox}

\begin{myverbbox}[\small]{\voutput}
    4
\end{myverbbox}
\inputoutputTable

\solutionSection

Можно, конечно, попробовать решить задачу полным перебором, однако, это крайне энергозатратно, медленно и неправильно. Поэтому перебор должен быть умным. Особенность задачи в том, что участник может посмотреть на вердикты по всем тестам и может управлять вердиктами. 

Например, можно печатать определенное значение функции и узнать в каких тестах оно является ответом. Так можно оценить максимальное значение функции и предположить, чему равно $d$.  

Также можно определять значение аргумента и при определенном условии генерировать ошибку, например, Time Limit Exceeded (к примеру, while True) или Runtime Error (10/0). Так можно определить четность/нечетность аргументов, кратность любому значению или даже само значение.

Если выявить несколько пар значений аргументов и значений функций в них, можно решить систему уравнений и определить используемые коэффициенты.

\includeSolutionIfExistsByPath{1st_tour/inf/try_3/task_09}