\assignementTitle{Анализ текста}{10}

Олеся занимается статистическим анализом текстов. Помогите ей посчитать, сколько каких слов 
встречается в тексте. Для этого:

\begin{enumerate}
    \item В скачанном тексте замените все знаки, не являющиеся латинскими буквами, на пробелы.
    \item Приведите текст к нижнему регистру.
    \item Для каждого слова посчитайте, сколько его вхождений есть в тексте. Выведите список пар слово — количество его повторений в порядке убывания количества вхождений. Если несколько слов входят в текст равное количество раз, укажите их в лексикографическом порядке.
\end{enumerate}

\explanationSection

Эталонный ответ и ваш будут сравниваться как последовательность слов без разделителей. Таким образом, ответы "answer 1" и "answer 01" будут считаться различными, а "answer 1" и "answer        1" одинаковыми. Для избежания ошибок, рекомендуем загружать ответ в виде файла в однобайтовой кодировке.

\sampleTitle{1}

\begin{myverbbox}[\small]{\vinput}
    Five little girls, of Five, Four, Three, Two, One:
    Rolling on the hearthrug, full of tricks and fun.
    
    Five rosy girls, in years from Ten to Six:
    Sitting down to lessons - no more time for tricks.
    
    Five growing girls, from Fifteen to Eleven:
    Music, Drawing, Languages, and food enough for seven!
    
    Five winsome girls, from Twenty to Sixteen:
    Each young man that calls, I say "Now tell me which you mean!"
    
    Five dashing girls, the youngest Twenty-one:
    But, if nobody proposes, what is there to be done?
    
    Five showy girls - but Thirty is an age
    When girls may be engaging, but they somehow don't engage.
    
    Five dressy girls, of Thirty-one or more:
    So gracious to the shy young men they snubbed so much before!
    
    Five passe girls - Their age? Well, never mind!
    We jog along together, like the rest of human kind:
    But the quondam "careless bachelor" begins to think he knows
    The answer to that ancient problem "how the money goes"!
\end{myverbbox}

\begin{myverbbox}[\small]{\voutput}
    five 9
    girls 9
    to 8
    the 7
    but 4
    of 4
    from 3
    one 3
    age 2
    and 2
    be 2
    for 2
    is 2
    more 2
    so 2
    that 2
    they 2
    thirty 2
    tricks 2
    twenty 2
    young 2
    along 1
    an 1
    ancient 1
    answer 1
    bachelor 1
    before 1
    begins 1
    calls 1
    careless 1
    dashing 1
    don 1
    done 1
    down 1
    drawing 1
    dressy 1
    each 1
    eleven 1
    engage 1
    engaging 1
    enough 1
    fifteen 1
    food 1
    four 1
    full 1
    fun 1
    goes 1
    gracious 1
    growing 1
    he 1
    hearthrug 1
    how 1
    human 1
    i 1
    if 1
    in 1
    jog 1
    kind 1
    knows 1
    languages 1
    lessons 1
    like 1
    little 1
    man 1
    may 1
    me 1
    mean 1
    men 1
    mind 1
    money 1
    much 1
    music 1
    never 1
    no 1
    nobody 1
    now 1
    on 1
    or 1
    passe 1
    problem 1
    proposes 1
    quondam 1
    rest 1
    rolling 1
    rosy 1
    say 1
    seven 1
    showy 1
    shy 1
    sitting 1
    six 1
    sixteen 1
    snubbed 1
    somehow 1
    t 1
    tell 1
    ten 1
    their 1
    there 1
    think 1
    three 1
    time 1
    together 1
    two 1
    we 1
    well 1
    what 1
    when 1
    which 1
    winsome 1
    years 1
    you 1
    youngest 1
\end{myverbbox}
\inputoutputTable

\includeSolutionIfExistsByPath{1st_tour/inf/try_3/task_08}