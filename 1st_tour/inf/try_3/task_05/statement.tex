\assignementTitle{Это сдвиг?}{10}

Мальчик Кирилл написал однажды на листе бумаги строчку, состоящую из больших и маленьких 
латинских букв, а после этого ушел играть в футбол. Когда он вернулся, то обнаружил, 
что его братик Дима написал под его строкой еще одну строчку такой же длины. 
Дима утверждает, что свою строчку он получил циклическим сдвигом строки Кирилла 
направо на несколько шагов (циклический сдвиг строки abcde на 2 позиции направо 
даст строку deabc). Однако Дима еще маленький и мог случайно ошибиться в большом 
количестве вычислений, поэтому Кирилл в растерянности — верить ли Диме?

\inputfmtSection

Первые две строки содержат $s_1$ и $s_2$ $(1\leq len(s_1) = len(s_2) \leq 10^4)$ — строки Кирилла и \
Димы соответственно. Строки состоят только из латинских букв.

\outputfmtSection

По данным строкам выведите единственное число — минимально возможный размер сдвига или $ -1 $, если Дима ошибся.

\markSection

Баллы за задачу будут начисляться пропорционально количеству успешно пройденных тестов.

\sampleTitle{1}

\begin{myverbbox}[\small]{\vinput}
    ieeeeieiieeieeeeiiiieeeii
    ieeeeiiiieeeiiieeeeieiiee
\end{myverbbox}

\begin{myverbbox}[\small]{\voutput}
    14
\end{myverbbox}
\inputoutputTable

\solutionSection

Задача решается с использованием z-функции от строки имеющей следующую структуру:
$$s + \# + t + t[:-1],$$ 
где $s$ -- первая строка, $t$ -- вторая строка, $t[-1]$ -- вторая строка без последнего знака, $\#$ -- не используемый в словах $s$ и $t$ знак. 

\includeSolutionIfExistsByPath{1st_tour/inf/try_3/task_05}