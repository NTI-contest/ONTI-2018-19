\assignementTitle{Пропускная способность}{10}

Антенна беспроводной связи типа $UR/2$ имеет радиус действия, равный $R$. Если телефон находится 
на расстоянии $s \in [0, R/2]$  скорость передачи данных для одного клиента равна $u$. Если же  
$s \in (R/2, R]$, скорость падает до значения $v$. За пределами этого радиуса $UR/2$ телефон этой 
антенной обслуживаться не будет.

В некотором государстве массово, почти на каждом шагу установлены антенны беспроводной связи типа 
$UR/2$. Вам нужно синхронизировать ваши данные на телефоне с облачным сервисом. 
С какой максимальной скоростью в заданной локации вы сможете это сделать, 
если технически устройством поддерживается параллельная передача данных через все доступные антенны?

\inputfmtSection
В первой строке через пробел подаются целые значения $R, u, v\space(-10^8\leq R\leq 10^8,\space 1\leq v \leq u \leq 100)$ — 
характеристики антенн.

В следующей строке целое число $n\space (1\leq n\leq 10^6)$  — количество антенн. 

Далее в $n$ строках пары целых чисел $x_i, y_i \space (-10^8\leq x_i, y_i \leq 10^8)$  — координаты антенн.

В следующей строке через пробел подаются $X, Y \space (-10^8\leq X, Y \leq 10^8)$ — ваши координаты.

\outputfmtSection
В единственной строке выведите целое число -- скорость синхронизации в заданной локации через заданную сеть приёмников.

\markSection

Баллы за задачу будут начисляться пропорционально количеству успешно пройденных тестов.

\sampleTitle{1}

\begin{myverbbox}[\small]{\vinput}
    10 5 3
    4
    0 0
    0 5
    10 0
    10 10
    0 0
\end{myverbbox}

\begin{myverbbox}[\small]{\voutput}
    13
\end{myverbbox}
\inputoutputTable

\includeSolutionIfExistsByPath{1st_tour/inf/try_3/task_06}