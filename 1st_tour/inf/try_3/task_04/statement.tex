\assignementTitle{Игра Жизнь}{10}

Правила этой игры просты:

\begin{itemize}
    \item Игра Жизнь проходит на клеточном поле, которое, традиционно, называется Вселенная.
    \item Каждая клетка может быть живой или мёртвой (живая — *, мёртвая — x).
    \item Поколения сменяются синхронно по следующим правилам:
    \item \begin{itemize}
        \item в пустой (мёртвой) клетке, рядом с которой ровно три живые клетки, зарождается жизнь;
        \item если у живой клетки есть две или три живые соседки, то эта клетка продолжает жить;
        \item в противном случае клетка умирает от одиночества или от перенаселённости.
    \end{itemize}
\end{itemize}

Для заданного начального состояния определите:

\begin{itemize}
    \item Вечная ли Жизнь во Вселенной?
    \begin{itemize}
        \item Если да, то укажите номер первой итерации, которая повторила любую из предыдущих.
        \item Если нет, то сколько итераций осталось до смерти Вселенной
        \item Под итерацией понимается процесс смены поколения во Вселенной.
    \end{itemize}
\end{itemize}

\inputfmtSection
В первой строке задается целое число $n \space(1\leq n \leq 10)$  — величина стороны Вселенной.

Далее в $n$ строках подаются значения клеточного поля.

\outputfmtSection
В первой строке Yes или No ответ на первый вопрос.

Во второй строке неотрицательное целое число — комментарий к ответу.

\markSection

Баллы за задачу будут начислены, если все тесты будут пройдены успешно.

\sampleTitle{1}

\begin{myverbbox}[\small]{\vinput}
    5
    xxxxx
    xx*xx
    ***xx
    xx*xx
    xxxxx
\end{myverbbox}
\begin{myverbbox}[\small]{\voutput}
    Yes
    10
\end{myverbbox}
\inputoutputTable

\sampleTitle{2}

\begin{myverbbox}[\small]{\vinput}
    5
    *xxxx
    x*xxx
    x**xx
    xxxxx
    xxxxx
\end{myverbbox}
\begin{myverbbox}[\small]{\voutput}
    No
    5
\end{myverbbox}
\inputoutputTable

\includeSolutionIfExistsByPath{1st_tour/inf/try_3/task_04}