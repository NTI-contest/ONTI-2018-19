\assignementTitle{Управление складом}{10}{}

Анна хочет исследовать динамику использования складского помещения. У нее есть сведения о суточном изменении количества свободных ячеек, и она хочет составлять запросы и узнавать, каким образом изменяется количество свободных ячеек на временных интервалах между любыми двумя указанными датами. Помогите Анне решить эту задачу.

\inputfmtSection

В первой строке целое число $n\space (1\leq n \leq 10^6)$ количество дней, за которые есть наблюдения. Во второй строке через пробел $n$ целых чисел $c_i\space (-100\leq c_i \leq 100)$ —- значения суточных изменений.В третьей строке целое число $m\space (1\leq m \leq 10^6)$  —- количество запросов Анны.Далее в $m$  строках по два числа через пробел $b_i$  и $e_i$  $(0\leq b_i \leq e_i \leq n-1)$  — первый и последний день отрезка, на котором необходимо произвести расчет.

\outputfmtSection

Для каждого из $m$ запросов выведите два числа: количество занятых за указанное время ячеек и количество высвобожденных.

\markSection

Баллы за задачу будут начисляться пропорционально количеству успешно пройденных тестов.

\explanationSection

Для считывания в Python вместо input() используйте sys.stdin.readline() во избежании TL.

\sampleTitle{1}

\begin{myverbbox}[\small]{\vinput}
    1
    -100
    1
    0 0
\end{myverbbox}

\begin{myverbbox}[\small]{\voutput}
    0 100
\end{myverbbox}
\inputoutputTable

\sampleTitle{2}

\begin{myverbbox}[\small]{\vinput}
    1
    100
    1
    0 0
\end{myverbbox}

\begin{myverbbox}[\small]{\voutput}
    100 0
\end{myverbbox}
\inputoutputTable

\sampleTitle{3}

\begin{myverbbox}[\small]{\vinput}
    4
    -10 5 -5 10
    5
    0 0
    1 1
    0 1
    0 3
    2 3
\end{myverbbox}

\begin{myverbbox}[\small]{\voutput}
    0 10
    5 0
    5 10
    15 15
    10 5
\end{myverbbox}
\inputoutputTable

\sampleTitle{4}

\begin{myverbbox}[\small]{\vinput}
    5
    10 20 -30 -10 100
    1
    0 4
\end{myverbbox}

\begin{myverbbox}[\small]{\voutput}
    130 40
\end{myverbbox}
\inputoutputTable

\solutionSection

Заранее подсчитаем количество занятых ячеек в период от первого дня до $i$-го для всех дней. И отдельно аналогичным образом посчитаем количество свободных. Тогда каждый поступаемый запрос можно обработать за $O(1)$.

\includeSolutionIfExistsByPath{1st_tour/inf/try_3/task_07}