\assignementTitle{Движение по шару}{10}

Люди перемещаются по поверхности нашей планеты. Они ходят, бегают, ездят и даже летают. И им всегда важно расстояние до объекта, пусть даже они почти никогда не двигаются по прямой. По координатам двух объектов в десятичных градусах определите расстояние между ними на поверхности Земли, если считаем её сферой с радиусом $R = 6371302$ м.

\inputfmtSection

В единственной строке через пробел подаются вещественные значения $ lat_1, long_1, lat_2, long_2 \space(-90 \leq lat_i \leq 90, -180 \leq long_i \leq 180) $ —- координаты двух объектов.

\outputfmtSection

Единственное целое число -— ответ на задачу в километрах. Округление дробной части производите к ближайшему целому.

\markSection

Баллы за задачу будут начисляться пропорционально количеству успешно пройденных тестов.

\sampleTitle{1}

\begin{myverbbox}[\small]{\vinput}
    55 37.620393 43.116418 131.882475
\end{myverbbox}

\begin{myverbbox}[\small]{\voutput}
    6458
\end{myverbbox}
\inputoutputTable

\includeSolutionIfExistsByPath{1st_tour/inf/try_2/task_08}