\assignementTitle{Размыкания}{10}

Буквально каждый из нас пользуется мобильным телефоном. Некоторые из нас не могут себе представить популярное устройство связи в будущем. А знаете ли вы, что лет эдак 40 назад никто не мог предположить о том, что телефоны станут не просто кнопочными, но и мобильными, да ещё и с сенсорным экраном. Раньше многие телефоны выглядели следующим образом:

\putImgWOCaption{8cm}{1}

Для такого телефонного аппарата (с дисковым номеронабирателем) набор номера абонента осуществляется следующим образом: при вращении диска по часовой стрелке до пальцевого упора контакты номеронабирателя замыкают линию, а при возвратном вращении линия размыкается такое число раз, которое соответствует набранной цифре. На рисунке показана временная диаграмма работы телефонного аппарата.

\putImgWOCaption{12cm}{2}

При наборе номера  8-800-000-00-00 в сумме будет произведено 106 размыканий (2 по 8 и 9 по 10). Посчитайте, сколько существует возможных различных номеров (11-значных комбинаций цифр, в том числе и с ведущими нулями), при которых возникает заданное количество размыканий $n$. 

\inputfmtSection

В отдельной строке целое число $n\space(0 \leq n \leq 1000)$ 

\outputfmtSection

Единственное число — ответ на задачу.

\solutionSection

В условиях ограниченного времени (5 минут) правильнее всего написать программу, вычисляющую ответ. 

\includeSolutionIfExistsByPath{1st_tour/inf/try_2/task_04}