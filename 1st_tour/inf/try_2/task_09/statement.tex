\assignementTitle{Длина пути}{10}

Робот умеет двигаться по траектории, задаваемой некоторой полиномиальной функцией $f(x)$ с 
целочисленными показателями. Свой путь он начинает в координате $(0, f(0))$  и двигается в 
сторону возрастания значений по оси абсцисс. Но, как вы знаете, на любое движение нужна 
энергия, наш робот не исключение. Он снабжен аккумулятором, которые позволяет ему 
проехать путь длиной $s$. Где окажется робот в конце пути?  

\inputfmtSection
В первой строке задается $n \space (0 \leq n \leq 5)$  — степень полинома, задающего 
траекторию робота.

Во второй строке через пробел задаются целочисленные значения $a_i \space (-10 \leq a_i \leq 10)$  — 
коэффициенты перед степенями многочлена, в порядке от старшей степени к младшей.

В третьей строке вещественное число $s \space (0 \leq s \leq 100)$  — длина пути робота.

\outputfmtSection
В отдельной строке абсцисса координаты, где остановится робот, с абсолютной точностью не менее $10^{-3}$.

\markSection

Баллы за задачу будут начисляться пропорционально количеству успешно пройденных тестов.

\sampleTitle{1}

\begin{myverbbox}[\small]{\vinput}
    0
    5
    100
\end{myverbbox}

\begin{myverbbox}[\small]{\voutput}
    100.00000002054185
\end{myverbbox}
\inputoutputTable

\sampleTitle{2}

\begin{myverbbox}[\small]{\vinput}
    1
    1 1
    100
\end{myverbbox}

\begin{myverbbox}[\small]{\voutput}
    70.71068001124502
\end{myverbbox}
\inputoutputTable

\sampleTitle{3}

\begin{myverbbox}[\small]{\vinput}
    2
    2 3 5
    100
\end{myverbbox}

\begin{myverbbox}[\small]{\voutput}
    6.350909999928737
\end{myverbbox}
\inputoutputTable

\includeSolutionIfExistsByPath{1st_tour/inf/try_2/task_09}