\assignementTitle{Инвентаризация}{10}{}

В некоторой очень крупной международной корпорации решили произвести учет продукции на складах всех филиалов, распределительных центров и магазинов. Каждый склад прислал список товаров, в котором каждая строка соответствует одному товару и содержит его название и количество. Посчитайте, сколько товаров каждого вида хранится на складах, и выводите эти значения по запросам.

\inputfmtSection

На вход подаются несколько списков товаров со складов $(N < 10^3)$. После каждого списка следует пустая строка.

В каждом списке некоторое количество записей $(M < 10^3)$. Одна запись записывается в отдельной строке.

Строка начинается с название товара $ s_i (1 \leq len(s_i) \leq 70) $, которое может состоять из заглавных/строчных латинских букв, цифр и пробелов между ними. Названия товаров в одном списке не повторяются. Заканчивается строка целым числом, отделенным пробелом, $ c_i (0 \leq с_i \leq 10^6) $ —- количеством элементов. 

Далее в отдельной строке записывается слово QUERIES .

После него в отдельных строках подаются $ K $ запросов $  q_i\space (1 \leq len(q_i) \leq 70), K\space (K \leq 10^5 )$, которые могут состоять из заглавных/строчных латинских букв, цифр и пробелов между ними. 

\outputfmtSection

Для каждого запроса выведите в отдельной строке единственное число —- количество данного товара на всех складах. Если товар отсутствует, выведите $ 0 $.

\explanationSection

Названия товаров регистрозависимы. Все названия начинаются и заканчиваются видимыми знаками (не разделителями). Нигде в исходных данных не встречаются два пробела подряд.

\markSection

Примерно в $ 30\% $ тестов названия товаров и запросы состоят из 1 слова. 

Примерно в $ 30\% $ тестов $  N \cdot M \geq 10 ^ 5 $.

Примерно в $ 40\% $ тестов $ K \geq 10^4 $

Баллы за задачу будут начисляться пропорционально количеству успешно пройденных тестов.

\sampleTitle{1}

\begin{myverbbox}[\small]{\vinput}
    blue ink 15 3 1

    QUERIES
    blue ink 15 3
    black paper
\end{myverbbox}

\begin{myverbbox}[\small]{\voutput}
    1
    0
\end{myverbbox}
\inputoutputTable

\sampleTitle{2}

\begin{myverbbox}[\small]{\vinput}
    robot 1 15
    detail2 10
    detail 1 5
    
    professional device v4 4
    detail2 12
    
    QUERIES
    robot  1
    detail2
    detail 2
    professional device
    professional device v4
    MakeIt kit
    robot 1 15
\end{myverbbox}

\begin{myverbbox}[\small]{\voutput}
    0
    22
    0
    0
    4
    0
    0
\end{myverbbox}
\inputoutputTable

\sampleTitle{3}

\begin{myverbbox}[\small]{\vinput}
    QUERIES
\end{myverbbox}

\begin{myverbbox}[\small]{\voutput}
    
\end{myverbbox}
\inputoutputTable

\solutionSection

Основная сложность задачи -- считывать поток данных до конца файла (не зная длины заранее) и разбивать строки с товарами на ключ-значение, правильно сохраняя общее количество товаров.

\includeSolutionIfExistsByPath{1st_tour/inf/try_2/task_09}