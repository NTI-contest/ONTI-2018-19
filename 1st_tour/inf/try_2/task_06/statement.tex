\assignementTitle{"Репликация"}{10}{}

Василий решил, что создание своей собственной системы управления базами данных (СУБД) — посильная для него задача, и даже немного подумал про отказоустойчивость. Он продумал собственную систему "репликации" (резервного копирования) данных, чтобы быть уверенным в их сохранности. Он активно использовал её для хранения некоторых своих измерений.Однажды алгоритм "репликации" дал сбой и Василий определил, что одна из записей в "реплику" была занесена дважды. Василий выгрузил в текстовый формат колонку с числовыми значениями идентификаторов записей, которые были представлены 32-битными целыми числами. Одна беда — они выгрузились в хаотичном порядке. Теперь у Василия два текстовых файла с числами: из базы и из реплики. Помогите как можно быстрее найти идентификатор дважды повторенной записи.

\inputfmtSection

В первой строке $ n\space (1\leq n \leq 10^6) $ — число записей в базе Василия.

Во второй строке $ n $ различных целых неотрицательных $ 32 $-битных значений $ a_i $, разделенных пробелом — идентификаторы записей в базе Василия.

В третьей строке $ n+1 $ целых неотрицательных $ 32 $-битных значений $ b_i $, разделенных пробелом — идентификаторы записей в "реплике" Василия. 

\outputfmtSection

Единственное число — ответ на задачу

\markSection

Примерно в $ 50\% $ тестов $ 10^5\leq n \leq 10^6 $.

Баллы за задачу будут начисляться пропорционально количеству успешно пройденных тестов.

\sampleTitle{1}

\begin{myverbbox}[\small]{\vinput}
    8
    22 23 11 8 21 2 6 18
    18 21 2 11 23 6 23 22 8
\end{myverbbox}

\begin{myverbbox}[\small]{\voutput}
    23
\end{myverbbox}
\inputoutputTable

\solutionSection
Так как значения в базе целочисленные и хранятся в двоичном виде, то можно воспользоваться логической 
операцией XOR. Её особеность в том, что $$\forall x: \: x \: XOR \: x = 0, x \: XOR \: y = y \: XOR \: x.$$ Таким образом, например, $$x \: XOR \: y \: XOR \: x \: XOR \: z \: XOR \: y = z.$$ Следовательно, если мы совершим эту операцию для всех считанных целых чисел (кроме количества в первой строке), то получим единственное значение, которое не повторялось.

\includeSolutionIfExistsByPath{1st_tour/inf/try_2/task_06}