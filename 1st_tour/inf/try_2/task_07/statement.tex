\assignementTitle{Параллелограмм}{10}{}

Игорь решил заняться необычной графикой. Сначала он по трём точкам на плоскости рисует треугольники. Затем ищет красивый способ дорисовать треугольник до параллелограмма. Красивым способом он считает такой, при котором:

\begin{enumerate}
\item У параллелограмма одна из диагоналей получается наидлиннейшей из возможных вариантов. 
\item Если таких вариантов построения несколько, Игорь из них выбирает тот, у которого сумма координат новой точки (абсциссы + ординаты) наибольшая. 
\item Ну а если и таких точек несколько, то из них он выбирает ту, у которой которой наибольшая абсцисса.
\end{enumerate}

По трем заданным точкам треугольника укажите координаты четвертой точки получаемого красивого паралелограмма.

\inputfmtSection

В трех строчках через пробел по паре целых чисел $ x_i, y_i\space (-10^7\leq x_i, y_i \leq 10^7) $ —- координаты вершин треугольника. Гарантируется, что они не лежат на одной прямой.

\outputfmtSection

Абсцисса и ордината точки — ответ на задачу

\markSection

Примерно в $ 50\% $ тестов $ -10^3 \leq x_i, y_i \leq 10^3 $.

Баллы за задачу будут начисляться пропорционально количеству успешно пройденных тестов.

\sampleTitle{1}

\begin{myverbbox}[\small]{\vinput}
    10 0 0 10 10 10
\end{myverbbox}

\begin{myverbbox}[\small]{\voutput}
    20 0
\end{myverbbox}
\inputoutputTable

\solutionSection

Сначала надо определить максмальную длину наибольшей возможной диагонали. Далее определяем список вершин тех паралелограммов, у которых хотя бы одна из диагоналей является максимальной. 

Треугольников с максимальной диагональю может быть не более 2, так как построить равносторонний треугольник с целыми координатами нельзя. Поэтому если есть два треугольника с максимальными диагоналями выбираем из них по дополнительным критериям. 

\includeSolutionIfExistsByPath{1st_tour/inf/try_2/task_07}