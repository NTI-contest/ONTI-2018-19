\assignementTitle{Статистика по салатам}{10}{}

В столовой учащиеся информационно-технологического класса решили провести анализ того, чем их кормят. Для исследований они выбрали овощной салат из капусты, горошка, кукурузы и фасоли, который повара столовой готовят достаточно часто. Ребята в каждой порции считали количество горошин, зернышек кукурузы и фасолин. Через некоторое время ребята ощутили маленький размер выборки и вовлекли в этот процесс большое количество учащихся школ, которых обслуживает тот же комбинат питания, что и их школу. Так за учебный год у ребят набралась неплохая выборка. Для заданной выборки определите, сколько в среднем ингредиентов каждого типа (горох, кукуруза и фасоль) встречаются в салате и каково медианное значение данных ингредиентов в салате. После этого для каждого типа ингредиентов отбросьте $10\%$  наименьших значений и $10\%$  наибольших, затем снова определите среднее значение и медиану.

\explanationSection

Если $10\%$ от количества элементов является вещественным числом, округлите его до целого в меньшую сторону. Для выборки из четного количества элементов медиана вычисляется как полусумма двух соседних значений в середине диапазона.

\inputfmtSection

В первой строке подается целое значение $ N \space (1 \leq N \leq 10^6) $ —- количество исследованных салатов. 

Далее в $ N $ строках через пробел подаются три целых числа $ a_i $, $ b_i $ и $ c_i\space (0 \leq a_i, b_i, c_i \leq 100) $ — количество единиц горошка, кукурузы и фасоли в $i$-м салате соответственно.

\outputfmtSection

В четырёх строках выведите по 3 числа -- требуемые параметры для гороха, кукурузы и фасоли соответственно, с точностью не ниже $10^{-6}$:

В первой строке —- среднее значение каждого из ингредиентов в полной выборке.

Во второй строке —- медианное значение каждого из ингредиентов в полной выборке.

В третьей строке —- среднее значение каждого из ингредиентов в усеченной выборке.

В четвертой строке —- медианное значение каждого из ингредиентов в усеченной выборке.

\markSection

Примерно в $ 50\% $ тестов $ 10^5 \leq N \leq 10^6$.

Баллы за задачу будут начисляться пропорционально количеству успешно пройденных тестов.

\sampleTitle{1}

\begin{myverbbox}[\small]{\vinput}
    3
    91 6 34
    64 31 16
    77 5 92
\end{myverbbox}

\begin{myverbbox}[\small]{\voutput}
    77.33333333333333 14.0 47.333333333333336
    77.0 6.0 34.0
    77.33333333333333 14.0 47.333333333333336
    77.0 6.0 34.0
\end{myverbbox}
\inputoutputTable

\sampleTitle{2}

\begin{myverbbox}[\small]{\vinput}
    12
    81 36 38
    17 89 30
    98 70 20
    74 68 20
    78 43 1
    0 17 31
    41 25 34
    14 12 55
    70 23 39
    43 84 83
    93 17 97
    38 91 25
\end{myverbbox}

\begin{myverbbox}[\small]{\voutput}
    53.916666666666664 47.916666666666664 39.416666666666664
    56.5 39.5 32.5
    54.9 47.2 37.5
    56.5 39.5 32.5
\end{myverbbox}


\inputoutputTable


\solutionSection
Медиана —- это такое число выборки, что половина из элементов выборки больше него или равна ему, а другая половина меньше него или равна ему. Для выборки из четного количества элементов медиана вычисляется как полусумма двух соседних значений в середине упорядоченной последовательности элементов выборки. Задача заключается в том, чтобы найти среднее (которое изменяется при усечении выборки) и найти медиану (которая не изменится при усечении выборки).

\includeSolutionIfExistsByPath{1st_tour/inf/try_2/task_05}