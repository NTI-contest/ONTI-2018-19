\assignementTitle{Фрактал}{10}

На сегодняшний день одним из самых быстро развивающихся и перспективных видов компьютерной графики является фрактальная графика. 
На этот раз мы и тебе предлагаем немного порисовать, но в достаточно простой геометрии. Основным повторяющимся элементом нашего 
рисунка будет отрезок длиной nn n  пикселей. 
\begin{itemize}
    \item Первый элемент располагается вертикально, считаем, что нижний конец его закреплен, 
    а верхний свободен. Его длина $s_1 = n$. 
    \item Каждый последующий элемент имеет меньшую длину, которая определяется по формуле 
    $s_{i+1}=a \cdot s_i \space // \space b$,  где знак $//$ обозначает целочисленное деление. Если $s_i = 0$, элемент не отображается.
    \item Для каждого элемента определяется точка крепления нового элемента. Этот элемент мы 
    делим в соотношении $c \space / \space d$ таким образом, чтобы отрезок у свободного конца был короче, чем у закрепленного. 
    Местом прикрепления выбирается пиксель, на который попала точка разбиения элементов на отрезки. Если 
    местоположение пикселя не определяется точно (точка достижения соотношения находится между двумя пикселями), 
    то положение места прикрепления определяется из двух пикселей по принципу близости к свободному концу.
    \item Новый элемент прикрепляется под углом $45^{\circ}$ к наиболее короткой части по часовой стрелке от 
    предыдущего. Для параметров $n = 1000, a = 935, b = 1000, c = 1, d = 1$  фрактал будет иметь вид:
\end{itemize}

\putImgWOCaption{8cm}{1}

Cколько будет закрашенных пикселей на рисунке?

\inputfmtSection
Строка, в которой указаны целочисленные параметры $n, a, b, c, d$ , разделенные пробелом, где 
$(10^{10000} \leq n \leq 10^{10001}$, $1 \leq a,\space b \leq 1000$, $0 < a/b \leq  0.935$, 
$1 \leq c$, $\space d \leq 10 )$

\outputfmtSection
Единственное целое число.

\sampleTitle{1}

\begin{myverbbox}[\small]{\vinput}
    20 2 5 1 2
\end{myverbbox}

\begin{myverbbox}[\small]{\voutput}
    32
\end{myverbbox}
\inputoutputTable

\explanationSection
\begin{enumerate}
    \item Первый элемент длиной $20$. Определяем длину следующего элемента: $20 \cdot 2\space // \space 5 = 8$. Определяем точку 
    крепления нового элемента: $20 \cdot 1 / (1 + 2) = 6 \frac{2}{3}$  — попадает на 7-й от свободного края пиксель. Крепим элемент 
    к первому элементу под углом $45^{\circ}$ к короткой части по направлению по часовой стрелке. 
    \item Второй элемент $8$. Определяем длину следующего элемента: $8 \cdot 2 \space // \space 5 = 3$. Определяем точку крепления 
    нового элемента: $8 \cdot 1 / (1 + 2) = 2 \frac{2}{3}$  — попадает на 3-й от свободного края пиксель. Крепим его ко второму 
    элементу под углом $45^{\circ}$ к короткой части по направлению по часовой стрелке. 
    \item Третий элемент 3. Определяем длину следующего элемента: $3 \cdot 2 \space // \space 5 = 1$. Определяем точку крепления 
    нового элемента: $3 \cdot 1 / (1 + 2) = 1$  -- попадает на стык, поэтому выбираем пиксель ближе к свободному краю. 
    Крепим его к третьему элементу под углом $45^{\circ}$ к короткой части по направлению по часовой стрелке.
    \item Четвертый элемент 1. Определяем длину следующего элемента: $1 \cdot 2 \space // \space 5 = 0$. Значит, этот элемент последний.
\end{enumerate}

На иллюстрации красным и сиреневым обозначены элементы фрактала, серым точки, крепления.

\putImgWOCaption{4cm}{2}

\includeSolutionIfExistsByPath{1st_tour/inf/try_2/task_05}