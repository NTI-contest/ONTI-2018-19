\assignementTitle{Queen}{10}{}

Валера имеет плохую память на номера телефонов. Особенно тяжело ему приходилось с номерами мобильных телефонов, у которых всегда приходится запоминать помимо 7 цифр номера, еще две, определяющие код оператора (8(9**)-***-**-**). Про телефонную книгу в телефоне он знал, но запоминание было принципом.

Узнав про системы счисления он возрадовался тому, что числа можно запоминать при помощи букв, да и при большом основании они будут ощутимо короче. Благодаря родителям он очень полюбил группу Queen, и когда пришла пора дарить первый телефон младшей сестренке, Валера выбрал ей номер так, что в некоторой системе счисления число из 9 цифр для запоминания было равно QUEEN и основание системы было наибольшее из возможных для данного формата номера телефона. Напоминаем, что при нехватке цифр для записи разрядов используются буквы латинского алфавита. Запишите в ответе те самые 9 цифр.

\solutionSection
Проще всего данную задачу решать методом программного перебора. Очевидно, что основание системы счисления не превышает 100, так как при помощи 5 цифр некоторой системы счисления кодируется девятизначное число в десятичной системе счисления.

\includeSolutionIfExistsByPath{1st_tour/inf/try_1/task_01}

\answerMath{976714307.}

