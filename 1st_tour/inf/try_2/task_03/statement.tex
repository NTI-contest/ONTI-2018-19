\assignementTitle{Little to Big}{10}{}

В современной вычислительной технике и цифровых системах связи информация обычно представлена в виде последовательности байтов. Если число не может быть представлено одним байтом, имеет значение, в каком порядке байты записываются в памяти компьютера или передаются по линиям связи. Часто выбор порядка записи байтов произволен и определяется только соглашениями.

В общем случае, для представления числа $M$, большего $255$, приходится использовать несколько байтов. При этом число $M$ записывается в позиционной системе счисления с основанием $256$:
$$ M=\sum\limits _{i=0}^{n}A_{i}\cdot 256^{i}=A_{0}\cdot 256^{0}+A_{1}\cdot 256^{1}+A_{2}\cdot 256^{2}+\dots +A_{n}\cdot 256^{n}. $$

Набор целых чисел $ A_{0},\dots ,A_{n} $, каждое из которых лежит в интервале от $ 0 $ до $ 255 $, является последовательностью байтов, составляющих $ M $. При этом $ A_{0} $—младший байт, $  A_{n}  $ -— старший байт числа $ M$.

Есть несколько способов записи целых чисел:

\begin{enumerate}
    \item Big-Endian —- запись числа от старшего байта к младшему,
    \item Little-Endian —- запись числа от младшего байта к старшему.
\end{enumerate}

Ваш компьютер поддерживает big-endian, а требуется работать с устройством, которое посылает 32-битные беззнаковые целые числа в формате little-endian. Напишите программу, которая адаптирует формат получаемого числа для работы на компьютере. 


\inputfmtSection

В единственной строке целое неотрицательное $ 32 $-битное число.

\outputfmtSection

Единственное число —- десятичная запись числа в формате big-endian, полученная путем перевода этого числа из little-endian.

\markSection

Баллы за задачу будут начислены, если все тесты будут пройдены успешно

\sampleTitle{1}

\begin{myverbbox}[\small]{\vinput}
3496683923
\end{myverbbox}
\begin{myverbbox}[\small]{\voutput}
2468965328
\end{myverbbox}
\inputoutputTable

\solutionSection

В этой задаче важно обратить внимание на то, что длина подаваемого на вход числа равна 32 битам, то есть 4 байтам. То есть, например, для числа 255 ответ 4278190080, а не 255.

\includeSolutionIfExistsByPath{1st_tour/inf/try_2/task_03}