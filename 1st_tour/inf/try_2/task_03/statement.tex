\assignementTitle{Расхождение в ДНК}{10}

Исследователи решили проанализировать участки ДНК двух образцов. 
Каждый участок ДНК имеет некоторый состав нуклеотидов. Нуклеотиды ДНК образованы одним из следующих азотистых 
оснований: аденин, гуанин, тимин, цитозин, которые кодируются в цепочках ДНК при помощи заглавных 
букв по первой букве основания А, Г, Т, Ц, соответственно.

Помогите учёным определить долю расхождений в цепочках ДНК, если под расхождением понимается неравенство 
соответствующих нуклеотидов в цепочках, находящихся на одной позиции, считая от левого края.

\inputfmtSection
На вход программе подаётся целое число $N\space(1\leq N \leq 10^7)$.  
Затем две строки в каждой из которых записана цепочка длиной $N$ из латинских букв $A$, $G$, $T$, $C$, 
соответствующих азотистым основаниям.

\outputfmtSection
Единственное число — ответ на задачу с точностью не ниже $10^{-7}$.

\markSection

Баллы за задачу будут начисляться пропорционально количеству успешно пройденных тестов.

\sampleTitle{1}

\begin{myverbbox}[\small]{\vinput}
10
AGTCCGTCAG
AGTGCCTCAG
\end{myverbbox}
\begin{myverbbox}[\small]{\voutput}
0.2
\end{myverbbox}
\inputoutputTable

\includeSolutionIfExistsByPath{1st_tour/inf/try_2/task_03}