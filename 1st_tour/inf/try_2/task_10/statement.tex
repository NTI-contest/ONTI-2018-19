\assignementTitle{Теория шести рукопожатий}{10}{}

Валентина решила проверить теорию "6 рукопожатий" в некоторой социальной сети. Согласно данной теории, все люди друзья со всеми, если не напрямую, то максимум через цепочку из 5 друзей. То есть если Маша подруга Даши, то они подруги через 1 рукопожатие. Если Даша -- подруга Саши, а Саша и Маша не подруги напрямую, то они подруги через 2 рукопожатия. Если у Саши есть друг Миша, с которым не знакомы Даша и Маша, то Миша и Даша -- друзья через 2 рукопожатия, а Миша и Маша -- через 3 (Маша - Даша - Саша - Миша).

Валентина выгрузила список дружеских связей в формате пар уникальных никнеймов. Если пользователь $a$ друг пользователю $b$, то $b$ -- друг пользователю $a$. Помогите Валентине проанализировать пользователей социальной сети.

\inputfmtSection

На вход программе подается целое число $ n\space(2\leq n\leq 200) $ -- количество пользователей социальной сети. 

Затем в $ n $ строках подаются никнеймы. Каждый никнейм уникален, состоит из строчных латинских знаков, не короче 3 знаков и не превышает 15 знаков по длине. 

Далее подается число $  m\space(1\leq m\leq 10000)$ -- количество выгруженных дружеских связей. 

Затем в $m$ строках записываются пары слов -- никнеймы пользователей социальной сети, которые являются прямыми (через 1 рукопожатие) друзьями. Данные пары в списке могут повторяться

\outputfmtSection

Выведите в отдельной строке Yes, если теория "6 рукопожатий" для данной соцсети истинна, No -- в противном случае.

\markSection

Баллы за задачу будут начисляться пропорционально количеству успешно пройденных тестов.

\sampleTitle{1}

\begin{myverbbox}[\small]{\vinput}
    8
    dasha
    masha
    sasha
    misha
    dima
    kolya
    igor
    valya
    7
    dasha masha
    misha dima
    masha sasha
    kolya igor
    sasha misha
    dima kolya
    kolya valya
\end{myverbbox}

\begin{myverbbox}[\small]{\voutput}
    Yes
\end{myverbbox}
\inputoutputTable

\solutionSection

В данной задаче можно воспользоваться алгоритмом Флойда-Уоршелл нахождения кратчайших путей между всеми парами вершин. В данном случае вершины -- пользователи социальной сети, ребра -- прямая связь между двумя пользователями. Вес ребра равен 1 -- если есть прямая связь, $\infty$ (в данной задаче любое число больше максимального количества пользователей) -- если прямой связи нет.

Если между какими-либо двумя вершинами путь превышает 6, то выводим "No". В противном случае "Yes".

\includeSolutionIfExistsByPath{1st_tour/inf/try_2/task_10}