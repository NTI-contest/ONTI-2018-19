\assignementTitle{}{10}

Вокруг точки с географическими координатами $24^{\circ}42’08''$ южной широты и \linebreak $63^{\circ}35’49''$
западной долготы практически полностью отсутствует лесной покров на площади
более 2 тысяч гектаров. Эта территория лишилась лесного покрова за очень короткий период времени 
в результате хозяйственной деятельности человека. Из них, по крайней мере, около 1,5 тысяч гектаров 
ближайших к указанной точке лишись своего лесного покрова в одном календарном году.

Воспользуйтесь доступными в интернете источниками открытых космических снимков и других пространственных данных. 
Например, порталами Всемирной лесной вахты (\url{https://www.globalforestwatch.org/map}), 
Worldview \\ (\url{https://worldview.earthdata.nasa.gov/}) американского космического агентства 
НАСА и LandLook Viewer (\url{https://landlook.usgs.gov/viewer.html}) Геологической службы США (USGS). С их помощью определите год (четырёхзначное число), 
когда это произошло.

Помните, что использование только одного источника информации может привести Вас к неверным выводам (особенно, если данный источник опирается на результаты массовой автоматизированной обработки космических снимков по стандартным алгоритмам, которые могут не учитывать региональных особенностей). Постарайтесь проверить Ваши выводы, по крайней мере, по двум разным источникам.

\answerMath{2009.}