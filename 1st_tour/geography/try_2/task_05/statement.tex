\assignementTitle{}{10}

Вокруг точки с географическими координатами $65^{\circ}30’2''$ северной широты и $41^{\circ}25’23''$ восточной долготы в период с 2010 по 2013 гг. исчез лесной покров на площади около 1300 гектаров. В результате какого процесса или явления данная территория утратила лесной покров? Выберите только один вариант из представленных ниже.

Воспользуйтесь доступными в интернете источниками открытых космических снимков и 
других пространственных данных. Например, геопорталами Яндекс Карты (\url{https://yandex.ru/maps/}), 
Google Планета Земля (\url{https://www.google.com/earth/}), Всемирной лесной вахты 
(\url{https://www.globalforestwatch.org/map}) и WorldView (\url{https://worldview.earthdata.nasa.gov/}) американского космического агентства НАСА.

\begin{enumerate}
    \item извержение вулкана
    \item лесной пожар
    \item расчистка территории под посевы сельскохозяйственных культур
    \item добыча алмазов открытым способом
    \item заготовка древесины
    \item массовая вспышка размножения насекомых-вредителей леса
    \item добыча угля открытым способом
    \item заболачивание (избыточное увлажнение) территории
    \item падение метеорита
    \item добыча нефти и газа
    \item испытание ядерного оружия
    \item расчистка территории под пастбище
\end{enumerate}

\solutionSection

\answerMath{4.}