\assignementTitle{}{10}

В феврале 1971 года в г.Рамсар (Иран) была принята Конвенция о водно-болотных угодьях, имеющих международное значение. Конвенция действует до сих пор и является глобальным международным договором, посвященным одному типу экосистем - водно-болотным угодьям (wetlands). 
Согласно определению Конвенции, к водно-болотным угодьям относится болота, поймы, реки, ручьи, каналы,
пресные и соленые озёра, заболоченные луга и леса, 
 приморские солёные марши и мангровые заросли, эстуарии, подводные морские луга, коралловые рифы и морские отмели и акватории глубиной не более шести метров при отливе, 
подземные карстовые водоемы, водохранилища и др. В настоящее время (по состоянию на сентябрь 2018 года) участниками Конвенции являются 170 стран-членов ООН. Советский Союз был среди первых подписантов Конвенции, которая вступила в нём в силу в 1977 году.

Подписавшие Конвенцию страны, в частности, берут на себя обязательство определить наиболее ценные участки водно-болотных угодий на своей территории для внесения в Список водно-болотных угодий, имеющих международное значение (Рамсарский список), и обеспечить их эффективную охрану. Каждая страна должна включить в список хотя бы одну свою территорию, соответствующую утверждённым критериям, чтобы присоединиться к Конвенции. От России, как правопреемника СССР, в Рамсарский список в настоящий момент 
включено 35 территорий.

На территории этой страны находится водно-болотное угодье, включенное в Рамсарский список самым первым, - \makebox[2cm]{\hrulefill}$^1$

На территории этой страны находится самое крупное по площади водно-болотное угодье, включенное в Рамсарский список, - \makebox[2cm]{\hrulefill}$^2$

На территории этой страны находится водно-болотное угодье из Рамсарского списка, включающее бОльшую часть крупнейшего в мире массива мангровых лесов, - \makebox[2cm]{\hrulefill}$^3$

На территории этой страны находится самая большая из нескольких территорий Рамсарского списка, охраняющих части одной из крупнейших (наряду с поймой Амазонки и болотами бассейна реки Конго) заболоченных территорий тропиков. Однако, бОльшая часть самого этого заболоченного региона находится вне каких-либо территорий Рамсарской конвенции. - \makebox[2cm]{\hrulefill}$^4$

Страна с наибольшей суммарной площадью водно-болотных угодий, включенных в Рамсарский список (по состоянию на сентябрь 2018 г.), - \makebox[2cm]{\hrulefill}$^5$

Страна, имеющая самое большое число водно-болотных угодий, включенных в Рамсарский список (по состоянию на сентябрь 2018 г.), - \makebox[2cm]{\hrulefill}$^6$

На территории этой страны находится крупнейшее болото северного полушария, не включенное в Рамсарский список, но входящее в так называемый "теневой список" территорий-кандидатов на включение, - \makebox[2cm]{\hrulefill}$^7$

На территории этой страны находится водно-болотное угодье, включенное в Рамсарский список последним по времени (в августе 2018 г.), - \makebox[2cm]{\hrulefill}$^8$

\solutionSection

\answerMath{
    \begin{enumerate}
        \item Австралия
        \item Бразилия
        \item Бангладеш
        \item Боливия
        \item Бразилия
        \item Великобритания
        \item Россия
        \item Мьянма
    \end{enumerate}
}