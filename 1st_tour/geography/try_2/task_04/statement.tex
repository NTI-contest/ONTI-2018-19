\assignementTitle{}{10}

Опираясь на данные портала Всемирной лесной вахты \\ (\url{https://www.globalforestwatch.org/}), определите, 
какая из перечисленных ниже тропических стран, согласно данным наблюдения из космоса, 
является лидером по абсолютным цифрам площади потерь лесного покрова (tree cover loss) за период 2013-2017 гг.

Для целей данного задания считать лесопокрытой любую территорию, которая имеет древесную растительность (как
естественного, так и искусственного происхождения) с сомкнутостью древесного полога (tree cover) не менее $10\%$ и высотой не менее 5 метров. (Близкое к этому определение леса используется в большинстве стран мира и международными организациями, хотя их данные часто очень сильно отличаются от результатов
космического мониторинга.)

Используйте показатель всех суммарных потерь древесного полога, включая его временные потери, компенсируемые восстановлением древесной растительности.

\begin{enumerate}
    \item Мадагаскар
    \item Бразилия
    \item Демократическая республика Конго
    \item Колумбия
    \item Индонезия
    \item Парагвай
    \item Вьетнам
\end{enumerate}

\answerMath{2.}