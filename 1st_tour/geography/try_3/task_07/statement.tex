\assignementTitle{}{10}

Почвы бассейна реки Амазонки в Южной Америке чрезвычайно бедны фосфором, который вымывается рекой и осадками. 
Однако, по оценкам учёных эти потери практически полностью компенсируются за счет песка и пыли, приносимых ветром 
за тысячи километров из Сахары и оседающих в джунглях Амазонии. Приводимые в научных публикациях цифры оценивают 
массу ежегодно оседающей в Южной Америке африканской пыли в 50 млн.тонн ежегодно.

Как показали исследования воздушных потоков и аэрозолей, проведённые с помощью космических снимков, а
 также оценки на основе наземных измерений, более половины этого количества происходит из весьма ограниченной 
 территории размером всего около $0,5\%$ территории Амазонии и $0,2\%$ территории Сахары. Уникальное сочетание 
 рельефа и климатических условий создает здесь естественную "аэродинамическую трубу" между двух горных массивов, 
 в которой ветер у поверхности земли может разгоняться до средних скоростей 12-13 метров в секунду и дуть около 100 дней в году.

Анализ приносимого песка показал, что он содержит большое количество останков диатомовых водорослей 
(которые собственно и являются основным источником фосфора), то есть происходит со дна высохшего озера. 
Действительно, есть достаточно свидетельств того, что данное понижение рельефа (депрессия) было в прошлом, 
при более влажном климате, заполнено водой древнего мегаозера. Как называется эта местность?

\soultionSection

\answerMath{Депрессия Боделе.}