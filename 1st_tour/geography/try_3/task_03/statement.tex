\assignementTitle{}{10}

Назовите озеро, видимое на этом спутниковом изображении, - одно из крупнейших в стране и на континенте. Введите его название по-русски в поле ниже.

\putImgWOCaption{10cm}{1}

Приведённое выше изображение получено со спутников Landsat 7 и Landsat 8. Это - не единовременный снимок, 
а синтез ряда безоблачных снимков за 2017 год, обработанных таким образом, чтобы компенсировать всегда 
имеющиеся различия в положении солнца, положении спутника, прозрачности атмосферы и пр. 
Обработка проведена лабораторией GLAD (Global Land Analysis \& Discovery; \url{https://glad.umd.edu/}) Географического факультета Университета Мэриленда (США). 
Данное изображение сделано с помощью портала Глобальные изменения лесного покрова (\url{http://earthenginepartners.appspot.com/science-2013-global-forest}).

Изображение представляет интенсивность электромагнитного излучения, отражённого от поверхности земли и принятого аппаратурой спутников, в условных цветах: не видимый человеческим глазом коротковолновый инфракрасный свет представлен красным, ближний инфракрасный (также не различимый глазом) - зелёным, видимый красный свет - синим. В таком цветовом синтезе хорошо видны различия в растительности и влажности поверхности.

\answerMath{Мар-Чикита.}