\assignementTitle{}{10}

Вокруг точки с географическими координатами $63^{\circ} 15’46''$ северной широты и $46^{\circ} 14’38''$ 
восточной долготы практически полностью отсутствует лесной покров на площади более 600 гектаров.

В результате какого процесса или явления данная территория не имеет лесного покрова? 
Выберите только один вариант из представленных ниже.

Воспользуйтесь доступными в интернете источниками открытых космических снимков и 
других пространственных данных. Например, геопорталами Яндекс Карты (\url{https://yandex.ru/maps/}), 
Google Планета Земля (\url{https://www.google.com/earth/}), Всемирной лесной вахты 
(\url{https://www.globalforestwatch.org/map}) и WorldView (\url{https://worldview.earthdata.nasa.gov/}) американского космического агентства НАСА.

\begin{enumerate}
    \item заболачивание (избыточное увлажнение) территории
    \item добыча алмазов открытым способом
    \item добыча нефти и газа
    \item расчистка территории под пастбище
    \item лесной пожар
    \item извержение вулкана
    \item добыча угля открытым способом
    \item расчистка территории под посевы сельскохозяйственных культур
    \item заготовка древесины
    \item падение метеорита
    \item массовая вспышка размножения насекомых-вредителей леса
    \item испытание ядерного оружия
\end{enumerate}

\soultionSection

\answerMath{5.}