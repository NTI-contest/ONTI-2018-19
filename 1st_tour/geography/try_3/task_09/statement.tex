\assignementTitle{}{10}

Леса этого засушливого региона Южной Америки исчезают сегодня быстрее любых других тропических лесов на планете - 
сводятся под пастбища и посевы сельскохозяйственных культур. Северная часть региона впервые была подробно исследована 
русским географом, этнографом, антропологом и лингвистом. Его экспедициями была впервые проведена топографическая 
съёмка этой местности, изучены культура и быт населявших территорию индейцев. Права коренных жителей региона этот 
человек защищал до конца своей жизни, в том числе на высоких государственных должностях. По его завещанию, его останки 
были переданы этим индейским племенам и до сих пор бережно ими сохраняются.

Потомственный военный, принимавший участие в 
Первой мировой войне и в Гражданской войне в России, он, вместе с другими выходцами из России, сыграл заметную роль в 
самой кровопролитной войне XX века в Латинской Америке, которая велась именно за регион, который он изучал. 
(Поддержка его экспедиций была, фактически, частью подготовки к войне одной из стран-участниц.) Сведения о его 
непосредственном участии в боевых действиях разнятся. Однако, он бесспорно привлекался к разработке боевых операций 
как знаток региона. А поддержка местных индейцев, с которыми он установил отношения, стала важным фактором победы в войне одной из сторон. 
Ему было присвоено звание почетного гражданина страны-победительницы, где он и прожил остаток жизни. 
По иронии судьбы, именно эта страна сегодня быстрее всего теряет естественные леса этого региона.

Назовите фамилию этого человека. Укажите только фамилию по-русски в поле ниже.

 

\answerMath{Иван Тимофеевич Беляев.}