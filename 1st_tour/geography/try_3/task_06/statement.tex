\assignementTitle{}{10}

Биом (\url{https://ru.wikipedia.org/wiki/%D0%91%D0%B8%D0%BE%D0%BC}) - совокупность экосистем одной природно-климатической зоны. В простейшем случае выделяют 6 
основных типов наземных биомов: тундра, хвойные леса, листопадные леса, тропические леса, 
степь и пустыня. Однако, чаще используются более сложные системы классификации. Так, 
Всемирный фонд дикой природы (WWF) для создания научной системы приоритетов в охране 
биологического разнообразия (\url{https://ru.wikipedia.org/wiki/Global_200}) нашей планеты выделил 867 наземных экорегионов (\url{https://ru.wikipedia.org/wiki/%D0%AD%D0%BA%D0%BE%D0%BB%D0%BE%D0%B3%D0%B8%D1%87%D0%B5%D1%81%D0%BA%D0%B8%D0%B9_%D1%80%D0%B5%D0%B3%D0%B8%D0%BE%D0%BD}), которые объединены в 14 биомов. 
Пожалуй, это одна из самых подробных и разработанных классификаций (\url{https://www.worldwildlife.org/publications/terrestrial-ecoregions-of-the-world}) такого рода в мире. На уровне биомов она достаточно близка к классификации природных зон, принятых в отечественной науке.

Поставьте в соответствие каждой стране из списка преобладающий на её территории биом, то есть биом, имеющий самую большую площадь в пределах данной страны. Используйте систему классификации биомов Всемирного фонда дикой природы (WWF). Выберите только ОДИН биом для каждой страны.

    \begin{enumerate}
        \item Аргентина
        \item Колумбия
        \item Намибия
        \item Нигерия
        \item Россия
        \item Франция
    \end{enumerate}
    
    \begin{enumerate}
        \item[a.] Тропические и субтропические влажные лиственные леса
        \item[б.] Тропические и субтропические сухие лиственные леса
        \item[в.] Тропические и субтропические хвойные леса
        \item[г.] Лиственные и смешанные леса умеренного пояса
        \item[д.] Хвойные леса умеренного пояса
        \item[е.] Бореальные леса (тайга)
        \item[ж.] Тропические и субтропические травянистые экосистемы (прерии, степи и пр.), саванны и кустарники
        \item[з.] Травянистые экосистемы (прерии, степи и пр.), саванны и кустарники умеренного пояса
        \item[и.] Затапливаемые травянистые экосистемы и саванны
        \item[к.] Горные травянистые экосистемы (луга, горные степи) и кустарники
        \item[л.] Тундра
        \item[м.] Средиземноморские леса, редколесья и кустарники
        \item[н.] Пустыни и ксерофитные кустарники
        \item[о.] Мангровые заросли        
    \end{enumerate}

\solutionSection

\answerMath{1 - з, 2 - а, 3 - н, 4 - ж, 5 - е, 6 - г.}