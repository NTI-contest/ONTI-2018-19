\assignementTitle{}{10}

Опираясь на данные портала Всемирной лесной вахты (\url{www.globalforestwatch.org}), 
определите, какая из перечисленных ниже тропических стран, согласно данным наблюдения из космоса, 
является лидером по доле (проценту) потерь лесного покрова (tree cover loss) за период 2013-2017 гг. 
от его общей площади.

Для целей данного задания считать лесопокрытой любую территорию, которая имеет древесную растительность 
(как естественного, так и искусственного происхождения) с сомкнутостью древесного полога (tree cover) 
не менее $10\%$ и высотой не менее 5 метров. (Близкое к этому определение леса используется в большинстве 
стран мира и международными организациями, хотя их данные часто очень сильно отличаются от результатов 
космического мониторинга.)

Используйте показатель всех суммарных потерь древесного полога в процентах от его общей площади по состоянию на 2010 год, включая его временные потери, компенсируемые восстановлением древесной растительности.

\begin{enumerate}
    \item Бразилия
    \item Индонезия
    \item Демократическая республика Конго
    \item Парагвай
    \item Колумбия
    \item Мадагаскар
    \item Вьетнам
\end{enumerate}

\answerMath{4.}