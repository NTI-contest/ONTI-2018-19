\assignementTitle{}{10}{}

Вокруг точки с географическими координатами $0^{\circ}13’11''$ южной широты и \linebreak $102^{\circ}53’33''$
восточной долготы расположен огромный массив плантаций быстрорастущих деревьев (эвкалипты и/или акации), выращиваемых на целлюлозу. 
По сведениям портала Всемирной лесной вахты (Global Forest Watch), древесина с этих плантаций поставляется на 
целлюлозно-бумажный комбинат одной из крупнейших мировых компаний на этом рынке - Asia Pulp \& Paper. 
Зная площади, занятые посадками разного возраста, можно оценить объём древесины, поставляемый на 
заводы компании с плантаций, и прогнозировать устойчивость их снабжения сырьем (в том числе оценить, 
будет ли компания вынуждена уничтожать естественные тропические леса или древесины с плантаций ей будет достаточно). 
Доступные сегодня космические снимки позволяют это сделать.

Участок площадью около 150 гектаров вокруг указанной точки является однородным насаждением, 
посаженным практически одновременно. Определите возраст деревьев на этом участке. 
Для этого воспользуйтесь доступными в интернете источниками открытых космических снимков и других 
пространственных данных. Например, порталами Всемирной лесной вахты (Global Forest Watch; \url{https://www.globalforestwatch.org/map}), 
LandLook Viewer (\url{https://landlook.usgs.gov/viewer.html}) Геологической службы США (USGS) и WorldView (\url{https://worldview.earthdata.nasa.gov/}) американского космического агентства НАСА. 
С их помощью определите год, когда на данном участке ПОСЛЕДНИЙ ПО ВРЕМЕНИ раз был полностью сведён древесный покров 
(вырублено и отправлено на переработку предыдущее поколение быстрорастущих деревьев). 
Посадка следующего поколения обычно происходит вскоре после этого. Рост и смыкание крон нового поколения посаженных 
деревьев также можно видеть на космических снимках.

Округлите Вашу оценку возраста до целого числа лет и внесите её в поле ниже.

Помните, что использование только одного источника информации может привести Вас к неверным выводам 
(особенно, если данный источник опирается на результаты массовой автоматизированной обработки 
космических снимков по стандартным алгоритмам, которые могут не учитывать региональных особенностей). 
Постарайтесь проверить Ваши выводы, по крайней мере, по двум разным источникам.

 

\answerMath{3.5.}