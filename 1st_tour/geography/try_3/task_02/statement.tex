\assignementTitle{}{10}{}

Точка с географическими координатами $2^{\circ} 30’00''$ южной широты и $64^{\circ} 30’00''$ западной 
долготы находится в пределах крупной особо охраняемой природной территории (ООПТ) - резервата устойчивого развития "Амана" 
(Amanã Sustainable \linebreak Development Reserve; \url{https://en.wikipedia.org/wiki/Aman%C3%A3_Sustainable_Development_Reserve}). 
Вместе с национальным парком "Жау" и ещё несколькими природными резерватами они образуют Комплекс резерватов Центральной Амазонии (\url{http://whc.unesco.org/en/list/998}) общей площадью свыше 60 тыс. кв.км. Это крупнейший комплекс ООПТ, сохраняющий влажные тропические леса бассейна реки Амазонки, объект Всемирного природного наследия ЮНЕСКО.

Антипод данной географической точки также находится внутри одного их крупнейших относительно сохранных массивов 
влажных тропических лесов своего региона и тоже в границах обширного национального парка. Назовите этот парк. Название дайте латинскими буквами на государственном языке страны, где он находится.

\answerMath{Kayan Mentarang.}