\assignementTitle{}{10}

Потеря части лесного покрова происходит почти во всех странах мира. Она может быть временной: через некоторое время древесный полог восстанавливается. Но может быть и практически необратимой - тогда говорят об обезлесивании (deforestation). Результаты спутникового мониторинга позволяют наблюдать потерю древесного полога (tree cover loss) в близком к реальному масштабу времени, а также получать ежегодную статистику по всему миру. С некоторыми данными можно ознакомиться, например, на портале 

Всемирной лесной вахты (\url{https://www.globalforestwatch.org/map}) или с помощью портала Глобальные изменения лесного покрова (\url{http://earthenginepartners.appspot.com/science-2013-global-forest}).

Лесной покров может исчезать и по естественным причинам. Например, в результате заболачивания территории или лесных пожаров, случающихся от молний. Но в подавляющем большинстве случаев наблюдаемые сегодня потери лесного покрова происходят в результате деятельности человека. (В том числе, по его вине происходит и подавляющее большинство лесных пожаров.) В разных частях света, в разных странах и регионах с различными системами ведения хозяйства непосредственные причины потерь древесного полога различны. Различна и степень его восстановления после антропогенного воздействия.

Выберите для каждой страны ведущую причину потери лесного покрова (постоянной или временной) в 2017 году. Выберите только ОДИН ведущий фактор для каждой страны.

\begin{multicols}{2}
    \begin{enumerate}
        \item Бразилия
        \item Демократическая республика Конго 
        \item Индонезия
        \item Канада
        \item Россия
        \item Финляндия
    \end{enumerate}
    
    \begin{enumerate}
        \item[a.] Добыча полезных ископаемых
        \item[б.] Животноводство (расчистка под пастбища)
        \item[в.] Заготовка древесины
        \item[г.] Лесные пожары
        \item[д.] Подсечно-огневое земледелие
        \item[е.] Расчистка под плантации древесных культур
    \end{enumerate}
\end{multicols}


\soultionSection

\answerMath{1 - б, 2 - д, 3 - е, 4 - г, 5 - г, 6 - в.}