\assignementTitle{}{10}

Вокруг точки с географическими координатами $65^{\circ}03’26''$ северной широты и $32^{\circ}16’59''$ 
восточной долготы практически полностью отсутствует лесной покров на площади более 2,5 тыс. гектаров.

В результате какого процесса или явления данная территория не имеет лесного покрова? Выберите только один вариант из представленных ниже.

Воспользуйтесь доступными в интернете источниками открытых космических снимков и других пространственных данных. 
Например, порталами Яндекс Карты (\url{https://yandex.ru/maps/}), Google Планета Земля (\url{https://www.google.com/earth/}), 
Всемирной лесной вахты (\url{https://www.globalforestwatch.org/map}) и WorldView (\url{https://worldview.earthdata.nasa.gov/}) Американского космического агентства НАСА.

\begin{enumerate}
    \item заготовка древесины
    \item падение метеорита
    \item массовая вспышка размножения насекомых-вредителей леса
    \item добыча нефти и газа
    \item лесной пожар
    \item расчистка территории под пастбище
    \item добыча алмазов открытым способом
    \item извержение вулкана
    \item заболачивание (избыточное увлажнение) территории
    \item расчистка территории под посевы сельскохозяйственных культур
    \item испытание ядерного оружия
    \item добыча угля открытым способом
\end{enumerate}

\soultionSection

\answerMath{9.}