\assignementTitle{}{10}

Вокруг точки с географическими координатами $61^{\circ} 23’ 19''$ северной широты и $122^{\circ} 14’ 31''$
восточной долготы отсутствует лесной покров на площади
более 4 тысяч гектаров.

Воспользуйтесь доступными в интернете источниками открытых космических снимков и других пространственных данных. 
Например, порталами Всемирной лесной вахты (\url{https://www.globalforestwatch.org/map}), Космоснимки.RU (\url{http://kosmosnimki.ru/})
российской компании Сканэкс, Worldview (\url{https://worldview.earthdata.nasa.gov/}) американского космического агентства НАСА и 
LandLook Viewer (\url{https://landlook.usgs.gov/viewer.html}) Геологической службы США (USGS). 
С их помощью определите год (четырёхзначное число), когда процесс или явление, 
в результате которого данная территория утратила свой лесной покров, имели место.

Помните, что использование только одного источника информации может привести Вас к неверным выводам (особенно, если данный источник опирается на результаты массовой автоматизированной обработки космических снимков по стандартным алгоритмам, которые могут не учитывать региональных особенностей). Постарайтесь проверить Ваши выводы, по крайней мере, по двум разным источникам.

\soultionSection

\answerMath{2011.}