\assignementTitle{}{10}

Река Окаванго в юго-западной Африке широко известна, как "река без устья", "река, впадающая в пустыню". Действительно, почти вся приносимая Окаванго вода испаряется в заболоченной и мелководной Дельте Окаванго - крупнейшей внутриматериковой дельте планеты. Эта уникальное водно-болотное угодье - главный источник воды в этом регионе и местообитание множества растений и животных. Про неё сняты фильмы, территория привлекает множество туристов.

Однако, утверждение, что река кончается здесь - не вполне точно. Часть воды течет через дельту дальше - в реки Тамалакане и Ботети (Ботлетли, Ботлетле) - и в сезон дождей дотекает до пересыхающих солёных озёр во впадине Макгадикгади. В особо полноводные годы вода из дельты питает расположенное южнее дельты озеро Нгами, а также - через обычно сухое русло реки Магвеквана (также известное как Селинда Спиллвэй) - попадает в болота Линянти (Linyanti) к северу от дельты, а оттуда - в реку Квандо, приток реки Замбези. Таким образом, часть воды Окаванго таки минует пойму, а небольшая её часть в полноводные годы, всё-таки, попадет в Индийский океан.

Подобные внутриматериковые дельты есть и на других реках Африки. Как и в случае Окаванго, они образовались в местах, где реки когда-то впадали в древние, высохшие сейчас озёра. (Так, Окаванго когда-то впадала в озеро Макгадикгади, заполнявшее одноименную впадину.) И хотя другие внутриматериковые дельты не так известны, отличия от Дельты Окаванго, скорее, количественные. Так, другая крупная африканская река, протекая через свою заболоченную внутриматериковую дельту теряет, в среднем, две трети своей воды и дотекает до страны, в которой находится её устье, существенно ослабленной. От своих притоков в этой стране данная река получает в шесть раз больше воды, чем приносит сама, что позволяет ей образовать при впадении в океан уже "настоящую", хорошо развитую дельту.

Назовите эту реку, которой в этом смысле "повезло" больше, чем Окаванго.

\answerMath{Нигер.}