\assignementTitle{}{6}

В молекулярной биологии широко используют
эндонуклеазы рестрикции, ферменты, которые умеют разрезать двухцепочечную ДНК в
определенном сайте. Сайт узнавания одной из рестриктаз представляет собой
последовательность GGATCC. Вы обрабатываете эндонуклеазой рестрикции кольцевую
ДНК одной бактерии. Известно, что содержание GC пар у бактерии составляет $60\%$.


Предполагая, что нуклеотдиды в геномной ДНК данной бактерии встречаются случайно, определите, какова ожидаемая длина фрагментов, полученных в результате обработки? Ответ
приведите в тысячах нуклеотидов, 
округлите до десятых долей.

\explanationSection

\answerMath{3086.}