\assignementTitle{}{6}

В молекулярной биологии широко используют
эндонуклеазы рестрикции, ферменты, которые умеют разрезать двухцепочечную ДНК в
определенном сайте. Сайт узнавания одной из рестриктаз представляет собой
последовательность GGATCC. Вы обрабатываете эндонуклеазой рестрикции кольцевую
ДНК одной бактерии. Известно, что содержание GC пар у бактерии составляет $60\%$.


Предполагая, что нуклеотдиды в геномной ДНК данной бактерии встречаются случайно, определите, какова ожидаемая длина фрагментов, полученных в результате обработки? Ответ
приведите в тысячах нуклеотидов, 
округлите до десятых долей.

\solutionSection

Длина фрагмента обратно пропорциональна частоте встречаемости сайта. 
Если предположить, что нуклеотдиды встречаются случайно, то вероятность встретить сайт 
AGATCT $p(AGATCT)=p(G) \cdot p(G) \cdot p(A) \cdot p(T) \cdot p(C) \cdot p(C)= (0,3)^4 \cdot (0,2)^2$. 
Тогда средняя длина фрагмента $l=1/p(AGATCT) =  3086$ п.н.

\answerMath{3086.}