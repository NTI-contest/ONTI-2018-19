\assignementTitle{}{8}

Первичная обработка зрительной информации происходит в
сетчатке глаза. Каждая биполярная клетка обладает рецептивным полем, в которое
входит несколько колбочек или палочек. По характеру ответа на освещение
рецептивного поля различают 2 типа биполярных клеток: on-клетки увеличивают частоту импульсов при освещении центральной
части рецептивного поля и уменьшают ее при освещении периферии рецептивного
поля; off-клетки реагируют на
освещение прямо противоположным образом. Аналогичные популяции on- и off-клеток
существуют также среди амакриновых и ганглиозных клеток. В рамках задачи
считайте, что если на клетку одновременно поступают тормозящий и возбуждающий
сигналы, ее активность не изменяется.

\putImgWOCaption{10cm}{1}

Рассмотрите предложенную схему из 6 нейронов и
выберите все верные утверждения.

\begin{enumerate}
    \item При попадании света в точку В произойдёт возбуждение клетки 6
    \item Данная сеть нейронов на выходе выдаёт различный ответ при освещении точек Б и Г
    \item Данная сеть нейронов на выходе выдаёт различный ответ при освещении точек А и Д
    \item Нейромедиаторы амакриновых клеток, вероятнее всего, приводят к открытию натриевых каналов в мембране ганглиозных клеток
    \item Подобное устройство нейронных сетей в сетчатке обеспечивает повышение углового разрешения видимого изображения
\end{enumerate}



\answerMath{1, 3.}