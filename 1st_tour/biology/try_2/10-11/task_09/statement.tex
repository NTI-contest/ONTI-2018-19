\assignementTitle{}{9}{}

Исследователи работают с двумя линиями
мышей.  Первая линия страдает генетически
обусловленным гипертиреозом, вторая линия — контрольная. Исследователи в
эксперименте изучили концентрацию в крови 
тироксина и измерили массу щитовидной железы у этих двух линий.
Результаты приведены на рисунке.

\putTwoImg{6cm}{1}{6cm}{2}
%\putImgWOCaption{12cm}{1}

Изучив рисунок, выберите утверждения,
которые не противоречат результатам эксперимента.

\begin{enumerate}
    \item Масса железы у контрольной линии составляет около 2 мг 
    \item Все образцы из линии 1 демонстрируют более высокий уровень тироксина, чем в линии 2
    \item У образцов линии 1 с повышенным содержанием тироксина проявляются признаки гипертиреоза
    \item Вероятно, гипертиреоз у линии 1 сопровождается разрастанием железы
    \item Для контрольной линии не наблюдается четко выраженной связи между концентрацией гормона и массой железы
\end{enumerate}



\answerMath{3, 4, 5.}