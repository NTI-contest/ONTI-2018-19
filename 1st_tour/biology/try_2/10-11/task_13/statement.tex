\assignementTitle{}{8}

В медицине широко известен феномен отраженной боли:
раздражение внутренних органов может привести к болевым ощущениям на
поверхности тела, при этом отраженная боль может возникнуть на значительном
расстоянии от истинного источника боли. Так, при стенокардии — остром дефиците
коронарного кровообращения — часто возникает резкая боль в левом плече и на
внутренней поверхности левой руки. Для объяснения явления отраженной боли был
предложен ряд теорий, в частности теория конвергенции. Основной идеей теории
конвергенции является схождение сигналов от чувствительных нервных окончаний с
кожи и внутренних органов на одном и том же нейроне спинного мозга.

Выберите все утверждения об отраженной боли,
которые можно объяснить с позиций теории конвергенции.

\begin{enumerate}
    \item отражение боли возможно только в одном направлении (болевое воздействие на кожу не приводит к развитию болевых ощущений во внутреннем органе из того же дерматома)
    \item отражённая боль развивается одновременно с локальной
    \item отражённая боль распространяется в пределах одного дерматома (участка тела, иннервируемого одним сегментом спинного мозга)
    \item отражённая боль не градуирована (не зависит от силы повреждения внутреннего органа)
    \item болевые пороги локальной и отражённой боли совпадают
\end{enumerate}

\explanationSection

\answerMath{2, 3, 5.}