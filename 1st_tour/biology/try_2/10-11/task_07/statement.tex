\assignementTitle{}{6}{}

В молекулярной биологии
широко используют эндонуклеазы рестрикции, ферменты, которые умеют разрезать
двухцепочечную ДНК в определенном сайте. Сайт узнавания одной из рестриктаз
представляет собой последовательность AGATCT. Вы обрабатываете эндонуклеазой
рестрикции кольцевую ДНК одной бактерии. Известно, что содержание GC пар у
бактерии составляет $60\%$. 
Предполагая, что нуклеотдиды в геномной ДНК данной бактерии встречаются случайно, определите, какова ожидаемая длина фрагментов, полученных в
результате обработки? Ответ приведите в количестве пар нуклеотидов.

\solutionSection

Длина фрагмента обратно пропорциональна частоте встречаемости сайта. 
Если предположить, что нуклеотдиды встречаются случайно, то вероятность встретить 
сайт AGATCT $p(AGATCT)=p(A) \cdot p(G) \cdot p(A) \cdot p(T) \cdot p(C) \cdot p(T)= (0,2)^4 \cdot (0,3)^2$. 
Тогда средняя длина фрагмента $l=1/p(AGATCT) =  6944$ п.н.

\answerMath{6944.}