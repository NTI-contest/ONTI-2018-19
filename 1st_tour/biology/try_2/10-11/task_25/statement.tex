\assignementTitle{}{9}

Исследователи работают с двумя линиями
мышей.  Первая линия страдает генетически
обусловленным гипертиреозом, вторая линия — контрольная. Исследователи в
эксперименте изучили концентрацию в крови 
тироксина и измерили массу щитовидной железы у этих двух линий.
Результаты приведены на рисунке.

\putTwoImg{7cm}{1}{7cm}{2}
%\putImgWOCaption{12cm}{1}

Изучив рисунок, выберите утверждения,
которые не противоречат результатам эксперимента.

\begin{enumerate}
    \item Концентрация тироксина линии 2 составляет в среднем около 60 нМ 
    \item Все образцы из линии 1 демонстрируют более высокую массу железы, чем в линии 2
    \item Вероятная причина гипертиреоза в линии 2, может быть связана с повышенным уровнем ТТГ
    \item Гипертиреоз у линии 2 сопровождается с разрастанием железы
    \item Для линии 1 наблюдается связь между концентрацией гормона и массой железы
\end{enumerate}

\explanationSection

\answerMath{1, 3, 5.}