\assignementTitle{}{8}{}

Угловое разрешение глаза – это минимальный угол
между 2 точками, при котором человек воспринимает 2 точки отдельно друг от
друга. Известно, что угловое разрешение глаза обратно пропорционально диаметру
зрачка. Днем испытуемый A с нормальным зрением при диаметре зрачка в 3  мм
различает две отдельные точки на расстоянии 1.45  миллиметра друг от друга с 5 
метров. Зрение испытуемого B в темноте имеет угловое разрешение в 2  угловых
минуты.

Испытуемый А и испытуемый В находятся в темноте на
расстоянии 40  м от двух светящихся точек, расположенных рядом.  Во сколько раз большее расстояние между двумя
светящимися точками должно быть, чтобы пациент В увидел эти точки как 2
отдельных объекта, по сравнению с минимальным расстоянием между точками для
пациента A (в темное время диаметр зрачка 6 мм)? Ответ округлите до целых
чисел, размером глазного яблока можно пренебречь.

\answerMath{4.}