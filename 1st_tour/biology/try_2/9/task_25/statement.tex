\assignementTitle{}{10}{}

Пруд
какой площадью (м$^2$) необходимо выкопать, чтобы прокормить двух
цапель, при условии, что масса одной цапли – 1,5 кг ($65\%$ от массы составляет
вода), биомасса фитопланктона – 250г/м$^2$ . Помимо фитопланктона в
пруду будет жить зоопланктон и рыба. Переход биомасс между уровнями вырастает
на $5\%$ с каждым новым уровнем, при этом от первого уровня ко второму переходит
$10\%$ биомассы. Ответ округлите до целых, в ходе решения числа не округляйте.

\solutionSection

$1,5 \cdot 0,65=0,975$кг сухая масса цапли 1,95 – 2х цапель. 
Переходы: фитопланктона-зоопланктон $10\%$, зоопланктон-рыба $15\%$, рыба – цапля $20\%$. 

$1,95/0,2=9,75$ – у рыбы. $9,75/0,15 = 65$ – у зоопланктона, $65/0,1=650$ – у фитопланктона. 

Площадь = $650кг/0,25=2600$ м$^2$.

\answerMath{2600.}