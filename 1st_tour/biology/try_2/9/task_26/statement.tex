\assignementTitle{}{8}

Для
измерения скорости проведения нервного импульса был предложен особый метод
измерения составного потенциала действия мышцы (compound muscle action
potential, CMAP). К исследуемому
двигательному нерву прикладывается два возбуждающих электрода на заданном
расстоянии друг от друга (дистальный,
расположенный ближе к мышце, и проксимальный,
расположенный ближе к ЦНС). К иннервируемой мышце прикладывается два
регистрирующих электрода: активный
регистрирующий электрод (G1) располагается на брюшке мышцы поверх концевой
пластинки мотонейрона, электрод
сравнения (G2) – возле сухожилия (см.рисунок 1). При возбуждении нерва
дистальным и проксимальным электродами единственным изменяющимся параметром
регистрируемого CMAP является латентный период, т.е. время от возбуждения нерва
до возбуждения мышцы (см.рисунок 2).

\putImgWOCaption{13cm}{1}

Рисунок
1. Схема установки для измерения CMAP. $DE$ – дистальный электрод, $G1$ – активный
регистрирующий электрод, $G2$ – электрод сравнения, $L$ – известное расстояние
между 2 возбуждающими электродами, PE – проксимальный электрод.

\putImgWOCaption{13cm}{2}

Рисунок
2. Кривая CMAP. $DL$  – латентный период при возбуждении нерва дистальным электродом,
$PL$  – латентный период при возбуждении нерва проксимальным электродом.

Перед вами фрагмент таблицы из лабораторного
журнала физиолога XX века, который измерял скорость проведения импульса на
препарате седалищного нерва варана. Проведите расчеты и укажите среднюю
скорость проведения импульса в метрах в секунду с точностью до одного знака
после запятой, например: $10.7$.  

\putImgWOCaption{7cm}{3}

\answerMath{39.7.}