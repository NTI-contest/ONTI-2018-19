\assignementTitle{}{10}{}

При
строительстве коттеджного посёлка застройщик уничтожил часть пашни, из-за чего
пострадало семейство лисиц, которым стало не хватать оставшейся пашни для того,
чтобы прокормиться. Пашня служит экологической нишей для следующих организмов:
лиса, мышь, ястреб, пшеница. Составьте из перечисленных организмов пищевую цепь (на каждом уровне только один вид организмов) и вычислите, на сколько гектаров нужно увеличить пашню, чтобы она снова могла
прокормить всех лис? Биомасса пшеницы за лето составляет 3 т/га, семейство лисиц
состоит из 13  особей. Чтобы выжить, одной лисице необходимо получать 2.5 кг
биомассы в месяц. Переход биомассы между уровнями составляет $10\%$, а площадь
пашни сейчас 1.5 га? Ответ округлите до сотых.

\solutionSection

$2,5 \cdot 13=32,5$ кг в месяц. У мышей: $32,5/0,1 = 325$кг, на уровне пшеницы: $325/0,1 = 3250$ кг в месяц. 
Пшеница за месяц: $3000/3=1000$кг имеем.\\ $3250-1500 = 1750$кг ещё надо. $1750кг/1000=1,75$ га.

\answerMath{1.75.}