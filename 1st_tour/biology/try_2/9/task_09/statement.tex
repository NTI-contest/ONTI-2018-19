\assignementTitle{}{9}{}

Составьте
пищевую цепь из следующих организмов: мелкая рыба, акула, фитопланктон, тюлень,
зоопланктон, и решите задачу:

Тюлени
получают 32000 ккал в месяц от организмов предыдущего трофического уровня.
Сколько ккал энергии продуценты могут тратить на собственные нужны в год, если
перенос энергии между уровнями одинаковый и подчиняется правилу Линдемана.
Ответ округлите до целых.

\solutionSection

Цепь фитопланктон $\rightarrow$ зоопланктон $\rightarrow$ рыба $\rightarrow$ тюлень $\rightarrow$ акула

На уровне фитопланктона в месяц $32000/0.1/0.1/0.1=32000000$ ккал. 
На себя в год: $32000000 \cdot 0.9 \cdot 12=345 600 000$ ккал

\answerMath{345600000.}