\assignementTitle{}{8}{}

Электрокардиограмма - метод
регистрации и исследования электрических полей, образующихся при работе сердца.
Это простой, но очень ценный метод, позволяющий изучать работу сердца. На рис.1
вы видите пики и интервалы ЭКГ, образующиеся при нормальной работе сердца. По
ЭКГ определяют частоту сердечных сокращений, положение сердца в грудной клетке,
нарушение в электрическом возбуждении сердца.

\putImgWOCaption{8cm}{1}

Рисунок 1. Стандартные зубцы и интервалы ЭКГ здорового человека.

В современной
медицине все чаще используются портативные кардиографы, которые способны
записывать ЭКГ круглосуточно. При каком расположении двух электродов на теле
человека возможна запись ЭКГ?

\begin{enumerate}
    \item Оба на правом предплечье
    \item Оба на левом предплечье
    \item Оба на правом бедре
    \item Один на груди, другой на животе
    \item Оба на левом бедре
\end{enumerate}

\answerMath{4.}