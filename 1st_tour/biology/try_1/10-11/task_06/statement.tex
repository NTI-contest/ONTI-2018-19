\assignementTitle{}{12}{}

В 1958 году Мезельсон, Сталь применили метод
центрифугирование в градиенте плотности для разделения молекул ДНК, содержащих
изотопы азота $^{14}$N и $^{15}$N. Мезельсон и Сталь выращивали
несколько поколений бактерий Escherichia
coli в среде, богатой $^{15}$N или $^{14}$N, а затем
центрифугировали их ДНК в градиенте плотности хлористого цезия. При этом более
тяжёлая, содержащая $^{15}$N, ДНК занимает положение ближе ко дну
центрифужной пробирки, а легкая $^{14}$N-ДНК – более высокое. 

Клетки E. coli,
росшие несколько поколений на содержащей $^{15}$N среде были перенесены
на $^{14}$N-содержащую среду. После первого и второго деления
исследователи выделяли ДНК из клеток и центрифугировали ее в градиенте плотности.
Результаты эксперимента можно увидеть на рисунке:

\putImgWOCaption{14cm}{1}

Верными утверждениями являются:

\begin{enumerate}
    \item Разделение молекул ДНК происходит по молекулярной массе
    \item Плотность раствора CsCl нарастает к дну центрифужной пробирки
    \item После первого деления, в одной цепи ДНК – только $^{15}$N, а в другой – только $^{14}$N
    \item Эксперимент опровергает полуконсервативный механизм репликации ДНК
    \item Несколько поколений на среде с $^{15}$N необходимы для вытеснения $^{14}$N из молекул ДНК
\end{enumerate}

\explanationSection

Данный эксперимент ПОДТВЕРЖДАЕТ полуконсервативный механизм репликации ДНК.

\answerMath{1, 2, 3, 5.}