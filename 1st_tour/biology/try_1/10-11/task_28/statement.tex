\assignementTitle{}{8}{}

На рисунке показан механизм
всасывания глюкозы из просвета кишечника в кровь. За счет работы мембранного
насоса,  Na+/K+ -
АТФ-азы (1), концентрация Na+ в эпителиоцитах ниже, чем концентрация
Na+  в крови и просвете кишки.
Это позволяет транспортировать глюкозу в эпителиоцит против ее
концентрационного градиента, сопрягая транспорт глюкозы с транспортом ионов
натрия по градиенту (3). Из эпителиоцита в кровь глюкоза поступает по градиенту
(2). Концентрация ионов калия, наоборот, выше в эпителиоцитах, чем в крови.
 
\putImgWOCaption{10cm}{1}

Выберите верные утверждения:

\begin{enumerate}
    \item Для всасывания глюкозы из кишки в кровь нужна энергия
    \item Использование ингибитора Na+/K+ - АТФ-азы увеличит эффективность поглощения глюкозы
    \item Калий выходит из клетки по градиенту его концентрации. Для этого не нужна энергия АТФ
    \item Мутация в гене белка-переносчика глюкозы (2) нарушит поглощение глюкозы
    \item Глюкоза может проходить через мембрану сама по себе, для этого не требуется специальный переносчик
\end{enumerate}

\answerMath{1, 3, 4.}