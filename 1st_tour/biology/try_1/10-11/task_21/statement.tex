\assignementTitle{}{12}{}

В 1952г. Херши и Чейз провели
эксперимент с  бактериофагом T2.
Бактериофаг Т2 состоит из белковой оболочки, внутри которой находится ДНК.
Херши и Чейз выращивали бактерий на среде, содержащей радиоактивный фосфор-32 и
на среде с радиоактивной серой-35. При добавлении бактериофагов к этим
бактериям, бактериофаги включали эти радиоактивные изотопы в состав нуклеиновых
кислот и белков. 

Радиоактивно-мечеными
бактериофагами инфицировали бактерии, свободные от радиоактивных изотопов,
после чего отмывали (при этом вирусные частицы оказывались в смыве). Когда к
бактериям добавлялись меченые фосфором-32 бактериофаги, радиоактивная метка
после инфицирования находилась в бактериальных клетках. Если же к бактериям
добавлялись бактериофаги, меченые серой-35, то метка обнаруживалась только в
смывах, но не в бактериальных клетках.

\putImgWOCaption{10cm}{1}

Верными утверждениями являются:

\begin{enumerate}
    \item Радиоактивный фосфор в эксперименте включался в состав белка
    \item Наличие в смывах метки возможно при инфицировании $^{32}$P меченными бактериофагами
    \item Из эксперимента следует, что именно ДНК проникает внутрь клетки
    \item Эксперимент опровергает гипотезу о роли белка в передаче генетической информации
    \item Бактериофаги меченные радиоактивными S и P не отличаются по способности инфицировать бактериальные клетки
\end{enumerate}

\explanationSection

Радиоактивный фосфор включается в состав нуклеиновых кислот (ДНК, РНК), радиоактивная сера входит 
в состав белков (аминокислот цистеина и метионина).

\answerMath{3, 4, 5.}