\assignementTitle{}{6}

Бегун бежит короткую
дистанцию. При таком режиме бега энергетические запасы мышц быстро истощаются,
а кислорода не хватает для полного окисления глюкозы. Известно, что при забеге,
в его мышцах распадается 5.4 г глюкозы. Молярная масса глюкозы 180 г/моль.

Предположив, что эта энергия получена исключительно в ходе гликолиза, рассчитайте количество образовавшегося в ходе забега АТФ. Ответ приведите в
молях, округлив до сотых.

\soultionSection

$5,4 \text{г} / 180 \text{г} / \text{моль} = 0,03$ моль. В ходе гликолиза из одного моль глюкозы 
образуется два моль АТФ. Следовательно, в ходе забега образовалось вдвое больше АТФ – 0,06 моль.

\answerMath{0.06}