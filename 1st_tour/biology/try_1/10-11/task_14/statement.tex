\assignementTitle{}{8}{}

Человек
получает значительную часть информации об окружающем мире с помощью зрения. На
рисунке А показан механизм восприятия света светочувствительными клетками
человека. При попадании квантов света опсин стимулирует обмен ГДФ на ГТФ в
трансдуцине, тем самым активируя этот белок. Активный трансдуцин активирует
фосфодиэстеразу, которая разрушает циклический гуанозинмонофосфат. Это приводит
к закрытию катионных каналов, в темноте обеспечивающих поступление ионов
кальция и натрия внутрь фоточувствительной клетки.

На рисунке Б показан принцип измерения мембранного потенциала (разности
потенциалов на мембране). Для большинства возбудимых клеток свойственно
отрицательное значение потенциала покоя. Показаны также типы изменения
мембранного потенциала. Например, поступление катионов внутрь клетки приводит к
уменьшению мембранного потенциала.

\putImgWOCaption{14cm}{1}

Внимательно рассмотрите
рисунок и выберите верные утверждения:

\begin{enumerate}
    \item Поступление в клетку ионов Ca++ приводит к деполяризации
    \item Деполяризация фоточувствительной клетки случается при попадании квантов света на опсин
    \item Свет инактивирует фосфодиэстеразу
    \item Ингибиторы фосфодиэстеразы приводят к гиперполяризации фоточувствительной клетки
\end{enumerate}

\answerMath{1, 4.}