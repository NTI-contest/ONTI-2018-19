\assignementTitle{}{6}{}

Дрожжи попали в 0.8 л бутылку с виноградным соком с
содержанием сахара $20\%$ . После некоторого времени содержание сахара упало до
$15\%$ . Какое количество АТФ образовалось в клетках дрожжей, если бутылка была
плотно закупорена, а все сахара в соке представлены глюкозой?  Молярная масса глюкозы $180$  г/моль. Ответ
приведите в молях, округлив до сотых

\solutionSection

0,8 кг $\cdot$ 0,05 = 0,04 кг = 40 г; 40 г/180 г/моль = 0,22 моль. В ходе гликолиза из одного моль глюкозы образуется два моль АТФ. Следовательно, в ходе забега образовалось вдвое больше АТФ – 0,44 моль.

\answerMath{0.44.}