\assignementTitle{}{8}

Электромиография – метод исследования электрической активности мышц. 
Обычно для получения электромиограммы (ЭМГ) в исследуемую мышцу пациента вводят тонкую иглу, 
содержащую в себе активный регистрирующий электрод, а на коже над мышцей закрепляют электрод сравнения. 
Сама ЭМГ представляет собой график зависимости электрического потенциала мышцы от времени.

Рассмотрите ЭМГ шести пациентов и определите состояние изучаемых мышц в момент исследования: 
сопоставьте пациентов и состояния их мышц.

Пациент 1

\putImgWOCaption{13cm}{1}

\newpage
Пациент 2

\putImgWOCaption{13cm}{2}

Пациент 3

\putImgWOCaption{13cm}{3}

Пациент 4

\putImgWOCaption{13cm}{4}

Пациент 5

\putImgWOCaption{13cm}{5}

\newpage
Пациент 6

\putImgWOCaption{13cm}{6}

Выберите верные утверждения:

        \begin{enumerate}
            \item Пациент 1 
            \item Пациент 2
            \item Пациент 3
            \item Пациент 4
            \item Пациент 5
            \item Пациент 6
        \end{enumerate}

        \begin{enumerate}
            \item[а.] Судорога (регулярные разряды определенных моторных единиц с низкой амплитудой и частотой 20-150 Гц)
            \item[б.] Миокимия (группы разрядов определенных моторных единиц с высокой амплитудой и частотой 5-60 Гц)
            \item[в.] Нейромиотония (регулярные разряды определенных моторных единиц со снижающейся амплитудой и частотой 150-250 Гц)
            \item[г.] Тремор покоя (нерегулярные группы разрядов переменной амплитуды продолжительностью 0.2-0.4 с)
            \item[д.] Комплексные повторяющиеся разряды (регулярные группы разрядов переменной амплитуды продолжительностью около 0.1 с)
            \item[е.] Покой
        \end{enumerate}

\answerMath{1 - в, 2 - б, 3 - а, 4 - е, 5 - д, 6 - г.}