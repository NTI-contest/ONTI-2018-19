\assignementTitle{}{8}{}

Человек
получает значительную часть информации об окружающем мире с помощью зрения. На
рисунке А показан механизм восприятия света светочувствительными клетками
человека. При попадании квантов света опсин стимулирует обмен ГДФ на ГТФ в
трансдуцине, тем самым активируя этот белок. Активный трансдуцин активирует
фосфодиэстеразу, которая разрушает циклический гуанозинмонофосфат. Это приводит
к закрытию катионных каналов, в темноте обеспечивающих поступление ионов
кальция и натрия внутрь фоточувствительной клетки.

На рисунке Б показан принцип измерения мембранного потенциала (разности
потенциалов на мембране). Для большинства возбудимых клеток свойственно
отрицательное значение потенциала покоя. Показаны также типы изменения
мембранного потенциала. Например, поступление катионов внутрь клетки приводит к
уменьшению мембранного потенциала.

\putImgWOCaption{14cm}{1}

Внимательно рассмотрите
рисунок и выберите верные утверждения:

\begin{enumerate}
    \item При попадании квантов света на опсин происходит гиперполяризация фоточувствительной клетки
    \item При попадании квантов света на опсин происходит деполяризация фоточувствительной клетки
    \item Потенциал покоя может быть измерен, если оба электрода на рисунке Б будут расположены вне клетки
    \item Мутация в гене трансдуцина может быть причиной врожденной слепоты
\end{enumerate}

\answerMath{1, 4.}