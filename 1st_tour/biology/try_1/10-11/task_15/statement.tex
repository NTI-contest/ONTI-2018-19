\assignementTitle{}{8}

Рассмотрите схему,
иллюстрирующую последовательность событий, приводящих к сокращению
кардиомиоцита. (1-потенциал действия активирует Са++канал, и Са++ входит в
клетку (2), 3-4 — Са++ выходит из саркоплазматического ретикулюма, 5 -
сокращение, 6 - расслабление, 7-8 — Са++ выходит из цитоплазмы. Известно, что кальций необходим для
запуска мышечного сокращения.

Выберите верные суждения.

\putImgWOCaption{14cm}{1}

\begin{enumerate}
    \item В расслабленной мышце концентрация кальция в цитоплазме ниже, чем при сокращении
    \item Вещество, блокирующее Са-каналы, приводит к нарушению способности кардиомиоци
    \item Кальций поступает в клетку против градиента концентрации
    \item Белок, обозначенный цифрой 9, является АТФазой
\end{enumerate}

\answerMath{1, 2, 4.}