\assignementTitle{}{8}{}

На рисунке показан механизм
всасывания глюкозы из просвета кишечника в кровь. За счет работы мембранного
насоса,  Na+/K+ -
АТФ-азы (1), концентрация Na+ в эпителиоцитах ниже, чем концентрация
Na+  в крови и просвете кишки.
Это позволяет транспортировать глюкозу в эпителиоцит против ее
концентрационного градиента, сопрягая транспорт глюкозы с транспортом ионов
натрия по градиенту (3). Из эпителиоцита в кровь глюкоза поступает по градиенту
(2). Концентрация ионов калия, наоборот, выше в эпителиоцитах, чем в крови.
 
\putImgWOCaption{10cm}{1}

Выберите верные утверждения:

\begin{enumerate}
    \item Для всасывания глюкозы из кишки необходима работа Na+/K+ - АТФ-азы
    \item В норме концентрация глюкозы в эпителиоците выше, чем в крови
    \item Натрий выходит из клетки по градиенту его концентрации. Для этого не нужна энергия АТФ
    \item Калий может проходить через мембрану сам по себе, для этого не требуется специальный канал
\end{enumerate}

\answerMath{1, 2.}