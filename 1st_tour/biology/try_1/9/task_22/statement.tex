\assignementTitle{}{5}

Выберите все типы активного транспорта из приведённого списка.

\begin{enumerate}
    \item конвекция
    \item осмос
    \item облегченная диффузия
    \item натрий-калиевый насос
    \item экзоцитоз
    \item эндоцитоз
    \item адвекция
   
\end{enumerate}

\explanationSection

Конвекция - вид теплообмена, возникающий самопроизвольно, при ней энергия передаётся струями и потоками, например, возникает в веществе при его неравномерном нагревании. Натрий-калиевый насос - активный транспорт против градиента концентрации. Адвекция - перемещение воздуха в горизонтальном направлении, пример пассивного транспорта. Эндоцитоз - захват клеткой веществ извне путём образования мембранных везикул. Экзоцитоз - процесс транспорта внутриклеточных веществ наружу, при этом внутренняя везикула сливается с мембраной клетки и содержимое везикулы выделяется наружу. Осмос - движение жидкости в сторону большей концентрации. Облегчённая диффузия - вещество переносится через мембрану по градиенту концентрации. 

\answerMath{4, 5, 6.}