\assignementTitle{}{8}

Гомеостаз – способность организма поддерживать постоянство
внутренней среды. Различные параметры (температура, концентрация веществ, рН)
поддерживаются при помощи разных механизмов. Сопоставьте механизмы регуляции и
уровни организации живого, на котором происходит регуляция данного параметра.

\begin{enumerate}
    \item Выделение слизи в дыхательных путях
    \item Увеличение биосинтеза актина и миозина в ответ на нагрузки
    \item Выделение адреналина при опасности
    \item Белковая буферная система крови
    \item Выделение кислоты желудком при попадании в него пищи
\end{enumerate}
    
\begin{enumerate}
    \item[а.] Молекулярный
    \item[б.] Организменный
    \item[в.] Органный
    \item[г.] Тканевой
    \item[д.] Клеточный
\end{enumerate}

\explanationSection

\begin{enumerate}
   \item Белковая буферная система крови - Молекулярный (Происходит присоединение или отсоединение протона на молекулярном уровне)
   \item Выделение адреналина при опасности - Организменный (Задействованы сразу несколько систем органов, поэтому уровень организменный)
   \item Увеличение биосинтеза актина и миозина в ответ на нагрузки - Клеточный (Регуляция экспрессии генов происходит на уровне клетки)  
   \item Выделение слизи в дыхательных путях - Тканевой (В данном процессе участвуют сразу несколько клеток нервной ткани - тканевой уровень)
   \item Выделение кислоты желудком при попадании в него пищи - Органный(Данный процесс регулируется паракринными сигналами на уровне желудка, при этом задействован весь орган)
\end{enumerate}

\answerMath{1 - г, 2 - д, 3 - б, 4 - а, 5 - в.}