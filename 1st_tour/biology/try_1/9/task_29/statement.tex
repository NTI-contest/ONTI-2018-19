\assignementTitle{}{8}{}

Гомеостаз – способность организма поддерживать постоянство внутренней среды. 
Различные параметры (температура, концентрация веществ, рН) поддерживаются при 
помощи разных механизмов. Сопоставьте механизмы регуляции и уровни организации живого, 
на котором происходит регуляция данного параметра.

\begin{enumerate}
    \item Увеличение биосинтеза соматотропного гормона
    \item Выделение инсулина при потреблении пищи
    \item Бикарбонатная буферная система крови
    \item Обратный захват нейромедиаторов нервными и глиальными клетками
    \item Выделение кислоты желудком при попадании в него пищи
\end{enumerate}
    
\begin{enumerate}
    \item[а.] Молекулярный
    \item[б.] Органный
    \item[в.] Организменный
    \item[г.] Тканевой
    \item[д.] Клеточный
\end{enumerate}

\answerMath{1 - д, 2 - в, 3 - а, 4 - г, 5 - б.}