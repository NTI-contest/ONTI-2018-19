\assignementTitle{}{5}{}

Какой тип бактерий не способен использовать кислород для производства энергии?

\begin{enumerate}
    \item аэротолерантные анаэробы
    \item хемогетеротрофы
    \item факультативные анаэробы
    \item облигатные аэробы
\end{enumerate}

\explanationSection

Хемогетеротрофы используют химическую энергию, освобождающуюся в ходе окисления органических веществ, получают энергию четырьмя способами: аэробным дыханием, неполным окислением, брожением и анаэробным дыханием. У факультативных анаэробов энергетические циклы проходят по анаэробному пути, но способны существовать при доступе кислорода. Облигатные аэробы нуждаются в кислороде для дыхания и не могут жить в его отсутствие. Аэротолерантные анаэробы - организмы, которые не погибают в присутствии кислорода. Но не способные переключиться на аэробный тип дыхания.  

\answerMath{1.}