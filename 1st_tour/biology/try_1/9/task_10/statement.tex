\assignementTitle{}{10}{}

На поле в 500 га за лето созревает 450 тонн зерна. В тоже время с гектара этого поля полёвки получают 4.5 центнера 
собственной массы. Сколько хищных птиц прокармливается с данного поля, если одной птице для выживания необходимо получать 15 кг биомассы в месяц. Перенос биомассы между уровнями одинаковый. Учитывайте, что полёвка на $70\%$ состоит из воды. Содержанием воды в зерне пренебречь. Ответ округлите до целых, в ходе решения числа не округляйте.

\solutionSection

$1500/500 = 3$ тонны с гектара. $0,45/3 =15\%$ - переход между уровнями. \linebreak $0,45 \cdot 500 \text{га}/3=75$тонн – масса 
полёвок в месяц. $75 \cdot 30\%= 22,5$тонн – биомасса. $22,5 \cdot 15\% =3,375$т – переходит к птицам. 
$3,375/0,015=225$ птиц

\answerMath{225.}