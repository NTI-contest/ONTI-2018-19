\assignementTitle{}{6}{}

Благодаря каким из нижеперечисленных механизмов бактерия может приобрести
устойчивость к антибиотикам?

\begin{enumerate}
    \item конъюгация
    \item репродукция
    \item трансформация
    \item трансдукция
    \item мутация
    \item устранение
\end{enumerate}

\explanationSection

При конъюгации происходит обмен генетическим материалом между двумя контактирующими бактериями, соответственно, одна может передать другой устойчивость к антибиотику. С помощью трансформации можно в лабораторных условиях внести в геном бактерии  чужеродный фрагмент, в том числе сделать бактерию резистентной. С помощью трансдукции посредством бактериофага можно передать генетический материал одной бактериальной клетки другой клетке. Мутация - один из основных природных механизмов приобретения устойчивости к антибиотикам. Репродукция и устранение не предполагают обмен генетическим материалом между разными клетками.

\answerMath{1, 3, 4, 5.}