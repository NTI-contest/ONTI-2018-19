\assignementTitle{}{8}{}

Электромиография – метод исследования электрической активности мышц. Обычно для получения
электромиограммы (ЭМГ) в исследуемую мышцу пациента вводят тонкую иглу, содержащую
в себе активный регистрирующий электрод, а на коже над мышцей закрепляют
электрод сравнения. Сама ЭМГ представляет собой график зависимости
электрического потенциала мышцы от времени. Рассмотрите ЭМГ шести
пациентов и определите состояние изучаемых мышц в момент исследования: сопоставьте пациентов и состояния их мышц. 

Пациент 1

\putImgWOCaption{13cm}{1}

Пациент 2

\putImgWOCaption{13cm}{2}

\newpage
Пациент 3

\putImgWOCaption{13cm}{3}

Пациент 4

\putImgWOCaption{13cm}{4}

Пациент 5

\putImgWOCaption{13cm}{5}

Пациент 6

\putImgWOCaption{13cm}{6}

Выберите верные утверждения:

\begin{enumerate}
    \item Самая высокая амплитуда наблюдается у пациента 6
    \item Нерегулярные группы разрядов переменной амплитуды продолжительностью 0.2-0.4 с можно наблюдать у пациента 5
    \item Мышца пациента 4 находится в покое
    \item Мышца пациента 3 судорожно сокращается (с частотой около 50 Гц происходят регулярные разряды моторных единиц)
\end{enumerate}

\answerMath{3, 4.}