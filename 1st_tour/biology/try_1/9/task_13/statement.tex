\assignementTitle{}{8}{}

В списке указан расход энергии мышцами при различных
видах физической активности в формате: "Вид физической активности — Расход энергии, ккал/час"

\begin{itemize}
    \item Сидение — 100
    \item Ходьба — 200
    \item Езда на велосипеде (9 км/ч) — 300
    \item Уборка снега лопатой — 480
    \item Бег (9 км/ч) — 600
    \item Гребля — 830
\end{itemize}
 
Определите, какое количество АТФ
(в килограммах) израсходовал Иван за день, если известно что:

\begin{itemize}
    \item На работу Иван не спеша ездит на велосипеде, дорога в каждую сторону занимает 20 минут.
    \item В рабочее время (8 часов в день) Иван сидит.
    \item Иван ведет здоровый образ жизни, поэтому, вернувшись с работы, он выходит побегать 45 минут (со скоростью 9 км/ч).
\end{itemize}

Считайте, что энергия гидролиза
АТФ до АДФ равна 14.5 кал/г. Ответ округлите до целых.

\solutionSection

Для начала необходимо определить сколько энергии расходуется на каждом 
участке пути (при этом учесть, что необходимо учитывать дорогу туда и 
обратно). Так, например, при езде на велосипеде затрачивается 300 ккал/час 
$\cdot$ (20 мин + 20 мин) =\\ = 300 ккал/час $\cdot$ 2/3 час = 200 ккал. Аналогично рассчитваются другие участки пути. В итоге получается:
 $(200 + 800 + 450) / 14,5 = 100.$

\answerMath{100}