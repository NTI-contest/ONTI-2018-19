\assignementTitle{}{9}{}

Составьте пищевую цепь из следующих организмов и решите задачу.

Даны следующие организмы: лягушки, мухи, манго, стрекозы. Продуценты выделяют 3500000 кДж энергии в день. 
Сколько энергии передадут стрекозы консументам следующего порядка за месяц (июнь), 
если передача энергии осуществляется по правилу Линдемана. Ответ округлите до целых.

\solutionSection

манго $\rightarrow$ муха $\rightarrow$ стрекоза $\rightarrow$ лягушка

$3500000 \cdot 0,1 \cdot 0,1  \cdot 0,1 \cdot 30 = 105000$кДж

\answerMath{105000.}