\assignementTitle{}{10}{}

В засушливое лето пересохла часть пруда, что привело к частичной гибели
рыбы. Сколько кг рыбы нужно выпустить в пруд, чтобы её хватило для того, чтобы
прокормить двух выдр, если сейчас выдры получают 450 ккал, а чтобы прокормиться
одна выдра должна получать 700 ккал. В 1кг биомассы рыбы запасается 650 ккал
энергии. Переход энергии между уровнями $14\%$. 
Вес 1 рыбы составляет 1,5 кг ($70\%$ воды). Ответ округлите до целых.

\solutionSection

$2 \cdot 700=1400$ нужно выдрам. Для этого у рыб должно быть $1400/0,14=10000$ ккал. Имеем $450/0,14=3214$. $10000-3214=6786$ ккал не хватает. \linebreak $6786/650 = 10,44$кг биомассы надо, 
$10,44/0,3=34,8=35$ кг.

\answerMath{35.}