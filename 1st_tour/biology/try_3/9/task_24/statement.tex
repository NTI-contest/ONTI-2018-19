\assignementTitle{}{9}

Сколько рыбы съел тюлень за февраль месяц, если за каждую неделю его
биомасса увеличивалась на 1,44кг.  Масса
одной рыбы составляет 2кг, при этом на $70\%$ рыба состоит из воды, а переход
биомассы между уровнями $12\%$. Ответ округлите до целых.

\solutionSection

$1,44/0,12=12$кг – сухой рыбы $12 \cdot 100/30=40$кг всего рыбы $40/2=20$ рыб в неделю $20 \cdot 4 = 80$ в месяц.

\answerMath{80.}