\assignementTitle{}{8}

Электрокардиограмма
— метод регистрации и исследования электрических полей, образующихся при работе
сердца. Это простой, но очень ценный метод, позволяющий изучать работу сердца.
По ЭКГ определяют частоту сердечных сокращений, положение сердца в грудной
клетке, нарушение в электрическом возбуждении сердца.

\putImgWOCaption{15cm}{1}

ЧСС как правило определяют,
рассчитывая расстояние между двумя соседними «R» пиками. Определите частоту
сердечных сокращений (уд/мин) на представленной записи ЭКГ, если скорость
движения ленты была 25 мм/сек. Ответ округлите до целых значений.

\explanationSection

\answerMath{60.}