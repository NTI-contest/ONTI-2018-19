\assignementTitle{}{8}

Один из методов
диагностики состояния легких — измерение скорости потока вдыхаемого и
выдыхаемого воздуха с последующим построением кривой зависимости скорости
потока от объема легких. Нормальная кривая изображена на рисунке зеленым
цветом. Изучите рисунок и выберите все верные утверждения.

\putImgWOCaption{8cm}{1}

Выберите верные утверждения:

\begin{itemize}
    \item Синяя кривая характерна для пациентов с фиброзом легких (разрастанием соединительной ткани)
    \item Графики ниже оси Х отражают процесс выдоха
    \item Желтая кривая характерна для пациентов с эмфиземой легких (патологическим расширением воздушных пространств мелких бронхиол)
    \item В норме при вдохе объем легких увеличивается примерно на 4.2 литра
\end{itemize}

\explanationSection

\answerMath{2, 4.}