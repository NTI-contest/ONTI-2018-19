\assignementTitle{}{10}{}

В 1958 г. Дж. Тейлор проращивал проростки бобов на среде содержащей 3H-меченный тимидинтрифосфат, метка включалась в делящиеся клетки и обнаруживалась в хроматидах хромосом в корешках проростка. После чего корешки помещались на среду без метки, клетки корешка, продолжающие активно делится, включали немеченый тимидин. Меченые хромосомы анализировали методом радиоавтографии: на препарат наслаивали тонкий слой эмульсии, который засвечивался радиоактивным излучением от метки, при этом из эмульсии выпадали зерна серебра, отмечая треки радиоактивных частиц. Препараты анализировали после первого и второго митотических делений во время метафазы.

Верными утверждениями являются:

\begin{enumerate}
    \item После первого деления метка присутствует во всех хроматидах метафазных хромосом
    \item После первого деления метка присутствует только в одной из хроматид метафазной хромосомы 
    \item После первого деления метка обнаруживается только в одной из гомологичных хромосом
    \item После второго деления метка присутствует только в одной из хроматид метафазной хромосомы
    \item После второго деления метка присутствует только в одной из гомологичных хромосом
\end{enumerate}



\answerMath{1, 4.}