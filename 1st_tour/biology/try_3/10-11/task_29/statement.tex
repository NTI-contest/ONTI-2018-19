\assignementTitle{}{8}

Мышечное
сокращение регулируется, в том числе, и на уровне спинного мозга. На рисунке
показана схема рефлекса, запускаемого сухожильным органом Гольджи. Частота
нервных импульсов, идущих от сухожильного органа Гольджи в спинной мозг,
увеличивается при растяжении органа Гольджи. Тормозный нейрон выделяет
тормозящий нейромедиатор в синаптическую щель, а возбуждающий нейрон –
возбуждающий медиатор. Мотонейрон стимулирует сокращение соответствующей мышцы.

\putImgWOCaption{10cm}{1}

Выберите
верные утверждения:

\begin{enumerate}
    \item Согласно данной схеме, растяжение сухожильного органа мышцы 1 стимулирует сокращение мышцы 2.
    \item Данный рефлекс защищает мышцы от чрезмерного напряжения.
    \item Разрушение тормозного нейрона не повлияет на возможность протекания данного рефлекса.
    \item Скорее всего, у мышцы 2 в сухожилии также есть сухожильный орган Гольджи.
    \item Данный рефлекс реализуется на уровне спинного мозга и не задействует головной мозг.
\end{enumerate}



\answerMath{1, 2, 4, 5.}