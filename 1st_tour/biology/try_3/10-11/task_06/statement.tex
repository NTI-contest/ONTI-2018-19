\assignementTitle{}{8}

При
проведении ПЦР аллели гена D дают
продукты амплификации длиной около 500 п.о. Аллель d отличается от аллеля D,
тем что у него есть участок, который узнает эндонуклеаза рестрикции. При
обработке эндонуклеазой продукт амплификации расщепляется на две части (длиной
200 и 300 п. о.). Исследователь имеет результаты генотипирования по гену D для двух групп (страдающих неким
заболеванием и здоровых). На рисунках ниже схематично представлены результаты
анализа продуктов ПЦР после рестрикции в агарозном геле 

Результат генотипирования среди заболевших:

\putImgWOCaption{15cm}{1}

Результат генотипирования в контрольной
группе:

\putImgWOCaption{15cm}{2}

Изучив результаты эксперимента, выберите
утверждения, которые следуют из полученных результатов.

\begin{enumerate}
    \item Пациент 4 – гетерозиготен по гену D
    \item Гетерозиготы наиболее устойчивы к заболеванию
    \item Все носители генотипа DD больны
    \item Гомозиготы DD обнаружены только среди больных
    \item Частота аллеля D  в контрольной группе равна 0,1
\end{enumerate}



\answerMath{1, 4, 5.}