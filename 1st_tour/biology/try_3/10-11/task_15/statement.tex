\assignementTitle{}{8}{}

Декомпрессионная
болезнь (ДКБ) - заболевание, возникающее вследствие быстрого подъёма
аквалангистов с глубины на поверхность и связанного с этим падения давления
вдыхаемой газовой смеси. При ДКБ газы, растворённые в крови, выделяются из
раствора в виде пузырьков, что приводит к закупориванию мелких сосудов.

Шотландский
физиолог Джон Скотт Холдейн вывел формулу, описывающую процесс выведения из
организма избытка инертного газа (азота) после погружения:

$T_{N2}= T_0 + (T_f - T_0)(1-(0.5)^{t/t_0})$ ,
где $T_{N2}$  - парциальное давление азота в крови [бар], $T_0$ -
исходное парциальное давление на момент подъёма, $T_f$  - равновесное
парциальное давление на данной глубине, $t$  - время пребывания на данной глубине
[мин], $t_0$  - срок полувыведения. Также Д.С. Холдейн создал модель
декомпрессии (процесса подъёма на поверхность), согласно которой подъём с
больших глубин должен осуществляться ступенчато, чтобы не позволить отношению
парциальных давлений азота в крови и во внешней среде превысить 2:1.

Аквалангист
погрузился на глубину в 30 метров на длительное время. Сколько времени должен
занять безопасный подъём, если:

\begin{enumerate}
    \item аквалангист при подъёме совершает 3 остановки по 20 или 40 мин на глубине 20, 10 и 5 метров; 
    \item $t_0$ составляет 20 минут;
    \item подъём с предыдущей остановки до следующей осуществляется по окончании тех 20 минут, в течение которых парциальное давление азота в крови снизится до
отметки, не более чем вдвое превосходящей внешнее давление на новой остановке;
    \item атмосферное давление на уровне воды равно 1 бар, спуск на каждые 10 метров под водой
соответствует увеличению внешнего давления на 1 бар;
    \item содержание азота в газовой смеси и в атмосфере примите равным $70\%$ (на заре
дайвинга состав газовых смесей в баллонах практически не отличался от состава
воздуха);
    \item временем перемещения между остановками можно пренебречь.
\end{enumerate}



\answerMath{80.}