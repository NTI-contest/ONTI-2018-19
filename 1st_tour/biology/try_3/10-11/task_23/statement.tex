\assignementTitle{}{10}

В 1963 г. Дж. Кэрнс выращивал бактерии E. coli на среде содержащей 3H-меченный
тимидинтрифосфат, метка включалась в делящиеся клетки и обнаруживалась в
кольцевых хромосомах E. coli.
Меченные хромосомы анализировались методом радиоавтографии: на препарат
наслаивали тонкий слой эмульсии, который засвечивался радиоактивным излучением
от метки, при этом из эмульсии выпадали зерна серебра, отмечая треки частиц.
Кэрнс обнаружил q-структуры, подобные приведенной на рисунке.

\putImgWOCaption{5cm}{1}

Верными утверждениями являются:

\begin{enumerate}
    \item ${}^{3}H$-меченный тимидинтрифосфат включается в синтезируемые цепи ДНК
    \item На участке А репликация бактериальной хромосомы уже прошла
    \item Структура А содержит одну репликативную вилку
    \item Репликация бактериальной хромосомы на рисунке еще не завершилась
    \item Структура А с ходом времени может уменьшаться
\end{enumerate}

\explanationSection

\answerMath{1, 2, 4.}