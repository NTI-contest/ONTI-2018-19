\assignementTitle{}{8}

Мышечное
сокращение регулируется, в том числе, и на уровне спинного мозга. На рисунке
показана схема рефлекса, запускаемого сухожильным органом Гольджи. Частота
нервных импульсов, идущих от сухожильного органа Гольджи в спинной мозг,
увеличивается при растяжении органа Гольджи. Тормозный нейрон выделяет
тормозящий нейромедиатор в синаптическую щель, а возбуждающий нейрон –
возбуждающий медиатор. Мотонейрон стимулирует сокращение соответствующей мышцы.

\putImgWOCaption{10cm}{1}

Выберите
верные утверждения:

\begin{enumerate}
    \item Согласно данной схеме, растяжение сухожильного органа мышцы 1 стимулирует расслабление мышцы 2.
    \item Этот рефлекс работает по принципу положительной обратной связи 
    \item Разрушение тормозного нейрона повлияет на возможность протекания данного рефлекса.
    \item Повреждения головного мозга напрямую отразятся на работе этого рефлекса
\end{enumerate}

\explanationSection

\answerMath{3.}