\assignementTitle{}{8}

Аутоиммунный тиреоидит – самое распространенное аутоиммунное
заболевание человека. По разным оценкам от него страдают от 1 до $3\%$ всего
человечества. Заболевание характеризуется тем, что клетки иммунной системы
атакуют тиреоциты, в результате снижается уровень тироксина и трийодтиронина в
крови. Представьте, что вы разрабатываете носимый аппарат, который способен
периодично вводить в кровь тироксин, чтобы компенсировать его недостаток. Вы
решили начать свои исследования с экспериментов на животных. Определите, в
каком порядке вы будете проводить данное исследование:

\begin{enumerate}
    \item При помощи статистических методов подобрать размеры контрольной и экспериментальных групп животных, чтобы получить статистически значимые результаты.
    \item Сформулировать основную гипотезу. Подобрать модель гипотиреоза, подходящую для изучения аутоиммунного тиреоидита.
    \item Обработать полученные данные. Представить их в виде картинок, диаграмм, таблиц и т.д., проверить с помощью статистических методов гипотезу.
    \item Сформулировать выводы, относительно полученных данных.
    \item Продумать дальнейшие эксперименты для улучшения вашей методики лечения аутоиммунного тиреоидита или внедрения вашего проекта в клиническую практику.
    \item Провести эксперименты с животными.
\end{enumerate}



\answerMath{2, 1, 6, 3, 4, 5.}