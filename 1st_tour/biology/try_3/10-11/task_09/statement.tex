\assignementTitle{}{6}

Дина
скушала булочку массой 56 г и, поскольку она тщательно следит за фигурой,
решила бегать до тех пор, пока не потратит энергию, заключенную в данной
булочке. Молярная масса глюкозы 180 г/моль, молярная масса воды 18 г/моль. Из 1
моль глюкозы в ходе окислительного метаболизма получается 32 моль АТФ. Считая,
что булочка целиком состоит из крахмала, а также зная, что энергоемкость
АТФ  составляет около 7,5 ккал/моль, и в
среднем человек при беге тратит 500 ккал/час, оцените сколько времени Дине
нужно бежать, ответ приведите в целых минутах. Ответ введите в виде числа

\solutionSection

При образовании крахмала из глюкозы, из-за образования гликозидной связи молярная 
масса уменьшается на молярную массу воды. Таким образом, 56 г крахмала соответствует 
$56 \text{г}/(180-18 \text{г/моль})=0,35$ молям глюкозы. Поскольку каждый моль глюкозы - соответствует 32 моль АТФ, 
а каждая АТФ – это около 7,5 ккал/моль, то общая энергетическая ценность составит около 83 ккал. 
Разделив на 500 ккал/час и умножив на 60 минут, получаем около 10 минут.

\answerMath{10.}