\assignementTitle{}{8}

Некоторые типы
диабета характеризуется тем, что клетки иммунной системы атакуют вырабатывающие
инсулин клетки. Это приводит к нарушению регуляции уровня сахара в крови, так
как не вырабатывается в достаточном количестве инсулин. Представьте, что вы
разрабатываете носимый аппарат, который способен вводить в кровь инсулин с
некоторой периодичностью, чтобы компенсировать его недостаток. Вы решили начать
свои исследования с экспериментов на животных. Определите, в каком порядке вы
будете проводить данное исследование:

\begin{enumerate}
    \item Сформулировать основную гипотезу. Подобрать модель сахарного диабета, подходящую для изучения данного типа заболевания.
    \item Сформулировать выводы, относительно полученных данных.
    \item При помощи статистических методов подобрать размеры контрольной и экспериментальных групп животных, чтобы получить статистически значимые результаты.
    \item Провести эксперименты с животными.
    \item Обработать полученные данные. Представить их в виде картинок, диаграмм, таблиц и т.д., проверить с помощью статистических методов гипотезу.
    \item Продумать дальнейшие эксперименты для улучшения вашей методики борьбы с симптомами сахарного диабета или внедрения вашего проекта в клиническую практику.
\end{enumerate}

\explanationSection

\answerMath{1, 3, 4, 5, 2, 6.}