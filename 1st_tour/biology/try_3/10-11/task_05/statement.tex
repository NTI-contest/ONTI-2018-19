\assignementTitle{}{6}

У бактерии в гене аргининовой тРНК, которая
ранее узнавала кодон AGA, произошла нуклеотидная замена, и она стала узнавать
кодон UGA (который обычно является стоп-кодоном). К возможным последствиям
подобной замены можно отнести:

\begin{enumerate}
    \item Трансляция белков, содержащих аргинин, будет внезапно прерываться 
    \item В транслируемых белках аргинин будет замещаться на иные аминокислоты 
    \item Некоторые из транслируемых белков увеличат свою длину
    \item Транскрипция тРНК будет завершаться досрочно
    \item Потребность клетки в аргинине возрастет
\end{enumerate}

\explanationSection

\answerMath{3, 5.}