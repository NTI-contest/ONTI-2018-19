\assignementTitle{}{8}

Один из методов
диагностики состояния легких — измерение скорости потока вдыхаемого и
выдыхаемого воздуха с последующим построением кривой зависимости скорости
потока от объема легких. Нормальная кривая изображена на рисунке зеленым
цветом. Изучите рисунок и выберите все верные утверждения.

\putImgWOCaption{10cm}{1}

Выберите верные утверждения:

\begin{enumerate}
    \item скорость потока считают положительной на вдохе и отрицательной на выдохе
    \item синяя кривая характерна для пациентов с эмфиземой легких (патологическим расширением воздушных пространств мелких бронхиол)
    \item фиолетовая кривая характерна для пациентов с пневмотораксом (попаданием воздуха в плевральную полость и коллапсом легкого)
    \item желтая кривая характерна для пациентов с фиброзом легких (разрастанием соединительной ткани)
    \item красная кривая характерна для пациентов с рецидивирующей стадией открытой формы туберкулеза легких
\end{enumerate}



\answerMath{1, 2, 4.}