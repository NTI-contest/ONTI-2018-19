\assignementTitle{}{6}{}

Люба выпила утром чай, в который положила 17 г
сахарозы и решила прогуляться, чтобы потратить всю энергию содержащуюся в
утреннем чае. Молярная масса глюкозы и фруктозы равна 180 г/моль, молярная
масса воды 18 г/моль. Из 1 моль глюкозы в ходе окислительного метаболизма
получается 32  моль АТФ. Зная, что
энергоемкость АТФ составляет около 7,5 ккал/моль, и в среднем человек при
ходьбе тратит 150 ккал/час, оцените сколько времени Любе нужно прогуливаться,
ответ приведите в целых минутах. Ответ введите в виде числа.

\solutionSection

При образовании крахмала из глюкозы, из-за образования гликозидной связи молярная масса уменьшается 
на молярную массу воды. Таким образом, 17 г сахара соответствует $17 \text{г}/(180-18 \text{г/моль})=0,105$ молям 
глюкозы. Поскольку каждый моль глюкозы - соответствует 32 моль АТФ, а каждая АТФ – это около 7,5 ккал/моль, 
то общая энергетическая ценность составит около 25 ккал. Разделив на 150 ккал/час 
и умножив на 60 минут, получаем около 10,1 минут, т.е. 10 минут.

\answerMath{10.}