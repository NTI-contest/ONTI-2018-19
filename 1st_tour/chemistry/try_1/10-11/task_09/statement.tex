\assignementTitle{}{10}{}

Титриметрическое определение буферной емкости ацетатного буфера по щелочи.

Для каждой строки выберите верное значение.

Отбирают (навеску/аликвоту/пробу) 10,00 мл ацетатного буфера в колбу для титрования, добавляют 
2 капли (фенолфталеина/метилоранжа/лакмуса) и титруют стандартным раствором (кислоты/щелочи/ацетата натрия) 
до появления неисчезающей розовой окраски. 

Титрование повторяют не менее 3 раз при условии, что разница в полученных значениях объемов 
раствора, добавленного из (бюретки/пипетки Мора/цилиндра) не превышает 0,1 мл. По полученным результатам рассчитывают 
буферную емкость раствора.

\answerMath{аликвоту, фенолфталеина, щелочи, бюретки.}