\assignementTitle{}{10}

Образец магния массой 7.68 г обработали разбавленной азотной кислотой. В результате реакции образовалось три различных азотсодержащих продукта, единственным газом из которых  является закись азота, занимающая после приведения к н.у. объем 448 мл. Рассчитайте массы двух других образовавшихся продуктов. В ответе укажите меньшую из них в граммах, округленную до десятых. Атомную массу элементов округлять до целых.



\answerMath{4.8.}