\assignementTitle{}{10}{}

К 128.1 г $22\%$-ного раствора $Na_2SO_4$ добавили
$30\%$-ный раствор серной кислоты, в результате образовался пересыщенный раствор $NaHSO_4$ 
(других растворенных веществ не оказалось). Какая
масса кислой соли может выпасть в виде кристаллов при данной температуре, если
ее растворимость составляет 28.5~г/100~г воды? Ответ, выраженный в граммах,
округлите до целых. Для расчета использовать молярные массы веществ также с
точностью до целых.



\answerMath{6.}