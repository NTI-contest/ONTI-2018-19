\assignementTitle{}{10}{}

Стандартизация
раствора ЭДТА методом комплексонометрического титрования.

Для каждой строки выберите верное значение.

Рассчитывают (навеску/аликвоту/пробу) соли $ZnSO_4 \cdot 7H_2O$, необходимую для приготовления требуемого объема 0.05 н. 
раствора. Взвешивают ее на (аналитических/технических/лабораторных) весах, переносят в (мерную/плоскодонную/
круглодонную) колбу, растворяют в воде и доводят объем до метки. 
Рассчитывают точную (массу/концентрацию/мольную долю) приготовленного раствора (моль-экв/л). 

В колбу для титрования переносят пипеткой аликвотную часть стандартного раствора соли цинка, добавляют 
20–25 мл аммиачного буфера и индикатор на кончике (стеклянной палочки/шпателя/металлической ложки). Титруют раствором ЭДТА до перехода сиреневой 
окраски раствора в синюю.

\answerMath{навеску, аналитических, мерную, концентрацию, шпателя.}