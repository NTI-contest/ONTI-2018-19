\assignementTitle{}{10}{}

Правило
Вант-Гоффа гласит, что скорость любой реакции при нагревании на 10 градусов
Цельсия возрастает в фиксированное число раз (от 2 до 4). Растворение образца
алюминия в растворе щелочи при $20^{\circ} C$ заканчивается за 36 минут, а при $40^{\circ} C$ такой 
же образец растворяется за 4 минуты. Сколько секунд потребуется для растворения 
такого же образца при температуре $65^{\circ} C$?
Ответ округлите до целых.



\answerMath{15.}