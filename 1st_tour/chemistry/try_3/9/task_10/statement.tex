\assignementTitle{}{10}{}

К 128,1 г 22$\%$-ного раствора Na2SO4 добавили 30$\%$-ный
раствор серной кислоты, в результате образовался пересыщенный раствор NaHSO4 (других
растворенных веществ не оказалось). Какая масса кислой соли может выпасть в
виде кристаллов при данной температуре, если ее растворимость составляет
28,5г/100г воды? Ответ, выраженный в граммах, округлите до целых. Для расчета
использовать молярные массы веществ также с точностью до целых.



\answerMath{6.}