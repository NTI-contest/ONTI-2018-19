\assignementTitle{}{20}

Лампы накаливания часто перегорают во время включения. 
Для того, чтобы увеличить срок их службы, было разработано устройство, позволяющее включать лампу постепенно. 
На графике показан рост квадрата силы тока, протекающего через лампу, с течением времени, начиная с момента 
включения электрической лампы (растет линейно с течением времени).  Квадрат силы тока на графике в амперах в 
квадрате, время в десятых долях секунды. Для решения задачи сопротивление лампы считайте постоянным и равным $220$ Ом. 

В реальности зависимость сопротивления от температуры играет существенную роль, однако учет этого сильно усложнит 
решение задачи.

\putImgWOCaption{5cm}{1}

Теплоёмкость вольфрама можно определить по таблице:

\putImgWOCaption{15cm}{2}

Какой массы $m$ (в граммах с точностью до сотых) нужно сделать спираль из вольфрама, 
чтобы за 3 секунды она успела разогреться до T = $1078.1 $ K? Считать, что $90 \% $ работы электрического тока идет на нагрев лампы. 
Начальная температура $300$ К.

\paramSection

$T$ в пределах от $1050$ до $1099$ градусов, шаг $0.1$ градус.  

Точность ответа  до $0.01$ грамма.

\solutionSection

За малое время $\Delta t$:  
$\Delta Q = m \cdot c \cdot \Delta T = 0.9 \cdot 220 \cdot I \cdot \Delta t$. Для нагрёва до температуры $T$ требуется: 
$Q = m \cdot (100 \cdot 135 + 200 \cdot (138+142+146)+(T-1000) \cdot 149)$ 
$$\dfrac{Q}{m}= 0.149 T - 50.3 = 0.149(T-1000) + 98.7$$

При этом это количество тепла можно посчитать, как $Q= U \cdot I \cdot \Delta t$, где средний ток можно вычислить из графика:
$$Q=\dfrac{0.3+0.345}{2} \cdot 3 \cdot 0.9 \cdot 220 = 191.565 \space \text{Дж}$$

Отсюда получаем ответ: $m=\dfrac{0.191565 \cdot 1000}{(0.149 \cdot (T-1000)+98.7)}$ грамм.

\answerMath{$\dfrac{0.191565 \cdot 1000}{(0.149 \cdot (T-1000)+98.7)}$.}