\assignementTitle{}{20}{}

Почти 40 лет назад советский ученый
Виктор Веселаго выдвинул гипотезу о существовании материалов с отрицательным
показателем преломления. Плоскопараллельная пластинка из такого материала
работала бы как линза, и была названа линзой Веселаго. Все
лучи, выходящие под малыми углами из $A$ , пройдя через плоскопараллельную пластину с 
отрицательным показателем преломления соберутся в точке $B$.

\putImgWOCaption{5cm}{1}

Пусть толщина пластины равна $d$ см, расстояние до точки $A$ от ближайшего края пластины
равно $x$ см, а показатель преломления линзы Веселаго равен $n < 0$. Чему равно $AB$? Ответ
дайте в сантиметрах с точностью до десятых.

Укажите ответ для заданных значений $n$, $d$, $x$.

\paramSection

$n$ в пределах от $-2.5$ до $-1.5$, шаг $0.1$;     

$d$ в пределах от $10$ см до $20$ см, шаг $1$ см; 

$x$ в пределах от $1$ см до $2$ см, шаг $0.1$ см.

\solutionSection

Пусть угол падения равен $\alpha$. Тогда луч, пройдя по горизонтали на расстояние~$x$ до линзы, по вертикали поднимется на 
$x \cdot tg(\alpha)$. Далее он пойдёт к нормали под углом~$\beta$, который определяется по закону преломления: 
$sin\beta=\left|\frac{sin\alpha}{n}\right|$. Далее он опустится на величину, равную $d \cdot tg(\beta)$. 
Потом выйдет из линзы снова под углом $\alpha$ и ему останется подняться на величину $(d \cdot  tg(\beta) - x \cdot tg(\alpha) )$, 
а по горизонтали это будет $(d \cdot  tg(\beta) - x \cdot tg(\alpha )) \cdot ctg(\alpha )$. 
Итак, получаем:  $AB=d+x+d \cdot \frac{tg(\beta)}{tg(\alpha )}-x \approx d \cdot (1 - \frac{1}{n})$.

\answerMath{$d \cdot (1 - \frac{1}{n})$.}