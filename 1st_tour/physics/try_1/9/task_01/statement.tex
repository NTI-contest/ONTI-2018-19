\assignementTitle{}{20}

Автоматическая колесная платформа-почтальон доставляет посылку. 

\putImgWOCaption{7cm}{1}

В начальный момент времени платформа находится в точке $А$ и ей надо доставить посылку в точку $C$. 
По полю платформа может ехать в любом направлении со скоростью $x$  км/ч, а на дороге (отрезок $BC$) ее скорость $y$ км/ч. 
Программа платформы разработана таким образом, чтобы она доставляла посылку за минимальное время. По
какой траектории будет двигаться платформа? Для ответа введите длину траектории
в метрах с точностью до десятых.


Укажите ответ для заданных значений $x$ и $y$.

\paramSection
$X$ в пределах от $20$ км/ч до $30$ км/ч, шаг $0.1$ км/ч.

$Y$ в пределах от $50$ км/ч до $80$ км/ч, шаг $0.1$ км/ч. 

Точность $L$ до $0.1$ метра.


\solutionSection

Согласно принципу Ферма в оптике, свет движется из начальной точки в конечную по пути, соответствующем минимальному времени движения. 
Данный принцип можно применить для решения данной задачи. 

\putImgWOCaption{7cm}{2}

Предположим, что дорога BC – это граница раздела двух сред, в 
которых свет распространяется со скоростями $v_1$ и $v_2$ соответственно. Мы будем пытаться определить траекторию луча, 
движущегося из точки $A$ в точку $С$, которая, согласно принципу Ферма, обеспечит минимальное время движения. Логично предполагая, 
что дрон хотя бы в самом конце пути к дому будет двигаться по проселочной дороге, получаем, что луч света часть пути должен 
распространяться вдоль границы двух сред. Такая ситуация реализуется при критическом угле падения на границу сред, 
при превышении которого в оптике происходит явление полного внутреннего отражения. Получаем, что для 
достижения минимального времени дрон, как и луч света, должен двигаться под углом, который определяется по формуле: 
$$cos \alpha =\dfrac{n_2}{n_1} = \dfrac{v_1}{v_2} = \dfrac{x}{y}.$$
Отсюда получаем длину пути:

$$600 \cdot \sqrt{ \dfrac{(y - x)}{(y^2 - x^2)} } +2000.$$

\answerMath{$600 \cdot \sqrt{ \dfrac{(y - x)}{(y^2 - x^2)} } +2000$.}