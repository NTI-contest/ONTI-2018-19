\assignementTitle{}{20}{}

Тонкая пластина из композитного материала состоит из двух слоев. 
Известно, что отношение коэффициента теплопроводности вещества, из которого состоит первый слой, 
к коэффициенту теплопроводности вещества, из которого состоит второй слой, равно $k$. 
Внешняя поверхность первого слоя поддерживается при температуре \linebreak $T_1 = 305$K, внешняя поверхность 
второго слоя при температуре $T_3 = 273$K. 

Как должны относиться толщины первого и второго слоев $(d_1 / d_2)$, если на границе между ними нужно 
обеспечить температуру $T_2$? Ответ дайте в виде числа, с точностью до десятых. 

Все температуры считайте постоянными.

Укажите ответ для заданных значений $k$, $T_2$.

\paramSection

$k$ от $2$ до $12$, шаг $0.5$.

$T_2$ от $280$ до $300$, шаг $0.1$.

\solutionSection

Заметим, что тепловой поток через пластину прямо пропорционален разности температур с разных сторон пластины, 
ее площади и обратно пропорционален его толщине. При этом тепловой поток через оба слоя должен быть одинаковым.

$\Phi= \kappa_1 \cdot S \cdot (T_1-T_2)/d_1 = \kappa_2 \cdot S \cdot (T_2-T_3)/d_2$, учитывая что $\kappa_1 / \kappa_2 = k$, получим: $k \cdot \frac{(305-T_2)}{(T_2-273)}$.

\answerMath{$k \cdot \dfrac{(305-T_2)}{(T_2-273)}$.}