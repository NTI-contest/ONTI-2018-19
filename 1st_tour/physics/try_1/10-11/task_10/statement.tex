\assignementTitle{}{10}

Автомобиль массы $M$ кг развивает максимальную мощность 608 лошадиных сил и максимальную скорость
301 км/ч на ровной дороге. 

Будем считать, что 1 л.с. = 0.74  кВт, а автомобиль тормозят постоянная и одинаковая во всех случаях сила трения о землю и сила сопротивления воздуха, пропорциональная квадрату скорости
движения машины. Для изучения характеристик автомобиля был поставлен следующий
эксперимент: автомобиль разгонялся до скорости $v_0$ , затем отключался двигатель и измерялся путь $S$ 
автомобиля до полной остановки (без включения тормоза за отсутствием такового).
Результаты эксперимента представлены в таблице:

\putImgWOCaption{15cm}{1}

Чему равен коэффициент пропорциональности $\mu$ в
силе сопротивления воздуха? Ответ дайте в Н $\cdot$ с$^2/$м$^2$ с точностью до десятитысячных.

Укажите ответ для заданного значения $M$.

\paramSection

$M$ меняется в диапазоне от $2400$ до $2500$ кг, шаг $1$ кг.

Точность ответа до  $0.0001$  H$\cdot$c$^2$/м$^2$.

\solutionSection

При малых скоростях путь линейно зависит от квадрата скорости, так как при них можно пренебречь сопротивлением воздуха: 
$\frac{S}{V_0^2} = \frac{M}{2 \cdot F}=1.6;  F=\frac{M}{3.2}.$ 

Коэффициент пропорциональности в выражении для силы сопротивления воздуха определим , зная максимальные скорость и мощность автомобиля. 
При  равномерном движении с максимальной скоростью выполняется соотношение:   $$\frac{P_{max}}{V_{max}} =F+\mu \cdot V_{max}^2 $$

\answerMath{$\left(\frac{608 \cdot 740}{\frac{301}{3.6}}-\frac{M}{3.2}\right)/\left(\frac{301}{3.6}\right)^2$.}