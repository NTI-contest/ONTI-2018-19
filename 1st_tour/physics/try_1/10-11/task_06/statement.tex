\assignementTitle{}{10}{}

Вольт-амперная характеристика реального источника ЭДС, используемого для питания
светодиода, задана таблицей. К этому источнику подсоединили последовательно резистор
сопротивления $R$ Ом и светодиод. Нужно спроектировать систему так, чтобы она светила максимально ярко. Какую
максимальную электрическую мощность на светодиоде можно получить, если доступны светодиоды с любыми параметрами зависимости падения напряжения и проходящего
через них тока? Считайте, что яркость света прямо пропорциональна мощности, выделяющейся на диоде. Ответ дайте в ваттах с точностью до десятых.

\putImgWOCaption{15cm}{1}

Укажите ответ для заданного значения $R$.

\paramSection

$R$ меняется в диапазоне от $1$ до $4$ Ом, шаг $0.2$ Ом. 

Точность ответа до $0.1$ Вт.

\solutionSection

Вольтамперная характеристика источника описывается формулой: \linebreak $U(I)=E-I \cdot r$.
Исходя из таблицы, находим, что $E = 6.5$ В, $r = 0.5$ Ом. Максимум мощности достигается, 
когда сопротивление $r+R$ равно тому отношению падения напряжения на светодиоде к силе 
электрического тока, который через него проходит. В таком случае:
$$E=I \cdot (r+R+r+R);  I=\frac{E}{2(R+r)}; P_{max}=I^2 \cdot (R+r) = \frac{10.5625 \cdot R + 0.5}{(R +0.5)^2}.$$

\answerMath{$\dfrac{10.5625 \cdot R + 0.5}{(R +0.5)^2}$.}