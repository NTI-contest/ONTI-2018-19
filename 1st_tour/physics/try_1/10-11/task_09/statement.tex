\assignementTitle{}{10}

Космическая станция, двигавшаяся на
геостационарной орбите с постоянной скоростью с выключенным двигателем, начала
ускоряться вдоль орбиты, свою орбиту при этом сохраняя. Точный динамический
акселерометр, установленный на станции, позволяет определить ее ускорение в 
любой момент времени. К несчастью, аппаратура, сломалась, и сигналом на Землю
передаются только средние значения ускорения, начиная с момента выхода из строя
оборудования. На рисунке показан график зависимости среднего тангенциального ускорения станции
от времени, начиная с момента начала ускорения. Время здесь указано в секундах,
полное ускорение станции в начале ускоренного этапа движения направлено под
углом $45$  градусов к её скорости. Определите, какое касательное ускорение имела
станция в момент времени $t$ , если
считать Землю однородным шаром с экватором $40\space000$ км, ускорение свободного
падения вблизи земной поверхности равным $9.8$ м/c$^2$. Можно считать, что станция всё время движется по той
же самой геостационарной орбите, увеличивая скорость, но при этом пренебрегать
изменением нормального ускорения станции в расчётах. Считайте, что динамический 
акселерометр определяет только тангенциальное ускорение в данном случае. Ответ дайте в м/c$^2$ с точностью до сотых.

\putImgWOCaption{7cm}{1}

Укажите ответ для заданного значений $t$.

\paramSection

$t$  меняется в пределах от $3$ до $10$ c, шаг $0.5$ c.

Точность ответа до $0.05$ м/c$^2$.

\soultionSection

Радиус геостационарной орбиты определяется из приравнивания угловой скорости обращения спутника вокруг 
планеты с угловой скоростью самой планеты:
$$m \cdot \omega^2 \cdot R_{orb}=G \cdot \dfrac{M \cdot m}{R_{orb}^2} ; R_{orb}=\sqrt[3]{\dfrac{G \cdot M}{\omega^2}}$$
Масса планеты не дана, но её можно посчитать, зная ускорение свободного падения на планете: 
$g=\dfrac{G \cdot M}{R^2} \Rightarrow G \cdot M = g \cdot R^2$. 

Отсюда $$R_orb=\sqrt[3]{\dfrac{g \cdot R^2}{\omega^2}}=\sqrt[3]{\dfrac{9.8 \cdot \left(\dfrac{40000000}{2 \cdot \pi}\right)^2}{\left(\dfrac{2 \cdot \pi}{24 \cdot 3600}\right)^2}} \approx 42191 \space \text{км}.$$
Начальное полное ускорение станции направлено под углом 45 градусов к её скорости, следовательно, начальное тангенциальное ускорение равно нормальному ускорению на этой геостационарной орбите. А оно, в свою очередь, равно:
$$a=\omega^2 \cdot R=\dfrac{42191 \cdot (2 \cdot \pi)^2}{(24 \cdot 3600)^2}  \cdot 1000=0.223  \text{м⁄с}^2$$ 
Таким образом, единица измерения ускорения на этом графике равна $a = 0.223$ м/c$^2$.

Динамический акселерометр определяет только тангенциальное ускорение у этой станции, потому что нормальное ускорение компенсируется 
силой тяжести. Для моментов времени $t > 2$ секунд среднее ускорение, измеряемое прибором, можно описать с помощью следующей формулы:  
$a_{\text{ср.}}(t)=a \cdot (\frac{t}{4}+0.5)$.

Среднее ускорение на участке, пройденном со второй секунды по секунду $t$, равно отношению разности скоростей к потраченному 
времени ($t - 2$ секунды):
$$a_{\text{ср.пер}}=\dfrac{v(t)-v(2)}{t-2}.$$

Так как на графике даны значения среднего ускорения, то: 
$$v(t)=a \cdot \left(\frac{t}{4}+0.5\right) \cdot (t-2)+v(2)=a \cdot \left(\dfrac{t^2}{4}-1\right)+v(2) $$
Чтобы получить отсюда ускорение, рассмотрим приращение за малое время $\Delta t$ скорости:
$$a(t)=\dfrac{(v(t+\Delta t)-v(t))}{\Delta t}=\dfrac{a}{4} \cdot \dfrac{(t+\Delta t)^2-t^2}{\Delta t}=\dfrac{a}{4} \cdot (2 \cdot t+\Delta t),\Delta t \rightarrow 0.$$
Отсюда получаем формулу для ускорения при $t>2$:  $a(t)=0.5 \cdot a \cdot t=0.112 \cdot t$.

\answerMath{$0.112 \cdot t$.}