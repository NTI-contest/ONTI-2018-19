\assignementTitle{}{10}

Для калибровки небольшого акселерометра его подвесили на нити длиной $L$ м и отпустили без начальной
скорости под некоторым острым углом к вертикали так, чтобы ускорение в его в
нижней точке равнялось по модулю ускорению в точке максимального отклонения.
Акселерометр показал, что минимальное полное ускорение достигается при угле
отклонения $\varphi$ , причём $cos(\varphi) = 0.8$. Считая, что
ускорение свободного падения равно $g = 9.8$  м/c$^2$, а акселерометр показывает в точке минимального
ускорения значение $X$  м/c2, определите относительную
погрешность $\varepsilon\%$  его измерения в этой
точке, т.е. отношение разницы между показанием акселерометра и 
полным ускорением к показанию акселерометра. Масса акселерометра $m$  кг. Ответ укажите в процентах с точностью до десятых
процента.

Укажите ответ для заданных значений $L$, $X$.

\paramSection

$m$ в пределах от $1$ до $3$ кг, шаг $0.1$ кг;  

$L$ в пределах от $1$ до $3$ метров, шаг $0.1$ метр;

$X$ в пределах от $8$ до $12$ м/c$^2$,  шаг $0.1$ м/c$^2$. 

Точность $\epsilon$ до $0.1 \%$.

\solutionSection

Закон сохранения энергии: $m \cdot g \cdot L \cdot (1-cos(fi_{max} ) )=\frac{m \cdot v_{max}^2}{2}.$

Равенство ускорений: $g \cdot sinfi_{max})=\frac{v_{max}^2}{L}.$

Отсюда получаем: $sin(fi_{max})+2 \cdot cos(fi_{max} )=2.$

Это уравнение имеет одно ненулевое решение в диапазоне $0 < fi_{max} < 90^{\circ}$:
 $$cos(fi_{max})=0.6;  sin(fi_{max} )=0.8.$$

Полное ускорение можно найти по теореме Пифагора:
$$a=g \cdot \sqrt{\frac{v^4}{g^2 L^2} +sin^2(fi)}=g \cdot \sqrt{4 \cdot (cos(fi)-cos(fi_{max}))^2+sin^2(fi)}= 9.8 \cdot \sqrt{0.52}$$

Теперь можно определить относительную погрешность:
$$\epsilon =\frac{|X-9.8 \cdot \sqrt{0.52}|}{X} \cdot 100 \% $$


\answerMath{$\sqrt{(X-9.8 \cdot \sqrt{0.52})^2} \cdot \frac{100}{X}.$}