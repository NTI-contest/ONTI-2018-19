\assignementTitle{}{10}{}

Для маркировки особенностей дна водоема решили
использовать придонные фонари,
излучающие свет однородно во все стороны. Фонари размещены на глубине $H$,
поверхность и дно водоема идеально ровные. При прохождении воды свет ослабляется в
среднем в $k$ раз на метр. Издалека к уединенному фонарю движется дрон и пролетает точно 
над источником света. Дрон летит над поверхностью воды, высотой полета можно пренебречь. 
Во время полета дрон анализирует силу светового сигнала. Найдите отношение максимальной силы сигнала к
минимальной с точностью до сотых. Там, где преломление возможно, изменение
коэффициента отражения в зависимости от угла падения в расчетах не учитывайте.
 Показатель преломления воды 1,33. Фонарь
можно считать точечным источником света. Характерные размеры сенсора дрона много меньше $H$.         

Найдите ответ задачи для заданных значений $H$ и $k$.

\paramSection

$H$ от $4$ до $16$, шаг $1$;

$K$ от $0,01$ до $0,05$, шаг $0,005$.

\solutionSection

Часть света не сможет выйти за поверхность воды из-за полного внутреннего отражения, эту долю можно оценить, 
как отношение телесного угла, определяемого критическим углом полного внутреннего отражения, к полному телесному углу $4\pi$.  
Кроме того, часть света рассеется в воде пропорционально $(1-k)^{H/1\text{м}}$.

\answerMath{$(1-k)^{-0.51186 \cdot H}$.}