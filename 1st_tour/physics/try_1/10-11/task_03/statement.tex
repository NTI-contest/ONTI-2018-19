\assignementTitle{}{10}{}

Космическое агентство запустило спутник Солнца, который двигается вокруг Солнца по круговой орбите со скоростью $V$ км/сек.

На сколько Вт/м$^2$ мощность потока солнечных лучей на спутнике больше, чем на расстоянии от Солнца, равном орбитальному радиусу планеты, которая получает максимальную полную мощность солнечного излучения из приведённых в таблице? Ответ дайте с точностью до целых.

Ниже приведена таблица параметров планет:

\putImgWOCaption{15cm}{1}

Диаметр и масса даны в диаметрах и массах Земли. 

1 астрономическая единица (а.е.) равна 149 597 870 700 метров. Земной
год считать равным ровно 365 суток по 24 часа.  

Мощность солнечного света попадающего на единицу поверхности на орбите Земли (солнечная постоянная) равна
1387  Вт/м$^2$. Гравитационную постоянную считать равной $6.67 \cdot 10^{-11}$ м$^3/($кг $\cdot$ с$^2)$,
для упрощения расчётов все орбиты считать круговыми.

Укажите ответ для заданного значения $V$ км/с.

\paramSection

$V$ в пределах от $10$ до $20$ км/сек, шаг  $0.1$ км/сек.

Точность ответа  до $1$ Вт.

\solutionSection

Солнечная масса может быть рассчитана по следующей формуле: 
$$M = \frac{(4 \cdot \pi ^2 \cdot a^3)}{(G \cdot T^2 )}$$

где $T$ -- период обращения планеты вокруг Солнца 

$a$ -- радиус орбиты планеты

$G$ -- гравитационная постоянная Ньютона.

Максимальную полную мощность солнечного излучения из приведённых в 
таблице получает Юпитер, так как у него максимальное отношение диаметра к радиусу орбиты.

Орбитальная скорость спутника равна:  $V=\sqrt{G \cdot \frac{M}{R}}$, 
значит:  $$R=\frac{G \cdot M}{V^2} = \frac{4 \cdot \pi ^2 \cdot a^3}{V^2 \cdot T^2}$$.
$$R=\frac{4 \cdot \pi ^2 \cdot 149~597~870~700^3}{V^2 \cdot (365 \cdot 24 \cdot 3600)^2} \approx \frac{(1.329 \cdot 10^{20})}{V^2} \approx \frac{(29.806 \text{км/сек} ^2)}{V} \cdot 1 \space \text{а.е.}$$

Мощность потока солнечных лучей на станции обратно пропорциональна квадрату расстояния до Солнца: $P=1387 \cdot \left( \frac{V}{29.806 \space \text{км/сек}} \right) ^4$

Аналогично для Юпитера: $P_{\text{ю}}=1387 \cdot \left(\frac{1}{5.2}\right)^4 \approx 51.3 \space \text{Вт} $.

\answerMath{$1387 \cdot (\frac{N}{29.806})^4-51.3$.}