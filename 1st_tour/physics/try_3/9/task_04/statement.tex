\assignementTitle{}{20}
Суперлинзой Всеселаго называют плоскопараллельную
пластинку из вещества с отрицательным показателем преломления, которая работает
как линза, если предмет находится вблизи её и имеет небольшие размеры в
сравнении с её толщиной.

Найдите расстояние между изображениями светящейся
точки в суперлинзе, если толщина её равна d см, расстояние до светящейся точки  от ближайшего края суперлинзы равно х см, а показатель преломления линзы Веселаго равен $n < 0$. Ответ дайте в сантиметрах с точностью до десятых.

Укажите решение для заданных значений $n$, $d$ см, $x$ см.

\paramSection

$n$ в пределах от -2.5 до -1.5, шаг 0.1;    

$d$ в пределах от 10 см до 20 см, шаг 1 см;  

$x$ в пределах от 1 см до 2 см, шаг 0.1 см.

\soultionSection

Пусть угол падения равен $\alpha$ . Тогда луч, пройдя по горизонтали на расстояние $x$ до линзы, 
по вертикали поднимется на $x \cdot tg(\alpha )$. Далее он пойдёт к нормали под углом  $\beta$ , который 
определяется по закону преломления: $sin\beta =\left|\dfrac{sin\alpha}{n}\right|$. Далее он опустится на 
величину, равную $d \cdot  tg( \beta )$. По пути пересечёт другие лучи в точке $C$ и создаст первое изображение,  
на расстоянии  $x \cdot tg(\alpha ) \cdot ctg( \beta )$  от левого края линзы. Потом выйдет из линзы снова под 
углом $\alpha$  и ему останется подняться на величину $(d \cdot  tg( \beta ) - x \cdot tg(\alpha ) )$, а по 
горизонтали это будет $(d \cdot  tg( \beta ) - x \cdot tg(\alpha )) \cdot ctg(\alpha )$. Итак, получаем:  
$$AB=d+x+d \cdot \dfrac{tg( \beta )}{tg(\alpha )}-x \approx d \cdot \left(1-\dfrac{1}{n}\right)$$
$AC=x+x \cdot tg(\alpha ) \cdot ctg( \beta ) \approx x \cdot \left(1+\dfrac{\alpha}{\beta} \right) \approx x 
\cdot (1-n);  CB=d \cdot \left(1-\dfrac{1}{n}\right)-x \cdot (1-n)$


\answerMath{$d \cdot (1-\dfrac{1}{n})-x \cdot (1-n)$.}