\assignementTitle{}{20}

Цилиндр из свинца, находящийся в вертикальном теплоизолирующем кожухе поддерживают снизу при температуре  
$T_1$ , а сверху при температуре  $T_2$ . Часть свинца расплавилась. Найдите высоту слоя расплавленного свинца, 
если  общая высота цилиндра $L$. Считайте, что теплопроводности жидкого и твердого свинца постоянны и 
равны соответственно 15.5 Вт/(м$\cdot$K) у жидкого и 31.6 Вт/(м$\cdot$K) у твердого. 
Температура плавления свинца 600К.Ответ дайте в сантиметрах с точностью до десятых.

Укажите решение для указанных значений $T_1$ K, $T_2$ K, $L$ см.

\paramSection

$T_1$ от 550 до 570, шаг 1

$T_2$ от 630 до 650, шаг 1

$L$ от 30 до 50, шаг 1

\solutionSection

Температура на границе раздела твердого и жидкого состояний равна температуре плавления свинца. 
Запишем равенство теплового потока в установившемся состоянии:

$\text{КАППА}_{\text{тв}} \cdot (T_0-T_1)/d_2 = \text{КАППА}_{\text{распл}} \cdot (T_2-T_0)/d_1$, 
где $d_1$ и $d_2$ – это толщины жидкого и твердого слоев. При этом $d_1+d_2 = L$.

\answerMath{$L/(1+\dfrac{31.6}{15.5} \cdot \dfrac{600-T_1}{T_2-600})$.}