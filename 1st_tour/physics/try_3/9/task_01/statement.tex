\assignementTitle{}{20}
Беспилотный
автомобиль рассчитал скоростной режим движения между городами. В таблице указаны средние скорости и соответствующие промежутки времени, которые автомобиль эти скорости сохраняет.

\putImgWOCaption{16cm}{1}

В результате автомобиль оказывается в пункте назначения.

Однако в системе
управления произошёл сбой, и перепутались местами значения
скоростей для всех этих 23 временных интервалов. 
Определите, сколько километров до места назначения мог не доехать
автомобиль в самом худшем случае и выведите в ответ время T в минутах, за которое он проедет это расстояние со скоростью $V$. Ответ дайте в минутах с точностью до 0,5 минуты.

Укажите ответ для заданных значений $V$ км/ч, $N$ мин.

\paramSection

$V$ в пределах от $90$ до $120$ км/ч, шаг  0.1 км/ч.  

$N$ в пределах от 1 до 4 минут, шаг  1 минута.

Точность ответа  до  0.1 минуты.

\solutionSection

Рассчитаем сначала, какое расстояние до города:
$$L=1 \cdot 60 +2 \cdot 40+ 3 \cdot 50+ 55 \cdot 3060+45 \cdot 2560+ 65 \cdot 3560+ 34 \cdot 2560+ 20 \cdot 560+ 75 \cdot 3560+ 50 \cdot 4060+$$
$$+ 60 \cdot 5060+80 \cdot 2060+40 \cdot 1560+ 32 \cdot 6560+ 28 \cdot 1560+20 \cdot 1060+ 45 \cdot 2560+ 50 \cdot 3560+ 60 \cdot 5060+$$
$$+30 \cdot 1060+40 \cdot 1560+50 \cdot 1560+N \cdot X60=72416+N \cdot X60$$  

Самый худший случай соответствует ситуации, когда более длинные интервалы автомобиль проходил с меньшими скоростями.

Отсортируем времена:

[N, 5, 10, 10, 15, 15, 15, 15, 20, 25, 25, 25, 30, 35, 35, 35, 40, 50, 50, 60, 65, 120, 180]

Отсортируем скорости в обратном порядке:

[X, 80, 75, 65, 60, 60, 60, 55, 50, 50, 50, 50, 45, 45, 40, 40, 40, 34, 32, 30, 28, 20, 20] 

Посчитаем сумму произведений соответствующих величин и поделим на 60:

$L_{min}=50513+N \cdot X60  ;  d=21856$

Отсюда ответ: $ T=21856 X \cdot 60= \frac{13130}{V}$

\answerMath{$\frac{13130}{V}$.}