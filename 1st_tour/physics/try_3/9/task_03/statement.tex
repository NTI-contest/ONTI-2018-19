\assignementTitle{}{20}{}

На орбите одного из перечисленных в таблице спутников планет Солнечной системы, находится исследовательская станция. Известно, что период обращения станции вокруг этого спутника на орбите высотой 100 км минимальный из всех возможных для станций, вращающихся вокруг спутников, приведенных в таблице, по такой орбите.

\putImgWOCaption{16cm}{1}

Спутник проведет на орбите N периодов обращения. 

Пренебрегая гравитационным действием других тел, кроме этого
спутника, а также считая орбиту круглой и спутник шаром, вычислите, сколько
земных суток станция проведет на орбите. 

Ответ приведите в сутках с точностью до целых.

Укажите решение для заданного значения $N$

\paramSection

$N$ в пределах от 200 до 1000, шаг 20;         

Точность ответа  до  1  суток.

\solutionSection

Для того, чтобы период обращения был минимальным, должно быть минимальным отношение суммы радиуса спутника  
и высоты орбиты к первой космической скорости на этой высоте. Вторая космическая скорость $v_2=\sqrt{2G \frac{M}{R}}$, 
а первая \linebreak  $v_1=\sqrt{G \frac{M}{R+h}}=\frac{v_2}{\sqrt{2}} \cdot \sqrt{\frac{R}{R+h}}$ .

Тогда период равен:  $$T=\frac{2 \cdot \pi \cdot (R+h)}{  v_1 }=\frac{2\sqrt{2} \cdot \pi \cdot (R+h)}{v_2}  \cdot \sqrt{\frac{R+h}{R}}.$$

Максимальный он для Ио, и равен:  $ T \approx 6827.63$ сек.


\answerMath{$0.0790 \cdot N$.}