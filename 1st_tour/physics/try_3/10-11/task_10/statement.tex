\assignementTitle{}{10}{}

При движении по автотрассам необходимо двигаться с постоянной
установленной на этой трассе скоростью. Это приводит к тому, что нередко самый
экономичный по расходу топлива маршрут не является самым коротким.  Пусть три города (A, B, C) 
соединены одной общей кольцевой
автотрассой в форме окружности и тремя радиальными.  Длина трассы AB
равна 300 км, трассы BC - 500 км, АС - 700
км, а круговая проходит через все города. 
При движении по трассе AB рекомендованная
скорость равна 70 км/ч, по BC - 50 км/ч, по АС - 30 км/ч. 

На кольцевой автотрассе рекомендованная скорость V км/ч. Найти оптимальный маршрут, проходящий из города   A по городам B, C и возвращающий обратно в A, на котором будет израсходовано наименьшее количество топлива. В
ответе записать, во сколько раз это количество топлива меньше, чем при движении
по маршруту только по кольцевой дороге (ответ больше единицы) с точностью до десятых.

Считать, что
расход топлива пропорционален мощности автомобиля и времени движения с этой мощностью. Сила тяги равна силе сопротивления воздуха, которая пропорциональна квадрату скорости
движения автомобиля.

Укажите решение для заданного значения $V$ км/ч.

\paramSection

$V$ в пределах от 55 до 65 км/час, шаг 0.1 км/час. 

Ответ $v$ с точностью до 0.1.

\solutionSection

$N=F_{\text{сопр}} \times v=k \times v^2 \times v=k \times v^3$                         $S=v \times t \Rightarrow t=\frac{S}{v} $ 

Тогда расход топлива Q:   $Q=\alpha Nt=k \times \alpha  \times v^2 \times S$, где $k$, $\alpha $-коэффициенты 
пропорциональности. 

$Q_AB=k \times \alpha  \times 70^2 \times 300; Q_BC=k \times \alpha  \times 50^2 \times 500; $
$Q_AC=k \times \alpha  \times 30^2 \times 700$

Пусть $\angle ABC=\alpha$ : $$cos\alpha = \frac{300^2+500^2-700^2}{2 \times 300 \times 500}=-\frac{1}{2}$$

Таким образом, $\alpha =120^{\circ}$.  Радиус кольцевой дороги: 

$$R=\frac{AC}{2 \times sin\frac{\alpha}{2}}= \frac{700}{2 \times sin60^{\circ}}=\frac{700}{\sqrt{3}} \text{км}.$$

Таким образом, чтобы найти оптимальный путь, нужно сложить произведения длин путей на квадраты скоростей.  
Проделав это, можно обнаружить , что оптимальный маршрут такой:  хорда $AB$, дуга $BC$, хорда $CA$;  или  хорда  
$AC$, дуга $CB$, хорда $BA$.

Найдем $sinA$:  $S=\frac{1}{2} AB \times BC sinA=\frac{1}{2} AB \times AC \times sinA$

$$sinA=\frac{BC}{AC} \times sin\alpha =\frac{5}{7} \times \frac{\sqrt{3}}{2}=\frac{5\sqrt{3}}{14}; \angle BOC=2\angle A=2arcsin\left(\frac{5\sqrt{3}}{14}\right)$$
$$L_{\text{дуги BC}}=2R \times arcsin\left(\frac{5\sqrt{3}}{14}\right) $$

Искомое  отношение $v$ равно:     
$$v=\frac{V^2 \times 2 \times \pi \frac{700}{\sqrt{3}}}{\sqrt{\frac{1400}{3} V^2}+30^2 \times 100+70^2 \times 300}$$


\answerMath{$\frac{V^2 \times 2 \times 3.14 \frac{700}{\sqrt{3}}}{\sqrt{\frac{1400}{3} V^2}+30^2 \times 100+70^2 \times 300}$.}