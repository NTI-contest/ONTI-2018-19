\assignementTitle{}{10}{}

В таблице показана зависимость напряжения
аккумуляторной батареи и плотности электролита от процента заряда в ней. В мастерской
оказалось 10 батарей с зарядом $95 \%$, 5 батарей с зарядом $65 \%$, 3 батареи с
зарядом $25 \%$, 2 батареи \linebreak с зарядом $40 \%$.

\putImgWOCaption{10cm}{1}
Кроме
того, там же оказалось 8 ламп с сопротивлением нитей накала X Ом и 12 ламп с
сопротивлением нитей накала Y Ом. К каждой лампе требуется
поставить аккумулятор так, чтобы суммарное освещение (и, соответственно,
выделяемая мощность) было максимальным. Чему станет равна суммарная выделяемая
мощность в лампах в этот момент? 

Сопротивление батарей много меньше X и Y. Ответ дайте в ваттах с точностью до десятых.

Укажите решение для заданных значений $X$ Ом, $Y$ Ом

\paramSection

$X$ в пределах от 11 до 15 Ом, шаг 0.1 Ом; 

$Y$ в пределах от 7 до 10 Ом, шаг 0.1 Ом; 

Точность ответа до 0.1 Ом.

\solutionSection

Чтобы мощность стала максимальной, требуется поставить батареи с большим напряжением к 
лампам с меньшим сопротивлением и наоборот. Тогда, по таблице:
$$P=\frac{10 \cdot 12.64^2}{Y} + \frac{2 \cdot 12.32^2}{Y}+\frac{3 \cdot 12.32^2}{X}+\frac{3 \cdot 12.00^2}{X}+\frac{2 \cdot 12.12^2}{X}$$


\answerMath{$\frac{10 \cdot 12.64^2}{Y} + \frac{2 \cdot 12.32^2}{Y}+\frac{3 \cdot 12.32^2}{X}+\frac{3 \cdot 12.00^2}{X}+\frac{2 \cdot 12.12^2}{X}$.}