\assignementTitle{}{10}{}

В одном из отсеков управляемого космического
корабля была обнаружена микротрещина. За время поиска трещины давление воздуха в
корабле успело упасть до 724 мм ртутного столба. 

Трещина была заклеена
пластырем из эпоксидной смолы. Считая трещину кругом диаметра 2 мм, оцените,
какой минимальной толщины h
должен быть пластырь, чтобы он не порвался сразу, если предел прочности
эпоксидной смолы равен N МПа. Давление воздуха в корабле считайте равным 724 мм ртутного столба, плотность ртути 13600 кг/м$^3$, $g=9.8$ м/c$^2$. Для оценки считайте, что напряжение материала распределяется равномерно по всему
куску пластыря, а сам он выгибается наружу в космическое пространство в виде
ровной полусферы,  $h << d = 2$ мм. Ответ приведите в микрометрах с точностью до сотых.

Примечание:
для устойчивого сдерживания атмосферного давления на протяжении длительного
времени требуется намного более толстый пластырь.

Укажите решение для заданного значений $N$ МПа

\paramSection

$N$ в пределах от 70 до 90, шаг 2;         

Точность ответа  $h_{min}$  до  0.02 мкм.

\solutionSection

Предел прочности нужно рассчитывать на касательное напряжение $P_s$. Чтобы посчитать его значение, 
приравняем работу $dA$, необходимую для радиального растяжения сферы из материала на $dr$, и приращение 
потенциальной энергии $dW$, связанной с касательным напряжением:

$dA=F_{\tau} \cdot dR=2 \cdot \pi \cdot P_s \cdot h \cdot R \cdot dR$; 

$dW=P \cdot \frac{2}{3} \cdot \pi \cdot ((R+dR)^3-R^3 )=P \cdot 2 \cdot \pi \cdot R^2 \cdot dR$ 

$2 \cdot \pi \cdot P_s \cdot h \cdot R \cdot dR=P \cdot 2 \cdot \pi \cdot R^2 \cdot dR$; 

$ P_s=P \cdot \frac{R}{h} \leq N$ МПа.  

Отсюда:
$h_{min}=\frac{P \cdot R}{N \cdot 10^6} \approx \frac{0.724 \cdot 9.8 \cdot 13600 \cdot 0.001}{N \cdot 10^6} \approx \frac{96.5}{N}$  мкм

Другой вариант решения.

Можно просто воспользоваться аналогией с поверхностным натяжением жидкости. 
Если поверхность жидкости - выпукла (вогнута), то при равновесии давление по разные стороны от 
неё будет неодинаковым:  $\Delta P= \frac{\sigma}{R}$ . Сила поверхностного натяжения тогда равна: 
$F_s= \sigma  \cdot 2 \cdot \pi \cdot R$.

Напряжение $P_s$ всюду одинаково и его можно посчитать на границе:  
$P_s=\frac{F_s}{2 \cdot \pi \cdot R \cdot h}= \frac{\sigma}{h}=P \cdot \frac{R}{h}$.


\answerMath{$\frac{96.5}{N}$.}