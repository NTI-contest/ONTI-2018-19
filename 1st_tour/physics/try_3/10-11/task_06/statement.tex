\assignementTitle{}{10}

Небольшой круглый катод радиуса $r$ находится на большом медном диске толщины $h$ мм. Диск окаймлён анодом в форме окружности радиуса
$R$. Оказалось, что каждые $t$ секунд  анода достигало $N \cdot 10^{18}$  электронов. Какая сила тока будет, если, не
меняя разности потенциалов, увеличить вдвое радиусы катода и анода? Ответ дайте в амперах с точностью до тысячных. Заряд электрона
принять равным \linebreak $e = 1.6 \cdot 10^{-19}$ Кл.

Укажите решение для заданных значений $N$, $h$ мм, $t$ c.

\paramSection

$h$ в пределах от 10 до 20 мм, шаг  1 мм;   

$t$ в пределах от 2 до 9 секунд, шаг 0.1 сек;

$N$ в пределах от 1 до 9, шаг 0.1.

Точность ответа  до  1 мА.

\soultionSection

Рассмотрим одну медную прямоугольную полоску, идущую от катода к аноду:

\putImgWOCaption{6cm}{1}

Пусть $x$ –  координата её слоя малой толщины $\Delta x$ от центра окружности, \linebreak $r<x<R$. По закону 
сохранения заряда плотность тока обратно пропорциональна $x$. Тогда падение потенциала на этом малом 
слое будет равно: $\Delta \varphi=\dfrac{J \cdot \rho  \cdot \Delta x}{2 \cdot \pi \cdot x \cdot h}$.  

То есть $J=\dfrac{\Delta \varphi \cdot 2 \cdot \pi \cdot x \cdot h}{\rho  \cdot \Delta x}$.

Если полоску вдвое увеличить по длине, оставив такое же разбиение на малые слои, то этот слой удлинится 
в 2 раза, расстояние до центра увеличится в 2 раза, а значит не изменится общий ток. $J = \dfrac{N \cdot e}{t}$.

\answerMath{$N \cdot 1.6 \cdot \dfrac{10^{-1}}{t}$.}