\assignementTitle{}{10}{}

Чтобы разогнать в глубоком космосе
космическую станцию массой 10000 кг  до скорости
2 км/c, надо запастись 1000 кг горючего. Какое минимальное количества \linebreak $M$ кг топлива потребуется,
чтобы сообщить космической станции массой в $N$ кг скорость 2 км/с, 
а затем затормозить её до скорости 1 км/c? Ответ дайте в килограммах с точностью до 20 кг.

Укажите условие для заданного значения $N$ кг.

\paramSection

$N$ в пределах от 20000 до 50000, шаг 100;         

Точность ответа $M$  до  20.
\solutionSection

Чтобы разогнать космическую станцию до 2 км/с и потом затормозить до 1 км/с, требуется столько же топлива, 
сколько для разгона до 3 км/с. Пусть при разгоне станции массой 10000 кг  на 1 км/с требуется $x$ кг горючего. 
Тогда составим таблицу:
\begin{tabular}{l l l}
    Этап разгона. &	Оставшаяся общая масса  &	Требуется потратить \\
    & на конец разгона. & горючего.\\
    & & \\
    От 2 до 3 км/c & 	10000 кг  &  	$x$ кг\\
    & & \\
    От  1 до 2 км/c & 	10000 кг  + $x$ кг	 & $x \cdot \left( 1 + \frac{x}{10000}\right)$  кг\\
    & & \\
    От 0 до 1 км/c & 	10000 кг  + $x$ кг + $x \cdot \left( 1 + \frac{x}{10000}\right)$  кг	 & $x \cdot \left( 1 + \frac{2x}{10000} + \frac{x^2}{10^8}\right)$   кг\\
\end{tabular}

По данным таблицы получаем общую массу горючего для станции \linebreak массы 10000 кг:
$x \cdot \left( 1 + \frac{x}{10000}\right)^2+x \cdot \left( 1 +\frac{x}{10000}\right)+x=x  \cdot (1+q+q^2 )$;  
$q=1 +\frac{x}{10000}$ 

Однако, для разгона на 2 км/с требуется: $x  \cdot (1+q)=1000 $. Решим уравнение:
$$x^2+2 \cdot 10^4 \cdot x-10^7=0; x= -10^4+\sqrt{10^8+10^7}  \approx 488.088; q\approx 1.0488088 $$

Тогда, для разгона на 3 км/с требуется: $M=\frac{N}{10000} \cdot x \cdot (1+q+q^2 )\approx 0.15369 \cdot N.$

Аналогичный ответ можно получить используя формулу Циолковского для этой задачи.

\answerMath{$0.15369 \cdot N$.}