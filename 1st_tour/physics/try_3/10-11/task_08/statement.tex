\assignementTitle{}{10}

Для защиты  от
вспышек на Солнце используют систему светофильтров. Светофильтр поглощает $N \%$ энергии падающего на 
него света, отражает назад $10 \%$, а остальной
свет пропускает. Сколько процентов M энергии
падающего света пропускает система из двух таких светофильтров? При расчётах
учесть пропускание многократно переотражённого света. Ответ дайте в процентах с точностью до сотых.

Укажите решение для заданного значения $N$.

\paramSection

$N$ в пределах от 20 до 50, шаг 1;         

Точность ответа $M$ до  0.01.

\soultionSection

Первый фильтр пропустит $x = 100 - N - 10 = 90 - N$ процентов падающего света. Второй фильтр пропустит сразу  
$x \cdot \dfrac{90-N}{100}$ процентов энергии исходного света и  $x \cdot 0.1$ отразит назад к первому фильтру, 
который переотразит  $x \cdot 0.1^2$. Тогда второй фильтр пропустит ещё $x \cdot \dfrac{90-N}{100} \cdot 0.1^2$ и 
$x \cdot 0.1^3$ снова отразится. Таким образом, для доли пропущенного вторым фильтром света получаем:
$$M=(90-N) \cdot (1+\dfrac{90-N}{100} \cdot 0.1^2+\dfrac{90-N}{100} \cdot (0.1^2 )^2+ \cdots) .$$

Это бесконечная геометрическая прогрессия.

\answerMath{$\dfrac{(0.9-N/100)^2}{0.99 \cdot 100}$.}