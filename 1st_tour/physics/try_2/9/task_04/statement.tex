\assignementTitle{}{20}
Когда свет идёт из среды с положительным показателем преломления в
среду с отрицательным, он отклоняется назад, оставаясь по ту же сторону
нормали, что и падающий. Карандаш, опущенный в такую среду, будет казаться
изогнутым наружу. Определите  показатель преломления $n < 0$
этой среды, если максимальный угол излома карандаша $\varphi$,
поставленного из воздуха в эту среду (угол 
поворота изображения в среде относительно карандаша в воздухе) равен $N$ градусов.   Дайте ответ с точностью до десятых. 

Считайте, что все остальные оптические свойства среды с отрицательным показателем преломления такие же, как и у среды с таким же по модулю положительным показателем.

Укажите решение для заданного значения $N$.

\paramSection

$N$ в пределах от $120$ до $150$ градусов, шаг $0.10^{\circ}$.

\soultionSection

Этот угол определяется из условия полного внутреннего отражения: $1=n \cdot sin\beta$.

\putImgWOCaption{5cm}{1}

$\varphi =90^{\circ}-arcsin(\frac{1}{n});$   $\frac{1}{n}=cos(\varphi );$   $n=\dfrac{1}{cos(\varphi )}$ 

\answerMath{$\dfrac{1}{cos(x/57.325)}$.}