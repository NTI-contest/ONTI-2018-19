\assignementTitle{}{20}
Космическое агентство решило разместить на астероиде,
который ближе всего подходит к Солнцу в перигелии орбиты, космическую станцию.
Список астероидов, из которых был проведен выбор, в таблице. Какая максимальная полезная
мощность P может
быть получена при прохождении перигелия на этой станции солнечными батареями,
если площадь фотоэлементов на ней равна $S$ м$^2$,
а КПД фотоэлемента  $N\%$? 

\putImgWOCaption{13cm}{1}

1 астрономическая единица (а.е.) равна $149~597~870~700$ метров. Земной год
считать равным ровно 365 суток по 24 часа.

Мощность
солнечного света попадающего на единицу поверхности на орбите Земли (солнечная
постоянная) равна 1387 Вт/м$^2$. Гравитационную постоянную считать равной
$6.67 \cdot 10^{-11}$ м$^3$/(кг $\cdot$ с$^2$),
 для упрощения расчётов все орбиты считать
круговыми.

Эксцентриситет орбиты определяется
по формуле:
$e=\sqrt{1-b^2/a^2}$  
 , где b — малая полуось, a —
большая полуось орбиты.

Укажите решение для заданных значений $S$ м$^2$, $N \%$

\paramSection

$S$ в пределах от $100$ до $200$ м$^2$, шаг  $1$ м$^2$.

$N$ в пределах от $20$ до $40 \%$, шаг  $1 \%$.

Точность ответа  до  $1$ Вт.

\solutionSection

Малая полуось равна:  $b=a \cdot \sqrt{1-e^2 }$ , $\rho =a-\sqrt{a^2-b^2 }=a \cdot (1-e)$ –  расстояние до 
перигелия. Если смотреть табличные данные, то наименьшее оно у Гебы и равно: $\rho =1.93116$ а.е.

Мощность потока солнечных лучей на станции обратно пропорциональна квадрату расстояния до Солнца, 
поэтому: $P=1350 \cdot \left(\frac{1}{1.93116} \right)^2 \cdot S \cdot \frac{N}{100}=3.719118 \cdot S \cdot N$.

\answerMath{$3.719118 \cdot S \cdot N$.}