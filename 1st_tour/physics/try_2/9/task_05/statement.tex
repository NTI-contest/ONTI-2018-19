\assignementTitle{}{10}
Дополним мысленный эксперимент Рассела. Допустим, что где-то вокруг Солнца по круговой орбите вращается абсолютно черный сферический литровый чайник, полностью заполненный водой. Температура воды в чайнике достигла $99^{\circ}C$ и не изменяется. Оцените за какое время остыла бы на $t^{\circ}$ вода в точно таком же чайнике, находящемся на орбите радиус  которой в $k$ раз больше. Начальная температура воды во втором чайника так же 
$99^{\circ}C$ . Теплоемкостью и толщиной стенок самого чайника можно пренебречь, так же как наличием у него носика и ручек. Чайник герметично закрыт. Ответ приведите в секундах, с точностью до целых.

Известно, что абсолютно черные тела излучают энергию в окружающее пространство согласно закону Стефана-Больцмана:  $P=\sigma T^4$, где $P$ – это мощность излучаемая с единицы площади поверхности тела, $Т$ – его абсолютная температура, а $\sigma=5,67 \cdot 10^{-8}$ Вт/(м$^2 \cdot K^4$) – постоянная Больцмана.Удельная теплоемкость воды 4200 Дж/кг $\cdot$ К.

Укажите решение для заданных значений $k$, $t$.

\paramSection

$K$ от $2$ до $5$ с шагом $0,5$

$t$ от $0.2$ до $1,5$ с шагом $0,1$

Допустимая погрешность ответа 3 секунды.

\solutionSection

Мощность получаемая телом от Солнца на исходной орбите должна быть равна мощности излучения тела. 
Мощность излучения тела при остывании на 1 градус изменится незначительно 
(порядка $1-2$Вт, чтобы учесть, возможную большую точность ответа участников диапазон ответа увеличен), 
а мощность получаемого тепла упадет пропорционально квадрату $k$. 
Соответственно, искомое время $$ \tau=\frac{4200 \cdot t \cdot 1}{4\pi r^2 \sigma T^4 (1-\frac{1}{k^2} )}=\frac{4200 \cdot t}{(52.5 \cdot (1-\frac{1}{k^2}))}.$$

\answerMath{$\frac{4200 \cdot t}{52.5 \cdot (1-\frac{1}{k^2})}$.}