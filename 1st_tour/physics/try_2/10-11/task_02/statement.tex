\assignementTitle{}{10}
Необходимо узнать характеристики
акселерометра, не разбирая его. Известно, что акселерометр работает по схеме реостата. Когда возникает сила инерции,
перемычка массы $m$ грамм, закрепленная на пружинах и регулирующая реостат длиной
20 см, сдвигается влево или
вправо в зависимости от направления ускорения. В положении равновесия перемычка
расположена ровно посередине реостата.

\putTwoImg{6cm}{1}{6cm}{2}

По таблице измерений зависимости силы тока от
ускорения акселерометра найдите величину коэффициента жесткости пружин, держащих перемычку в приборе. Пружины
одинаковые. Известно, что
внутреннее сопротивление источника напряжения равно $r = 1$ Ом, а напряжение равно $E = 20$ В.  Масса
перемычки $m$.

\putImgWOCaption{13cm}{3}

Считать, что сопротивление активной части реостата прямо
пропорционально её используемой длине (от перемычки до левого края).

Пользуясь данными таблицы, определите величину $k$ с точностью до $0,2$ Н/м.

Укажите решение для заданного занчения $m$ г.

\paramSection

$m$ в пределах от $100$ до $500$ грамм, шаг $1$ грамм.  

Точность ответа  до  $0.2$  Н/м.

\soultionSection

Предположим, максимальное сопротивление реостата равно $2 \cdot R_0$. Тогда, когда перемычка находится 
ровно посередине, оно будет равно $R_0$. А при сдвиге перемычки в левую сторону на $х$ – оно изменится 
пропорционально длине: \linebreak $R(x)=R_0 \cdot (1+\dfrac{x}{0.1})$ . Здесь $0.1$ – половина длины реостата. 
Запишем второй закон Ньютона для перемычки:
$m \cdot a=2 \cdot k \cdot x;  x=\dfrac{m \cdot a}{2 \cdot k}$   . По закону Ома:   
$\dfrac{E}{R+r}$; $R=R_0 \cdot \left(1+\dfrac{m \cdot a}{2 \cdot k \cdot 0.1}\right)$; 
$I=  \dfrac{E}{r+R_0+\dfrac{R_0 \cdot m \cdot a}{2 \cdot k \cdot 0.1}}$ . 

Используя таблицу:
$$\dfrac{18.16530427}{16.65278934}=\dfrac{1+R_0+\dfrac{R_0 \cdot m \cdot 20}{2 \cdot k \cdot 0.1}}{1+R_0+\dfrac{R_0 \cdot m \cdot 10}{2 \cdot k \cdot 0.1}}; \Rightarrow  1.51251493 \cdot (1+R_0 )=\dfrac{50 \cdot m \cdot R_0}{k} \cdot 15.1402441$$
$$1+R_0 \approx   \dfrac{R_0 \cdot m}{k}   \cdot 500.5;  18.16530427 \cdot 10^{-3} \cdot (1+R_0+50 \cdot \dfrac{R_0 \cdot m}{k}=20$$
$(1+R_0 ) \cdot \left(\dfrac{1+50}{500.5}\right) \approx 1101;  R_0 \approx 1000;   \dfrac{R_0 \cdot m}{k}=2;$ 
$k \approx 500 \cdot m \cdot \text{Н/(м} \cdot \text{кг}) $  

Выражаем $m$ в граммах, поэтому делим на $1000$ в формуле:  $k \approx 0.5 \cdot m \cdot \text{Н/(м} \cdot \text{гр)}   .$

\answerMath{$0.5 \cdot m$.}