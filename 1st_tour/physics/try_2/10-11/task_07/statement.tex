\assignementTitle{}{10}
Видеокамера наблюдения крепится тремя присосками в углу
комнаты к двум стенам и потолку (стены вертикальные, потолок горизонтальный).   Каждую
присоску можно считать идеально герметичной и абсолютно плотно примыкающей к поверхности,
имеющей в прижатом состоянии форму диска радиуса R. Каков
максимальный вес камеры, которую можно цеплять, если коэффициент трения
присосок о стены и пол равен $f$, атмосферное
давление равно $10^5$ Па, ускорение свободного падения $g = 10$ м/с$^2$. Считать, что
сила присасывания перпендикулярна поверхности и зависит только от давления. Ответ дайте в ньютонах с точностью до десятых.

Укажите ответ для заданных значений $R$ мм, $f$.

\paramSection

$R$  в пределах от $20$ до $30$ мм, шаг $0,1$ мм;

$f$ в пределах от $0,3$ до $0,5$, шаг $0,01$;

Точность ответа $m_{max}$: до $0,1$ Н.

\solutionSection

Пусть $F$ –сила присасывания, $T$-реакция троса.

$mg=T+2F_{\text{тр}}$     $T=F;$     $ F_{\text{тр}} \leq fF;$   $F=p \times \pi R^2$  

$mg \leq F \times (1+2f);$     $m_{max}=\dfrac{F}{g}(1+2f)=\dfrac{p\pi R^2}{g} \times (1+2f)$

\answerMath{$\dfrac{0.1 \cdot 3.14 \cdot R^2}{10} \cdot (1+2 \cdot f)$.}