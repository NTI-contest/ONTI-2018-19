\assignementTitle{}{10}{}

Батискаф спускают на дно океана с корабля, а потом он начинает подниматься с выключенным двигателем. 
Для регулирования устоявшейся скорости всплытия сбрасывают балласт. 

В таблице показана устоявшаяся скорость движения вверх при разном количестве сброшенного балласта. 

\putImgWOCaption{13cm}{1}

В расчётах можно считать, что сила сопротивления движению состоит из двух частей – трения, пропорционального скорости, и 
аэродинамического сопротивления, пропорционального квадрату скорости, а объём батискафа не меняется при подъёме.

Сколько балласта $\Delta m$ нужно сбросить, чтобы установилась скорость $V$ м/c?  Ответ дайте в тоннах, с точностью до сотых.

Укажите условие для заданного значения $V$ м/с

\paramSection

$V = N + x$, 

$N$ в пределах от $3$ до $8$, шаг $1$, 

$x$ в пределах от $0.3$ до $0.7$, шаг $0.01$. 

Ответ $\Delta m$ с точностью до $0.01$ тонн.

\solutionSection

Устоявшаяся скорость движения означает, что ускорение равно нулю. Решение этого квадратного уравнения 
в общем виде имеет вид:  $v=-A+\sqrt{B-C \cdot m}$ .

Так как $m = m_0 - \Delta m$, где $m_0$ – начальная масса вместе с полным балластом, а при полной загрузке, 
судя по таблице, батискаф не поднимается, то можно записать:

$v=-A+\sqrt{p \cdot \Delta m-q}$, $p \cdot \Delta m=q+(v+A)^2$,   $A,p,q>0$

Используя табличные данные, можно определить все эти коэффициенты. 
Например, запишем для трёх соседних столбцов из таблицы:

\begin{equation*} 
    \begin{cases}
        p \cdot \Delta m_1=q+(v_1+A)^2\\
        p \cdot \Delta m_2=q+(v_2+A)^2 \space \Rightarrow\\
        p \cdot \Delta m_3=q+(v_3+A)^2
    \end{cases}
\end{equation*}
\begin{equation*} 
    \begin{cases}
        p \cdot \Delta m_2-\Delta m_1=v_2^2-v_1^2+2 \cdot A \cdot (v_2-v_1 ) \space \Rightarrow \\
        p \cdot \Delta m_3-\Delta m_2=v_3^2-v_2^2+2 \cdot A \cdot (v_3-v_2 )
    \end{cases}
\end{equation*}

$$0=v_3^2+v_1^2-2 \cdot v_2^2+2 \cdot A \cdot (v_3+v_1-2 \cdot v_2 )  $$
$A=-\frac{1}{2} \cdot \frac{v_3^2+v_1^2-2 \cdot v_2^2}{v_3+v_1-2 \cdot v_2}$; 
$p=\frac{v_2^2-v_1^2+2 \cdot A \cdot (v_2-v_1 )}{\Delta m_2-\Delta m_1};$ $q=p \cdot \Delta m_1-(v_1+A)^2 $

Запишем результаты, посчитанные по разным соседним тройкам столбцов:

\begin{tabular}{l l l l l}
    Столбцы & 3-4-5 & 4-5-6	& 5-6-7	& 6-7-8 \\
    A &	1.00021 & 1.0007 &	1.00053 &	0.998415 \\
    P &	1.00049 &	1.00042 &	1.00029 &	0.99934 \\
    q & 	1.50107 &	1.49959 &	1.49957 &	1.50285
\end{tabular}

Таким образом, можно положить $p = 1$, $A = 1$, $q = 1.5$, погрешность вычисления коэффициентов не 
больше $0.2 \%$ (самый худший результат получается в третьем столбце). 

Тогда $\Delta m=1.5+(V+1)^2$ – итоговая формула. 


\answerMath{$1.5+(V+1)^2$.}