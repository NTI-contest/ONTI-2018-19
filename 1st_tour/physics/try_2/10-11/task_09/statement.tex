\assignementTitle{}{10}
Два известных
германских астронома, Хеллер и Михаэль Хиппке, предложили
вариант путешествия к альфе Центавра, которое позволит микрозонду выйти
на орбиту этой тройной звездной системы и, при желании, вернуться
назад на Землю.

Ключом к реализации этой задачи является гигантский солнечный парус
площадью примерно в S = 100 тысяч
квадратных метров, изготовленный из сверхлегкого материала – графена,
волокна из нанотрубок или других "плоских" субстанций.
Такой парус сможет достичь альфы Центавра примерно за 90 лет. При
достаточно большом удалении от Солнца можно не учитывать гравитационное
притяжение в уравнении движения и учитывать только световое давление.  

Считая, что только часть  падающих на парус фотонов поглощаются парусом, а остальные отражаются зеркально назад, оцените, во сколько раз сила давления, которую оказывают на парус солнечные лучи, на расстоянии L  от Солнца, больше силы гравитационного притяжения к Солнцу.  Ответ дайте в виде числа с точностью до десятков.

Энергия Е и импульс фотона p связаны соотношением $E = c \cdot p$, где c  – скорость света, равная $3 \cdot 10^8$ м/c. Мощность
солнечного света, попадающего на единицу поверхности на орбите Земли (солнечная
постоянная) равна  $q = 1387$ Вт/м$^2$.  Силу гравитационного притяжения Солнца при решении задачи
оценивать из приближения, что Земля вращается вокруг Солнца с периодом ровно
365 суток по 24 часа по круговой орбите. Масса микрозонда вместе с парусом
равна 10 грамм. Одну астрономическую единицу принять равной ровно 150 миллионов
километров.

Доля фотонов поглощенных парусом $x\%$, расстояние от зонда до Солнца $L$ а.е.

\paramSection

$N$ в пределах от $1000$ до $5000$, шаг  $1$.  

$x$ в пределах от $80$ до $100 \%$, шаг  $1 \%$.  

Точность ответа  до  $10$.

\soultionSection

Сначала найдём силу гравитационного притяжения. Она обратно пропорциональна квадрату расстояния до 
Солнца, а на расстоянии, равном одной астрономической единице, она равна центробежной силе, то есть:  
$$F_{\text{солн}}=\dfrac{m}{N^2}  \cdot \dfrac{v^2}{R}=m \cdot \omega ^2 \cdot R=\dfrac{4 \cdot \pi^2 \cdot R \cdot m}{T^2 \cdot N^2}.$$

Теперь найдём силу давления солнечного света. Мощность солнечного света, попадающего на единицу поверхности паруса, 
обратно пропорциональна квадрату расстояния до Солнца.  Если фотон поглощается, то по закону сохранения импульса импульс фотона прибавляется к солнечному парусу, а если зеркально отражается – то прибавляется удвоенный импульс фотона:
$$\Delta \dfrac{E}{c} \left( \dfrac{x}{100}+2 \cdot \left(1-\dfrac{x}{100} \right) \right)=\dfrac{E}{c} \left(2-\dfrac{x}{100}\right); F_{\text{свет}}=  \dfrac{\Delta p}{\Delta t}=\dfrac{q \cdot S}{c \cdot N^2} \left(2-\dfrac{x}{100}\right) $$

Тогда получаем:
  $$F_{\text{свет}}/F_{\text{солн}} = \dfrac{q \cdot S \cdot T^2}{4 \cdot c \cdot \pi^2 \cdot R \cdot m} \left(2-\dfrac{x}{100}\right)=\dfrac{1387 \cdot 10^5 \cdot (365 \cdot 24 \cdot 3600)^2}{4 \cdot 3 \cdot 10^8 \cdot 3.14159^2 \cdot 1.5 \cdot 10^{11} \cdot 0.01} \cdot$$  
  $$\cdot \left(2-\dfrac{x}{100}\right) \approx 7764.58 \cdot (2-\dfrac{x}{100})$$

\answerMath{$7764.58 \cdot (2-\dfrac{x}{100})$.}