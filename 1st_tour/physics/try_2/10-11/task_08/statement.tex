\assignementTitle{}{10}
Солнечные электростанции башенного типа состоят из башни
коллектора и группы зеркал гелиостатов, которые непрерывно поворачиваются так,
чтобы солнечный свет отражался на башню коллектор. Пусть наша электростанция
состоит из $N$ зеркал, каждое с площадью $S$ м$^2$, будем считать эти зеркала идеальными и отражающими
на одну и ту же плоскую поверхность так, что в каждый момент солнечные зайчики
от всех зеркал совпадают. Оцените во сколько раз освещенность плоской
поверхности в коллекторе отличается от максимальной освещенности, которую можно
получить от Солнца с помощью лупы диаметром $d$ см, если фокусное расстояние лупы 2 см. Найдите $E_{\text{зеркал}}/E_{\text{лупы}}$ с точностью
до десятых.

Считайте, что
свет от зеркал не рассеивается и не поглощается при прохождении расстояния до
коллектора. Угловой размер Солнца – 0,01 рад.

Укажите решение для заданных значений $N$, $S$ м$^2$, $d$ см.

\paramSection

$N$ меняется от $10000$ до $10000$ с шагом в $1000$

$S$ от $1$ до $10$ с шагом в $1$ м$^2$

$d$ от $0,05$ до $0.1$ с шагом в $0,01$

\solutionSection

Диаметр изображения Солнца в фокальной плоскости линзы равен 
$D = F \alpha$, где  $\alpha$ – угловой размер Солнца. Тогда освещенность изображения, 
создаваемого линзой равна: $E_{\text{лупы}} = E \cdot \frac{d^2}{F^2 \cdot \alpha^2}$, где $E$ – освещенность 
поверхности зеркала или лупы.

Считая, что все отражение от зеркала попадает на поверхность, освещенность поверхности от $N$ зеркал будет 
равна $E_{\text{зеркал}} = N \cdot E$

Отсюда ответ:
		$K = N \cdot 4 \cdot 0.0001/d^2$


\answerMath{$N \cdot 4 \cdot 0.0001/d^2$.}