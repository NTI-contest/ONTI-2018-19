\assignementTitle{}{10}
Для определения положения облака, из которого ударила
молния, под землей закопали датчик звука на глубине $h=5$ метров. Датчик 
зафиксировал две звуковые волны – первая пришла через время $t_1$ после
вспышки, вторая через время $t_2$.  На какой
высоте $H$ находится это облако,
если скорость звука в почве равна \linebreak $v_1 = 600$ м/сек, а скорость
звука в воздухе $v_2 = 330$ м/сек? В расчётах
считать, что $h << H$, $h << L$, $L >  H$, где L расстояние по горизонтали от датчика до точки находящейся непосредственно под местом откуда ударила молния. Первая
звуковая волна пришла от места удара молнии о землю, вторая – из облака. Ответ дайте в метрах с точностью до десятков.

\putImgWOCaption{6cm}{1}

Укажите ответ для заданных параметров $ t_1$, $t_2$.

\paramSection

$t_1$ в пределах от $20$ до $24$ секунд, шаг  $0.1$ сек.  

$t_2$ в пределах от $25$ до $30$ секунд, шаг  $0.1$ сек.

Точность ответа  до $10$ метров.

\soultionSection

Согласно принципу Ферма в оптике, свет движется из начальной точки в конечную по пути, 
соответствующем минимальному времени движения. Данный принцип можно применить для решения данной задачи, 
чтобы понять, как будет распространяться звук.

$$t_1=\sqrt{\dfrac{h^2+L^2}{v_1} )}; \space h^2+L^2=v_1^2 t_1^2$$
$h<<H$, значит можно считать $\beta  \approx 90^{\circ}$. $sin\beta  \approx 1$, следовательно,  
$sin \alpha  \approx \dfrac{v_2}{v_1}$ 
$$t_2=\dfrac{H}{v_2  \times cos \alpha}+\dfrac{L-H \times tg \alpha }{v_1}; 
cos \alpha =\sqrt{1-\left(\dfrac{v_2}{v_1}\right)^2}$$
$h<<L$, значит $L \approx v_1 t_1$. $t_2 v_2 \times cos \alpha =H+L \dfrac{v_2}{v_1} \times cos \alpha -H \dfrac{v_2}{v_1} \times sin \alpha$  
$H \times \left(1-\left(\dfrac{v_2}{v_1} \right)^2 \right)= \linebreak = v_2 \times cos \alpha  \times (t_2-t_1 )$

$$H=\dfrac{v_1 \times v_2 \times (t_2-t_1 )}{\sqrt{v_1^2-v_2^2}}$$

\answerMath{$\dfrac{600 \cdot 330 \cdot (t_2-t_1)}{\sqrt{600 \cdot 600-330 \cdot 330}}$.}