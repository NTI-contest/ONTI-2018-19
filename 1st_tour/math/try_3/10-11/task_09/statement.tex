\assignementTitle{}{10}{}

В течение пяти часов 1 сентября Женя наблюдал за воздушными шариками в небе. По мере того, как утренние линейки проходили, шаров в небе становилось все меньше -- так, с каждым часом за час пролетало не больше шаров, чем в предыдущий час. Суммарно Женя насчитал $100$ шаров, пролетевших в небе мимо его окна. Причем, суммарно за второй и четвертый час Женя увидел не больше шаров, чем за первый и третий. Какое минимальное число шаров Женя мог увидеть суммарно за $1$, $3$ и $5$ часы?

\solutionSection

Чтобы увидеть минимальное число шаров суммарно за $1$, $3$ и $5$ часы, надо увидеть максимальное число шаров за $2$ и $4$ часы. Если суммарно за второй и четвертый час Женя увидел не больше шаров, чем за первый и третий, то максимизируя число шаров за $2$ и $4$ часы, в $1$ и $3$ Женя увидит столько же. За $5$ час Женя может не увидеть ни одного шара. При этом остальные шары могут быть распределены равномерно по оставшимся часам. Минимум суммы увиденных шаров за $1$, $3$ и $5$ часы равен 50.


\answerMath{50.}