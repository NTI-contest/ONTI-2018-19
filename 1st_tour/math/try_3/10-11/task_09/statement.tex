\assignementTitle{}{10}{}

В течение пяти часов 1 сентября Женя наблюдал за воздушными шариками в небе. По мере того, как утренние линейки проходили, шаров в небе становилось все меньше -- так, с каждым часом за час пролетало не больше шаров, чем в предыдущий час. Суммарно Женя насчитал $100$ шаров, пролетевших в небе мимо его окна. Причем, суммарно за второй и четвертый час Женя увидел не больше шаров, чем за первый и третий. Какое минимальное число шаров Женя мог увидеть суммарно за $1$, $3$ и $5$ часы?

\solutionSection

Пусть $a$, $b$, $c$, $d$, $e$ — количество шаров в $1$, $2$, $3$, $4$ и $5$ часы соответственно. $a\geq  b \geq c \geq d \geq e$, $a + b + c + d + e = 100$, $b + d \leq a + c$, требуется найти $min(a + c + e)$. Данная задача сводится к поиску $max (b + d)$. Так как $b + d \leq a+c$, а $a + b + c + d + e = 100$, то $b + d \leq 100/2 = 50$. Такой случай легко придумать: $a,b,c,d,e = 25,25,25,25,0$. Таким образом, $min (a + c + e) = 100 - max(b + d) = 50$.

\answerMath{50.}