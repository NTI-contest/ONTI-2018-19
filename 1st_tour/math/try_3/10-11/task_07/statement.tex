\assignementTitle{}{10}

Если из резервуара выливают воду, уровень воды $H$ в нём меняется в зависимости от времени $t$ следующим образом: $H(t)=at^2+bt+c$. Пусть $t_0$ — момент окончания слива. Известно, что в этот момент выполнены равенства $H(t_0)= H'(t_0)=0$. В течение какого времени вода из резервуара будет полностью вылита, если за первый час слилась половина уровня? 

Округлите ответ до ближайшего целого.

\solutionSection

$$H(t_0) = at_0^2 + bt_0 + c = 0,$$
$$H'(t_0) = 2at_0 + b = 0.$$

Выразим $b$ и $c$ через $a$ и $t_0$:
$$b = -2at_0,$$
$$c = at_0^2.$$

Пусть $\tau$ -- время достижения половинного уровня.
$$2H(\tau) = H(0),$$
$$2(a\tau^2 + (-2at_0)\tau+at_0^2) = at_0^2,$$
$$2\tau^2 - 4t_0\tau + t_0^2 = 0,$$
$$D = 16t_0^2 - 4\cdot 2t_0^2 = 8t_0^2,$$
$$\tau = \frac{4t_0\pm 2\sqrt{2}t_0}{2\cdot 2}=t_0\pm t_0/\sqrt{2}.$$

Так как $\tau < t_0$ из условия, то $\tau = t_0\cdot(2 -\sqrt{2})/2$, то есть $t_0 = \tau(2 + \sqrt{2})$, то есть $[t_0] = [1\cdot(2 + \sqrt{2})] = 3.$

\answerMath{3.}