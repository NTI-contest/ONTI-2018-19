\assignementTitle{}{10}

Артур и Саша играли в игру — по очереди выписывали натуральные числа на бумагу. 
В итоге оказалось, что на бумаге выписано 15 чисел, причем, наименьшее из чисел можно представить как 
$x+1$, $x > 1$, а все остальные числа — последовательность $(1+x^n)$ , где $n$  — натуральный показатель степени, 
изменяющийся от 2 до 15. Артуру показалось, что выписанных чисел слишком много и он зачеркнул часть 
из них таким образом, чтобы все оставшиеся на бумаге числа были взаимно простыми. 
Какое наименьшее количество чисел мог зачеркнуть Артур?

\soultionSection

\answerMath{11.}