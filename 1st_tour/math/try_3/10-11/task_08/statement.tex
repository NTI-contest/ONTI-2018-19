\assignementTitle{}{10}

Ира и Паша расставляют стулья вокруг круглого стола. После того, как все стулья были расставлены, ребята решили их пересчитать — они начали ходить по кругу в одном направлении, но начиная с разных стульев. Известно, что стул, который Паша посчитал седьмым, у Иры оказался под двадцатым номером, а тот стул, который Ира посчитала седьмым, у Паши был 94 м. Сколько стульев было расставлено вокруг стола?

\solutionSection

\putImgWOCaption{4cm}{1}

Считаем по обеим дугам количество стульев, включая один из двух крайних: $20 - 7 + 94 - 7 = 100$.

\answerMath{100.}