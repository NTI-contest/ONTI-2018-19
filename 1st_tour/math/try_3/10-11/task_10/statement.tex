\assignementTitle{}{10}{}

На столе стоят песочные часы высоты $16$ см, представляющие собой два соединенных усеченных одинаковых конуса. Радиус горлышка (отверстия, через которое сыпется песок) равен 1 см. Тангенс угла раствора конусов равен $4/3$. Чему равен объем песочных часов в см$^3$? 

Ответ округлите до ближайшего целого.

\solutionSection

\putImgWOCaption{8cm}{1}

Рассмотрим один из конусов.

\putImgWOCaption{8cm}{2}

$$\tg^2\alpha + 1 = \frac{1}{cos^2\alpha},$$
$$cos\alpha = \frac{3}{5}.$$ 

Пусть $OC = O'C = x$, по теореме косинусов: 
$OO'^2 = x^2 + x^2 - 2x\cdot x\cos\alpha = \frac{4x^2}{5},$
$x = \sqrt{5}.$

Опустим перпендикуляр $CO''$ к $OO''$. 
$$CO'' = \sqrt{x^2 - {OO''^2}} = \sqrt{5 - 1} = 2$$ 

\putImgWOCaption{8cm}{3}

Из подобия треугольников $BB'' = 5$. Объем песочных часов, т.е. удвоенный объем усеченного конуса, равен $$V=\frac{2}{3}\pi O''B''(OO''^2 + OO''BB'' + BB''^2)=\frac{2\pi}{3}\cdot 8 \cdot(1 + 5 + 25) = 519.409 \approx 519.$$

\answerMath{519.}