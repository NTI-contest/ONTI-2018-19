\assignementTitle{}{10}

Дан треугольник $ABC$ и $H$ — точка пересечения высот этого треугольника. 
Пусть $D$ — середина отрезка $BC$, $E$ — середина отрезка $АС$. 
Кроме того, медианы треугольника $AED$ пересекаются в точке $H$. Найдите градусную меру угла $\angle ABC$. 

Ответ укажите с точностью до десятитысячных.

\solutionSection

\putImgWOCaption{6cm}{1}

Рассмотрим $\triangle ABC$: $DE, EF, DF$ -- медианы, $AA', BB'$ -- высоты. $BD = DC$. В $\triangle ADE$: медианы пересекаются в точке $H$, которая разделяет их в отношении $2:1$. Из подобия треугольников очевидно, что $DA':A'C = 1:2$. 

Пусть $AD = x$, $A'C = 2x$. $FH = 2 \cdot (HE + HE/2) - HE = 2HE = 2x$ По свойству паралелограмма, образованного медианами $FE$ и $BC$, перпендикулярами $AA'$ и $FF'$: $FH = F'A' = 2x$ и $FF'=HA'$, $F'D = x$. Также из подобия треугольников $BF' = 2x$, $FF'=HA'=AH$. 

Рассмотрим $\triangle BA'H\sim\triangle AHE$(по 2 углам). $A'B:A'H=A'H:HE \Rightarrow A'H = \sqrt{A'B\cdot HE} = \sqrt{4x^2} = 2x = EF'=BF'$. 

Так, $\triangle BF'F$ -- прямоугольный равнобедренный, и угол при основании $\angle FBF' = 45^\circ$.\\

\answerMath{45.}