\assignementTitle{}{10}

Паша загадал число $x$: это неправильная дробь с натуральным числителем и со знаменателем, равным 9. Далее он вычислил еще три числа умножил $x$ на $5$, на $2$ и на $4$. Затем округлил эти три числа по правилам округления до целого и сложил между собой, в итоге он получил 120. Каким был числитель у неправильной дроби?

\solutionSection

Найдем первое значение для перебора: $\frac{5x+2x+4x}{9}\approx120 \Leftrightarrow x \approx 98$. Проверим значение: $$\left[\frac{5\cdot98}{9}\right] + \left[\frac{2\cdot98}{9}\right] + \left[\frac{4\cdot98}{9}\right] = 54 + 22 + 44 = 120.$$

\answerMath{98.}