\assignementTitle{}{20}

Артур и Саша играли в игру — по очереди выписывали натуральные числа на бумагу. В итоге оказалось, что на бумаге выписано 15 чисел, причем, наименьшее из чисел можно представить как $x+1$, $x > 1$, а все остальные числа — последовательность $(1+x^n)$ , где $n$  — натуральный показатель степени, изменяющийся от 2 до 15. Артуру показалось, что выписанных чисел слишком много и он зачеркнул часть из них таким образом, чтобы все оставшиеся на бумаге числа были взаимно простыми. Какое наименьшее количество чисел мог зачеркнуть Артур?

\solutionSection

Все натуральные числа вида $(1 + x^n)$, где $n$ -- нечетное число, делятся нацело на $(1+x)$.

Таким образом, вычеркнем все числа, где $n$ = 1, 3, 5, 7, 9, 11 и 13. 15 оставляем.

Далее заменим $x^2$ на $y$. По аналогии вычеркнем числа, где n = 2, 6, 10. 14 оставляем.

Далее заменим $x^3$ на $z$. $n$ = 3, 9 -- вычеркнуты. 15 оставляем. 

Далее заменим $x^4$ на $w$. По аналогии вычеркнем число, где $n$ = 4. 12 оставляем.

Далее заменим $x^5$ на $v$. $n$ = 5 вычеркнуто. 15 оставляем. 

Далее нет смысла перебирать, так как $3n > 15$. Оставшиеся числа взаимнопростые. Таким образом, минимально мы вычеркнули 11 чисел (1, 2, 3, 4, 5, 6, 7, 9, 10, 11, 13).

\answerMath{11.}