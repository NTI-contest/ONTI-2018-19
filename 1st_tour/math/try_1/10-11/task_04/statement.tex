\assignementTitle{}{10}

На уроке геометрии нарисовали окружность. На дуге $BC$ этой окружности, описанной около равностороннего треугольника $ABC$, 
взята произвольная точка $P$. Выразите отрезок $AP$ через отрезки $ВР$ и $СР$. 

Укажите длину $AP$, если $BP = 3$, $CP = 4$. Ответ введите с точностью до десятитысячных.

\solutionSection

Так как $\triangle ABC$ -- равносторонний, то $\angle BAC = \angle ABC = 
\angle ACB = 60^{\circ}$ и \linebreak $AB=AC$. $\angle ABC = \angle APC$, так как опираются на $\smile AC$. $\angle ACB = \angle APB$, так как опираются на $\smile{AB}$. Таким образом, $\angle APB = \angle APC = 60^{\circ}$. По теореме косинусов 

\begin{equation*} 
    \begin{cases}
    AB^2 = AP^2 + BP^2 - 2 \cdot AP\cdot BP\cdot \cos(\angle APB),\\
    AC^2 = AP^2 + CP^2 - 2 \cdot AP\cdot CP\cdot \cos(\angle APC).
    \end{cases}
\end{equation*}

\putImgWOCaption{5cm}{1}

Так как $AB = AC$ и $\angle APC = \angle APB = 60^{\circ}$, получим:

$$AP^2 + BP^2 - AP\cdot BP\cdot = AP^2 + CP^2 - AP\cdot CP,$$
$$BP^2 - CP^2 = AP\cdot (BP - CP),$$
$$AP = BP - CP.$$

Подставим значения $BP = 3$ и $CP = 4$ и получим $AP = 7$.

\answerMath{7.}