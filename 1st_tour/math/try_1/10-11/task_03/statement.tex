\assignementTitle{}{10}{}

В химико-биологическом классе 25 учащихся. Для дежурства по школе всегда наугад выбирают двоих. Вероятность того, что оба дежурных окажутся мальчиками, равна $3/25$. Сколько в классе девочек?

\solutionSection

Пусть N -- количество учащихся в классе, А -- количество юношей в классе. Вероятность того, что оба дежурных окажутся мальчиками составляет 
$$P = \frac{A}{N}\cdot \frac{A - 1}{N - 1}.$$

Подставим из условия $N = 25$, $P = 3/25$ и получим следующее уравнение:
$$\frac{A}{25}\cdot\frac{A - 1}{24} = \frac{3}{25},$$
$$A^2 - A - 3 \cdot 24 = 0.$$

Уравнение имеет только один положительный корень $A = 9$. Значит, девочек в классе $25 - 9 = 16$.

\answerMath{16.}