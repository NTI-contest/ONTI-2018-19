\assignementTitle{}{10}

У Лады на прикроватной тумбочке стоят часы с циферблатом. Они показывают текущее время суток от 00.00.00 до 23.59.59. Однако, сосед Дима решил перепрошить часы, и теперь, если на часах должны загореться ровно четыре цифры 3, циферблат перестает гореть. Сколько времени в течение суток часы не показывают время, если всё остальное время они работают корректно? Ответ укажите в секундах.

\solutionSection

Каждая комбинация цифр на циферблате отображается ровно $1$ минуту, следовательно в задаче требуется найти количество соответствующих комбинаций. В группе цифр, отображающей часы может встретиться только $0$ или $1$ тройка. В остальных -- $0$, $1$ или $2$. Четыре тройки можно получить из следующих комбинаций: $X_1 = Q([0]:[2]:[2])$, $X_2 = Q([1]:[1]:[2])$, $X_3 = Q([1]:[2]:[1])$. $Q$ -- количество комбинаций в соответствии с количеством троек в группах цифр циферблата, соответствующих часам, минутам, секундам. $X_1 = 21 \cdot 1 \cdot 1$. $X_2 = 3 \cdot 14 \cdot 1$. $X_3 = 3 \cdot 1 \cdot 14$. $X = X_1 + X_2 + X_3 = 21 + 42 + 42 = 105$.

\answerMath{105.}