\assignementTitle{}{10}

Коптер летит над поверхностью огромного поля. 

В какой-то момент времени он оказывается в точке $(0,3,6)$ в заданной ортогональной системе координат с осями $Ox, Oy$ и $Oz$. 

Найдите расстояние от коптера до земли, если в той же системе координат поле можно считать плоскостью, заданной уравнением $2x+4y-4z-6=0$. Ответ укажите с точностью до десятитысячных.

\solutionSection

Расстояние между точкой с координатами $(x_0, y_0, z_0)$ и плоскостью, задаваемой уравнением $Ax + By + Cx + D = 0$ вычисляется по формуле:
$$S = \frac{|Ax_0+By_0+Cz_0 + D|}{\sqrt{A^2+B^2+C^2}}$$

Для заданных точки и плоскости $$S = \frac{|2 \cdot 0 + 4 \cdot 3 - 4 \cdot 6 - 6|}{\sqrt{2^2 + 4^2 + 4^2}} = 3.$$

\answerMath{3.}