\assignementTitle{}{20}

Маша и Андрей, будущие математики, развлекались на перемене. Маша написала на доске 4 различных натуральных числа. Андрей выписал значения наибольших общих делителей для каждой из шести пар чисел. Получилось, что для одной из пар НОД  равен 1, для другой — 2, для третьей — 3, для четвертой — 4, для пятой — 5, а для шестой — $X$. Найдите наименьшее возможное значение $X$?

\solutionSection

Пусть на доске записаны 4 числа $a$, $b$, $c$ и $d$. Пусть НОД$(a, b) = 2$, тогда НОД$(c, d) = 4$ быть не может, так как НОД всех пар будет кратным 2. Значит, пусть НОД$(a, c) = 4$ и тогда $d$ будет нечётным числом.

Исходя из записанных равенств, можно выписать следующие:
$$a = 4\cdot a_4 = 2 \cdot a_2$$
$$b = 2\cdot b_2$$
$$c = 4\cdot c_4$$

При этом НОД$(a_2, b_2) = 1$ и НОД$(a_4, c_4) = 1$. Очевидно также, что НОД$(b, c) = 2 \cdot x$ -- то есть это последняя искомая пара. Попробуем подобрать значения $a$, $b$, $c$ и $d$ так, чтобы они удовлетворяли всем равенствам и из НОД оставшихся пар равнялись указанным значениям. Например, $a = 4$, $b = 10$, $c = 12$ и $d = 15$. Таким образом, наименьшее возможное значение равно 2.

\answerMath{2.}