\assignementTitle{}{20}

Дан некоторый острый угол $\alpha = 60^{\circ}$. На одной из его сторон отмечены точки $A_1$ и $A_2$, 
на другой стороне отмечена точка $B$. 

Вершина угла — $H$. Известно, что  $HA_1=2$, $A_1A_2=8$. 
При какой величине отрезка $HB$ величина острого угла между прямыми $A_1B$ и $A_2B$  будет максимальна? 
Ответ введите с точностью до десятитысячных.

\soultionSection

Обозначим $\alpha = \angle BAA_2=60^{\circ}$, $\beta = \angle A_1BA_2$, $x = BH$. 

\putImgWOCaption{5cm}{1}

По теореме косинусов выразим $A_1B$ и $A_2B$ и подставим значения $A_1 = 2B$ и $A_1A_2 = 8$ из условия:
$$A_1B^2 = x^2 + A_1H^2 - 2xA_1H\cdot \cos\alpha = x^2 + 4 - 2x,$$
$$A_2B^2 = x^2 + A_2H^2 - 2xA_2H\cdot \cos\alpha = x^2 + 100 - 10x.$$

Рассмотрим $\triangle A_1BA_2$, по теореме косинусов:
$${A_1A_2}^2 = A_1B^2 + A_2B^2 - 2A_1B\cdot A_2B \cos\beta.$$
Выразим $\cos\beta$:
$$\cos\beta = \frac{A_1B^2 + A_2B^2 - {A_1A_2}^2}{2A_1B\cdot A_2B} = \frac{x^2 + 4 - 2x + x^2 + 100 - 10x - 64}{2\sqrt{x^2 + 4 - 2x} \cdot \sqrt{x^2 + 100 - 10x}} =  $$
$$=\frac{x^2 - 6x + 20}{\sqrt{x^2 + 4 - 2x} \cdot \sqrt{x^2 + 100 - 10x}}.$$

Для максимизации острого угла $A_1BA_2$ требуется, найти $x$, при котором достигается $min(\cos\beta)$. Решим уравнение $(\cos\beta)'_x = 0$ и найдем точку минимума. $x = 2\sqrt{5} \approx 4.472135955$.

\answerMath{4.472135955.}