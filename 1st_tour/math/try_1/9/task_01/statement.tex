\assignementTitle{}{20}{}

Гоша взял у друга 11 гаек M6 (ГОСТ 5916-70) и положил в карман рюкзака. Согласно ГОСТу 1  гайка M6 весит 1.254 грамма. И вот незадача, придя домой, Гоша насчитал в кармане 12  внешне одинаковых гаек! Одна из них была из того набора, что когда-то был куплен на блошином рынке, и, по его личному опыту, такие гайки имеют меньший вес, около грамма, а также сами по себе более низкого качества менее прочные. 

У Гошиного папы есть весы, состоящие из двух больших чаш на двух концах рычага. За какое минимальное количество взвешиваний можно найти ту самую низкокачесвенную гайку?

\solutionSection

Если мы имеем $n$ внешне одинаковых объектов, после одного взвешивания останется в худшем случае $\lceil n \rceil$ объектов, среди которых есть отличающийся по весу. Таким образом, если $n = 12$, то, в худшем случае, понадобятся 3 взвешивания $12\Rightarrow 4\Rightarrow 2\Rightarrow 1$.

\answerMath{3.}