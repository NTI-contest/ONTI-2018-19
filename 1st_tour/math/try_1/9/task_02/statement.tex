\assignementTitle{}{20}{}

Витя приклеивал цифры номера квартиры на дверь. Этот номер состоит из трех цифр. В процессе приклеивания Вите пришла необычная мысль, что если в номере квартиры поменять местами две последние цифры и сложить получившееся число с исходным, то получится номер его школы! Юноша учился в школе 1187. Найдите все такие номера квартир, и если Витя живет в квартире с наименьшим из них, то в какой квартире он живет?

\solutionSection

Пусть номер квартиры равен $\overline{abc} = 100\cdot a + 10\cdot b + c$, где $a$, $b$ и $c$ -- цифры числа. Число с перевернутыми двумя последними цифрами при этом будет равно $\overline{acb} = 100\cdot a + 10\cdot c + b$. Затем решим уравнение в целых числах 
$$100 \cdot a + 10\cdot b + c + 100\cdot a + 10\cdot c + b = 1187,$$ 
$$200 \cdot a + 11\cdot b + 11\cdot c = 1187$$ с учетом ограничений на то, что $a$, $b$ и $c$ -- цифры в десятичной системе счисления.
$$200\cdot a + 11\cdot (b + c) = 200 \cdot 5 + 11 \cdot 17.$$

Таким образом, возможные номера квартир $589$ и $598$. Выбираем наименьший -- 589.

\answerMath{589.}