\assignementTitle{}{20}

На доске записаны 5 чисел: сначала некоторое рациональное $a=n/y$ ($n$ и $y$ натуральные взаимно простые числа), затем $x$ и далее $x+2$, $x+3$ и $x+4$. При каком наименьшем значении $a$ произведение всех пяти чисел всегда будет натуральным для любого натурального $x$? В ответе напишите целое число $y$.

\solutionSection

В данной задаче надо проверить делимость произведения чисел. $x\cdot (x + 2)\cdot(x + 3) \cdot \linebreak \cdot (x + 4)$ гарантированно делится на $2$ и на $3$, так как содержит произведение последовательно идущих $3$ чисел. Делимость на остальные числа при любом $x$ гарантировать нельзя.

\answerMath{6.}