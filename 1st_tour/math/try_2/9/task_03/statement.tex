\assignementTitle{}{20}{}

Коля -- очень любознательный юноша. Он решил провести исследование. Для различных действительных чисел $a$ он решил найти такое наибольшее целое \linebreak число $x$, чтобы выполнялось следующее: 

\begin{enumerate}
    \item $a$ лежит в интервале $(1,2)$,
    \item $a^2$  лежит в $(2,3)$,
    \item $a^3$  лежит в $(3,4)$,
    \item и так далее до показателя степени $x$.
\end{enumerate}

Помогите Коле выяснить, каким же может быть максимальное значение \linebreak числа $x$, при котором существует хотя бы одно значение $a$, удовлетворяющее условиям?

\solutionSection

В данной задаче необходимо найти такое натуральное максимальное $n$, при котором $\exists a \forall x \in N: x < n \Rightarrow x < a^x < x + 1$. Выпишем границы интервалов $(\sqrt[x]{x}, \sqrt[x]{x + 1})$ для нескольких первых значений $x$.\\

\begin{tabular}{l l l}
    x & \text{Левая граница a} & \text{Правая граница a} \\
    $1$ & $1$ & $2$ \\
    $2$ & $\sqrt[2]{2} = 1.41421356$ & $\sqrt[2]{3} = 1.73205081$\\
    $3$ & $\sqrt[3]{3} = 1.44224957$ & $\sqrt[3]{4} = 1.58740105$\\
    $4$ & $\sqrt[4]{4} = 1.41421356$ & $\sqrt[4]{5} =  1.49534878$\\
    $5$ & $\sqrt[5]{5} = 1.37972966$ & $\sqrt[5]{6} = 1.43096908$\\
    \dots & \dots & \dots \\
\end{tabular}\\

По таблице видно, что $\sqrt[5]{6} < \sqrt[3]{3}$. Таким образом, $n = 4$ -- наибольшее значение, удовлетворяющее условию задачи.

\answerMath{4.}