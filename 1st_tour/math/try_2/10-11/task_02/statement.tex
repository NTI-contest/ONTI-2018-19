\assignementTitle{}{10}{}

Мотоциклист поднимается на холм. Его движение в ортогональной системе координат $xOy$ можно описать законом $y = ax^2+bx+c$, где $a$ и $b$ -- некоторые неизвестные постоянные коэффициенты. Известно, что во время своего движения мотоциклист побывал в точках с координатами $(0,1)$, $(1,2)$, $(2,1)$. Найдите координаты вершины холма. Ответ укажите в формате "$(x, y)$"\, где $x$ и $у$ — значения абсциссы и ординаты с точностью до десятитысячных.

\solutionSection

Подставим точки в уравнение и решим систему:

\begin{equation*} 
    \begin{cases}
    1 = a\cdot 0^2 + b\cdot 0 + c,\\
    2 = a\cdot 1^2 + b\cdot 1 + c,\\
    1 = a\cdot 2^2 + b\cdot 2 + c;
    \end{cases}
\end{equation*}

\begin{equation*} 
    \begin{cases}
    c = 1,\\
    a + b = 1,\\
    2a + b = 0;
    \end{cases}
\end{equation*}

\begin{equation*} 
    \begin{cases}
        a = -1,\\
        b = 2, \\
        c = 1. \\
    \end{cases}
\end{equation*}

Уравнение параболы с вычисленными коэффициентами $y = -x^2 + 2x + 1$. Найдем абсциссу точки перегиба: 
$$y'_x=0 \Leftrightarrow -2x + 2 = 0 \Leftrightarrow x = 1. $$

При x = 1 y = 2.

\answerMath{$(1,2)$.}