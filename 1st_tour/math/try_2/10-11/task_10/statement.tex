\assignementTitle{}{10}{}

Соня, Андрей и Егор живут в домах $A$, $B$, $C$ соответственно. Эти дома соединены прямыми улицами -- Cадовой, Огородной и Персиковой. Известно, что Садовая и Персиковая улицы пересекаются у дома Сони под углом $45$ градусов, Персиковая и Огородная -- под углом $60$ градусов у дома Егора и, наконец, Огородная и Садовая пересекаются под окнами у Андрея. Равноудаленно от домов $B$ и $C$ внутри треугольника $ABC$ построили магазин. Известно, что прямая улица, которая соединяет магазин и дом Егора, пересекается с Персиковой под углом $15$ градусов. Между домом Сони и магазином также есть прямая улица. Под каким углом она пересекается с Садовой?

Ответ укажите в градусах с точностью до десятитысячных.

\solutionSection

Обозначим искомый угол буквой $\alpha$. 

\putImgWOCaption{7cm}{1}

$\angle BAC = 45^\circ \Rightarrow \angle OAC = 45^\circ - \alpha$. 
$\angle OCA = 15^\circ \Rightarrow \angle AOC = 120^\circ + \alpha$. \linebreak
$\angle OCB = \angle OBC = 60^\circ - 15^\circ = 45^\circ \Rightarrow \angle BOC = 90^\circ$. 
$\angle ABO = 30^\circ$. По теореме синусов:
$$\frac{\sin\alpha}{OB} = \frac{\sin 30^\circ}{AO} \Rightarrow AO = \frac{OB}{2\sin\alpha}$$
$$\frac{\sin 15^\circ}{OA} = \frac{\sin (45^\circ - \alpha)}{OC} \Rightarrow AO = \frac{OC\sin 15^\circ}{\sin (45^\circ - \alpha)}$$ \\

Так как $OB = OC$, 
$$\sin (45^\circ - \alpha) = 2\sin\alpha \sin 15^\circ \Rightarrow \alpha = 30^\circ.$$

\answerMath{30.}