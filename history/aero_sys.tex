\chapter*{История ОНТИ}


\section*{Профиль <<Аэрокосмические системы>>}

Современная аэрокосмическая робототехника требует широкого кругозора в различных отраслях науки и производства, имеющих непосредственное отношение к задачам проектирования, создания, программирования, администрирования и эксплуатации роботов. Таким образом для отбора участников олимпиады по профилю «Аэрокосмические системы» проводилось комплексное тестирование компетенций и фундаментальных знаний из разных областей науки.

В первом отборочном этапе участники решали задачи по физике и информатике по темам, знания о которых будут востребованы во втором и заключительном этапах олимпиады.

Второй отборочный этап содержал следующие задания:
\begin{enumerate}
    \item Программирование платформы Arduino - как пример задач по программированию микроконтроллеров управляющих всеми системами аэрокосмических аппаратов.
    \item Задачи по 3Д-моделированию - как пример широкого спектра задач по проектированию и созданию частей и элементов всех типов устройств, использующихся в аэрокосмической отрасли.
    \item Задачи по администрированию и программированию в среде Robot Operating System (ROS). Данная мета-операционная система является стандартом де-факто в робототехнике и широко используется ведущими аэрокосмическими компаниями и агентствами для прототипирования и создания собственных робототехнических комплексов.
\end{enumerate}

В заключительном этапе индивидуальный предметный тур проводился по дисциплинам физика и информатика. Данные дисциплины содержат наибольшее количество компетенций используемых специалистами аэрокосмической отрасли.

В командном туре заключительного этапа задача была направлена на создание и испытание ровера-планетохода. Команда Участников должна была:
\begin{itemize}
    \item собственными силами собрать ровер;
    \item спроектировать и собрать полезную нагрузку;
    \item изготовить электрическую схему ее подключения;
    \item настроить и в процессе испытаний и работы администрировать ровер, работающий под управлением ROS;
    \item запрограммировать ровер на выполнение управляющих команд;
    \item осуществлять эксплуатацию ровера в процессе испытаний, параллельно определяя и устраняя проблемы, возникающие в программном обеспечении ровера, не имея с ним прямого контакта.
\end{itemize}

Совокупность данных задач является прототипом реальных проектов создания аэрокосмических систем и представляет собой кросс-функциональное задание, содержащее в себе запрос на все ключевые компетенции, требующиеся в реальных инженерных командах. Кроме очевидных, компетенций по программированию, моделированию, проектированию и электротехнике, решение данного задания предполагает знания основ проектного управления и командного взаимодействия. Кроме этого в процессе испытаний и работы ровера возникает группа задач по эксплуатации и обслуживанию сложных технических систем, которые хоть и не представляют из себя классических фундаментальных научных дисциплин, тем не менее являются сложными административно-техническими компетенциями.