\begingroup
\pagestyle{empty}

\section*{Введение}

\subsubsection*{Направление: Дополненная реальность}

В современном мире дополненная реальность (AR — Augmented Reality) является трендом и в скором времени может найти применение в широком ряде сфер деятельности. Технология обогащает видимую картину мира информацией, взятой из контекста, оказывает влияние на пользовательский опыт, снижая стоимость взаимодействия, минимизируя когнитивную нагрузку и усиливая внимание. Профиль позволяет заглянуть за пределы определенной сегодня для AR-технологии роли, провести собственные исследования и эксперименты в области разработки программно-аппаратной составляющей AR-интерфейсов, привлечь наиболее творческую часть молодежи к разработке новых инструментов и сценариев контекстного компьютинга.

Отборочные этапы Олимпиады НТИ и подготовительные мероприятия выстроены так, чтобы в течение года учащиеся смогли качественно подготовиться к выполнению финального задания. Профиль включает в себя задачи по двум школьным предметам: информатика и математика и соревнования по решению инженерных задач. 

Первый отборочный этап проводится индивидуально в сети Интернет с использованием онлайн-платформы Stepik. Для каждой возрастной группы (9 класс или 10-11 класс) предлагается свой набор заданий, отражающий углубленное изучение указанных выше дисциплин и соответствующий предметным компетенциям уровня участников школьных олимпиад. Участники, преодолевшие порог по сумме баллов по обоим предметам, переходят в следующий этап олимпиады.

Второй этап является командным и состоит из задач, призванных не только выявить финалистов, но и помочь участникам подготовиться к решению комплексной задачи финала. Предлагаемые задачи рассчитаны на разные роли участников в команде: веб-программист, AR-архитектор, 3D-конструктор, специалист по логистике и геоданным. В период работы над этими заданиями, каждый участник нарабатывает необходимую базу знаний и умений согласно своей профессиональной направленности в команде. Тематика самой сложной задачи второго этапа была связана с фантастическим «Миром Полудня» братьев Стругацких. От участников требовалось разработать комплексное программное обеспечение. Предполагалась, что пока одни члены команды решают задачу, связанную с распознаванием объектов, анализируя более 1000 изображений; другие разрабатывают архитектуру, интерфейсы, 3D-модели и функционал для мобильного AR-приложения, визуализирующего легенду задачи.

Параллельно с первым и вторым отборочными этапами для участников проводятся учебно-тренировочные сборы (УТС), в том числе, с привлечением региональных ЦМИТ-ов, площадок федеральной сети детских технопарков «Кванториум». 

Задача финала посвящена созданию AR-браузера для проведения тематической экскурсии по городу. Это реальный и востребованный в регионах, ориентированных на туризм, продукт – информационная система, состоящая из целого спектра, обеспечивающих функционирование ее программных модулей. В данной ситуации каждый участник занимался подготовкой своего компонента программного обеспечения: веб-ресурса, динамической AR-карты, AR-навигатора, калькулятором маршрутов, 3D-моделей объектов и сцен. Высокую роль при выполнении заданий финала в данном профиле играет творческая составляющая коллектива, готовность к проведению различных экспериментов и самостоятельных исследований по поиску новых оригинальных решений для дизайна интерфейсов, сценариев взаимодействия и архитектуры приложения. 

Помимо командной задачи участникам необходимо решить индивидуально задачи по предметам информатика и математика, которые схожи по темам с задачами первого этапа, но имеют более высокую сложность.

\subsubsection*{Направление: Виртуальная реальность}

В современном мире виртуальная реальность (VR — Virtual Reality) является популярным направлением и находит применение во многих прикладных областях. Технологии VR позволяют погрузить пользователя в созданное разработчиками виртуальное трехмерное пространство и взаимодействовать с объектами цифрового мира, что дает возможность неоднократно пережить и получить опыт в условиях максимально приближенных к реальным. Направление «Виртуальная реальность« дает возможность школьникам попробовать свои силы в разработке, регулярно обновлять свои знания и опыт, адаптироваться и разрабатывать системы под различные конфигурации VR-оборудования. 

Отборочные этапы Олимпиады НТИ и подготовительные мероприятия выстроены так, чтобы учащиеся могли качественно подготовиться к выполнению финального задания. Олимпиада по направлению включает в себя соревнования по двум школьным предметам: информатика и математика. 

Первый отборочный этап (решение задач по предметам) проводится индивидуально в сети Интернет с использованием онлайн-платформы Stepik. Для каждой возрастной группы (9 класс или 10-11 класс) предлагается свой набор заданий, отражающий углубленное изучение указанных выше дисциплин и соответствующий предметным компетенциям уровня школьных олимпиад. Участники, преодолевшие порог по сумме баллов по обоим предметам, попадают в следующий этап олимпиады.

Второй этап является командным и состоит из задач, призванных помочь участникам подготовиться к комплексному решению заданий заключительного этапа. Предлагаемые задачи учитывают разные роли участников в команде: программист, 2D/3D художник, геймдизайнер. В период работы над командными заданиями каждый участник получает опыт в своей профессиональной направленности или роли в команде. Область виртуальной реальности подразумевает выполнение более 50\% задач по 3D моделированию, включая текстуринг и анимацию. 

Параллельно с первым и вторым отборочными этапами для участников проводятся учебно-тренировочные сборы в региональных ЦМИТах, на площадках федеральной сети детских технопарков «Кванториум». 

Задача финального этапа заключается в разработке упрощенного варианта игры «Фактория» с применением технологии виртуальной реальности. Задание для состязаний и критерии разработаны таким образом, чтобы каждая команда могла достичь максимального результата только при условии распределения ролей и задач между собой. В качестве ключевых параметров разработки учитываются качество кода, реализованный функционал, а также художественное оформление приложения. 

Помимо командной задачи участникам необходимо решить индивидуальные задачи по предметам информатика и математика, которые схожи по темам с задачами первого этапа, но имеют более высокую сложность.
\clearpage
\endgroup