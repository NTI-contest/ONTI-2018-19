%\chapter*{История ОНТИ}

\begingroup
\pagestyle{empty}

\section*{Введение}

Профиль «Наносистемы и наноинженерия» состоял из двух отборочных этапов и заключительного этапа. Задачи и задания отборочных этапов разрабатывались таким образом, чтобы отобрать участников, способных справиться с заданием финала и подготовить их к его выполнению. 

В первом отборочном этапе участникам предлагались задачи продвинутого школьного уровня по физике, химии и биологии, что позволяло отобрать школьников, готовых к выполнению заданий второго этапа. 

Во втором отборочным этапе командам требовалось работать с симулятором химического производства. Работа заключалась в моделировании синтеза наноразмерных полупроводников - квантовых точек, и проектирования на их основе солнечных батарей, дисплея и биометок. Эти задачи заметно выходили за рамки школьной программы в целом, однако необходимая для их решения информация (или ссылки на необходимые источники) предлагалась в методических пособиях вместе с задачами. Это было сделано для того, чтобы познакомить участников с реальными проблемами, находящимися на стыке нанотехнологий, медицины, энергетики и современного производства электроники, дать представление о процессе инженерной разработки в целом. Задачи требовали применения знания школьной программы по химии, физике и биологии в сочетании с предоставленной информацией к решению задач непривычного школьникам формата. Кроме того, теоретические материалы к задачам второго этапа готовили участников к пониманию задачи финала. 

В финале, помимо индивидуального тура (олимпиадных задач по биологии, химии и физике), участникам была предложена командная практическая задача. Она состояла в самостоятельном синтезе перовскитных квантовых точек и создании на их основе прототипов RGB-дисплеев. 

Для оценивания синтезированных квантовых точек применялись критерии, созданные на основе стандартов мировых производителей электроники с использованием квантовых точек. Задача была построена таким образом, что участники имели возможность вернуться к начальному этапу и начать всё сначала, если совершили ошибку. Выполнение задания предполагало командную работу и четкую организацию своего времени. Для понимания основ применяемых методов участникам требовалось знание школьной программы по физике и химии, а также материалов второго этапа. Участникам в ходе задачи были предложено ответить на вопросы о сути применяемых ими методик, а также закрепить в них понимание тех процессов, с которыми они столкнулись. Это было сделано для того, чтобы выявить команду, наиболее глубоко и полно разбирающуюся в тематике финальной задачи. Мы хотели подобрать интересную, вариативную и выполнимую школьниками в сжатые сроки задачу из области нанотехнологий, которую можно было бы реализовать в рамках имеющегося оборудования. От участников требовались не столько глубокие знания химии и физики вне школьной программы, сколько способность применять полученную информацию на практике и принимать самостоятельные решения.

\clearpage
\endgroup