%\chapter*{История ОНТИ}

\begingroup
\pagestyle{empty}

\section*{Введение}

Профиль «Нейротехнологии» в 2018-2019 учебном году был посвящен разработке системы мониторинга состояния человека для выявления сонливости и потери внимания у водителя. Данное направление требует знаний в таких областях, как программирование (основы машинного обучения, компьютерное зрение), электроника, физиология человека (особенности обработки биосигналов человека).

Отборочные этапы Олимпиады НТИ по профилю и подготовительные мероприятия были выстроены так, что в течение года учащиеся могли  подготовиться к решению задачи на заключительном этапе олимпиады.

В первом отборочном дистанционном этапе (индивидуальный) участникам были предложены задания по двум предметам – информатика и биология. По информатике проверялись базовые основы программирования, по биологии был сделан упор на темы, связанные с работой нервной системы, мозга человека и др. (при этом оставаясь в рамках тем школьной программы по биологии).

Во втором отборочном дистанционном этапе участникам были предложены командные задачи по обработке и анализу биосигналов человека (электроэнцефалограмма, электрокардиограмма), а также задачи по компьютерному зрению. Решая задачи второго этапа, участники изучали и реализовывали различные алгоритмы и методы обработки данных, нарабатывая необходимую базу знаний и умений для решения задач на финальном этапе.

Параллельно с первым и вторым отборочным этапам участники могли принять участие в хакатонах, где была предоставлена возможность работы с различными датчиками и оборудованием, которое использовалось в финальных соревнованиях. Для тех, кто не мог принять участие очно, мог ознакомиться с планом хакатонов (задачи, полезные материалы).

В финальных очных соревнованиях участникам предстояло применить все ранее полученные знания для разработки системы мониторинга состояния водителя за рулем, чтобы точнее и быстрее выявить признаки потери внимания водителя. В составе команды предполагалось от 2 до 4 человек, где у каждого был свой функционал: сборка и отладка системы, программирование контроллеров, постановка эксперимента, интерпретация данных, построение физиологической модели, реализация различных методов машинного обучения и др.  Помимо командной задачи на финальном этапе участникам необходимо было решить индивидуальные задачи по предметам - информатика и биология, которые схожи по темам с задачами первого этапа, но имели более высокую сложность.

\clearpage
\endgroup