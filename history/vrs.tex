%\chapter*{История ОНТИ}

\begingroup
\pagestyle{empty}

\section*{Введение}

Профиль «Водные робототехнические системы» в 2018-2019 уч. году был посвящен разработке подводных устройств и использованию автономных необитаемых подводных аппаратов для решения конкретных задач в воде. 

Решение заданий профиля требовало знаний школьного уровня по информатике и физике: программирование, геометрия, гидродинамика. Кроме базовых школьных знаний и навыков для решения задач профиля также требовалось самостоятельное освоение следующих тем: теория автоматического управления (пропорциональный регулятор), алгоритмы распознавания образов, а также умения программировать на языке С++. Методические материалы для самостоятельного освоения были предоставлены во время подготовительных мероприятий к заключительному этапу через вебинары.

Отборочные этапы и подготовительные мероприятия были выстроены так, чтобы в течение года учащиеся имели возможность подготовиться к финальным задачам профиля.

\textbf{Первый отборочный этап}

Первый отборочный этап проводился индивидуально в сети интернет, работы оценивались автоматически средствами системы онлайн-тестирования. Целью данного этапа являлся отбор участников, обладающих базовыми знаниями в области физики и информатики.  Для каждой возрастной группы (9 класс или 10-11 класс) предлагался свой набор задач по физике, задания по информатике были общими для всех классов. 

\textbf{Второй отборочный этап}

Цель второго дистанционного отборочного этапа состоялаа в том, чтобы отобрать наиболее подготовленных участников для участия в очных финальных состязаниях. 

Для этого участникам предлагалось выполнить ряд заданий в симуляторе, где необходимо было запрограммировать подводного робота выполнять задания, аналогичные тем, что предстоит сделать в финале на настоящем подводном аппарате. 

Например, команды должны были запрограммировать робота на стабилизацию по курсу и глубине. Робот должен был распознавать «на лету» разные объекты под водой: прямоугольники, круги, квадраты и принимать решение в зависимости от того, что удалось распознать.

Данные задания предлагались с целью подготовки участников к финалу олимпиады НТИ по следующим направлениям:
\begin{itemize}
    \item укрепление теоретической и практической базы в области программирования подводных роботов;
    \item знакомство с алгоритмами компьютерного зрения;
    \item изучение API функций среды программирования MUR\_IDE, с которой придется работать в финале.
\end{itemize}

\textbf{Финал. Индивидуальная часть}

Участники решали в индивидуальном порядке предметные задачи по физике и информатике (предметный тур). Задачи предметного тура преследовали следующие цели:
\begin{itemize}
    \item объективно проверить индивидуальные знания участников;
    \item косвенно оценить индивидуальный вклад участников в результат командной работы;
    \item Вес этих задач в финальной оценке участников составлял 40\%.
\end{itemize}

\textbf{Финал. Командная часть}

На заключительном этапе в командном туре участникам предлагалось спроектировать и изготовить подводную мини-торпедную установку, которую можно подключить к конструктору MUR. Для этого каждая команда получила исчерпывающую информацию о протоколе обмена данных с периферийными устройствами MUR и о распиновке разъема подключения. 

Далее команда должна была выполнить задание в симуляторе.

Итоговое задание: выполнить задание, которые ранее решалось на симуляторе. Робот должен был выполнить некоторые задания в бассейне, например, пройти по полоске и торпедировать мишень.

В финале участники должны были продемонстрировать высокий уровень подготовки по следующим направлениям:
\begin{itemize}
    \item высокий уровень знаний и навыков в области электронной инженерии и программирования встраиваемых систем;
    \item высокий уровень знаний и навыков в области программирования автономных подводных аппаратов;
    \item высокий уровень знаний и навыков в области компьютерного зрения;
    \item владеть навыками командной работы и делегирования в условиях соревнований и высокой конкуренции.
\end{itemize}

\clearpage
\endgroup