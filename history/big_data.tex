\chapter*{История ОНТИ}


\section*{Профиль <<Большие данные и машинное обучение>>}

В 2019 году организаторы профиля «Большие Данные и Машинное Обучение»  приняли решение двигаться в сторону самообучающихся систем без учителя. Данное направление набирает популярность ввиду применимости ко многим задачам современности и появлением математического аппарата обучения с подкреплением. Такое направление сочетает в себе как классическое машинное обучение, так и создание управляемых агентов в виртуальной среде. 

В качестве командной задачи на заключительном этапе олимпиады предлагалось реализовать агента, управляющего квадрокоптером. Данный агент должен выводить квадрокоптер на необходимое направление полета и изменять его скорость, управляя мощностями каждого из винтов коптера.

В целях подготовки школьников к решению командной задачи заключительного этапа и отбора школьников с необходимыми компетенциями в первом отборочном этапе проверялись общие знания в области математики и информатики.

В рамках первого этапа по предмету математика проверялись компетенции школьников в области теории чисел, геометрических функций, решения уравнений. Данные знания являются базовыми для расчета метрик и свойств моделей. Также проверялись знания по геометрии в пространстве (стереометрия), использования трехмерной системы координат и вычислений в ней.  

Пример задачи из первого этапа олимпиады (математика):\\
\textit{Коптер летит над поверхностью огромного поля. В какой-то момент времени он оказывается в точке $(0; 3; 6)$ в заданной ортогональной системе координат с осями $Ox$, $Oy$ и $Oz$. Найдите расстояние от коптера до земли, если в той же системе координат поле можно считать плоскостью, заданной уравнением $2x+4y-4z-6 = 0$. Ответ укажите с точностью до десятитысячных.}

В рамках предмета информатика проверялись умения разрабатывать алгоритмы для решения разных задач и создавать программы, реализующие различные алгоритмы, в том числе работа с геометрическими построениями, которые были необходимы для решения командной задачи. Также проверялись знания теории сложности и готовность обрабатывать относительно большие объемы данных (до 100К входные чисел). Школьники тренировались в написании программ для решения геометрических задач на плоскости и в пространстве и оптимизационных задач, которые необходимы для управления роботом.

Пример задачи из первого этапа олимпиады (информатика):\\
\textit{Задача 2.1.5. Робот и камушки (25 баллов)

Ильнар в одной из комнат увидел странного робота. Во время выполнения алгоритма, он доставал из мешка разные камушки. Причем он никогда не доставал один и тот же камень два раза и говорил, сколько камушков в мешке такого же цвета. Разработчик робота рассказал Ильнару, что из-за ошибки в коде робот ровно один раз всегда ошибается. Теперь Ильнару интересно, а сколько минимально может быть камушков в мешке.}

Второй этап продолжил логику первого, однако с более высоким классом задач. Кроме функции отбора участников в финал второй этап был для участников обучающим, т.к. на втором этапе предлагалось уже непосредственно решать задачи машинного обучения, требующие построения прогнозных моделей. Многие участники были не знакомы с данными методами и, решая задачи, учились с помощью ресурсов, предложенных организаторами олимпиады.  Опыт построения моделей поможет школьникам строить прогнозные модели для коптера при решении задач на финальном (заключительном) этапе.

Пример задачи\\
\textit{В качестве тренировочной выборки вам даны некоторые данные GPS-навигации о движении транспорта в Красноярске. 
\url{https://docs.google.com/spreadsheets/d/1QdZJT05NiqgCDPYrPHJBWCK2SeORK2Fo6HNODT7cgE/edit?usp=sharing}
В качестве признаков имеются средняя скорость, время в пути, пройденная дистанция, rating – оценочный параметр, выражающий насколько данное средство передвижения удобно для перемещения по городу (3 – удобно, 2 – нормально, 1 – плохо), а также ответ – 0car0 или 0bus0.

Помимо этого вам дана тестовая выборка(test.csv), набор столбцов которой отличается от обучающей только отсутствием столбца car\_or\_bus. 

Как вы уже могли догадаться, ваша задача – обучиться на обучающей выборке и научиться предсказывать для тестовой выборки тип каждого из указанных в ней средств передвижения (car или bus).}

Для прицельной подготовки к задаче командного этапа организаторами профиля был проведен онлайн хакатон для всех участников по похожей задаче (\url{http://nti-contest.ru/konkursmts/}). В рамках хакатона школьники имели возможность ознакомиться с основными принципами обучения с подкреплением. На хакатон участники тренировались работать в среде GYM, на основе которой реализовалось командное задание заключительного этапа олимпиады. 