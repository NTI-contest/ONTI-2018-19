%\chapter*{История ОНТИ}


\section*{Введение}

В 2018-2019 уч. году в профиле “Автономные транспортные системы” создавалась модель транспортной системы для перемещения самого транспорта и перевозки грузов. Для выполнения комплексной задачи необходимо было обеспечить ориентацию соответствующих транспортных средств на полигоне и взаимодействие каждого транспорта с системой города. В рамкам можели города были использованы 2 вида транспорта, взаимодействующих между собой (машина и квадракоптер). Данное направление требует знаний в областях электроники, программирования, моделирования, конструирования. Среди общих компетенций также стоит отметить умение работать в команде, правильно распределять ресурсы, умение ставить задачи и строить подробный план действий с возможными форс-мажорами. Отборочные этапы и подготовительные мероприятия были выстроены так, чтобы в течение года учащиеся смогли подготовиться к финальной задаче. 

В первом отборочном дистанционном индивидуальном этапе участники решали задачи по двум предметам - физика и информатика. По физике задачи были направлены на определение уровня базовых школьных знаний, а по информатике - на высокие навыки и уровень программирования, умение составлять и программировать сложные алгоритмы.

После первого тура участники, прошедшие во второй тур объединялись в команду и совместно решали отборочные дистанционные задания. В рамках второго тура были представлены три задачи, каждая из которых непосредственно были связаны с финальной задачей. Так, участникам предлагалось сделать 3D-модель объекта по предоставленным чертежам, написать программу для распознавания изображений с камеры, написать программу для работы дрона (квадракоптера) в заданных условиях. Решая задачи второго этапа, участники изучали и реализовывали алгоритмы обработки и фильтрации данных, нарабатывая необходимую базу знаний и умений в области работы с машинным зрением, моделированием, ключевыми библиотеками и функциями языка Wiring, необходимыми для решения задачи на финальных соревнованиях.

Параллельно с первым и вторым отборочным этапам школьники также могли принять участие в хакатонах, где они могли непосредственно поработать с различными датчиками и оборудованием, которое использовалось в финальных соревнованиях.

В финальных очных соревнованиях участникам предстояло применить все ранее полученные знания для сборки, пайки, программирования и отладки двух устройств – моделей транспортных средств, которые бы передвигались по полигону, взаимодействуя с элементами инфраструктуры (дорожная разметка, дома, посадочные площадки, светофоры, дорожные знаки и прочее). Для реализации задачи перемещения груза участники должны были спроектировать сам груз и захватывающе-сбрасывающее устройство для коптера в соответствии с техническим заданием. Кроме этого, участникам необходимо было скоординировать боа устройства в единую систему через сервер для обеспечения доставки груза в соответствии с поставленным заданием. 

Помимо командной задачи в финальном этапе участникам необходимо было решить индивидуальные задачи по предметам информатика и физика, которые схожи по темам с задачами первого этапа, но имели более высокую сложность.