
%\chapter*{История ОНТИ}


\begingroup
\pagestyle{empty}

\section*{Введение}

\textbf{Логика построения задач профиля} — последовательная демонстрация процессов разработки миссии и проектирования малых космических аппаратов участникам олимпиады. На каждом этапе, от предметных задач первого этапа до финала, происходит постепенное и глубокое погружение в задачи инженерии космических систем на примере малого КА (космического аппарата, спутника) для задач ДЗЗ (дистанционного зондирования Земли). Знания и умения, полученные при подготовке к Олимпиаде и в работе над задачами, позволяют участникам понять специфику разработки малых и сверхмалых КА, основы расчета миссии и циклограмм, требования, накладываемые на служебные системы различными видами полезных нагрузок. 

\textbf{На первом этапе} происходит проверка знаний школьников в области физики и информатики, позволяющие осуществить предварительный отбор и подготовку участников по темам, востребованным в работе над задачей финала.

\textbf{Во втором этапе} задачи погружают в специфику профиля: классические задачи по физике и задачи в симуляторе космических полетов требуют понимания основ орбитальной механики и позволяют знакомить участников с принципами построения орбитальных группировок ДЗЗ. Последняя группа задач посвящена оптике и позволяет знакомить с основными понятиями оптических систем, подготавливая участников к работе над инженерной задачей финала. Для объективной оценки результатов используется учебный симулятор космических полетов, который позволяет решать необходимые для финала задачи, в том числе в области оптимизации состава КА, точной стабилизации и ориентации на орбите, передачи данных на центр управления полетом, перехода на оптимальную орбиту и пр.

Для подготовки к решению предлагаемых задач используется несколько тренировочных заданий симулятора, решая которые, можно ознакомиться с характером задач и понять, какие именно области необходимо изучить более детально для дальнейшего участия в олимпиаде по данному профилю.

\textbf{Задачи предметного тура финала} по физике и информатике нацелены на проверку необходимых знаний и помощь участнику в работе над задачей командного тура. Задачи  построены по принципу вовлечения: результат решения более легких задач может быть использован в решениях более сложных, последовательное решение задач помогает участнику набрать больше баллов.

\textbf{Во время подготовки к финалу} проводятся хакатоны, где участники работают с конструктором микроспутников и специально разработанным набором электронных и оптических компонентов, а также профессиональным симулятором космических полетов. Данный набор содержит упрощенные функциональные макеты основных модулей КА, представленного на финале. Работа с элементами конструктора и алгоритмами, реализуемыми на нем, приближена к реальным задачам проектирования космических систем. На хакатонах участники знакомились с конструктором и основой задачи одноосной стабилизации и ориентации КА, необходимой для полного решения задачи финала. 

\textbf{Итоговая задача финала} направлена на сборку всех компонентов заданий в единый комплекс: происходит разработка прототипа КА с полезной нагрузкой - камерой и оптической системой, а также системой автоматизации процесса ориентации, съемки и беспроводной передачи снимков на судейский центр управления. Вместе с этим команда разрабатывает проект группировки КА, состоящей в том числе из КА соответствующего типа, для съемки Северного Морского Пути. 

\clearpage
\endgroup