%\chapter*{История ОНТИ}


\section*{Введение}

\textbf{Олимпиадные задания по профиль «Беспилотные авиационные системы» Олимпиады НТИ} включают в себя задачи по двум школьным предметам информатика и физика и командную инженерную задачу.

Задание командного тура финала инженерной олимпиады состояла в том, что необходимо было решить задачу поиска объекта на местности в автоматическом режиме с использованием БПЛА самолетного типа и технического зрения (работа с элементами и системами БПЛА, моделирование полета, разработка системы автоматического управления, техническое зрение, проведение испытаний).

Для решения этого задания участникам необходимо было иметь соответствующие знания по физике и информатике, а также необходимо было:
\begin{itemize}
    \item разобраться в принципах полета БПЛА самолётного типа, научиться работать с элементами БПЛА (датчики, исполнительные механизмы);
    \item решить задачу определения текущего местоположения БПЛА и уметь регистрировать его показания;
    \item разобраться, что такое регуляторы, и в принципах построения системы автоматического управления, используя симулятор провести настройку регуляторов по соответствующим критериям;
    \item научиться работать с техническим зрением по детектированию заданного объекта;
    \item проверить работоспособность созданного программного обеспечения (системы автоматического управления (САУ) БПЛА самолётного типа) каждой командой путём лётных испытаний на аэродроме.
\end{itemize}

Основной акцент при формировании командного задания был направлен на четкое разделение на подзадачи, при успешном решении которых участники достигали поставленной цели. Каждая подзадача в финале имела свой уровень сложности и соответствующий балл.

Подзадачи были распределены по соответствующим этапам. На первом этапе участники решали подготовительные задачи по физике и информатике. На втором этапе был сделан переход к работе по определению параметров полета, ориентации и навигации, моделированию полета в упрощенной форме. Задачи носят междисциплинарный характер и в упрощенной форме воссоздают инженерную задачу заключительного этапа. В заключительном этапе все, полученные на предыдущих этапах, знания участники использовали при работе с реальным БПЛА самолетного типа (работа с датчиками, электродвигателем, моделированием полета и элементами технического зрения). Такой подход даёт возможность участникам глубоко понять объект исследования и принцип его работы.