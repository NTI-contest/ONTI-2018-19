\chapter{Введение}

Олимпиада Национальной технологической инициативы  (далее – Олимпиада \linebreak НТИ)\footnote{Национальная технологическая инициатива (НТИ) — это программа мер, нацеленная на формирование принципиально новых рынков и создание условий для глобального технологического лидерства России к 2035 году. Задача по созданию НТИ поставлена Президентом Российской Федерации 4 декабря 2014 года в Послании к Федеральному собранию.} – это командная инженерная олимпиада школьников, завершающаяся разработкой действующего устройства, системы устройств или компьютерной программы. Олимпиада является проектом Агентства стратегических инициатив, элементом дорожной карты НТИ «Кружковое движение» и ключевым механизмом вовлечения инженерно – ориентированных школьников в образовательные программы высшего образования, ориентированные на рынки НТИ. Оператором Олимпиады НТИ  является некоммерческая организация – Ассоциация участников технологических кружков. Профили Олимпиады НТИ выбраны на основе приоритетов Национальной технологической инициативы: «Автономные транспортные системы», «Большие данные и машинное обучение», «Системы связи и дистанционного зондирования Земли», «Интеллектуальные энергетические системы», «Нейротехнологии», «Инженерные биологические системы: агробиотехнологии и геномное редактирование», «Интеллектуальные робототехнические системы», «Беспилотные авиационные системы», «Композитные технологии», «Когнитивные технологии», «Аэрокосмические системы», «Наносистемы и наноинженерия», «Технологии беспроводной связи», «Умный город», «Передовые производственные технологии», «Виртуальная и дополненная реальность», «Анализ космических снимков и геопространственных данных», «Водные робототехнические системы» и «Программная инженерия финансовых технологий».

Цель Олимпиады НТИ: поддержка школьников в стремлении решать технологические вызовы XXI века (что подразумевает включение их в решение технологических задач переднего края и, одновременно, повышение социальной значимости такой работы старшеклассников через льготы к поступлению). Эта цель лежит в рамках большей цели Кружкового движения: формирование и подготовка команд, способных запускать глобальные технологические проекты, менять мир, создавая новые общественные практики. Именно участники этих команд должны будут через 10-15 лет «перезапустить» НТИ: создать собственные рынки и глобальные прорывные компании.  Важной особенностью олимпиады является то, что в части отборочного и в заключительном этапах участники выполняют задания в командах по 2-4 человека. Умение работать в команде - важный навык человека 21 века. Команды формируются на основе компетентностного принципа, различные компетенции участников в одной команде позволяют найти оригинальное нестандартное решение задачи.  В командах участники планируют свою работу, обсуждают, ищут совместно решения, распределяют роли - часто один участник выполняет несколько ролей. Комплексные инженерные задачи, которые решают участники, не под силу решить отдельно взятому школьнику. Задачи разработаны таким образом, что декомпозируются на несколько подзадач, за решение, которых берутся участники согласно своей роли в команде. Каждый участник несет ответственность за результат работы команды. Поэтому, при подведении итогов олимпиады определяются не только победители в личном зачете, но и команда-победитель. 

Целевыми победителями Олимпиады НТИ являются школьники, способные реализовывать сложные технические проекты в прорывных областях. Олимпиада должна выделять команды участников с особыми характеристиками мышления, коммуникации и действия, необходимыми для решения задач НТИ. Победители и призеры Олимпиады НТИ должны показывать высокие результаты в области применения предметных знаний в практической работе. Одновременно с этим, система подготовки Олимпиады НТИ должна предоставлять участникам инструменты для подготовки и получения недостающих знаний и практических навыков.

\section*{Первый год проведения олимпиады}

Олимпиада НТИ была впервые проведена в 2015/2016 учебном году. В отборочных этапах олимпиады приняли участие несколько тысяч школьников, около ста из них были приглашены к участию в заключительном этапе по профилям «Большие данные и машинное обучение», «Системы связи и дистанционного зондирования Земли», «Интеллектуальные энергетические системы», «Автономные транспортные системы». Заключительный этап Олимпиады и торжественные мероприятия проводились в ВДЦ «Орленок». 

В 2015/2016 учебном году победители и призеры олимпиады могли воспользоваться возможностью добавить дополнительные 10 баллов к сумме баллов за вступительные экзамены, в случае если они поступали в вузы-организаторы Олимпиады НТИ. 

\section*{Второй год проведения олимпиады}

В 2016/2017 учебном году Олимпиада проводилась во второй раз по 12 профилям, количество зарегистрированных для участия школьников увеличилось более чем в три раза и достигло 12 тыс., в отборочных этапах приняли активное участие 4 тыс. школьников, на заключительный этап прибыло 306 участников.  

Заключительный этап Олимпиады и торжественные мероприятия проводились на площадке Образовательного центра «Сириус», в лабораториях и помещениях Парка Наук и Искусств. Вечером проходили лекции и неформальные встречи с представителями технологических компаний. 

В 2016/2017 учебном году четыре профиля Олимпиады НТИ («Автономные транспортные системы», «Большие данные и машинное обучение», «Системы связи и дистанционного зондирования Земли», «Интеллектуальные энергетические системы») входили в Перечень олимпиад школьников, таким  образом победители и призеры смогли воспользоваться льготами при поступлении в вузы России (в зависимости от правил приема конкретного вуза). Победители и призеры новых профилей также могли воспользоваться бонусами при поступлении в вузы, которые имеют статус «организатор Олимпиады НТИ».

\section*{Третий год проведения олимпиады}

В отборе на Олимпиаду 2017/2018 учебного года приняло участие более 20 тыс. школьников, подавших более 50 тыс. заявок на различные профили, число которых увеличилось до 17. В финал вышли 578 участников Олимпиады из 51 региона РФ:

\putImgWOCaption{16cm}{history/info/map}

Финал стал распределенным и проходил с февраля по апрель 2018 года: Олимпиаду приняли Образовательный центр «Сириус», МАИ, МИФИ, ТПУ, Университет Иннополис, СПбПУ, ДВФУ, УрФУ. В 2017/2018 учебном году девять из 14 профилей Олимпиады НТИ включены в Перечень олимпиад школьников (приказ Минобрнауки России от 30.08.2017 № 866) и дают льготы при поступлении в вузы.

Важная составляющая подготовки участников к финалу Олимпиады –  открытые для всех желающих хакатоны, вебинары и мастер-классы. Программы этих мероприятий разработаны педагогами профилей Олимпиады НТИ специально для регионов так, чтобы их можно было провести на минимальном количестве оборудования. Сеть региональных партнеров Олимпиады со статусом Методическая площадка или Площадка подготовки растет с каждым годом, и в 2018 году, на третий год проведения Олимпиады, их количество достигло 110, всего проведенных мероприятий по подготовке  (соревнований, хакатонов, сборов) –  более 50. Информация о партнерских площадках размещена в специальном разделе официального сайта олимпиады: \url{http://nti-contest.ru/places_to_prepare/}.

\section*{Четвертый год проведения олимпиады}

В отборе на Олимпиаду 2018/2019 учебного года приняло участие более 36 тыс. школьников, подавших более 70 тыс. заявок на различные профили, число которых увеличилось до 19. В финал вышли 1053 участника Олимпиады из 60 регионов РФ.

Финал стал распределенным и проходил с марта по апрель 2019 года: Олимпиаду приняли МФТИ, МАИ, МИФИ, ТПУ, Университет Иннополис, СПбПУ, ДВФУ, НГУ, НовГУ, Московский Политех, ИГУ, ИРНИТУ и ряд других площадок. В 2018/2019 учебном году 13 из 19 профилей Олимпиады НТИ включены в Перечень олимпиад школьников (приказ №32н от 28 августа 2018 года Министерства науки и высшего образования Российской Федерации) и дают льготы при поступлении в вузы.

В олимпиаде в 2018/2019 учебном году впервые были проведены синхронные по времени распределенные финалы на площадках в разных городах в рамках одного профиля: Нейротехнологии (ДВФУ, НГУ, МФТИ, НовГУ), ИЭС (ИГУ, МИФИ), Нанотехнологии (Школа Летово, НГУ), АТС (Московский политех, НовГУ). Участники распределенных финалов имели одинаковые задания, критерии оценивания и единый рейтинг участников.

\section*{График проведения заключительных этапов\\Олимпиады НТИ 2018/2019 гг.}

\begin{center}
    \small
    \begin{longtable}{|p{7.5cm}|p{2.5cm}|p{5cm}|}
        \hline
        \textbf{Площадка проведения} & \textbf{Даты проведения} & \textbf{Перечень профилей Олимпиады НТИ} \\
        \hline
        \textbf{Университет Иннополис}
        
        (г. Иннополис) & 3-11 марта 2019 г. & Интеллектуальные робототехнические системы\\
        \hline
        \textbf{Университет Иннополис}
        
        (г. Иннополис) & 6-11 марта 2019 г. & Программная инженерия финансовых технологий\\
        \hline
        \textbf{Школа Летово}

        (г. Москва)

        \textbf{Новосибирский государственный университет}

        (г. Новосибирск) & 11-16 марта 2019 г. & Наносистемы и наноинженерия \\
        \hline
        \textbf{Московский политехнический университет} 

        (г. Москва) & 11-16 марта 2019 г. & Инженерные биологические системы: Агробиотехнологии \\
        \hline
        \textbf{Московский авиационный институт}

        (г. Москва) & 11-16 марта 2019 г. & Беспилотные авиационные системы \\
        \hline
        \textbf{Томский политехнический университет} 

        (г. Томск) & 12-17 марта 2019 г. & Умный город \\
        \hline
        \textbf{Иркутский национальный исследовательский технический университет}

        (г. Иркутск) & 13-19 марта 2019 г. & Технологии беспроводной связи\\
        \hline
        \textbf{Иркутский государственный университет} 

        (г. Иркутск)

        \textbf{Национальный исследовательский ядерный университет «МИФИ»}
        
        (г. Москва) & 13-19 марта 2019 г. & Интеллектуальные энергетические системы \\
        \hline
        \textbf{Иркутский государственный университет} 

        (г. Иркутск) & 13-19 марта 2019 г. & Технологии виртуальной и дополненной реальности: Дополненная реальность\\
        \hline
        \textbf{Дальневосточный федеральный университет}

        (г. Владивосток) & 18-23 марта 2019 г. & Виртуальная и дополненная реальность: Виртуальная реальность \\
        \hline
        \textbf{Дальневосточный федеральный университет} 

        (г. Владивосток) & 18-23 марта 2019 г. & Водные робототехнические системы \\
        \hline
        \textbf{Московский физико-технический институт}

        (г. Москва)

        \textbf{Новосибирский государственный университет}

        (г. Новосибирск) 

        \textbf{Новгородский государственный университет имения Ярослава Мудрого}

        (г. Великий Новгород)

        \textbf{Дальневосточный федеральный университет}

        (г. Владивосток) & 18-23 марта 2019 г. & Нейротехнологии \\
        \hline 
        \textbf{Московский физико-технический институт}

        (г. Москва),

        \textbf{Новосибирский государственный университет}

        (г. Новосибирск) & 18-23 марта 2019 г. & Инженерные биологические системы: Геномное редактирование \\
        \hline
        \textbf{Московский физико-технический институт}

        (г. Москва) & 18-23 марта 2019 г. & Большие данные и машинное обучение \\
        \hline
        \textbf{АО «ИПК Машприбор» ГК Роскосмос} 

        (г. Королев)

        \textbf{Детский технопарк «Кванториум»}

        (г. Королев) & 26-31 марта 2019 г. & Системы связи и дистанционного зондирования Земли \\
        \hline
        \textbf{АО «ИПК Машприбор» ГК Роскосмос}

        (г. Королев)

        \textbf{Детский технопарк «Кванториум»}

        (г. Королев) & 26-31 марта 2019 г. & Аэрокосмические технологии \\
        \hline
        \textbf{АО «ИПК Машприбор» ГК Роскосмос}

        (г. Королев)

        \textbf{Детский технопарк «Кванториум»}

        (г. Королев) & 26-31 марта 2019 г. & Анализ космических снимков и пространственных геоданных Земли \\
        \hline
        \textbf{Санкт-Петербургский университет Петра Великого,}

        \textbf{Академия цифровых технологий}

        (г. Санкт-Петербург) & 01-06 апреля 2019 г.	& Передовые производственные технологии \\
        \hline
        \textbf{Московский государственный психолого-педагогический университет}
        
        (г. Москва) & 02-06 апреля 2019 г. & Когнитивные технологии \\
        \hline
        \textbf{Московский государственный технический университет им. Н.Э. Баумана}

        (г. Москва) & 07-12 апреля 2019 г. & Композитные технологии \\
        \hline
        \textbf{Московский политехнический университет}
        (г. Москва)

        \textbf{Новгородский государственный университет имения Ярослава Мудрого}

        (г. Великий Новгород) & 08-14 апреля 2019 г. & Автономные транспортные системы\\
        \hline        
    \end{longtable}
\end{center}

\section*{Организаторы и партнеры Олимпиады НТИ}

Оргкомитет олимпиады представлен ректорами крупнейших политехнических и инженерных вузов России, руководителями технологических компаний и представителями государственных органов.

\textbf{Вузы-соучредители олимпиады:}
\begin{itemize}
    \item ФГБОУ ВО «Московский политехнический университет»;
    \item ФГАОУ ВО «Санкт-Петербургский политехнический университет Петра Великого»;
    \item ФГАОУ ВО «Национальный исследовательский Томский политехнический университет»;
    \item ФГАОУ ВО «Дальневосточный федеральный университет».
\end{itemize}

\textbf{Технологические партнеры}

Олимпиада НТИ проводится при поддержке технологических партнеров, количество которых увеличилось, по сравнению с прошлым годом,  среди них: РВК (Российская венчурная компания) и АСИ (Агентство стратегических инициатив по продвижению новых проектов)  –  в роли со-организаторов и генеральных партнеров выступают: Аэрофлот, ПАО «Сухой», ОАК, Роскосмос, ФИОП Роснано, МТС, Газпром нефть, Фонд новых форм развития образования, сеть детских технопарков «Кванториум», Спутникс, Полюс-НТ, BiTronicsLab, КРОК,  Инфосистемы Джет, Лоретт, Коптер Экспресс, АсРоботикс, Образование будущего и др. Полный список организаторов и партнеров олимпиады размещен в соответствующем разделе на официальном сайте: \url{http://nti-contest.ru/about/}.

\textbf{Вузы-организаторы профилей Олимпиады НТИ:}
\begin{itemize}
    \item АНО ВО «Университет Иннополис»; 
    \item ФГАОУ ВО «Национальный исследовательский ядерный университет «МИФИ»;
    \item ФГАОУ ВО «Московский физико-технический институт (государственный университет)»;
    \item ФГБОУ ВО «Московский авиационный институт (национальный исследовательский университет)»;
    \item ФГБОУ ВО «Сибирский государственный университет науки и технологий имени академика М.Ф. Решетнева»;
    \item ФГАОУ ВО «Новосибирский национальный исследовательский государственный университет»;
    \item АНО ВО «Сколковский институт науки и технологий»;
    \item ФГБОУ ВО «Московский государственный психолого-педагогический университет»;
    \item ФГБОУ ВО «Московский государственный технический университет имени Н.Э. Баумана (национальный исследовательский университет)»;
    \item ФГБОУ ВО «Новгородский государственный университет имени Ярослава Мудрого»;
    \item ФГБОУ ВО «Иркутский государственный университет»;
    \item ФГБОУ ВО «Новосибирский государственный технический университет»;
    \item ФГАОУ ВО «Национальный исследовательский Нижегородский государственный университет им. Н.И. Лобачевского»;
    \item ФГБОУ ВО «Московский технический университет связи и информатики».
\end{itemize}

К работе методической комиссии был привлечен профессорско-преподавательский состав вузов-организаторов и представители реального сектора экономики. Объективную оценку работы осуществляет жюри, представленное основателями технологических компаний, а также представителями вузов-организаторов.  

Вузы-организаторы, входящие в Оргкомитет Олимпиады НТИ, ведут непрерывную работу с талантливыми школьниками.

\textbf{МГТУ им. Н.Э. Баумана (Национальный Исследовательский Университет)} является организатором профиля «Композитные технологии»

МГТУ им. Н.Э. Баумана — ведущий технический ВУЗ России. На кафедрах ВУЗа обучается больше 20 тысяч студентов, решающих задачи разработки авиационной, ракетной, военной техники, ядерной и альтернативной энергетики, радиоэлектроники и передового информационного и медицинского оборудования.

МГТУ им. Н.Э. Баумана активно работает со школьниками со всей страны. В ВУЗе работает Центр довузовской подготовки, где учащиеся могут получить дополнительные знания по школьным предметам основного образования.

На базе МГТУ им. Н.Э. Баумана успешно функционирует межотраслевой инжиниринговый центр «Композиты России». Помимо научно-исследовательской деятельности, посвященной разработке и внедрению новых материалов в промышленность, Композиты России активно занимаются образованием: как школьным, так и высшим. Помимо Олимпиады Национальной технологической инициативы, центр  является основателем и организатором конкурса «Composite Battle» в 2015, 2016, 2017 и 2018 годах. Чемпионат получил положительные отзывы международного научного сообщества: благодарственные письма от участников из Беларуси, Китая, Италии, Мьянмы и Египта.

Главные партнеры профиля:

ПАО «ОАК» - российское публичное акционерное общество, объединяющее крупнейшие авиастроительные предприятия России.

ООО «Ниагара» - специализируется на производстве и реализации тканых и нетканых каркасов для композиционных материалов, изделий из угле- и стеклопластика, а также высокотемпературных теплоизоляционных материалов, является разработчиком и поставщиком оборудования (в том числе стенда для проведения суперфинала).

ООО «Викрон» - компания НТИ, реализует проект «Высокоавтоматизированное дистанционное зондирование почв и земель с применением БПЛА для расчета NDVI и формирования отчетов и рекомендаций об объекте исследования» в рамках гранта «Фонда содействия развитию малых форм предприятий в научно-технической сфер».

\section*{Структура отбора участников Олимпиады НТИ}

Соревнование проходит в три этапа. Первый и второй отборочные этапы проходили с октября по декабрь 2018 года в заочной форме на интернет-платформе «Stepik» (\url{http://stepik.org}) и в инженерных онлайн-симуляторах.

Отборочные этапы сопровождались различными подготовительными мероприятиями, среди которых были дистанционные мероприятия (вебинары), мероприятия для самостоятельной подготовки (онлайн-курсы), мероприятия направленные на командообразующую деятельность (специальные встречи, интенсивы, очные курсы на площадках по подготовке, создана специальная интерактивная форма формирования и подбора членов команд на платформе Олимпиады НТИ), мероприятия, направленные на получение практических навыков (интенсивы).

Заключительный этап Олимпиады состоит из двух частей: индивидуальное решение предметных задач по выбранным профилям и командная разработка инженерного решения с испытанием  его на стенде. Задание второй части заключительного этапа имеет свою специфику для каждого профиля.

\section*{Информация о профиле}

Профиль «Композитные технологии» проводился в МГТУ им. Н.Э. Баумана (Национальный Исследовательский Университет). В отборочных турах школьники решали задачи по физике и химии на первом этапе, и задачи из композитных технологий на втором. Участники должны были верно ответить, какова роль армирующего наполнителя в композиционном материале, в какой последовательности необходимо укладывать вакуумный пакет при технологическом методе вакуумной инфузии, по какой формуле можно определить модуль упругости композиты, а также решали задачи, посвященные тепловым эффектам и механике многослойного композита. Финалисты профиля, разбившись на команды, самостоятельно разработали конструкцию кессона крыла среднемагистрального пассажирского самолёта из композиционных материалов. Задача финала согласована и разработана совместно с партнерами профиля: ПАО «ОАК», ООО «Ниагара» и ООО «Викрон».

В основу профиля легли следующие технологические барьеры НТИ:

\begin{itemize}
    \item Силовые конструкции планера нового типа, в том числе: из композиционных материалов, новые композиционные материалы.
    \item Апробации тестового оборудования, разработанного для тематических учебных центров и разработчиков НТИ.
    \item Увеличение скорости разработки материалов и обеспечение возможности оперативного изменения свойств материалов под требования проектируемых конструкций, с учетом необходимости снижения себестоимости производства до уровня лучших мировых образцов (минимум на 20\%).
\end{itemize}

Также задание финала базируется на следующих технологических задачах:

\begin{itemize}
    \item Разработка элементов конструкций БПЛА с увеличенной грузоподъемностью, продолжительностью и дальностью полета.
    \item Разработка стрингерных панелей для композиционного крыла ближне-среднемагистрального самолета, элементов конструкции ПАК ДА.
    \item Разработка методик и инструментов создания новых материалов или конструкций из них, позволяющие снизить себестоимость производства по сравнению с материалами/конструкциями с аналогичными свойствами.
\end{itemize}

\putImgWOCaption{10cm}{history/info/composite/1}

\begin{center}
    На фото: Участники профиля «Композитные технологии» за работой
\end{center}

\section*{Работа с участниками}

Организаторы Олимпиады заинтересованы в дальнейшем сопровождении ее участников. Практика показывает, что школьники  –  участники Олимпиады НТИ также заинтересованы в дальнейшем сотрудничестве. В организации заключительного этапа Олимпиады НТИ 2018/2019 учебного года в качестве волонтеров приняли участие победители и призеры Олимпиады НТИ 2017/2018 учебного года, студенты первых курсов из различных регионов России. Участники заключительного этапа 2018/2019 учебного года из числа учеников одиннадцатого класса также выразили желание принять участие в организации олимпиады и подготовке участников в качестве волонтеров.  

В 2018/2019 годах число партнерских мероприятий Олимпиады увеличилось: на странице \url{http://nti-contest.ru/participants/posle_finala/} представлен список мероприятий, организаторы которых специально приглашают участников Олимпиады и дают им бонусы при конкурсном отборе.

Так, члены команд-победителей финалов Олимпиады были приглашены на образовательный интенсив «Остров 10-22», проходящий в Сколково летом 2019 года.  На авиационную смену в «Артеке» получили приглашение лучшие участники профиля «Беспилотные авиационные системы». Отбор на июльскую проектную смену в Образовательный центр «Сириус» предполагает дополнительные баллы для призеров и победителей Олимпиады НТИ.

\section*{Партнерство с инженерными соревнованиями}

Оргкомитет Олимпиады НТИ, в свою очередь, ежегодно утверждает перечень инженерных мероприятий и конкурсов, победители которых, могут принять участие в заключительном этапе Олимпиады, минуя отборочные. В 2016/2017 учебном году таковыми мероприятиями являлись: IT-хакатон GoTo, инженерно-конструкторские школы «Лифт в будущее»,  всероссийский форум «Будущие интеллектуальные лидеры России» и World Skills High Tech. 

В 2017/2018 учебном году льготы предоставлялись победителям мероприятий: всероссийский форум «Будущие интеллектуальные лидеры России», чемпионат \linebreak «WorldSkills Abu Dhabi» и «World Skills High Tech (Junior)», Воздушно-инженерная школа МГУ, региональный этап международных соревнований по подводной робототехнике «Russia Far-East MATE ROV Competition», Всероссийская Робототехническая Олимпиада.

В 2018/2019 году напрямую во второй этап Олимпиады получили доступ победители Региональных чемпионатов WorldSkills Junior Russia, Всероссийской робототехнической олимпиады, Олимпиады «Шаг в Будущее», Russia Far-East MATE ROV Competition, Russian Self-Driving Challenge, трека «Микробный топливный элемент» конкурса icet2018.ru, конкурса «Энергопрорыв» 2017/2018, Всероссийской олимпиады по 3D технологиям «Робофинист», Олимпиада «Кибервызов» компании Ростелеком, проектных смен Образовательного центра «Сириус» и всероссийских олимпиад школьников 1-3 уровней.

\section*{Подготовка участников}

Для вовлечения участников в олимпиаду были разработаны «Урок НТИ» и «Демо-этап» олимпиады, благодаря чему участники могли определиться с выбором профилей и попробовать свои силы.

«Урок НТИ» (\url{http://nti-contest.ru/ntilessonteacher/}) – это инициатива Олимпиады НТИ, проходившая в сентябре 2018 года и направленная на распространение информации об НТИ среди школьников и привлечение их к Олимпиаде НТИ через проведение уроков и занятий в школах и учреждениях дополнительного образования. Учебный материал для проведения «Урока НТИ» сформирован в виде конструктора, с помощью которого учителя могли собрать урок по теме НТИ. Урок позволяет познакомить учащихся НТИ и  с профилями Олимпиады НТИ, организовать практическую работу по решению задач в рамках выбранного профиля. Для участия в проекте «Урок НТИ» зарегистрировалось 2185 педагогов.

Демо-этап Олимпиады НТИ (\url{https://stepik.org/course/24389/syllabus}) – это публикация задач олимпиады в открытом доступе. Демо-этап создан для знакомства с задачами по профилям олимпиады, тренировки и испытания собственных знаний и умений решать непростые инженерные задачи.  Прежде чем определиться с участием в олимпиаде и выбором профиля,  потенциальные участники и их наставники могут познакомиться с задачами и выбрать наиболее интересный для себя профиль.

Чтобы участники могли восполнить недостаток практических компетенций и изучить оборудование, на котором им предстоит работать на заключительном этапе Олимпиады НТИ, разработчики направлений представляют методические материалы для самостоятельной практики и самоподготовки, проводят вебинары для участников и педагогов с ответами на вопросы и подбирают подготовительные курсы, совместно с площадками подготовки проводят хакатоны для участников с возможностью попробовать на практике фрагменты финальной задачи. 

Команды разработчиков профилей с целью эффективной подготовки к Олимпиаде НТИ создали видео разборы задач 2 этапа, которые доступны на канале Олимпиады НТИ, \url{https://www.youtube.com/channel/UCZV1CNpOrDNj7tuWuf35lgw/playlists} в 2018 году разработан курс (веб-сайт: \url{https://stepik.org/course/15697/syllabus}) по подготовке школьников к Олимпиаде НТИ на основе контента олимпиады 2017/2018 учебного года. Курс содержит задачи предметных треков 1 и 3 этапа по предметам: математика, физика, информатика, химия и биология и задачи 2 этапа по профилям олимпиады. Курс может использоваться наставниками и самими участниками для подготовки к олимпиаде следующего года. Формат курса максимально приближает участников к реальным условиям олимпиады.

Все указанные материалы находятся в свободном доступе и размещены на официальном сайте олимпиады, на страницах профилей в разделе «Материалы для участников». 

Материалы по профилю «Композитные технологии»: \url{https://nti-contest.ru/profiles/compmat/}

Олимпиада НТИ является промежуточным итогом работы по реализации дорожной карты НТИ «Кружковое движение»: подготовка к ней велась в фаблабах, ЦМИТах, детских технопарках, на базе активных школ и лицеев, центров дополнительного образования по всей России. Рабочая группа «Кружковое движение» НТИ направлена на развитие технологического сообщества, объединяющего школьников и студентов, ориентированных на инженерную деятельность на рынках НТИ, самодеятельных технических энтузиастов, лидеров технологических кружков, разработчиков педагогических технологий, технологических предпринимателей, популяризаторов науки и технологий.

\section*{Популяризация Олимпиады НТИ}

Олимпиада НТИ широко освещается в различных средствах массовой информации (телевидение, печатные издания, электронные издания). В период с 15 августа 2018 года (начало подготовки к регистрациям) до 3 апреля 2019 года, по данным Медиалогии, Олимпиада НТИ упоминалась в СМИ 2 301 раз, из них 633 раза на федеральном уровне, 1655 на региональном уровне, 13 на зарубежном уровне. 

Во время проведения отборочных этапов Олимпиада НТИ освещалась в федеральных, массовых, родительских, образовательных и иных медиа («ИТАР-ТАСС», «РИА-Новости», «Интерфакс», «Такие Дела», «Летидор», «Дети.Мэйл.ру», «Индикатор», «Занимательная робототехника», «Чердак.Ру», «Habrahabr», «Rusbase», \linebreak «Учёба.Ру»), официальных образовательных порталах и порталах органов государственности власти в регионах  (Новосибирск, Санкт-Петербург, Великий Новгород, Иннополис, Томск, Владивосток, Калининград, Тюмень, Курск, Курган, Тамбов, Мурманск, Новогород, Вологда и т.д.), в печатных изданиях («Российская газета», «Известия»). Радио «Медиаметрикс», программа «Выбор Родителей» под руководством автора самого большого блога для родителей в России.  Кампания по привлечению шла также в научно-популярных группах и группах вузов и площадок партнеров (МАИ, НГУ, Абитуриент НГУ,  ДВФУ, Школьники ДВФУ, Абитуриенты ДВФУ, Мурманский Арктический государственный университет, СибГУ им. М.Ф. Решетнева, Кванториум, АНО ДТ Красноярский кванториум,  НовГУ, ТПУ, абитуриенты ТПУ, МГППУ, Московский Политех, школа Летово, технопарк Академгородка, Сколтех, Иннополис, группы довузовского управления университета Иннополис, Политех Петра, ИТМО, ИРНИТУ, студенты ИРНИТУ, АУ УР РЦИиОКО, Детский технопарк INGENERIKA, Инкубатор Профи, Центр компетенций для детей Поколение 2035, Лаборатория НГУ Инжевика, Чеченский государственный университет, Айти школа Орбита, Фонд Книту, Фонд Золотое сечение, ЦМИТЫ Коптер, Ноосфера, Рекорд, Уникум, Stem-Байкал, Роболатория, Академия Технолаб, Образовательный проект для подростков Tula Teens ,  Проектория, ЯКласс, Фаблаб Политех и другие).  Заключительный этап Олимпиады НТИ в 2018/2019 учебном году проходил при участии журналистов таких печатных изданий, как «Российская газета», «Известия»; федеральных телевизионных каналов («Россия 24» (6 репортажей), «Общественное телевидение России»), федеральных новостных агентств («РИА-Новости», «ИТАР-ТАСС», «Интерфакс») научно-популярных порталов  «Rusbase», «Habrahabr», «Индикатор», «Такие Дела», радиостанций («Радио России», «МедиаМетрикс») Профильные издания освещали соответствующие направления Олимпиады НТИ («Крылья Родины» – «Беспилотные авиационные системы»). В ходе финалов Олимпиады НТИ были инициированы события, вызывающие дополнительный интерес как со стороны участников, так и со стороны СМИ. Так, например в рамках финала в Новосибирском государственном университете, участники встретились с Нобелевским лауреатом Хироси Амано, информация об этом событии была распространена ведущими федеральными агентствами и телеканалами. Разработка победителей профиля «Нейротехнологии», привлекла внимание известной актрисы Екатерины Варнавы, которая написала о своих впечатлениях в блоге с аудиторией 5 млн. 400 тысяч человек, позитивную реакцию на ее пост о победителях олимпиады продемонстрировали больше 75 тысяч пользователей. 

Широкое освещение мероприятий заключительного этапа имеет своей целью распространение информации среди потенциальных участников Олимпиады НТИ будущего года – учеников 7-11 классов и направлено на привлечение талантливых школьников со всей России и активное участие их родителей. В минувшем году была проведена большая работа с целевой аудиторией родителей, чьи дети учатся в 7-11 классах (появилась собственная передача «Выше среднего» на радио Медиаметрикс, регулярно выходят материалы на портале для активных родителей «Летидор», были инициированы эфиры в передача автора самого большого блога для родителей в России (1,6 млн.человек). 

Для привлечения внимания участников к конкретным профилям Олимпиады НТИ ведется точечная работа по освещению их разработок и задач. Инициированы эфиры на радио «Медиаметрикс», тексты в таких медиа как «Rusbase», «Понедельник», «Executive», «БОСС».

В отдельное направление выделена работа с финалистами Олимпиады НТИ с особенными достижениями. Регулярно, а не только в период проведения финалов, инициируются и выходят публикации в таких медиа как: «Российская Газета», «Известия», «Такие Дела», «Индикатор», «RusBase», «Летидор», «Дети Мэйл ру», радио «Медиаметрикс», запущен сервис подкасты в социальной сети ВК, его героями становятся как финалисты, так и разработчики профилей, партнеры, учредители и организаторы Олимпиады НТИ. 

Список лучших материалов об Олимпиаде: \url{http://nti-contest.ru/publications/}.

\chapter{Профиль «Композитные технологии»}

Задача командного тура финала профиля «Композитные технологии»
Олимпиады Национальной технологической инициативы состоит в
разработке конструкции и изготовлении сегмента крыла
среднемагистрального самолёта из композиционных материалов,
выполненного по кессонной схеме. За три дня участники должны разработать
конструктивно-компоновочную схему своей конструкции, определиться с
применяемыми материалами и изготовить изделие, полностью отвечающее
требованиям технического задания.

В основе композитных технологий лежат физика и химия, т.к.
структура композиционных материалов образована как физической, так и
химической связью компонентов. Поэтому на первом, предметном этапе
профиля участники решали задачи именно по физике и химии.
Вопросы второго, командного этапа профиля разрабатывались таким
образом, чтобы теоретически подготовить участников к работе в финале,
чтобы у них сформировалось четкое понимание того, что такое
композиционный материал, как он работает с точки зрения прочности, какая
роль у того или иного компонента.

Самый первый вопрос «Что из перечисленного отличает
композиционный материал от традиционных?» является очень важным, т.к.
затрагивает принципиальное отличие композитов от привычных нам
материалов. Именно наличие границы между фазами, т.е. различными
структурами в композите накладывает на него как ограничения, так и
преимущества, а также особенности производства, с чем участники
столкнулись во время финала.

Также во время финала участники должны были выбрать материал, из
которого им предстояло изготавливать свою конструкцию. Сделать это
можно было с помощью правила смеси для композита, и соответствующий
вопрос был предложен во втором этапе.

На втором этапе участники решали задачу расчета прочности образца
при трёхточечном изгибе, т.к. именно такой вид испытаний изделий
участников был реализован во время финала.
Помимо этого, участники во время командного этапа познакомились с
понятием адгезии, видами связующих компонентов, типом течения
эпоксидных связующих, классификацией композитов и способами их
исследования.

В конце второго этапа участники решали достаточно сложные задачи,
посвященные анализу слоистых материалов и конструкций.
Композиционный материал, в большинстве случаев, является слоистым, и
это требует от конструктора знания особенностей расчета слоистых систем.

Не представляет особой сложности рассчитать, например, теплопроводность
металла определенной толщины, однако в случае трёхслойной композитной
панели задача усложняется. Еще больше трудностей возникает при оценке
жесткости многослойных систем. Если у металла один параметр отвечает за
жесткость, и для определения модуля упругости нужно решить уравнение с
одним неизвестным, то у слоистого композита за это отвечает матрица с
девятью константами жесткости.