\chapter*{История ОНТИ}


\section*{Профиль <<Передовые производственные технологии>>}

Участие в олимпиаде по профилю «Передовые производственные технологии» (ППТ) предполагает комплекс знаний и умений учащихся разработать и изготовить (относительно) сложное техническое изделие, включая:
\begin{itemize}
    \item Умение сконструировать изделие в САПР.
    \item Умение использовать технологии “быстрого прототипирования” (3D-печать, лазерная резка, фрезерование) для изготовления деталей устройства по разработанным 3D-моделям. 
    \item Умение использовать программируемую электронику (например, на базе микроконтроллера Ардуино) для “оживления” созданного устройства.
    \item Знание элементов робототехники и/или машинного зрения для повышения автономности и функциональности устройства.  Предполагается умение программировать на любом высокоуровневом языке (например, Python).
\end{itemize}

Чтобы подчеркнуть специфику профиля, само разрабатываемое изделие зачастую является разновидностью координатного устройства с ЧПУ.  Таким образом, участники профиля должны показать как свое умение использовать цифровые технологии производства, так и понимание принципов работы и умение сконструировать и изготовить действующую модель несложного устройства такого класса.

Команда по профилю ППТ состоит из 3-х человек, их роли условно обозначены как конструктор-технолог, электронщик, программист.  На практике, крайне желательно, чтобы компетенции членов команды существенно перекрывались, и во время соревнований команда могла перераспределять ресурсы между этими ролями.

\textbf{Первый отборочный этап} профиля «Передовые производственные технологии» является общеобразовательным. Участники решают задания по физике и информатике. Высокий уровень знаний по данным предметам необходим для продолжения участия во втором этапе, однако сам по себе недостаточен для успешного решения как заданий 2-го этапа, так и командного задания финального этапа. 

\textbf{Во втором этапе} участникам предлагается задание, соответствующее профилю ППТ, которые не только проверят готовность участников к финалу, но и помогут к нему подготовиться. Задание 2-го этапа состоит из задач, которые решаются на платформе Stepik, и практической части. В задачах на платформе Stepik имеются разделы для конструктора (проектирование в САПР) и для электронщика-программиста (задачи по основам электроники и по программированию Ардуино). 

Практическую часть задания 2-го этапа можно рассматривать как фрагмент либо сильно упрощенное подобие финальной задачи. Участникам предлагается спроектировать и изготовить несложное устройство, а затем запрограммировать его перемещение (в олимпиаде 2018-2019 учебного года это была модель мостового крана с перемещаемой кареткой).

Успешно выполнив все компоненты сильно растянутого по времени задания 2-го этапа, команда должна быть в целом подготовлена к уровню сложности задания финального (очного) этапа.

\textbf{Командное задание финала} заключалось в конструировании и изготовлении заданного автоматического устройства. Например, в олимпиаде 2018-2019 учебного года участникам нужно было разработать и построить модель автономного мостового крана для сортировочной станции железной дороги. Кран должен двигаться по крайним путям сортировочной станции (на полигоне - по специальным направляющим) и перемещать маркированные контейнеры заданным образом между платформами. Для определения типов контейнеров и позиций их перемещения должно использоваться «машинное зрение» с помощью камеры, размещенной на каретке мостового крана.

Выполнение этого задания включает в себя моделированием в САПР, изготовление его деталей на 3D-принтерах и лазерном станке, сборку и наладку электронной начинки, программирование контроллера устройства, а также написание управляющего приложения на ПК с использованием библиотеки машинного зрения OpenCV.
