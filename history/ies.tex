%\chapter*{История ОНТИ}

\begingroup
\pagestyle{empty}

\section*{Введение}

Профиль «Интеллектуальные энергетические системы» в 2018-2019 уч. году был посвящен моделированию энергетических систем ближайшего будущего, которые базируются на уже существующих по отдельности, но еще не работающих в комплексе, технологиях. Важное место в профиле занимали экономические модели, которые в настоящее время пока не распространены. Решение заданий по профилю требовали знаний школьного уровня по математике: теория вероятностей, геометрия, основы математического анализа. Кроме базовых школьных знаний и навыков для решения задач профиля также требовалось самостоятельное освоение следующих тем: теория аукционов, теория игр, программирование на языке Python, основы численных методов в решении математических задач. Большинство задач требовали практической реализации их решения в виде или в составе программ — управляющего скрипта энергосистемы и вспомогательных инструментов для принятия решений.

Методические материалы для самостоятельного освоения были предоставлены в период проведения второго отборочного этапа через обучающие вебинары. Отборочные этапы и подготовительные мероприятия были выстроены так, чтобы в течение года учащиеся имели возможность подготовиться к финальной задаче профиля.

На первом отборочном этапе (дистанционном индивидуальном) участники решали предметные задачи по информатике и математике. Этот этап имеет своей задачей: отобрать тех, кто уверенно владеет школьной программой; дать возможность сильным участникам из разных городов найти друг друга и объединиться в команды; актуализировать знания именно по тем предметным областям, которые понадобятся в дальнейшем.

На втором отборочном этапе (дистанционном командном) участники формировали команды и работали над общим командным решением. Функций у этого этапа несколько: 
\begin{itemize}
\item отбор команд — сложность совокупности задач была рассчитана на  командную работу 3-4 участников; 
\item ознакомление участников с математическими методами, на которых будет построено решение задач финала; 
\item предъявление командам требований к уровню программирования, идентичных тем, что будут предъявлены им на финале.
\end{itemize}

Во время второго этапа участники решали задачи, которые отражали часть большой командной задачи финала и знакомились с базовыми концепциями. Работали с вероятностными прогнозами, графами, аукционами, задачами на оптимизацию. Параллельно со вторым отборочным этапом для участников были организованы хакатоны, где участники могли получить и отработать навыки, необходимые для решения финальной задачи и задач второго этапа: теория и практика аукционов, программирование на языке Python, совместная работа над программой.

Командная задача заключительного этапа представляла собой комплексную интегральную задачу, полное оптимальное решение которой достаточно сложно. Однако, полное приближенное решение была способна найти любая команда. Качество и сложность используемых приближений отражало глубину понимания, знаний и уровень способностей участников. Финальная командная задача заключалась в экономическом и энергетическом моделирование энергосистемы и конкуренции предложенных участниками решений. Участники разрабатывали экономические стратегии, работали с прогнозами погоды и потребления, писали скрипты по управлению энергообъектами. Система оценки была полностью автоматизирована и спроектирована таким образом, чтобы однозначно и численно оценить качество найденных и реализованных участниками приближённых решений. 

Помимо командной задачи на заключительном этапе участникам необходимо было решить индивидуальные задачи по предметам информатика и математика. Предметные задачи заключительного этапа преследовали следующие цели: объективно проверить индивидуальные знания участников; косвенно оценить индивидуальный вклад участников в результат командной работы. Вес этих задач в финальной оценке участников составлял 40\%.

\textbf{Второй отборочный этап} включает задачи, для решения которых достаточно школьных знаний и умений программы 10-11 класса и навыков использования школьных знаний для решения новых задач. Второй этап Включает 8 задач, целью которых является подготовка к командному туру на заключительном этапе.  Задания второго этапа очерчивают области предметных знаний, необходимых для участия в профиле. Методические рекомендации к данному этапу позволяют определить области знаний и навыков для самостоятельного изучения.

В таблице приведены номера задач второго этапа, элементы решения которых или, полученное в результате решения, понимание могут быть использованы для решения задач командного тура заключительного этапа.
\begin{center}
\small
\begin{longtable}{|p{2cm}|p{8cm}|p{5cm}|}
\hline
\textbf{№ задачи} & \textbf{№ задачи на командном туре в финале, в которой применимо} & \textbf{№ задачи на командном туре в финале, в которой применимо} \\
\hline
1& Нацелена на развитие навыков работы с вероятностями, мультиагентными системами, аукционами.

Для решения задачи необходимы разделы математики: теория вероятностей, элементы теории аукционов. Специальных навыков из информатики задача не требует, однако желателен уровень культуры программирования не ниже среднего.& Составление и адаптация стратегии для аукционов, система поддержки принятия решений на аукционе.\\
\hline
2& Нацелена на развитие алгоритмического мышления и понимания процессов, происходящих в системах альтернативной энергетики.

Для решения задачи желательна высокая культура программирования, специальных навыков из математики и информатики задача не требует.& Расчёт предельной цены объекта, работа с биржей электроэнергии, управляющие скрипты\\
\hline
3& Нацелена на развитие понимания теории игр.

Для решения из математики нужны элементы теории игр (турниры по Аксельроду). Специальных знаний из информатики задача не требует.& Система поддержки принятия решений на аукционе\\
\hline
4& Нацелена на развитие пространственного мышления и навыков решения оптимизационных задач.

Для решения требуются элементарные знания из стереометрии и тригонометрии, владение методом скорейшего спуска или поиска выпуклой оболочки множества точек на плоскости.& Оптимальное расположение солнечной батареи.\\
\hline
5& Нацелена на развитие владения теорией игр, алгоритмического мышления, навыков программирования.

Требует понимания критерия наилучшего гарантированного результата, владения техникой мемоизации.& Составление и адаптация стратегии для аукционов, система поддержки принятия решений на аукционе.\\
\hline
6& Нацелена на развитие навыков работы с большими массивами данных и алгоритмического мышления.

Специальных знаний по математике не требует.

Из информатики требует базовых навыков программирования.& Вычисление полного энергетического баланса на основании данных прогнозов, вычисление экономического баланса энергосистемы на основании данных прогнозов, управляющие скрипты\\
\hline
7& Нацелена на развитие навыков моделирования мультиагентных систем.

Специальных навыков из математики не требует.

Из информатики необходимы базовые навыки работы с графами.& Работа с биржей электроэнергии.\\
\hline
8& Нацелена на развитие алгоритмического мышления и навыков численного моделирования.

Из информатики требует среднего уровня навыков программирования.

Из математики требует понимания (хотя бы интуитивного) принципов работы критериев остановки численных алгоритмов.& Система поддержки принятия решений на аукционе, Вычисление полного энергетического баланса на основании данных прогнозов\\
\hline
\end{longtable}
\end{center}

Все задачи второго этапа решались на платформе Stepik. В заданное время всем участникам открывался доступ к задаче (условие и проверяющая система). Во всех задачах участники должны были написать программу и загрузить её текст на сервер. Большинство задач требовало базовых навыков программирования на языке Python версии 3, поскольку это является необходимой частью задания заключительного этапа.

Решение, загруженное любым участником команды, засчитывалось всем участникам команды. Неправильные попытки решения учитывались только при одинаковых баллах двух команд. Такой ситуации не возникло.

\textbf{Заключительный этап: индивидуальная часть}

\textbf{Задачи по математике}

В таблице приведены номера задач предметного индивидуального тура заключительного этапа, элементы решения которых или полученное в результате решения понимание могут быть использованы для решения задач командного тура в финале. Такое соотнесение навыков позволяет лучше учесть личный вклад в командный результат.

\begin{center}
\small
\begin{longtable}{|p{2cm}|p{6cm}|p{7cm}|}
\hline
\textbf{№ задачи}&\textbf{Знания и навыки, на выявление и развитие которых направлена задача}&\textbf{№ задачи на командном туре в финале, в которой применимо}\\
\hline
1& Решение задачи требует навыков работать с множественными условиями и системами уравнений.& Система поддержки принятия решений на аукционе, составление и адаптация стратегии для аукционов, управляющие скрипты\\
\hline
2& Решение задачи требует владения элементарной теорией вероятностей.& Вычисление полного энергетического баланса на основании данных прогнозов, система поддержки принятия решений на аукционе\\
\hline
3& Для решения задачи нужно владение элементарной теорией вероятностей, навыки комбинаторики и пространственного мышления.& Вычисление полного энергетического баланса на основании данных прогнозов, система поддержки принятия решений на аукционе, управляющие скрипты\\
\hline
4& Для решения задачи нужно владение элементарной теорией вероятностей и техниками работы с бесконечными системами.& Вычисление баланса энергорайонов энергосистемы, вычисление полного энергетического баланса на основании данных прогнозов\\
\hline
5& Решение задачи требует владения алгебраической геометрией и планиметрией.& Система поддержки принятия решений на аукционе \\
\hline
\end{longtable}
\end{center}

\textbf{Задачи по информатике}

В таблице приведены номера задач предметного индивидуального тура, элементы решения которых или, полученное в результате решения, понимание могут быть использованы для решения задач экспериментального тура в финале. Такое соотнесение навыков позволяет лучше учесть личный вклад в командный результат.

\begin{center}
\small
\begin{longtable}{|p{2cm}|p{7cm}|p{6cm}|}
\hline
\textbf{№ задачи}&\textbf{Знания и навыки, на выявление и развитие которых направлена задача}&\textbf{№ задачи на командном туре в финале, в которой применимо} \\
\hline
1& Для решения задачи необходимы разделы информатики и математики, посвященные следующим темам: системы счисления, десятичные дроби, рациональные и иррациональные числа.& нет\\
\hline
2& Для решения задачи необходимы разделы математики и информатики, посвященные следующим темам: решение задач типа «рюкзак», теория вероятностей.& Система поддержки принятия решений на аукционе, составление и адаптация стратегии для аукционов\\
\hline
3& Для решения задачи необходимы разделы математики и информатики, посвященные следующим темам: теория вероятностей, полный перебор, работа чтением данных из числовых массивов.& Вычисление полного энергетического баланса на основании данных прогнозов, система поддержки принятия решений на аукционе.\\
\hline
4& Для решения задачи необходимы разделы математики и информатики, посвященные следующим темам: циклы, условные операторы, работа с двумерными массивами.& Управляющие скрипты\\
\hline
5& Для решения задачи необходимы разделы математики и информатики, посвященные следующим темам: графы, свёртка дерева.& Вычисление баланса энергорайонов энергосистемы, вычисление полного энергетического баланса на основании данных прогнозов \\
\hline
\end{longtable}
\end{center}

\textbf{Заключительный этап: командная часть}

Командный тур заключительного этапа проводился на специально разработанном стенде-тренажере «ИЭС» (Подробное устройство стенда представлено в Приложении к материалам заданий). Участникам предстояло работать с моделью энергосистемы, в которой присутствует большое количество альтернативных источников энергии, накопителей энергии и широкие возможности для экономических решений. В работе над задачей участники изучали: физические и экономические параметры энергосетей; сильные и слабые стороны альтернативной энергетики, взаимосвязь инженерных и экономических решений и проблему качества и надежности электросетей.

Для решения командной задачи участникам требовались навыки и знания из школьной программы, а также, самостоятельно приобретенные на хакатонах, и навыки и знания, которые участники приобрели при решения задач второго этапа:
\begin{itemize}
\item Умение программировать на языке Python.
\item Умение работать со случайными величинами в программных вычислениях.
\item Умение решать базовые оптимизационные задачи.
\item Реализация решений математических задач в виде алгоритмов и программ.
\item Понимание правил и решения закрытого аукциона первой цены.
\item Командная работа над программным кодом.
\item Работа со сравнительно большими рядами данных в математических задачах.
\item Работа с рядами данных в алгоритмах.
\item Понимание принципов работы и характеристик ветрогенераторов.
\item Практическое использование теории вероятности.
\end{itemize}

Успешное решение задачи финала предполагает освоение и развитие следующих знаний и навыков:
\begin{itemize}
\item Умение работы в команде.
\item Умение самостоятельно выделять и формулировать подзадачи.
\item Понимание принципов оценки риска.
\item Программирование, в том числе на языке Python.
\item Работа с рядами данных, как аналитических, так и алгоритмических.
\item Работа с системами с инерцией.
\item Практическое использование теории вероятности, в том числе в программных вычислениях.
\end{itemize}

\clearpage
\endgroup