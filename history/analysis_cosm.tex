\chapter*{История ОНТИ}


\section*{Профиль <<Анализ космических снимков и геопространнственных данных>>}

Анализ космических снимков – сложная и комплексная отрасль современных знаний и технологий, которая требует одновременно:
\begin{itemize}
    \item глубоких знаний географии и смежных областей, понимание основных экологических закономерностей, социальных и экономических реалий конкретной территории; 
    \item умений и навыков работы с пространственными данными и географическими информационными системами, знакомство с основными источниками данных в интернете, умение искать специфическую информацию, а также производить статистические расчёты.
\end{itemize}

Поэтому на первом отборочном этапе проверялись знания и навыки участников по двум предметам – географии и информатике. 

Задачи первого этапа по географии были направлены на выявление у участников следующих знаний и навыков (или способностей их быстро освоить в случае необходимости):
\begin{itemize}
    \item Знаний в области физической географии, мира и отдельных регионов, особенно в части природных зон, растительности и ландшафтов, распределения и характера растительности в зависимости от различных физических факторов: рельефа, гидрологии, геологии, климата и пр.
    \item Знаний в области экономической географии и основных видов природопользования изучаемых регионов, воздействия деятельности человека на природные экосистемы и ландшафты.
    \item Понимания основных экологических закономерностей и функционирования экосистем и растительных сообществ, основных принципов строения лесных экосистем, первичных и вторичных сукцессий, лимитирующих факторов, воздействия человека на окружающую среду и природные экосистемы.
    \item Навыков поиска информации и источников данных в интернете, прежде всего, – картографической и пространственной информации.
    \item Умения пользоваться общедоступными картографическими порталами (интерактивными картами) в интернете – как картами общего профиля (типа Яндекс Карты, Google Maps и пр.), так и специализированными тематическими порталами, прежде всего, по лесной тематике. В частности уделялось внимание умению производить с помощью геопорталов простейшие измерения и расчёты, анализировать пространственную информацию.
    \item Навыков анализа найденной информации, умение сопоставлять и сравнивать информацию из разных источников.
\end{itemize}

Большинство задач на знание физической и экономической географии подбирались таким образом, чтобы ответы невозможно было найти простым поиском в интернете. В ходе решения задач участники также познакомились с общедоступными космическими снимками и распознаванием объектов на них.

Задачи первого этапа по информатике были направлены на выявление у участников, прежде всего навыков поиска и анализа специальной информации по теме профиля в интернете, а также на проверку умения программировать. Участникам предлагалось написать программы пересчёта географических координат, распознавания графических образов (аналог анализа космических снимков), а также анализировать статистические данные по лесному покрову разных стран.

Второй отборочный этап был направлен не только на отбор, но и на обучение участников  по ходу решения задач работе с географическими информационными системами и пространственными данными, включая космические снимки. Все задачи требовали работы с реальными пространственными данными, которые нужно было самостоятельно найти и получить из открытых источников. В заданиях использовались реальные космические снимки из открытых источников, которые используются исследователями и инженерами всего мира во «взрослых» проектах. В качестве программных инструментов предлагались использовать бесплатное программное обеспечение ГИС с открытым кодом, – прежде всего, QGIS и дополнительные модули к нему. Также участники могли выбрать любой знакомый им программный пакет или написать программу обработки самостоятельно. В ходе выполнения заданий второго этапа участники должны были освоить следующие навыки:
\begin{itemize}
    \item Поиск и отбор необходимых пространственных данных (электронных карт) из различных источников.
    \item Выборка пространственных объектов по определённым признакам.
    \item Редактирование и корректировка векторных пространственных данных.
    \item Перевод пространственных данных из одной системы координат в другую.
    \item Геообработка и пространственный анализ двух наборов данных.
    \item Расчёт площадей и пространственной статистики с помощью инструментов ГИС.
    \item Поиск и отбор необходимых космических снимков на специализированных порталах.
    \item Подготовка скачанных снимков к работе, объединение спектральных каналов в многоканальное («цветное») изображение и его визуализация.
    \item Визуальное дешифрирование – выделение объектов на космических снимках.
    \item Применение автоматических алгоритмов дешифрирования для выделения однородных объектов и классификации растительного покрова.
    \item Комбинирование информации из открытых геопорталов и результатов собственного анализа космических снимков.
    \item Проверка (верификация) результатов дешифрирования космоснимков и расчет ошибок.
\end{itemize}

Наиболее сложные задачи второго этапа непосредственно подводили участников к темам и объектам задач заключительного этапа.

На заключительном этапе олимпиады команды участников должны были решить реальную практическую задачу на обновление тематической карты по конкретному региону на базе анализа космических снимков. Для успешного решения задач финала им было необходимо применить все или большинство навыков, полученных в ходе решения задач второго отборочного этапа. При этом было необходимо применить для анализа знания из различных областей географии, особенно те, которые проверялись в ходе первого отборочного этапа по географии. При этом увеличившийся объём космических снимков, которые необходимо обрабатывать, и разнообразие выделяемых объектов не позволяли просто механически применить подходы, использованные в ходе второго этапа. Требовалось придумать новые методы работы и/или автоматизировать применение более простых методов. В то же время, задачи финала оставляли за участниками полную свободу выбора методов и алгоритмов. В том числе, возможность использовать визуальное дешифрирование вместо алгоритмов автоматической классификации, которое, в ряде случаев, может оказаться более эффективно.