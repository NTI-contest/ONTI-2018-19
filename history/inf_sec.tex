%\chapter*{История ОНТИ}


\begingroup
\pagestyle{empty}

\section*{Введение}

«Информационная безопасность» важная и востребованная в современном мире область знаний, которая развивается на стыке информационных технологий, физики и математики. Олимпиада по профилю «Информационная безопасность» проводилась в два этапа. Отборочный этап — дистанционно на специально разработанной для олимпиады платформе \url{https://cyberchallenge.rt.ru/}, и заключительный этап — очно. 

В отборочном этапе школьники решали задачи, решение которых опирается на глубокие знания информатики и математики, а также на знание современных IT-технологий:
\begin{itemize}    
    \item Задачи по криптографии (crypto);
    \item Задачи по стеганографии (stegano);
    \item Задачи по проведению программно-технической экспертизы и расследованию инцидентов (forensics);
    \item Задания по поиску и эксплуатации веб-уязвимостей (web);
    \item Задания по исследованию программ в условиях отсутствия исходного кода (reverse);
    \item Задания по программированию подсистем безопасности (professional program-ming and coding).
\end{itemize}    

Знания и навыки, приобретенные на отборочном этапе будут востребованы при решении задач финала. Недостающие знания участники могли получить на онлайн-курсах и из методических материалов, представленных на сайте олимпиады. Специально для участников были проведены вебинары, погружающие их в технологии и методы, используемые в направлении информационная безопасность. 

На заключительном этапе финалисты проводили аудит безопасности устройств умного города и автомобиля Tesla Model S, а также разрабатывали модели защиты и способы устранения данных уязвимостей. Особенностью профиля является эмуляция работы на реальной критической инфраструктуре, требующая совокупности технических компетенций, системно-целостного видения проблем обеспечения информационной безопасности, представления о природе возникновения типичных угроз, а также навыков практической реализации мероприятий защиты от них.


\clearpage
\endgroup