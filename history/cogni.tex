
%\chapter*{История ОНТИ}


\begingroup
\pagestyle{empty}

\section*{Введение}

Профиль «Когнитивные технологии» в 2018-2019 учебном году представил задачу дистанционного («мысленного») управления объектом оператором по заданному маршруту с несколькими сегментами. Задача предусматривала: 
\begin{itemize}
\item создание интерфейса мозг-компьютер, включая его тестирование; 
\item использование интерфейса мозг-компьютер для подготовки операторов (диагностика нужных качеств и обучение членов команды); 
\item отбор оператора с нужными качествами; управление объектом в «реальных» условиях. 
\end{itemize}

Данное направление требует знаний основ программирования  (основы машинного обучения) и психофизиологии человека. Система отборочных этапов Олимпиады НТИ по профилю позволяла участникам в течение года погрузиться в проблематику и подготовиться к решению задачи на заключительном этапе олимпиады.

В первом отборочном дистанционном этапе (индивидуальный) участникам были предложены задания по двум предметам — информатика (основы программирования) и биология (в рамках школьной программы).

Во втором отборочном дистанционном этапе участникам были предложены командные задачи по когнитивной науке, психофизиологии и программированию. Решение заданий второго этапа позволило участникам отработать необходимые навыки и способы решения, которые потребуются для решения задачи заключительного этапа, а также более подробно познакомиться с предметной областью задачи финала.

В заключительном очном этапе олимпиады участникам предстояло применить все ранее полученные знания для создания ИМК (интерфейс мозг-компьютер), включая его тестирование;  	использования ИМК для подготовки операторов (диагностика нужных качеств и обучение членов команды);  	отбора оператора с нужными качествами;    управления объектом в «реальных» условиях. Команды были сформированы по 3-4 участника в каждой. Каждый участник команды выполнял  определенные функции: инженер, программист и психофизиолог. 

Помимо командной задачи на заключительном этапе участникам необходимо было решить индивидуальные задачи по предметам — информатика и биология, которые схожи по темам с задачами первого этапа, но имели более высокую сложность.

\clearpage
\endgroup