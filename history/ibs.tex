\begingroup
\pagestyle{empty}

\section*{Введение}

Олимпиада по профилю «Инженерные биологические системы»  проводится по двум школьным предметам: химия и биология и в виде состязания команд школьников по решению практических инженерно-биологических задач.

\subsubsection*{Направление: Геномное редактирование}

Финальная задача профиля по направлению «Геномное редактирование» для 10-11 классов была посвящена анализу продуктов работы системы редактирования генома в культуре клеток. Участники должны были определить РАМ-последовательность, которая была использована для редактирования фрагмента ДНК. Для успешного решения финальной задачи участникам требовалось знание инструментов биоинформатического анализа, а также навыки работы в молекулярно-биологической лаборатории. 

Для подготовки участников к решению задач заключительного этапа, задания первого и второго этапов олимпиады были посвящены основам молекулярной биологии, биохимии, молекулярной генетики, биотехнологии. Участникам были предложены для ознакомления литературные источники, статьи, онлайн и видеокурсы, которые позволят получить недостающие знания. Решение заданий первого и второго этапов позволили закрепить эти знания. 

Для анализа последовательностей нуклеиновых кислот в настоящее время используют различные базы данных (например, NCBI и многие другие), интернет-сервисы (например, Tide и другие) и специальное программное обеспечение (например, Mega, uGene и другие). Для анализа секвенограмм и сайтов рестрикции плазмидной ДНК используют программу SnapGene Viewer. Использование этих программ и других IT-ресурсов позволили учащимся без посещения молекулярно-биоло-гической лаборатории и специального оборудования приблизиться к освоению методов биоинформатического анализа, без сомнений, являющихся практиками будущего. 

Освоение основ генетической инженерии, молекулярной биологии, биотехнологии во время подготовительных хакатонов позволили понять суть процессов, происходящих с нуклеиновыми кислотами в лаборатории, на практическом опыте применить полученные теоретические знания. Методы генетической инженерии и геномного редактирования сегодня активно используются в лабораториях по всему миру и, без сомнения, являются практиками будущего. 

Задания направления «Геномное редактирование» соответствуют дорожным картам Национальной технологической инициативы по подготовке специалистов рынков Хелснет и Фуднет. Победители и призеры олимпиады приглашаются на стажировки в лабораториях партнеров: МФТИ, ИХБФМ СО РАН, Биокад, Вектор-бест и других организациях. 

\subsubsection*{Направление: Агробиотехнологии}

Задания по направлению «Агробиотехнологии» для учащихся 10-11 классов были ориентированные на цитологию, микробиологию, работу c бактериями; задания для учащихся 9 классов были ориентированные на ботанику, зоологию, питательные цепи и перенос энергии в природе.   

Первый отборочный этап проводился индивидуально в сети Интернет (платформа Stepik), работы оценивались автоматически средствами системы онлайн-тестирова-ния. Для каждой возрастной группы (9 класс или 10-11 классы) предлагался свой набор задач по химии и биологии. В общей сложности, на решение задач первого отборочного этапа участникам выделялся ~1 месяц, в течение которого участникам предоставлялось 3 попытки. На решение задач по химии и ввод ответа предоставлялся 1 день. По биологии задачи открывались на 2-3 часа в первой половине дня, и во второй с целью охвата всех часовых поясов РФ. Ни один блок заданий не открывался повторно. В связи с этим, для реализации 3 попыток было  разработано по 3 варианта заданий по химии и по 6 вариантов по биологии. 
 
Второй отборочный этап проводился в командном формате в сети Интернет (платформа Stepik). Задания разрабатывались отдельно для каждой возрастной категории участников. Продолжительность второго отборочного этапа — 1,5-2 месяца. Работы оценивались автоматически средствами stepik. Задачи носили междисциплинарный характер и в упрощенной форме воссоздавали типовые инженерные и биологические задачи, которые участникам необходимо будет решать на финале.

Задачи разбивались на несколько блоков и открывались последовательно. На каждый блок заданий участникам давалась неделя. Поскольку второй этап являлся командным, решения принимались только от одного участника команды. Для этого на платформе Stepik было создано два параллельных курса. В одном публиковались только условия заданий и доступ предоставлялся всем участникам, прошедшим во второй тур профиля. Во втором публиковались условия с возможностью ввода ответа, и доступ предоставлялся только одному (или единственному) участнику от команды.

В 2018/2019 учебном году задачи второго этапа были разработаны для двух возрастных групп. Для 9 классов задания были посвящены проектированию и работе с аквапонными установками и замкнутыми системами. Для 10-11 классов задания были посвящены микробиологии и работе с нитрифицирующими бактериями. 

Заключительный этап олимпиады состоял из двух частей: индивидуальное решение задач по предметам (химия, биология) - предметный тур и командное решение инженерно-биологических задач - командный тур. На индивидуальное решение задач отводилось по 2 часа на один предмет. Для каждой возрастной группы (9 класс или 10-11 классы) предлагался свой набор задач по химии и биологии. Решение каждой задачи давало определенное количество баллов. Участники получали оценку за решение задач в совокупности по всем предметам данного профиля (биология и химия), которая в 2018/2019 учебном году вносила 40\% вклада в общий результат участника.

В командной части заключительного этапа участникам предлагалось  решать конкретную задачу, относящуюся к ключевым направлениям развития современной науки и реальных представителей биотехнологических компаний. 

В направлении «Агробиотехнологии» для двух возрастных групп участников предлагалось два разных уровня проработки задач. Финалисты 9-х классов работали на уровне отдельных организмов и систем. Участники 10-11-х классов решали задачи клеточного и молекулярного уровня. Подобное разделение было выбрано разработчиками профиля в связи со значительными отличиями школьных программ 9 и 10-11 классов. 

В 2018/2019 учебном году на заключительном этапе участники 9-х классов работали с системами замкнутого водоснабжения, состоящих из 4х модулей: аквакультура (рыба), гидропоника (растения), биофильтр (бактерии и фильтраты-беспозвоночные) и модуль с водными растениями. Участникам в ходе финальных испытаний необходимо было решать как практические задачи (работа с аквапонными установками, отслеживание параметров, балансировка систем, введение дополнительных модулей в систему), так и теоретические задачи (разработка систем автоматизации основных процессов, расчет оптимальных параметров систем и т.д.). 

Участники 10-11 классов решали задачу подбора оптимального микробиологического наполнения биофильтра аквапонных установок (с которыми работали девятиклассники). Участники разрабатывали оптимальную форму и наполнение модулей биофильтра, работали с нитрифицирующими бактериями, определяли оптимальный состав питательной среды.

Для успешного выполнения заданий обеих возрастных групп требовались глубокие знания в области биологии, навыки поиска и работы с открытыми источниками информации, а также умения решения инженерных задач.

Продолжительность работы над практическими задачами  составляли 4 рабочих дня. 



\clearpage
\endgroup