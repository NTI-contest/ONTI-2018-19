%\chapter*{История ОНТИ}

\begingroup
\pagestyle{empty}
\section*{Введение}

Тематикой профиля «Умный город» в 2018/19 учебном году было создание системы оптимизации управления процессами жизнеобеспечения целого города. Внедрение систем управления, как правило, осуществляется на нескольких уровнях: индивидуальное жилье, общегородская инфраструктура, промышленность.

Для успешного решения задач, связанных с оптимизацией систем управления, участникам необходимы глубокие знания по физике (электроника, схемотехника) и информатике (программирование микроконтроллеров, написание мобильных приложений). Заключительный этап олимпиады является логическим завершением подготовительного цикла, включающего в себя два отборочных этапа: индивидуальный и командный.

Первый отборочный этап проводился индивидуально on-line. Участники решали задания в рамках школьных предметов по физике и информатике. Целью данного этапа являлся отбор участников, обладающих базовыми знаниями в области указанных предметов и соответствующих разделов (информатика - основы программирования, физика - электричество, основы электроники).

Целью второго этапа являлось выявление наиболее подготовленных школьников для участия в очных финальных состязаниях, а также пробудить интерес к изучению тематики данного профиля. На втором этапе были предложены более сложные задания, которые наиболее эффективно решать именно в команде. В ходе данного этапа участники приобретали дополнительные знания, получали опыт командной работы, которая им потом пригодилась в ходе финального испытания, т.к. в подавляющем большинстве участник финала соревновались  сформированными на втором этапе командами. В заданиях второго этапа был сделан упор на актуализацию основ кинематики, электродинамики, оптики, радиоэлектроники и электрических цепей.  Также участникам были предложены задания, мотивирующие на изучение языков программирования, разработки алгоритмов и приобретение навыков декомпозиции. Задания второго этапа позволили выделить наиболее подготовленных участников, которые с прошли в заключительный этап.

В качестве подготовительных мероприятий к заключительному этапу участники могли принять участие в хакатонах, где имели возможность потренироваться в программировании микроконтроллеров, проверить в действии работу различных датчиков, с которыми им придется работать на финале.

В заключительном этапе участники должны были максимально применить в действии все полученные ранее знания и навыки. В рамках финальных испытаний по профилю «Умный город» командам было предложено решить задачи по следующим направлениям:
\begin{itemize}
\item автоматизация процессов сбора показаний приборов учета и управления поставкой ресурсов;
\item  обеспечение безопасности в детских учебных учреждениях;
\item  организация автоматических частных пассажироперевозок;
\end{itemize}

Данные задачи в той или иной степени уже реализуются в крупных городах. В частности, автоматизация ЖКХ является одной из наиболее широко внедряемых технологии цифровизации города, что подтверждает актуальность направления. Перед командами также стояла задача построить информационную сеть, которая является связующим звеном между всеми проектируемыми объектами города.

Кроме командной задачи участники должны были решить индивидуально задачи предметного тура по физике и информатике, которые имели более высокую сложность по сравнению с заданиями по предметам в отборочных этапах.

\clearpage
\endgroup