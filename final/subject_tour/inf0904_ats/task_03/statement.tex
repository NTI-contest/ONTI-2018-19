\assignementTitle{}{20}{}

В некотором районе города имеется $N$ светофоров. Каждый из них имеет 4 состояния: зелёный, жёлтый, красный, красный и жёлтый. Светофор умеет переключаться между ними только в таком порядке. Состояния зациклены, то есть после сигнала "красный и жёлтый" будет сигнал "зелёный". Каждый светофор хранит информацию о длительности каждого состояния в секундах.

В начале каждго часа (в момент времени 0) все светофоры переходят в начало состояния "красный", независимо от их предыдущего состояния. Требуется найти самое раннее время (в секундах), в которое все светофоры района находятся в состоянии "зеленый".

\inputfmtSection

В первой строке содержится число $N$ — число светофоров в районе. $1 \leq N \leq 100000$.

Далее следует $N$ строк, каждая из которых содержит в себе 4 числа GREEN ORANGE RED REDORANGE — время каждого состояния в секундах.

\outputfmtSection

Вывести самый ранний момент времени, в который ВСЕ светофоры будут в состоянии "зеленый". Если такого момента времени нет, то вывести -1. Отсчет моментов времени начинается с 0.

\sampleTitle{1}

\begin{myverbbox}[\small]{\vinput}
    2 
    3 1 2 1 
    4 2 1 2 
\end{myverbbox}
\begin{myverbbox}[\small]{\voutput}
    3
\end{myverbbox}
\inputoutputTable

\solutionSection

В данной задаче можно создать массив состояний, и писать в нем количество светофоров, которые горят зеленым. Так как цикличность ограничена часом, а интервал детектирования – секундой, то размер массива должен быть равен 3600 (количество секунд в 1 часе). Данный массив легкр заполнить во время итерации по светофорам и инкрементируя значения, если светофор горит зеленым. Затем в данном массиве ищется число, равное числу сфетофоров. Если это число есть, то это означает, что существует момент времени, когда все светофоры горят зеленым.

\includeSolutionIfExistsByPath{final/subject_tour/inf0904_ats/task_03}