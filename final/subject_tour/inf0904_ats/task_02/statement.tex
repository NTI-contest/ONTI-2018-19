\assignementTitle{}{20}{}

Для автоматизации сбора данных о инфраструктуре городов было запущено приложение, считывающее карту города и выводящее её в виде двумерного массива из . и \#, где \# — это фрагмент дороги, а . — любая другая поверхность.

Ваша задача состоит в том, чтобы по полученной карте определить число улиц в городе.

Улицей называется горизонтальная или вертикальная линия, состоящая из более чем одного участка дороги, ограниченная с начала и конца не дорожными участками.

\inputfmtSection

Числа $W$ и $H$ — ширина и высота карты. $1\leq W\leq 1000$, $1\leq H\leq 1000$.

$H$ строк, каждая длиной $W$, содержащие . и \#, где \# — это фрагмент дороги, а . — любая другая поверхность.

\textbf{\textit{Гарантируется, что:}}

\begin{enumerate}
    \item Первая и последняя строки содержат только . , а также первый и последний символ каждой строки - . .
    \item Каждая улица шириной в 1 символ.
    \item Все улицы либо горизонтальные, либо вертикальные.
    \item Длина каждой улицы больше 1.
    \item Никакие 2 улицы не соприкасаются более чем в 1 точке.
\end{enumerate}

\outputfmtSection

Одно целое число дорог.

\sampleTitle{1}

\begin{myverbbox}[\small]{\vinput}
    6 7
    ......
    ..#.#.
    .####.
    ..#.#.
    .##.#.
    ..#.#.
    ......
\end{myverbbox}
\begin{myverbbox}[\small]{\voutput}
    4
\end{myverbbox}
\inputoutputTable

\sampleTitle{2}

\begin{myverbbox}[\small]{\vinput}
    7 7
    .......
    .......
    .#####.
    ..#....
    .##....
    ..#....
    .......
\end{myverbbox}
\begin{myverbbox}[\small]{\voutput}
    3
\end{myverbbox}
\inputoutputTable

\includeSolutionIfExistsByPath{final/subject_tour/inf0904_ats/task_02}