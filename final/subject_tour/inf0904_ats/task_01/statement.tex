\assignementTitle{}{20}{}

В городе имеется несколько маршрутов общественного транспорта. Каждый из них содержит свой список остановок. Одинаковые остановки на разных маршрутах называются одинаково. Каждые 2 остановки маршрута соединены одной прямой улицей. Никакие 3 остановки не стоят на одной улице. Считается, что все улицы в городе двунаправлены и автомобили могут ездить как с остановки $A$ до $B$, так и обратно по одной и той же дороге.

Требуется определить МИНИМАЛЬНО возможное количество дорог в городе на основании полученных маршрутов.

\inputfmtSection

В первой строке $N$ — число маршрутов. $1\leq N\leq 10000$.

Далее идут $N$ строк с заглавными латинскими буквами, обозначающими остановки. Длина каждой строки не превышает 1000. Гарантируется, что остановок в городе не более 26. В маршруте имеется как минимум 2 различные остановки.

\outputfmtSection

Одно число — МИНИМАЛЬНО возможное количество дорог в городе.

\sampleTitle{1}

\begin{myverbbox}[\small]{\vinput}
    1
    ABCBADE
\end{myverbbox}
\begin{myverbbox}[\small]{\voutput}
    4 
\end{myverbbox}
\inputoutputTable

\sampleTitle{2}

\begin{myverbbox}[\small]{\vinput}
    2
    ABCDEFG
    GFEDCBA
\end{myverbbox}
\begin{myverbbox}[\small]{\voutput}
    6
\end{myverbbox}
\inputoutputTable

\solutionSection

Создаем таблицу переходов и заполняем её 0. Каждый раз, когда маршрут связывает две остановки, изменяем оба элемента массива, отвечающих за переходы туда и обратно, на 1. После занесения всей информации достаточно просуммировать все значения в таблице и разделить на 2, так как все дороги двунаправлены.

\includeSolutionIfExistsByPath{final/subject_tour/inf0904_ats/task_01}