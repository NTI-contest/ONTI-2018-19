\assignementTitle{}{20}{}

Городской отдел сбора статистики исследует загруженность общественного транспорта. Для этого в конце каждого дня каждый автобус отправляет информацию о средней загруженности транспорта.

Загруженность представляет собой число от 0 до 1, являющееся отношением среднего числа пассажиров в автобусе за этот день к максимальному числу мест. Эти данные обрабатываются на сервере в режиме получения ответов на запросы $A$ $B$, где $A$ и $B$ — позиции автобусов в рейтинге популярности. Рейтинг популярности является номером автобуса в списке всех автобусов, отсортированным по убыванию загруженности. На каждый запрос следует вывести среднюю загруженность всех автобусов, которые в рейтинге не ниже автобуса с большей позицией в запросе, но не выше автобуса с меньшей. Ваша задача - написать такую программу.

\inputfmtSection

$M$ и $N$ — количество автобусов и количество запросов. $1 \leq M \leq 100000$, $1 \leq N \leq 100000$.

$M$ строк, содержащих числа с плавающей точкой - средние загруженности автобусов.

$N$ строк, содержащих запросы типа $A$ $B$, где $A$ и $B$ — позиции автобусов в рейтинге популярности.

\outputfmtSection

$N$ строк, содержащих ответы на запросы. Каждая строка должна содержать единственное число с плавающей точкой, с точностью в 4 знака после запятой.

\sampleTitle{1}

\begin{myverbbox}[\small]{\vinput}
    4 5
    0.4
    0.2
    0.1
    0.3
    1 4
    2 4
    4 3
    3 2
    1 1
\end{myverbbox}
\begin{myverbbox}[\small]{\voutput}
    0.2500
    0.2000
    0.1500
    0.2500
    0.4000
\end{myverbbox}
\inputoutputTable

%\includeSolutionIfExistsByPath{final/subject_tour/inf0904_ats/task_04}