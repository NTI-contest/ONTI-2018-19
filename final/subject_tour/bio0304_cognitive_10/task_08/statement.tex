\assignementTitle{}{10}{}

В нервной системе передача сигнала между нейронами происходит с помощью сигнальных веществ — нейромедиаторов. Медиатор может действовать как возбуждающий или как тормозной, затрудняющий возбуждение. Ацетилхолин, глутамат, серотонин и норадреналин обычно действуют как возбуждающие нейромедиаторы. ГАМК, глицин обычно действуют как тормозные.

Для прекращения действия сигнала медиатор удаляется из “рабочей области”, закачиваясь в одну из клеток или расщепляясь на месте. Например, холинэстераза расщепляет ацетилхолин, а моноаминоксидаза — норадреналин, серотонин, дофамин и другие, похожие по химической структуре.

Эпилептический припадок – нарушение функций ЦНС, вызванное чрезмерной нейрональной активностью. При генерализованном клонико-тоническом эпилептическом припадке в процесс вовлечены все области головного мозга, вплоть до ствола.

В настоящее время существует достаточно много вариантов лечения эпилепсии.

Укажите, какими эффектами могут обладать препараты против эпилепсии:
\begin{enumerate}
    \item[А.] Антагонисты ГАМК
    \item[Б.] Антагонисты глутамата
    \item[В.] Блокада потенциал-зависимых натриевых каналов
    \item[Г.] Ингибирование обратного захвата серотонина
    \item[Д.] Ингибирование моноаминоксидазы
\end{enumerate} 

Перечислите \textbf{русские} буквы, соответствующие \textbf{верным} уверждениям, \textbf{в алфавитном порядке без разделителей}.

\answerMath{БВ.}