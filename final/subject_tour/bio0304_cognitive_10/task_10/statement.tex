\assignementTitle{}{3}{}

Известно, что забота о потомстве у крыс приводит к активации работы некоторых генов в клетках гиппокампа у потомства. Механизмом является снятие метилирования некоторых нуклеотидов в промоторах этих генов.

\textbf{\textit{Краткая справка:}}

Метилирование нуклеотидов не приводит к изменению последовательности ДНК, но метилированная последовательность по-другому считывается белками, регулирующими работу генома.

Отсортируйте утверждения от ошибочного к верному.

\begin{enumerate}
    \item Эти гены присутствуют как в ядрах клеткок мозга, так и в ядрах клеток других тканей
    \item Забота о потомстве приводит к изменению генотипа у детенышей
    \item Ингибирование ДНК-метилаз, работающих в области данных генов, приведет к эффекту, как при высоком уровне заботы о потомстве
\end{enumerate}

\commentsSection

Эпигенетические изменения не влияют на последовательность ДНК, а геном — это совокупность информации, записанной именно в ДНК. Ингибирование ДНК-метилаз может привести к описываемому эффекту, но эти же ферменты могут работать и в областях других генов, и мы не можем сказать, как это скажется в итоге. Наконец, все гены (за исключением генов иммуноглобулинов) одинаковы во всех клетках организма, хотя работают они в разных тканях по-разному — наиболее верное суждение. 

\answerMath{2, 3, 1.}