\assignementTitle{}{10}{}

Метод «patch-clamp» позволяет изолировать фрагмент клеточной мембраны с одиночным каналом с помощью специальной микропипетки. Более того, исследователи имеют возможность контролировать разность потенциалов (напряжение) между сторонами мембраны и измерять ток, создаваемый движением ионов через открытый канал. Вам представлена 8 записей зависимости тока (пА) через одиночные Na-каналы из мышечных клеток крысы от времени (мс)) с помощью данного метода. В ходе эксперимента на мембрану подавали напряжение (Vp), составляющее -60 мВ (подъем линии на рис. а). В остальное время на мембране поддерживалось напряжение, характерное для невозбужденного состояния (потенициал покоя, базовая линия на рис. а).  Исходя из представленных данных, мы напрямую можем сделать следующие выводы о функционировании Na-каналов (справочная информация: пико- – $10^{-12}$, См = Ом$^{-1}$):

\putImgWOCaption{10cm}{1}

Укажите верные утверждения:
\begin{enumerate}
    \item[А.] Могут самопроизвольно открываться при потенциале покоя
    \item[Б.] Отсутствует инактивированное состояние после открытия
    \item[В.] Проводимость одного порядка 20 пСм
    \item[Г.] Проводимость одного порядка 20 мСм
    \item[Д.] Ток $Na^+$ во всех наблюдаемых случаях направлен в одну сторону
\end{enumerate}

Перечислите \textbf{русские} буквы, соответствующие \textbf{верным} уверждениям, \textbf{в алфавитном порядке без разделителей}.

\answerMath{АВД.}