\assignementTitle{}{21}{}

Известно, что кошки мурлычут в разных ситуациях. Вы предполагаете, что они мурлычут, чтобы выпросить у хозяев еду. Другая гипотеза — они мурлычут всегда, когда голодны.

Вам нужно поставить эксперимент, позволяющий выяснить, какая из гипотез более верна. У вас и ваших друзей есть 10 кошек, за каждой из которых можно установить наблюдение. Допустим, вы уже знаете, в какое время суток и в каких условиях каждая из этих кошек бывает голодна, а когда — нет (т.е. уверены в каждый момент эксперимента, кошка голодна или сыта). Также допустим, что кошки не мурлычут по другим поводам. Опишите, как вы будете проводить свой эксперимент. Обоснуйте, зачем вы делаете то или иное измерение. Продумайте наперед 3 возможных исхода эксперимента и напишите их. Как бы вы интерпретировали результаты каждого из этих трех исходов?