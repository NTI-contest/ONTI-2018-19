\assignementTitle{}{8}{}

Синаптическая пластичность – это способность нейрона  регулировать эффективность синаптической передачи с другим нейронов под действием различных эффектов , которые можно разделить на 2 большие группы: кратковременные и долговременные.

Утверждения:
\begin{enumerate}
    \item[А.] Длительность эффекта от миллисекунд до минут, реже — часов
    \item[Б.] Увеличение эффективности выброса медиатора (вещества, передающие сигнал в синапсе между аксоном предыдущего и последующего нейронов) под действием повторного потенциала, проходящего по аксону, вследствие остаточной концентрации $Ca^{2+}$ внутри клетки, сформированной в результате недавнего предыдущего потенциала
    \item[В.] Впервые была описана на примере гиппокампа – части височной доли мозга, играющей важную роль в формировании памяти, эмоций, ориентации
    \item[Г.] Увеличение количества AMPA глутаматных рецепторов, для чего необходимо, а также формирование новых синапсов
\end{enumerate} 

Виды памяти:
\begin{enumerate}
    \item Кратковременная синаптическая пластичность
    \item Долговременная синаптическая пластичность
\end{enumerate}

Выберите, каким видам пластичности скорее соответствуют приведенные утверждения.

\begin{tabular}{|p{1.5cm}|p{1.5cm}|p{1.5cm}|p{1.5cm}|}
    \hline
    А&Б&В&Г\\
    \hline
    &&&\\
    \hline
\end{tabular}

В ответ запишите последовательность \textbf{4 цифр} из таблицы. Например, 1111.

\explanationSection

Миллисекунды - это кратковременные эффекты в масштабе мозга. Так как классификация довольно условна, то совокупность кратковременных ответов может захватывать и часовые масштабы. 
Утверждение б описывает явления нейрональной фасилитации, относящейся к кратковременным эффектам. Об этом можно догадаться по необходимости наличия недавнего предыдущего возбуждения
Долговременная синаптическая пластичность была впервые описана для гиппокампа. Об этом можно догадаться, т.к. там сосредоточены такие сложные и длительные процессы, как память, ориентация и прочее.

Увеличение количества рецепторов и формирование синапсов требуют времени на активацию экспрессии, трансляции и транспортировки множества белковые молекул, что занимает время. Поэтому это долговременные процессы.

\answerMath{1122.}