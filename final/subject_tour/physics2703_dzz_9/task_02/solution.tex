\solutionSection

\begin{enumerate}
    \item Определим площадь города:
    $$S=\frac{N}{\sigma}=\frac{180\cdot10^3}{20\cdot10^3}=9\text{ км}^2$$
    Найдём размер кадра (сторону прямоугольника a):
    $$a=\sqrt S=\sqrt{9\cdot10^6}=3000\text{ м}$$
    Тогда разрешение камеры составляет
    $$L=3000\text{ м }/2000\text{ пикселей} = 1.5\text{ метра}$$
    
    \answerMath{$1.5\text{ метра}$}
    %2
    \item Из подобия треугольников имеем, что проекция пикселя L связана с размером пикселя фотоприёмника:
    $$\frac{L}{\delta}=\frac{h}{f}$$
    Тогда для фокусного расстояния f имеем:
    $$f=\frac{\delta}{L}\cdot h=\frac{7.5\cdot10^{-6}}{1.5}\cdot500\cdot10^3=2.5\text{ м}$$
    
    \answerMath{$2.5\text{ м}$}
    %3
    \item Для обеспечения перекрытия нужно, чтобы период съёмки не превышал время пролёта стороны снимка подспутниковой точкой («тенью» от спутника на поверхности Земли). Тогда:
    $$\frac{1}{\nu_\text{кадр}}=\frac{2000\cdot L}{V_\text{пст}}$$
    Скорость подспутниковой точки (можно взять из ответа к задаче 9-1-2, высоты орбит одинаковы):
    \begin{eqnarray}\nonumber
    V_\text{пст}=V_\text{орб}\cdot\frac{R_\text{зем}}{R_\text{зем}+H}=\frac{R_\text{зем}}{R_\text{зем}+H}\cdot\sqrt{\frac{G\cdot M_\text{зем}}{R_\text{зем}+H}}=
    \\\nonumber
    =\frac{6370\cdot10^3}{6870\cdot10^3}\cdot\sqrt{\frac{6.67\cdot10^{-11}\cdot5.97\cdot10^{24}}{6870\cdot10^3}}=7059.18\text{ м/с}
    \end{eqnarray}
    Для кадровой частоты получаем:
    $$\nu_\text{кадр}=\frac{V_\text{пст}}{2000\cdot L}=\frac{7059.18}{2000\cdot1.5}=2.353\text{ Гц}$$
    
    \answerMath{$2.353\text{ Гц}$}
    %4
    \item Общий объём памяти под один снимок будет:
    $$I=2000\cdot2000\cdot10=4\cdot10^7\text{ бит}$$
    Тогда общее количество снимков, которое можно записать в память спутника, составит c учётом того, что  1 Байт=8 бит:
    $$n=\frac{N}{\frac{1}{8}\cdot I}=\frac{10\cdot10^9\text{ Байт}}{\frac{1}{8}\cdot4\cdot10^7}=2\cdot10^3\text{ снимков}$$
    Длина полосы определяется как:
    $$l=n\cdot a=2\cdot10^3\cdot3\cdot10^3=6\cdot10^6\text{ м}=6\cdot10^3\text{ км}$$ 
    
    \answerMath{$6\cdot10^3\text{ км}$}
    %5
    \item Полоса захвата спутника составляет $b=3000\cdot1.5=4500\text{ метров}$
    Тогда статистически для покрытия всей Московской области нужно 	совершить N витков:
    $$N=\frac{l}{b}=\frac{320\cdot10^3}{4500}=71.1\text{ витков}=72\text{ витка}$$
    Здесь мы учли, что за один виток спутник дважды пересекает широту Москвы, но при этом одно пересечение бесполезно, так как происходит на ночной половине витка.\\
    Период обращения спутника определяется как:
    \begin{eqnarray}\nonumber
    T_{sat}=\frac{2\cdot\pi\cdot(R_\text{зем}+h)}{V_\text{орб}}=\frac{2\cdot\pi\cdot(R_\text{зем}+h)}{\sqrt{\frac{G\cdot M_\text{зем}}{R_\text{зем}+h}}}=2\cdot\pi\cdot(R_\text{зем}+h)\cdot\sqrt{\frac{R_\text{зем}+h}{G\cdot M_\text{зем}}}=
    \\\nonumber
    =6.28\cdot(6370+500)\cdot1000\cdot\sqrt{\frac{(6370+500)\cdot1000}{6.67\cdot10^{-11}\cdot5.97\cdot10^{24}}}=5667\text{ c}
    \end{eqnarray}
    Тогда оцениваем общее время съёмки Московской области:
    $$T_{all}=T_{sat}\cdot N=5667\cdot72=408\cdot10^3\text{ c}=4.72\text{ суток}$$
    
    \answerMath{$4.72\text{ суток}$}
    \end{enumerate}