\assignementTitle{Спутниковая фотосъемка}{}{2}

Спутник движется по круговой орбите Земли, имеющей высоту h=500 км над уровнем моря. Масса Земли 
$5,97 \cdot 10^{24}$ кг, а радиус 6370 км.

\begin{enumerate}
    \item На спутнике размещена фотокамера с размером фотоприёмной матрицы 2000$\cdot$2000 пикселей. Каким будет 
    разрешение камеры, если она должна обеспечить съёмку города квадратной формы с населением $N=180$ тысяч 
    человек  за 1 кадр? Плотность населения в городах принять $\sigma=20$ тыс чел на км$^2$. Оптическая ось 
    камеры направлена вертикально вниз.
    \item Какое фокусное расстояние должен иметь объектив камеры, чтобы обеспечить разрешение, определённое в п.1. 
    при высоте орбиты  $h=500$~км? Размер пикселя фотоприёмника \linebreak $\delta=7,5$~мкм.
    \item Определите минимально допустимую частоту кадров для указанных выше условий съёмки, чтобы соседние 
    изображения имели перекрытие друг с другом. Камера ориентирована вертикально вниз, причём две оси кадра 
    расположена поперёк и вдоль направления полёта соответственно.
    \item Какую длину может иметь изображение-полоса, составленное из соприкасающихся снимков, если запас 
    памяти на спутнике составляет 10 миллиардов Байт?  На хранение одного пикселя в памяти отводится 10 бит.
    \item Оцените время, за которое спутник отснимет всю Московскую область. Считать, что орбита полярная 
    (проходит над полюсами Земли), а съёмка интересующих объектов производится только на дневной стороне витка 
    (при любых углах возвышения Солнца над горизонтом). Камера ориентирована вертикально вниз, причём две оси 
    кадра расположена поперёк и вдоль направления полёта соответственно. Ширина Московской области в направлении 
    запад-восток составляет $l=320$ км.
\end{enumerate}