\assignementTitle{Энергосистемы спутника}{}{2}

Спутник вращается по круговой орбите высотой $H=500$ км. Съёмка Земли выполняется с помощью оптико-электронной 
камеры. Фотоприёмник представляет собой матрицу размером 3000$\cdot$1800 пикселей, широкая сторона матрицы 
расположена поперёк направления полёта спутника. Размер пикселя матрицы составляет $\delta=12$ мкм. Для хранения 
одного пикселя изображения требуется 10 бит памяти. Радиус Земли 6370 км.

\begin{enumerate}
    \item Сколько снимков может сделать камера, если ёмкость её памяти, где хранятся снимки перед передачей на 
    Землю по радиоканалу,  составляет $N=67,5$ миллиардов байт.
    \item Какую максимальную площадь между сеансами передачи снимков на Землю может отснять спутник, если 
    фокусное расстояние объектива составляет $f=5$ м и оптическая ось направлена вертикально вниз?
    \item Наземная антенная станция может обеспечить приём полученных снимков, если спутник виден под углом не 
    менее $\theta=20$ градусов к горизонту. Станция находится под орбитой спутника и способна принимать информацию 
    со скоростью $p=51,5$ Мбит/с. Какой объём информации и сколько снимков может передать спутник на Землю за 
    один сеанс связи? 
    \item Какую площадь должны иметь солнечные батареи спутника, чтобы камера могла снять не менее 100 
    изображений на одном витке? Общая потребляемая мощность в режиме съёмки и последующей обработки составляет 
    100 Вт, обработка изображения занимает 0,5 секунд. Учесть, что за 1 виток спутник производит до 10 
    перенацеливаний, каждое из которых требует энергозатрат 169,507 кДж. КПД солнечных батарей $\eta=15\%$, 
    потерями в заряжаемом ими аккумуляторами пренебречь. В диапазоне длин волн, к которым чувствительны солнечные 
    батареи, поток света от солнца составляет $\Phi=400$ Вт/м$^2$. Принять, что солнечные батареи поворачиваются на 
    дневной стороне витка так, что ориентированы перпендикулярно солнечным лучам.
    \item Оценить, на сколько градусов нагреются кремниевые солнечные батареи спутника за время пролёта по 
    дневной стороне витка? Принять, что для всех длин волн они поглощают $\epsilon=80\%$ солнечного света и 
    являются серым телом, для которого применим закон Стефана-Больцмана. При этом серость тела также составляет 
    0,8 . Поток света от Солнца составляет $\phi=1300$ Вт/м$^2$  интегрально для всех длин волн. Начальная 
    температура 327 К, удельная теплоёмкость кремния 678 Дж/(кг$\cdot$К), поверхностная плотность батареи $\rho=1,5$ кг/м$^2$. 
    Принять, что солнечные батареи поворачиваются на дневной стороне витка так, что ориентированы перпендикулярно 
    солнечным лучам. Теплоотдача излучением осуществляется одинаково с обеих сторон батареи. Поглощением света 
    от Земли и Луны пренебречь. Постоянная Стефана-Больцмана составляет 5,67$\cdot$10$^{-8}$ Вт/(м$^2\cdot$К$^4$). 
    Принять, что изменение температуры намного меньше начальной температуры батареи.
\end{enumerate}