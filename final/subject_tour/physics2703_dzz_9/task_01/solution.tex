\solutionSection

\begin{enumerate}
    \item Составим уравнение на основе закона всемирного тяготения:
    $$\frac{m\cdot V_\text{орб}^2}{R_\text{зем}+h}=\frac{GmM_\text{зем}}{(R_\text{зем}+h)^2}$$
    Тогда для орбитальной скорости получаем
    $$V_\text{орб}=\sqrt{\frac{G\cdot M_\text{зем}}{R_\text{зем}+h}}=\sqrt{\frac{6.67\cdot10^{-11}\cdot5.97\cdot10^{24}}{(6370+500)\cdot10^3}}=7613.28  \text{м/с}$$
    
    \answerMath{$7613.28\text{ м/с}$}
    %2
    \item Угловая скорость спутника: 
    $$\omega_\text{спутн}=\frac{V_\text{орб}}{R_\text{зем}+h}=\sqrt{\frac{G\cdot M_\text{зем}}{(R_\text{зем}+h)^3}}$$
    Из этого находим скорость тени спутника по поверхности Земли
    \begin{eqnarray}\nonumber
    V_\text{тени}=\omega_\text{отн}\cdot R_\text{зем}=\frac{V_\text{орб}}{R_\text{зем}+h}\cdot R_\text{зем}=\sqrt{\frac{G\cdot M_\text{зем}}{(R_\text{зем}+h)^3}}\cdot R_\text{зем}=
    \\\nonumber
    =\sqrt{\frac{6.67\cdot10^{-11}\cdot5.97\cdot10^{24}}{(6370+500)^3\cdot 10^9}} \cdot6370\cdot10^3=7059.18\text{ м/с}
    \end{eqnarray}
    
    \answerMath{$7059.18\text{ м/с}$}
    %3
    \item За время между съёмкой изображений спутник пролетит расстояние L и при этом
    $$\frac{0.5\cdot L}{h}=\tg(\alpha)$$
    Тогда для времени между снимками имеем:
    $$T_\text{между}=\frac{L}{V_\text{орб}}=\frac{2\cdot\tg(\alpha)\cdot h}{V_\text{орб}}$$
    При этом автомобиль пройдёт расстояние S:
    $$S=u\cdot T_\text{между}$$
    Количество пикселей, на которое сместился автомобиль, составит:
    $$N=\frac{S}{d}=\frac{u\cdot T_\text{между}}{d}=\frac{u}{d}\cdot\frac{2\cdot\tg(\alpha)\cdot h}{V_\text{орб}}=\frac{10}{1.5}\cdot\frac{2\cdot0.1\cdot500\cdot10^3}{7613.28}=88\text{ пикселей}$$
    
    \answerMath{$88\text{ пикселей}$}
    %4
    \item Так как ограничения по угловой скорости вращения спутника нет, то считаем, что от угла 90 градусов до 45 градусов происходит разгон вращения, а от 45 градусов до 0 градусов - замедление вращения. Именно в таком режиме вращения врем поворота спутника минимально, угол в 45 градусов – половина от общего угла поворота. Здесь 0 градусам соответствует направление оптической оси камеры на Землю. Тогда:
    $$T_\text{разгон}=T_\text{замедл}=\sqrt{\frac{2\cdot(\frac{\pi}{180}\cdot45)}{\varepsilon}}$$
    Общее время переориентации спутника
    $$T_\text{поворот}=2\cdot T_\text{разгон}=2\cdot\sqrt{\frac{2\cdot\frac{\pi}{4}}{\varepsilon}}=2\cdot\sqrt{\frac{1.571}{0.01571}}=20\text{ c}$$
    
    \answerMath{$20\text{ c}$}
    %5
    \item При заданных условиях Землю можно считать плоской. 
    
    Тогда для угла крена $\theta$:
    $$\theta=\arctg\left(\frac{d\cdot\frac{1}{\sqrt 2}}{h}\right)=\arctg\left(\frac{125\cdot\frac{1}{\sqrt 2}}{500}\right)=10\text{ градусов}$$
    Угол тангажа равен углу крена, 10 градусов, так как по условию, город находится строго к юго-востоку от Москвы.
    
    \answerMath{$10\text{ градусов}$}
\end{enumerate}