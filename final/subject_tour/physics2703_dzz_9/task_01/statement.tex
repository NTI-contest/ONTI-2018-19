\assignementTitle{Ориентация спутников}{}{1}

Спутник движется по круговой орбите Земли, имеющей высоту $h=500$ км над уровнем моря. Масса Земли $5,97 \cdot 10^{24}$ кг,
а радиус 6370 км.

\begin{enumerate}
    \item Найти линейную скорость движения спутника по орбите.
    \item Найти скорость бега «тени спутника» на поверхности Земли (скорость подспутниковой точки) без учёта 
    вращения Земли.
    \item Для построения стереоскопических изображений небольших площадок спутник делает пару снимков местности. 
    Первый снимок получается, когда камера смотрит вперёд с углом отклонения по тангажу $+\alpha$, то есть до пролёта 
    над снимаемой площадкой. Второй снимок получается, когда камера смотрит назад с углом отклонения по тангажу 
    $-\alpha$, то есть  после пролёта над снимаемой площадкой. Отклонение от вертикали производится только в 
    плоскости орбиты. В плоскости, перпендикулярной направлению полёта спутника, никаких поворотов не выполняется.
    \item Известно, что ось камеры должна быть отклонена от вертикали по тангажу на $\alpha=5,729$ градусов. 
    Разрешение (проекция пикселя) каждого из снимков составляет $d=1,5$ метра. На сколько пикселей будут сдвинуты 
    изображения автомобиля, движущегося по прямой дороге перпендикулярно направлению полёта спутника со скоростью 
    $u=36$ км/ч?

    На спутнике размещена фотокамера, предназначенная для съёмки поверхности Земли. При съёмке в основном режиме 
    съёмки оптическая ось объектива должна быть направлена вертикально вниз. Однако незадолго до подачи команды 
    на съёмку в центре управления полётами обнаружили, что объектив камеры направлен перпендикулярно плоскости 
    орбиты. Для возвращения спутника в исходное состояние необходимо повернуть его вокруг продольной оси. 
    Сколько времени займет переориентация спутника, если система управления может обеспечить угловое ускорение 
    $|\epsilon|=0,01571$ радиан/с$^2$ при перевороте спутника вокруг его продольной оси? Принять, что максимальная 
    угловая скорость вращения спутника очень большая. Изначально спутник не вращается вокруг продольной оси, 
    во время съёмки этого тоже не должно быть.
    \item Орбита спутника полярная, то есть он движется с севера на юг. При подлёте к Москве по полярной 
    орбите возникла срочная необходимость провести съёмку города Луховицы, находящегося в $d=125$ км строго к 
    юго-востоку от центра Москвы (по прямой на поверхности Земли). Естественно, что осуществить это можно только 
    при помощи наклона линии визирования (съёмка с креном и тангажом). Определите требуемые углы крена и тангажа, 
    если первоначально оптическая ось камеры была направлена вертикально вниз. Принять, что спутник выполнил 
    перенацеливание к моменту пролёта над центром Москвы.

    При отклонении по крену имеет место вращение оси объектива в плоскости, перпендикулярной направлению полёта 
    спутника. Угол крена отсчитывается от плоскости орбиты. 

    При отклонении по тангажу имеет место вращение оси объектива в плоскости орбиты. Угол тангажа отсчитывается 
    от плоскости, перпендикулярной направлению полёта.
\end{enumerate}