\assignementTitle{Задача 1. Стевина на вас нет!}{20}{}

Десятичные дроби, которыми мы пользуемся, не появились сами, а были изобретены — их придумал в конце 16 века математик Симон Стевин.

Чтобы найти десятичную дробь для числа, меньшего единицы, мы делим единичный отрезок на 10 одинаковых частей, и записываем номер отрезка (начиная с нуля), в который попало число. Затем этот отрезок снова делим на 10 частей, и так и далее, пока не получим достаточную точность. Если число больше единицы, то похожий процесс можно проводить и в обратную сторону, тогда мы найдём десятичные цифры слева от десятичной запятой.

В десятичных дробях, к которым мы привыкли, все цифры, стоящие на фиксированной позиции, отвечают за отрезки одинаковой длины, а позиция цифры отвечает за масштаб — при смещении вправо по записи числа масштаб увеличивается в 10 раз на один шаг.

Но числам не обязательно быть такими! Каждая цифра может быть особенной и отвечать за отрезок своей, особой длины!

Выберем следующие 10 цифр особенными:

\putImgWOCaption{12cm}{1}

(все цифры в таблице — нормальные, безликие десятичные).
Вам нужно будет сконвертировать набор из 20 десятичных чисел в особенные.

\inputfmtSection

20 вещественных чисел, по одному в каждой строке.

\outputfmtSection

20 чисел в такой же последовательности, но записанные в особенной форме.

\solutionSection

Для чисел, меньших единицы, цифры находятся таким же процессом, что и для десятичных: делением отрезка на десять и нахождением, в какой попадает число. Отличием является только то, что длина отрезка уменьшается на как степени 10, а менее предсказуемо, что, однако, на работу такого алгоритма не влияет.

Для чисел, больших единицы, нужно найти, сколько степеней «десяти» в них помещается. Нетривиальный момент в том, что «единица» составляет не 1/10, а 0.2265234857 от «десяти». Соответственно, основание искомой степени составляет не 10, а 4.41455329406. Нужно найти, при делении на какую степень 4.41455329406-ти число становится меньше единицы, разделить его на 4.41455329406 в этой степени, преобразовать число, а затем сдвинуть десятичную запятую вправо на число символов, равное найденной степени.

\includeSolutionIfExistsByPath{final/subject_tour/inf1503_ies/task_01}

\answerMath{Для решения нужно верно преобразовать все 20 случайно заданных чисел.}