\assignementTitle{Задача 2. Трудный выбор}{10 + 14}{}

Вы отвечаете за техническое обслуживание и модернизацию местной электростанции. Пришло время очередной модернизации, и, открыв корпоративную почту, вы увидели набор дорогостоящих предложений по модернизации отдельных модулей электростанции.

С помощью специалистов вы определили вероятность отказа модуля за расчётный период без модернизации и с ней для каждого предложения. Вероятности независимы между предложениями, но в случае отказа одного из модулей регистрируется авария на всей электростанции.

Вам необходимо минимизировать вероятность аварии, при этом уложившись в бюджет на модернизацию.

Подсчитайте, во сколько раз уменьшится вероятность аварии при наиболее рациональном наборе принятых предложений.

\inputfmtSection

Первая строка — количество предложений $N\space(7\leq N \leq 15)$ и бюджет $M$  в млн.руб. $(70\leq M \leq 120)$ через пробел. Далее $N$ строк с тремя числами через пробел: стоимость предложения в млн. руб., вероятность отказа модуля без реализации и после реализации предложения.

\outputfmtSection

Единственное число — во сколько раз уменьшится вероятность аварии. Допустимо расхождение не более, чем на 0.01

\markSection

Задача имеет два варианта, один с небольшим размером входных данных и лимитами времени выполнения, достаточными для решения задачи перебором. Второй вариант имеет размеры входных данных и лимиты времени выполнения, исключающие перебор. За решение первого варианта: 10 баллов, за решение второго — 14 баллов.

\sampleTitle{1}

\begin{myverbbox}[\small]{\vinput}
    7 99
    28 0.2521 0.1965
    11 0.3811 0.2454
    9 0.4621 0.03367
    12 0.4761 0.3627
    5 0.3171 0.0548
    30 0.4007 0.3879
    34 0.2658 0.2053
\end{myverbbox}
\begin{myverbbox}[\small]{\voutput}
    4.401
\end{myverbbox}
\inputoutputTable

\solutionSection

Это задача о рюкзаке с мультипликативной целевой функцией. Каждый вариант преобразуется к 1/<уменьшение вероятности отказа>, находится вектор, максимизирующий данную величину, затем преобразуется в ответ. Кроме владения техникой решения задач типа «рюкзак» и умения их распознавать, нужно также владение теорией вероятности на уровне понимания независимых и дополнительных событий.

\includeSolutionIfExistsByPath{final/subject_tour/inf1503_ies/task_02}
