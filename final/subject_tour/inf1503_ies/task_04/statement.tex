\assignementTitle{Задача 4. Тьюринг 2D}{6}{}

Дан прототип двумерной машины Тьюринга, описывающей программу движения курсора по полю. Программа представлена матрицей из $N$ строк и $M$ столбцов. В каждой клетке матрицы, соответствующей определённой точке поля, содержится команда перемещения курсора:
l — влево на 1 клетку
r — вправо на 1 клетку
u — вверх на 1 клетку
d — вниз на 1 клетку

Поле зациклено, при переходе через границу поля курсор выходит с другой его стороны.

В начале выполнения курсор установлен в точке $(0,0)$.

Очевидно, что рано или поздно выполнение программы зациклится. Вам нужно определить, в какой точке матрицы начинается цикл.

\inputfmtSection

Первая строка — количество строк $M$  и столбцов $N\space (5\leq M,N\leq 20)$ через пробел. 
Далее $M$ строк длиной $N$, содержащих символы из множества \{l,r,u,d\}.

\outputfmtSection

Два числа через пробел — строка и столбец ячейки, где начинается цикл. Нумерация начинается с нуля!

\sampleTitle{1}

\begin{myverbbox}[\small]{\vinput}
    4 4
    rrrd
    ldll
    urru
    rlrl
\end{myverbbox}
\begin{myverbbox}[\small]{\voutput}
    1 3
\end{myverbbox}
\inputoutputTable

\solutionSection

Решение рассчитано на простую реализацию. Храним посещённые клетки в булевой матрице. Начинаем с (0, 0), читаем символ по этим координатам. Вправо — увеличение столбца, вниз — увеличение строки. Вышли за пределы матрицы — присвоили координаты с другого конца. Попали на посещённую ранее клетку — выводим её координаты и завершаем выполнение.

\includeSolutionIfExistsByPath{final/subject_tour/inf1503_ies/task_04}
