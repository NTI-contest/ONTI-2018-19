\assignementTitle{Задача 3. Некубические кубики}{13}{}

Привычный нам шестигранный кубик при броске выдает любое число очков на верхней грани с равной вероятностью (1/6).

В ваше же распоряжение попали необычные кубики, у которых вероятности выпадения для каждой грани отличаются от привычной 1/6.

Подсчитайте вероятность того, что в результате броска на верхних гранях в сумме будет 21 точка.

\inputfmtSection

6 строк, в каждой через пробел по 6 вещественных чисел $p_{ij}\space(1\leq i, j\leq 6)$, представляющих вероятность выпадения $j$ точек на $i$-м кубике. Гарантируется, что распределения для кубиков корректны, т.е. $\sum_{j-1}^6 a_{ij} = 1 (1\leq i\leq 6) $.

\outputfmtSection

Единственное вещественное число — вероятность выпадения 21 точки на всех кубиках. Допускается расхождение с ответом не более, чем на 0.00001\markSection

\sampleTitle{1}

\begin{myverbbox}[\small]{\vinput}
    0.11195 0.25259 0.0337 0.39412 0.16683 0.04081
    0.09433 0.1281 0.17695 0.11216 0.31335 0.17511
    0.04646 0.07009 0.13142 0.43325 0.14784 0.17094
    0.18595 0.25611 0.2401 0.11454 0.08692 0.11638
    0.18818 0.12266 0.09079 0.02577 0.51837 0.05423
    0.00422 0.24524 0.09319 0.11028 0.35783 0.18924
\end{myverbbox}
\begin{myverbbox}[\small]{\voutput}
    0.123123
\end{myverbbox}
\inputoutputTable

\solutionSection

Нужно написать программу, которая сначала находит всех комбинаций 6 кубиков, имеющие сумму 21, а затем вычислить и сложить вероятности выпадения всех этих комбинаций.

\includeSolutionIfExistsByPath{final/subject_tour/inf1503_ies/task_03}
