\assignementTitle{Задача 2}{15}{}

Использование фитнес-трекеров стало очень популярно  в настоящее время. Эти устройства помогают людям мониторить свою активность, стимулируют вести здоровый образ жизни и в некоторых случаях позволяют заподозрить отклонения в здоровье. 

Большинство моделей с пульсометрами измеряют частоту сердечных сокращений (ЧСС) только в состоянии покоя, так как движения вызывают большие погрешности измерений. Однако ряд специализированных трекеров осуществляют мониторинг ЧСС даже в движении с определенной заданной периодичностью.

При разработке новых методик и программ фитнес-тренировок учитывают, в том числе, измерение показателей ЧСС в процессе тренировки. Так как фитнес — не спорт высоких достижений, а элемент здорового образа жизни (ЗОЖ), то и нагрузки должны быть сбалансированными. Для оценки сбалансированности нагрузки выбирается фокус-группа, участникам которой надевают специализированные трекеры. Для текущего исследования были выбраны те, которые записывают ЧСС за равные промежутки времени.

На основании данных с датчиков и данных о возрасте участников фокус-группы определите насколько оказалась интенсивной тренировка. Интенсивность в каждый момент тренировки определяется формулой Карвонена:

 $$BT = (MAXB - BB) \cdot I + BB$$

$BT$ — ЧСС во время тренировки в ударах в минуту

$MAXB$ — максимальная ЧСС

$I$ — интенсивность (в %).

Можно преобразовать эту формулу, чтобы она показывала требуемую интенсивность:

$$I = \dfrac{BT-BB}{MAXB-BB}$$

$$MAXB = 220 - Y$$

$Y$ — возраст

Выведите статистику по тренировочным зонам, выделяя продолжительность зон и их смену.


\begin{tabular}{|c|c|c|c|}
Зона интенсивности&Название&Границы интенсивности, \% \\
Нетренировочная зона&RELAXING&[0, 50]\\
Восстановительная зона&VERY LIGHT&(50, 60]\\
Зона лёгкой активности&LIGHT&(60, 70]\\
Аэробная зона&MODERATE&(70, 80]\\
Анаэробная зона&HARD&(80, 90]\\
Максимальные усилия&MAXIMUM&(90, $\infty$)\\
\end{tabular}


\inputfmtSection

В первой строке через пробел подаются четыре целых числа $Y\space(14\leq Y\leq 80)$, $BB\space(40\leq BB\leq 100)$, $P\space(10\leq P\leq 120)$ и $N\space(300\leq N\cdot P\leq 2\cdot 60\cdot 60)$ — возраст участника фокус-группы, пульс в состоянии покоя, период, за который измеряет средний пульс пульсометр и количество измерений соответственно.

Далее через пробел записываются $N$ целых чисел $H_i\space(40\leq H_i\leq 230)$ — последовательные измерения среднего значения ЧСС за период, соответствующий характеристике измерителя.

\outputfmtSection

Выведите смену зон и соответствующую им продолжительность по времени, каждую в отдельной строке. 

\markSection

Баллы за задачу будут начисляться пропорционально количеству успешно пройденных тестов.

\sampleTitle{1}

\begin{myverbbox}[\small]{\vinput}
    26 79 60 120
    79 79 106 117 143 156 181 180 177 174 175 177 177 175 174 172 169 190 189 187 184 186 186 185 183 180 178 175 174 171 168 167 170 170 169 171 171 170 170 171 172 171 170 169 172 173 170 170 173 170 174 176 179 182 180 181 178 180 182 185 183 184 188 186 185 185 186 189 193 190 192 191 192 188 186 187 186 183 183 182 182 182 179 179 180 178 176 173 173 173 172 171 172 173 172 172 173 174 172 166 163 159 149 145 144 137 127 120 119 118 116 107 105 99 99 98 94 94 93 90
\end{myverbbox}
\begin{myverbbox}[\small]{\voutput}
    RELAXING 240
    VERY LIGHT 60
    LIGHT 60
    HARD 600
    MODERATE 60
    MAXIMUM 480
    HARD 240
    MODERATE 660
    HARD 60
    MODERATE 180
    HARD 120
    MODERATE 120
    HARD 60
    MODERATE 60
    HARD 540
    MAXIMUM 1200
    HARD 720
    MODERATE 60
    HARD 420
    MODERATE 120
    LIGHT 120
    VERY LIGHT 180
    RELAXING 840
\end{myverbbox}
\inputoutputTable

\solutionSection

В данной задаче необходимо аккуратно вычислять зону интенсивности для каждого измерения. В случае, если зона не меняется, увеличиваем продолжительность нахождения в данной зоне. В случае же её смены, надо выводить продолжительность.
