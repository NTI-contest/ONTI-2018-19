\assignementTitle{}{20}{}

Особое место в научных исследованиях является выявление корелляций (зависимостей) между явлениями. Например, известно, что с возрастом у людей уменьшается средняя частота сердечных сокращений (ЧСС), с определенного возраста человек становится ниже, количество больних диабетом 2 типа возрастает также с увеличением возраста.

Корелляции бывают разные. Одной из распространенных зависимостей является линейная. У вас есть значения измерений, характеризующие какие-либо величины. Проверьте гипотезу о том, что значения данных параметров зависят линейно от возраста. Причём не обязательно, чтобы все точки лежали на одной прямой, а были на минимальном расстоянии от некоторой неё. При этом эти расстояния не должны превышать некоторого заданного значения в большинстве точек. 

Интерн медицинского ВУЗа на основе публикаций в Интернете выяснил, что прямая вида $y = kx + b$ — наилучшая линейная аппроксимация, характеризующая зависимость некоторого параметра от возраста и на практике собрал статистику с одного медицинского учреждения. Определите среднее расстояние от точек, соответсвующих измерениям, до заданной линейной аппроксимации, а также долю точек, находящихся на расстоянии больше допустимого.

\inputfmtSection

В первой строке через пробел подаются 4 целых числа: $N\space (1\leq N\leq 10^6)$, $k\space (-10^6\leq k\leq 10^6)$, $b\space (-10^6\leq b\leq 10^6)$ и $S\space (0\leq S\leq 10^8)$ — количество измерений, значения коэффициентов линейного уравнения заданного вида, наибольшее допустимое отклонение.

Далее в $N$ строках через пробел подаются по два целых числа: $x_i\space (14\leq x_i\leq 100)$, $y_i\space (0\leq y_i\leq 10^8)$.

\outputfmtSection

Два вещественных числа через пробел с точностью не ниже $10^{-6}$ — ответ на задачу.

\markSection

Баллы за задачу будут начисляться пропорционально количеству успешно пройденных тестов.

\sampleTitle{1}

\begin{myverbbox}[\small]{\vinput}
    7 -1 200 10
    20 185
    25 164
    83 123
    64 161
    33 172
    20 193
    45 148
\end{myverbbox}
\begin{myverbbox}[\small]{\voutput}
    7.2730983208 0.1428571429
\end{myverbbox}
\inputoutputTable

\solutionSection

Выведем формулу нахождения расстояния от точки до прямой вида $y = kx + b$. Приведем к виду $Ax + By + C = 0: A = k, B = -1, C = b$.
Расстояние от точки $M$ до прямой равно $\frac{|AM_x + BM_y + C|}{\sqrt{A^2 + B^2}} = \frac{|kM_x-y+b|}{\sqrt(k^2 + 1)}$. Действия по поиску расстояния необходимо проделать над всеми точками и на основе полученных значений найти количество точек, расстояние до которых больше указанного значения, также при подсчете надо получать сумму расстояний для нахождения среднего значения. 
