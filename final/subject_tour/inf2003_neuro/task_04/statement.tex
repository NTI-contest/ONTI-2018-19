\assignementTitle{Задача 4}{10}{}

Международная статистическая классификация болезней и проблем, связанных со здоровьем (англ. International Statistical Classification of Diseases and Related Health Problems) — документ, используемый как ведущая статистическая и классификационная основа в здравоохранении. Раз в десять лет пересматривается под руководством Всемирной организации здравоохранения (ВОЗ). Международная классификация болезней (МКБ) является нормативным документом, обеспечивающим единство методических подходов и международную сопоставимость материалов. В нем содержится перечень заболеваний, классифицированных по разделам. 

Каждое заболевание или проблема со здоровьем имеет ряд признаков, которые позволяют правильно поставить диагноз. Эти признаки при каждом из заболеваний появляются с определенными вероятностями. У вас есть справочник заболеваний, в котором прописаны симптомы с вероятностью их встречаемости при указанных заболеваниях. Определите список наиболее вероятных диагнозов по жалобам пациента и результатам клинических анализов.

Методика оценки диагнозов из справочника:

\begin{enumerate}
    \item Введем коэффициент $K = 0$. Далее сравним список жалоб с информацией в справочнике по каждому заболеванию.
    \item За каждый симптом заболевания, обнаруженный у пациента, добавляем к $K$ число $p$ — вероятность возникновения конкретного симптома у заболевающего.
    \item За каждый симптом заболевания, отсутствующий у пациента, вычитаем из $K$ число $p$.
\end{enumerate}

\inputfmtSection

В первой строке подается целое число $N\space (10\leq N \leq 10^5)$ и $M\space (1\leq N \leq 100) $ — количество заболеваний в справочнике и количество симптомов у пациента. 

Далее для каждого из $N$ заболеваний задаются следующие параметры:
\begin{itemize}
\item В первой строке через пробел подаются $C_i\space(1\leq len(C_i) \leq 10)$ и $K\space(1\leq K \leq 10)$ — код заболевания по МКБ-10 (содержит латинские буквы, цифры и точки) и количество симптомов 
\item Для каждого из симптомов в отдельной строке через пробел задаются $S_{ij}\space(1\leq len(S_{ij}) \leq 10)$ и $p_{ij}\space(0\lt p_{ij} \leq 1)$ — названия симптомов из латинских букв или цифр без пробелов и соответствующая вероятность встречаемости
\item Затем в отдельной строке через пробел подаются $H_{i}\space(1\leq len(H_{i}) \leq 10)$ — названия симптомов пациента, состоящие из латинских букв без пробелов.
\end{itemize}

\outputfmtSection

Выведите наиболее вероятный диагноз. Если в списке есть два и более равновероятных, выведите их в отдельных строках алфавитном порядке.

\markSection

Баллы за задачу будут начисляться пропорционально количеству успешно пройденных тестов.

\sampleTitle{1}

\begin{myverbbox}[\small]{\vinput}
    3 3
    D11.1 3
    S1 0.5
    S2 0.5
    S3 1.0
    D33.2 4
    S1 1.0
    S2 0.5
    S3 0.5
    S4 1.0
    D23.1 2
    S1 1.0
    S3 1.0
    S1 S3 S4
\end{myverbbox}
\begin{myverbbox}[\small]{\voutput}
    D23.1
    D33.2
\end{myverbbox}
\inputoutputTable

\solutionSection

В данной задаче необходимо найти симптомы заболеваний в наборе симптомов пациента и получить некоторое числовое значение, характеризующее каждый диагноз. Для этого необходимо в некотором виде сохранить диагнозы и соответствующие симптомы, зачитать жалобы/симптомы пациента и только затем перебрать симптомы диагнозов. Затем все диагнозы сортируются в первую очередь по значению, во вторую в лексикографическом порядке.

\includeSolutionIfExistsByPath{final/subject_tour/inf1303_bas/task_04}