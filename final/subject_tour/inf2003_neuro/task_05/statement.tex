\assignementTitle{}{25}{}

У некоторых вирусных заболеваний есть несколько стадий (сведения представлены для вируса краснухи):

\begin{enumerate}
\item Сначала вирус попадает в организм человека и начинает активно размножаться.
\item В некоторый момент человек становится заразным (начиная с 7 дня заражения).
\item Появляются первые признаки заболевания (от 11 до 25 дней).
\item Признаки заболевания исчезают (через 3 дня после возникновения первых признаков).
\item Человек перестает быть заразным, у него формируется иммунитет к вирусу (через 7 дней после возникновения признаков).
\end{enumerate}
Мы же рассмотрим вирус, который никак себя внешне не проявляет. Он чрезвычайно заразен (если люди находятся в помещении некоторое время, заражаются все). Этим вирусом можно заразиться и переболеть всего один раз, днем заражения считается первый день контакта с другими зараженными в заразном периоде. После болезни у человека возникает стойкий иммунитет. Вакцины от данной болезни не существует.

В нулевой день человек с идентификатором 0 привез данный вирус из некоторой поездки. Он был единственным заболевшим среди всех.
\inputfmtSection

В первой строке через пробел подаются два целых числа $a\space (1\leq a \leq 20)$ и $b\space(a \leq b \leq 20)$ — первый и последний день заразного периода начиная с момента заражения человека (заражение происходит в день 0)

Во второй строке подаётся целое число $N\space(1\leq N\leq 10^4)$ — число встреч.

Далее для каждой из $N$ встреч подаются 2 строки.

\begin{itemize}
\item Первая из них содержит два целых числа $D\space(1\leq D\leq 10^4)$ и $M\space(2\leq M\leq 100)$ — день встречи (начиная с 0) и количество участников встречи соответственно.
\item Вторая содержит $M$ чисел $Q\space(0\leq Q\leq 10^5)$ — идентификаторы участников.
\end{itemize}

\outputfmtSection

Выведите через пробел номера точно заразившихся на перечисленных встречах в порядке возрастания, если других контактов каждый из них не имел.

\sampleTitle{1}

\begin{myverbbox}[\small]{\vinput}
    2 3
    5
    1 3
    0 4 10
    2 3
    0 2 3
    4 4
    0 8 7 9
    5 3
    0 2 4
    7 6
    8 5 0 4 6 1
\end{myverbbox}

\begin{myverbbox}[\small]{\voutput}
    0 1 2 3 4 5 6 8
\end{myverbbox}
\inputoutputTable