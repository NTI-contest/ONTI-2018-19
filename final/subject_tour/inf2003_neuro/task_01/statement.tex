\assignementTitle{}{10}{}

В некотором городе согласно статистике местной службы мониторинга здоровья населения:
\begin{itemize}
\item половина населения близоруки,
\item треть имеют заболевания опорно-двигательного аппарата,
\item 20\% населения — IT специалисты,
\item 15\% имеют проблемы с опорно-двигательным аппаратом и и при этом идеальное зрение, 
\item четверть населения полностью здоровы.
\item 30\% IT-специалистов ведут здоровый образ жизни, делают зарядку, следят за своей утомляемостью, соблюдают правила работы за компьютером,
\item только 10\% IT-специалистов, ведущих здоровый образ жизни, нездоровы.
\end{itemize}

Определите наибольшее возможное в городе количество IT-специалистов со сколиозом (заболевание опорно-двигательного аппарата) и близорукостью одновременно, если население города составляет 60000.

\solutionSection

Пусть $A$ — множество близоруких людей в городе, $B$ — множество людей с заболеваниями опорно-двигательного аппарата в городе.
$|A| = \frac{1}{2} \cdot 60000 = 30000$ — число элементов множества $A$.  
$|B| = \frac{1}{3} \cdot 60000 = 20000$ — число элементов множества $B$.
$|!A \& B| = \frac{15}{100} \cdot 60000 = 9000$ 
$|A \& B| = |B| \\ |!A \& B| = 11000$ — количество людей, имеющих оба недуга ($A$ и $B$).

Максисмум больных IT-специалистов (без учета общего числа людей, имеющих оба недуга) равен: 
$$\big(\frac{1}{5} \cdot \big(1 - \frac{3}{10}\big) + \frac{1}{5} \cdot \frac{3}{10} \cdot \frac{1}{10}\big) \cdot 60000 = \frac{73}{500} \cdot 60000 = 8760$$


$8760 < 11000$, следовательно, ответ на вопрос задачи — 8760.

\answerMath{8760}