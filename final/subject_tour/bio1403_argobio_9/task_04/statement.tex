\assignementTitle{}{}{4}

Растения на садовом участке в 8 соток выделяют кислород, который затем 
используется микроорганизмами в процессе нитрификации (окисления аммиака до нитрата). Известно, что суммарно в неделю
с 1 м$^2$ растения поглощают 6 л воды для синтеза кислорода. Нитрифицирующие бактерии забирают 5\% выделенного кислорода. 
Запишите уравнения протекающих реакций и вычислите количество
нитрата, которое образуется на садовом участке за месяц.
