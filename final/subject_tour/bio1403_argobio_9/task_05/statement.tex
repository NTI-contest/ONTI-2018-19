\assignementTitle{}{13}{5}

В лесном массиве площадью 16 га обитают зайцы, лисы, олени, белки и совы. Вместе с мелкими растениями леса они 
составляют единую трофическую цепочку, на каждом уровне которой энергия между разными видами распределяется 
равнозначно, а переход между уровнями одинаковый. Известно, что консументы самого высокого порядка в этой цепи 
получают 62259 ккал энергии, а с 1 га леса продуценты выделяют 228000 ккал. Такая пищевая цепь поддерживает жизнь 
всех входящих в неё видов, обеспечивая их необходимым для выживания количеством энергии. Нарисуйте данную пищевую 
цепь и укажите количество энергии у каждого её участника.
 
При строительстве нового жилого квартала застройщик вырубил 1,8 га массива, что в первую очередь сказалось на 
популяции продуцентов, на 35\% сократило численность зайцев, нарушив тем самым баланс и равномерность 
распределения энергии по пищевой цепи. Каким видам в первую очередь будет не хватать энергии после вырубки, 
сколько? В стрессовой ситуации один из видов данной пищевой цепи может начать употреблять вид, на который в 
нормальных условиях не охотится. Хватит ли в таком случае ему энергии для поддержания существования своей 
популяции? О каких видах идёт речь?
