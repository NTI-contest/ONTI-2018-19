\solutionSection
Шарик пройдет решетку только в случае если его координаты его центра находятся в диапазоне $(\pm 2.5, \pm 2.5)$ см от центра ячейки. Т.е. \textbf{окно безопасности} равно $5\cdot 5 = 25$ кв. см. Так как у нас есть $N$ ячеек, есть $N$ \textbf{окон безопасности}. Вероятность шарика попасть в любое окно пропорционально отношению площадей \textbf{окна безопасности} и площади ячейки с прутьями. Так как ширина прутьев 10 см, можно считать, что к каждой ячейке добавляются по 5 см с каждой стороны. Поэтому площадь ячейки с прутьями $30\cdot 30 = 900$ см$^2$. 

\[p = N \cdot  25 / N \cdot  900 = 25 / 900 = 0.027778\]
\[N = p \cdot  100000  = 2777.8 = 2778\]

\answerMath{2778.}

