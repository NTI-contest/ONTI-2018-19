\solutionSection
Минимальное возможное время без учета ожидания следующего маршрута равно:
\[ T = \delta_{AQ} + \delta_{QS} + \delta_{SZ}\]

Такое время поездки достигается если маршруты №1 и №2 встретяться на остановке $Q$ и на этом же круге №2 встретится с №3 на $S$. Обозначим время посадки на $A$ как $t$, к моменту времени $t+23$ маршруты №1 и №2 должны совершить полный круг каждый. Поэтому остаток от деление $t+23$ на 59 и 37 равно нулю:

\[ (t + 23) \% 59 = 0\]
\[ (t + 23) \% 37 = 0\]

Так как 59 и 37 простые числа искомое $t$ должно удовлетворять следующему выражению:

\[ t = 59 \cdot 37 \cdot n - 23\]

где $n$ - целое положительное число.

Так же известно что в момент времени $t+23+19+41$ пассажир должен быть на остановке $Z$. Так как автобус №3 начинает курсировать с $Z$ то это время так же должно соответствовать целому числу кругов:
\[ (t + 23 + 19 + 41) \% 89 = 0\]
\[ t = 89 \cdot m - 23 - 19 - 41\]
\[ 59 \cdot 37 \cdot n - 23 = 89 \cdot m - 23 - 19 - 41\]
\[ 59 \cdot 37 \cdot n = 89 \cdot m - 60\]
\[ m = (59 \cdot 37 \cdot n + 60)/89 \]

Подбираем $n=65$ что соответствует моменту времени $t = 141872$.

\answerMath{141872.}

