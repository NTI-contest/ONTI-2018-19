\solutionSection
Движение мячика происходит только в направлении Z, поэтому точкой пересечения будет $(x_0, y_0, 0)$.  Условием касания мячика и камеры будет:

\[R + D/2 == \sqrt{x_0^2 + y_0^2} \]

Таким образом задача сводится к двум неизвестным. Искомая величина смещения выражается как:
\[\delta = (R + D/2) - \sqrt{x_0^2 + y_0^2}\]

Угловой размер (угловой диаметр) равен $\theta = 2 \cdot arctg(R/L)$, $L$ это расстояние  от камеры до мячика, $R$ - радиус мячика. Выразим $L$ можно через угловой размер:
\[ L = \frac{R}{\tg{\frac{\theta}{2}}} \]

С другой стороны $L$:
\[L^2 = x_0^2 + y_0^2 + z^2 \]
\[L^2 = x_0^2 + y_0^2 + (z_0 + v_z \cdot t)^2 \]

В момент времен $t_1$ и $t_2$:
\[L_1^2 = x_0^2 + y_0^2 + (z_0 + v_z \cdot t_1)^2 \]
\[L_2^2 = x_0^2 + y_0^2 + (z_0 + v_z \cdot t_2)^2 \]

Вычтем одно из другого и выразим $z_0$:
\[ z_0 = \frac{L_1^2 - L_2^2 - v_z^2  \cdot  (t_1^2 - t_2^2)}{2 \cdot v_z \cdot (t_1 - t_2)} \]

Из любого уравнения выше можем выразить сумму квадратов $x_0$ и $y_0$:
\[x_0^2 + y_0^2 = L_1^2 - (z_0 + v_z \cdot t_1)^2 \]

Значения $L$ можно вычислить из выражений выше.

Подставив выражение выше найдем что:
\[ z_0 =  113.1\]
\[ \sqrt{x_0^2 + y_0^2} =  22.4294109 ...\]

Найдем смещение необходимое для касания.
\[ \delta = (9.5 + 12.75) - 22.4294109 ... = -0.17941 ... = -0.18\]

\answerMath{-0.18.}

\markSection

\begin{enumerate}
    \item Определено условия касания (5 баллов)
    \item Определена связь координат мячика и углового размера (10 баллов) 
    \item Величина смещения определена правильно (20 баллов)
\end{enumerate}