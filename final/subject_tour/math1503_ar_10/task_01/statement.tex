\assignementTitle{}{35}{1}

Мячик радиуса 9.5 см вылетает из некой точки пространства $(x_0, y_0, z_0)$, $ z_0 > 0 $, со скоростью $(0, 0, -4.85)$ cм/c. В точке $(0, 0, 0)$ находится камера и снимает движения мячика, луч зрения камеры совпадает с положительным направлением оси Z. В момент времени $t_1 = 5$ c угловой размер (угловой диаметр) мячика относительно камеры равен $T_1=$ 11\textdegree50'14.309'', в момент $t_2=11.25$ c –$T_2=$17\textdegree $14'4.454''$. Угловой размер можно найти из треугольника составленного из расстояния до мячика и его радиуса. Камера находится в корпусе диаметром $D = 25.5$ см. 

На какое расстояние перпендикулярно оси $Z$ нужно подвинуть мячик, чтобы он только коснулся корпуса камеры. 

Смещение к оси $Z$ считать отрицательным, от оси $Z$ положительным. 
Ответ округлите до сотых долей сантиметра.

\putImgWOCaption{8.5cm}{1}

Иллюстрация углового размера:

\putImgWOCaption{7cm}{2}