\assignementTitle{}{}{4}

Манипулятор состоит из двух плечей эффективной длины $L$. Основание первого плеча установлено на валу сервопровода S1, закрепленного в начале координат. Сервопривод S2 закреплен на конце первого плеча, а к его валу прикреплено основание второго плеча. К концу второго плеча прикреплен карандаш (см. рисунок). Сервоприводы могут поварачиваться в диапазоне +-90 градусов. При положение сервоприводов (0\textdegree, 0\textdegree) оба плеча расположены вдоль положительного направления оcи X. Если смотреть с положительного направления Z - увеличение угла поворота сервопривода, соответствует движению плеча, в основании которого установлен сервопривод, по часовой стрелке. Уменьшение - к движению против часовой стрелки.

\putImgWOCaption{10cm}{1}

Найти выражение параметрического задания углов поворота сервоприводов $s_1 = f(t), s_2 = g(t)$ для рисовки части оси X, которую позволяет отрисовать конструктивные ограничения манипулятора. Так же дайте диапазон изменения параметра $t$ с точностью до градуса. Линия должна быть орисована в положительном направлении оси X.