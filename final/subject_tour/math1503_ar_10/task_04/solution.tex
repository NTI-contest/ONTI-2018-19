\solutionSection
Из-за конструктивных особенностей данный манипулятор может отрисовать только часть оси X от $(\sqrt{2}l, 0)$ до $(2l, 0)$. Плечи манипулятора с осью X формируют равнобедренный треугольник. Если угол между первым манипулятором и осью X равен $s_1=\alpha$, то угол между вторым манипулятором и осью X так же должен быть $\alpha$. Угол лежаший напротив основания равен $180 - 2\alpha$. Следовательно модуль угла поворота второго сервопривода равен $s_2 = 180 - (180 - 2\alpha) = 2\alpha$, направление противоположно повороту S1, поэтому $s_2 = -2\alpha$. Таким образом углы задаются как:

\[ s1 = t; s2 = -2t\]

Когда карандаш находится в положении $(\sqrt{2}l, 0)$ это соответствует равнобедренному прямоугольному треугольнику $\alpha$ = 45\textdegree. Так как относительно оси X плечо должно отклониться против часовой стрелки то угол поворота $S1$ равен -45\textdegree. При положении $(2l, 0)$, как следует из условия, оба сервопривод повернуты на 0\textdegree. 

\answerMath{$s1 = \pm t; s2 = \mp 2t$; $t=-45$\textdegree \dots 0\textdegree.}