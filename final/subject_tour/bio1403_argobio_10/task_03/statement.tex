\assignementTitle{}{10}{3}

Для определения активности азотфиксации, которая происходит в почве благодаря азотфиксирующим бактериям, учёные применили ацетиленовый метод, который основан на свойстве нитрогеназы восстанавливать не только азот, но и ацетилен. Они использовали специальный флакон, в который добавляли сначала навеску почвы (10г), затем глюкозу, а после инкубации вводили ацетилен. С помощью газового хроматографа и калибровочной кривой они определили концентрации этилена и ацетилена через 2 часа после начала инкубации почвы с ацетиленом.

Запишите реакции с ионами, катализируемые нитрогеназой для азота и ацетилена, и вычислите активность азотфиксации (в мг фиксированного азота на 1 кг почвы за 1 час), если известно, что объём флакона 5 мл, концентрации этилена и ацетилена по калибровочной кривой равны 7,5M и 1,5M соответственно, а исходная концентрация азота в флаконе равна 3,3M. Выход реакции азотфиксации в 3 раза меньше, чем реакции восстановления ацетилена.
