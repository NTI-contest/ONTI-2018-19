\assignementTitle{}{10}{7}

Одним из бурно развивающихся методов неинвазивной диагностики состояния сердечно-сосудистой системы является фотоплетизмография. Принцип метода состоит в измерении динамики поглощения инфракрасного света пальцем руки или ноги пациента с помощью пульсового оксиметра. Во время прохождения пульсовой волны по сосудам пальца объём крови в пальце возрастает и поглощение ИК излучения увеличивается. Типичная фотоплетизмограмма выглядит следующим образом (см. рисунок 1). Пульсовая волна, соответствующая одному сердечному циклу, изображена на рисунке 2.

\putImgWOCaption{12cm}{1}

Рисунок 1. Фотоплетизмограмма: сверху – первичные данные, внизу – результат после удаления шумов.

\putImgWOCaption{12cm}{2}

Рисунок 2. Пульсовая волна, зарегистрированная пульсовым оксиметром, закреплённым на указательном пальце руки.

Выберите все верные утверждения:

А. у пациента, чья фотоплетизмограмма изображена на рисунке 1, наблюдается умеренная синусовая аритмия с частотой сердечных сокращений 60-67 уд/мин;

Б. пик №1 на рисунке 2 соответствует моменту начала сокращения левого предсердия;

В. пик №4 на рисунке 2 соответствует волне, возникающей в артериальном русле после закрытия аортального клапана;

Г. фотоплетизмограф в сочетании со сфигмоманометром может использоваться для измерения систолического давления пациентов с пульсом, не прослушивающимся при помощи стетоскопа;

Д. при внутривенном введении фенилэфрина (вазоконстриктор) амплитуда волн фотоплетизмограммы резко возрастёт.

В качестве ответа введите последовательность из русских букв без разделителей.