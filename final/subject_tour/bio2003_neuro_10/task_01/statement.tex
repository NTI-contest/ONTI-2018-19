\assignementTitle{}{6}{1}

Равновесный потенциал для иона — это значение разности потенциалов на мембране, при котором входящий и выходящий токи данного иона уравновешены. Потенциал покоя — значение разности потенциалов на мембране клетки в отсутствии стимула. Значение потенциала покоя на исследуемом нейроне равно -70 мВ (внутри мембрана заряжена более отрицательно, чем снаружи).

Рассчитайте значение равновесного потенциала для ионов натрия, калия и хлора по формуле:

$$E = -\dfrac{R\cdot T}{F\cdot Z}\ln \dfrac{[C_1]}{[C_2]},$$

где $Z$ — заряд иона, $T$ — температура по шкале Кельвина ($298$ K), $F$ — постоянная Фарадея ($96485$ Кл/моль), $R$ — газовая постоянная ($8,314$ Дж/(моль$\cdot$К)), $C_1$~— концентрация иона внутри клетки, $C_2$ — концентрация иона снаружи клетки.

Используйте следующие концентрации ионов:

\begin{tabular}{|l|c|p{5cm}|}
    \bf ион & \bf концентрация внутри, ммоль (i) & \bf концентрация снаружи, ммоль (o) \\
    калий & 140 & 2.5 \\
    натрий & 10 & 145 \\
    хлор & 10 & 110 \\
\end{tabular}

Ответы дайте в мВ, округлите до ближайшего целого числа. Укажите через пробел три целых числа: величины соответствующие равновесным потенциалам ионов натрия, калия и хлора (порядок важен). Знак $-$ пишите вплотную к значению.

Если вы не смогли вычислить одну из величин, проставьте в соответствующем месте 0 (в ответе обязательно должно быть три целых числа). 


