\assignementTitle{}{8}{5}

Аркуатное ядро гипоталамуса играет ключевую роль в регуляции аппетита и пищевого поведения. Здесь находятся рецепторы гормона лептина, вырабатываемого жировой тканью. Этот гормон подавляет чувство голода и тормозит пищевое поведение. Лептин — пептидный гормон, кодируемый геном Ob. Рецепторы лептина кодируются гормоном Db. Исследователь обнаружил мутацию, приводящую к потере функции гена Ob (аллель ob), а также мутацию, приводящую к потере функции рецептора лептина (аллель db). Было поставлено скрещивание мышей, гетерозиготных по обоим генам.

Какой процент мышей в потомстве (F1) будет страдать ожирением? Считайте, что мутации рецессивны.
Какой процент мышей среди мышей, страдающих ожирением, удастся вылечить инъекциями лептина?
Введите через пробел два целых числа: ответы на вопросы 1 и 2 (порядок важен), в процентах, округленные до целого (знак \% не нужно писать).