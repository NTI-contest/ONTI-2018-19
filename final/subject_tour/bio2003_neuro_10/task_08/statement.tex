\assignementTitle{}{10}{8}

Потенциал действия (ПД) — возбуждение, передающееся по мембране возбудимых клеток. В покое мембранный потенциал равен некоторой постоянной отрицательной величине, называемой потенциалом покоя. В клетке высокая концентрация калия, вне клетки — кальция и натрия. В мембране есть каналы для этих ионов, причем калиевые каналы всегда открыты. Под воздействием внешних факторов (сигнала от другой клетки или повышения потенциала на соседнем участке той же клетки) потенциал может повышаться и даже превышать ноль — мембрана деполяризуется. При достижении критического уровня деполяризации (КУД) открываются потенциал-зависимые натриевые каналы (ПЗНК), натрий входит в клетку, и мембрана деполяризуется. Затем ПЗНК закрываются, и насос восстанавливает распределение ионов. При этом потенциал покоя восстанавливается — мембрана реполяризуется. В мышечных клетках, кроме того, между пиком деполяризации и реполяризацией происходит выход ионов кальция из эндоплазматической сети (саркоплазматического ретикулума) в цитоплазму клетки.

Мотонейроны проводят возбуждение к мышцам. Если на скелетную мышцу подать два сигнала подряд, то сокращения сольются в одно, более сильное.

Водитель ритма — группа клеток в сердце, периодически создающих ПД и обеспечивающих автономную работу сердца. Пучки Гиса и волокна Пуркинье — часть миокарда, которая передает ему возбуждение от водителя ритма, и может сама вызывать редкие слабые сокращения.

Установите соответствие между графиками и потенциалами действия.

\begin{enumerate}
\item \putImgWOCaption{12cm}{1}
\item \putImgWOCaption{12cm}{2}
\item \putImgWOCaption{12cm}{3}
\item \putImgWOCaption{12cm}{4}
\item \putImgWOCaption{12cm}{5}
\end{enumerate}

А. ПД мотонейрона
Б. ПД скелетной мышцы
В. ПД типичного кардиомиоцита
Г. ПД пучка Гиса и волокон Пуркинье
Д. ПД водителя ритма

В качестве ответа введите последовательно 5 русских букв без разделителей.