\assignementTitle{}{5}{3}

Сакситоксин — один из сильнейших нейротоксинов, вызывающих паралитическое отравление моллюсками. Моллюски добывают эти токсины, поглощая планктонных динофитовых водорослей из родов Alexandrium, Gymnodinium, синезелёных водорослей Aphanizomenon и др. Мишенью сакситоксина являются потенциал-зависимые $Na^+$-каналы. Ниже приведена схема взаимодействия сакситоксина с ионным каналом.

\putImgWOCaption{12cm}{1}

Выберите верные утверждения.

А. отравление сакситоксином приводит к вялому параличу мышц (мышцы расслабляются)

Б. цитоплазма нейрона обозначена на рисунке буквой А, внешняя среда - буквой Б

В. моллюски, вызывающие отравление, относятся к классу Головоногие

Г. синтез сакситоксинов характерен только для эукариот

Д. связывание сакситоксина с ионным каналом обратимо

Перечислите русские буквы, соответствующие верным уверждениям, в алфавитном порядке без разделителей.



