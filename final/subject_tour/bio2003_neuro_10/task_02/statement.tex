\assignementTitle{}{6}{2}

При открытии ионных каналов возникающий ионный ток стремится приблизить значение мембранного потенциала к равновесному потенциалу для данного иона.

1. Вы подействовали на нейрон веществом, приводящим к открытию потенциал-зависимых калиевых каналов. В какую сторону будет направлен ток калия сразу после действия вещества?

А. Внутрь клетки
Б. Наружу клетки
В. Изменения тока не будет

2. Вы подействовали на нейрон веществом, приводящим к открытию потенциал-зависимых натриевых каналов. В какую сторону будет направлен ток натрия сразу после действия вещества?

А. Внутрь клетки
Б. Наружу клетки
В. Изменения тока не будет

3. Вы подействовали на нейрон веществом, приводящим к открытию хлорных каналов. В какую сторону будет направлен ток хлора сразу после действия вещества?

А. Внутрь клетки
Б. Наружу клетки
В. Изменения тока не будет

Ответьте на вопросы и в качестве ответа введите последовательность из 3 русских букв без разделителей в порядке следования вопросов. Регистр букв не важен.



