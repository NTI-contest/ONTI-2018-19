\assignementTitle{}{10}{13}

Экспериментатор ввел в сердце лягушки (см.предыдущую задачу) вещество атропин, после чего произвел кратковременную электрическую стимуляцию вагосимпатического нервного ствола — нерва, который содержит как симпатические, так и парасимпатические волокна. Механограмма сердца после введения атропина и электрической стимуляции показана на рисунке ниже:

\putImgWOCaption{12cm}{1}

Время на одну маленькую клетку — 0.5 секунд.

Примите во внимание, что в отсутствии атропина электрическая стимуляция вагосимпатического ствола не приводила к значительным изменениям частоты сердечных сокращений.  

Выберите верные утверждения и введите в алфавитном порядке заглавные русские буквы, им соответствующие.

А. После введения атропина и электрической стимуляции частота сердечных сокращений возросла до примерно 60 ударов в минуту.
Б. Активность сердца не регулируется вагосимпатическим нервным стволом.
В. Результаты данного опыта объясняются тем, что атропин является блокатором действия парасимпатической нервной системы на сердце.
Г. Результаты данного опыта объясняются тем, что атропин является блокатором действия симпатической нервной системы на сердце.
Д. При каждом сокращении первый пик на механограмме отражает сокращение желудочков, а второй пик — сокращение предсердий.

В качестве ответа введите последовательность из русских букв без разделителей.