\assignementTitle{}{20}{5}

Для синтеза аммиака смесь азота и водорода с плотностью по гелию 2.125 и общим количеством вещества 40 моль 
поместили в замкнутый реактор с ванадиевым катализатором и рабочим объемом 10 л при температуре 200$^{\circ}$С и 
повышенном давлении. Через 20 минут давление в реакторе уменьшилось на 10\% (при той же температуре). 

\begin{enumerate}
    \item Вычислите степень превращения азота в аммиак и содержание аммиака в конечной реакционной смеси в 
    объемных процентах.
    \item Рассчитайте среднюю скорость раcходования азота за указанный промежуток времени.
    \item Константа равновесия реакции $N_2 + 3H_2 = 2NH_3$ при данной температуре составляет 1,5 моль$^3$/(л$^3 \cdot$с). 
    Находилась ли система в равновесии в указанный момент? Ответ обоснуйте. Укажите направление изменения 
    концентрации каждого участника процесса в течении последующих 5 минут. 
    \item Каким рН будет обладать раствор, полученный при пропускании получившейся (через 20 минут 
    после начала реакции) газовой смеси через 1 л чистой воды? Константа основности аммиака составляет $1.8*10^{-5}$    
\end{enumerate}