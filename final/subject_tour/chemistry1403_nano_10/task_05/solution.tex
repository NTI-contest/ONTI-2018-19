\solutionSection

\begin{enumerate}
\item Средняя молярная масса начальной смеси составляет $$M = 2.125\cdot4 = 8.5 \: \text{г/моль},$$ 
откуда можно рассчитать ее состав. Пусть было $x$ моль $N_2$, тогда количество $H_2$ составляет $40-x$ моль, и выражение для средней молярной массы: 
$$8.5 = (x\cdot28 + (40-x)\cdot2)/40,$$ 
откуда $х = 10, n(N_2) = 10$ моль, $n(H_2) = 30$ моль. Если прореагировало y моль азота, то общее количество газов в равновесной смеси составляет 
$$(10 - у) + (30 - 3у) + 2у = 40-2у, \: \text{и}\: \frac{(40-2y)}{40} = 0.9,$$ то есть $y = 2$ моль. $$\varphi(NH_3) = \frac{2y}{(40-2y)} = \frac{4}{36} = \frac{1}{9} = 11.11\%.$$ 
Степень превращения $$\alpha = \frac{y}{10} = 20\%.$$
\item За 20 минут количество азота уменьшилось с 10 моль до $10 - y = 8$ моль, то есть изменение концентрации произошло с 1 М до 0.8 М (объем реактора — 10 л).
Средняя скорость $$r = \frac{\Delta c}{\Delta t} = \frac{(1-0.8)}{20} = 0.1 \: \text{моль/л мин}^{-1}.$$
\item Рассчитаем произведение реакции (величину, аналогичную константе равновесия) на момент времени через 20 минут после начала реакции. Концентрации веществ составляли 
$$С(NH_3) = \frac{2y}{10} = 0.4 \: \text{M}, \: C(H_2) = \frac{(30 - 3y)}{10} = 2.4 \: \text{М}, \: C(H_2) = 0.8 \: \text{M}$$

$$Q = \frac{C_2(NH_3)}{(C(N_2)C^3(H_2)} = 1.3\frac{\text{моль}^3}{(\text{л}^3\cdot\text{с})}.$$
Это величина меньше константы равновесия, значит концентрации азота и водорода продолжат уменьшаться, а аммиака — увеличиваться. Равновесие на указанный момент еще не было достигнуто. 
\item Пропускание газовой смеси через 1 л воды будет равносильно пропусканию 4 моль $NH_3$ (азот и водород не взаимодействуют с водой), то есть задача сводится к расчету $pH$ 4 M раствора аммиака. 
$$[OH]^-=(K\cdot C)^\frac{1}{2} = (4\cdot1.8\cdot10^{-5})^\frac{1}{2} = 8.48\cdot10^{-3}$$
$$pOH = 2.07,\;pH = 11.93$$
\end{enumerate}

\additionalCriteria

Определение количеств газов — \textit{3 балла}, расчет объемных долей — \textit{2 балла}, степени превращения — \textit{2 балла}.

Определение скорости — \textit{2 балла}

Верное указание направления изменения концентрации для каждого реагента — \textit{3 балла}, разумное объяснение — \textit{3 балла}.

Расчет $рН$ — \textit{5 баллов}.

Итого \textit{20 баллов}.