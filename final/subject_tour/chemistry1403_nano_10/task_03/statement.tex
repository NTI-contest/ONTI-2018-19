\assignementTitle{}{26}{3}

Хемилюминесценция элемента Х в одной из его аллотропных модификаций  описана не только в научной, но и в 
художественной литературе. Известно, что подобное свечение возникает в результате последовательных реакций Х с 
кислородом. В результате полного окисления образуется высший оксид элемента Х, массовая доля кислорода в котором 
составляет 56.3\%. 

\begin{enumerate}
    \item Определите элемент Х и укажите все известные вам его аллотропные модификации. 
    \item Для элемена Х известно большое количество кислот. Изобразите по одной стуктурной формуле кислоты в 
    степени окисления Х: +1, +3, +4 и +5. Для каждой из кислот напишите уравнение ее реакции с избытком раствора 
    гидроксида натрия.
    \item Рассчитайте pH 0,1М раствора средней натриевой соли трехосновной кислоты с Х в степени окисления +5. 
    (константа кислотности  кислоты по третьей ступени - $K_3=1.3 \cdot 10^{-12}$, считайте, что гидролиз идет только по первой ступени).   
\end{enumerate}

