\solutionSection

\begin{enumerate}
\item $X$ — это фосфор. Наиболее распространенные аллотропные модификации - белый, красный и черный.
\item Формулы кислот:

\putImgWOCaption{10cm}{1}

Уравнения реакций:
$$H_3PO_2 + NaOH = NaH_2PO_2 + H_2O,$$
$$H_3PO_3 + 2NaOH = Na_2HPO_3 + 2H_2O,$$
$$H_4P_2O_6 + 4NaOH = Na_4P_2O_6 + 4H_2O,$$
$$H_3PO_4 + 3NaOH = Na_3PO_4 + 3H_2O,$$
\item$$Na_3PO_4 + H2O\leftrightarrow Na_2HPO_4 + NaOH$$
$$PO_4^{3-} + H_2O\leftrightarrow HPO_4^- + OH^-$$

$$K = K_\text{гидролиза} =\frac{K_w}{K_\text{3}}=\frac{10^{-14}}{(1.3\cdot10^{-12})}=7.7\cdot10^{-3}$$
Рассчитаем степень гидролиза:
$$h^2 =\frac{K}{c}$$
$$h = 0.28$$
$$[OH^-] = h\cdot c = 0.28\cdot0.1 = 0.028$$
$$pOH = -lg[OH^-] = 1.5$$
$$pH = 14 - 1.5 = 12.5$$
\end{enumerate}

\additionalCriteria

Определение $X$~-- \textit{1~балл}, проверка расчетом по массовой доле в высшем оксиде~-- \textit{1~балл}, за каждую модификацию~-- по \textit{1~баллу}, каждая структурная формула кислоты~-- по 2 балла, каждое уравнение реакции~-- по \textit{2~балла}, верный расчет $pH$~-- \textit{5~баллов}. Итого максимум \textit{26 баллов}.