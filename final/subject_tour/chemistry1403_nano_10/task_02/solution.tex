\solutionSection
\begin{enumerate}
\item Очевидно, что вещество $Х$ является неметаллом, поскольку из металлов алюминий и цинк растворяются в щелочах, а комплексные гидроксиды, которые при этом получаются, достаточно тяжело перевести обратно в форму простого вещества-металла. Разбавленная соляная кислота, которая вызывает появление простого вещества, не проявляет окислительных или восстановительных свойств, поэтому можно сделать предположение, что вещество $X$ образовалось в результате сопропорционирования, а в веществе $Y$ элемент, из которого состоит $Х$, находится в различных степенях окисления. С учетом того, что вещество $Х$ твердое, это сера, а $Y$ — тиосульфат натрия $(\omega(S) = \frac{64}{158} = 0.405)$, неустойчивый в кислотах. Газ $Z$ — сернистый газ. Уравнения реакций:
$$Na_2S_2O_3 + 2HCl = 2NaCl + S + SO_2 + H_2O,$$
$$SO_2 + Br_2 + 2H_2O= 2HBr + H_2SO_4,$$
$$4S + 6NaOH = 2Na_2S + Na_2S_2O_3  + 3H_2O,$$  
\item Тиосульфат натрия
\item Попадая в пищу, сернистый газ связывается, образуя, в зависимости от среды, сульфиты, гидросульфиты или растворы сернистой кислоты. Все они легко связывают кислород воздуха, не давая идти процессам окисления в пище и сохраняя продукт в исходном виде. Пример уравнения: 

$2Na_2SO_3 + O_2 = 2Na_2SO_4$ (возможны и другие)
\item При увеличении концентрации соляной кислоты станет невозможным ее безопасное использование, так как она будет поражать кожу. Кроме того, большое количество выделяемого сернистого газа также может вызвать отравление.  
\end{enumerate}

\additionalCriteria

Формулы $X$, $Y$, $Z$ — по \textit{1 баллу}, подтверждение массовой доли для $Y$ — \textit{1 балл}, уравнения реакций — по \textit{2 балла} каждое (без коэффициентов — \textit{1 балл}), название вещества $Y$~— \textit{1 балл}, объяснение консервирующего действия с реакцией — \textit{3 балла}, без нее — \textit{2 балла},  каждый поражающий фактор — по \textit{2 балла}. Итого \textit{18 баллов}.