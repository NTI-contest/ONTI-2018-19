\assignementTitle{}{18}{2}

Наночастицы простого вещества $X$ нашли свое применение в медицине как основное действующее вещество против 
чесоточного клеща. Само лекарственное средство применяется наружно и представляет собой два раствора, наносимых 
на пораженный участок кожи один за другим, в результате которого происходит взаимодействие компонентов лекарства 
и выделение вещества $X$, а также газа с резким запахом $Z$, обесцвечивающим бромную воду. Одним из компонентов 
лекарственного средства является разбавленная соляная кислота, а вторым - раствор вещества $Y$, про которое 
известно, что оно может быть получено растворением избытка $X$ в концентрированном растворе гидроксида натрия,  
а массовая доля в $Y$ элемента, из которого состоит вещество $X$, составляет 40.5\%.

\begin{enumerate}
    \item Определите вещества $X$ - $Z$, напишите уравнения всех описанных в тексте задачи реакций.
    \item Назовите вещество $Y$.
    \item Газ $Z$ является широко распространенным консервантом для пищевой промышленности (Е220). На чем основано его действие? 
    Приведите пример уравнения реакции, где $Z$ проявляет себя как консервант.    
    \item Передозировка вещества $X$ при наружном применении не предоставляет опасности для человека, 
    однако при увеличении концентраций $Y$ и соляной кислоты возникают побочные эффекты. Перечислите два 
    таких эффекта и укажите вещества, ответственные за их возникновение. 
\end{enumerate}

