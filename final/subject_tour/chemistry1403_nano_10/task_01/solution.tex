\solutionSection

\begin{enumerate}
\item Пользуясь общими представлениями о поведением металлов с амфотерными свойствами соединений (алюминий, цинк), составляем аналогичное уравнения растворения металла в щелочах. Тогда газ Б~-- водород, а соединение В~-- комплексный гидроксид: $K_2[A(OH)_4]$, $K_2[A(OH)_6]$ или $K_3[A(OH)_6]$, в зависимости от степени окисления металла. Выбрать правильную формулу можно, составляя уравнения через массовую долю металла c молярной массой $x$. Доходим до варианта: $\frac{x}{(39\cdot2+x+17\cdot4)}=0.449$, откуда $x = 119$ (олово).

Уравнение реакции: $Sn+2KOH+2H_2O=K_2[Sn(OH)_4]+H_2$
\item «Заражение» происходит в результате охлаждения: при низких температурах белое олово переходит в другую аллотропную модификацию, представляющую собой серый порошок. При $-33$ и ниже этот процесс идёт особенно стремительно. 
\item Уравнение альфа - распада: $_{83}^{299}Bi=_{81}^{205}Tl+_2^4He$. Металл Г — висмут.
\end{enumerate}

\additionalCriteria

Определение А, Б, B~-- по \textit{1~баллу}, перебор и расчет~-- \textit{4~балла}, уравнение реакции~-- \textit{3~балла}. Объяснение причины чумы~-- \textit{3~балла}, уравнение распада~-- \textit{2~балла}, нахождение металла Г~-- \textit{1~балл,} максимум \textit{16 баллов}.