\assignementTitle{}{16}{1}

Как известно, чума - одно из самых ужасных потрясений, которое переживало человечество на протяжении 
своей истории. Однако кроме «биологической» чумы человечеству известна также «химическая» чума, названная 
в честь металла A, унесшая жизни членов экспедиции Скотта, направлявшейся к Южному полюсу в 1912 году. 
Причиной этого события являлось то, что баки с горючим, запаянные этим металлом, протекли, и экспедиция 
осталась без топлива в антарктических климатических условиях.

\begin{enumerate}
    \item Металл A при комнатной температуре серебристо-белого цвета, растворяется в горячих щелочах и 
    образует амфотерные оксиды и гидроксиды. При реакции A с водным раствором гидроксида калия при нагревании 
    получается газ Б, проявляющий восстановительные свойства и соединение В с массовой долей металла 44,9\%. 
    Расшифруйте соединения А - В, напишите уравнения реакций, ответ подтвердите расчетом.
    \item В чем причина «заражения чумой» металла А? Какие изменения произошли с металлом А во время экспедиции?
    \item «Заражение» можно предотвратить не только  правильными условий работы с металлом, но и использованием 
    стабилизатора, например, металла Г. Известно, что в результате альфа - распада природного изотопа Г 
    образуется стабильный изотоп $_{81}^{205}Tl$. Определите металл Г и напишите уравнение реакции его альфа - распада.
\end{enumerate}