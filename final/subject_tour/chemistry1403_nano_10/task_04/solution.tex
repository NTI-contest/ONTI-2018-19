\solutionSection

\begin{enumerate}
\item Количество вещества цинка — $\frac{2.6}{65} = 0.04$ моль, количество добавленной кислоты — $0.06$ л $\cdot\:2$ M $= 0.12$ моль, соляная кислота в избытке. По окончании выделения водорода по реакции $Zn + 2HCl = ZnCl_2 + H_2$ количество $ZnCl_2$ в конечном растворе — $0.04$ моль, количество оставшейся кислоты — $0.12-2\cdot0.04 = 0.04$ моль. С учетом разбавления на 100 мл, имеем концентрацию соляной кислоты $c = 0.04$ моль$/0.1$ л $= 0.4$ М
\item Рассмотрим уравнения реакций, последовательно протекающих в каждом из экспериментов.
\item 
\begin{enumerate}
\item [\textbf{А:}]
$$HCl + NaOH = NaCl + H_2O\;\textbf{(A1)};$$
$$ZnCl_2 + 2NaOH = Zn(OH)_2\downarrow + NaCl\;\textbf{(A2)};$$
$$Zn(OH)_2 + 2NaOH = Na_2[Zn(OH)_4]\;\textbf{(A3)}.$$
Осадок начнёт выпадать после того, как будет нейтрализована вся кислота.

В 10 мл \textbf{раствора 1} содержится 0.004 моль $HCl$, следовательно, на её нейтрализацию потребуется 0.004 моль $NaOH$.
За один шаг в \textbf{раствор 1} попадает 1 мл 0.4 M $NaOH$, что составляет $0.4\cdot0.001=0.0004$ моль.

Тогда, за десять шагов в растворе 1 нейтрализуется вся кислота, а на 11 шаге начнёт выпадать осадок.
\item [\textbf{В:}]
$$NaOH+HCl = NaCl + H_2O\;\textbf{(B1)};$$
$$4NaOH + ZnCl_2 = Na_2[Zn(OH)_4] + 2NaCl\;\textbf{(B2)};$$
$$Na_2[Zn(OH)_4] + 2HCl = Zn(OH)_2\downarrow + 2NaCl + 2H_2O\;\textbf{(B3)};$$
$$Zn(OH)_2 + HCl = ZnCl_2 + H_2O\;\textbf{(B4)}$$
Осадок начет выпадать после того, как израсходуется избыток щелочи.

В 10 мл \textbf{раствора 2} содержится 0.004 моль $NaOH$. В одной порции \textbf{раствора 1} содержится 0.0004 моль $HCl$ и 0.0004 моль $ZnCl_2$.\\
По уравнениям реакции \textbf{B1} и \textbf{B2} на 1 эквивалент $HCl$ требуется 1 эквивалент $NaOH$, а на 1 эквивалент $ZnCl_2$ — 4 эквивалента $NaOH$, следовательно за один шаг расходуется 0.02 моль $NaOH$. Избыток щелочи закончится после 2-х шагов, а на третьем шаге начнет протекать реакция \textbf{В3}.
\end{enumerate}
\item Максимально возможная масса гидроксида цинка (II) в обоих экспериментах соответствует количеству ионов $Zn^{2+}$ в аликвоте $10$ мл: 
$$m = c\cdot V\cdot MZn(OH)_2 = 0.4 \: \text{М} \cdot(0.01 \: \text{л})\cdot 99 \: \text{ г/моль} = 0.4 \: \text{г}.$$
\item Если дать твердой щелочи постоять в помещении, особенно в котором активно выделяется углекислый газ, произойдет поглощение последнего по уравнению $2NaOH + CO_2 = Na_2CO_3 + H_2O$, т.е. образуется некоторое количество карбоната натрия, который при смешивании с соляной кислотой из раствора 1 будет причиной высвобождения углекислого газа $CO_3^{2-} + H^+ = CO_2 + H_2O$.
\end{enumerate}

\additionalCriteria

Расчет молярной концентрации кислоты — \textit{2 балла},

Уравнения для каждой системы — по \textit{1 баллу} за каждое уравнение (всего максимум \textit{7 баллов} — \textit{3 балла} для эксперимента А и \textit{4 балла} для эксперимента В), расчет/обоснование — для каждого случая — по \textit{2 балла}, просто ответ — \textit{1 балл вместо 12 максимально возможных},

Расчет максимальной массы осадка $Zn(OH)_2$ — \textit{2 балла},

Объяснение — \textit{2 балла}, уравнение реакции — еще \textit{2 балла}.

\textit{Всего 20 баллов}.