\assignementTitle{}{20}{4}

Для приготовления раствора 1 навеску цинка массой 2.6 г поместили в мерную колбу на 100 мл, добавили 60 мл 2М 
соляной кислоты, дождались полного выделения газа, а затем довели до метки с помощью того же самого раствора 
кислоты. Для приготовления раствора 2 навеску гидроксида натрия массой 4 г поместили в мерную колбу на 250 мл и 
довели до метки с помощью дистиллированной воды. 

Эксперимент А: к 10 мл раствора 1 медленно добавляли раствор 2, измеряя объем добавленного раствора 
(с шагом в 1 мл).

Эксперимент В: к 10 мл раствора 2 медленно добавляли раствор 1, измеряя объем добавленного раствора 
(с шагом в 1 мл).

\begin{enumerate}
    \item Рассчитайте молярную концентрацию соляной кислоты в растворе 1 после его приготовления.
    \item Через сколько шагов в эксперименте А начнет образовываться осадок? А в эксперименте В? Ответ 
    обоснуйте и подтвердите расчетами. Напишите уравнения реакций, идущих в экспериментах А и В, указав 
    последовательность их протекания.
    \item Найдите максимальное значение массы выпавшего осадка гидроксида цинка (II) в каждом из экспериментов.
    \item Если растворы 1 и 2 готовить не накануне эксперимента, а за год до него, то при их смешивании можно 
    будет наблюдать появление пузырьков бесцветного газа. Объясните причину этого явления, напишите уравнение 
    реакции.
\end{enumerate}