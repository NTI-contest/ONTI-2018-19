\solutionSection

\begin{enumerate}
    \item Сила пропорциональна площади паруса, т.к. толщина и плотность постоянны, 
	масса паруса так же пропорциональна площади. Следовательно сила прямо пропорциональна массе паруса.

    \answerMath{в 3 раза.}
\end{enumerate}

\additionalCriteria

\begin{itemize}
    \item Приведен расчет или рассуждение, говорящее о прямой пропорциональности между силой и массой — 8 баллов.
    \item Получен правильный числовой ответ — 2 балла.
    
\end{itemize}

\begin{enumerate}
    \item[2.] Скорость перестанет увеличиваться, когда в системе отсчёта, связанной с бруском, ветер будет параллелен парусу: $u=V\cdot ctg(\alpha)=1\cdot \sqrt{3}= \sqrt{3}$ м/c.
    
    \answerMath{$\sqrt{3}$ м/c.}
\end{enumerate}

\additionalCriteria

\begin{itemize}
    \item Найдено условие, при котором скорость будет максимальна — 4 балла.
    \item Получено выражение для этой скорости — 4 балла.
    \item Получен правильный числовой ответ — 2 балла.
\end{itemize}

\begin{enumerate}
    \item[3.] Пусть искомый угол равен $x$. Тогда сила, которую создаёт ветер, дуя в парус, пропорциональна $sin(x)$, а её проекция на киль пропорциональна 
    \linebreak $cos(90^\circ - (60^\circ - x)) = sin(60^\circ - x)$. Это значит, что общая сила пропорциональна $f(x) = sin(x) \cdot sin(60^\circ - x)$. Производная функции $f(x)$ равна $sin(60^\circ - 2x)$, следовательно, максимум силы достигается при $x = 30^\circ$.
    \answerMath{под углом $30^\circ$.}
\end{enumerate}

\additionalCriteria

\begin{itemize}
    \item Приведен расчет равнодействующей силы для паруса — 4 балла.
    \item Получено значение проекции этой силы вдоль паруса — 2 балла.
    \item Показано (тригонометрически или с помощью производной) при каком значении угла достигается 
    максимум для этой проекции — 2 балла.
    \item Получен правильный числовой ответ — 2 балла.
\end{itemize}

Так же правильным (и даже более полным) нужно признать решение, в котором показывается, что в общем 
случае ставить парус нужно под углом составляющим половину угла между курсом парусника и ветром.

\begin{enumerate}
    \item[4.] Поплывёт вперёд. Рассмотрим начальный момент движения, когда парусник неподвижен и вентилятор гонит газ к парусу. Пусть он передаёт воздуху импульс P, тогда сам он приобретает импульс -P.  Молекулы воздуха, столкнувшись с парусом, не прилипают к нему, а отскакивают, поэтому приращение нормальной к парусу составляющей импульса всегда больше до двух раз этой составляющей. Соответственно, импульс, переданный струёй газа парусу, больше (до двух раз) того, который она получила от вентилятора. 
    Следовательно, парусник поплывёт вперёд.

    \answerMath{Парусник поплывёт вперёд.}
\end{enumerate}

\additionalCriteria

\begin{itemize}
    \item Приведено рассуждение показывающее, что равнодействующая сила будет направлена вперед или указан закон сохранения импульса — 8 баллов.
    \item Получен правильный ответ — 2 балла.
\end{itemize}

\begin{enumerate}
    \item[5.] Скорость ветра относительно паруса v равна 10 м/c. Ускорение рассчитаем в инерциальной системе 
    отсчёта, которая движется со скоростью парусника в данный момент. За время $\Delta t$ 
    масса воздуха, ударившаяся о парус, равна $\rho vS \Delta t$. 
    Импульс этой массы воздуха равен $\rho v^2S \Delta t$. 
    При абсолютно упругом ударе переданный импульс в 2 
    раза больше, при абсолютно неупругом равен этому 
    значению. Следовательно, переданный парусу воздухом 
    импульс за время $\Delta t$ можно оценить как:
    $$\rho \cdot v^2\cdot S\cdot \Delta t< \Delta P<2\cdot \rho \cdot v^2\cdot S\cdot \Delta t$$
    
    Отсюда можно оценить ускорение:
    
    $$\frac{\rho \cdot v^2\cdot S}{m}< a<2\cdot \frac{\rho \cdot v^2\cdot S}{m}$$
    
    Осталось вычислить плотность воздуха, исходя из данных условия. По закону Менделеева-Клапейрона:
    
    $$p\cdot V=\frac{m}{\mu} \cdot R\cdot T \rightarrow  \rho =\frac{m}{V}=\frac{p\cdot \mu}{R\cdot T}=\frac{\rho_{Hg}\cdot g\cdot h\cdot \mu}{R\cdot T}$$
    
    Итак, получаем оценку: 
    
    $$a_{min}=\frac{\rho_{Hg}\cdot g\cdot h\cdot \mu\cdot v^2\cdot S}{R\cdot T\cdot m}=\frac{13600\cdot 9.81\cdot 0.75\cdot 0.029\cdot 10^2\cdot 10}{8.31\cdot 293\cdot 200} \approx 5.96 \: \text{м/c}^2$$
    
    \answerMath{$5.96 \: \text{м/c}^2 < a<11.92 \: \text{м/c}^2 $.}
\end{enumerate}

\additionalCriteria

\begin{itemize}
    \item Записан импульс переносимый воздухом — 2 балла.
    \item Записан закон сохранения импульса для удара воздуха о парус — 2 балла.
    \item Учтено, что удар может быть разной упругости  (получено неравенство для импульсов или ускорений) — 2 балла.
    \item Вычислена плотность воздуха — 2 балла.
    \item Получен полный правильный ответ — 2 балла. Если получено только одно значение для граничных случаев (абс. упругий или абс. неупругий удары) — 1 балл.   
\end{itemize}