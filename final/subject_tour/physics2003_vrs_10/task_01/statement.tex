\assignementTitle{На парусной яхте против ветра}{50}{1}

Парусные суда с косым парусным вооружением могут ходить под небольшим углом к направлению ветра, 
практически против него. Объяснить это можно следующим образом.

Ветер толкает парус всегда под прямым углом к его плоскости.

\putTwoImg{6cm}{1}{5cm}{2}

Так как ветер напирает равномерно на всю поверхность паруса, то заменяем силу давления ветра силой $R$, 
приложенной к середине паруса. Эту силу разложим на две: силу $Q$, перпендикулярную к парусу, и силу $Р$, 
направленную вдоль него (см. рис. вверху, посередине). Последняя сила толкает парус, так как трение 
ветра о холст незначительно. Остается сила $Q$, которая толкает парус под прямым углом к нему. 
Ускорение свободного падения 9.81~м/c$^2$, универсальная газовая постоянная 8.31~Дж/(моль$\cdot$К).

Рисунок справа показывает, как можно идти против ветра. Перпендикулярная движению сила $R$ встречает сильное 
сопротивление воды за счёт того, что киль делается глубоким. Можно считать, что смещения перпендикулярно 
курсу судна не возникает. Кроме того, мы пренебрежем креном судна. Остаётся только сила $S$. При расчётах и 
рассуждениях в задачах можно считать, что молекулы воздуха, столкнувшись с парусом, не прилипают к нему, а 
отскакивают, но не абсолютно упруго.

\begin{enumerate}
    \item Во сколько раз увеличится сила тяги, если массу паруса увеличить в 3 раза? Материал и толщина паруса при этом не меняются.
    \item По рейке без трения может скользить брусок. На бруске укреплён парус, угол между ним и рейкой равен 
    $\alpha = 30^{\circ}$. На парус дует ветер со скоростью \linebreak $V = 1$~м/c перпендикулярно рейке. Какую 
    максимальную скорость может развить брусок за счёт разгона ветром? Сопротивление воздуха не учитывать.

    \putImgWOCaption{7cm}{3}

    \item Пусть угол между направлением движения парусника и ветром равен 60 градусов, парусник плывёт против ветра. Под каким углом к ветру надо поставить парус, чтобы сила S была максимальной?
    \item Будет ли двигаться парусная лодка вперёд, если на корме установить мощный вентилятор и направить поток воздуха в парус? Считать, что весь поток летит в парус. Ответ поясните.
    \item Пусть парусник плывёт вдоль ветра, а парус развёрнут перпендикулярно ветру так, чтобы ветер давал 
    максимальную силу.  Скорость ветра равна 20 м/c, парусник уже набрал скорость 10 м/c. Оцените ускорение 
    парусника в этот момент, пренебрегая сопротивлением воды. Площадь паруса равна 10 м$^2$, масса судна с 
    грузом 200 кг, температура воздуха 20$^{\circ}C$, давление 750 мм.рт.столба, плотность ртути 13600 кг/м$^3$, 
    молярная масса воздуха 29~г/моль.
\end{enumerate}