\assignementTitle{Эхолот подводного робота}{}{2}

Эхолот позволяет постоянно мониторить толщу воды под дроном, тем самым отслеживая перепады рельефа дна. Общая структурная схема эхолота выглядит так:

\putImgWOCaption{8cm}{1}

Тактовый генератор G1 управляет взаимодействием узлов прибора и обеспечивает его работу в автоматическом режиме. 
Генерируемые им короткие (0.1 с) прямоугольные импульсы положительной полярности повторяются каждые 10 с.

Своим фронтом эти импульсы устанавливают цифровой счетчик РС1 в нулевое состояние и закрывают приемник А2, 
делая его нечувствительным к сигналам на время работы передатчика. Спадом тактовый импульс запускает передатчик 
А1, и излучатель-датчик BQ1 излучает в направлении дна короткий (40 мкс) ультразвуковой зондирующий импульс. 
Одновременно открывается электронный ключ S1, и колебания образцовой частоты 7500 Гц от генератора G2 поступают 
на цифровой счетчик РС1. По окончании работы передатчика приемник А2 открывается и приобретает нормальную 
чувствительность. Эхосигнал, отраженный от дна, принимается датчиком BQ1 и после усиления в приемнике 
закрывает ключ S1. Измерение закончено, и индикаторы счетчика РС1 высвечивают измеренную глубину. Очередной 
тактовый импульс вновь переводит счетчик РС1 в нулевое состояние, и процесс повторяется.

\begin{enumerate}
    \item Какую часть времени работы эхолота датчик BQ1 работает как излучатель?
    \item Оцените, исходя из данных условия, погрешность измерения расстояния до дна эхолотом, если скорость звука в морской воде равна 1.5~км/сек. 
    \item На рисунке изображён профиль поперечной волны, распространяющейся вправо. В каком направлении 
    движется в данный момент частица среды, обозначенная буквой А? Чему равна скорость движения точки А, если 
    частота волны равна 75 кГц, скорость 1.5 км/c, а кривую на рисунке можно считать синусом.
    
    \putImgWOCaption{7cm}{2}

    \item Узкий пучок волн частоты $\nu = 75$ кГц направлен от неподвижного эхолота дрона к приближающейся подводной лодке. 
    Определить скорость $V$ подводной лодки, если частота $\nu_1$ биений (разность частот колебаний источника и сигнала, 
    отраженного от лодки) равна 50 Гц. Скорость u звука в морской воде принять равной 1.5 км/с.
    \item Электрическая схема приёмника А2 состоит из конденсатора ёмкости \linebreak $C = 50$~мкФ и 
    индуктивно-резистивных элементов. Какой максимальный заряд и напряжение накапливаются на 
    этом конденсаторе, если максимальный общий ток, проходящий через конденсатор, равен 15 А? 
    Частота принимаемого ультразвука равна 75 кГц.
    
\end{enumerate}