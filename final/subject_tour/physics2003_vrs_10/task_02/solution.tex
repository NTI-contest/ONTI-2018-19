\solutionSection

\begin{enumerate}
    \item BQ1 излучает на протяжении 40 мкс каждый раз, 
    когда происходит спад тактового импульса. Тактовый 
    импульс повторяется раз в \linebreak 10 секунд. Это значит, 
    что излучатель находится в рабочем состоянии каждые 
    40 мкс из 10 секунд, то есть $x = 4\cdot 10^{-6}$.
    
    \answerMath{$4\cdot 10^{-6}$.}
\end{enumerate}

\additionalCriteria

\begin{itemize}
    \item Указано (или правильно использовано в вычислении)  условие работы излучателя – 2 балла.
    \item Указано (или правильно использовано в вычислении) время повтора импульса – 2 балла.
    \item Получено выражение для доли времени – 4 балла.
    \item Получен правильный числовой ответ – 2 балла.
\end{itemize}

\begin{enumerate}
    \item[2.] Погрешность измерения расстояния эхолотом 
    определяется одним тактом колебания образцовой частоты 
    7500~Гц, используемой для работы счётчика. Так как эта 
    частота означает, что счётчик переключается 7500 раз за 
    секунду, то погрешность измерения времени $\Delta t = 1/7500$ сек. 
    Если ультразвуковой сигнал дошёл до дна, отразился и 
    вернулся обратно за общее время $t$, то расстояние 
    вычисляется как: $L=\frac{v\cdot t}{2}$. Отсюда следует, 
    что погрешность измерения расстояния равна:  $$\Delta L=\frac{v\cdot \Delta t}{2}=\frac{1500\cdot 1}{2\cdot 7500}= 0.1 \: \text{метр}.$$
    
    \answerMath{0.1 метр.}
\end{enumerate}

\additionalCriteria

\begin{itemize}
    \item Определено основное условие на погрешность измерения  – 4 балла.
    \item Получено выражение для расстояния до дна – 2 балла.
    \item Получено выражение для погрешности измерения – 2 балла.
    \item Получен правильный числовой ответ – 2 балла.
\end{itemize}

\begin{enumerate}
    \item[3.] Чтобы ответить на этот вопрос, начертим профиль волны, который будет иметь место спустя небольшой интервал времени, учитывая, что волна распространяется вправо.
    \putImgWOCaption{7cm}{3}

    Из рисунка видно, что скорость точки А направлена вверх. 
    Модуль скорости точки А равен скорости волны $v = 1.5$ км/сек, так как угловой коэффициент 
    наклона синуса в начале координат равен 1.

    \answerMath{1.5 км/сек.}
\end{enumerate}

\additionalCriteria

\begin{itemize}
    \item Определено направление скорости точки А  – 5 баллов.
    \item Найден модуль скорости точки А – 5 баллов.    
\end{itemize}

\begin{enumerate}
    \item[4.] Частота звука, принимаемого подводной лодкой, 
    равна:
    $$\nu_l=\frac{V+u}{u} \nu$$

    Тогда, частота звука, отражённого от дна (принимаемого локатором), равна:
    $$\nu_2=\frac{u}{u-V} \nu_l$$

    Отсюда получаем:    $$\nu_2=\frac{V+u}{u-V}  \nu$$

    Разность частот колебаний источника и сигнала, 
    отражённого от лодки, равна:
    $$\nu_1=\nu_2-\nu=\nu\cdot \frac{2\cdot V}{u-V}$$

    Отсюда находим скорость приближения подводной лодки к 
    дрону:
    
    $$V=u\cdot \frac{\nu_1}{\nu_1+2\cdot \nu}=1500 \frac{50}{50+2\cdot 75000} \approx 0.500 \: \text{м/c}$$
    
    \answerMath{0.5 м/c.}
\end{enumerate}

\additionalCriteria

\begin{itemize}
    \item Найдена частота звука отраженного от дна – 4 балла.
    \item Найдена разность частот колебаний источника и сигнала – 2 балла.
    \item Получено выражение для скорости лодки – 4 балла.
    \item Получен правильный числовой ответ – 2 балла.
\end{itemize}

\begin{enumerate}
    \item[5.] Чувствительный приёмник работает как 
    колебательный контур. 

    Тогда  максимальная сила тока и максимальный заряд на 
    нём связаны соотношением $I = w\cdot q$, где $w=2\cdot \pi\cdot f=6.28\cdot 75000\approx 471240 \: \text{c}^{-1}$.  Значит, максимальный заряд равен: 
    $$q_{max}=\frac{15}{471240}\approx 31.8 \: \text{мкКл}.$$
    Максимальное напряжение: 
    $$U_{max}=\frac{q_{max}}{C}=\frac{31.8}{50}\approx 0.64 \: \text{В}.$$
        
    \answerMath{$q_{max}\approx 31.8$ мкКл, $U_{max}\approx 0.64$ В.}
\end{enumerate}

\additionalCriteria

\begin{itemize}
    \item Указано, что приемник работает как колебательный контур (или этот факт используется в рассуждениях или вычислениях) – 2 балла.
    \item Найдена собственная частота колебательного контура – 2 балла.
    \item Указана связь между амплитудами заряда и тока – 1 балл.
    \item Получено правильное значение величины заряда – 2 балла.
    \item Получено правильное значение величины напряжения – 2 балла.  
\end{itemize}