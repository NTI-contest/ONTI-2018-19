\solutionSection

\begin{tabular}{l|}
    Дано: \\
    $E = 400$ В/м \\
    $l = 1$ м \\
    $B=10^{-2}$ Тл \\
    $R = 0.3$ м \\
    \hline \\
    $t$ - ?
\end{tabular}

Кинетическая энергия частицы, приобретаемая в однородном электрическом поле
  
$$\frac{mv^2}{2} = qEl, \: \text{откуда} v = \left( \frac{2qEl}{m}\right)^{\frac{1}{2}}$$

Записываем 2 закон Ньютона для частицы, движущейся в поперечном магнитном поле
 
$$\frac{mv^2}{R} = qvR, \: \text{откуда} \frac{q}{m} = \frac{v}{RB} = \frac{\left(\frac{2qEl}{m}\right)^{\frac{1}{2}}}{RB}$$

Возводя в квадрат левую и правую часть, имеем
 
$$\frac{q}{m} = \frac{2El}{R^2B^2} = \frac{2 \cdot 4 \cdot 10^2 \cdot 1}{0.09 \cdot 10^{-4}} \approx 0.9 \cdot 10^8 \: \text{Кл/кг}$$

\begin{enumerate}
    \item[а)] При движении в электрическом поле, согласно 2 закону Ньютона, ускорение $a = qE/m=const$, движение равноускоренное.
 
$$l = v_0 t + \frac{at^2}{2}; \: t = \frac{-v_0 + \sqrt{v_0^2 + 4 \frac{a}{2}l}}{a}$$
$$v_{k} = v_0 +at = \left( v_0^2 + \frac{2qEl}{m}\right)^{\frac{1}{2}}$$

Запишем закон Ньютона при движении в магнитном поле:

$$\frac{mv_k^2}{R_1} = qv_kB; \: R_1 = \frac{mv_k}{qB}; \: \frac{R_1}{R} = \frac{v_1}{v} = \left(\frac{v_0^2m}{2qEl} + 1\right)^{\frac{1}{2}}$$

Радиус вращения частицы в магнитном поле вырос.
\item [б)] $v_0=v_1+v_2$,  где 

$v_1 = v_0 \cos \alpha$ – скорость частицы вдоль силовой линии $E$,

$v_2 = v_0 \sin \alpha$ – скорость частицы перпендикулярно $E$.
\end{enumerate}
Согласно пункту а)  $v_1 = v_0 \cos \alpha \left(\frac{v_0^2m}{2qEl} + 1\right)^{\frac{1}{2}}$.

Далее рассмотрим два предельных случая.

\begin{enumerate}
    \item $v_2 \perp E, \: v_2 || B, \: v_1 \perp B$  

    $$R_1 = \frac{mv_1}{qB} = R \left(\frac{mv_1}{2qEl} + 1\right)^{\frac{1}{2}}$$

    За счет $v_2$ частица смещается вдоль силовой линии $B$, 
    образуя спираль. Радиус спирали $R_1$, а шаг спирали $h=v_2 \cdot T$, 
    где $T$ – период обращения, $T=2 \pi R_1/v_1$.
    \item $v_1 \perp B, \: v_2 \perp B$
    
    В этом случае частица влетает в поперечное магнитное поле со 
    скоростью $v=v_1+v_2$.
    
    Модуль скорости $v=\sqrt{v_1^2 + v_2^2}$.
    
    Радиус вращения $R_2 = \frac{mv}{qB}$.
    
    В магнитном поле радиус вращения возрос. Спирального движения нет.
\end{enumerate}

Удельный заряд исследуемой частицы соответствует элементарной 
частице – протону.

$$q_p=1.6 \cdot 10^{-19} \: \text{Кл}; \: m_p=1.8\cdot 10^{-27} \text{кг}$$
$$\frac{q_p}{m_p}=\frac{1.6\cdot 10^{-19}}{1.8 \cdot 10^{-27}} = 0.9 \cdot 10^8 \: \text{Кл/кг}$$

Если начальная скорость частицы $v_0$ параллельна вектору $E$, то радиус вращения в магнитном поле возрастает:
 
$$\frac{R_1}{R} = \left( \frac{v_0^2m}{2qEl} + 1\right)^{\frac{1}{2}}$$

Если начальная скорость частицы $v_0$ направлена под углом $\alpha$ к вектору $E$, 
то траекторию движения можно обосновать из рисунка:

\putImgWOCaption{7cm}{1}

Составляющая $(v_0)_{\perp}$ может составлять с равной 
вероятностью различные углы с вектором $B$ в интервале от $0$ до $360^\circ$.

Если $(v_0)_{\perp}$ в проекции на $B$ равна $0$, то она 
складывается с $(v_0)_{||}$ векторно и увеличивает радиус 
вращения.

Если имеется не равная нулю проекция $(v_0)_{\perp}$ на $B$, 
то она вызывает смещение частицы вдоль вектора $B$ так, что 
траектория становится винтовой линией.