\assignementTitle{}{}{4}

В измерительном приборе, установленном на космической станции, покоящаяся заряженная частица вначале 
попадает в однородное электрическое поле напряженностью $E=400$~В/м и проходит вдоль силовой линии 
расстояние $l=1$~м,  а затем влетает в однородное поперечное магнитное поле с индукцией $B=10^{-2}$~Тл и 
движется в нем по окружности с радиусом $R=0,3$~м. Вычислить удельный заряд частицы, равный отношению ее заряда к массе. 
Учесть, что в магнитном поле на частицу действует сила Лоренца, направленная к центру окружности и равная 
произведению заряда частицы на ее скорость и на величину магнитной индукции. Как изменится траектория частицы, если:

\begin{enumerate}
    \item[a)] частица влетает в прибор с начальной скоростью $v_0$ вдоль силовой линии вектора $E$;
    \item[б)] частица влетает в прибор с начальной скоростью $v_0$ под углом $\alpha$ к силовой линии электрического поля.
\end{enumerate}