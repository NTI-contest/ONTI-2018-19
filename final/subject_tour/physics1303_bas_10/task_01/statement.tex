\assignementTitle{}{22}{1}

Водород при нормальных условиях (давление $P_0=10^5$~Па, температура $T=300$~К) находится внутри сферы радиусом $R=10$~см. В результате внешнего воздействия $0.1\%$ молекул 
были ионизованы, т.е. от этих молекул было оторвано по одному электрону и эти электроны были изъяты из объема. Найти работу, которую надо затратить, чтобы удалить еще 
один электрон с поверхности сферы на бесконечность. Предполагаем, что заряды распределены по объему внутри сферы равномерно. Заряд электрона $e = 1.6 \cdot 10^{-19}$ Кл, постоянная 
Больцмана $k=1.38 \cdot 10^{-23}$ Дж/К, $\epsilon_0=8.85 \cdot 10^{-12}$ Ф/м.
