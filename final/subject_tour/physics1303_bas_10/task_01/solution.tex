\solutionSection

\begin{tabular}{l|}
    Дано: \\
    $R = 10$ см \\
    $\alpha = 0.1\%$ \\
    $e = 1.6 \cdot 10^{-19}$ Кл \\
    $k = 1.38 \cdot 10^{-23}$ Дж/К \\
    $\epsilon_0 = 8.85 \cdot 10^{-12}$ Ф/м \\
    $P_0 = 10^5$ Па \\
    $T= 300$ К \\
    \hline \\
    $A$ - ?
\end{tabular}

Исходное число молекул водорода находится из уравнения идеального газа

$$PV = NkT$$
$$N = \frac{PV}{kT} = \frac{10^5 \cdot \frac{4}{3} \pi \cdot 10^{-3}}{1.38 \cdot 10^{23} \cdot 3 \cdot 10^2} \approx 10^{23} \: \text{молекул}$$

Число ионизированных молекул

$$N_{i, e} = 10^{-3} \cdot N = 10^{20} \: \text{ионов и электронов}$$

Заряд внутри сферы после удаления электронов

$$Q = e \cdot N_i = 1.6 \cdot 10^{-19} \cdot 10^{20} = 16 \: \text{Кл}$$

Потенциал заряженной сферы

$$\phi(R) = \frac{1}{4 \pi \epsilon_0} \cdot \frac{Q}{R} = \frac{1}{4\pi \cdot 8.85 \cdot 10^{-12}} \cdot \frac{16}{10^{-1}} = 14.4 \cdot 10^{11} \: \text{В}$$

Работа по удалению электрона (принимаем ноль потенциал на бесконечности)

$$A = e \Delta \phi = 1.6 \cdot 10^{-19} \cdot 14.4 \cdot 10^{11} = 23 \cdot 10^{-8} \: \text{Дж}$$

\answerMath{$A = 23 \cdot 10^{-8}$ Дж.}