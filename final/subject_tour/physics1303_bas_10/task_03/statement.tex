\assignementTitle{}{}{3}

Жилой отсек космической станции имеет объем $V=40$ м$^3$ и заполнен воздухом при нормальных условиях (давление $P_0=10^5$ Па, температура $Т=300$ К). 
В результате попадания метеорита образовалась микротрещина, в которую вытекает в секунду в среднем $\Delta N=10^{22}$ молекул. Считая процесс изотермическим, 
рассчитать время, которое отводится космонавтам для обнаружения и устранения утечки воздуха, если допустимое уменьшение давления в жилом отсеке оценивается в $20\%$. 