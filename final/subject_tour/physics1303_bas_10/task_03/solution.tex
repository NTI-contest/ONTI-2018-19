\solutionSection

\begin{tabular}{l|}
    Дано: \\
    $V = 40$ м$^3$ \\
    $\Delta N = 10^{22}$ молекул/с \\
    $\alpha= 20\%$ \\
    $P_0 = 10^5$ Па \\
    $T = 300$ К \\
    \hline \\
    $t$ - ?
\end{tabular}

Начальное количество молекул воздуха в жилом отсеке при нормальных условиях

$$N_0 = \frac{P_0V}{kT} = \frac{10^5 \cdot 40}{1.38 \cdot 10^{-23} \cdot 3 \cdot 10^2} \approx 9.7 \cdot 10^{26} \: \text{молекул}$$

При $T=const$ уменьшение давление на $20\%$ означает, что и количество молекул в отсеке уменьшается на $20\%$, следовательно 
 
$0.2 N_0 = \Delta N \cdot t$, откуда

$$t = \frac{0.2 N_0}{\Delta N} = \frac{0.2 \cdot 9.7 \cdot 10^{26}}{10^{22}} = 19300 \: c = \frac{19300}{3.6 \cdot 10^3} = 5.36 \: \text{часа}$$

\answerMath{$t = 5.36$ часа.}
