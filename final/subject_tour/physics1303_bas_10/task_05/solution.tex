\solutionSection

\begin{tabular}{l|}
    Дано: \\
    $\nu = 1$ моль \\
    $T = 27^\circ C$ \\
    $V = 300$ м/с \\
    \hline \\
    $W_\text{cp}, W_\text{вн}$ - ?
\end{tabular}

Кинетическая энергия направленного движения молекул гелия 
после остановки летательного аппарата переходит в тепловую

$$\frac{3}{2}kT_0 + \frac{m_{He} \cdot \mu^2}{2} = \frac{3}{2} k T_1$$

где $T_0=27+273=300$ К,

$T_1$ – температура молекул газа после остановки,

$m_{He}$ – масса молекулы гелия.

$$m_{He} = \frac{k\mu}{R} = \frac{1.38 \cdot 10^{-23} \cdot 4 \cdot 10^{-3}}{8.31} = 0.66 \cdot 10^{-26} \: \text{кг}$$
 
где $k$ – постоянная Больцмана, $R$ – универсальная газовая постоянная.

$$W_{\text{ср}} =\frac{3}{2}T_1= \frac{3}{2} k T_0 +\frac{m_{He} \nu^2}{2} = \frac{3}{2} \cdot 1.38 \cdot 10^{-23} \cdot 300 + \frac{0.66 \cdot 10^{-26} \cdot 300^2}{2} \approx 6.5 \cdot 10^{-21} \: \text{Дж}$$
 
Внутренняя энергия идеального газа равна сумме кинетических энергий молекул, участвующих в хаотическом движении

$$W_\text{внутр} = W_\text{ср} \cdot N_a = 6.5 \cdot 10^{-21} \cdot 6.02 \cdot 10^{23} = 39.1 \cdot 10^2 \: \text{Дж} \approx 3.9 \: \text{кДж}$$

\answerMath{$W_\text{ср} \approx 6.5 \cdot 10^{-21}$ Дж, $W_\text{вн} \approx 3.9$ кДж.}
