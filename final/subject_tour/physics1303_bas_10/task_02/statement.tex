\assignementTitle{}{}{2}

Предположим, что башню высотой $h=300$ м поставили перпендикулярно поверхности Земли на полюсе и на экваторе. 
Подсчитать время падения камня $t$, опущенного с вершины башни без начальной скорости. На какое расстояние от 
вертикали $\Delta h$ отклонится камень при падении на Землю? Исследовать качественно, как изменятся значения  $\Delta h$ и  
$t$, если башню установить между полюсом и экватором посередине. Трением о воздух пренебречь, принять $R_\text{Земли} = 6400$ км, 
ускорение свободного падения считать равным $g=10$ м/с$^2$ и на экваторе, и на полюсе.