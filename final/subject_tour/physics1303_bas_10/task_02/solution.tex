\solutionSection

\begin{tabular}{l|}
    Дано: \\
    $h = 300$ м \\
    $R_\text{З} = 6400$ км \\
    $g= 10$ м/с$^2$ \\
    \hline \\
    $\Delta h, t$ - ?
\end{tabular}

При падении камня с башни на полюсе вращение Земли на движение камня не влияет.

$$h = \frac{gt^2}{2}, \: \text{откуда} \: t=\sqrt{\frac{2h}{g}} = \sqrt{\frac{2\cdot 300}{20}} \approx 7.75 \: \text{c}$$
 
Отклонения от вертикали нет.

Изобразим (без соблюдения масштабов) Земной шар и башню на экваторе (вид со стороны полюса).
 
\putImgWOCaption{6cm}{1}

$v_\text{Б}$ – линейная скорость верхнего конца башни, связанная с вращением Земли;

$v_\text{З}$ – линейная скорость поверхности Земли на экваторе, связанная с вращением Земли;

$T=24$ часа – период обращения Земли;

$\omega =2 \pi /T$ – угловая скорость вращения. 

$$\omega \approx 7.3\cdot 10^{-5} \: c^{-1}$$
$$v_\text{Б}= \omega (R_\text{З}+h)= 7.3\cdot 10^{-5}\cdot 6.4003\cdot 10^6 \approx 467.222 \: \text{м/с}$$
$$v_\text{З}=\omega \cdot R_\text{З}=7.3\cdot 10^{-5}\cdot 6.4\cdot 10^6=467.2 \: \text{м/с}$$

Время падения на экваторе равно $t = 7.75$ с.

Отклонение от вертикали 
$$\Delta h=( v_\text{Б} - v_\text{З})\cdot t=0.022\cdot 7.75 \approx 0.17 \text{м}.$$

Изобразим  (без соблюдения масштабов) земной шар и башню в данном случае. В более крупном масштабе отдельно изобразим башню с указанием действующих сил на камень на уровне верхней точки башни.

\putTwoImg{5cm}{2}{5cm}{3}

$$r_1 = R_\text{З} \cos 45^\circ = R_\text{З} \frac{\sqrt{2}}{2}$$

$$r_2 = (R_\text{З} + h) \cos 45^\circ = (R_\text{З} + h) \frac{\sqrt{2}}{2}$$
 
$F_1=mg$ – сила притяжения к Земле, действующая на камень по закону всемирного тяготения.

$F_2=m\omega^2r_2$ – центробежная сила, действующая на камень в верхней точке башни.

$F=F_1+F_2$ – результирующая сила притяжения камня к Земле.

$\alpha$ – угол между $F_1$ и $F$.

$\beta=45^\circ$ - по условию задачи.

Скорость вращения камня в верхней точке башни 

$$v_\text{Б} = \omega r_2 = \omega (R_\text{З} + h) \frac{\sqrt{2}}{2}$$

Скорость вращения основания башни

$$v_\text{З} = \omega r_1 = \omega R_\text{З} \frac{\sqrt{2}}{2}$$

Значение $F$ находится по теореме косинусов  

$$F = \sqrt{F_1^2 + F_2^2 -2F_1F_2 \cos \alpha}$$

$$F_1 = mg; \: F_2 = m \omega^2 (R_\text{З} + h) \frac{\sqrt{2}}{2};$$
$$\frac{F_2}{F_1} = \frac{\omega^2(R_\text{З} + h) \frac{\sqrt{2}}{2}}{g} = \frac{53.3 \cdot 10^{-10} \cdot 6.4003 \cdot 10^{6} \cdot 0.7}{10} = 23.9 \cdot 10^{-4}$$
 
Полагаем $g = 10$ м/с$^2$, $t = 7.75$ c.

$\Delta h$, связанное с вращением Земли в данном случае

$$\Delta h = (v_\text{Б} - v_\text{З})t = \omega \left[ (R_\text{З} + h) \frac{\sqrt{2}}{2} - R_\text{З}\frac{\sqrt{2}}{2} \right] \cdot t = 0.17 \frac{\sqrt{2}}{2} \approx 12 \: \text{м}$$
 
По теореме синусов  

$$\frac{F_2}{\sin \alpha} = \frac{F}{\sin 45^\circ}; \: \sin \alpha = \frac{F_2}{F} \sin 45^\circ = 23.8 \cdot 10^{-4} \cdot \frac{\sqrt{2}}{2} \approx 16.7 \cdot 10^{-4}$$

Следовательно, отклонением камня по меридиану в сторону экватора можно пренебречь.
 
\answerMath{$\Delta h = 0.17$ м; $t = 7.75$ с. В случае установки башни между полюсом и экватором посередине $\Delta h = 0.12$ м, $t = 7.75$ с.}
