\solutionSection
Способ 1
Визуализируем условия задачи в виде таблицы посещаемости офиса компании:

\begin{tabular}{ | l | l | l | l | l | l | l | l |}
\hline
  -      & Пн & Вт & Ср & Чт & Пт & Сб & Всего \\ \hline
1 камера & + & + & + & + & + & + & 510 \\
2 камера & + & + & + & - & - & - & 392 \\
3 камера & - & + & - & - & + & - & 220 \\
4 камера & - & - & + & + & - & + & 208 \\
5 камера & - & - & - & + & + & + & 118 \\
\hline
\end{tabular}

За исключение понедельника каждый день упоминается 3 раза. Это приводит к двойному учету посетителей четырьмя последними камерами во все дни кроме понедельника. Таким образом, искомое количество посетителей в понедельник можно найти из следующего выражения:

\begin{displaymath}
2*510-(392+220+208+118)=1020-938=82
\end{displaymath}

Способ 2

Составим систему уравнений:

\begin{equation*}
   \begin{cases}
    \text{ПН}+\text{ВТ}+\text{СР}+\text{ЧТ}+\text{ПТ}+\text{СБ}=510,\\
    \text{ПН}+\text{ВТ}+\text{СР} =392, \\
    \text{ВТ}+ \text{ПТ}=220, \\
    \text{СР}+\text{ЧТ}+ \text{СБ}=208. \\
    \text{ЧТ}+\text{ПТ}+\text{СБ}=118. \\ 
   \end{cases}
  \end{equation*}

Решив систему уравнений, можно попытаться найти ответ. Однако, следует заметить, решение системы значительно упрощается, если обратить внимание на тот факт, что ответ можно получить, вычитая уравнения (3) и (4) из уравнения (1), т.е. ПН =82.
\answerMath{В офисе компании ALT-Future в понедельник побывало 82 человека.}