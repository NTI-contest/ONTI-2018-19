\assignementTitle{}{}{2}

Существуют три кольцевых маршрута автобусов №1, №2, №3. Первый пересекается со вторым на одной остановке Q, 
второй пересекается с третим на одной остановке S. Первый и третий маршруты не пересекаются. Автобусы курсируют 
по часовой стрелке. Время полного круга каждого из маршрутов: $L_1 = 59$, $L_2 = 37$ и $L_3 = 89$ минут 
соответственно. Автобусы №1, №2 начинают движение одновременно в момент времени $t=0$ на остановке Q.  
Маршрут №3 начинает движение с остановки Z, когда маршрут №2 проходит остановку S.
Пассажиру нужно доехать с остановки A маршрута №1, от которой до остановки Q ехать $\delta_{AQ} = 23$ 
минуты. От S до Z $\delta_{SZ} = 41$ минуты. Время движения от Q до S равно $\delta_{QS} = 19$ минуты. 
Найти первое возможное время посадки на останоке А так, что время в пути до Z минимально. Время считать 
в минутах от начала движения маршрутов №1 и №2, автобусы курсируют бесконечно. Пересадки считать 
моментальными, пересадка возможна если автобусы оказываются на одной остановке в одно и то же время с 
точностью до минуты и во все последующие моменты времени.