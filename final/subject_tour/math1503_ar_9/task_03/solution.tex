\solutionSection

Пусть x>5 моделей делает ЗD-дизайнер за 1 час, тогда стажер за один час делает x-2 трехмерные модели. Пусть также ЗD-дизайнер выполняет заказ за t часов, где t – целое число. Составим уравнение: 

\begin{displaymath}
xt=2(x-2)(t-1)
\end{displaymath}\\
\begin{displaymath}
t=\frac{2x-4}{x-4}
\end{displaymath}\\
\begin{displaymath}
t= 2 + \frac{4}{x-4}
\end{displaymath}

Дробь 4/(x-4) должна быть целым числом. При x>5 это возможно, когда x=6 или x=8. В первом случае получаем, что t=4, во втором – t=3. В обоих случаях техническое задание проекта содержит заказ на изготовление xt=24 трехмерные модели.

\answerMath{Техническое задание, поступившее в компанию ALT-Future, содержало заказ на разработку 24 трехмерных моделей}