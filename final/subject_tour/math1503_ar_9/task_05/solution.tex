\solutionSection

Рассмотрим треугольник ABC. 
\putTwoImg{6cm}{3}{6cm}{4}

Если считать, что все стороны квадрата раны 6, то треугольник будет иметь стороны 3, 4, 5. Радиус трех нижних окружностей равен 1, тогда нужно доказать, что радиус вписанной в ABC окружности так же равен 1: 
Рассмотрим треугольник IBC. У него основание BC и высота r, \Rightarrow Sibc =  \frac{BC*r}{2}
Точно так же в треугольниках ICA И IAB площади будут равны:
\begin{displaymath} 
\frac{CA*r}{2} 
\end{displaymath}\\
\begin{displaymath} 
\frac{AB*r}{2};
\end{displaymath}

Учитывая, что площадь треугольника \begin{displaymath}  ABC = \frac{AC*CB}{2}, \end{displaymath} получаем следующее равенство: 
\begin{displaymath}
\frac{AC*AB}{2} = \frac{BC*r}{2} + \frac{CA*r}{2} +  \frac{AB*r}{2}; \Rightarrow 
\end{displaymath}\\
\begin{displaymath}
AC*AB = (BC + CA + AB)*r;
\end{displaymath}\\

Подставляя BC=3, CA=4, AB=5 получаем r=1.\\

Что и требовалось доказать.