\assignementTitle{Задача 4}{35}{}

С расстояния $100$ метров в стену с отверстиями разного диаметра последовательно бросают мячи, которые имеют разный радиус. Мячи всегда бросают из одной позиции. Необходимо определить количество шаров пролетевших сквозь стену. При этом движение мяча рассматривается как линейное, т.е. без учета гравитации. Для каждого шара задан угловой размер и координаты на расстоянии в $50$ метров после броска. Пролет мяча считается только при чистом пролете мяча через отверстие, т.е. если мяч не задел край.

\inputfmtSection

В первой строке задается целое число $n\space (1\leq n\leq 100)$ количество брошеных шаров. Во второй строке задано целое число $h\space(1\leq h\leq 100)$ — количество отверстий в стене. В третьей строке через пробел заданы координаты точки, из которой бросают шары $(x, y, z)$. В следующих $h$ строках заданы координаты отверстий в стене и диаметр отверстий через пробелы $(x, y, z, d)$. За ними следуют координаты мячей и угловой размер на $50$ метрах в минутах $(x_{50}, y_{50}, z_{50}, s)$. Координаты, размеры мячей и размеры отверстий представлены целыми числами. Угловой размер задается в минутах.

\outputfmtSection

Единственное целое число — количество шаров, пролетевших сквозь стену.

\sampleTitle{1}

\begin{myverbbox}[\small]{\vinput}
    14
    6
    100 10 10
    0 10 10 4
    0 17 10 1
    0 4 4 2
    0 2 12 2
    0 20 14 3
    0 24 5 2
    50 10 10 120
    50 10 10 350
    50 9 10 100
    50 9 9 150
    50 24 4 1
    50 13 7 100
    50 8 7 200
    50 7 7 10
    50 6 7 100
    50 12 3 200
    50 11 5 100
    50 20 14 4
    50 2 3 120
    50 23 3 1
\end{myverbbox}
\begin{myverbbox}[\small]{\voutput}
    3
\end{myverbbox}
\inputoutputTable

\includeSolutionIfExistsByPath{final/subject_tour/inf1503_ar/task_04}
