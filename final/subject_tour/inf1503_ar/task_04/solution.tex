\solutionSection

Линейный размер мяча определяется $size = 2\cdot L\cdot \sin(D/2)$, где $L$ расстояние до объекта, $D$ — угловой размер. Координаты попадания мяча в стену определяются исходя из того, что нормаль из начальной точки и прямая проходящая через начальную точку и точку на 50 метрах образуют подобные треугольники по двум углам, при заданных условиях коэффициент равен 2. Таким образом координаты ударения мяча в стену можно определить по удвоенному смещению начальной точки от точки на 50 метрах прибавленному к координатам начальной точки. Определение пролета мяча сквозь стену определяется следующим образом: $r_1 > (d + r_2)$ 

\codeExample

\inputPythonSource
%\inputminted[fontsize=\footnotesize, linenos]{python}{final/subject_tour/inf2003_vr/task_01/source.cpp}