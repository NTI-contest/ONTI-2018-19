\assignementTitle{Задача 3}{30}{}

В заданной матрице определить максимально длинную последовательность из значений, которые отличаются друг от друга на $1$, вывести полученную последовательность в порядке возрастания. Последовательность должна состоять из чисел, которые прилегают друг к другу, т.е. между числами нет значения $0$. Минимальное число от которого может начинаться последовательность $\geq$. Значение $0$ не участвует в составлении последовательности. У каждого значения в последовательности может быть не больше двух соседних значений включая диагонали. Если найдено несколько последовательностей одинаковой длины, то вывести последовательность с максимальной суммой.

\inputfmtSection

В первой строке через пробел заданы $h, \space w\space (10\leq h, w \leq 50)$ — высота и ширина матрицы соответственно. В следующих строках заданы значения матрицы по строкам и столбцам.

\outputfmtSection

Последовтельность целых чисел, каждое число в новой строке — ответ на вопрос задачи.

\sampleTitle{1}

\begin{myverbbox}[\small]{\vinput}
    10 20
    0 0 0 0 0 0 0 0 0 0 0 0 0 0 0 0 0 0 0 0
    0 0 0 0 0 1 2 0 0 0 0 0 0 9 8 7 6 0 0 0
    0 0 0 2 0 0 0 0 0 0 0 0 0 10 0 0 0 0 0 0
    0 0 0 3 0 0 0 0 5 0 0 0 0 11 0 0 0 0 1 0
    0 0 0 4 0 0 0 0 6 0 0 0 0 12 0 0 0 0 2 0
    0 0 0 5 0 0 0 0 7 0 0 0 0 13 14 15 0 0 3 0
    0 0 0 6 0 0 0 0 8 0 0 0 0 0 0 0 0 0 4 0
    0 0 0 7 0 0 0 0 9 10 11 12 13 14 0 0 0 0 0 0
    0 0 0 0 0 0 0 0 0 0 0 0 0 0 0 0 0 0 0 0
    0 0 0 0 0 0 0 0 0 0 0 0 0 0 0 0 0 0 0 0
\end{myverbbox}
\begin{myverbbox}[\small]{\voutput}
    6
    7
    8
    9
    10
    11
    12
    13
    14
    15
\end{myverbbox}
\inputoutputTable

\includeSolutionIfExistsByPath{final/subject_tour/inf1503_ar/task_03}
