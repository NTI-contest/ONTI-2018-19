\solutionSection

Первым шагом в решении после считывания матрицы может быть выборка последовательностей из неё. Для этого подойдёт рекурсивный обход или вложенный цикл. Для избежания дублирования выборки можно отмечать уже обработанные ячейки в отдельном массиве. Далее отбираем для каждой последовательности подпоследовательность — числа, отличающиеся только на единицу друг от друга, например, 3, 4, 5, 6. Отобранные последовательности ранжируем (сортируем) по сумме элементов.

\codeExample

\inputPythonSource
%\inputminted[fontsize=\footnotesize, linenos]{python}{final/subject_tour/inf2003_vr/task_01/source.cpp}