\assignementTitle{Задача 1}{15}{}

Космический корабль путешествует от одной планеты к другой планете. Необходимо определить за какое наименьшее количество прыжков корабль достигнет точки назначения. Корабль может прыгать только к ближайшей планете и не может посещать уже ранее посещенные планеты.

\inputfmtSection

В первой строке задано количество звезд $n\space (2 \leq n \leq 100)$. Вторая строка содержит номера стартовой и конечной планет разделенные пробелом, нумерация начинается с $0$. Следующие строки содержат целочисленные координаты планет в трехмерном пространстве $x,\space y,\space z\space (-100\leq x, y, z\leq 100)$ разделенные пробелом. 

\outputfmtSection

В отдельной строке единственное целое число — ответ на задачу.

\sampleTitle{1}

\begin{myverbbox}[\small]{\vinput}
    111
    0 5
    0 0 0
    2 3 0
    4 1 2
    -2 3 0
    7 1 3
    6 5 4
    3 2 7
    1 1 1
    6 8 -8
    1 -1 -1
    2 -5 2
\end{myverbbox}
\begin{myverbbox}[\small]{\voutput}
    5
\end{myverbbox}
\inputoutputTable

\includeSolutionIfExistsByPath{final/subject_tour/inf1503_ar/task_01}