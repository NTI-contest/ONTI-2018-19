\solutionSection

Считаем координаты планет в трёхмерном пространстве в массив. Нам также потребуется вспомогательная функция, рассчитывающая расстояние между точками в пространстве (по теореме Пифагора) и массив для учёта уже посещённых планет. Начинаем со стартовой планеты и двигаемся к ближайшей (определяя расстояния до всех не посещённых ранее планет), считая число перемещений до тех пор, пока не достигнем точки назначения. Важно учесть, что стартовая и конечная точка могут совпадать.

\codeExample

\inputPythonSource
%\inputminted[fontsize=\footnotesize, linenos]{python}{final/subject_tour/inf2003_vr/task_01/source.cpp}