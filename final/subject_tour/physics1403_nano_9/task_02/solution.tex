\solutionSection


\begin{enumerate}
    \item Напряжение на резисторе равно разности напряжений между напряжением источника и напряжением на диоде. Тогда:
    $$R=\frac{(U-U_D)^2}{P_0}=\frac{(4.5-3)^2}{0.03}=75\text{ Ом}$$
    \answerMath{$R=\frac{(U-U_D)^2}{P_0}=75\text{ Ом}$}
    \end{enumerate}
    
    \markSection

    \begin{itemize}
    \item Найдено напряжение на резисторе
    \item Получено выражение для сопротивления
    \item Получен правильный численный ответ.
    \end{itemize}
    \begin{enumerate}
    \item [2.] Найдем силу тока, текущего в цепи:
    $$I=\frac{U_R}{R}=\frac{1.5}{75}\text{ А}=20\text{ мА}$$
    Тогда 
    $$P=IU=0.02\cdot1.5=0.03\text{ Вт}$$
    Учитывая, что только излучение составляет только 25\% от выделяющейся мощности:
    $$P_c=kIU=0.25\cdot0.02\cdot1.5=7.5\text{ мВт}$$
    \answerMath{$P_c=kIU=7.5\text{ мВт}$}
    \end{enumerate}
    
    \markSection

    \begin{itemize}
    \item Найдена сила тока в цепи.
    \item Получено выражение для мощности излучения.
    \item Получен правильный численный ответ.
    \end{itemize}
    \begin{enumerate}
    \item [3.] Из графика видно, что мощность делится между синим, зеленым и красным элементами диода как 4/9, 3/9 и 2/9 соответственно.\\ 
    Тогда на синий свет приходится мощность $P_B=\frac{4}{9}\cdot7.5=3.33 \: \text{ мВт}$\\
    на зеленый: $P_G=\frac{3}{9}\cdot7.5=2.5\:\text{ мВт}$\\
    на красный: $P_R=\frac{2}{9}\cdot7.5=1.66\:\text{ мВт}$\\
    Энергия, которую переносит один фотон может быть выражена через его частоту, как:
    $$E=h\nu=\frac{hc}{\lambda}$$
    Где, $h = 6.63\cdot10^{-34}$ Дж$\cdot$с – постоянная Планка.\\
    Тогда 
    $$N=\frac{P}{E}=\frac{P\lambda}{hc}$$
    \answerMath{\\$N_B=7.46\cdot10^18$ частиц/с\\
    $N_G=6.72\cdot10^{18}$ частиц/с\\
    $N_B=5.36\cdot10^{18}$ частиц/с}
    \end{enumerate}
    
    \markSection

    \begin{itemize}
    \item Найдена мощность, приходящаяся на каждый цвет
    \item Записано выражение для энергии фотона
    \item Получено выражение для числа частиц
    \item Получен правильный численный ответ.
    \end{itemize}
    \begin{enumerate}
    \item [4.] Из закона преломления выразим показатели преломления для крайних значений и приравняем их к выражению формулы Коши.
    $$\frac{\sin{\alpha_0}}{\sin{\alpha_1}}=a+\frac{b}{\lambda_1^2}$$
    $$\frac{\sin{\alpha_0}}{\sin{\alpha_2}}=a+\frac{b}{\lambda_2^2}$$
    Вычитая одно из другого получим выражение для $b$
    $$b=\frac{\sin{\alpha_0}\cdot\left(\frac{1}{\sin{\alpha_2}}-\frac{1}{\sin{\alpha_1}}\right)}{\frac{1}{\lambda_2^2}-\frac{1}{\lambda_1^2}}$$
    Углы $\alpha_1$ и $\alpha_2$ можно найти из геометрии:
    $$\alpha_1=\arctg\left(\frac{l_2}{H}\right)=0.356$$
    $$\alpha_2=\arctg\left(\frac{l_1+l_2}{H}\right)=0.365$$
    $$b\approx12650\text{ нм}^2$$
    После этого a можно найти из любого выражения для закона преломления:	
    $$a=\frac{\sin{\alpha_0}}{\sin{\alpha_1}}-\frac{b}{\lambda_1^2}=1.37$$
    \answerMath{$b=\frac{\sin{\alpha_0}\cdot\left(\frac{1}{\sin{\alpha_2}}-\frac{1}{\sin{\alpha_1}}\right)}{\frac{1}{\lambda_2^2}-\frac{1}{\lambda_1^2}}=12650\text{ нм}^2$, $a=\frac{\sin{\alpha_0}}{\sin{\alpha_1}}-\frac{b}{\lambda_1^2}=1.37$}
    \end{enumerate}
    
    \markSection

    \begin{itemize}
    \item Записан закон преломления 
    \item Записано равенство показателя преломления, выраженного через закон преломления и через формулу Коши.
    \item Получено выражения для краевых углов спектра.
    \item Получено выражение для b
    \item Получено значение для b значение
    \item Получено выражение для a
    \item Получено значение для a
    \end{itemize}
    
    \begin{enumerate}
    \item [5.] Как видно из рисунка, свет выйдет из пластины параллельным пучком. Таким образом, если пренебречь потерями энергии, можно сказать, что независимо от расположения экрана ширина «радуги» будет одинаковой.
    \begin{center}
    \putImgWOCaption{8cm}{3}
    \end{center}
    Воспользовавшись рисунком и решением предыдущего пункта запишем:
    $$L=2H(\tg\alpha_2-\tg\alpha_1)=2.005\text{ см}$$
    \answerMath{$L=2H(\tg\alpha_2-\tg\alpha_1)=2.005\text{ см}$}
    \end{enumerate}
    
    \markSection

    \begin{itemize}
    \item Отмечено, что ширина спектра не зависит от расстояния до экрана.
    \item Записано выражение для ширины.
    \item Получен правильный численный ответ.
    \end{itemize}