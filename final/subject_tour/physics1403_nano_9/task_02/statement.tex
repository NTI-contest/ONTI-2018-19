\assignementTitle{Оптика}{}{2}

Для работы светодиода необходимо напряжение не выше $U_d = 3$В. RGB-светодиод подключен к источнику постоянного 
напряжения $U = 4,5$В сопротивлением, которого можно пренебречь. Для того, чтобы светодиод мог работать, в цепь 
подключили резистор. Доля мощности, выделяющаяся в виде излучения на светодиоде $k = 25\%$, светодиод излучает 
свет с длинами волн $\lambda_1=445$нм, $\lambda_2=535$нм и $\lambda_3=640$нм. Скорость света считайте равной 
$c=3 \cdot 10^8$ м⁄с, $h = 6,63 \cdot 10-{34}$ Дж $\cdot$ с – постоянная Планка

\begin{enumerate}
    \item Каково сопротивление резистора, подключенного в цепь, если на нем выделяется тепловая мощность $P_0 = 30$мВт?
    \item Какова мощность излучения диода при этом?
    \item В условия предыдущих пунктов по спектру излучения светодиода оцените число фотонов каждого цвета, излучаемых светодиодом за секунду. Для простоты считайте, что в создании каждого цвета у светодиода участвует строго одна длина волны.
    \putImgWOCaption{7cm}{1}
    \item Одним из подтверждений волновой природы света считается явление дисперсии, которое можно наблюдать 
    как разложение белого света на составляющие его цвета. Это явление возникает из-за того, что коэффициент 
    преломления вещества зависит от длины волны света, хотя и не очень сильно. Это явление было эмпирически 
    описано Огюстеном Коши, в честь которого получила название формулы Коши: $n(\lambda)=a+\frac{b}{\lambda^2} + \frac{m}{\lambda^4}$  
    Третье слагаемое мы использовать не будем, т.к. вклад его совсем мал.
    
    При падении узкого луча света от светодиода на плоскую прозрачную пластинку толщиной $H = 100$ см под углом 
    $\alpha = 30^{\circ}$, фотоматрица под пластинкой зафиксировала три сигнала, соответствующих разным цветам. 
    При этом “радуга” имела ширину (расстояние между крайними точками) $l_1 = 1$ см, а началась на расстоянии 
    $l_2 = 37,2$см по горизонтали от точки падения луча. Найдите коэффициенты $a$ (безразмерная) и $b$ (в нм$^2$) в 
    формуле Коши для материала пластинки.
    \putImgWOCaption{6cm}{2}
    \item Какой ширины будет “радуга” на экране расположенном над прозрачной пластинкой параллельно ей, если вместо фотоматрицы под пластинкой будет находиться зеркало?
\end{enumerate}