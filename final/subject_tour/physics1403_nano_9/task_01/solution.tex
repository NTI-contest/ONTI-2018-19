\solutionSection

\begin{enumerate}
\item Если шарик не меняет свою глубину плавания, то его средняя плотность равна плотности жидкости. Следовательно, его масса может быть вычислена как:
$$m=\rho\cdot\frac{4}{3}\pi R^3=1260\cdot\frac{4}{3}\cdot3.14\cdot3^3\cdot10^{-9}=1.4\cdot10^{-3}\text{ кг.}$$
\answerMath{$m=\rho\cdot\frac{4}{3}\pi R^3=1.4\cdot10^{-3}\text{ кг.}$}
\end{enumerate}

\markSection

\begin{itemize}
\item Записан закон плавания тела или сказано, что для плавания без всплытия плотности тела и жидкости должны быть равны.
\item Получено правильное числовое значение с точностью до десятых.
\end{itemize}

\begin{enumerate}
\item [2.] Из закона сохранения импульса следует, что после соударения шарики остановятся. Следовательно, вся кинетическая энергия шариков выделится, перейдя во внутреннюю. По условию задачи 40\% этой энергии пошло на нагрев. Соответственно:
$$2\frac{mV^2}{2}=2\cdot0.4cm\cdot dt$$
$$V^2=2\cdot0.4c\cdot dt$$
$$V=\sqrt{2\cdot0.4c\cdot dt}=\sqrt{2\cdot0.4\cdot1250\cdot0.1}\text{ м/с}=10\text{ м/с}$$
\answerMath{$V=\sqrt{0.8c\cdot dt}=10\text{ м/с}$}
\end{enumerate}

\markSection

\begin{itemize}
\item Записан закон сохранения импульса или указано, что после соударения тела остановятся.
\item Записан закон сохранения энергии.
\item Вычислено правильное значение скорости.
\end{itemize}
\begin{enumerate}
\item [3.] Из решения 1 и 2 задачи найдем теплоемкость шарика $C_\text{ш}=m\cdot C=1.4\cdot10^{-3}\cdot1250=1.75\text{ Дж/С}^{\circ}$\\
Запишем уравнение теплового баланса после броска первого и после броска второго шариков:
$$C_\text{ж}\Delta t_1=C_\text{ш}(t_0-\theta)$$
Где $\theta$ – установившаяся температура, а $t_0$ – начальная температура шарика.
$$C_\text{ж}\Delta t_2+C_\text{ш}\Delta t_2=C_\text{ш}(t_0-\theta_2 )$$
Вычитая первое из второго получим: $C_\text{ж}(\Delta t_1-\Delta t_2 )=2C_\text{ш}\Delta t_2$
Отсюда $C_\text{ж}=\frac{2C_\text{ш}\Delta t_2}{(\Delta t_1-\Delta t_2)}=38.5\text{ Дж/К}$\\
\answerMath{$C_\text{ж}=\frac{2C_\text{ш}\Delta t_2}{(\Delta t_1-\Delta t_2)}=38.5\text{ Дж/К}$}
\end{enumerate}

\markSection

\begin{itemize}
\item Найдена теплоемкость шарика.
\item Записано уравнение теплового баланса для первого случая
\item Записано уравнение теплового баланса для второго случая.
\item Вычислено правильное значение теплоемкости.
\end{itemize}
\begin{enumerate}
\item [4.] Заметим, что для плавающего тела сила Архимеда равна силе тяжести, тогда при разбавлении можно записать второй закон Ньютона в следующей форме:
$$mg-\frac{\rho gV}{N}-\alpha U_1=0$$
Или
$$mg\left(1-\frac{1}{N}\right)-\alpha U_1=0$$
Откуда:
$$\alpha=\frac{mg\left(1-\frac{1}{N}\right)}{U_1}=\frac{0.006\cdot9.8\cdot0.5}{0.04}=0.735\text{ (Н}\cdot\text{с)/м}$$
\answerMath{$\alpha=\frac{mg\left(1-\frac{1}{N}\right)}{U_1}=0.735\text{ (Н}\cdot\text{с)/м}$}
\end{enumerate}

\markSection

\begin{itemize}
\item Указано условие плавания тела
\item Записан второй закон Ньютона для тонущего тела
\item Вычислено правильное значение коэффициента вязкого трения.
\end{itemize}
\begin{enumerate}
\item [5.] Пусть масса тела, которое движется вниз $m1$, а тела, которое движется вверх – $m2$. Т.к. объемы тел одинаковы, понятно, что $m1 > m2$, из чего следует, что импульс тела движущегося вниз больше, чем импульс тела, движущегося вверх. Следовательно, после соударения слипшиеся тела продолжат двигаться вниз до тех пор, пока не остановятся.\\
Запишем закон сохранения импульса для момента соударения:
$$m_1V-m_2V=(m_1+m_2)U$$
Откуда: 
$$U=\frac{(m_1-m_2)}{(m_1+m_2 )}V$$
Запишем второй закон Ньютона для тел:
$$m_1 g-F_1-F_\text{Арх}=0$$
$$m_2 g+F_1-F_\text{Арх}=0$$
где $F_1$ – сила сопротивления для тел, одинаковая в силу их одинакового размера, формы и среды, в которой они движутся.\\
Преобразуя эти уравнения получим:
$$(m_1+m_2)g=2F_\text{Арх}$$
$$(m_1-m_2)g=2F_1$$
Отсюда
$$\frac{(m_1-m_2)}{(m_1+m_2)}=\frac{F_1}{F_\text{Арх}}=\frac{K}{N}$$
Запишем теорему о кинетической энергии для движения тела после удара. Будем считать, что объем тела при ударе не изменился.
$$-\frac{(m_1+m_2)U^2}{2}=\left((m_1+m_2)g-2F_\text{Арх}-F_2\right)l$$
Где $F_2$ – сила сопротивления действующая на образовавшееся тело, а $l$ – искомое расстояние.\\
Заметим, что сила тяжести полностью компенсирует силу Архимеда. Т.е. в после остановки тело будет находиться в равновесии.\\
Откуда 
$$l=\frac{(m_1+m_2)U^2}{2F_2}=\frac{(m_1+m_2)\left(\frac{K}{N}\right)^2V^2}{2F_2}$$
Заметим, что $(m_1+m_2)=2F_\frac{\text{Арх}}{g}$\\
Тогда:
$$l=\frac{\frac{2F_\text{Арх}}{g}\left(\frac{K}{N}\right)^2V^2}{2F_2}=\frac{\left(\frac{K}{N}\right)^2V^2}{gN}=0.004\text{ м}$$
\answerMath{$l=\frac{\left(\frac{K}{N}\right)^2V^2}{gN}=4\text{ мм}$}
\end{enumerate}

\markSection

\begin{itemize}
\item Замечено, что после соударения тело будет двигаться вниз
\item Записаны законы Ньютона для движения тела
\item Записан закон сохранения энергии или теорема о кинетической энергии
\item Получено итоговое выражение.
\item Получен правильный численный ответ.
\end{itemize}