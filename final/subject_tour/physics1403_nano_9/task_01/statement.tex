\assignementTitle{Вязкая жидкость}{}{1}

\begin{enumerate}
    \item Шарик диаметром 6 мм плавает, не меняя глубины, в жидкости плотности $\rho=1,26$~г/см$^3$. Какова масса шарика?
    \item При абсолютно неупругом соударении двух таких шариков, двигавшихся вдоль одной прямой навстречу друг другу с одинаковыми скоростями, 
    40\% выделившейся энергии пошло на их нагрев. С какой скоростью двигались шарики перед соударением, если в результате удара они нагрелись на 
    $\Delta t = 0,1^{\circ}C$? Удельная теплоемкость шарика равна $C=1250$~Дж/кг$^{\circ}$С.
    \item В жидкость кинули такой шарик, нагретый до некоторой температуры выше, чем температура жидкости. После установления теплового 
    равновесия, температура жидкости оказалась на $\Delta t_1 = 2,4^{\circ}C$ выше, чем была первоначально. После этого, не вынимая первый 
    шарик, в воду кинули еще один такой же, нагретый до той же первоначальной температуры, что и первый. После установления теплового 
    равновесия во второй раз, температура жидкости оказалась еще на $\Delta t_2 = 2,2^{\circ}C$ выше, относительно установившейся 
    после броска первого шарика. Какова теплоемкость жидкости, если теплоемкость шариков такая же, как в пункте 2? 
    Теплоемкостью сосуда пренебрегите.
    \item Пусть этот шарик находится в равновесии в глубине жидкости. Жидкость разбавили так, что ее плотность 
    стала меньше в $N = 2$ раза, и шарик начал тонуть, причем установившаяся скорость движения оказалась постоянной и равной $U_1 = 4$ см/с. 
    Определите коэффициент вязкого трения для этого шарика в разбавленной жидкости, 
    считая, что сила вязкого трения пропорциональна первой степени скорости.
    \item Два разных шарика одинакового объема движутся в жидкости вдоль одной вертикальной прямой со скоростями 
    $V = 30$~см/c, так что их скорости постоянны, равны по модулю, но направлены в противоположных направлениях. 
    Где эти два шарика остановились после абсолютно неупругого удара, относительно точки, где удар произошел, 
    если сила средняя сопротивления после удара составила $N = 126\%$ от силы Архимеда действовавшей на один шар и 
    $K = 154\%$ от силы сопротивления действующей на один шар до соударения. Турбулентными процессами пренебрегите. 
    Ускорение свободного падения $g = 10$~м/c$^2$, сила сопротивления пропорциональна первой степени скорости, 
    причем коэффициент пропорциональности считайте зависящим только от свойств среды и геометрических свойств 
    тела. Суммарный объем тел при взаимодействии не изменяется. Ускорение свободного падения $g=9,8$~м/c$^2$.   
\end{enumerate}
