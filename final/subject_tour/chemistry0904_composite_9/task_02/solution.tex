\solutionSection
\begin{enumerate}
    \item Предположим, соединение \textbf{А} содержит 1 атом кислорода. Тогда:\\
    53.3 – 16 (атомная масса кислорода)\\
    46.7 – х\\
    х = 14.\\
    Подходят два соединения:\begin{enumerate}
    \item[1)]Оксид азота (II) – $NO$, однако он не является высшим оксидом для азота, т.к. азот находится в V группе.
    \item[2)]Оксид лития – $Li_2O$ также не удовлетворяет условиям задачи, так как способен взаимодействовать с водой.\\
    Если соединение \textbf{A} содержит два атома кислорода, пропорция выглядит следующим образом: \\
    53.3 – 32 (атомная масса кислорода)\\
    46.7 – х \\
    х = 28 = $Ar(Si)$\\
    Оксид кремния (IV) – $SiO_2$ – полностью удовлетворяет условиям задачи. Следовательно, элемент \textbf{X} – кремний ($Si$), соединение \textbf{А} – $SiO_2$.
    \end{enumerate}
    \item Уравнение реакции:\\
    $4NaOH_\text{(расплав)} + SiO_2 = Na_4SiO_4 + 2H_2O$\\
    или \\
    $2NaOH + SiO_2 = Na_2SiO_3 + H_2O$\\
    Клей на основе силикатов калия или натрия – силикатный клей.
    \item Пусть соединение \textbf{B} содержит один атом кислорода:\\
    47\% – 16\\
    53\% – х\\
    х = 18, не подходит ни одному элементу ПС\\
    Если соединение \textbf{В} содержит 2 атома кислорода:\\
    47\% – 32\\
    53\% – х\\
    х = 36 – не подходит ни одному элементу ПС, наиболее близкая атомная масса у хлора, однако, оксид хлора $ClO_2$ легко взаимодействует с водой.\\
    Если соединение \textbf{В} содержит 3 атома кислорода:\\
    47\% – 48\\
    53\% – х\\
    х = 54 – соответствует 2-м атомам алюминия. Оксид алюминия (III) – $Al_2O_3$ – полностью удовлетворяет условиям задачи, следовательно, соединение \textbf{B} – $Al_2O_3$, а элемент \textbf{Y} – $Al$.
    \item $Al_2O_3 + 2NaOH + 3H_2O = 2Na[Al(OH)_4]$ (\textbf{реакция 2})\\
    $Al_2O_3 + 6HCl = 2AlCl_3 + 3H_2O$ (\textbf{реакция 3})\\
    Оксиды, способные взаимодействовать и с кислотами, и с щелочами называют амфотерными.
\end{enumerate}