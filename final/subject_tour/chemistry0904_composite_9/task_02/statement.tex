\assignementTitle{Задача 2 «Цеолиты»}{14}{2}
Цеолиты – пористые структурированные соединения, каркас которых образован оксидами элементов \textbf{X} и \textbf{Y}. Цеолиты давно стали излюбленными материалами для создания водных фильтров и композитных материалов (катализаторов и ионообменников), что связано с их высокой гидролитической и термической стабильностью. Химическая (гидролитическая) стабильность обусловлена химией элементов \textbf{X} и \textbf{Y} – оксиды этих элементов не взаимодействуют с водой. Более того, высший оксид (соединение \textbf{A}) элемента \textbf{X} не взаимодействует даже с растворами щелочей и вступает в реакцию только с одной кислотой. Массовая доля кислорода в соединении \textbf{A} составляет 53.3\%.
\begin{enumerate}
\item[1.] Определите элемент \textbf{X} и напишите формулу и название соединения \textbf{A}. Ответ подтвердите расчётом.
\end{enumerate}
Соединение \textbf{А} реагирует с расплавами щелочей, давая в результате реакции соль - основу советского канцелярского клея.
\begin{enumerate}
\item[2.] Напишите уравнение реакции соединения A с расплавом NaOH (\textbf{реакция 1}). Как называется клей, производимый на основе солей элемента \textbf{Х}? 
\end{enumerate}
Элемент \textbf{Y} – третий по распространенности в земной коре. Оксид элемента \textbf{Y} (соединение \textbf{B}) образует многие драгоценные камни (рубин, сапфир) и содержит 47\% кислорода по массе
\begin{enumerate}
\item[3.] Уставите элемент \textbf{Y} и напишите формулу и название соединения \textbf{B}. Ответ подтвердите расчётом.
\end{enumerate}
Соединение \textbf{В} способно реагировать с растворами кислот и щелочей.
\begin{enumerate}
\item[4.] Напишите уравнения реакций взаимодействия \textbf{B} c раствором NaOH (\textbf{реакция 2}), а также с раствором соляной кислоты (\textbf{реакция 3}). Как называют класс оксидов, способных взаимодействовать и с кислотами, и с щелочами?
\end{enumerate}

