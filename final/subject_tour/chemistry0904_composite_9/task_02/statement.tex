\assignementTitle{Среда – не только день недели}{11}{2}

У ученых естественнонаучных специальностей слово «среда» ассоциируется не только с третьим днём недели, 
но и с показателем кислотности среды в растворе. Самой известной шкалой, используемой для выражения 
кислотности среды, является шкала, выражаемая в единицах pH. Диапазон шкалы pH – от 0 до 14.

\begin{enumerate}
    \item Напишите формулу для расчёта значения pH в растворе. Рассчитайте pH в водных растворах:
    \begin{enumerate}
        \item[a)] 0,3 М $HCl$;
        \item[б)] 0,4 М $CH_3COOH$.
    \end{enumerate}
        
    Для расчётов: соляная кислота диссоциирует полностью, а константа диссоциации уксусной кислоты $Ka = 1,8 \cdot 10^{-5}$
\end{enumerate}

Работа с клетками в лаборатории ведётся в определённых условиях. В том числе, учёные строго следят за значением 
pH. Растворы, способные поддерживать постоянную, заданную концентрацию ионов водорода при многократном разбавлении 
и добавлении небольших количеств кислоты или щелочи, называются буферными (или просто буферами). Молекулярные 
биологи обычно работают в диапазоне pH от 5,5 до 7, поэтому предпочитают фосфатные буферы. Такие буферы готовятся 
из гидрофосфатов и дигидрофосфатов щелочных металлов.

pH в буферном растворе можно рассчитать по формуле:

$pH = pKa + lg \frac{C_0 (A^-)}{C_0 (HA)}$, где $pKa = -lgKa$ – логарифмическая функция от константы диссоциации кислоты $HA$, 
$C_0A^-$ – заданная концентрация аниона, $C_0HA$ – заданная концентрация кислоты.

\begin{enumerate}
    \item[2.] Напишите уравнение реакции диссоциации $H_3PO_4$ по первой и второй ступени.
    \item[3.] Рассчитайте pH буферного раствора, содержащего 0,5 М $KH_2PO_4$ и 1 М $K_2HPO_4$.
    
    Для расчетов: $pKa_2 (H_3PO_4) = 7,2$, считать, что соли диссоциируют нацело.

    \item[4.] Можете ли вы и ваши коллеги использовать раствор из пункта 3 для исследований в области молекулярной биологии? Аргументируйте свой ответ, основываясь на проведенных в пункте 3 расчетах.
\end{enumerate}

