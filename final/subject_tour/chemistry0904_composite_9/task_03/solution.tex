\solutionSection
\begin{enumerate}
    \item Сульфат меди – $CuSO_4$. Безводный сульфат меди – белый порошок.
    \item $5\text{ л} = 5000\text{ мл} = 5000\text{ см}^3 => m_\text{раствора} = 5000\text{ см}^3\cdot1.03\text{ г/см}^3 = 5150$ г\\
    $m(CuSO_4) = m_\text{раствора}\cdot\omega(CuSO_4) = 5150\cdot0.03 = 154.5$ г
    \item Медный купорос – $CuSO_4\cdot5H_2O$. Соединения такого типа называют кристаллогидратами.
    \item Типы связей:
    \begin{itemize}
    \item ковалентная (между серой и кислородом);
    \item ионная (между катионом и анионом), 
    \item водородная (между сульфатом и координированной или внешнесферной водой)
    \end{itemize}
    \item Масса чистого $CuSO_4$, необходимая для приготовления такого раствора была рассчитана в пункте 2 и составляет 154.5 г.\\
    Рассчитаем массовую долю $CuSO_4$ в медном купоросе:\\
    $$\omega(CuSO_4) = \frac{m(CuSO_4)}{m(CuSO_4\cdot5H_2O)} = \frac{160}{250} = 0.64$$
    Следовательно, сульфат меди в медном купоросе составляет 64\%, остальное – вода.\\
    Для того, чтобы в растворе оказалось нужное количество сульфата нужно взять:\\
    64\% – 154.5 г\\
    100\% – х г\\
    Х = 241.4 г медного купороса $CuSO_4\cdot5H_2O$.\\
    Добавкой кристаллизационной воды можно пренебречь, поскольку её доля от общего числа воды в растворе составит меньше двух процентов. Проверим расчёт:\\
    $$\omega(CuSO_4) = \frac{m(CuSO_4)}{m_\text{раствора}} = \frac{154.5}{5241.4} = 0.0295 \thicksim 0.03 = 3\%$$
    \item Нет, нельзя, поскольку цинк вытесняет медь из растворов её солей.\\
    Уравнение реакции:\\
    $CuSO_4 + Zn = ZnSO_4 + Cu\downarrow$
    
    
\end{enumerate}