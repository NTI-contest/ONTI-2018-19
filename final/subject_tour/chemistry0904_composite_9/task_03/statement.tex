\assignementTitle{Задача 3 «Садовые расчеты»}{13}{3}

Возможно, ваши родственники-садоводы опрыскивали растения голубым раствором с целью избавиться от вредителей. В садоводстве, в зависимости от сезона, применяются 1 – 3\% растворы сульфата меди, который и даёт голубое окрашивание.
\begin{enumerate}
\item[1.] Напишите формулу сульфата меди. Какой цвет имеет безводный сульфат меди? 
\item[2.] Рассчитайте массу безводного сульфата меди, необходимую для приготовления 5 литров 3\% раствора. Плотность конечного раствора считать равной 1.03 г/см$^3$.
\end{enumerate}
Безводный сульфат меди – очень гигроскопичное вещество. В магазине садовод может купить только медный купорос – пятиводный сульфат меди.
\begin{enumerate}
\item[3.] Напишите формулу медного купороса. Как называются соединения такого типа?
\item[4.] Какие типы химических связей присутствуют в твердом медном купоросе?
\item[5.] Рассчитайте массу медного купороса, которую необходимо взять садовнику для приготовления 5 литров 3\% раствора (такого же, как в пункте 2). 
\item[6.] Как вы думаете, можно ли готовить раствор медного купороса в оцинкованном ведре? Аргументируйте свой ответ. Если вы считаете, что в системе могут проходить химические реакции, обязательно напишите их.
\end{enumerate}