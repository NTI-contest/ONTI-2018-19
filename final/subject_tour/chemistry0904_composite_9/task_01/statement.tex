\assignementTitle{Семь бед – один ответ}{12}{1}

Желание получить протез, полностью подчиняющийся сознанию человека, давно преследует реабилитационные центры. Однако, конструкторы протезов пока разводят руками – трудности есть и в информационных технологиях, и в подходящих материалах. \\
Специалисты считают, что основой композитного материала, который можно использовать для создания высокотехнологичных протезов, могла бы стать аллотропная модификация (вещество \textbf{A}) элемента \textbf{X}. Вещество \textbf{A} обладает развитой структурой, по форме напоминающей полый цилиндр, не токсично для человека и способно проводить ток.\\
Элемент \textbf{Х} образует два оксида \textbf{B} и \textbf{С} с массовой долей кислорода 57.1\% и 72.7\% соответственно
\begin{enumerate}
\item[1.] Определите элемент \textbf{X}, а также напишите формулы оксидов \textbf{B} и \textbf{C}. Ответ подтвердите расчетом.
\end{enumerate}
Элемент \textbf{Х} – рекордсмен по количеству аллотропных модификаций. Сейчас научному сообществу известно больше десяти простых веществ, образованных элементом \textbf{Х}. 
\begin{enumerate}
\item[2.] Как называется аллотропная модификация X, образующая вещество \textbf{А}? Напишите названия ещё трёх аллотропных модификаций X, известных вам.
\end{enumerate}
Природный газ преимущественно состоит из бинарного вещества \textbf{D}, образованного элементом \textbf{Х} и водородом.
\begin{enumerate}
\item[3.] Установите формулу вещества \textbf{D}, если известно, что его плотность по воздуху составляет 0.55. Какая гибридизация у атома элемента \textbf{X} в веществе \textbf{D}?
\end{enumerate}
На реакции сгорания природного газа сейчас построено множество бытовых процессов – от работы двигателей автобусов до функционирования целых электростанций. 
\begin{enumerate}
\item[4.] Напишите уравнение и рассчитайте тепловой эффект реакции полного сгорания вещества \textbf{D}\\
\textit{Теплоты образования веществ: $Q_\text{обр}$ (вещества $D_\text{(газ)}$) = 76 кДж/моль,\\ $Q_\text{обр}(CO_{2(\text{газ})})$ = 393.5кДж/моль, $Q_\text{обр} (H2O_\text{(жидкость)})$ =  286 кДж/моль}
\end{enumerate}

