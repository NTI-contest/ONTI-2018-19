\solutionSection
\begin{enumerate}
\item Предположим, в оксиде с меньшим содержанием кислорода содержится один атом O. Составим пропорцию:\\
$57.1\% – 16$ (атомная масса кислорода)\\
$43\% – x$\\
$x = 12$ – соответствует атомной массе углерода. Следовательно, оксид \textbf{B} – $CO$.

Второй известный оксид углерода – $CO_2$.

Рассчитаем массовую долю кислорода в диоксиде углерода:
$$\omega(O) = \frac{16\cdot2}{(16\cdot2+12)} = 0.727$$
\item Вещество А образует аллотропная модификация углерода – углеродные нанотрубки.\\
Верными считаются любые аллотропные модификации, характерные для углерода.\\
Возможные варианты ответа – алмаз, графит, графен, фуллерен, аморфный углерод, карбин.
\item $M(\text{вещества} D) = 29\cdot0.55 = 16$ г/моль\\
Тогда число атомов водорода в веществе D составит:\\
$n(H) = 16-12/M(H) = 4$\\
Формула вещества \textbf{D}: $CH_4$\\
Гибридизация углерода в веществе $D - sp^3$.
\item Уравнение реакции:\\
$CH_4 + 2O_2 = CO_2 + 2H_2O$\\
Тепловой эффект:\\
$\Delta Q = Q_\text{продуктов} - Q_\text{реагентов} = 2\cdot286 + 393.5 - 76 = 889.5$ кДж.
\end{enumerate}