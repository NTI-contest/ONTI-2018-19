\assignementTitle{Кто ищет, тот всегда найдёт}{10}{4}

Гэмфри Дэви прославился тем, что смог выделить щелочные металлы с помощью действия электрического тока на их соли. Но до своего триумфа, Дэви потерпел и множество неудач: дело в том, что раствор и расплав соли щелочного металла при электролизе ведут себя по-разному.
\begin{enumerate}
\item[1.] Напишите уравнение реакции электролиза расплава (\textbf{реакция 1}) и раствора (\textbf{реакция 2}) хлорида натрия. 
\end{enumerate}
Стоит отметить, что не все катионы такие «капризные», как катионы щелочных металлов. Выделить некоторые металлы можно с помощью электролиза даже из раствора их солей.
\begin{enumerate}
\item[2.] Соли каких металлов могут образовывать на катоде одинаковые продукты при электролизе расплавов и растворов? 
\item[3.] Приведите один пример соли, при электролизе раствора и расплава которой образуются одинаковые продукты и на катоде, и на аноде. Напишите уравнение реакции электролиза этой соли (\textbf{реакция 3}).
\end{enumerate}
Один из промышленных способов получения водорода построен на реакции электролиза растворов гидроксидов щелочных металлов. Этот изящный метод был известен ещё со времен Дэви и всё ещё остаётся популярным среди тех, кому необходим особо чистый водород.
\begin{enumerate}
\item[4.] Напишите уравнение реакции электролиза раствора гидроксида калия (\textbf{реакция 4}). Какова роль гидроксида калия в этом процессе?
\item[5.] Приведите два примера получения водорода в лаборатории.
\item[6.] Как хранят щелочные металлы в лаборатории? Почему выбраны такие условия хранения?
\end{enumerate}