\solutionSection
\begin{enumerate}
    \item Электролиз расплава (\textbf{реакция 1}): $2NaCl = 2Na_\text{(мет)} + Cl_2\downarrow$\\
    Электролиз раствора (\textbf{реакция 2}): $2NaCl + 2H_2O = 2NaOH + H_2\downarrow + Cl_2\downarrow$
    \item Соли металлов, стоящих в ряду напряжения после водорода, при электролизе в растворах и  расплавах выделяют на катоде металл.
    \item Примером соли, при электролизе расплава и раствора которой образуются одинаковые продукты, может служить $CuCl_2$.\\
    Уравнение электролиза: $CuCl_2 = Cu + Cl_2\downarrow$\\
    Верным считается любой пример соли, соответствующий условию – одинаковые продукты при электролизе расплава и раствора.
    \item Электролиз раствора (\textbf{реакция 4}): $\textit{\textcolor[rgb]{0.5,0.5,0.5}{\sout{KOH}}} + 2H_2O = 2H_2 + O_2 +\cancel{\textcolor[rgb]{0.5,0.5,0.5}{KOH}}$\\
    Гидроксид калия не участвует в реакции, а увеличивает электропроводность раствора, не давая побочных продуктов.
    \item Примеры:\\
    $Zn + 2HCl = ZnCl_2 + H_2 \uparrow$\\
    $2Al +2NaOH + 6H_2O= 2Na[Al(OH)_4] + 3H_2\uparrow$\\
    $Сa + 2H_2O = Ca(OH)_2 + H_2\uparrow$
    \item Щелочные металлы хранят под слоем керосина или масла, чтобы избежать окисления поверхности кислородом воздуха или взаимодействия с парами воды.
\end{enumerate}