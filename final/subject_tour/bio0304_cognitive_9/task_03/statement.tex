\assignementTitle{}{10}{}

Ретикулярная формация — это структура ствола головного мозга, участвующая в поддержании и смене таких физиологических состояний, как бодрствование, парадоксальная фаза сна («быстрый сон») и ортодоксальная фаза сна («медленный сон»), а также играющая важную роль в физиологии внимания. Внимание можно наблюдать в состоянии бодрствования, а также (с некоторыми допущениями) во время быстрой фазы сна. 

\putImgWOCaption{15cm}{1}

Пояснения к иллюстрации: 5HT – серотонин, ACh – ацетилхолин, LC – голубое пятно, NA – норадреналин, PPN - педункулопонтийное тегментальное ядро, RN – ядро шва.

\begin{table}[H]
    \begin{center}
        \begin{tabular}{|p{2cm}|p{2cm}|p{2cm}|p{2cm}|}
            \hline
            Нейромедиатор & Бодрствование & “Быстрый” сон & “Медленный” сон \\
            \hline
            Ацетилхолин & ++ & + & -- \\
            \hline
            Норадреналин & + & -- & -- \\
            \hline
            Серотонин & + & -- & ++ \\
            \hline
        \end{tabular}
    \end{center}
\end{table}

Рассмотрите иллюстрацию и таблицу. Выберите все верные утверждения:
\begin{enumerate}
    \item[А.] Серотонин в данной системе выступает в роли возбуждающего нейромедиатора
    \item[Б.] Нонсенс-мутация (приводящая к возникновению стоп-кодона)  в гене норадреналинового рецептора, расположенного на мембране клеток педункулопонтийного тегментального ядра, может привести к бессоннице
    \item[В.] Применение слабого ингибитора ацетилхолинэстеразы (фермента, разрушающего ацетилхолин в синаптической щели) может привести к повышению концентрации внимания
    \item[Г.] При пробуждении электрическая активность голубоватого ядра увеличивается сильнее, чем электрическая активность педункулопонтийного тегментального ядра
    \item[Д.] Активность данных структур поддаётся прямому сознательному контролю
\end{enumerate} 

Перечислите \textbf{русские} буквы, соответствующие \textbf{верным} уверждениям, \textbf{в алфавитном порядке без разделителей}.

\explanationSection

Наибольшее количество серотонина соответствует минимуму прочих нейромедиаторов, следовательно, серотонин в данной системе выступает в качестве тормозного нейромедиатора. Максимальный уровень норадреналина согласно схеме приходится на бодрствование, значит, при отсутствии рецепторов к норадреналину будет наблюдаться не бессонница, а напротив, отсутствие способности поддерживать бодрствование. Согласно приведённой схеме, уровень внимания коррелирует с активностью ацетилхолиновых рецепторов. Значит, при слабом ингибировании ацетилхолинэстеразы, равносильному для клетки-акцептора повышению активности холинэргических клеток, уровень внимания может возрасти. Однако следует помнить, что при избыточном ингибировании ацетилхолинэстеразы могут развиться крайне негативные эффекты, такие как паралич мускулатуры или судороги. Пункт Г напрямую следует из таблицы. Указанные структуры располагаются в стволе мозга, следовательно, прямому сознательному контролю не подвергаются.

\answerMath{ВГ.}