\assignementTitle{}{12}{}

Сопоставьте названия и описание методов исследования анатомии и активности головного мозга (нейровизуализация) с данными, которые получаются при таких исследованиях.

Список методов:
\begin{enumerate}
    \item[А.] Магнитно-резонансная томография (МРТ). Метод основан на разном распределении молекул воды в тканях человека. С помощью ядерного магнитного резонанса обнаруживаются ядра атомов водорода, входящие в состав молекул воды. Послойное сканирование мозга позволяет изучать анатомию в трех измерениях.
    \item[Б.] Электроэнцефалография (ЭЭГ). В данном методе с кожи головы регистрируются токи, отражающие суммарную электрическую активность мозга.
    \item[В.] Функциональная магнитно-резонансная томография (фМРТ). Метод является модификацией метода МРТ. Проводится запись МРТ-томограммы во времени. Активность разных отделов мозга отражается на изменении кровотока через эти отделы и на уровне насыщения гемоглобина кислородом. Оксигенированный и деоксигенированный гемоглобин по-разному реагирует на магнитное поле, изменяя МРТ сигнал.
    \item[Г.] Диффузионная спектральная томография. Метод позволяет регистрировать направление диффузии молекул воды в мозге. Поскольку диффузия воды в нейроне ограничена клеточной мембраной, а также гидрофобной миелиновой оболочкой, данный метод позволяет визуализировать крупные нервные волокна в мозге.
    \item[Д.] Диффузная оптическая томография (ДОТ). В данном методе регистрируется поглощение ближнего инфракрасного света гемоглобином в сосудах коры больших полушарий. Оксигенированный и деоксигенированный гемоглобин по-разному поглощает инфракрасный свет. Это позволяет исследовать функциональную активность коры. Однако из-за небольшой проникающей способности инфракрасного света сквозь ткани черепа и мозга (примерно 1 см), данный метод, в отличие от фМРТ, не позволяет исследовать более глубокие структуры мозга.
    \item[Е.] Позитронно-эмиссионная томография (ПЭТ). В кровь пациента вводится вещество, которое содержит радиоактивную метку (например, фтордезоксиглюкоза). В зависимости от функционального состояния, клетки мозга по разному поглощают это вещество. Специальный томограф позволяет локализовать радиоактивный сигнал в тканях мозга. В отличие от фМРТ данный метод не позволяет напрямую изучать анатомию мозга - только разную функциональную активность его зон.
\end{enumerate}

Данные:
\begin{enumerate}
    \item
    \putImgWOCaption{9cm}{1}
    \item
    \putImgWOCaption{9cm}{2}
    \item
    \putImgWOCaption{9cm}{3}
    \item
    \putImgWOCaption{9cm}{4}
    \item
    \putImgWOCaption{9cm}{5}
    \item
    \putImgWOCaption{9cm}{6}
\end{enumerate}

Укажите последовательность из \textbf{6 русских букв без разделителей} (методы), соответствующие каждому из предложенных изображений, указанных в порядке нумерации. Например, АБВГДЕ.

\answerMath{БВДЕАГ.}