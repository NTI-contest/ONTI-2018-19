\assignementTitle{}{10}{}

\putImgWOCaption{15cm}{1}

Вам представлена форма одного из нейромедиаторов человека. Выберите верные ответы:
\begin{enumerate}
    \item[А.] Вещество №1 используется для синтеза нейромедиатора
    \item[Б.] Вещество №2 используется для синтеза нейромедиатора
    \item[В.] Нейромедиатор синтезируется путем прямой химической модификации аминокислот
    \item[Г.] Данный нейромедиатор можно найти как в центральной (головной и спинной мозг), так и периферической нервной системе
    \item[Д.] Нейромедиатор инактивируется с помощью эстераз — ферментов, катализирующих гидролитическое расщепление сложных эфиров (X-COO-X)
\end{enumerate} 

Перечислите \textbf{русские} буквы, соответствующие \textbf{верным} уверждениям, \textbf{в алфавитном порядке без разделителей}.

\solutionSection

Ацетилхолин синтезируется из ацетил-КоА и холина. Значит ответы А и Б верные, А - В не верный

Ацетилхолин является медиатором как в центральной, так и в периферической системах. Ответ Г верный

В ацетилхолине есть сложноэфирная связь, которая разрушается ацетилхолинэстеразой в синаптической щели. Ответ Д верный.


\answerMath{АГД.}