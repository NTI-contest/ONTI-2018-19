\assignementTitle{}{10}{}

Внимание — это избирательная направленность восприятия на тот или иной объект. Для маркетологов особенно важно изучение непроизвольного внимания, поскольку понимание его механизмов позволяет эффективнее воздействовать на потребителей. В основе непроизвольного внимания лежит ряд базовых принципов, основанных на особенностях раздражителей и субъекта воздействия. Совместите данные особенности с конкретными примерами (все примеры вымышленные, любое совпадение является случайным).

Особенности:
\begin{enumerate}
    \item Контраст между схожими раздражителями
    \item Абсолютная сила раздражителя
    \item Относительная сила раздражителя
    \item Новизна раздражителя
    \item Изменения в раздражителе
    \item Сознательный интерес субъекта к объекту внимания
    \item Физиологическое состояние субъекта
\end{enumerate}

Примеры:
\begin{enumerate}
    \item[А.] жевательная резинка со вкусом моркови
    \item[Б.] Вывеска в виде замысловатого механизма с множеством движущихся деталей
    \item[В.] Название фирмы с латинской буквой другого цвета в русском слове
    \item[Г.] Размещение афиш концертов классической музыки вблизи консерватории
    \item[Д.] Громкая звуковая реклама
    \item[Е.] Размещение кафе вблизи крупного музея, есть и пить в котором запрещено
    \item[Ж.] Иллюминация, переливающаяся всеми цветами радуги
    \item[З.] Тусклая неоновая вывеска в ночное время суток
\end{enumerate}

\begin{tabular}{|p{1.5cm}|p{1.5cm}|p{1.5cm}|p{1.5cm}|p{1.5cm}|p{1.5cm}|p{1.5cm}|p{1.5cm}|}
    \hline
    А&Б&В&Г&Д&Е&Ж&З \\
    \hline
    &&&&&&&\\
    \hline
\end{tabular}

Перечислите \textbf{8 цифр из таблицы} (особенности), соответствующих примерам. Например, 75318642.

\answerMath{45162753.}