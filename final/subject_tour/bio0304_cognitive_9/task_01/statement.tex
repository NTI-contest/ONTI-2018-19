\assignementTitle{}{8}{}

Электроэнцефалограмма – один из методов исследования активности головного мозга, заключающийся в регистрации с помощью электродов электрических сигналов, т.н. ритмов головного мозга.

Ритмы головного мозга – диагностируемые электрические колебания мозга – меняются в зависимости от статуса нервной системы человека.

Существует общее правило: чем выше уровень мозговой активности, чем выше частота генерирующихся в мозге ритмов.

Соотнесите состояния нервной системы и присущие им ритмы:

Ритмы:

\begin{enumerate}
    \item 
    \putImgWOCaption{10cm}{1}
    \item 
    \putImgWOCaption{10cm}{2}
    \item 
    \putImgWOCaption{10cm}{3}
    \item 
    \putImgWOCaption{10cm}{4}
\end{enumerate}

Состояния нервной системы:
\begin{enumerate}
    \item[А.] Бодрствование с закрытыми глазами
    \item[Б.] Бодрствование с открытыми глазами
    \item[В.] Глубокий сон
    \item[Г.] Засыпание
\end{enumerate} 

Перечислите 4 \textbf{русские} буквы, соответствующие изображениям с 1 по 4 \textbf{без разделителей}. Например, ГВБА.

\answerMath{ВАБГ.}