\assignementTitle{}{7}{}

Клетки мозга, передающие нервный импульс, характеризуются:
\begin{enumerate}
    \item[А.] Отсутствием ядра
    \item[Б.] Присутствием митохондрий
    \item[В.] Наличием шероховатого ЭПР
    \item[Г.] Наличием пластид
    \item[Д.] Наличием жгутиков
    \item[Е.] Наличием клеточной стенки
    \item[Ж.] Наличием центриолей
\end{enumerate} 

Перечислите \textbf{русские буквы}, соответствующие \textbf{верным} уверждениям, \textbf{в алфавитном порядке без разделителей}.

\explanationSection

Нейроны содержат ядро, митохондрии, шероховатый ЭПР, центриоли. Жгутиков у них нет, а тем более пластид и клеточной стенки.

\answerMath{БВЖ.}