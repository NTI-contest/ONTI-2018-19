\assignementTitle{}{8}{}

Сопоставьте симптомы и повреждения различных отделов мозга

\begin{enumerate}
    \item Мозжечок
    \item Лобная доля коры
    \item Гиппокамп
    \item Затылочная доля полушарий
\end{enumerate}

\begin{enumerate}
    \item[a)] Больной ведет себя вполне адекватно, его кратковременная память работает, но через несколько дней не может ничего вспомнить из того, что делал.
    \item[б)] Происходит нарушение узнавания предметов при сохранении зрительного восприятия: пациент не узнаёт знакомых предметов, не знает их значения, но может отобрать предметы, аналогичные предъявленному.
    \item[в)] Мышечная гипотония, нарушения координации, крупный тонический спонтанный нистагм (непроизвольные частые колебательные движения глаз)
    \item[г)] У больного нарушается речь. Больной хватает окружающие его предметы и начинает ими оперировать, даже если для этого нет никаких мотивов.
\end{enumerate}

\answerMath{1 - в, 2 - г, 3 - а, 4 - б.}