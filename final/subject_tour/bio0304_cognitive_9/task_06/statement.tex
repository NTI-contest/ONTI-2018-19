\assignementTitle{}{21}{}

Известно, что кошки мурлычут в разных ситуациях. Вы предполагаете, что они мурлычут, чтобы выпросить у хозяев еду. Другая гипотеза — они мурлычут всегда, когда голодны.

Вам нужно поставить эксперимент, позволяющий выяснить, какая из гипотез более верна. У вас и ваших друзей есть 10 кошек, за каждой из которых можно установить наблюдение. Допустим, вы уже знаете, в какое время суток и в каких условиях каждая из этих кошек бывает голодна, а когда — нет (т.е. уверены в каждый момент эксперимента, кошка голодна или сыта). Также допустим, что кошки не мурлычут по другим поводам. Опишите, как вы будете проводить свой эксперимент. Обоснуйте, зачем вы делаете то или иное измерение. Продумайте наперед 3 возможных исхода эксперимента и напишите их. Как бы вы интерпретировали результаты каждого из этих трех исходов?

\solutionSection

Во-первых, необходимо придумать способ отследить мурлыканье кошки вне зависимости от того, понимает ли она, что рядом человек (3 балла). Можно поставить диктофон в комнате, где она находится, или использовать микрофон (а прослушивать его в другой комнате) (2 балла за адекватную идею). 

Нужно зафиксировать, сколько раз (или сколько минут) кошка мурлыкала за определенный отрезок времени, когда рядом не было никого, и сколько — когда рядом был хозяин (2 балла).

Также нужно зафиксировать базовый уровень мурчания (3 балла): когда кошка не голодна (в двух условиях: рядом с хозяином и когда никого нет). Это можно также назвать  отрицательным контролем. Этот результат нужно отнимать от того, что получится в основном эксперименте, соответственно (когда кошка голодна)

Аналогичные эксперименты нужно повторить для разных кошек в одинаковых условиях. 

Возможные исходы (после вычитания базового уровня): одна кошка мурлычет одинаково, 7 кошек — когда рядом хозяин, 2 кошки — чаще, когда рядом нет хозяина. 

2 кошки мурлычут чаще, когда рядом хозяин, 7 кошек — не зависимо от присутствия хозяина, 1 кошка мурлычет чаще, когда хозяина нет. Такой результат говорит, что мурлыканье не зависит от чувства голода

И так далее. По 1 баллу за пример и 3 балла за верную интерпретацию (12 всего за подзадачу). 

Другим вариантом верного решения будет собрать кошек в одну комнату и проверять, как они ведут себя в присутствии хозяина или в присутствии постороннего человека. 