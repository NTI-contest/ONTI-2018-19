\assignementTitle{Задача 3. «История вещества Х»}{14}{3}
В 1787 году известный французский химик Клод Луи Бертолле проводил эксперименты по растворению ядовитого желто-зеленого газа \textbf{А} с резким запахом \textbf{в холодной воде} [реакция 1] и обнаружил, что получившийся раствор обладает отбеливающими свойствами благодаря образующемуся в ходе реакции кислородсодержащему веществу \textbf{В} (также в ходе реакции образуется вещество \textbf{С}, не обладающее отбеливающей активностью). Позднее по этой технологии небольшая французская компания «Societé Javel» наладила производство «отбеливающего раствора». Этот раствор обладал высокой активностью, но имел важный недостаток: вещество \textbf{В} - один из главных компонентов раствора - нестабильно. Позднее этот процесс был модифицирован А. Лабарраком, и желто-зеленый газ \textbf{А} стали пропускать не через воду, а через раствор вещества \textbf{D}, называемого также «каустической содой» [реакция 2]. Этот раствор был назван по имени французского фармацевта «лабарраковой водой» и получил широкое распространение как дезинфектор и отбеливающий агент, а вещество \textbf{Х}, являющееся солью вещества \textbf{В}, и на сегодняшний день является наиболее распространенным и широко используемым в мире бытовым дезинфицирующим и отбеливающим средством.\\
Дополнительно известно, что вещество \textbf{D} содержит следующие химические элементы: Na – 57.5 масс. \%, H – 2.5 масс. \%, О – 40.0 масс. \%.
\begin{enumerate}
\item Определите вещества \textbf{A-D}, а также вещество \textbf{Х}, назовите вещества \textbf{A-D}, \textbf{Х}.
\item Напишите уравнения упомянутых реакций [1-2].
\item Как было отмечено в условии задачи, отбеливающей активностью обладает вещество \textbf{В}, однако в продажу поступает его соль – вещество \textbf{Х} (вследствие нестабильности водных растворов вещества \textbf{В}). Объясните, почему \textbf{водные растворы} вещества \textbf{Х} также обладают отбеливающей активностью.
\end{enumerate}