\solutionSection
\begin{enumerate}
\item По описанию ясно, что ядовитый желто-зеленый газ с резким запахом – это \textbf{хлор} (\textbf{A} - \textbf{Cl$_2$}). При пропускании газообразного хлора через холодную воду образуется смесь соляной (HCl) и хлорноватистой кислот (HClO). Поскольку в условии отмечено, что вещество \textbf{В} содержит кислород, то можно сделать вывод, что вещество \textbf{В} – это \textbf{HClO} (\textbf{хлорноватистая кислота}), а вещество \textbf{С} – \textbf{HCl} (\textbf{соляная кислота}).\\
Массовые доли позволяют определить состав вещества \textbf{D} («каустической соды»). Пусть имеется 100 г. вещества \textbf{D}, тогда натрия в его составе 57.5 г., водорода – 2.5 г и кислорода – 40.0 г. Вычислим количества вещества каждого из элементов в соединении: n(Na) = 57.5 г : 23 г/моль = 2.5 моль; n(H) = 2.5 г : 1 г/моль = 2.5 моль; n(O) = 40 г : 16 г/моль = 2.5 моль. n(Na) : n(H) : n(O) = 1 : 1 : 1. Таким образом, соединение \textbf{D} – \textbf{NaOH} (\textbf{гидроксид натрия}).\\
Соединение \textbf{Х} образуется при пропускании хлора через раствор гидроксида натрия. Кроме того, известно, что оно является солью хлорноватистой кислоты. Отсюда ясно, что \textbf{Х} – \textbf{NaClO} – \textbf{гипохлорит натрия}.
\item Реакция 1: $Cl_2 + H_2O = HCl + HClO$.\\
Реакция 2: $Cl_2 + 2NaOH = NaCl + NaClO + H_2O$.
\item Гипохлорит натрия - $NaClO$ – соль, образованная катионом сильного основания и анионом слабой кислоты, поэтому в водных растворах происходит гидролиз по аниону по уравнению: $ClO^- + H_2O = HClO + OH^-$. Наличие в растворах гипохлорита натрия хлорноватистой кислоты и обуславливает их отбеливающие свойства.


\end{enumerate}
