\assignementTitle{ Небо было желтым, как латунь}{12}{4}
Латунь - многокомпонентный сплав, состоящий преимущественно из металлов \textbf{Х} и \textbf{Y}. Помимо металла \textbf{Y}, в качестве легирующих добавок также могут входить олово, свинец, марганец, никель, железо. Латунь находит широкое применение как материал для изготовления деталей машин, морских судов и самолетов.\\
Навеску латуни массой 7.0 г. (состав: 60 \% металла \textbf{Х} и 40 \% металла \textbf{Y}) растворили в избытке соляной кислоты, при этом выделилось 0.959 л газа (при н.у.). Нерастворившуюся часть навески отфильтровали и взвесили. Ее масса составила 4.2 г.\\
Еще одну навеску латуни (7.0 г.) растворили в избытке концентрированной азотной кислоты, при этом образовался раствор голубого цвета. При добавлении избытка раствора гидроксида натрия выпал осадок голубого цвета, а раствор обесцветился. Осадок отфильтровали, прокалили и взвесили. Масса черного осадка после прокаливания составила 5.26 г.
\begin{enumerate}
\item Установите металлы \textbf{Х} и \textbf{Y}. Ответ подтвердите расчетами.\\
\textit{Получаемые значения количества вещества при расчетах используйте с точностью до тысячных.}
\item Запишите уравнения реакций (6 шт.), проведенных для установления качественного состава сплава.
\end{enumerate}