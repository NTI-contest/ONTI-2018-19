\solutionSection
\begin{enumerate}
    \item При растворении навески латуни в избытке соляной кислоты произошло растворение только 2.8 г. металла, что составляет 40 \% от исходной массы навески. Из этого можно сделать предположение, что металл \textbf{Х} в соляной кислоте не растворяется, а реакция идет только с металлом \textbf{Y}. Иначе говоря, металл \textbf{Х} стоит в ряду напряжений металлов после водорода, а металл \textbf{Y} - до. Запишем в общем виде реакцию растворения металла \textbf{Y}  в соляной кислоте:
    $$Y + kHCl = YCl_k + (k/2)H_2$$
    Найдем количество вещества выделившегося водорода: $n(H_2) = 0.959$ л $: 22.4$ л/моль$ = 0.043$ моль. Количество вещества металла Y, вступившего в реакцию, равно $n(Y) = 2.8 г : M(Y)$ г/моль, а отношение $n(H2) :  n(Y) = k/2$. Тогда $k/2 = (0.043\cdot M(Y))/2.8. M(Y) = 32,56\cdot k$. При $k = 1$ молярная масса близка к массе серы, но этот вариант не подходит хотя бы потому, что \textbf{Y} - металл. При $k = 2, M(Y) = 65$ г/моль, \textbf{Y - Zn}.\\
    В концентрированной азотной кислоте растворятся оба металла. Судя по образованию голубого раствора, голубого осадка гидроксида при добавлении щелочи и черного осадка при прокаливании, можно предположить, что металл \textbf{Х} - это \textbf{Cu}.\\
    Этот же ответ можно получить и путем вычислений: при растворении металла \textbf{Х} в концентрированной азотной кислоте образуется нитрат металла \textbf{Х}, при добавлении гидроксида натрия - гидроксид металла \textbf{Х}, а при прокаливании - оксид металла \textbf{Х}. Запишем формулу оксида в виде XOt и аналогично предыдущему случаю найдем соотношение между t и M(X). Количество вещества Х, вступившего в реакцию:
    $$n(X) = (0.6\cdot7) : M(X) = 4.2 : M(X).$$
    $$n(XO_t) = 5.26 : (M(X) + 16\cdot t).$$
    $$n(X) = n(XO_t).$$
    $$4.2 : M(X) = 5.26 : (M(X) + 16\cdot t).$$
    $$M(X) = 63,4\cdot t.$$
    При $t=1$ , \textbf{X - Cu}. Остальные варианты не подходят, поскольку получается, что либо \textbf{Х} - неметалл (что противоречит условию задачи), либо радиоактивный элемент, либо металл, химические свойства которого не согласуются с условием задачи.
    \item $[1]$ $Zn + 2HCl = ZnCl_2 + H_2$\\
    $[2]$ $Cu + 4HNO_{3(\text{к})}  = Cu(NO_3)_2 + 2NO_2 + 2H_2O$\\
    $[3]$ $Zn + 4HNO_{3(\text{к})} = Zn(NO_3)_2 + 2NO_2 + 2H_2O (\text{можно также принять} NO, N_2O)$\\
    $[4]$ $Cu(NO_3)_2 + 2NaOH_\text{(изб.)} = Cu(OH)_2 + 2NaNO_3$\\
    $[5]$ $Zn(NO_3)_2 + 4NaOH_\text{(изб.)} = Na_2[Zn(OH)_4] + 2NaNO_3$\\
    $[6]$ $Cu(OH)_2 = CuO + H_2O$
\end{enumerate}