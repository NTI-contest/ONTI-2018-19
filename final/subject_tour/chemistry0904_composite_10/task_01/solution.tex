\begin{enumerate}
    \item Структурная формула гидразина:
    \begin{center}
    \putImgWOCaption{8cm}{1}
    \end{center}
    Связи N-H – ковалентные полярные, N-N – ковалентная неполярная. Водородная связь образуется между атомом водорода и электроотрицательным элементом (F, O, N, Cl). Таким образом, молекулы гидразина в жидком состоянии могут образовать между собой водородные связи.
    \item Ион гидразиния – $N_2H_5^+$. Связь образуется по донорно-акцепторному механизму.
    \item $N_2H_4 = N_2 + 2H_2$\\
    $O_2 + 2H_2 = 2H_2O$\\
    Зная плотность жидкого гидразина, можно рассчитать массу 292 литров топлива.
    $m(N_2H_4) = \rho(N_2H_4)\cdot V(N_2H_4)$ = 292 л $\cdot$ 1010 г/л = 294 920 г. Найдем количество вещества гидразина $n(N_2H_4) = m(N_2H_4) : M(N_2H_4)$ = 294 920 г : 32 г/моль = 9216.25 моль. По уравнениям реакций $n(H_2O) = 2\cdot n(N_2H_4)$ = 18432.5 моль, тогда $m(H_2O) = n(H_2O)\cdot M(H_2O)$ = 18432.5 моль$\cdot$ 18 г/моль = 331 785 г. Поскольку плотность воды $\sim$ 1 г/мл, то $V(H_2O)$ = 331785 мл $\approx$ 331.8 л. Таким образом, расчеты Марка оказываются весьма оптимистичными.
    \item В п.3 было посчитано, что в 292 л содержится 9216.25 моль гидразина. При сгорании 1 моля гидразина выделится 50 кДж тепла. Таким образом, можно составить пропорцию и найти количество теплоты, которое выделится при сгорании 292 л гидразина.\\
    1 моль – 50 кДж\\
    9216.25 моль – х кДж\\
    х = 460812.5 кДж.\\
    
\end{enumerate}