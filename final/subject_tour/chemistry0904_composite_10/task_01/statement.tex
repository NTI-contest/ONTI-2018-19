\assignementTitle{По следам Марка Уотни}{11}{1}

В 2011 году вышел роман Энди Вейра «Марсианин», а в 2015 году одноименный фильм. Наверняка вы читали книгу, которая стала бестселлером, или смотрели фильм, собравший 630 миллионов долларов. Напомним, что главный герой - Марк Уотни, оказался один на «красной планете» после того, как все остальные сотрудники миссии «Арес-3» вынуждены были покинуть Марс из-за надвигавшейся песчаной бури.

По счастливой случайности автономный жилой модуль остался неповреждённым после песчаной бури, а значит, у героя оставался шанс продержаться ещё 4 года до прилёта следующей миссии «Арес-4». Чтобы обеспечить себе пропитание, Марк Уотни принял решение выращивать картофель, а для этого ему необходима была вода. Благодаря наличию оксигенатора, Марк всегда имел запас кислорода, а водород он получил из гидразина ($N_2H_4$), который в изобилии имелся как запас топлива. 

\begin{enumerate}
    \item[1.] Запишите структурную формулу гидразина и обозначьте тип каждой из химических связей в молекуле (ковалентная полярная/неполярная, ионная и т.д). Как вы думаете, могут ли молекулы гидразина в жидком состоянии образовывать межмолекулярные водородные связи?  Ответ обоснуйте.
\end{enumerate}
Подобно аммиаку молекулы гидразина могут присоединить протон с образованием иона гидразиния. 
\begin{enumerate}
    \item[2.] Запишите формулу иона гидразиния. По какому механизму происходит образование связи между гидразином и протоном?
\end{enumerate}
Итак, у Марка был следующий план: «я очень медленно пропускаю гидразин над иридиевым катализатором, и гидразин превращается в N$_2$ и H$_2$. Я направляю водород в ограниченное пространство и сжигаю его».

Оптимистичный прогноз Марка Уотни был следующим: «В баках плещется 292 литра топлива [гидразина] — достаточно, чтобы получить едва ли не 600 литров воды!».
\begin{enumerate}
\item[3.] Запишите уравнения реакций, по которым Марк планировал получить воду. Оцените, насколько верны расчеты Марка, если плотность жидкого гидразина составляет 1.01 г/см$_3$.
\end{enumerate}

Марк Уотни в своем журнале отмечал: «Главный параметр, который мне нужно было наблюдать — температура. Разложение гидразина чрезвычайно экзотермично. Поэтому я проводил реакцию очень осторожно, внимательно наблюдая за показаниями термопары».

\begin{enumerate}
\item[4.] Рассчитайте, какое количество теплоты выделится при разложении имеющихся в распоряжении Марка 292 литров гидразина, если в реакции разложения гидразина выделяется 50 кДж/моль тепла. 
\end{enumerate}

