\assignementTitle{Металл-органические каркасы}{13}{2}

В 2000-х годах большое внимание ученых-исследователей получил новый класс соединений, названных металл-органическими каркасами. Эти соединения состоят из ионов металлов, соединенных между собой органическими молекулами, при этом происходит формирование бесконечных одномерных, двумерных или трехмерных структур (рис.1).

\putImgWOCaption{8cm}{4}

\begin{center}
    Рисунок 1. Схематическая иллюстрация объединения ионов металлов и органических молекул в трехмерную структуру
\end{center}


Уникальной особенностью этих соединений является крайне высокая площадь поверхности (на сегодняшний день самая большая среди всех известных пористых материалов).Один грамм металл-органического каркаса (а это примерно объем одной горошины) может иметь площадь поверхности сравнимую с размерами 40 теннисных кортов!

Металл-органические каркасы находят широкое применение в различных областях: для адсорбции и разделения газов, катализа, хранения и доставки лекарств, создания люминесцентных и флуоресцентных материалов, а также сенсоров. Композитные материалы на основе металл-органических каркасов и полимерных матриц применяют как селективные мембраны для разделения смесей газов. Например, металл-органический каркас \textbf{CuBDC}, состоящий из ионов меди и органической молекулы X находит применение в составе мембраны для разделения смесей CO$_2$ и CH$_4$.\\

Синтез этого органического линкера можно осуществить по схеме:
$$CaC_2\xrightarrow{H_2O}\textbf{A}\xrightarrow[450^{\circ}]{C_\text{акт}}\textbf{B}\xrightarrow[AlCl_3]{CH_3Cl}\textbf{C}\xrightarrow[H_3PO_4]{CH_2=CH-CH_3}\textbf{D}\xrightarrow[H_2SO_4]{KMnO_4}\textbf{X}$$
\begin{enumerate}
    \item Изобразите структурные формулы всех зашифрованных веществ (\textbf{A-D}, \textbf{X}), если известно, что соединение \textbf{Х} не образует внутримолекулярные водородные связи.
    \item Какой изомерный продукт образуется в реакции превращения \textbf{С $\rightarrow$ D}? Нарисуйте структурную формулу изомера, а также объясните причину его образования с точки зрения электронных эффектов заместителей.
    \item Напишите уравнение реакции окисления соединения \textbf{D} перманганатом калия в сернокислой среде
\end{enumerate}
