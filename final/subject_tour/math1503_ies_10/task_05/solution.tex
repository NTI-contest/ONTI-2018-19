\solutionSection

\begin{enumerate}
    \item [Способ 1] \;
    
    \putImgWOCaption{10cm}{1}

    Согласно условию требуется найти наименьшее расстояние меду прямой $у=x-3$ и параболой $y_1=2x^2+3$ и проверить выполнение дополнительного условия. Выберем на параболе произвольную точку $M (x,2x^2+3)$. Запишем функцию $f(x)$ расстояния между точкой, принадлежащей параболе и прямой $x-y-3=0 (A=1, B=-1, C=-3)$:
    $$f(x)=\frac{|Ax+By+C|}{\sqrt{A^2+B^2}}=\frac{|x-y-3|}{\sqrt{2}}=\frac{|x-2x^2-6|}{\sqrt{2}}$$
    Для нахождения минимума функции, находим производную функции (модуль остается) и приравниваем к нулю. 
    $$f^{'}(x)=\frac{1}{\sqrt{2}}|1-4x|=0$$
    $$x=\frac{1}{4}$$
    Проверим выполнение достаточного условия экстремума. 
    $$f^{''}(x)=\frac{1}{\sqrt{2}}|0-4|>0$$
    Выполняется для всех х, т .е. функция f(x) достигает минимума при  $x=\frac{1}{4}$
    $$f\left(\frac{1}{4}\right)_{min} =\frac{1}{\sqrt{2}}\left|\frac{1}{4}-\frac{1}{8}-6\right|=\frac{47}{8\sqrt{2}}$$
    Искомая дорога – перпендикуляр из этой точки на прямую, и он не пересекает окружности города.\\
    \answerMath {Оптимальное расстояние для перевоза груза $\frac{47}{8\sqrt{2}}\approx4.15$ км}
    \item [Способ 2]
    Минимальное расстояние – это нормаль к прямой, пересекающая параболу в некоторой точке М, координаты которой можно определить, записав уравнение касательной $f^{'}(x)=4х$ и потребовав, чтобы угол наклона касательной к этой точке соответствовал углу наклона исходной прямой $у=x-3$. Получим точку пересечения с абсциссой $x=14$ Далее находим расстояние от точки M до прямой $478\sqrt{2}\approx4.15$ км
    
    \end{enumerate}