\solutionSection

Вероятность линии в первой сроке (все доминошки должны опираться на линию 1), а так как события независимы, тогда: $p_1 = \left(\frac{1}{9}\right)^5$.
\begin{table}[h]
\begin{center}
{\setlength{\extrarowheight}{9pt}
\begin{tabular}{|*{11}{c|}{m{90pt}}}
\hline 10 &\;\;\;\;\;\;\;\;&\;\;\;\;\;\;\;\;&\;\;\;\;\;\;\;\;&\;\;\;\;\;\;\;\;&\;\;\;\;\;\;\;\;&\;\;\;\;\;\;\;\;&\;\;\;\;\;\;\;\;&\;\;\;\;\;\;\;\;&\;\;\;\;\;\;\;\;&\;\;\;\;\;\;\;\;\\
\hline 9 &&&&&&&&&&\\
\hline 8 &&&&&&&&&&\\
\hline 7 &&&&&&&&&&\\
\hline 6 &&&&&&&&&&\\
\hline 5 &&&&&&&&&&\\
\hline 4 &&&&&&&&&&\\
\hline 3 &&&&&&&&&&\\
\hline 2 &&&&&&&&&&\\
\hline 1 &&&&&&&&&&\\
\hline \;\;\;\;\;\;\;\;& 1 & 2 & 3 & 4 & 5 & 6 & 7 & 8 & 9 & 10 \\
\hline &\multicolumn{5}{c}{5-ти элементный блок № 1}&&&&&\\
\hline &&\multicolumn{5}{c}{ 5-ти элементный блок № 1}&&&&\\
\hline &&&&&&\multicolumn{5}{c|}{ 5-ти элементный блок №6} \\
\hline
\end{tabular}}
\end{center}
\end{table} \\
\begin{enumerate}
\item Вероятность линии во второй строке (доминошки могут стоять на линии 1 или на линии 2), так как события независимы: $p_2 =\left(\frac{2}{9}\right)^5$.
\item Вероятности на линиях $3 - 9: p_3 = p_4 = p_5 = p_6 = p_7 = p_8 = p_9 = p_2$.
\item Вероятность появления линии в строке 10 означает, что доминошка стоит на линии 9, т.е. $p_{10} = p_1$.
\item Заполнение перовой линии, означает и заполнение второй и, следовательно, такой вариант уже учтен при расчете вероятности появления второй линии. Тоже утверждение справедливо для 10-й и 9-й линии.
\item Таким образом, вероятность появления 5-ти элементной линии в пятиэлементном блоке, равна сумме вероятностей: $p_2+p_3+p_4+p_5+p_6+p_7+p_8+p_9 = 8\cdotp2 = 8\cdot\left(\frac{2}{9}\right)^5$.
\item Таких блоков 5-ти элементных блоков у нас 6. Таким образом, вероятность появления линии $6\cdot8\cdot\left(\frac{2}{9}\right)^5$.
\end{enumerate}
\answerMath {$48\cdot\left(\frac{2}{9}\right)^5$.}