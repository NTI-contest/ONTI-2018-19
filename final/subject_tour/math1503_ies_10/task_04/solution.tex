\solutionSection

Шарик пройдет решетку только в случае если его координаты его центра находятся в диапазоне $(\pm2.5, \pm2.5)$ см от центра ячейки. Т.е. окно безопасности равно $5\cdot5=25$ кв. см. Так как у нас есть N ячеек, есть N окон безопасности. Вероятность шарика попасть в любое окно пропорционально отношению площадей окна безопасности и площади ячейки с прутьями. Так как ширина прутьев 10 см, можно считать, что к каждой ячейке добавляются по 5 см с каждой стороны. Поэтому площадь ячейки с прутьями $30\cdot30 = 900$ кв. см.\\
$$p = N\cdot\frac{25}{N}\cdot900 = \frac{25}{900} = 0.027778$$
$$N = p\cdot100000 = 2777.8 = 2778$$
\answerMath {2778.}

\markSection

Дано соотношение вероятности мячика попасть (или миновать) в решетку (5 баллов) 

Дан верный ответ (5 баллов).
