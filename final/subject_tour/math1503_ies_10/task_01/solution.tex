\solutionSection

\begin{enumerate}
    \item[Способ 1] Визуализируем условия задачи в виде таблицы посещаемости офиса компании:
    \begin{table}[h]
        \small
    \begin{center}
    {\setlength{\extrarowheight}{9pt}
    \begin{tabular}{|*{9}{c|}{m{90pt}}}
    \hline $-$ & & Пн & Вт & Ср & Чт & Пт & Сб & Всего \\
    \hline 1 камера & & + & + & + & + & + & + & 510 \\
    \hline 2 камера & & + & + & + & $-$ & $-$ & $-$ & 392 \\
    \hline 3 камера & & $-$ & + & $-$ & $-$ & + & $-$ & 220 \\
    \hline 4 камера & & $-$ & $-$ & + & + & $-$ & + & 208 \\
    \hline 5 камера & & $-$ & $-$ & $-$ & + & + & + & 118 \\
    \hline
    \end{tabular}}
    \end{center}
    \end{table} \\
    За исключение понедельника каждый день упоминается 3 раза. Это приводит к двойному учету посетителей четырьмя последними камерами во все дни кроме понедельника. Таким образом, искомое количество посетителей в понедельник можно найти из следующего выражения:
    $$2\cdot510 - (392 + 220 + 208 + 118) = 1020 - 938 = 82$$
    \item[Способ 2] Составим систему уравнений:
    \begin{gather}
    \text{ПН}+\text{ВТ}+\text{СР}+\text{ЧТ}+\text{ПТ}+\text{СБ}=510\\
    \text{ПН}+\text{ВТ}+\text{СР}=392\\
    \text{ВТ}+\text{ПТ}=220\\
    \text{СР}+\text{ЧТ}+\text{СБ}=208\\
    \text{ЧТ}+\text{ПТ}+\text{СБ}=118
    \end{gather}
    Решив систему уравнений, можно попытаться найти ответ. Однако, следует заметить, решение системы значительно упрощается, если обратить внимание на тот факт, что ответ можно получить, вычитая уравнения (3) и (4) из уравнения (1), т.е. $\text{ПН}=82$.
\end{enumerate}
\answerMath {В офисе компании в понедельник побывало 82 человека.}

\markSection

Правильно составлена система уравнений или есть таблица учета наблюдений посещаемости офиса компании (5 баллов).

Получено правильное решение (15 баллов).

