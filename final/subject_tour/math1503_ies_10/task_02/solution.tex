\solutionSection

\begin{enumerate}
    \item[Способ 1] По условной вероятности. Во всех событиях, которые приводят к описываемому в условии результату – В забирает исправный светильник (с вероятность 1), поэтому его удаляем из рассотрения сразу.   П – Петр берет исправный светильник (выбирая из 3 из которых 2 исправных, 1 нет), т.е. берет любой за исключением неисправного, вероятность события: $\left(1-\frac{1}{3}\right)$, 2) Ф – то же самое при условии что на один исправный светильник в наборе меньше, вероятность $\left(1-\frac{1}{2}\right)$,  3) $\overline{\text{Л}}$ – берет оставшийся неисправный светильник, вероятность 1. Вероятность события ПВФ$\overline{\text{Л}}$, при выполнении всех описанных условий равна $\frac{2}{3}\cdot\frac{1}{2}\cdot1=\frac{1}{3}$\\
    \answerMath {$Р=\frac{1}{3}$} \\
    Примечание. Вариант ответа $\frac{3}{4}\cdot\frac{2}{3}\cdot1\cdot1=\frac{1}{2}$ (дает ошибочный ответ, если не исключить сразу того, кто не выбирает лампочки)
    \item[Способ 2] 2.1. Считаем варианты всех возможных событий методом перебора арифметически\\
    1,2,3 – исправные лампочки, – неисправная \\
    \begin{table}[h]
    \begin{center}
    {\setlength{\extrarowheight}{9pt}
    \begin{tabular}{|*{7}{c|}{m{90pt}}}
    \hline П & 1 & 2 & 3 & 1 & 2 & 3 \\
    \hline Ф & 2 & 3 & 1 & 3 & 1 & 2 \\
    \hline В & 3 & 1 & 2 & 2 & 3 & 1 \\
    \hline Л & $-$ & $-$ & $-$ & $-$ & $-$ & $-$ \\
    \hline
    \end{tabular}}
    \end{center}
    \end{table} \\
    6 благоприятствующих событий. Всего событий $6\cdot(4-1)= 18$, так как события где В неисправная лампочка исключается.\\
    \answerMath {$\frac{6}{18} =\frac{1}{3}$}\\
    Примечание. Вариант простого перебора, при котором лампочки 1,2,3 неразличимы тоже может учитываться как правильный, поскольку при делении числа вариантов $n!$ (событий, в данном случае $3!$) при подсчете вероятности сокращаются.  
    \item[Способ 3] Рисуем дерево графов, считаем методом перебора геометрически.\\ \answerMath {$\frac{6}{18} =\frac{1}{3}$}
\end{enumerate}

\markSection

Определено и описано правильное событие (2 балла). 

Правильно рассчитаны условные вероятности событий (2 балла). 

Правильно получен ответ (1 балл).
