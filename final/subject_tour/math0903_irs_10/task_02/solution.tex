\solutionSection
\begin{enumerate}    
   \item [а)] $x=y=z=2$ является решением данного уравнения, так как $2^{2018}+2^{2018}=2\cdot2^{2018}=2^{2019}$.

   \item [б)]  Тройки чисел $(2n^{2019};2n^{2019};2n^{2018})$ для любого натурального $n$ являются решениями: $$(2n^{2019})^{2018}+(2n^{2019})^{2018}=2\cdot2^{2018}n^{2019\cdot2018}=(2n^{2018})^{2019}.$$

   \item [в)] Воспользуемся тождеством: 
    $$
    a^n(a^n+b^n)^n+b^n(a^n+b^n)^n=(a^n+b^n)\cdot(a^n+b^n)^n=(a^n+b^n)^{n+1}.
    $$
    Возьмем $a$ любое нечетное число, $b$ -- четное и $n=2018$. Тогда решениями уравнения являются все тройки чисел $x=a(a^{2018}+b^{2018}),$ $y=b(a^{2018}+b^{2018})$, $z=a^{2018}+b^{2018}.$ Нечетность $z$ очевидна.
\end{enumerate}

\additionalCriteria
ОЦЕНКА: только за пункт а) --- 10 баллов, за пункт б) --- 20 баллов (включает пункт а)), за пункт в) --- 40 баллов (включает первые два пункта)
