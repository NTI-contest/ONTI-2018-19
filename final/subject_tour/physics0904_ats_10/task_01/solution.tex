\solutionSection

\begin{enumerate}
    \item Для того, чтобы пролететь расстояние $H$ и остановиться дрон должен первую половину пути разгоняться с максимально возможным ускорением, а вторую половину пути с таким же ускорением тормозить.
    
    Время, за которое дрон достигнет половины высоты $\tau =\sqrt{\frac{H}{a}}$, на остановку уйдет такое же время.
    
    Следовательно полное время будет равно $2\tau =2 \sqrt{\frac{H}{a}} = 2.8 \: \text{с}$.

    \answerMath{$2\tau =2 \sqrt{\frac{H}{a}} = 2.8 \: \text{с}$.}

    \markSection
    
    \begin{itemize}
        \item Указано (или использовано в решении) условие на движение дрона для достижения минимального времени – 5 баллов.
        \item Получено выражение для минимального времени – 3 балла
        \item Получен правильный числовой ответ – 2 балла.
    \end{itemize}

    \item Т.к. дрон должен остановиться на высоте $H$, изменение его кинетической энергии равно 0. Следовательно, работа, совершенная двигателями равна работе силы тяжести, т.е. $А = mgH$. Среднюю мощность можно выразить как отношение совершенной работы времени, за которое она совершена (оно рассчитано в п.1).

    $$P=\frac{A}{2\tau} =\frac{mgH}{2\sqrt{\frac{H}{a}}}=20 \: \text{Вт}$$

    \answerMath{$P=\frac{mgH}{2\sqrt{\frac{H}{a}}}=20 \: \text{Вт}$.}

    \markSection
    
    \begin{itemize}
        \item Записан закон сохранения энергии или теорема о кинетической энергии – 3 балла.
        \item Получено выражение для совершенной двигателем работы – 2 балла
        \item Получено выражение для средней мощности – 2 балла
        \item Получен правильный числовой ответ – 2 балла.
    \end{itemize}

    \item Для решения задачи можно использовать аналогию с оптикой. Угол, под которым необходимо двигаться для того, чтобы пройти две среды с разными (но постоянными) скоростями движения в них, может быть определен по закону преломления. В нашем случае, мы можем считать прямую, по которой движется монорельс, за границу раздела сред. Тогда минимальному времени будет соответствовать движение под критическим углом полного внутреннего отражения, который равен:
    
    $$\alpha_\text{кр}=arcsin\left( \frac{V_\text{дрона}}{V_\text{монорельса}}\right)=arcsin\left(\frac{1}{2}\right)=30^\circ$$ 

    Теперь задача поиска расстояний, которые пройдет поезд становится чисто геометрической:

    \putImgWOCaption{7cm}{2}
    
    $$AC= l_1-l_2  tg\alpha_\text{кр}$$
    $$CB= l_2/\cos\alpha_\text{кр}$$ 
    Искомое время 
    $$\tau =\frac{AC}{V_\text{монорельса}} + \frac{CB}{V_\text{дрона}} =\frac{l_1-l_2  tg\alpha_\text{кр}}{V_\text{монорельса}} +\frac{\frac{l_2}{\cos\alpha_\text{кр}}}{V_\text{дрона}} =2232 \: \text{c}=37 \: \text{мин}.$$

    \answerMath{$\tau =\frac{l_1-l_2  tg\alpha_\text{кр}}{V_\text{монорельса}} +\frac{\frac{l_2}{\cos\alpha_\text{кр}}}{V_\text{дрона}} =2232 \: \text{c}=37 \: \text{мин}$.}

    \markSection

    \begin{itemize}
        \item Выдвинут любой разумный критерии для минимальности времени – 2 балла
        \item Найден угол, под которым необходимо двигаться дрону – 2 балла
        \item Найдена точка, до которой едет монорельс или соответствующее расстояние – 2 балла
        \item Найдено расстояние, которое пролетит дрон – 1 балл
        \item Получено выражение для полного времени движения – 1 балл
        \item Получен правильный числовой ответ для времени – 2 балла
    \end{itemize}

    \item Задача может быть сведена к предыдущей, если перейти в систему отсчёта, связанную с воздухом. Нужно учесть, что в такой системе отсчета точка B будет удаляться от неподвижного, относительно воздуха дрона, а скорость поезда окажется равной $V_\text{монорельса}^\prime= V_\text{монорельса} +U$, а соответственно изменится и критический угол: $\alpha_\text{кр}^\prime=arcsin\left(\frac{V_\text{дрона}}{V_\text{монорельса}^\prime}\right))$.
 
    \putImgWOCaption{7cm}{3}

    Тогда искомые расстояния: $C^\prime B^\prime=  l_2/\cos \alpha_\text{кр}^\prime$; $AC^\prime= l_1-l_2  tg\alpha_\text{кр}^\prime+BB^\prime$, где $BB^\prime=U\tau$  – расстояние, на которое переместится точка $B$ в системе отсчёта, связанной с воздухом.

    Уравнение на время движения $\tau$:
    
    $$\tau =\frac{l_1-l_2  tg\alpha_\text{кр}^\prime+U\tau}{V_\text{монорельса}+U}+\frac{l_2/\cos\alpha_\text{кр}^\prime}{V_\text{дрона}}$$
    
    Откуда: 

    $$ \tau =\frac{l_2 \cdot (V_\text{монорельса}+U)+V_\text{дрона} \cdot \cos\alpha_\text{кр}^\prime \cdot (l_1-l_2  tg\alpha_\text{кр}^\prime)}{(V_\text{монорельса}+U) \cdot V_\text{дрона} \cdot \cos\alpha_\text{кр}^\prime-V_\text{дрона} U \cos\alpha_\text{кр}^\prime}$$

    \answerMath{3230 c = 54 мин.}

    \markSection

    \begin{itemize}
        \item Предложен переход в другую систему отсчета – 2 балла
        \item Найден угол, под которым необходимо двигаться дрону – 2 балла
        \item Найдена точка, до которой едет монорельс или соответствующее расстояние – 2 балла
        \item Найдено расстояние, которое пролетит дрон – 1 балл
        \item Получено выражение для полного времени движения – 1 балл
        \item Получен правильный числовой ответ для времени – 2 балла
    \end{itemize}

    \item Для того, чтобы облететь препятствие, повернувшись на минимальный угол, дрону необходимо направиться в крайнюю точку препятствия по направлению ветра под углом $\alpha$.
    
    Тогда можно записать условие на угол $\alpha$:
    $$ \left\{
        \begin{aligned}
            \frac{L}{2}=(V sin\alpha +U)\tau\\ 
            H=V \tau  \cdot \cos\alpha
        \end{aligned}
    \right. $$

    Из этой системы получим квадратное уравнение на синус угла: 
    $$sin^2\alpha \left(1+\left(\frac{L}{2H}\right)^2 \right)+2 \frac{U}{V} sin\alpha + \left(\left(\frac{U}{V}\right)^2+\left(\frac{L}{2H}\right)^2\right)=0$$

    Откуда единственный имеющий физический смысл корень: $sin\alpha =0.49, т.е. \alpha \approx 30^\circ$.

    \answerMath{$\alpha \approx 30^\circ$.}

    \markSection
    
    \begin{itemize}
        \item Записано условие для облета препятствия дроном – 2 балла
        \item Записаны соответствующие уравнения из законов движения – 2 балла
        \item Получено квадратное уравнение для функции угла – 4 балла
        \item Получен правильный числовой ответ для угла – 2 балла
    \end{itemize}
\end{enumerate}