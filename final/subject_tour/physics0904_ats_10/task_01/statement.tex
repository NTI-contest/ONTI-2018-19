\assignementTitle{Доставка дронами}{50}{1}

\begin{enumerate}
    \item Дрон покоится на поверхности земли. Двигатели дрона могут сообщить ему ускорение не больше, чем 20 м/с$^2$ в любую сторону. За какое минимальное время он сможет зависнуть без скорости на высоте $H = 40$ м над точкой, где он стоял? Считайте, что направление ускорения и его величину дрон может изменять мгновенно.
    \item Какова при этом минимальная средняя мощность двигателей дрона? Масса дрона $m = 140$ г.
    \item Система доставки между городами А и Б состоит из двух элементов – монорельса, который ходит по прямой проходящей через город А со скоростью $V_1 = 72$ км/ч и может сбросить груз в любой точке этой прямой и дронов, которые подбирают груз и везут по прямой от места, где груз сброшен, в город Б со скоростью $V_2 = 36$ км/ч. Система настроена так, чтобы доставлять грузы за минимальное время. Каково это время, если расстояния $l_1 = 10$ км, $l_2 = 20$ км, а временем перекладывания груза можно пренебречь?
    
    \putImgWOCaption{5cm}{1}

    \item Как изменится ответ в задач пункта 3, если между городами А и Б дует ветер. Ветер дует параллельно путям монорельса по направлению от Б к А. Ветер не мешает движению монорельса (его скорость относительно земли не изменяется), скорость относительно воздуха у дрона остается такой же, как и в п.3.  Скорость ветра 5 м/с.
    \item Дрон летит перпендикулярно путям монорельса со скоростью 36 км/ч. Дует такой же ветер как в п.4.  На пути дрона на высоте его полета оказался закрепленный дирижабль длиной \linebreak $L = 10$~м. Его скорость относительно земли равна нулю. На какой минимальный угол должен повернуть дрон, чтобы облететь дирижабль, если дрон обнаружил его, когда был на расстоянии \linebreak $H = 4.4$ м от середины дирижабля, а его скорость относительно земли направлена строго в центр дирижабля?
\end{enumerate}