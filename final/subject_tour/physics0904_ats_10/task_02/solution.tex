\solutionSection

\begin{enumerate}
    \item Поскольку слои очень тонкие, можно считать, что половину объема материала составляет вещество А, а вторую половину – вещество Б. Тогда искомая теплоемкость:
    
    $$C_1 \rho_1  \frac{V}{2}+C_2 \rho_2  \frac{V}{2}=C_1 \rho_1  \frac{S \cdot d}{2}+C_2 \rho_2 \frac{S \cdot d}{2}=118.5 \: \text{Дж/(кг}^\circ C)$$

    \answerMath{118.5 Дж/кг$^\circ C$.}

    \markSection

    \begin{itemize}
        \item Сделано предположение о распределении веществ А и Б в материале (или использовано в вычислении) – 4 балла
        \item Получено выражение для теплоемкости – 4 балла
        \item Получен правильный числовой ответ – 2 балла
    \end{itemize}

    \item Слои материала можно рассматривать как отдельные конденсаторы. 
    
    Максимальная емкость будет при расположении слоев перпендикулярно обкладкам, т.к. можно рассматривать такую систему, как набор параллельных конденсаторов. Емкости в этом случае складываются:

    $$С=\frac{\varepsilon S\varepsilon_0}{d}$$
    $$C_1=C_{11}+C_{12}=\frac{\varepsilon_0 \varepsilon_1 S}{2d}+\frac{\varepsilon_0 \varepsilon_2 S}{2d}=\frac{\varepsilon_0 S}{2d} (\varepsilon_1+\varepsilon_2 )=13.2 \: \text{пФ}$$

    \answerMath{13.2 пФ.}

    \markSection

    \begin{itemize}
        \item Сделано предположение о том, что слои можно рассматривать как отдельные конденсаторы – 2 балла
        \item Указано, что емкость максимальна в случае расположения слоев перпендикулярно обкладкам – 2 балла
        \item Получено выражение для емкости системы – 4 балла
        \item Получен правильный числовой ответ – 2 балла
    \end{itemize}

    \item В случае, если слои расположены параллельно обкладкам, такую систему можно рассматривать как систему последовательно подключенных конденсаторов. Для нее: 

    $$C_2=\frac{C_{21} C_{22}}{C_{21}+C_{22}}=\frac{\frac{2\varepsilon_0 \varepsilon_1 S}{d} \cdot \frac{2\varepsilon_0 \varepsilon_2 S}{d}}{\frac{2\varepsilon_0 \varepsilon_1 S}{d}+\frac{2\varepsilon_0 \varepsilon_2 S}{d}}=\frac{2\varepsilon_1 \varepsilon_2}{\varepsilon_1+\varepsilon_2} \cdot \frac{S\varepsilon_0}{d}=7.4 \: \text{пФ}$$

    Заряд, протекающий через провода при переключении можно найти, как:
    $$\Delta q=C_1U – C_2U = (C_1 – C_2)U$$
    $$\frac{C_1 U^2}{2}-\frac{C_2 U^2}{2}+Q=U \Delta q=U^2 (C_1-C_2)$$
    $$Q=U^2 (C_1-C_2 )-\frac{U^2 (C_1-C_2 )}{2}=\frac{U^2 (C_1-C_2 )}{2}=0.43 \: \text{нДж}$$

    \answerMath{0.43 нДж.}

    \markSection

    \begin{itemize}
        \item Получено выражение для емкости в случае слоев параллельно обкладкам – 2 балла
        \item Получено численное значение этой емкости (или рассчитано внутри общей формулы) – 2 балла.
        \item Получено выражение для заряда, протекающего через цепь – 2 балла
        \item Записан закон сохранения энергии – 2 балла
        \item Получено выражение для выделившегося тепла – 1 балл
        \item Получен правильный числовой ответ – 1 балл
    \end{itemize}
    
    \item Запишем тепловой поток для случая слоев параллельных обкладкам: 
    
    $$\Phi_\text{парал}=\frac{2S\ae_1 \ae_2}{\ae_1+\ae_2}\cdot \frac{\Delta t}{d}$$
    
    Запишем тепловой поток для случая слоев перпендикулярных обкладкам (если тепло не перетекает из слоя в слой):
    
    $$\Phi_\text{перп}=\frac{(\ae_1+\ae_2 )S \Delta t}{2d}$$

    Тогда:
    
    $$\frac{\Phi_\text{парал}}{\Phi_\text{перп}}=\frac{4\ae_1 \ae_2}{(\ae_1+\ae_2 )^2} = \frac{4\ae_2}{\ae_1 \left(1+\frac{\ae_2}{\ae_1} \right)}=\frac{4 \: \text{K}}{1+\text{K}}=4/3$$

    \answerMath{4/3.}

    \markSection
    \begin{itemize}
        \item Получено выражение для потока  в случае слоев параллельно обкладкам – 4 балла
        \item Получено выражение для потока  в случае слоев перпендикулярно обкладкам – 4 балла
        \item Получено выражение для отношения – 2 балла
        \item Получен правильный числовой ответ – 2 балл
    \end{itemize}

    \item Напряжение на таком конденсаторе можно выразить, как  сумму напряжений на слоях:
    $$U = U_\text{A} \cdot N + U_\text{Б} \cdot N$$
    Заряды на всех конденсаторах будут равны (т.к. мы можем рассматривать систему как совокупность последовательных конденсаторов), следовательно:
    $$C_\text{A}U_\text{A} = С_\text{Б}U_\text{Б}$$
    $$C_\text{A}=\frac{S\varepsilon \varepsilon_\text{A}}{\Delta}$$ 
    $$C_\text{Б}=\frac{S\varepsilon \varepsilon_\text{Б}}{\Delta}$$ 

    Отсюда:
    $$ \left\{
        \begin{aligned}
            U_\text{A}=\frac{C_\text{Б}}{C_\text{A}}U_\text{Б} \\
            U_\text{A}+U_\text{Б}=\frac{U}{N}
        \end{aligned}
    \right. $$

    $$U_\text{A}=\frac{U}{1+\frac{C_\text{A}}{C_\text{Б}}} \qquad U_\text{Б}=\frac{U}{1+\frac{C_\text{Б}}{C_\text{А}}}$$
    Условие наличия пробоя: 

    $$E_\text{A} = U_\text{A}/\Delta \qquad E_\text{d} = U_\text{Б}/\Delta \qquad E_\text{A}=\frac{2U_\text{A} C_\text{Б}}{d(C_\text{A}+C_\text{Б})} \qquad E_\text{d}=\frac{2U_\text{Б} C_\text{А}}{d(C_\text{A}+C_\text{Б})}$$
    
    Напряжение, при котором возникает пробой в каждом из слоев:
    $$U_\text{A}=\frac{d \cdot E_\text{A} (C_\text{A}+C_\text{Б})}{2C_\text{Б}}=18 \: \text{В}$$
    $$U_\text{Б}=\frac{d \cdot E_\text{d} (C_\text{A}+C_\text{Б})}{2C_\text{А}}=9.6 \: \text{В}$$
    Чтобы пробить конденсатор надо пробить оба слоя, следовательно:
    $$U=max(U_\text{A};U_\text{Б})$$
    \answerMath{18 В.}

    \begin{itemize}
        \item Записано условие возникновения пробоя – 4 балла
        \item Получены выражения для предельных напряжений на слоях –2 балла
        \item Правильно сделан вывод о необходимости выбора большего из напряжений – 2 балла
        \item Получен правильный числовой ответ – 2 балл
    \end{itemize}
\end{enumerate}
