\assignementTitle{Слоистый материал}{50}{2}

Материал состоит из двух веществ (назовем их А и Б), расположенных последовательно одно за другим в несколько слоев одинаковой толщины $\Delta = 10$ мкм. Диэлектрическая проницаемость одного из веществ $\varepsilon_\text{А} = 10$, второго $\varepsilon_\text{Б} = 2$; удельная теплоемкость вещества А $c_\text{A} = 1.3$ кДж/(кг$\cdot^{\circ}C$), вещества Б $c_\text{Б} = 2.3$ кДж/(кг$\cdot^{\circ}C$). Плотность вещества А – $\rho_\text{А} = 2000$ кг/м$^2$, плотность вещества Б – $\rho_\text{А} = 3000$~кг/м$^2$. Диэлектрическая проницаемость вакуума $\varepsilon_0 = 8.85\cdot10-12$ (Кл$^2$/(Н$\cdot$м$^2$).

\begin{enumerate}
    \item Какую теплоемкость будет иметь такой плоский конденсатор с расстоянием между обкладками $d = 1$ см и площадью обкладок $S = 25$ см$^2$, если пространство между обкладками полностью заполнено данным материалом? (Теплоемкостью обкладок пренебрегите).
    \item Какую максимальную электрическую емкость будет иметь этот конденсатор, если все пространство между обкладками заполнено данным материалом?
    \item В этом конденсаторе, подключенном к источнику постоянного напряжения $U = 12$В заменили наполнитель из данного материала, слои в котором расположены параллельно обкладкам на наполнитель из такого же материала, но расположенного так, что слои в нем находятся перпендикулярно обкладкам. Какое количество тепла рассеялось при этой замене на подводящих проводах? 
    \item Из данного материала делают теплоотводящую трубку. Как надо расположить материал при создании трубки (слои параллельно сечению трубки или перпендикулярно), чтобы мощность теплоотвода оказалась наибольшей? Во сколько раз отличаются мощности теплоотвода в этих двух случаях, если коэффициенты теплопередачи веществ отличаются в два раза $K_\text{A}/K_\text{Б} = 2$. Считайте, что при расположении материала со слоями перпендикулярно сечению трубки, передачи тепла между слоями материала не возникает.
    \item Известно, что напряжение пробоя вещества А – $E_{max\text{A}} = 600$ В/м, а вещества Б – $E_{max\text{Б}} =$ \linebreak $= 1600$~В/м.  Можно ли конденсатор, слои материала в котором направленны параллельно обкладкам, подключать к напряжению 24В? Каково максимальное напряжение, при котором не возникнет пробой?
\end{enumerate}