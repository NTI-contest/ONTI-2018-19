\assignementTitle{Экситон Бора}{}{1}

Одной из важнейших моделей в теории квантовых точек является представление об экситоне~- 
связанной паре электрон-”дырка”. В самом простом приближении, по аналогии с моделью атома Бора, 
экситон можно рассматривать их как две материальные точки, вращающиеся вокруг общего центра масс, 
одна из которых имеет заряд +е, а вторая -е. Можно также построить простейшую теорию поведения экситона, 
воспользовавшись аналогией с моделью атома Бора. В качестве правила квантования нужно взять следующее: 
${m_e \cdot V_e \cdot a_e + m_h \cdot V_h \cdot a_h=\frac{nh}{2 \pi}}$, где $m_e$ и $m_h$  - эффективные массы электрона 
и «дырки», $V_e$ и $V_h$ - их скорости, $a_e$ и $a_h$~- расстояния от электрона и дырки соответственно 
до оси вращения, $h=6,63 \cdot 10^{-34}$  Дж $\cdot$ c – постоянная Планка, а $n$  – натуральное число. 
Будем считать эффективные массы электрона и “дырки” постоянными и равными 25\% от массы свободного электрона 
$m_{\text{e св}}=9,1 \cdot 10^{-31}$кг. Отличие от обычной массы возникает из-за взаимодействия электрона с полем кристаллической решетки.

Экситон чувствителен ко внешним воздействиям, в частности к тепловому движению частиц в кристалле. 
Если полная энергия экситона окажется больше по модулю, чем кинетическая энергия теплового движения, 
экситон разрушится. Среднюю кинетическую энергию теплового движения в кристалле будем считать равной 
$E=kT$, где $k=1,38 \cdot 10^{-23}$ Дж/К  - постоянная Больцмана, а $Т$ - абсолютная температура кристалла. В квантовой точке число степеней 
свободы электрона в экситоне ограничено до нуля, что приводит к увеличению энергии связи и, 
как следствие, температуры, при которой он может наблюдаться. 

Заряд электрона $e=1.6 \cdot 10^{-19}$ Кл, электрическая постоянная $\epsilon_0=8.85 \cdot 10^{-12}$  $\frac{\text{Кл}^2}{\text{Н} \cdot \text{м}^2}$, скорость 
света $c = 3 \cdot 10^8$~м/c.

\begin{enumerate}
    \item Найдите возможные (допустимые правилом квантования) угловые скорости вращения частиц в этой 
    модели для свободного экситона диаметром 20~нм. Под диаметром экситона будем понимать расстояние 
    между электроном и «дыркой».
    \item Найдите энергию основного состояния для экситона диаметром 20~нм в веществе с диэлектрической проницаемостью $\epsilon=11.7$?
    \item При какой температуре может наблюдаться основное состояние такого экситона?
    \item Определите минимальную длину волны света, которую может излучить возбужденный свободный экситона, основное состояние которого стабильно при температуре не выше 300~К.
    \item Найдите диапазон углов, под которыми будет наблюдаться спектр первого порядка для излучения свободного экситона при переходе из произвольного возбужденного состояния в основное состояние, если наблюдение производится с помощью решетки с шагом $d= 4 \cdot 10^4$~м. 
    Основное состояние стабильно при температуре 300~К.    
\end{enumerate}

\textit{(В качестве ответа допустимо указать арксинусы углов).}