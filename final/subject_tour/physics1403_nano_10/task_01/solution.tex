\solutionSection
\begin{enumerate}
\item Будем рассматривать электрон и «дырку» как две одинаковые материальные точки. Центр их вращения находится в центре масс, т.е. по середине указанного расстояния.\\
Тогда, исходя из предложенного правила квантования можно найти возможные для данного экситона угловые скорости вращения электрона и «дырки» (в силу одинаковых масс они будут одинаковыми):\\
$2m_e\left(\frac{l}{2}\right)^2 \omega=\frac{hn}{2\pi}$, где $\omega$ – угловая скорость вращения, $l$ – диаметр экситона, $m$ – эффективная масса.
Отсюда возможные угловые скорости:
$$\omega=\frac{hn}{\pi ml^2}=\frac{6.63\cdot10^{-34}}{3.14\cdot0.25\cdot9.1\cdot10^{-31}\cdot(20\cdot10^{-9})^2 } n\text{ 1/с}= 2.3\cdot10^{12}n\text{ 1/с}\text{, где }n = 1,2,3…$$
\answerMath{$\omega=\frac{hn}{\pi ml^2}=2.3\cdot10^{12}n\text{ 1/с}\text{, где }n = 1,2,3…$}
\end{enumerate}
\additionalCriteria
\begin{itemize}
\item Записано условие квантования в применении к данной задаче - 5
\item Получен ответ для одного случая. - 2
\item Записан ответ для всех возможных орбит (выражено через целое число n в примере). - 3
\end{itemize}
\begin{enumerate}
\item[2.] Рассмотрим полную энергию экситона в нашем приближении. Она состоит из кинетической энергии движения электрона и «дырки» и потенциальной энергии их электростатического взаимодействия.\\
$$E=2\frac{mV^2}{2}-\frac{1}{4\pi\varepsilon_0\varepsilon}\cdot\frac{e^2}{l}$$
Знак минус перед потенциальной энергией возникает из-за разных знаков заряда электрона и «дырки».\\
Запишем так же второй закон Ньютона для электрона (для дырки он будет записываться точно так же, в силу равенства масс по условию).
$$\frac{mV^2}{\frac{l}{2}}=\frac{1}{4\pi\varepsilon_0\varepsilon}\cdot\frac{e^2}{l^2}$$
Расстояние до оси вращения в два раза меньше расстояния между электроном и «дыркой».
Из этих двух уравнений получаем полную энергию экситона диаметром $l$:\\
$$E=-\frac{1}{2}\cdot\frac{1}{4\pi\varepsilon_0\varepsilon}\cdot\frac{e^2}{l}$$
Заметим, что она отрицательна и вдвое меньше потенциальной энергии взаимодействия между электроном и «дыркой».
Найдем эту энергию:
$$E=-\frac{1}{2}\cdot\frac{1}{4\pi\varepsilon_0\varepsilon}\cdot\frac{e^{2}}{l}=-\frac{1.6^2\cdot10^{-38}}{2\cdot4\cdot3.14\cdot8.85\cdot10^{-12}\cdot11.7\cdot20\cdot10^{-9}}\text{ Дж}\thickapprox-0.5\cdot10^{-21}\text{ Дж}$$
\answerMath{$E=-\frac{1}{2}\cdot\frac{1}{4\pi\varepsilon_0\varepsilon}\cdot\frac{e^{2}}{l}\thickapprox-0.5\cdot10^{-21}\text{ Дж}$}
\end{enumerate}
\additionalCriteria
\begin{itemize}
\item Записано выражение для полной энергии экситона 
\item Записан второй закон Ньютона
\item Получено правильное выражение для полной энергии
\item Получено числовое значение
\end{itemize}
\begin{enumerate}
\item[3.] Воспользовавшись решением предыдущей задачи и условием разрушения экситона, которое дано в условии, запишем:
$$E=\left|-\frac{1}{2}\cdot\frac{1}{4\pi\varepsilon_0\varepsilon}\cdot\frac{e^2}{l}\right|\leqslant kT$$
Отсюда $\left|-\frac{1}{2}\cdot\frac{1}{4\pi\varepsilon_0\varepsilon}\cdot\frac{e^2}{l}\right|/k\leqslant T$ или $T \geqslant\frac{0.5\cdot10^{-21}}{1.38\cdot10^{-23}}=36K$\\
\answerMath{Такой экситон может существовать при температуре ниже $T\leqslant36К$}
\end{enumerate}
\additionalCriteria\begin{itemize}
\item Записано выражение сравнения энергии экситона с средней энергией теплового движения.
\item Получено правильное числовое значение
\end{itemize}
\begin{enumerate}
\item[4.] Минимальная длина волны света соответствует максимальной частоте, которая в свою очередь соответствует максимальной выделяемой энергии. Максимальная энергия будет выделяться экситоном при переходе из состояния «бесконечно удаленного» в основное состояние. Такая энергия, по аналогии с моделью атома Бора будет равна по модулю энергии основного состояния экситона, которая может быть определена из условия его стабильности.\\
$$E=kT$$
Тогда 
$$E=kT=h\upsilon=\frac{hc}{\lambda}$$
Отсюда:$\lambda=\frac{hc}{kT}=\frac{6.63\cdot10^{-34}\cdot3\cdot10^8}{1.38\cdot10^{-23}\cdot300}=48$ мкм\\
\answerMath{$\lambda=\frac{hc}{kT}=48$ мкм}
\end{enumerate}
\additionalCriteria
\begin{itemize}
\item Учтено, что полная энергия основного состояния может быть получена из условия стабильности экситона.
\item Записано выражение для частоты или длины волны излучения.
\item Получено правильное числовое значение
\end{itemize}
\begin{enumerate}
\item[5.] Диапазон в котором будет поглощать такой экситон будет лежать между 48мкм, которые были обнаружены в предыдущем пункте и излучением при переходе из второго энергетического состояния в основное. По аналогии с моделью Бора можно записать:\\
$\delta E=E_0\frac{1}{n^2}-\frac{1}{k^2}$, для перехода из n-го состояния в k-тое. В нашем случае, для перехода из основного – 1го состояния во 2ое, $\delta E_12=E_0\left(\frac{1}{1^2}-\frac{1}{2^2}\right)=\frac{3}{4}E_0$.\\
Соответствующая длина волны может быть определена аналогично предыдущему пункту:\\
$$\lambda=\frac{hc}{\frac{3}{4}kT}=64\text{ мкм}$$
Таким образом, диапазон излучения основного состояния такого экситона 48-64мкм\\
Отсюда, используя известное выражение для направления на максимумы дифракционной решетки $d\sin\varphi=k\lambda$, можно найти диапазон углов:
$$\sin\varphi_{max}=\frac{\lambda_{max}}{d}=\frac{64\cdot10^{-6}}{4\cdot10^{-4}}=0.16$$					
$$\sin\varphi_{min}=\frac{\lambda_{min}}{d}=\frac{48\cdot10^{-6}}{4\cdot10^{-4}}=0.12$$
$$\varphi\in(\arcsin(0.12);\arcsin(0.16))$$
\answerMath{$\varphi\in[\arcsin(0.12);\arcsin(0.16)]$}
\end{enumerate}
\additionalCriteria
\begin{itemize}
\item Получено значение максимальной длины волны
\item Получено значение минимальной длины волны
\item Записано выражение для направления на максимумы для дифракционной решетки
\item Получено правильное числовое значение
\end{itemize}
