\assignementTitle{Броуновские частицы}{}{2}

Броуновские частицы - очень маленькие частицы твердого вещества, совершающие хаотическое движение в 
жидкостях или газах находящиеся в термодинамическом равновесии с молекулами жидкости или газа, в которых 
они находятся. $k = 1,38 \cdot 10^{-23}$ Дж/К - постоянная Больцмана.

\begin{enumerate}
    \item Определите среднеквадратичную скорость броуновской частицы в газе при комнатной температуре 
    (23 градуса Цельсия). Для простоты считайте частицу кубом со стороной 0,5~мкм. 
    Средняя плотность частицы $\rho=2330$~кг/м$^3$.
    \item Пусть эта частица совершает броуновское движение в азоте. Во сколько раз ее среднеквадратичная 
    скорость меньше среднеквадратичной скорости молекул азота?
    \item Частица, указанная в первом пункте находится в азоте. Сколько частиц азота должны ударять ее 
    строго перпендикулярно одной грани, чтобы она приобрела скорость равную посчитанной выше средне 
    квадратичной вдоль направления перпендикулярного этой грани. Удары считайте абсолютно упругими. 
    Температура азота $t = 300$~K. Молярная масса азота $\mu_N= 0,014$~кг/моль.
    \item Частица, находившаяся в воде на глубине $H$ начинает всплывать без начальной скорости. На глубине 
    $h = H/2$ ее скорость перестает меняться и становится равной $U$, причем скорость $U$ равна скорости, 
    которую приобрела бы эта частица при свободном падении с высоты $H/2$. Считая, что вся работа сил 
    сопротивления переходит в тепло, а температура частицы не изменилась, найдите какая доля выделившегося 
    тепла выделилась до установления постоянной скорости. Размер частицы много меньше $H$ и $h$. Плотность 
    жидкости в $n=3$ раза больше средней плотности частицы.
    \item В теплоизолированный сосуд объемом $V = 4$~л, высотой $h = 40$~см, содержащий одноатомный газ, 
    засыпали сверху 10~г мелкодисперсного порошка броуновских частиц, не сообщая им кинетической энергии, 
    при этом давление в сосуде изменилось на величину dP. Найдите это изменение давления. Считайте, что 
    броуновские частицы распределятся по сосуду равномерно. Ускорение свободного падения $g =9,8$ м/c$^2$.
\end{enumerate}