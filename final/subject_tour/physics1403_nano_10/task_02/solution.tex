\solutionSection
\begin{enumerate}
\item Поскольку броуновская частица находится в состоянии термодинамического равновесия с газом ее внутренняя энергия равна внутренней энергии окружающего его газа.
$$E=\frac{3}{2}kT$$
тогда можно записать:
$$\frac{m<V>^2}{2}=\frac{3}{2}kT$$
или:
$$\frac{\rho a^3<V>^2}{2}=\frac{3}{2}kT$$
где a – сторона куба.
$$\sqrt{<V>^2}=\sqrt{\frac{3kT}{\rho a^3}}=\sqrt{\frac{3\cdot1.38\cdot10^{-23}\cdot296}{2330\cdot5^3\cdot10^{-21} }}=0.0064\text{ м/c}$$
\answerMath{$\sqrt{<V>^2}=\sqrt{\frac{3kT}{\rho a^3}}=0.0064\text{ м/c}$}
\end{enumerate}
\additionalCriteria
\begin{itemize}
\item Замечено, что энергия броуновской частицы равна энергии окружающего ее газа.
\item Записано выражение равенства кинетических энергий.
\item Получено правильное числовое значение для среднеквадратичной скорости или ее квадрата.
\end{itemize}
\begin{enumerate}
\item[2.] Среднеквадратичная скорость молекул азота может быть определено по такой же формуле 
$$\frac{m_{N_2}<V_{N_2}>^2}{2}=\frac{3}{2}kT$$
Тогда
$$\frac{<V_{N_2}>}{<V>}=\sqrt\frac{m}{m_{N_2}}$$
$$m_{N_2}=\frac{\mu_{N_2}}{N_A}$$
$$m=\rho a^3$$
$$\frac{<V_{N_2}>}{<V>}=\sqrt{\frac{N_A\rho a^3}{\mu_{N_2}}}=\sqrt{\frac{6\cdot10^{23}\cdot2330\cdot5^3\cdot10^{-21}}{0.028}}=7.9\cdot10^4$$
\answerMath{$\frac{<V_{N_2}>}{<V>}=\sqrt{\frac{N_A\rho a^3}{\mu_{N_2}}}=7.9\cdot10^4$}
\end{enumerate}
\begin{itemize}
\item Замечено, что энергия броуновской частицы равна энергии окружающего ее газа.
\item Записано выражение для отношения среднеквадратичных скоростей
\item Выражены массы молекулы азота и частицы
\item Получено правильное числовое значение отношения среднеквадратичных сокростей.
\end{itemize}
\begin{enumerate}
\item[3.] В среднем молекулы ударяют частицу со всех сторон одинаково и средний импульс, который они передают равен нулю. Однако, если число молекул ударивших в одну грань окажется выше среднего, частица может получить существенный импульс в сторону перпендикулярную грани.\\
Импульс который приобретет частица равен $p=2N\cdot p_\text{м},$ где $N$ – число частиц, $p_\text{м}$ – импульс одной молекулы а множитель 2 возникает из-за абсолютно упругого удара.\\
Импульс молекулы может быть выражен через ее кинетическую энергию, как:\\
$p_\text{м}=\sqrt{2mE}=\sqrt{2m\cdot\frac{3}{2}kT}$, масса одной молекулы может быть выражена через молярную массу и число Авогадро: $m=\frac{\mu}{N_A}$, что для азота дает $m=\frac{\mu}{N_A}=\frac{0.028}{6.02\cdot10^{23}}=4.65\cdot10^{-26}$ кг
$$N=\frac{M\sqrt{<V>^2}}{\sqrt{2m\cdot\frac{3}{2}kT}}=\frac{\rho a^3\sqrt{<V>^2}}{\sqrt{2m\cdot\frac{3}{2}kT}}=$$
$$=\frac{2330\cdot5^3\cdot10^{-24}\cdot0.0064}{\sqrt{2\cdot4.65\cdot10^{-26}\cdot1.5\cdot1.38\cdot10^{-23}\cdot300}}=7.76\cdot10^5$$
\answerMath{$N=\frac{\rho a^3\sqrt{<V>^2}}{\sqrt{2m\cdot\frac{3}{2}kT}}=7.76\cdot10^5$}
\end{enumerate}
\begin{itemize}
\item Записан закон сохранения импульса. Если не учтен множитель 2, ставится половина баллов.
\item Записан импульс для молекулы азота.
\item Получено правильное числовое значение с точностью до десятых.
\end{itemize}
\begin{enumerate}
\item[4.] В тепло превращается работа силы трения. Запишем эту работу.\\
До установления постоянной скорости:
$$A_1=\rho gV(H-h)-mg(H-h)-\frac{mU^2}{2}$$
где $\rho$ – плотность жидкости, а $V$ – объем тела.\\
После установления постоянной скорости:
$$A_2=\rho gVh-mgh$$
Искомая величина равна:
$$\frac{A_1}{A_1+A_2}=\frac{\rho gV(H-h)-mg(H-h)-\frac{mU^2}{2}}{\rho gVH-mgH-\frac{mU^2}{2}}$$
Заметим, что по условию $\frac{mU^2}{2}=mg\frac{H}{2}$, а $\rho gV=nm$, тогда:
$$\frac{A_1}{A_1+A_2}=\frac{nmg\frac{H}{2}-mg\frac{H}{2}-mg\frac{H}{2}}{nmgH-mgH-mg\frac{H}{2}}=\frac{(n-2)\frac{H}{2}}{(n-1)H-\frac{H}{2}}=\frac{1}{3}$$
\answerMath{$\frac{A_1}{A_1+A_2}=\frac{(n-2)}{2(n-1)-1}=\frac{1}{3}$}
\end{enumerate}
\begin{itemize}
\item Записан закон сохранения энергии для первой половины пути.
\item Записан закон сохранения энергии для второй половины пути или для всего пути.
\item Кинетическая энергия частицы выражена через потенциальную энергию.
\item Сила Архимеда выражена через вес частицы.
\item Получено правильное числовое значение.
\end{itemize}
\begin{enumerate}
\item[5.] Броуновские частицы в силу своего хаотического движения могут быть описаны так же, как и газ.\\
Запишем закон сохранения энергии, т.к. сосуд теплоизолирован:
$$\frac{3}{2}n_\text{Б}VkT_1+\frac{3}{2}nVk(T_1-T_0 )=mg\frac{H}{2}$$
Считая, что частицы в сосуде распределены равномерно, их потенциальная энергия уменьшилась на величину $mg\frac{H}{2}$.\\
Так же можно записать основное уравнение МКТ для приращения давления. Оно произойдет за счет изменения давления газа и в существенно меньшей степени из-за давления броуновских частиц.
$$dP=n_\text{Б}kT_1+nk(T_1-T_0)$$
Домножив второе уравнение на $V$, а первое на $2/3$, увидим, что
$$mg\frac{H}{3}=VdP$$
Отсюда: 
$$dP=mg\frac{H}{3V}=0.01\cdot9.8\cdot\frac{0.4}{3\cdot4\cdot10^{-3}}=3.3\text{ Па}$$
\answerMath{$dP=mg\frac{H}{3V}=3.3$ Па}
\end{enumerate}
\begin{itemize}
\item Сделано предположение, что броуновские частицы можно рассматривать как газ
\item Записан закон сохранения энергии
\item Записано выражение для приращения давления
\item Получено правильное числовое значение с точностью до десятых.
\end{itemize}
