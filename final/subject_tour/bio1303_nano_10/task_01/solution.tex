\solutionSection
\begin{enumerate}
\item
\begin{enumerate}
\item [1 –] Тиллакоид \textit{(1 балл)}
\item [2 –] Строма \textit{(2 балла)}
\item [3 –] Грана \textit{(1 балл)}
\item [4 –] Люмен \textit{(1 балл)}
\end{enumerate}
\item Упорядочивание белков в пространстве и сбор компонентов фотосистем; увеличение площади, на которой возможно размещение белков;  разделение объемов люмена и стромы – дает возможность накопить трансмембранный протонный (электрохимический) градиент; разделение  компонентов мембранной цепи в пространстве;  возможность регуляции реакций фотосинтеза. \textit{(5 баллов, за любые 3 из перечисленных.}
\item Система улавливающая свет и преобразующая его энергию в энергию химических связей (пигменты и их белковое окружение, фотосистема).
Система, обеспечивающая за счет этой энергии транспорт протонов внутрь люмена.
Система, обеспечивающая синтез АТФ за счет протонного градиента. (АТФ-синтаза) \textit{(10 баллов)}.
\end{enumerate}