\solutionSection Предположим что за 2 описанных признака отвечают 2 гена. Исходя из того, что указан гетерогаметный пол и пол потомков во всех парах – по крайней мере один из этих признаков наследуется сцеплено с полом. $\textit{(1 балл)}$\\
Т.к. от брака белых вуки рождаются только белые дети, т.е. нет расщепления, вероятно все белые вуки гомозиготны. При этом, поскольку в браке с окрашенными вуки рождаются окрашенные дети, а от браков окрашенных вуки бывают белые дети – вероятно белый – рецессивный признак. $\textit{(1 балл)}$\\
Рассматривая ситуацию с черным окрасом можно было бы прийти к такому же выводу, однако оказывается, что черный может быть носителем белого. При этом тип шерсти и белый окрас наследуются не сцеплено, а цветные ( по крайней мере черный) окрасы связаны с типом шерсти $\textit{(2 балла)}$. Тогда, либо окрас определяется 2 генами и еще одним определяется тип шерсти, либо имеют место неаллельные взаимодействия генов. Белые потомки черных вуки всегда – мужского пола. Учитывая что он – гетерогаметный (обозначим как $ХУ$), вероятно белая окраска сцеплена с полом. Обозначим ген белой шерсти как $W\textgreater w$, где $W$ – не белая, $w$ – белая. Вероятно этот ген расположен на $X$-хромосоме ( рецессивная аллель проявляется у потомков мужского пола, а потомки женского пола – носители). Тогда белый мужчина будет иметь генотип $X^wY$, окрашенный – $X^WY$, белая женщина –  $X^wX^w$, а женщины с генотипом $X^wX^W$ и $X^WX^W$ будут окрашены. Отношения между геном $W/w$ и геном (генами) окраски – рецессивный эпистаз. $\textit{(4 балла)}$\\
Все известные нам потомки черного цвета – кудрявые. При этом существуют не-черные кудрявые индивиды, но черных гладкошерстных не известно. Соответственно можно предположить либо плейотропное действие гена окраски так же на тип шерсти, либо предположить что за окрас отвечают 2 гена, один из которых наследуется сцепленнос геном кудрявости либо так же имеет плейотропное действие. $\textit{(2 балла)}$\\
Предположим первый вариант, т.к. черный появляется в потомстве вуки других окрасов, т.е. они вероятные носители. Попробуем решить задачу исходя из того, что за окрас отвечает ген $A$, имеющий 3 аллельных варианта $As$ – чепрачный, $Aу$ – соболиный и $A$ – черный. Предположительно черный окрас – рецессивный относительно остальных (см. выше), будем обозначать его а. Так же ген $A$ каким-то образом определяет тип шерсти – гш или кш. $\textit{(2 балла)}$\\\\
Р:$\hspace{110pt}$ $\male$ чепрачный гш $\times$ $\female$ чепрачный кш
$$X^WY\hspace{3pt}AsA? \times X^wX^W\hspace{3pt}AsA?$$
F1:$\hspace{20pt}$ $\male$ чепрачный гш + $\male$ чепрачный кш + $\male$ белый кш + $\female$ чепрачный гш
$$X^WY\hspace{3pt}AsA?\hspace{30pt}X^WY\hspace{3pt}AsA?\hspace{10pt}\hspace{35pt}X^wY\hspace{3pt}A?A?\hspace{25pt}X^WX?\hspace{3pt}AsA?$$
Исходя из наличия белого сына, мать обозначена как XwXW  - окрашенная носительница гена белой шерсти. $\textit{(1 балл)}$\\\\
Р:$\hspace{110pt}$ $\male$ белый кш (F1) $\times$ $\female$ черный кш
$$X^wY\hspace{3pt}A?A?\hspace{34pt}X^wX^W\hspace{3pt}aa$$
F2$\hspace{60pt}$белый кш $\male+\female+$ соболиный кш $\male+\female+$ черный кш $\male+\female$\\
Одна из дочерей и один из сыновей белый у окрашенной матери и белого отца – значит мать – носительница гена белого. Как мы выяснили ранее генотип черных вуки – $aa$, значит генотип матери известен. Поскольку так же у этой пары есть черные дети с генотипами $X^wX^W$ $aa$ и $X^WYaa$ соответственно, отец является носителем аллеля черной окраски: $X^wY$ $A?a$. Тогда, если бы не влияние гена $W$, наблюдалось бы расщепление по фенотипу $1:1$ по окраске: не-черная: черная. Значит вторая аллель окраски у белого отца – соболиная, т.к. именно так окрашены его не-черные окрашенные потомки. Соответственно один из его родителей должен быть носителем аллели $Aу$, которая, вероятно рецессивна по отношению к аллели $As$, а второй – носителем рецессивной аллели $a$. $\textit{(2 балла)}$\\
Интересно, что в этой семье, все вуки кудрявые и все несут хотя бы одну аллель $a$.\\
Так же и во второй паре $\male$ чепрачный кш $\times\female$ соболиный кш, судя по тому, что есть потомки черного цвета оба несут  аллель черного и оба кудрявые.\\ Можно предположить следующую плейотропию: аллель $a$ рецессивен по отношению к остальным аллелям при определении цвета, но при этом его присутствие хотя бы в одной копии дает кудрявую шерсть.\\
Тогда генотип исходной пары:\\
Р:$\hspace{110pt}$ $\male$ чепрачный гш $\times$ $\female$ чепрачный кш
$$X^WY\hspace{3pt}AsAy \times X^wX^W\hspace{3pt}Asa$$
F1:$\hspace{20pt}$ $\male$ чепрачный гш + $\male$ чепрачный кш + $\male$ белый кш + $\female$ чепрачный гш
$$X^WY\hspace{3pt}AsAs\hspace{30pt}X^WY\hspace{3pt}Asa\hspace{10pt}\hspace{35pt}X^wY\hspace{3pt}Aya\hspace{25pt}X^WX?\hspace{3pt}AsAs$$
$\textit{(2 балла)}$\\\\
Р(F1):$\hspace{63pt}$ $\male$ чепрачный гш(F1) $\times$ $\female$ соболиный кш
$$X^WY\hspace{3pt}Asa \times X^wX^W\hspace{3pt}Aya$$
Генотип жены чепрачного кш сына:  $X^wX^W\hspace{3pt}Aya$ – потому что у них есть белые сыновья, т.е. она носительница белого, она имеет соболиный окрас за счет аллели $Ay$ и, поскольку у них есть черные дети она – носительница черного так же как и ее муж. У них могут  быть следующие потомки:\\
\begin{table}[h]
\begin{center}
{\setlength{\extrarowheight}{9pt}
\begin{tabular}{|*{5}{c|}{m{70pt}}}
\hline & $Ay$ $X^W$ & $a$ $X^W$ & $Ay$ $X^w$ & $a$ $X^w$ \\
\hline $As$ $X^W$ & $AsAy$ $X^WX^W$ & $Asa$ $X^WX^W$ & $AsAy$ $X^WX^w$ & $Asa$ $X^WX^w$ \\
\hline & Чепрачный гш \female & Чепрачный гш \female & Чепрачный гш \female & Чепрачный гш \female \\
\hline $a$ $X^w$ & $Aya$ $X^WX^W$ & $aa$ $X^WX^W$ & $Aya$ $X^WX^w$ & $aa$ $X^WX^w$ \\
\hline & Соболиный кш \female & Черный кш \female & Соболиный кш \female & Черный кш \female \\
\hline $As Y$ & $AsAy$ $X^WY$ & $Asa$ $X^WY$ & $AsAy$ $X^wY$ & $Asa$ $X^wY$\\
\hline & Чепрачный гш \male & Чепрачный кш \male & Белый гш \male & Белый кш \male \\
\hline $a$ $Y$ & $Aya$ $X^WY$ & $aa$ $X^WY$ & $Aya$ $X^wY$ & $aa$ $X^wY$ \\
\hline & Соболиный кш \male & Черный кш \male & Белый кш \male & Белый кш \male\\
\hline
\end{tabular}}
\end{center}
\end{table} \\
То есть соболиного гладкошерстного сына у этой пары быть не может, и значит он – не родной внук своего деда. \textit{(3 балла)}

