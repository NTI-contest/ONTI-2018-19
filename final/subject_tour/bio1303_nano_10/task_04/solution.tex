\solutionSection

\begin{enumerate}
    \item Дафния – планктонный рачок, живущий в толще воды, где нет укрытий. У нее есть 2 основных типа врагов – хищников, которые на нее охотятся. Это рыбы (некрупные – мальки этого года и просто мелкие) и беспозвоночные. Рыбы охотятся с помощью зрения, поэтому для защиты от них выгодно уменьшать размер тела и быть как можно прозрачней. Беспозвоночные же ориентируются на колебания воды, и не слишком превосходят дафний размерами, поэтому необходимо быть достаточно крупным, чтобы «не пролезть в рот», а заметность при этом неважна. Соответственно летом, когда появляется молодь рыб дафнии выгодно быть маленькой и прозрачной, но для сохранения размера необходимы выросты. К осени рыбы становятся слишком крупными чтобы охотиться на дафний и выгодно иметь крупное тело без выростов. \textit{(10 баллов)}
    \itemЭто энергетически не выгодно – эти структуры не используются при размножении или питании как структуры тела, но при этом тратится энергия на синтез хитина из которого они состоят. Летом обходиться без них тяжелее чем с ними, но как только просто крупное тело оказывается достаточным для защиты «вооружённые» клоны дафний начинают проигрывать по скорости размножения «безоружным» и вытесняются ими. \textit{(8 баллов)} Кроме того, вероятно рост шлема и иглы провоцируется хим. Веществами выделяемыми в воду самими хищниками, что происходит только когда они присутствуют в озере. \textit{(2 балла)}
    \end{enumerate}