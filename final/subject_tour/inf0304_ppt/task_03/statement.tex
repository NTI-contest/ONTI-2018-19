\assignementTitle{}{20}{}

В умном городе планируют расположение парковок. Известно, что город состоит из нескольких районов. Кроме того, каждый район является самонепересекающимся многоугольником (не обязательно выпуклым).

Известен список всех точек границ каждого района (в направлении по или против часовой стрелки).

Исходя из статистики, ученые вывели максимальную площадь, на которой должна находиться хотя бы одна парковка. Также ученые определили, что число парковок должно зависить не от общей площади города, а от площади отдельных районов. Требуется определить, какое минимальное число парковок следует разместить в городе.

\inputfmtSection

Строка, содержащая два числа: целое число $N$ $(1 \leq N \leq 10)$ — количество районов и число с плавающей запятой $S$ — максимальная площадь, на которой должна находиться хотя бы 1 парковка.

Далее для каждого района:

Строка с числом $M$ $(1 \leq M \leq 20)$ — количеством точек, описывающих район.

$M$ строк с целыми числами $X$ $(-10^9 \leq X \leq 10^9)$ и $Y$ $(-10^9 \leq Y \leq 10^9)$ — координатами кадой точки.

Координаты указаны в порядке соединения, но направление может быть как по, так и против часовой стрелки.

\outputfmtSection

Минимальное число парковок, которое нужно разместить в городе.

\sampleTitle{1}

\begin{myverbbox}[\small]{\vinput}
    1 1.5
    4
    0 1
    1 0
    0 -1
    -1 0
\end{myverbbox}
\begin{myverbbox}[\small]{\voutput}
    2
\end{myverbbox}
\inputoutputTable

\solutionSection

Суть задачи заключается в верном нахождении площади района. Подсчитаем площадь методом трапеций. Так как обход идет в одном направлении, ненужные площади сократятся из-за разных знаков, и получится необходимый ответ. Однако, порядок обхода точек неизвестен, поэтому после расчетов нужно взять площадь по модулю. Остальные расчеты не подразумевают какой-либо сложной логики и основываются на делении с округлением в большую сторону).

\includeSolutionIfExistsByPath{final/subject_tour/inf0304_ppt/task_03}