\assignementTitle{}{25}{}

Автоматизированный школьный учитель помогает ученикам разбирать разные темы. Однако в системе случился сбой, и модуль "Умножение столбиком"\ был удалён.

Напишите программу, которая показывает выполнение умножения столбиком для двух целых положительных чисел.

\inputfmtSection

Через пробел подаются два положительных целых числа $A$ и $B$, не превышающие $10^4$.

\outputfmtSection

Все строки должны быть одинаковой длины. Если длина какой-либо строки меньше, то необходимо дописать её пробелами слева.

Сначала две строки с числами $A$ и $B$, выровнеными по правому краю. Первым символом в строке с числом $B$ должен быть "*".

Затем строка, состоящая из знаков \"-\". 

Количество символов равняется длине самой длинной строки в записи.

Затем, строки, являющиеся слагаемыми при вычислении в столбик. Шаги записываются в порядке возрастания разрядов взятых цифр числа $B$.

Строка, состоящая из \"-\". Количество символов равняется длине самой длинной строки в записи.

Строка с результатом умножения.

\sampleTitle{1}

\begin{myverbbox}[\small]{\vinput}
    2 3
\end{myverbbox}
\begin{myverbbox}[\small]{\voutput}
     2
    *3
    --
     6
    --
     6
\end{myverbbox}
\inputoutputTable

\sampleTitle{2}

\begin{myverbbox}[\small]{\vinput}
    2111 38
\end{myverbbox}
\begin{myverbbox}[\small]{\voutput}
     2111
    *  38
    -----
    16888
    6333 
    -----
    80218
\end{myverbbox}
\inputoutputTable

\solutionSection

В данной задаче сложность представляют 2 основных момента:
\begin{enumerate}
    \item Длины строк – они должны вмещать все числа и знак умножения. Для этого требуется найти произведение стандартным способом и его длину в символах.
    \item Множители и сумма. Для упрощения задачи можно не писать реальную логику умножения в столбик, а умножать лишь нужную цифру множителя на другой множитель целиком и выводить их в нужной позиции.
\end{enumerate}

\includeSolutionIfExistsByPath{final/subject_tour/inf0304_ppt/task_05}