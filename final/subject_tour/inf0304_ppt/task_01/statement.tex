\assignementTitle{}{10}{}

Дисграфия — стойкое нарушение процесса письма, обусловленное отклонениями от нормы в деятельности тех анализаторов и психических процессов, которые обеспечивают письмо. 

Далее приведены два из видов дисграфии:

\begin{itemize}
    \item пропуски согласных при их стечении (дожи-доди-> дожди) — в задаче учитываются все согласные, находящиеся рядом (могут быть пропущены сразу 2 и более из группы, например, "приветствие" -> "привесие");
    \item пропуски гласных (пошл-пшли-пшл-> пошли, тчка-точк-тчк-> точка, озис-азис-зис-> оазис).
\end{itemize}

Сколькими способами возможно допустить подобные ошибки в предложении:

\begin{center}
    \textbf{То, что сегодня наука, – завтра техника.}
\end{center}

если одновременно могут быть совершены ошибки только одного типа?

\solutionSection

Обозначим $E$ — общее возможное количесво способов допустить ошибки.
$E = E_1 + E_2$ — сумма количеств способов совершить ошибки 1 и 2 типа (согласные и гласные).

В предложении 4 группы смежных согласных: "чт"{}, "дн"{}, "втр"{}, "хн"{}. Следовательно, так как мы можем в каждом случае либо ошибиться заданным числом способов, либо не ошибиться (вариант, где мы не ошиблись нигде, не подходит), $E_1 = 3\cdot 3\cdot 7 \cdot 3 - 1 = 188$.

При этом в предложении 13 гласных. Следовательно, мы можем ошибиться в любой из них или не ошибиться (вариант, где мы не ошиблись нигде, не подходит). $E_2 = 2^{13} - 1 = 8191$

Таким образом, $E = 188 + 8191 = 8379$.

\answerMath{8379.}