\assignementTitle{}{10}{}

Дисграфия — стойкое нарушение процесса письма, обусловленное отклонениями от нормы в деятельности тех анализаторов и психических процессов, которые обеспечивают письмо. 

Далее приведены два из видов дисграфии:

\begin{itemize}
    \item пропуски согласных при их стечении (дожи-доди->дожди) — в задаче учитываются все согласные, находящиеся рядом (могут быть пропущены сразу 2 и более из группы, например, "приветствие" -> "привесие");
    \item пропуски гласных (пошл-пшли-пшл-> пошли, тчка-точк-тчк-> точка).
\end{itemize}

Сколькими способами возможно допустить подобные ошибки в предложении:

\begin{center}
    \textbf{То, что сегодня наука, – завтра техника.}
\end{center}

если одновременно могут быть совершены ошибки только одного типа?

\answerMath{8379.}