\assignementTitle{}{20}{}

Некоторые электронные инженерные системы используют логические формулы в своих расчетах. Однако с ростом количества систем формулы стали неоптимальны — некоторые переменные могут быть сокращены, так как не влияют на результат выражения.

Определите, какие переменные не влияют на результат выражения.

\inputfmtSection

В первой строке подается целое число $N$ $(1 \leq N \leq 10)$ — число переменных в выражении.

Далее подаются $2^N$ строк, содержащих $2$ значения, разделенных пробелом:
\begin{enumerate}
    \item строка из $N$ символов 0 или 1 — значений переменных, участвующих в выражении в порядке наименования от $X_0$ до $X_{N-1}$.
    \item результат выражения при заданных значениях переменных: $0$ или $1$.
\end{enumerate}

Гарантируется, что данные не противоречат друг другу, и все строки полностью покрывают возможные значения функции.

\outputfmtSection

Список переменных, которые не влияют на результат, в порядке возрастания их индексов через пробел. Если таких переменных нет, то выводить "OK".

\sampleTitle{1}

\begin{myverbbox}[\small]{\vinput}
    2
    00 0
    01 1
    10 0
    11 1
\end{myverbbox}
\begin{myverbbox}[\small]{\voutput}
    X0
\end{myverbbox}
\inputoutputTable

\sampleTitle{2}

\begin{myverbbox}[\small]{\vinput}
    2
    11 1
    00 0
    10 1
    01 1
\end{myverbbox}
\begin{myverbbox}[\small]{\voutput}
    OK
\end{myverbbox}
\inputoutputTable

%\includeSolutionIfExistsByPath{final/subject_tour/inf0304_ppt/task_02}