\assignementTitle{}{25}{}

В умном городе тестируется автоматический планировщик зданий. В задачи планировщика входит расстановка стен, огораживающих комнаты друг от друга. Изначально известны размеры прямоугольного здания — это ширина $W$ и длина $L$, а также два числа $R$ и $S$, о которых будет рассказано позже. Затем все здание делится на две комнаты согласно числу $R$, но только если каждая из двух комнат по площади больше $S$. Данная процедура повторяется для каждой из комнат до тех пор, пока никакую комнату нельзя будет разбить на две более маленьких с выполнением вышеперечисленных условий. Гарантируется, что $S \leq W \cdot L$.

$R$ представляет собой число от $0$ до $1$, означающее соотношение площади одной из разделённых комнат к площади исходной. Например, для комнаты размером $40$ соотношение $0.75$ будет означать, что её нужно будет разбить на комнаты с площадями $10$ и $30$. Требуется написать программу, которая выведет число комнат после создания плана, основываясь на числах $W$, $L$, $R$ и $S$.

\inputfmtSection

На вход программе через пробел подаются числа с плавающей точкой: $W$, $L$, $R$, $S$, ($0 \leq W$, $L$, $R$, $S \leq 1000$).
\begin{itemize}
    \item $W$ — ширина здания,
    \item $L$ — длина здания,
    \item $R$ — соотношение сторон комнат (например, соотношение $0.75$ будет означать, что нужно будет разбить комнату на комнаты с площадями 1X и 3X),
    \item $S$ — максимальная площадь комнаты, которая не может находиться в здании.
\end{itemize}

\outputfmtSection

Число комнат после работы автоматического планировщика.

\sampleTitle{1}

\begin{myverbbox}[\small]{\vinput}
    3 4 0.6666 2
\end{myverbbox}
\begin{myverbbox}[\small]{\voutput}
    3
\end{myverbbox}
\inputoutputTable

\includeSolutionIfExistsByPath{final/subject_tour/inf0304_ppt/task_04}