\assignementTitle{Определяем среду}{15}{3}

Растворы многих кислот и щелочей трудно отличить друг от друга: они бесцветны и не имеют запаха. До появления специальных приборов – pH-метров, которые определяют кислотность среды по изменению значения потенциала на электродах, химики различали кислоты и щелочи с помощью специальных химических веществ – кислотно-основных индикаторов, окраска которых зависит от pH раствора. Одним из наиболее известных таких индикаторов, который мог встречаться Вам на уроках химии в школе, является метиловый оранжевый.

\begin{enumerate}
    \item Какую окраску метиловый оранжевый имеет в щелочной среде, а какую в кислой?   
\end{enumerate}

Метиловый оранжевый можно получить взаимодействием веществ F и J, схема синтеза которых приведена ниже.
\putImgWOCaption{16cm}{1}

\begin{enumerate}
    \item[2.] Приведите структурные формулы веществ A-J и напишите названия для соединений A, B, C и D. Отметим, 
    что соединение F представляет собой  крайне неустойчивую соль, которую не выделяют и получают in 
    situ (на месте).
\end{enumerate}

