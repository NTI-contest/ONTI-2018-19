\solutionSection

\begin{enumerate}
    \item 
    Формула для расчёта pH:\\
    $рH = -lg[H^+]$, где $[H^+]$ – равновесная концентрация ионов водорода в растворе.
    \begin{enumerate}
    \item[а)] HCl – сильная кислота, следовательно, диссоциирует нацело:
    $$HCl = H^+ + Cl^-$$
    Тогда, $[H^+] = C_0(HCl) = 0.3$ М\\
    $$pH = - \lg0.3 = 0.52$$
    \item[б)] $CH_3COOH$ – слабая кислота и диссоциирует с константой 
    $$k_a = 1.8\cdot 10^{-5}: \: CH_3COOH = H^+ + CH_3COO^-$$
    $$k_a = \frac{[H^+]\cdot[CH_3COO^-]}{[CH_3COOH]} = 1.8\cdot10^{-5}$$
    Пусть $[H^+] = [CH_3COO^-] = x$, тогда $[CH_3COOH] = 0.4 - x$\\
    Поскольку константа диссоциации порядка $10^-5$, вкладом х в концентрацию непродиссоциировавшей кислоты можно пренебречь, следовательно,\\$[CH_3COOH] = 0.4$\\
    Решаем квадратное уравнение:\\
    $$1.8\cdot10^{-5} = \frac{х2}{0.4} \Rightarrow x =\sqrt{0.4\cdot1.8\cdot10^{-5}} = 0.0026$$
    Тогда $[H^+] = 0.0026$ М\\
    $$pH = - \lg0.0026 = 2.57$$
    \end{enumerate}
    \item Первая ступень:\\
    $$H_3PO_4  \Leftrightarrow H^+ + H_2PO_4$$
    Вторая ступень:\\
    $$H_2PO_4  \Leftrightarrow H^+ + HPO_4^2$$
    \item В предложенном буферном растворе:\\
    $pKa = 7.2$\\
    $C_0(A^-) = C(HPO_4^2) = 0.5$ М\\
    $C_0(HA) = C(H_2PO_4) = 1$ М\\
    Подставим в формулу:\\
    $pH = pK_a + \lg\frac{C_0(A^-)}{C_0(HA)} = 7.2 + \lg\frac{0.5}{1} = 7.2 - 0.3 = 6.9$
    \item Да, поскольку полученное значение $(pH = 6.9)$ лежит в удобном для молекулярных биологов диапазоне $pH$ $5.5 - 7$. 
    \end{enumerate}