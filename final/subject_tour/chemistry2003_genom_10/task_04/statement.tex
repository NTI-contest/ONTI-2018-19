\assignementTitle{Аминокислоты}{10}{4}

Аминокислоты – мономеры или своеобразные «кирпичики», из которых формируются белки. Знакомство с 
аминокислотами зачастую происходит раньше, чем на уроках биологии и химии в старшей школе. Многим 
из нас врачи в период олимпиад, экзаменов или весеннего авитаминоза советуют употреблять препарат, 
содержащий аминокислоту X. 

Аминокислоту Х можно получить по схеме 1:
\putImgWOCaption{7cm}{1}

\begin{enumerate}
    \item Изобразите структурные формулы соединения A и кислоты X. Как называется кислота X?
\end{enumerate} 

При растворении кислоты X в воде происходит перераспределение зарядов. В растворе образуется цвиттер-ион.
    
\begin{enumerate}    
    \item[2.] Изобразите структуру цвиттер-иона кислоты X.
    \item[3.] Рассчитайте массу кислоты Х, необходимую для приготовления двух литров раствора этой кислоты в воде с концентрацией 0,01 М.
\end{enumerate}    

Один из производителей медицинского препарата на основе кислоты Х утверждает, что его препарат состоит из кислоты Х и склеивающих веществ в соотношении 98:2. Однако, несколько независимых исследований показало, что содержание кислоты Х в препарате этого производителя составляет примерно 85\%.
    
\begin{enumerate}
    \item[4.] Рассчитайте число таблеток кислоты Х, необходимое для приготовления раствора из пункта 3. 

    Для расчетов: считайте, что вы взяли таблетки недобросовестного производителя и содержание в них кислоты Х 85\%, а масса одной таблетки составляет 100 мг.
\end{enumerate}