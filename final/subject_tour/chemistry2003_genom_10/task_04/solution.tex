\solutionSection

\begin{enumerate}
    \item \putImgWOCaption{9.5cm}{2}

    До появления номенклатуры кислоту X называли глицином или аминоуксусной кислотой.
    \item Структура цвиттер-иона глицина:\\
    \putImgWOCaption{3cm}{3}
    \item $M (\text{глицин}) = 2\cdot M(C) + M(N) + 5\cdot M(H) + 2\cdot M(O) = 12\cdot 2 +14 + 5 +16\cdot2 =$ \linebreak $=75$ г/моль.\\
    $n (\text{глицина}) = C(\text{глицина})\cdot V\text{р-ра} = 0.01\text{ моль/л}\cdot2\text{ л} = 0.02$ моль.\\
    $m = M\cdot n = 75\cdot0.02 = 1.5$ г$ = 1.5$ г глицина требуется для приготовления 2~л 0.01 М раствора.
    \item Для приготовления 100 мл 0.01 М требуется 75 мг глицина. Составим пропорцию:\\
    85\% – 1.5 г\\
    100\% – х г.\\
    Отсюда, х = 1.76 г = 1760 мг \\
    Число таблеток $n = \frac{1760}{100} = 17.6$ таблеток (18 таблеток).
    \end{enumerate}