\solutionSection

\begin{enumerate}
    \item Единственный элемент, имеющий белую и красную аллотропную модификацию – фосфор \textbf{P} (элемент \textbf{X}).
    Соответственно, веществом \textbf{А} является красный фосфор, а веществом \textbf{Б} – белый.
    В молекуле белого фосфора четыре атома P расположены в вершинах тетраэдра.
    \item $P_2O_3$ – оксид фосфора (III); $2P + 3O_{2\text{(недостаток)}} = P_2O_3$ (\textbf{реакция 1})\\
    $P_2O_5$ – оксид фосфора (V); $2P + 5O_{2(\text{избыток})} = P_2O_5$ (\textbf{реакция 2})
    \item $P_2O_5 + 3H_2O = 2H_3PO_4$ (\textbf{реакция 3})\\
    $H_3PO_4$ – кислота D
    \item Универсальным источником энергии всех биохимических процессов считается \textbf{аденозинтрифосфат} (\textbf{АТФ}).\\
    Структурная формула АТФ:\\
    \putImgWOCaption{7cm}{1}
    
\end{enumerate}