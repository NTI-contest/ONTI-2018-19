\assignementTitle{По мотивам «Марсианина»}{14}{1}

Вы приземлились на Марсе и тут ваш космический корабль сломался. Ваша ситуация намного лучше, 
чем у героя известного фильма «Марсианин» - ваша система коммуникации с Землей не нарушена и к вам 
уже выслали спасательную миссию. Однако вы, как истинный исследователь, отправились изучать планету. 
На красной планете вы обнаружили красное простое вещество A, образованное элементом Х. Вы сразу же отметили 
следующие свойства обнаруженного вами вещества:

Простое вещество А не блестит, не растворяется в воде. При нагревании вещество А возгоняется и 
образует белое вещество B.

\begin{enumerate}
    \item Какой элемент образует вещества А и B? Какой состав и строение имеют молекулы вещества В?
    \item Напишите формулы всех известных вам оксидов элемента Х, назовите эти оксиды по номенлатуре, а также напишите уравнения реакции их образования из простых веществ.
\end{enumerate}

Известно, что высший оксид элемента Х образует кислоту средней силы D. Кислотные остатки D встречаются в структуре многих биологических молекул. 

\begin{enumerate}
    \item[3.] Напишите уравнение реакции образования кислоты D из оксида.
    \item[4.] Изобразите структурную формулу молекулы – универсального источника энергии для всех биохимических процессов, в состав которой входит кислотный остаток D.
\end{enumerate}