\assignementTitle{Царство}{100}

В тридевятом царстве $n$ городов и $n - 1$ дорог, по которым можно из любого города добраться в любой другой (возможно, проезжая через другие города). Города пронумерованы от $1$ до $n$, а дороги -- от $1$ до $n - 1$. Все дороги с двусторонним движением.

Автомобилисты царства последнее время сильно озабочены ценами на бензин. Они называют пару городов $(i, j)$ странной, если из города $i$ можно добраться до города $j$ и стоимость бензина в городе $i$ больше, чем в городе $j$.

А царь озабочен народными волнениями и хочет перекрыть все дороги. На время. А также хочет всегда знать степень недовольства своих подданных. Одним из факторов, влияющих на степень недовольства, является количество пар странных городов.

Ваша задача -- определить количество пар странных городов до перекрытия дорог и после каждого очередного перекрытия дороги. Не справитесь - не сносить вам головы!

\inputfmtSection

В первой строке вводится целое число $n$ ($2 \le n \le 3 \cdot 10^5$).

Во второй строке вводится $n$ целых чисел $c_i$ ($1 \le c_i \le 10^9$) - стоимость бензина в $i$-ом городе.

В следующих $n - 1$ строках содержатся описания дорог: пары чисел $a_i$ и $b_i$ ($1 \le a_i, b_i \le n$), говорящие о том, что $i$-ая дорога проложена между городами $a_i$ и $b_i$.

В последней строке задается порядок перекрытия дорог, то есть $n - 1$ целых чисел $q_i$ ($1 \le q_i \le n-1$) - номер дороги, перекрытой $i$-ой по счёту.

Гарантируется, что до перекрытия дорог в царстве можно было из любого города добраться в любой другой, а в конечном итоге все дороги перекрыли.

\outputfmtSection

Выведите в одной строке $n$ целых чисел - количество пар странных городов до перекрытия дорог и после каждого очередного перекрытия дороги.

Заметьте, что ответ может не помещаться в 32-битный тип данных.

\exampleSection

\sampleTitle{1}

\begin{myverbbox}[\small]{\vinput}
3
1 2 3
1 2
2 3
1 2
\end{myverbbox}
\begin{myverbbox}[\small]{\voutput}
3 1 0
\end{myverbbox}
\inputoutputTable

\sampleTitle{2}

\begin{myverbbox}[\small]{\vinput}
5
1 2 3 1 2
1 2
2 3
3 4
4 5
2 3 1 4
\end{myverbbox}
\begin{myverbbox}[\small]{\voutput}
8 4 2 1 0
\end{myverbbox}
\inputoutputTable

\includeGradingIfExistsByPath{final/subject_tour/inf0903_irs/task_03}

\includeSolutionIfExistsByPath{final/subject_tour/inf0903_irs/task_03}