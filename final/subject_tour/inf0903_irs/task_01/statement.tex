\assignementTitle{Игра}{100}

Как-то Ильнар и Азат придумали игру для двух человек. На доске пишется число $n$, после чего игроки делают свои ходы.

Ходят по очереди. Первым ходит Ильнар. За один ход игрок должен заменить написанное на доске число (обозначим его $m$) на число $m-x$, где $x$ является степенью числа $2$, то есть $x = 2^k$ для некоторого целого неотрицательного числа $k$, и выполняется условие $1 \le x \le m$. Игрок, который не может сделать ход, проигрывает.

Определите, кто выиграет при оптимальной игре.

\inputfmtSection

В первой строке вводится одно целое число $n$ ($0 \le n \le 10^9$) — начальное число.

\outputfmtSection

Выведите \textit{Ilnar won!}, если выиграет Ильнар, или \textit{Azat won!}, если выиграет Азат.

\exampleSection

\sampleTitle{1}

\begin{myverbbox}[\small]{\vinput}
2
\end{myverbbox}
\begin{myverbbox}[\small]{\voutput}
Ilnar won!
\end{myverbbox}
\inputoutputTable

\sampleTitle{2}

\begin{myverbbox}[\small]{\vinput}
0
\end{myverbbox}
\begin{myverbbox}[\small]{\voutput}
Azat won!
\end{myverbbox}
\inputoutputTable

\includeGradingIfExistsByPath{final/subject_tour/inf0903_irs/task_01}

\includeSolutionIfExistsByPath{final/subject_tour/inf0903_irs/task_01}
