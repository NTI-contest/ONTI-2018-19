\assignementTitle{Бактерии}{100}

На планете Нирокку существует $n$ различных видов бактерий. Виды бактерий пронумерованы от $1$ до $n$. В лаборатории "Кадзи"\ хранятся экземпляры некоторых из этих видов.

В лаборатории существует $n$ колоний бактерий, в $i$-ой из которых находятся все имеющиеся в лаборатории бактерии $i$-го вида. Некоторые колонии могут пустовать. Также стоит отметить, что в "Кадзи"\ поддерживают условия для существования, но не размножения бактерий.

Вечером в лабораторию приезжает министр экспериментов, который хочет увидеть коллекцию бактерий. Учёные лаборатории решили угодить ему: они хотят, чтобы суммарное количество бактерий стало равно любимому числу министра. Для этого учёные воспользуются раствором ускорения роста численности бактерий в колонии.

Раствор можно применять только к тем колониям, количество бактерий в которых не менее двух и ранее к ним не применяли данный раствор. В результате действия раствора количество бактерий в колонии увеличится в $k$ раз. Число $k$ может быть любым целым числом, большим $1$, и выбирается заново перед каждым применением раствора, то есть оно может меняться.

До вечера остаётся мало времени, и потому учёные успеют применить раствор не более чем к одной колонии. Также учёные потеряли и не могут найти информацию о том, к каким колониям уже применялся раствор. В случае повторного применения раствора к некоторой колонии происходит взрыв, последствия которого до приезда министра устранить не является возможным.

Помогите учёным определить, можно ли угодить министру, не рискуя при этом взорвать лабораторию.

\inputfmtSection

В первой строке два целых числа:

$1 \le n \le 10^5$ -- количество различных видов бактерий на планете Нирокку;

$0 \le m \le 10^{18}$ -- любимое число министра экспериментов.

Во второй строке $n$ целых чисел:

$0 \le a_{i} \le 10^6$ -- количество бактерий $i$-го $(1 \le i \le n)$ вида в лаборатории "Кадзи".

\outputfmtSection

В случае если угодить министру невозможно, выведите строку "No"\ (без кавычек).

В случае отсутствия необходимости применять раствор, выведите строку "Yes"\ (без кавычек).

В остальных случаях выведите строку "Yes i k"\ (без кавычек), где $i$ -- номер колонии, к которой учёным можно и нужно применить раствор, а $k$ -- количество раз, в которое необходимо увеличить численность $i$-ой колонии.

Если правильных ответов несколько, выведите любой из них.

\exampleSection

\sampleTitle{1}

\begin{myverbbox}[\small]{\vinput}
5 5
0 0 1 2 0
\end{myverbbox}
\begin{myverbbox}[\small]{\voutput}
Yes 4 2
\end{myverbbox}
\inputoutputTable

\sampleTitle{2}

\begin{myverbbox}[\small]{\vinput}
2 7
3 3
\end{myverbbox}
\begin{myverbbox}[\small]{\voutput}
No
\end{myverbbox}
\inputoutputTable

\sampleTitle{3}

\begin{myverbbox}[\small]{\vinput}
4 8
0 3 2 3
\end{myverbbox}
\begin{myverbbox}[\small]{\voutput}
Yes
\end{myverbbox}
\inputoutputTable

\includeGradingIfExistsByPath{final/subject_tour/inf0903_irs/task_02}

\includeSolutionIfExistsByPath{final/subject_tour/inf0903_irs/task_02}
