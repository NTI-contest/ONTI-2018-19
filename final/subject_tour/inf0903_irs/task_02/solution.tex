\solutionSection

Запишем в переменную $sum$ текущее суммарное количество бактерий.

Посмотрим на значение $m - sum$.

Если оно отрицательно, то ответ \textit{No}, так как мы не можем уменьшить
суммарное количество бактерий.

Если равно нулю, то очевидно, что ответ – \textit{Yes}.

Подробнее рассмотрим случай, когда значение $m - sum$ положительно.

Из условия понятно, что если число $a_i$ удовлетворяет некоторому условию,
то мы можем умножить его на любое целое число, которое больше $1$.

Что это за условие?

Не должно существовать числа, которое можно умножить на целое число $k$
$(k > 1)$ и таким образом получить число $a_i$.

Не трудно заметить, что число $a_i$ должно быть простым
(не иметь делителей кроме $1$ и самого себя),
иначе возможно, что к $i$-ой колонии уже применяли раствор.

Таким образом нужно определить, существует ли такое простое $a_i$, что
$sum - a_i + a_i \cdot k = m$ , где число $k$ – целое и больше $1$.

Проверку на простоту можно сделать с помощью, например, решета Эратосфена,
поскольку $a_i \le 10^6$ для всех $i$.

Асимптотика: $O(n + max(a_i) \cdot \log (\log (max(a_i))))$

\codeExample

\inputPythonSource
%\inputJavaSource
%\inputCPPSource