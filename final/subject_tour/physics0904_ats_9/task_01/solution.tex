\solutionSection

\begin{enumerate}
    \item Так как автомобиль прошел поворот с максимально возможной скоростью, сила трения покоя действующая на его колеса (вдоль радиуса) также была максимальна. Следовательно, во втором случае она будет такой же.
    Запишем второй закон Ньютона для первого и второго случаев:
    $$ \left\{
        \begin{aligned}
        \frac{mV_0^2}{R_0} =\mu gm\\
        \frac{mV_1^2}{R_1} =\mu gm\\
        \end{aligned}
    \right. \rightarrow \frac{V_0^2}{V_1^2}=\frac{R_0}{R_1} \rightarrow V_1 = \sqrt{\frac{3R_0}{R_0}} V_0= \sqrt{3} V_0=62 \: \text{км/ч}$$

    \markSection

    \begin{itemize}
        \item Указано (или использовано в решении) что сила трения будет максимальной и одинаковой в обоих случаях – 3 балла.
        \item Записаны выражения второго закона Ньютона или аналоги – 3 балла
        \item Получено выражение для скорости – 2 балла
        \item Получен правильный числовой ответ – 2 балла.
    \end{itemize}

    \item[2.] Задача аналогична предыдущей, но необходимо учесть вклад наклона в изменение ускорения. Вдоль радиуса поворота будет действовать проекция силы реакции опоры кроме силы трения, кроме того, из-за наклона изменится и сама сила трения. Запишем соответствующие уравнения для второго закона Ньютона:
    $$ \left\{
        \begin{aligned}
            \frac{mV_0^2}{R_0} =  \mu gm\\
            \frac{mV_1^2}{R_0} = \mu mg cos^2  \alpha+N sin \alpha= \mu mg cos^2 \alpha+mg cos \alpha sin\alpha
        \end{aligned}
    \right. $$
    Откуда:
    $$\frac{V_1^2}{V_0^2}=\frac{2 \mu  cos^2 \alpha+sin2 \alpha}{2 \mu}$$ 
    $$V_1=V_0 \sqrt{cos^2 \alpha+\frac{sin2\alpha}{2 \mu}} =44.7 \: \text{км/ч}$$
    \answerMath{44.7 км/ч}   

    \markSection

    \begin{itemize}
        \item Записаны выражения второго закона Ньютона или аналоги – 5 баллов
        \item Получено выражение для скорости – 3 балла
        \item Получен правильный числовой ответ – 2 балла.
    \end{itemize}

    \item[3.] 	Запишем закон сохранения энергии для случая до начала подъема:
    $$\frac{mV_0^2}{2}-mgl_1=\frac{mV_1^2}{2}$$
    Отсюда конечная скорость: $V_1=\sqrt{V_0^2-2mgl_1}$

    Учтем, что при ударе в начале подъема сохранится только проекция импульса вдоль плоскости подъема, следовательно: $V_2 = V_1 \cdot cos\alpha $

    Запишем закон сохранения энергии после начала подъема:
    $$\frac{mV_2^2}{2}=\mu mg cos\alpha  l_2+mgl_2  sin\alpha =gl_2 (sin\alpha +\mu  cos\alpha  )$$
    $$(V_0^2-2\mu gl_1 )  cos^2\alpha =2gl_2 (sin\alpha +\mu  cos\alpha )$$
    Откуда:
    $$\mu =\frac{V_0^2  cos^2\alpha -2gl_2  sin\alpha}{2g(l_1  cos^2\alpha +l_2  cos\alpha )}=0.64$$
    \answerMath{Ответ: 0.64}    

    \markSection

    \begin{itemize}
        \item Получена скорость перед подъемом – 2 балла
        \item Получена скорость после удара в начале подъема – 2 балла
        \item Записан закон сохранения энергии во время подъема - 2 балла
        \item Получено выражение для коэффициента трения – 2 балла
        \item Получен правильный числовой ответ – 2 балла.
    \end{itemize}

    \item[4.] Т.к. коэффициент трения больше тангенса угла наклона поверхности после остановки автомобиль останется стоять на тормозе.

    Соответственно, единственное необходимое условие для выполнения задачи – автомобиль должен суметь доехать до начала подъема. Тогда необходимая начальная скорость находится из закона сохранения энергии:
    $$\frac{mV_3^2}{2}= \mu gl_1$$
    $$V_3 \geq \sqrt{2gl_1}=11.18 \: \text{м/с}$$
    \answerMath{11.18 м/с.}

    \markSection

    \begin{itemize}
        \item Указано, что независимо от начальной скорости автомобиль остановится на наклонной плоскости – 2 балла
        \item Указано, что при этом до наклонной плоскости еще надо доехать – 1 балла
        \item Записан закон сохранения энергии - 2 балла
        \item Получено выражение для скорости в виде неравенства – 3 балла (в случае равенства – 1 балл)
        \item Получен правильный числовой ответ – 2 балла.
    \end{itemize}

    \item[5.] 	Для того, чтобы была возможна такая ситуация с одной стороны трение сила трения должна быть достаточно большой, чтобы не было проскальзывания, а с другой стороны – вращающий момент должен опрокидывать машину.
    
    \putImgWOCaption{7cm}{2}

    Из первого условия ограничение на скорость: 
    $$ma \geq  \mu mg$$
    $$V \leq \sqrt{\mu gR}=320 \: \text{км/ч}$$
    
    Для того, чтобы оценить второе условие:
    $mg \cdot DE \geq ma \cdot BD$ (относительно оси поворота, проходящей через землю),
    найдем плечи силы тяжести и «силы» инерции.
    $$AC = H \cdot tg \alpha $$
    $$BC = H/cos \alpha $$
    $$CD = (a/2 – H \cdot tg \alpha ) \cdot sin \alpha $$
    $$BD = BC + CD = H/cos \alpha  + (a/2 – H \cdot tg \alpha ) \cdot sin \alpha $$
    $$DE = (a/2 – H \cdot tg \alpha ) \cdot cos \alpha $$
    
    Тогда ограничение на скорость при которой еще не возникает переворот:
    $$V \leq \sqrt{\frac{gR(a/2-H tan \alpha )cos \alpha}{(a/2-H tan \alpha )sin \alpha +H/cos \alpha}}=34.6 \: \text{км/ч}$$
    
    \answerMath{$34.6 \: \text{км/ч} < V \leq 320 \: \text{км/ч}$.}

    \markSection

    \begin{itemize}
        \item Найдено условие не проскальзывания – 2 балла
        \item Найдено условие опрокидывания – 4 балла
        \item Получено нижнее численное ограничение на скорость - 2 балла
        \item Получено верхнее численное ограничение на скорость – 2 балла
    \end{itemize}
\end{enumerate}