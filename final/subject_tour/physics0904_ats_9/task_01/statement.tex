\assignementTitle{Тормози на повороте}{50}{1}

\begin{enumerate}
    \item Автомобиль прошел поворот, представляющий собой часть окружности, с максимальной скоростью, при которой не возникает проскальзывания колес $V_0 = 36$ км/ч. С какой скоростью (если пренебречь опрокидыванием) сможет пройти автомобиль поворот с втрое большим радиусом, так чтобы проскальзывания не возникло? Поверхность дороги горизонтальная.
    \item С какой скоростью сможет пройти автомобиль этот поворот (из условия п.1), если на всем протяжении поворота дрога будет иметь наклон на угол $\alpha = 15^{\circ}$, так как показано на рисунке (машина на рисунке движется на наблюдателя).  Коэффициент трения между колесами автомобиля и дрогой равен $\mu =0.3$.
    
    \putImgWOCaption{5cm}{1}

    \item Автомобиль, двигавшийся по прямой со скоростью $V_0 = 72$ км/ч начал резкое торможение с максимально возможной силой трения на расстоянии $l_1 = 10$ м до подъема. Угол подъема $\alpha=30^{\circ}$. Известно, что до того момента, как скорость автомобиля стала равной нулю, он проехал такое же расстояние $l_2 = 10$ м вдоль подъема, продолжая тормозить. Каков коэффициент трения между дорогой и колесами? Считайте, что во время торможения колеса автомобиля не прокручиваются, коэффициент трения между колесами и дорогой до и во время подъема одинаковый, силой сопротивления воздуха пренебрегите. Ускорение свободного падения $g = 9.8$ м/с$^2$. Обратите внимание, что при начале подъема автомобиль испытывает удар.
    \item При какой начальной скорости автомобиль из пункта 3. сможет остановиться этой на наклонной плоскости и не съезжать вниз (оставаясь на тормозе)? Считайте, что наклонная плоскость сколь угодно длинная.
    \item Расстояние между колесами автомобиля на одной оси $a = 1.9$ м, высота, на которой расположен центр масс автомобиля, $h = 0.9$ м. Центр масс расположен точно посередине между колесами. Коэффициент трения между дорогой и колесами $\mu = 0.4$. Известно, что поворачивая с радиусом поворота $R = 45$ м автомобиль накренился на $30^{\circ}$ (с вертикалью) и затем перевернулся, колеса автомобиля при этом не проскальзывали по дороге. С какой скоростью мог двигаться автомобиль? Ускорение свободного падения $g = 9.8$ м/c$^2$.
\end{enumerate}