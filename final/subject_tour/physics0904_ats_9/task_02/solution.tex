\solutionSection

\begin{enumerate}
    \item Найдем массу фигурки, как массу вытекшего из сопла пластика:
    $$m =  \rho  \cdot v = \rho  \cdot S \cdot v \cdot \tau$$
    $$m= 1250 \: \text{кг/м}^3 \cdot 3.14 \cdot \frac{(0.4 \cdot 10^{-3})^2}{4} \cdot \frac{7.2}{60} \cdot 40 \cdot 60 \: \text{кг}=0.045 \: \text{кг}=45 \: \text{г}$$
    \answerMath{45 г.}

    \markSection
    \begin{itemize}
        \item Получено выражение для массы фигурки – 5 баллов
        \item Получен верный численный ответ – 5 баллов
    \end{itemize}
     
    \item Из уравнения теплового баланса:
    $$\nu =\frac{Cyg \cdot m \cdot \Delta t}{I \cdot U \cdot \tau}=\frac{Cyg \cdot \rho  \cdot S \cdot v \cdot \tau \cdot \Delta t}{I \cdot U \cdot \tau}=\frac{Cyg \cdot \rho  \cdot S \cdot v \cdot \Delta t}{I \cdot U}=0.138=13.8\%$$
    \answerMath{13.8\%}

    \markSection
    \begin{itemize}
        \item Получено выражение количество тепла нужного для нагрева материала – 4 балла
        \item Получено выражение для количества тепла, выделившегося в проводах – 4 балла
        \item Получен верный численный ответ – 2 баллов
    \end{itemize}
    
    \item Запишем число необходимых линий, напечатанных принтером (+1 возникает из- за того, что линии нужны с обеих сторон).
    
    $$K=N(\frac{20}{2}+1)=110$$
    $$M=N(\frac{10}{2}+1)=60$$
    
    Общая длина линий напечатанных принтером:
    $$L = 110 \cdot 10 \: \text{см} +60 \cdot 20 \: \text{см} = (1100 + 1200) \: \text{см} = 2300 \: \text{см}$$

    Чистое время печати:
    $$\tau = 2300 \: \text{см}/720 \: \text{см/мин} = 3.2 \: \text{мин} = 191.7 \: \text{с}$$

    Выделившееся тепло:
    $$Q = I \cdot U \cdot \tau = 4 \cdot 8 \cdot 191.7 = 6.13 кДж$$

    \answerMath{6.13 кДж.}

    \markSection
    
    \begin{itemize}
        \item Найдено число линий в сетках  – 2 балла (если число линий без крайних – 1 балл)
        \item Найдено время печати – 4 балла
        \item Получено выражение для электрической работы – 2 балла
        \item Получен верный численный ответ – 2 баллов 
    \end{itemize}

    \item Тепловой поток, отводящий тепло, пропорционален площади поверхности, которая в свою очередь пропорциональна квадрату расстояний $\Phi \approx S \approx =l_2$
    
    Количество тепла, запасенного в материале пропорционально теплоёмкости материала, которая в свою очередь пропорциональна масса, а следовательно объему и кубу расстояний 

    $$Q \approx C \approx m \approx V \approx l_3$$
    
    Тогда, время на отвод тепла $$\tau \approx \frac{Q}{\Phi} \rightarrow \tau \sim  l^3/l^2 \sim l \rightarrow \tau$$ увеличится 2 раза.

    \answerMath{$\tau$ увеличится в 2 раза.}

    \markSection
    
    \begin{itemize}
        \item Указано, что за отвод тепла отвечает площадь поверхности – 2 балла 
        \item Указано, что запасенное тепло пропорционально объему тела – 2 балла
        \item Сделана оценка для времени охлаждения  – 2 балла
        \item Получен верный численный ответ – 2 баллов 
    \end{itemize}

    \item Для того, чтобы снизить напряжение необходимо подключить резистор последовательно. Тогда на нем б будет напряжение U = 4~В, следовательно: $P = 4 \: \text{В} \cdot 4 \: \text{А} = 16 \: \text{Вт}$ (учитывая, что при последовательном соединении ток через резистор будет таким же, как и на установке, а там он стал номинальным, раз установка работает. Рассеиваемое тепло:
    $$T=\sqrt[4]{\frac{UI}{\sigma S}}=323 \: \text{К}=50^\circ C$$
    \answerMath{$T=\sqrt[4]{\frac{UI}{\sigma} S}=50^\circ C$.}

    \markSection
    
    \begin{itemize}
        \item Указано, что резистор необходимо подключить последовательно  – 2 балла 
        \item Найдена выделяющаяся на резисторе мощность – 2 балла
        \item Получено выражение для температуры – 4 балла
        \item Получен верный численный ответ – 2 баллов 
    \end{itemize}
\end{enumerate}
