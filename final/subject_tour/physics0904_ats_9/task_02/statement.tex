\assignementTitle{3D-принтер}{50}{2}

Диаметр сопла 3D – принтера $d = 0.4$ мм, скорость струи при печати $V = 720$ см/мин; параметры пластика, который используется для печати следующие: плотность $1.25$ г/см$^3$, удельная теплоемкость $1.8$ кДж/(кг$\cdot^{\circ}C$), температура стеклования (размягчения) – $50^{\circ}C$, температура в вязкой фазе, которая используется при печати – $150^{\circ}C$. Температура в комнате $20^{\circ}C$.

\begin{enumerate}
    \item Какова масса пластиковой фигурки, на печать которой ушло 40 минут?
    \item При печати ток через нагревательный элемент принтера составил $4А$, при напряжении $8В$. Какая доля энергии расходуется при этом на нагрев пластика для печати? Считайте, что пластик до нагрева долго находился в комнате.
    \item С помощью принтера изготовили сетку со сторонами $a= 20$ см и $b = 10$ см в $N = 10$ слоев. Ячейка сетки представляет собой квадрат со стороной $L = 2$ см. Сколько электроэнергии было потрачено на нагрев пластика? Параметры нагревателя такие же, как и в пункте 2.
    \item Во сколько раз увеличится время застывания напечатанного кольца, если при печати использовать сопло с вдвое большим диаметром, а все остальные параметры оставить теми же?
    \item Нагревательный элемент может работать при напряжении на нем в $8В$ и токе $4А$, однако, источник постоянного напряжения подключенный к нему выдает $12В$. Для того, чтобы нагревательный элемент смог работать к нему подключили дополнительный резистор. До какой температуры нагреется резистор, если для рассеяния тепла к нему прикреплен радиатор с эффективной площадью $260$ см$^2$? Учитывайте только потери тепла через радиатор за счет излучения. Известно, что нагретые тела излучают энергию, причем мощность этого излучения подчиняется закону Стефана-Больцмана: $W=\sigma T^4S$,  где $T$ – температура излучающей поверхности в градусах Кельвина, $\sigma=5.67\cdot10^{-8}$ Вт/(м$^2\cdot$К$^4$) – постоянная Стефана-Больцмана, $S$ – площадь излучающей поверхности.
\end{enumerate}