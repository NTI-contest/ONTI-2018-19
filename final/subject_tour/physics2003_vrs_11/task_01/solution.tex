\solutionSection

\begin{enumerate}
    \item Орбитальная скорость спутника $V_\text{орб}$
    $$V_\text{орб}=\sqrt{\frac{G \cdot M_\text{зем}}{R_\text{зем}+H}}$$
    Тогда скорость подспутниковой точки с учётом круглости Земли можно определить как:
    $$V_\text{пст} = V_\text{орб} \cdot \frac{R_\text{зем}}{R_\text{зем}+H} = \frac{R_\text{зем}}{R_\text{зем}+H} \cdot \sqrt{\frac{G \cdot M_\text{зем}}{R_\text{зем}+H}}=\frac{6370 \cdot 10^3}{6870 \cdot 10^3} \cdot \sqrt{((6.67 \cdot 10^{-11} \cdot 5,97 \cdot 〖10〗^24)/(6870〖10〗^3 ))}=7059,18  м/с$$
    
\end{enumerate}