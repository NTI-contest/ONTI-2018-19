\assignementTitle{Задача 5}{25}{}

Разработанный беспилотный дрон умеет выполнять следующие комманды:

MOV X — передвижение в точку $X_0 + X$, где $X_0$ это координата дрона до выполнения операции,
REP N — повторить следующий блок команд $N$ раз,
BLB — начало блока,
BLE — конец блока,
END — конец программы.

Дрону передаются комманды, которые он должен выполнить. Изначально дрон находится в точке $0$.

Требуется определить координату дрона после выполнения всех команд.

\inputfmtSection

На вход поступает неопределённое число строк с коммандами.

Гарантируется, что:

\begin{itemize}
\item Только одна комманда в строке.
\item В программе нет ошибок (например, отсутствие BLE на каждый BLB)
\item Последняя комманда в программе — END.
\item Координата дрона в любой момент выполнения программы не выйдет за пределы $(-10^9, 10^9)$.
\item Конструкция BLB ... BLE всегда идёт сразу за коммандой REP N.
\end{itemize}

Для любой комманды MOV X верно $(-10^9 \leq X \leq 10^9)$, где $X$ — целое.

Для любой комманды REP N верно  $(0 \leq N \leq 10^9)$, где $N$ — целое.


\outputfmtSection

Одно число — координата дрона на момент завершения программы.

\sampleTitle{1}

\begin{myverbbox}[\small]{\vinput}
    MOV 2 
    REP 3 
    BLB 
    MOV 1 
    BLE
    END
\end{myverbbox}

\begin{myverbbox}[\small]{\voutput}
    5
\end{myverbbox}
\inputoutputTable

\solutionSection

Для решения задачи при заданных ограничениях можно использовать рекурсию.

\includeSolutionIfExistsByPath{final/subject_tour/inf1303_bas/task_05}