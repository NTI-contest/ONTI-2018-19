\assignementTitle{}{10}{}

Алгоритм автономного управления БПЛА измеряет в процессе полета 3 значения $\alpha$, $\beta$ и $\gamma$. Все 3 показателя вещественного типа с 2 знаками после запятой. $$-1.00\leq \alpha\leq 1.00, 0.00\leq\beta\leq 3.00 \: \text{и} \: 1.00\leq\gamma\leq10.00.$$

Вы знаете, что параметр $\alpha$ может за секунду измениться в любую из сторон не более чем на $0.07$, $\beta$ — $0.01$, $\gamma$ — $0.11$.

Зная начальные значения параметров (всегда постоянные), определите наименьшее количество бит, которое необходимо затратить на хранение информации об изменении параметров в течение 1 минуты, если изменение каждого из показателей кодируется наименьшим количеством бит?

\solutionSection

Изменения параметров $\alpha$, $\beta$, $\gamma$ принимают 15, 3 и 23 различных значения соответственно. Таким образом, для их кодирования необходимо 4 бита, 2 бита и 5 битов соответственно. В сумме за 1 секунду тратится 11 битов. За 60 секунд — 660 битов. 

\answerMath{660}