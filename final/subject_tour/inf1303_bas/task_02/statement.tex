\assignementTitle{Задача 2}{20}{}

На стене висят умные часы со светодиодным табло. Они показывают в формате ЧЧММ от $0000$ до $2359$. С целью экономии энергии при смене времени они переключают минимальное число диодов.

Цифры представляются следующим образом:



..... ..... ..... ..... ..... ..... ..... ..... ..... .....\\
.\#\#\#. ...\#. .\#\#\#. .\#\#\#. .\#.\#. .\#\#\#. .\#\#\#. .\#\#\#. .\#\#\#. .\#\#\#.\\
.\#.\#. ...\#. ...\#. ...\#. .\#.\#. .\#... .\#... ...\#. .\#.\#. .\#.\#.\\
.\#.\#. ...\#. .\#\#\#. .\#\#\#. .\#\#\#. .\#\#\#. .\#\#\#. ...\#. .\#\#\#. .\#\#\#.\\
.\#.\#. ...\#. .\#... ...\#. ...\#. ...\#. .\#.\#. ...\#. .\#.\#. ...\#.\\
.\#\#\#. ...\#. .\#\#\#. .\#\#\#. ...\#. .\#\#\#. .\#\#\#. ...\#. .\#\#\#. .\#\#\#.\\
..... ..... ..... ..... ..... ..... ..... ..... ..... .....

Здесь \# показаны горящие светодиоды, а . - погашеные.

Например, исходное время 20:19 будет выглядеть на табло как

...................
.\#\#\#.\#\#\#.....\#.\#\#\#.
...\#.\#.\#.\#...\#.\#.\#.
.\#\#\#.\#.\#.....\#.\#\#\#.
.\#...\#.\#.\#...\#...\#.
.\#\#\#.\#\#\#.....\#.\#\#\#.
...................

Через минуту оно сменится на  

...................
.\#\#\#.\#\#\#...\#\#\#.\#\#\#.
...\#.\#.\#.\#...\#.\#.\#.
.\#\#\#.\#.\#...\#\#\#.\#.\#.
.\#...\#.\#.\#.\#...\#.\#.
.\#\#\#.\#\#\#...\#\#\#.\#\#\#.
...................


В данном случае минимальным числом диодов для переключения является $10$.

Найдите минимальное число для перехода из текущего времени во время, которое будет через минуту.

\inputfmtSection

Целое $4$-значное число $N$ с ведущими нулями, обозначающее текущее время в формате ЧЧММ.

\outputfmtSection

Целое число, являющееся минимальным числом диодов, состояние которых необходимо для изменения времени на $1$ минуту вперед.

\markSection

Баллы за задачу будут начисляться пропорционально количеству успешно пройденных тестов.

\sampleTitle{1}

\begin{myverbbox}[\small]{\vinput}
    2019
\end{myverbbox}
\begin{myverbbox}[\small]{\voutput}
    10
\end{myverbbox}
\inputoutputTable

\solutionSection

Сначала необходимо вычислить время (часы и минуты), которое будет через минуту от заданного. Далее необходимо сравнить соответствующие цифры циферблата для исходного и измененного времени. Количество несовпадений нужно аккумулировать в переменную, которая и будет ответом.

\includeSolutionIfExistsByPath{final/subject_tour/inf1303_bas/task_02}
