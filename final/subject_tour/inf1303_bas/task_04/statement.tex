\assignementTitle{Задача 4}{25}{}

Беспилотные транспортные средства в умном городе умеют общаться между собой с помощью зашифрованных сообщений. Для этого они выбирают ключ — такое число $A$, являющееся произведением двух простых чисел.

Напомним, что простыми числами называют натуральные числа больше $1$, которые делятся только сами на себя и на $1$.

Требуется написать программу, которая будет определять, подходят ли числа под данные требования.

\inputfmtSection

Число $N\space (1 \leq N \leq 10^4)$ — количество чисел-кандидатов.

Далее следует $N$ строк, в каждой содержится одно целое число\\ $ M\space (2 \leq M < 10^9)$ — возможный ключ.

\outputfmtSection

$N$ строк, которые содержат YES, если $N$-ое число-кандидат подходит по требованиям и NO в противном случае.

\markSection

Баллы за задачу будут начисляться пропорционально количеству успешно пройденных тестов.

\sampleTitle{1}

\begin{myverbbox}[\small]{\vinput}
    1
    6
\end{myverbbox}
\begin{myverbbox}[\small]{\voutput}
    YES
\end{myverbbox}
\inputoutputTable

\sampleTitle{2}

\begin{myverbbox}[\small]{\vinput}
    2
    7
    10
\end{myverbbox}
\begin{myverbbox}[\small]{\voutput}
    NO
    YES
\end{myverbbox}
\inputoutputTable

\solutionSection

Для решения задачи необходимо подсчитать для каждого возможного ключа количество множителей. При заданных ограничениях подсчёт можно осуществлять перебором делителей до $\sqrt{M}$.

\includeSolutionIfExistsByPath{final/subject_tour/inf1303_bas/task_04}