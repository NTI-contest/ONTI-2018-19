\assignementTitle{}{25}{}

Беспилотные транспортные средства в умном городе умеют общаться между собой с помощью зашифрованных сообщений. Для этого они выбирают ключ — такое число $A$, являющееся произведением двух простых чисел.

Напомним, что простыми числами называют натуральные числа больше $1$, которые делятся только сами на себя и на $1$.

Требуется написать программу, которая будет определять, подходят ли числа под данные требования.

\inputfmtSection

Число $N\space (1 \leq N \leq 10^4)$ — количество чисел-кандидатов.

Далее следует $N$ строк, в каждой содержится одно целое число\\ $ M\space (2 \leq M < 10^9)$ — возможный ключ.

\outputfmtSection

$N$ строк, которые содержат YES, если $N$-ое число-кандидат подходит по требованиям и NO в противном случае.

\markSection

Баллы за задачу будут начисляться пропорционально количеству успешно пройденных тестов.

\sampleTitle{1}

\begin{myverbbox}[\small]{\vinput}
    1
    6
\end{myverbbox}
\begin{myverbbox}[\small]{\voutput}
    YES
\end{myverbbox}
\inputoutputTable

\sampleTitle{2}

\begin{myverbbox}[\small]{\vinput}
    2
    7
    10
\end{myverbbox}
\begin{myverbbox}[\small]{\voutput}
    NO
    YES
\end{myverbbox}
\inputoutputTable

\solutionSection

Данная задача может быть решена с помощью 2 программ. 

Первая программа будет выполнятся на компьютере участника 1 раз и писать в консоль все простые числа до некоторого числа, большего чем квадратный корень из максимально возможного числа, так как максимальным делителем может быть корень из самого числа. Данная программа может использовать любой алгоритм нахождения простых чисел.

Полученный текст копируется во вторую программу в массив простых чисел, и внутри нее во всех числах ищутся простые делители перебором по массиву.

Следует отметить, что простые делители могут быть одинаковыми.

\includeSolutionIfExistsByPath{final/subject_tour/inf1303_bas/task_04}