\assignementTitle{}{20}{}

На современной автостоянке установили Шлагбаум Универскально Распознающий Автомобили (Ш.У.Р.А.), который пропускает только те машины, которые есть у него в базе данных. Однако машин настолько много, что Ш.У.Р.А. нуждается в помощи. Напишите программу, которая будет отвечать на запросы, содержащие номера автомобилей и выносить вердикт, можно ли пропустить данный автомобиль или нет.

\inputfmtSection

В первой строке записаны 2 числа:
$N\space (1 \leq N \leq 10^5)$ — число номеров автомобилей в базе данных.
$M\space (0 \leq M \leq 10^5)$ — число запросов к базе.

Следующая строка содержит $N$ слов в формате LLNNNLRR, где $L$ — буквы латинского алфавита, а $N$ и $R$ — цифры от $0$ до $9$. Эта строка содержит список всех занесенных в базу данных номеров автомобилей.

Далее идут $M$ строк, содержащих номер в том же формате, что и в базе данных. Это номера автомобилей, которые хотят проехать через шлагбаум.

\outputfmtSection

$M$ строк, содержащих ответы на соответствующие запросы. Если в запросе указан номер, существующий в базе данных, то выводить YES, иначе NO.

\markSection

Баллы за задачу будут начисляться пропорционально количеству успешно пройденных тестов.

\sampleTitle{1}

\begin{myverbbox}[\small]{\vinput}
    3 4
    DD820G70 RR001D80 ID000Q24
    DD820G70
    ID000Q25
    DD020G70
    RR001D80
\end{myverbbox}
\begin{myverbbox}[\small]{\voutput}
    YES
    NO
    NO
    YES
\end{myverbbox}
\inputoutputTable

\solutionSection

Данную задачу можно решить несколькими способами. В приведенном решении каждая входная строка превращается в уникальное число. Массив таких чисел сортируется, после чего каждая входная строка кодируется тем же алгоритмом, что и входная, после чего в сохраненном массиве бинарным поиском ищется такое же число.

\includeSolutionIfExistsByPath{final/subject_tour/inf1303_bas/task_03}