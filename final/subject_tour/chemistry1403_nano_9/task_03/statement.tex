\assignementTitle{}{25}{3}

Имеется 3 монеты одинакового размера (диаметром 20 мм и толщиной 0.8 мм), но отчеканенные  из трех разных чистых 
металлов - железа, меди и золота, а также кристаллический хлорид железа (III), платиновая проволока, несколько 
химических стаканов и вода.

\begin{enumerate}
    \item Как с помощью этих объектов покрыть слоем меди железную монету? Кратко опишите последовательность действий, напишите уравнения реакций, которые будут происходить в описанной Вами методике осаждения меди.
    \item Как с помощью этих объектов покрыть слоем меди золотую монету? Кратко опишите последовательность действий, напишите уравнения реакций, которые будут происходить в описанной Вами методике осаждения меди.
    \item Рассчитайте толщину нанослоя меди, покрывающего железную монету, если медная монета растворилась в ходе 
    процесса нанесения этого слоя на 3\% и вся медь из раствора осела на железной монете. Считайте, что 
    образовавшийся слой имеет одинаковую толщину по всей поверхности монеты. Объем монеты рассчитывается по 
    формуле $V = \pi r2h$, где $r$ - радиус, а $h$ - толщина монеты.    
\end{enumerate}

