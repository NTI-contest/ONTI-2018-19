\solutionSection

\begin{enumerate}
    \item Если медную монету опустить в раствор хлорида трехвалентного железа, пойдет реакция $Cu + FeCl_3 = CuCl_2 + FeCl_2$. 
    Через некоторое время в полученный раствор, не содержащий избытка $FeCl_3$, опускают железную монету, при 
    этом извлекают медную. Осаждение меди на железе: $Fe + CuСl_2 = Cu + FeCl_2$.
    \item Создав гальваническую пару $Au/Fe$, соединив металлы платиновой проволокой, следует погрузить ее в 
    раствор с $СuCl_2$ (получение - см выше). При этом медь будет выделяться на золотой монете, которая станет катодом.
    \item Объем монеты составляет $V = 3.14 \cdot 10 \cdot 10 \cdot 0.8 = 251.2$ мм$^3$, растворившейся части - 
    $V_1 = 251.2 \cdot 0.03 = 7.53$ мм$^3$. Площадь поверхности монеты $S = 2 \cdot \pi r2 + S_{\text{ребра}}= 678$ мм$^2$. 
    Считая слой равномерным, получаем толщину $d = V_1/S = 11.1$ мкм.
\end{enumerate}

Уравнения - 4 балла, разумное описание методики 6 баллов

Методика выделения меди на золотой монете - 8 баллов

Расчет толщины слоя - 7 баллов, итого 25 баллов