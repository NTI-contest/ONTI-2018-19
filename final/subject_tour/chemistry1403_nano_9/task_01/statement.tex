\assignementTitle{}{20}{1}

Открытую колбу объемом 2 л заполнили при н.у. сухим газообразным аммиаком, после чего внесли внутрь 
разогретое платиновое кольцо. После этого кольцо раскалилось докрасна (1), стали заметны следы бурого газа 
(2), а на стенках сосуда стал образовываться белый налет (3).

\begin{enumerate}
    \item Объясните наблюдаемые явления (1)-(3), напишите уравнения всех протекающих химических процессов.
    \item Белое вещество на стенках сосуда хорошо растворимо в воде. Оцените максимально возможную молярную 
    концентрацию этого вещества в растворе, полученном при заполнении этого сосуда водой до краев.
    \item Как образуется бурый газ в атмосфере? Напишите уравнения реакций и укажите условия их протекания.
    \item Одним из продуктов реакции, протекающих при нагревании и прокаливании твердого вещества, 
    образующегося на стенках сосуда, является газообразное вещество, имеющее сладковатый запах и 
    использующееся в пищевой промышленности в качестве пропеллента для изготовления взбитых сливок. 
    Назовите газ двумя способами и напишите уравнение реакции описанного выше термического разложения.
\end{enumerate}