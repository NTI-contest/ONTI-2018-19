\solutionSection

\begin{enumerate}
    \item При каталитическом окислении аммиака образуется $NO$, причем реакция настолько экзотермическая, 
    что поверхность катализатора разогревается докрасна. Бурый газ - $NO_2$, легко образующийся в 
    результате окисления $NO$ на воздухе. В присутствии кислорода воздуха $NO_2$ также соединяется с парами воды, 
    образующимися при горении аммиака, давая азотную кислоту, которая соединяется с оставшимся в колбе аммиаком. 
    В результате образуется нитрат аммония - соль, имеющая ионное строение и посему твердая в указанных условиях.
    Уравнения реакций:
    $$4NH_3 + 5O_2 = 4NO + 6H_2O$$
    $$2NO + O_2 = 2NO_2$$
    $$4NO_2 + O_2 + 2H_2O = 4HNO_3$$
    $$HNO_3 + NH_3 = NH4NO_3.$$
    \item При оценке будем исходить из того, что вначале колба была полностью заполнена аммиаком, тогда его количество в такой 
    колбе составляет $$n = \nu/Vm = 2/22.4 = 0.09 \: \text{моль}.$$ Если допустить, что все остальные реагенты находилися в 
    избытке, то максимальное количество образовавшейся $HNO_3$ также составляет 0.09 моль. Но для того, чтобы 
    образовался нитрат аммония, необходимо, чтобы половина атомов азота пошла на образование катиона аммония, 
    поэтому максимальное возможное количество нитрата аммония составляет также половину количества атомов азота, 
    то есть 0.45 моль. Так объем колбы равен 2 л, то предельная концентрация составляет $$c = \nu/V_{\text{колбы}}=0.45/2 = 0.225 \: \text{М}.$$
    \item В атмосфере $NO_2$ может образоваться во время грозы. Грозовой разряд катализирует реакцию между кислородом и азотом с образованием $NO$, 
    который далее окисляется до $NO_2$.
    \item Уравнение реакции разложения:  $NH_4NO_3 = N_2O + 2H_2O$, названия газа - оксид азота (I), закись азота, 
    веселящий газ.
\end{enumerate}

\additionalCriteria

За каждое уравнение - \textit{2 балла (без коэффициентов - по 1)}, наличие объяснения с указанием продуктов, но без реакций - \textit{4 балла}.

Разумная оценка - \textit{4 балла}

Каждое уравнение - по \textit{2 балла}, если нет уравнений, но есть указание на грозу - всего \textit{2 балла}.

Реакция разложения нитрата аммония - \textit{2 балла}, каждое название - \textit{1 балл}.
