\solutionSection

\begin{enumerate}
    \item Вещество А - йодид калия.

    Реакция 1: $2KI + 2H_2SO_4 = I_2 + K_2SO_4 + SO_2 + 2H_2O$

    Реакция 2: $4KI + O_2 + 2CO_2 = I_2 + 2K_2CO_3$
    
    Или точнее, $6KI + O_2 + 2CO_2 = 2KI_3 + 2K_2CO_3$ (засчитываются оба варианта)

    $KI + AgNO_3 = AgI \uparrow + KNO_3$

    Иодид серебра - нерастворимое вещество желтого цвета с требуемой массовой долей $\omega(Ag)= M(Ag)/M(AgI) = 108/234 = 0.4615$.
    \item Во времена Наполеона, вероятнее всего, использовали обыконовенный дымный порох, состоящий из угля, серы и калийной селитры. Возможны уравнения горения:
    
    $2KNO_3 + S + 3C = K_2S + N_2 + 3CO_2$ или \\$10KNO_3 + 3S + 8C = 2K_2CO_3 + 3K_2SO_4 + 6CO_2 + 5N_2$
    
\end{enumerate}

\additionalCriteria

Состав порошка - \textit{2 балла}, уравнения - по \textit{2 балла}, подтверждение по массовой доле - \textit{2 балла}

Состав пороха - \textit{2 балла},  уравнение горения - \textit{3 балла}. Всего \textit{15 баллов}
