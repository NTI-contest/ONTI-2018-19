\assignementTitle{}{15}{4}

Во времена наполеоновских войн французским войскам очень не хватало калийной селитры для производства пороха, 
поэтому активно разрабатывались альтернативные способы его получения. Одним из таких способов был процесс 
высушивания и обжига в медных котлах морских водорослей. При этом, помимо целевого продукта, на дне котла 
постепенно скапливалось белое кристаллическое вещество А. Его состав помог установить случай, в результате 
которого на большую порцию этого порошка пролили концентрированную серную кислоту. Это привело к бурному 
выделению фиолетовых паров, легко конденсировавшихся на холодных поверхностях (реакция 1). Также известно, 
что при длительном хранении на воздухе это вещество постепенно желтеет (реакция 2), а массовая доля серебра 
в осадке, появляющемся при добавлении к водному раствору данного порошка раствора нитрата серебра (реакция 3), 
составляет 46,15\%.

\begin{enumerate}
    \item Установите состав вещества А. Ответ подтвердите расчетами.
    \item Напишите уравнения реакций (1)-(3).
    \item Из каких веществ состоит порох? Напишите возможное уравнение его горения.
\end{enumerate}