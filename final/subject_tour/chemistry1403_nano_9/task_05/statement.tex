\assignementTitle{}{25}{5}

Для приготовления раствора 1 навеску алюминия массой 2.7 г поместили в мерную колбу на 200 мл, добавили 80 мл 5М 
соляной кислоты, дождались полного выделения газа, а затем довели до метки с помощью дистиллированной воды. Для 
приготовления раствора 2 навеску гидроксида натрия массой 1.60 г поместили в мерную колбу на 250 мл и довели до 
метки с помощью дистиллированной воды.

\begin{enumerate}
    \item Рассчитайте молярную концентрацию кислоты по окончании приготовления раствора 1.
    \item Рассчитайте массовые доли солей в растворе, полученном при добавлении к 50 мл раствора 1 50 мл 
    раствора 2 и последующем пропускании 5.6 л (н.у.) сухого аммиака. Напишите уравнения всех протекающих реакций. 
    Считайте, что плотности растворов 1 и 2 равны плотности чистой воды.
\end{enumerate}