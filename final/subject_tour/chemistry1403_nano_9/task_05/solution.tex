\solutionSection

\begin{enumerate}
    \item $$2Al + 6HCl = 2AlCl_3 + 3H_2, n(Al) = 2.7/27 = 0.1 \: \text{моль},$$ 
    $$n(HCl) = c \cdot V = 5 \cdot 0.08 = 0.4 \: \text{моль}.$$
    
    Соляная кислота в избытке, с алюминием прореагирует только 0.3 моль, 
    $$n(HCl)_\text{ост} = 0.4 - 0.3 = 0.1\: \text{моль}.$$

    С учетом разбавления до 200 мл $$c = n/V = 0.1/0.2 = 0.5 \: \text{M}$$
    \item В 50 мл раствора 1 содержится 0.025 моль $HCl$ и 0.025 моль $AlCl_3$, в 50 мл раствора 2 - 0.008 моль
    
    $$NaOH (c(NaOH) = 1.6/40/0.25= 0.16\: \text{M},$$
    $$n(NaOH) = 0.16 \cdot 0.05 = 0.008\: \text{моль}.$$ 
    В присутствии кислоты невозможно 
    выпадение гидроксида алюминия, поэтому первой будет реагировать кислота: $$NaOH + HCl = NaCl + H_2O.$$ 
    
    Данного 
    количества $NaOH$ не хватает для полной нейтрализации кислоты, поэтому алюминий останется в виде соли $AlCl_3$ 
    перед добавлением аммиака. По уравнению реакции, количество образовавшегося $NaCl$ также составляет 0.008 моль.
    
    Рассчитаем количество добавленного аммиака: $$n(NH_3) = 5.6/22.4 = 0.25\: \text{моль}.$$ Этого количества хватит, чтобы не 
    только нейтрализовать оставшуюся кислоту, но и осадить весь алюминий в виде гидроксида:

    $$NH_3 + HCl = NH_4Cl,$$ 
    $$3NH_3 + AlCl_3 + 3H_2O = 3NH_4Cl + Al(OH)_3 \downarrow.$$
    
    По первой реакции образуется $$n(NH_4Cl) = n(HCl)_\text{ост} = 0.025 - 0.008 = 0.017\: \text{моль},$$
    
    по второй - 
    
    $0.025 \cdot 3 = 0.075$ моль $NH_4Cl$, итого в конечной смеси находятся две соли - хлориды натрия и аммония, 
    $n(NH_4Cl) = 0.075 + 0.017 = 0.092$ моль, $n(NaCl) = 0.008$ моль. 
    
    Тогда $$m(NH_4Cl) = 0.075 \cdot 53.5 = 4.92 \: \text{г},$$
    $$m(NaCl) = 0.008 \cdot 58.5 = 0.468 \: \text{г}.$$

    Масса конечного раствора складывается из масс аммиака и растворов (плотность которых равна плотности воды по 
    условию) за вычетом гидроксида алюминия: $$m_{\text{р-ра}} = 100 + 0.25 \cdot 17 - 0.025 \cdot 78= 102.3 \: \text{г}.$$
    
    Тогда $$\omega(NH_4Cl) = 4.8\%, \omega(NaCl) = 0.5\%.$$
    
\end{enumerate}

\additionalCriteria

Расчет концентрации - \textit{5 баллов}, 

Уравнения реакций - по \textit{2 балла}, Расчет массовых долей - \textit{1 баллов}, не учтена масса гидроксида алюминия - \textit{6 баллов}, арифметическая ошибка, сохраняющая схему химических процессов - \textit{8 баллов}, всего - \textit{25 баллов}
