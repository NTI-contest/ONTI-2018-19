\solutionSection

\begin{enumerate}
    \item $2Al + 6HCl = 2AlCl_3 + 3H_2, n(Al) = 2.7/27 = 0.1$ моль, $n(HCl) = c \cdot V = 5 \cdot 0.08 = 0.4$ моль. 
    Соляная кислота в избытке, с алюминием прореагирует только 0.3 моль, $n(HCl)_\text{ост} = 0.4 - 0.3 = 0.1$ моль. 
    С учетом разбавления до 200 мл $c = n/V = 0.1/0.2 = 0.5$M
    \item В 50 мл раствора 1 содержится 0.025 моль $HCl$ и 0.025 моль $AlCl_3$, в 50 мл раствора 2 - 0.008 моль 
    $NaOH (c(NaOH) = 1.6/40/0.25= 0.16$M, $n(NaOH) = 0.16 \cdot 0.05 = 0.008$ моль. В присутствии кислоты невозможно 
    выпадение гидроксида алюминия, поэтому первой будет реагировать кислота: $NaOH + HСl = NaCl + H_2O$. Данного 
    количества $NaOH$ не хватает для полной нейтрализации кислоты, поэтому алюминий останется в виде соли $AlCl_3$ 
    перед добавлением аммиака. По уравнению реакции, количество образовавшегося $NaCl$ также составляет 0.008 моль.
    
    Рассчитаем количество добавленного аммиака: $n(NH3) = 5.6/22.4 = 0.25$ моль. Этого количества хватит, чтобы не 
    только нейтрализовать оставшуюся кислоту, но и осадить весь алюминий в виде гидроксида:

    $NH_3 + HCl = NH_4Cl, 3NH_3 + AlCl_3 + 3H_2O = 3NH_4Cl + Al(OH)_3 \downarrow.$ 
    
    По первой реакции образуется $n(NH_4Cl) = n(HCl)_\text{ост} = 0.025 - 0.008 = 0.017$ моль, по второй - 
    $0.025 \cdot 3 = 0.075$ моль $NH_4Cl$, итого в конечной смеси находятся две соли - хлориды натрия и аммония, 
    $n(NH4Cl) = 0.075 + 0.017 = 0.092$ моль, $n(NaCl) = 0.008$ моль. Тогда $m(NH4Cl) = 0.075 \cdot 53.5 = 4.92$ г,  
    $m(NaCl) = 0.008 \cdot 58.5 = 0.468$ г. 

    Масса конечного раствора складывается из масс аммиака и растворов (плотность которых равна плотности воды по 
    условию) за вычетом гидроксида алюминия : $m_{\text{р-ра}} = 100 + 0.25 \cdot 17 - 0.025 \cdot 78= 102.3$ г. 
    Тогда $\omega(NH_4Cl) = 4.8\%, \omega(NaCl) = 0.5\%$. 
    
\end{enumerate}

Расчет концентрации - 5 баллов, 

Уравнения реакций - по 2 балла, Расчет массовых долей - 11 баллов, не учтена масса гидроксида алюминия- 6 баллов, арифметическая ошибка, сохраняющая схему химических процессов - 8 баллов, всего - 25 баллов
