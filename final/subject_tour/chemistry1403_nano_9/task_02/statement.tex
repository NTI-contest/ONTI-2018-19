\assignementTitle{}{15}{2}

Если в помещении становится душно, мы открываем окно и корректируем газовый состав воздуха, заменяя при 
проветривании воздух на свежий. Восполнять запас кислорода в воздухе помогают растения с помощью реакции 
фотосинтеза, суммарное уравнение которой, как известно, включает углекислый газ в качестве одного из реагентов 
и кислород в качестве одного из продуктов. В случае длительного пребывания в замкнутом пространстве (например, в батискафах или в космосе) приходится использовать другие способы «регенерации» воздуха для 
дыхания. Понятно, что можно взять с собой баллоны с кислородом, однако более «прогрессивным» является 
использование веществ, способных поглощать углекислый газ и выделять кислород в ходе одной и той же реакции, 
по сути, восполняющих фотосинтетическую роль растений.

\begin{enumerate}
    \item Одним из веществ, ведущих себя подобным образом, является $X$ - бинарное соединение щелочного металла с кислородом, 
    массовая доля которого составляет 41\%. Установите формулу вещества $X$ и приведите уравнение его реакции с углекислым 
    газом, если известно, что в состав соединения $X$ входит 4 атома.
    \item Рассчитайте время нахождения в герметичном помещении объемом 10 м$^3$ человека, за которое доля 
    содержащегося в воздухе кислорода снизится с 21 до 16 объемных процентов (примерно столько кислорода 
    содержится в выдыхаемом воздухе), если считать, что дыхание происходит с частотой 20 вдохов в минуту, 
    и за 1 вдох человек заменяет 25 мл кислорода на 25 мл углекислого газа.
    \item Какую массу вещества $X$ необходимо взять для того, чтобы в течение 1 часа 
    количество кислорода в помещении можно было бы поддержать на начальном уровне? На 
    сколько граммов при этом потяжелеет само вещество $X$? Считайте молярный объем в данных условиях равным 24 л/моль.    
\end{enumerate}

