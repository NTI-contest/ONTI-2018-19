\solutionSection

\begin{enumerate}
    \item Бинарное соединение щелочного металла с кислородом имеет формулу $M_aO_b$, где $a+b=4$. Очевидно, что 
    вещество $X$ не может быть оксидом, поскольку при реакции оксида щелочного металла с углекислым газом 
    происходит реакция соединения, а по условию в продуктах реакции должен присутствовать кислород. Значит в 
    соединении Х больше одного атома кислорода. Предположим, что $b=2$, тогда $M(a \cdot M) = 32/0.41 - 32 = 46$ 
    г/моль. Это масса двух атомов натрия. Формула вещества Na2O2. Уравнение реакции $Na_2O_2 + CO_2 = Na_2CO_3 + O_2$.
    \item 5\% от 10 м$^3$ - это 500 л, за 1 минуту человек потребляет $20\cdot 25$мл = 0,5 л кислорода, откуда 
    следует, что этого объема хватит на 1000 минут или 1000/60 = 16 часов 40 минут. Однако духота будет ощущаться 
    гораздо раньше, так как в реальности количество потребляемого с каждым вдохом кислорода будет понижаться из-за 
    понижения его концентрации в помещении и органы начнут ощущать недостаток кислорода.
    \item Для 1 часа комфортного дыхания человеку потребуется $0.5 \cdot 60 = 30$ л кислорода, откуда $n(O_2) = V/V_m = 30/24 = 1.2$ моль. Тогда 
    по уравнению реакции $n(Na2CO3) = 2.4$ моль, $m =n \cdot M = 254.4$ г. Масса твёрдого вещества увеличится на 
    массу поглощённого углекислого газа за вычетом выделившегося кислорода: $\Delta m = m(CO_2) - m(O_2) = 2.4 \cdot 44 - 1.2 \cdot 32 = 67.2$ г.
\end{enumerate}

Расчет - 2 балла (подбор с проверкой также засчитывается), уравнение реакции - 2 балла

Расчет времени - 4 балла

Расчет необходимой массы пероксида натрия - 4 балла, утяжеление твердого вещества - 3 балла, всего - 15 баллов

