\solutionSection
\begin{enumerate}

    \item[a)] Во-первых, покажем, что у нас есть только одно число, модуль которого больше $1$. Если бы таких чисел было бы несколько, то можно было бы выбрать два числа $a, b$ с наибольшими модулями и тогда их произведение было бы больше любого из чисел на доске, что противоречит условию. Аналогичные рассуждения можно провести для чисел, меньших единицы по модулю.
	
	Во-вторых, если у нас есть число не равное $\pm 1$, то $-1$ не может входить в набор. Иначе бы мы получили пару чисел больших единицы по модулю или меньших и вернулись бы к предыдущим двум рассуждениям. Следовательно, всего чисел не больше трех. Пример: $\dfrac{1}{2}, 1, 2$.

	\item[б)] Заметим, что всевозможные наборы чисел, которые могут быть выписаны на доске, выглядят так: $\dfrac{1}{n}, 1, n$ (или их какое-то подмножество). Следовательно, максимально возможная сумма чисел будет достигаться при $n = 2019$ (можно показать, что при увеличении $n$ сумма всех трех чисел будет увеличиваться: например, взяв производную от суммы, как функции, зависящей от $n$).
 
\end{enumerate}	

\answerMath{ б) $2020\frac{1}{2019}$.}

\additionalCriteria

\begin{enumerate}

    \item[a)] \textit{(15 баллов)}
	
	$+5$ баллов за рассуждения о том, почему нет больше $1$
	числа, большего единицы.
	
	$+5$ баллов за рассуждения о том, почему нет больше $1$
	числа, меньшего единицы.
	
	$+5$ баллов за рассуждения о том, почему нет $-1$ в наборе.
	
	\item[б)] \textit{(5 баллов)}

	$+3$ балла за рассуждения о том, почему подходящий набор
	будет составлен из трех чисел: единицы, числа $n$ и его обратного.
	
	$+1$ балл за рассуждения, 
	почему при увеличении $n$ сумма увеличивается.
	
	$+1$ балл за пример.
		
\end{enumerate}