\solutionSection
Обозначим $\angle FBC = \alpha$. Тогда можно показать, что $AD \cdot \tg \alpha = EC$ (можно, например, опустить высоты из $E$ и $D$ на $AC$ и приравнять их, воспользовавшись тем, что $\angle CAB = \alpha$).
	 
Пусть точки $M$ и $N$ получаются пересечением лучей $AL$ и $LD$ с прямой $BC$. Соединим точки $A$ и $N$. Тогда $D$ будет ортоцентром треугольника $AMN$.
	
Так как $AB$ и $NL$~---~высоты в треугольнике $AMN$, то $\angle LBM = \angle MAN = \alpha$. Кроме того, равны углы $NMA$ и $ADL$. Поэтому треугольники $ADL$ и $NLM$ подобны, откуда
$$\frac{AD}{MN} = \frac{AL}{LN} = \ctg \angle MAN = \ctg \alpha \Rightarrow AD \tg \alpha = MN.$$
Из двух равенств выше имеем искомое $MN = EC$.

\additionalCriteria

$+10$ баллов за вывод формулы $AD \cdot \tg \alpha = EC$.
	
	$+5$ баллов за замечание, что $D$ ортоцентр треугольника $AMN$.
	
	$+10$ баллов за вывод формулы $AD \tg \alpha = MN$.
	
	\underline{Замечания} 
\begin{enumerate}
	\item За полностью правильное решение, 
	которое отличается от авторского баллы не снижать
	(в случае наличия ошибок, снижать баллы пропорционально 
	их тяжести и влияния на решение задачи).
\end{enumerate}