\solutionSection

\begin{enumerate}

    \item[a)]  Допустим, что всего у нас $n$ шаров ($n > 2$) в $3$-х корзинах. Посчитаем вероятность того, что все корзины не пустые, 
	с помощью формулы включений-исключений
	$$p = 3 \cdot \left(\frac{2}{3}\right)^{n} - 3 \cdot \left(\frac{1}{3}\right)^{n}.$$
	Заметим, что в наших корзинах остается $16$ определенных шаров
	всегда, поэтому в этом пункте ответ будет получаться подстановкой
	$n = 16$ в формулу выше. 

	\item[б)] Здесь у нас повторяется тоже, что и в предыдущем пункте,
	но $n$ может принимать разные значения с разной вероятностью.
	Тогда с вероятностью $\dfrac{1}{2}$ останется $n = 15$ шаров,
	с вероятностью $\dfrac{1}{4}$ останется $n = 14$ шаров, \ldots
	Тогда итоговой вероятностью будет в каждом случае произведение
	вероятности того, что все три корзины не пустые при определенном
	числе шаров и вероятности того, что после выбора Пети останется
	это число шаров:
	$$P = \frac{1}{2^{15}} + \frac{1}{2^{15}} + \frac{1}{2^{14}} + 
	\sum\limits_{n=3}^{15} \left(3 \cdot \left(\frac{2}{3}\right)^{n} - 3 \cdot \left(\frac{1}{3}\right)^{n}\right) \cdot \frac{1}{2^{16-n}}.$$
	В этой сумме первые три слагаемых~---~это случаи, когда останется
	$0$, $1$ и $2$ шара. Это сумму можно было не сокращать, хотя имеет
	место следующие выкладки.
	
	Сократим немного выражение:
	$$P = \frac{1}{2^{13}} + 3 \cdot \frac{1}{2^{13}}
	\cdot \sum\limits_{t = 0}^{12}
	\left(\cdot \left(\frac{2}{3}\right)^{t+3} - 3 \cdot \left(\frac{1}{3}\right)^{t+3}\right) \cdot 2^{t}.$$
	По формуле суммы геометрической прогрессии имеем
	$$P = \frac{1}{2^{13}} + \frac{3}{2^{13}} \cdot\left\{ \left(\frac{2}{3}\right)^3 
	\cdot \frac{1 - \left(\frac{2}{3}\right)^{13}}{1 - \frac{2}{3}} + 
    \left(\frac{1}{3}\right)^3 \cdot \frac{1 - \left(\frac{1}{3}\right)^{13}}{\frac{1}{3} - 1}\right\}$$
    
\end{enumerate}	

\answerMath{а), б) см. решение.}

\additionalCriteria

\begin{enumerate}
	\item[a)] \textit{(5 баллов)}
	\item[] 
		$+2$ балла за рассуждения, почему можно рассматривать $16$ шаров
		вместо $17$.
	
		$+3$ балла за использование формулы включений-исключений.
	
	\item[б)] \textit{(20 баллов)}

		$+5$ баллов за рассуждения о том, почему, как и в первом
		пункте, в каждом случае у нас вероятность зависит только от
		числа оставшихся шаров.
	
		$+5$ баллов за рассуждения о том, почему надо 
		перемножать вероятности.

		$+10$ баллов за правильно выписанную формулу и объяснения,
		почему она так выглядит (включая вынесенные первые слагаемые).
\end{enumerate}	

	\underline{Замечания} 
\begin{enumerate}
	\item Если во втором пункте не разобран случай с $n = 0, 1, 2$,
	т.~е. сумма итоговая неправильна и нет указаний нигде о том,
	что в этих случаях будет, то балл за формулу снижается на $5$ баллов.
\end{enumerate}