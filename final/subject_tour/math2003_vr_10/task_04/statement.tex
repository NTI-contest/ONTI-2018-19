\assignementTitle{}{25}{4}

Перед Васей и Петей стоят три корзины (желтая, белая и красная) 
	и лежат $17$ пронумерованных шаров. Сначала Вася берет по шару
	и с равной вероятностью кладет в одну из корзин. 
	Затем Петя достает шар с номером $1$ из той корзины, в которой он лежит. 
	 
	\begin{enumerate}
	\item[а)] \textit{(5 баллов)} Какова вероятность того, что после этого
	среди корзин будет хотя бы одна пустая?
	\item[б)] \textit{(20 баллов)} После этого Петя с вероятностью $\dfrac{1}{2}$
	достает из корзин шар с номером $2$, или с вероятностью $\dfrac{1}{4}$
	он достает из корзин два шара: с номером $2$ и $3$, или с вероятностью
	$\dfrac{1}{8}$ он достает из корзин три шара: с номером $2$, $3$ и $4$ \dots
	Все $16$ оставшихся шаров он достает с вероятностью $\dfrac{1}{2^{15}}$.
	\end{enumerate}
	 Какова вероятность того, что после этого
	среди корзин будет хотя бы одна пустая?
	(в этом пункте ответ может быть получен в виде суммы 
	некоторого выражения; при его получении дальнейшие 
	сокращения можно не проводить)