\assignementTitle{(20 баллов)}{20}{2}

В двух различных узлах бесконечной сетки сидят заяц и волк. 
	Они оба могут
	перемещаться только по сторонам клеток. Первым двигается заяц,
	потом волк, после него заяц... За один свой ход волк 
	преодолевает $v$ сторон клеток, а заяц --- $z$ 
	(двигаются они строго по сторонам клеток).
    
    \begin{enumerate}
    	\item[а)] \textit{(10 баллов)} Пусть скорости у них равны. 
	Докажите, что заяц может так
	действовать, чтобы волк его никогда не догнал. 
        \item[б)] \textit{(10 баллов)} Пусть скорость волка $v = z+1$.	
	Докажите, что при любом расположении персонажей 
    волк сможет догнать зайца.
\end{enumerate}