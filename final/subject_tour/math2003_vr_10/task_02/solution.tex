\solutionSection
\begin{enumerate}

    \item[a)] Пусть заяц всегда будет двигаться от волка. Заметим, что он всегда может двигаться так, чтобы кратчайшее расстояние между ним и волком увеличилось ровно на $z$. Тогда, как бы волк ни двигался, но после его хода расстояние между ними не сократится более, чем на $v = z$. Поэтому за два хода (зайца и волка) расстояние между ними не уменьшится, а следовательно, если оно изначально было положительным, то и после любого хода будет оставаться таким же.
	
    \item[б)] В этом пункте нам надо смотреть на погоню со стороны волка. Он может повторить все движения зайца и при этом еще пройти на одну клетку больше. Тогда пусть он будет последним передвижением сокращать расстояние между ними. При таком поведении расстояние между волком и зайцем будет сокращаться всегда хотя бы на единицу после их двух ходов. Так как в начале игры оно было конечным, то наступит момент, когда волк догонит зайца.
    
\end{enumerate}

\additionalCriteria

\begin{enumerate}

    \item[a)] \textit{(10 баллов)}
	
	$+10$ баллов за рассуждения о том, почему после двух ходов
	заяц сможет не уменьшить расстояние между ним и волком.

	\item[б)] \textit{(10 баллов)}

	$+10$ баллов за рассуждения о том, почему волк может
	сокращать расстояния между ним и зайцем.
\end{enumerate}

	\underline{Замечания}
\begin{enumerate}
	\item Может быть приведено более оптимальное решение, где
	волк не повторяет движения зайца, а просто сокращает расстояния
	между ними, двигаясь по кратчайшему пути. В этом случае, если
	не приведено обоснований, почему после двух ходов расстояние между
	ними сократится, то ставится половина баллов.
	
	\item За решения без обоснований и пояснений, почему волк
	догонит или не догонит зайца, оцениваются нулем баллов.
\end{enumerate}	