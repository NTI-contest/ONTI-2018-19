\assignementTitle{Универсальный гонщик}{50}{1}

В романе Жюля Верна «Властелин мира» описывается машина-амфибия «Грозный», способная двигаться под водой, 
по суше, по поверхности воды и по воздуху. 

Длина этой машины 10 метров. Предположим, что когда она двигается под водой, она имеет продолговатую 
форму, очень близкую к цилиндру радиуса 1.5 метра. Пусть масса такой машины равна 7000 кг. 
Сила сопротивления воды при быстром движении рассчитывается для такой машины по следующей формуле:
$$F=C \cdot \frac{\rho  \cdot v^2}{2} \cdot S$$
Здесь C = 0.82 – коэффициент формы, S – характерная площадь, для подобной формы оцениваемая как $V^{2/3}$, 
$\rho  = 1000$ кг/м$^3$ – плотность воды, $v$ – скорость движения.

Объём цилиндра можно вычислить по формуле:  $V=\pi \cdot r^2 \cdot L$.

При быстром движении по суше верна такая же формула, только в качестве плотности следует брать 
плотность воздуха, которая равна  $\rho_\text{возд} = 1.29$ кг/м$^3$. По суше машина движется на колесах Коэффициент скольжения трения между колёсами и землёй равен 0.1, 
Полезная мощность машины 1.5~МВт.

Ускорение свободного падения принять равным 9.81 м/c$^2$.

При движении в воде считать, что машина умеет отталкиваться от воды, создавая некоторую силу тяги F, предельное 
значение которой ограничено мощностью машины.

\begin{enumerate}
    \item За какое время машина-амфибия проплывёт 10 км под водой, используя свой двигатель на полную мощность, 
    оставаясь при этом на одной глубине? 
    \item До какой температуры нагреется эта машина, если в него ударит молния, если считать, что тепло по машине 
    распространится равномерно, а её средняя удельная теплоёмкость на единицу массы равна 490 Дж/(кг$\cdot$К) и на 
    неё налепился в снежном облаке слой мокрого снега толщиной в 1 см (удельная теплота плавления 330 кДж/кг, 
    плотность 350 кг/м$^3$)?  Энергию молнии принять равной 300 миллионов джоулей, а начальную температуру 
    автомобиля вместе со снегом за ноль градусов по Цельсию. Считать, что снег налепился только на боковую 
    поверхность цилиндрической формы.
    \item Какую максимальную скорость эта машина может развивать на суше? Получите ответ с точностью до 1 см/c. По 
    земле машина двигается на обычных колёсах.
    \item Эта машина, плавая на поверхности воды, погружается на глубину 0.466 метра.
    
    \putImgWOCaption{5cm}{1}

    Вода ночью теплее, чем воздух, и происходит теплопередача – воздух охлаждает, а вода нагревает машину. 
    Считая, что машина плывёт ночью, оцените устоявшуюся температуру машины, если воздух имеет температуру 5 
    градусов по Цельсию, а вода 15 градусов по Цельсию.
    \item Известно, что максимальная скорость, с которой эта машина может погружаться вертикально вниз в воде 
    составляет 2.27 м/c. Во сколько раз надо увеличить мощность двигателя этой машины, чтобы удвоить эту скорость?
\end{enumerate}