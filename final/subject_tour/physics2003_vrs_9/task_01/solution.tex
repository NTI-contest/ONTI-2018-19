\solutionSection

\begin{enumerate}
    \item Используя двигатель на полную мощность, машина будет двигаться с некоторой постоянной скоростью. Запишем второй закон Ньютона в проекции на направление движения: 
    $$F- C \cdot \frac{\rho \cdot v^2}{2} \cdot S=0; \:  F= C \cdot \frac{\rho \cdot v^2}{2} \cdot S$$ 
    
    Мощность равна произведению силы на скорость, откуда можно найти скорость:
    $$N=F \cdot v=C \cdot \frac{\rho \cdot v^3}{2} \cdot S; \: v= \sqrt[3]{\frac{2 \cdot N}{C \cdot \rho \cdot S}}=\sqrt[3]{\frac{2 \cdot 1.5 \cdot 10^6}{0.82 \cdot 10^3 \cdot (10 \cdot \pi \cdot 1.5^2 )^{\frac{2}{3}}}}= 5.98 \: \text{м/c}$$
    
    При движении с постоянной скоростью время движения будет равно:
    $$t=\frac{10000}{5.98}=1672 \: \text{секунды}$$
    
    \answerMath{$t = 1672$ секунды}
\end{enumerate}

\additionalCriteria
    \begin{itemize}
        \item Указано, что машина будет двигаться с постоянной скоростью – 2 балла.
        \item Правильно записан второй закон Ньютона – 3 – балла.
        \item Записано выражение для мощности или его аналог, позволяющий найти скорость – 1 балл.
        \item Получено правильное выражение для времени – 2 балла.
        \item Получен правильный числовой ответ – 2 балла.
    \end{itemize}

    \begin{enumerate}
    \item[2.] Запишем уравнение теплового баланса. Вся энергия, 
    которая есть в молнии, уходит на плавление снега, 
    нагревание воды и нагревание автомобиля:
    $$Q=m \cdot \lambda +m \cdot c_1 \cdot t+M \cdot c_2 \cdot t \rightarrow t=  \frac{Q-m \cdot r}{m \cdot c_1+M \cdot c_2}$$

    Осталось посчитать массу снега $m$. Эту массу можно найти, умножив среднюю плотность на объём снега, равный 
    разности объёмов двух цилиндров:
    $$m=\rho \cdot \pi \cdot L \cdot ((r+a)^2-r^2 )=350 \cdot \pi \cdot 10 \cdot (1.51^2-1.5^2 ) \approx 331 \: \text{кг}$$
    $$t= \frac{300 \cdot 10^6-m \cdot 330000}{m \cdot 330+7000 \cdot 490} \approx 53.9^\circ \: C$$

    \answerMath{$53.9^\circ C$.}
\end{enumerate}  

    \additionalCriteria
    \begin{itemize}
        \item Указано, что энергия молнии пойдет на плавление снега, нагрев получившейся воды и автомобиля 
        или эти соображения использованы при составлении уравнений – 2 балла.
        \item Правильно записано уравнение теплового баланса – 3 балла.
        \item Получено выражение для массы снега – 1 балл.
        \item Получено правильное выражение для температуры  – 2 балла.
        \item Получен правильный числовой ответ – 2 балла.      
    \end{itemize}

\begin{enumerate}
    \item[3.] Максимальную скорость можно найти из записи второго закона Ньютона. Запишем его в 
    проекции на ось направления движения, учитывая то, что автомобиль едет за счёт силы трения колёс 
    о землю (больше ему не от чего отталкиваться):
    $$m \cdot g \cdot f-C \cdot \frac{\rho_\text{возд} \cdot v^2_{max}}{2} \cdot S=0$$
    $$v_{max}=\sqrt{\frac{m \cdot g \cdot f}{C \cdot \frac{\rho_\text{возд}}{2} \cdot S}} \approx 27.56 \: \text{м/c}.$$

    \answerMath{27.56 м/c.}
\end{enumerate}

\additionalCriteria
    \begin{itemize}
        \item Отмечено (или учтено при записи уравнений), что машина двигается вперед за счет силы трения колес о поверхность – 1 балл.
        \item Записан второй закон Ньютона для движения машины с постоянной скоростью – 4 балла.
        \item Записано выражение для скорости  –  \textit{3 балла}.
        \item Получен правильный числовой ответ – \textit{2 балла}.             
    \end{itemize}

\begin{enumerate}
    \item[4.] Поток тепла пропорционален произведению разности температур и площади поверхности, 
    через которую поток проходит. Для того чтобы машина находилась в тепловом равновесии тепловой поток 
    от воды к машине и от машины к воздуху должны быть равны.
    
    Найдем площади соприкосновения: 
    
    Площадь под линией состоит из равнобедренного треугольника и сектора, дополнительного к сектору с 
    углом $2 \alpha $, причём $h=R \cdot (1-\cos\alpha  )$ . Отсюда:
    $$\cos\alpha =1-\frac{h}{R} \approx 0.689;   \alpha  \approx 0.811 \: \text{рад}$$\

    Тогда площадь, граничащая с воздухом, относится к площади, граничащей с водой, как $0.811 \div (2 \pi-0.811)$. 
    
    Так как поток тепла пропорционален разности температур и площади поверхности, то получаем уравнение теплового баланса:
    $$0.811 \cdot (t-5)=(2 \pi-0.811) \cdot (15-t) \rightarrow t \approx 13.71^\circ C$$
    \answerMath{$13.71^\circ C$.}
\end{enumerate}

\additionalCriteria
    \begin{itemize}
        \item Использована в любой форме формула для теплового потока – 1 балл.
        \item Отмечено (или учтено при записи уравнений), должно быть тепловое равновесие, приводящее к равенству потоков – 3 балла.
        \item Записано уравнение теплового баланса – 2 балла.
        \item Вычислено отношение площадей соприкосновения (допустимо ставить 1 балл, если отношение вычислено неправильно или не вычислено, но сами площади вычислены.
        \item Получен правильный числовой ответ – 2 балла.
    \end{itemize}

\begin{enumerate}
    \item[5.] Запишем второй закон Ньютона для погружения машины. Будем считать, что для 
    самого быстрого погружения она движется торцом вниз, помогая себе двигателями. Тогда:
    $$C \cdot \frac{\rho \cdot v^2}{2} \cdot S+\rho \cdot g \cdot V-m \cdot g \cdot V-F=0 $$
    
    Отсюда можно выразить мощность двигателя:
    $$N=F \cdot v=\left(C \cdot \frac{\rho \cdot v^2}{2} \cdot S+\rho \cdot g \cdot V-m \cdot g\right) \cdot v$$
    При удвоении скорости новая мощность будет равна:
    $$N_1=C \cdot \frac{8 \cdot \rho \cdot v^3}{2} \cdot S+2 \cdot v \cdot (\rho \cdot g \cdot V-m \cdot g)$$
    
    Отсюда:
    
    $$\frac{N_1}{N}=\frac{C \cdot \frac{8 \cdot \rho \cdot v^3}{2} \cdot S+2 \cdot v \cdot (\rho \cdot g \cdot V-m \cdot g)}{\left(C \cdot \frac{\rho \cdot v^2}{2} \cdot S+\rho \cdot g \cdot V-m \cdot g \right) \cdot v}=$$
    $$=\frac{C \cdot \frac{8 \cdot \rho \cdot v^2}{2} \cdot S+2 \cdot (\rho \cdot g \cdot V-m \cdot g)}{C \cdot \frac{\rho \cdot v^2}{2} \cdot S+\rho \cdot g \cdot V-m \cdot g}$$
    $$\frac{N_1}{N} \approx 2.33$$
    \answerMath{2.33.}
\end{enumerate}

\additionalCriteria
    \begin{itemize}
        \item Отмечено, что для наибольшей скорости машина погружается вертикально вниз, вперед торцом – 1 балл.
        \item Записан второй закон Ньютона для погружения машины с постоянной скоростью – 2 балла.
        \item Записано выражение для мощности двигателя при погружении  – 1 балл (можно ставить 3 балла, если второй закон Ньютона не был записан отдельно, а сразу использован для мощности).
        \item Записано выражение для мощности при погружении с удвоенной скоростью~– 2 балла.
        \item Получен правильный числовой ответ – 2 балла.
        
    \end{itemize}