\assignementTitle{Температурный градиент}{50}{2}

Подводный дрон, представляющий из себя полый шар, получает энергию из разницы температуры морской воды на разной 
глубине. Набирая воду у поверхности океана, дрон увеличивает свою среднюю плотность и спускается на дно, а 
глубоко под водой использует разность температур между запасённой водой и водой вокруг для получения 
электричества с КПД 60\%. 

Радиус дрона равен 30 см, объём шара $V=\frac{4}{3} \pi r^3$, масса незаполненного дрона 50~кг, средняя 
плотность материала, из которого сделан дрон, равна 7200 кг/м$^3$ (полость для воды при расчете этой 
плотности не учитывается).

Напряжение, необходимое для работы системы управления дрона, равно 220 вольт, общее сопротивление его 
электрической схемы равно 800 Ом.

До пятого пункта считать, что плотность воды равна $\rho_0=1000$ кг/м$^3$. Теплоёмкость воды равна 
4200 Дж/(кг$\cdot ^{\circ}$C). На каждые 1000 метров температура воды понижается на 11 градусов по Цельсию.

\begin{enumerate}
    \item Какое минимальное количество воды дрон может взять с собой на дно?
    \item Какое максимальное количество воды дрон может взять с собой на дно?
    \item Какое максимальное количество энергии дрон может получить, охладив воду при спуске на 1500 метров?
    \item Какое минимальное количество раз в году дрону нужно спускаться на 1.5~км, чтобы пополнять запас энергии, 
    необходимый для работы его системы управления? Считать, что в году ровно 365 суток по 24 часа.
    \item За счёт того, что плотность воды увеличивается с понижением температуры, при спуске вниз в 
    наполненном дроне образуется свободное место и можно непрерывно догружать ещё воду по мере погружения 
    вниз, тем самым увеличивая количество энергии, которое можно получить. Какую дополнительную массу 
    воды можно подгрузить, если действовать таким образом, когда полностью заполненный водой дрон опустится 
    на дно, находящееся на глубине 1.5 км, если на поверхности температура равна 31 градус Цельсия, а 
    плотность воды линейно зависит от её температуры: $\rho=\rho_0 (1-\alpha T)$, где для воды коэффициент $\alpha$ 
    принять равным:
    \begin{itemize}
        \item $0.53 \cdot 10^{-4} K^{-1}$ (при температуре $5-10^{\circ}C$);
        \item $1.50 \cdot 10^{-4} K^{-1}$ (при температуре $10-20^{\circ}C$);
        \item $3.02 \cdot 10^{-4} K^{-1}$ (при температуре $20-40^{\circ}C$);
        \item $4.58 \cdot 10^{-4} K^{-1}$ (при температуре $40-60^{\circ}C$);
        \item $5.87 \cdot 10^{-4} K^{-1}$ (при температуре $60-80^{\circ}C$).
    \end{itemize}

    В расчётах считать, температура с глубиной снижается строго линейно, а количество подгружаемой воды на каждом интервале изменения температур из указанных приблизительно прямо пропорционально глубине погружения.
\end{enumerate}