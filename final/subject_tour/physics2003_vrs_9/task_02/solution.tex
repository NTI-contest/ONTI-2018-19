\solutionSection

\begin{enumerate}
    \item Минимальным будет количество воды, при котором 
    средняя плотность дрона будет равна плотности воды 
    (при большем количестве дрон погружается). 

    Тогда:
    
    $$\rho_\text{в}=\frac{m+m_\text{в}}{\frac{4}{3} \pi r^3} \rightarrow  m_\text{в}= \frac{4}{3} \pi r^3 \rho_\text{в}-m=\frac{4}{3} \pi  \cdot 0.3^3 \cdot 1000-50 \approx 63.1 \: \text{кг} .$$  

    \answerMath{63.1 кг}
\end{enumerate}

\additionalCriteria

\begin{itemize}
    \item Указано условие на минимальное количество воды или записаны условия плавания – 4 балла.
    \item Получено правильное выражение для массы воды – 4 балла.
    \item Получен правильный числовой ответ – 2 балла.
\end{itemize}

\begin{enumerate}
    \item[2.] Максимальное количество воды можно вычислить, 
    считая, что при этом количестве воды внутри дрона не 
    останется свободного места. Для вычисления свободного 
    объёма внутри дрона нужно из общего его объёма вычесть 
    тот объём, который занимает материал дрона. После этого 
    умножаем на плотность воды:
    $$m_{\text{в}max}=\rho_\text{в} V_\text{св}=\rho_\text{в} \left(\frac{4}{3} \pi r^3-\frac{m}{\rho_\text{ср}} \right)=1000 \cdot \left(\frac{4}{3} \pi \cdot 0.3^3-\frac{50}{7200}\right) \approx 106.15 \: \text{кг}$$   

    \answerMath{106.15 кг.}
\end{enumerate}

\additionalCriteria

\begin{itemize}
    \item Указано условие на максимальное количество – 2 балла.
    \item Получено выражение для свободного объема дрона – 3 балла.
    \item Получено правильное выражение для массы воды – 3 балла.
    \item Получен правильный числовой ответ – 2 балла.
\end{itemize}

\begin{enumerate}
    \item[3.] Охладив воду при спуске на 1500 метров, 
    можно добиться снижения температуры на $\Delta T_0=11 \cdot \frac{1500}{1000}=16.5^\circ C$. Учитывая КПД, 
    максимальную массу воды и её теплоёмкость, получаем общее количество энергии:
    $$Q=m_{\text{в}max} \cdot c \cdot \Delta T \cdot \eta =106.15 \cdot 4200 \cdot 11 \cdot \frac{1500}{1000} \cdot 0.6 \approx 4.41 \: \text{МДж}$$
    
    \answerMath{4.41 МДж.}
\end{enumerate}

\additionalCriteria

\begin{itemize}
    \item Найдено изменение температуры при погружении – 2 балла.
    \item Записано количество энергии, которое запасено в воде – 3 балла.
    \item Записано запасенная энергия, с учетом КПД преобразования – 3 балла.
    \item Получен правильный числовой ответ – 2 балла.
\end{itemize}

\begin{enumerate}
    \item[4.] Чтобы узнать, сколько раз в году нужно спускаться, нужно поделить количество 
    энергии, которое требуется дрону на непрерывную работу 
    в течение года, на $Q$, полученное из предыдущего пункта 
    задачи и округлить вверх до целого числа:

    $$N\geq\frac{\frac{U^2}{R} \cdot 3600 \cdot 24 \cdot 365}{Q}=\frac{\frac{220^2}{800} \cdot 3600 \cdot 24 \cdot 365}{4413717} \approx 432.27 \rightarrow N_{min}=433$$
    
    \answerMath{433.}
\end{enumerate}

\additionalCriteria

\begin{itemize}
    \item Рассчитано количество энергии, которое необходимо дрону на год – 4 балла.
    \item Получено выражение для количества погружений – 3 балла.
    \item Произведено правильное округление (вверх) – 1 балл.
    \item Получен правильный числовой ответ – 2 балла.
\end{itemize}

\begin{enumerate}
    \item[5.] Рассмотрим погружение дрона, находящегося на 
    некоторой глубине, на малое расстояние $\Delta x$ вниз. 
    Тогда при погружении на это расстояние уменьшится 
    температура, возрастёт плотность и уменьшится объём воды. 
    Распишем это уменьшение и выведем коэффициент 
    пропорциональности в приближённой формуле для массы 
    подгружаемой воды в этот момент:
    $$\Delta m \approx \beta \cdot \Delta x; \:  \Delta T=-\frac{11\cdot x}{1000}; \: \rho=\rho_0 (1-\alpha T)  \rightarrow \Delta \rho=-\alpha \cdot \Delta T=\frac{11\cdot x\cdot \alpha}{1000}$$
    $$m=\rho\cdot V=const  \rightarrow  -\frac{\Delta \rho}{\rho} \approx \frac{\Delta V}{V} \rightarrow  \Delta V \approx -V\cdot \frac{\Delta \rho}{\rho} \rightarrow $$
    $$\rightarrow \Delta m \approx m\cdot \frac{11\cdot x\cdot \alpha}{1000} \rightarrow \beta  \approx \frac{11\cdot \alpha \cdot m}{1000}$$
    
    Здесь массу  $m=106.15$ кг , можно считать примерно постоянной. 
    
    Это выражение можно использовать в случае погружения в одном температурном диапазоне.
    
    Исходя из того, что каждый километр температура снижается на 11 градусов по Цельсию, получаем, что первый километр спуск идёт в третьем температурном диапазоне, а оставшиеся 500 метров – во втором.
    
    Тогда общая $\Delta m$ может быть рассчитана следующим образом:
    $$\Delta m \approx 11\cdot 106.15 \cdot (3.02\cdot 10^{-4}+0.5\cdot 1.50\cdot 10^{-4} ) \cong 0.44 \: \text{кг}$$
    
    \answerMath{0.44 кг.}
\end{enumerate}

\additionalCriteria

\begin{itemize}
    \item Получено выражение для изменения объема воды или массы воды, которую можно долить – 6 баллов.
    \item Учтено, что погружение происходит в разных температурных диапазонах – 2 балла.
    \item Получен правильный числовой ответ – 2 балла.
\end{itemize}