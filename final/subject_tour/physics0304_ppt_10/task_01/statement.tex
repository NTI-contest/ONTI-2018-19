\assignementTitle{Тормозные колодки}{50}{1}

\textbf{}\\
\putImgWOCaption{5cm}{1}
\begin{enumerate}
\item Отношение площади тормозной колодки к квадрату её толщины было увеличено в 3 раза при сохранении её 
полной массы, объёма и теплоёмкости. Во сколько раз изменилась скорость её охлаждения за счёт теплопроводности? 
\item Колёса автомобиля представляют собой 4 диска диаметром 0.48 м и массой 9.2 кг. Автомобиль едет со скорость 
100 км/ч, его полная масса равна 500 кг. К цилиндрической поверхности его колёс прижимаются тормозные колодки с 
силой 200 Н. Коэффициент трения колодки о диски 0.4. Через сколько времени автомобиль остановится? Каков будет 
тормозной путь? Ускорение свободного падения принять за $g = 9.81$ м/c$^2$. Считать, что все колёса вносят 
одинаковый вклад в торможение автомобиля, а колёса не проскальзывают.
\item [3.] Какой толщины должны быть чугунные тормозные колодки, чтобы они не начали плавиться ни в одном из 
четырёх колёс автомобиля при аварийной остановке, если площадь каждой из них равна 50 см$^2$, а передние колёса 
тормозят в 2 раза сильнее задних? Считать, что почти всё тепло передаётся колодкам, так как теплопроводность 
колёс маленькая, скорость автомобиля перед началом торможения 180 км/ч, масса 500 кг. Теплоёмкость чугуна принять 
за 400 Дж/(кг$\cdot^{\circ}$С), температура плавления чугуна $1200^{\circ}C$, в начале торможения температура колодок $50^{\circ}C$. Плотность чугуна принять равной 7500 кг/м$^3$
\item [4.] Особенностью устройства тормозных колодок является их самотормозящее действие – сила трения 
может расти неограниченно. Чтобы понять этот эффект, рассмотрите следующую задачу. Опирающаяся на доску тяжёлая 
балка может поворачиваться в шарнире А вокруг горизонтальной оси. При каком условии её невозможно выдернуть вправо? В расчётах пренебречь деформацией балки и доски, считать коэффициент трения постоянным.  
\item [5.] При расчётах торможения автомобиля бывает важно учесть зависимость силы трения от скорости. 
Предположим, в условиях пункта 2, что к постоянной силе трения $200$ Н надо добавить поправку, пропорциональную скорости автомобиля с некоторым коэффициентом пропорциональности. Чему равен этот коэффициент, если оказалось, что тормозной путь равен $L = 100$ метров, а время торможения равно $t = 10$ секунд?
\end{enumerate}