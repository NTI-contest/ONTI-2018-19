\solutionSection

\begin{enumerate}
    \item Скорость охлаждения за счёт теплопроводности пропорциональна мощности теплового потока и обратно пропорциональна теплоёмкости. Тепловой поток пропорционален площади тормозной колодки:
    $$ \left\{
        \begin{aligned}
            \frac{S_2}{h_2^2}:\frac{S_1}{h_1^2} = 3;\\
            S_1 \cdot h_1=S_2 \cdot h_2;
        \end{aligned}
    \right.
    \Rightarrow \frac{S_2}{S_1}^3=3 \Rightarrow  \frac{S_2}{S_1} = \sqrt[3]{3}$$   

    Следовательно, скорость охлаждения увеличилась в  $\sqrt[3]{3} \approx 1.44$ раз.
    
    \answerMath{$\frac{S_2}{S_1} \sqrt[3]{3} \approx 1.44$.}

    \markSection

    \begin{itemize}
        \item Указано что скорость охлаждения пропорциональна тепловому потоку и обратно ппропрциональна теплоемкости – 3 балла
        \item Указано как изменится тепловой поток – 3 балла
        \item Получено соотношение для отношения скоростей – 2 балла
        \item Получен правильный числовой ответ – 2 балла
    \end{itemize}

    \item Сила трения колодок в одном колесе равна $0.4 \cdot 200 = 80$ Н. Так как колёса не проскальзывают, то мощность торможения можно считать равной произведению суммы всех сил трения колодок на скорость автомобиля:
    $$N=4 \cdot 80 \cdot \nu=\frac{\Delta E}{\Delta t}, \: E=\frac{m \cdot \nu^2}{2} \rightarrow \frac{\Delta E}{\Delta t}= m \cdot \nu \cdot a.$$

    Отсюда получаем:
    $$320=m \cdot a; \: a=\frac{320}{500}=0.64 \: \text{м/c}^2 $$
    
    Таким образом, торможение будет равноускоренным. Время торможения равно:
    $$t=\frac{\nu}{a}=\frac{100}{0.64} \cdot \frac{1000}{3600} \: \text{c}=43.4 \: \text{с}$$

    Тормозной путь составит: 
    $$L=\frac{\nu \cdot t}{2}=\frac{100 \cdot 43.4}{2} \cdot \frac{1000}{3600}=602.8 \: \text{м}$$

    \answerMath{$L=\frac{\nu \cdot t}{2}=602.8$ м, $t=\frac{\nu}{a}=43.4$ сек.}

    \markSection

    \begin{itemize}
        \item Записан второй закон Ньютона и закон сохранения энергии, позволяющие оченить время торможения – 2 балла
        \item Получено выражение для времени торможения – 3 балла
        \item Получено выражение для тормозного пути – 3 балла  
        \item Получено правильное числовое значение для времени – 1 балл
        \item Получено правильное числовое значение для пути – 1 балл
    \end{itemize}

    Если передние колёса тормозят в 2 раза сильнее, то на них выделяется в 2 раза больше мощность, следовательно, на одном переднем колесе выделяется треть всей мощности. Так как потерями тепла пренебрегаем, то можно считать, что треть всей кинетической энергии перешла в нагрев колодки переднего колёса ($360 \: \text{км/ч} = 50 \: \text{м/c}$).

    $$\frac{m \cdot \nu^2}{2}=\rho \cdot S \cdot d \cdot c \cdot \Delta T \rightarrow d=\frac{m \cdot \nu^2}{2 \cdot S \cdot \rho \cdot c \cdot \Delta T}=$$
    $$=\frac{500 \cdot 50^2}{2 \cdot 3 \cdot 50 \cdot 10^{-4} \cdot 7500 \cdot 400 \cdot 1150} \approx 1.2 \: \text{см}.$$

    \answerMath{$d=\frac{m \cdot \nu^2}{2 \cdot S \cdot \rho \cdot c \cdot \Delta T} \approx 1.2$ см.}

    \markSection

    \begin{itemize}  
        \item Указано (или использовано), что на передних колесах будет выделяться треть всей мощности – 3 балла
        \item Записан закон сохранения энергии – 3 балла
        \item Получено выражение для толщины колодки – 2 балла
        \item Получено правильное числовое значение – 2 балла
    \end{itemize}

    \item Рассмотрим прежде всего действующие силы. 
    
    \putImgWOCaption{7cm}{2}
    

    На балку действуют сила тяжести $m \cdot \overrightarrow{g}$, нормальная сила реакции доски $\overrightarrow{N}$, сила трения со стороны доски $\overrightarrow{F}$, направленная в сторону движения доски, и сила реакции шарнира. Направление последней силы заранее неизвестно, но оно нам и не нужно, так как можно рассматривать моменты сил, действующих на балку, относительно оси вращения, а вычисление самой этой силы нам не нужно. Тогда это уравнение имеет вид:
    $$\frac{mg}{2} \cdot sin \beta - N \cdot sin \beta -F \cdot cos \beta =0$$

    Теперь запишем второй закон Ньютона для сил, действующих на доску:
    $$T-F-F_1=0$$
    $$N_1-m_1 g-N=0$$
    
    Для сил трения можно записать: $$F=\mu \cdot N_1,\: F_1= \mu_1 \cdot N_1.$$

    С помощью этих уравнений можно определить:
    $$N=\frac{sin\beta}{sin \beta -\mu \cdot cos \beta} \cdot \frac{mg}{2}; \: T= \mu_1 m_1 g+\frac{\mu_1+\mu}{1-\mu \cdot ctg \beta} \cdot \frac{mg}{2}\:-\: \text{если} \mu \cdot ctg \beta < 1$$  
    
    Получается, доску невозможно выдвинуть вправо при $\mu \cdot ctg \beta \geq 1$ . Это связано с тем, что момент силы трения направлен так, что приводит к увеличению нормальной реакции N, а значит увеличивается сила трения скольжения, что при $\mu \cdot ctg \beta \geq 1$ приводит к неограниченному возрастанию силы трения при попытке выдвинуть доску вправо.
    \answerMath{$\mu \cdot ctg \beta \geq 1$.}

    \markSection

    \begin{itemize}
        \item Записано уравнение моментов сил – 3 балла
        \item Записан второй закон Ньютона – 3 балла
        \item Получено условие на невозможность сдвига – 4 балла 
    \end{itemize}

    \item Аналогично пункту 2:
    $$N=4 \cdot 200 \cdot \nu \cdot 0.4 \cdot (1+k \cdot \nu)=\frac{\Delta E}{\Delta t}=m \cdot \nu \cdot a$$

    Отсюда получаем:
    $$\frac{\Delta \nu}{\Delta t}=0.64 \cdot (1+k \cdot \nu); \: \Delta \nu=0.64 \cdot (\Delta t+k \cdot \Delta s)$$    
    Так как коэффициенты постоянны, то от малых приращений можно перейти к большим конечным:  $$\nu=0.64 \cdot (t+k \cdot s) \rightarrow k=\frac{\nu-0.64 \cdot t}{0.64 \cdot s}=\frac{\frac{100}{3.6}-0.64 \cdot 10}{0.64 \cdot 100}=0.334  \: \text{(Н} \cdot \text{с)/м} .$$
    
    \answerMath{$0.334  \: \text{(Н} \cdot \text{с)/м}$.}

    \markSection

    \begin{itemize}
        \item Записан второй закон Ньютона и закон сохранения энергии, позволяющие оценитьмощность при торможении – 2 балла
        \item Получено выражение для ускорения – 3 балла
        \item Записано выражение для коэффициента – 3 балла
        \item Получен верный численный ответ – 2 балла
    \end{itemize}
\end{enumerate}