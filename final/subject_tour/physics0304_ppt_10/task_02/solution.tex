\solutionSection

\begin{enumerate}
    \item Если пренебречь переотражением отражённого звука, то новая длительность импульса должна определяться разностью времён прихода звука, отражённого от ближней точки квадрата и отражённого от угла квадрата. В первом случае звук летит 1 метр, во втором случае $\sqrt{2}$ метра. Разность времён равна: $\Delta t = \frac{\sqrt{2}-1}{340}=1218$ мкс. Так как исходная длительность была 1000 мкс, то она становится 2218 мкс. Длительность становится больше в 2.218 раз.
    
    \answerMath{в 2.218 раз.}

    \markSection

    \begin{itemize}
	    \item Найдено выражение для разности хода звука – 5 баллов
	    \item Получено верное числовое значение – 5 баллов
    \end{itemize}

    Такой конструкцией будет окружность достаточного малого радиуса R, чтобы время прихода отражённого звука была не больше длительности самого импульса:
    
    $$\tau \geq \frac{2 R}{\nu } \rightarrow R \leq \frac{\nu  \cdot t}{2}= \frac{340\cdot 0.001}{2}=0.17\: \text{м}$$  

    \answerMath{например, окружность радиуса 17 см.}

    \markSection
    
    \begin{itemize}
        \item Указано, что конструкцией будет окружность – 2 балла
	    \item Дано объяснение, помечу именно окружность – 2 балла
	    \item Найдено, что максимальный радиус этой окружности будет равен $\frac{\nu \cdot t}{2}$ – 2 балла
	    \item Указано, что радиус может быть и меньше максимального – 2 балла
	    \item Получены верные численные значения – 2 балла.
    \end{itemize}

    \item При идеальном зеркальном отражении на приёмник попадает звуковая волна только с двух направлений – от центра и от точки границы, находящейся на линии, соединяющей центр шара и приёмник. Сначала в приёмник попадает волна, которая прошла расстояние $x$, затем она же отражённая – прошла расстояние $(2R-x)$. Потом доходит волна, отражённая от противоположной точки – она прошла расстояние $2R+x$. После она же отражённая – прошла расстояние $(4R-x)$ и так далее. Разности пройденных расстояний равны $(2R-2x)$ и $2х$ между соседними приходами импульсов. Итак, получаем систему неравенств: 
    $$ \left\{
        \begin{aligned}
            2R-2x \leq \tau \cdot \nu \\
            2x \leq \tau \cdot \nu.
        \end{aligned}
        \right.$$  
    Если $R= \tau \cdot \nu$, то эта система имеет единственное решение $$x = 0.5 \tau \nu  = 17 \: \text{см}.$$
    
    \answerMath{на расстоянии 17 см.}

    \markSection
    
    \begin{itemize}
	    \item Указано, какие волны приходят в приемник – 2 балла
	    \item Указаны разности хода между соседними импульсами – 2 балла
	    \item Замечено, что $R= \tau \cdot \nu$  – 2 балла
	    \item Найдено выражение для расстояния  – 2 балла
	    \item Получены верные численные значения – 2 балла
    \end{itemize}

    \item Мощность нагрева пропорциональна интенсивности звука, а интенсивность звука обратно пропорциональна квадрату расстояния. Минимальное расстояние до стены в $\sqrt{3}$ раза меньше максимального, значит излучаемая мощность там будет в 3 раза больше. Тепло рассеивается в окружающую среду пропорционально разности температур, следовательно:  $T_{max}-T=3\cdot (T_{min}-T)$. По условию $T_{max}=T_{min}+10$. $$T_{min}+10-T=3\cdot (T_{min}-T); \: 2\cdot T_{min}=2\cdot T+10\rightarrow T_{min} - T.$$
    
    \answerMath{на 5 К больше.}

    \markSection

    \begin{itemize}
	    \item Учтено, что интенсивность звука и, следовательно, мощность нагрева будет падать обратно пропорционально расстоянию – 2 балла
	    \item Указаны места с минимальной и максимальной мощностью нагрева – 2 балла
	    \item Записано уравнение теплового баланса – 2 балла
	    \item Найдено выражение для разницы температур  – 2 балла
	    \item Получены верные численные значения – 2 балла
    \end{itemize}

    \item Для расчёта движения луча звука, идеально отражающегося от гладкой поверхности, можно отражать не луч, а всю область пространства, в которой он распространяется, рассматривая его движение в «зазеркалье». Отразим пирамиду от каждой грани при каждом прохождении луча. Быстрее всего может он вернуться обратно тогда, когда от каждой грани было ровно одно отражение. В силу перпендикулярности рёбер плоскости граней при симметрии относительно друг друга остаются на месте. Три таких отражения равносильны центральной симметрии относительно вершины пирамиды. Длина любого луча, идущего от точки с основания, не может быть больше расстояния между плоскостью основания и центрально симметричной к ней.  Этот минимум достигается, поэтому она равна удвоенному значению высоты пирамиды. Объём пирамиды равен: $\nu =a^3=\frac{1}{3} S\cdot h$.  Основание представляет собой треугольник со сторонами $\sqrt{13}$, $\sqrt{10}$, $\sqrt{5}$. Можно посчитать любым способом, что его площадь равна 3.5. 
    
    Тогда $h = \frac{3}{3.5} = \frac{6}{7}$. Следовательно, искомый путь равен $\frac{12}{7}$ метра, звук пройдёт его за время $t = \frac{12}{340 \cdot 7} = 5.04$ мс.

    \answerMath{5.04 мс.}

    \markSection

    \begin{itemize}
        \item Верно рассмотрено распространение лучей – 3 балла
	    \item Найдено выражение для времени пути – 5 баллов
	    \item Получены верные численные значения – 2 балла
    \end{itemize}
\end{enumerate}