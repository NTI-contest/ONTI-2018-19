\assignementTitle{Идеальное отражение звука}{50}{2}

Был разработан новый материал, который идеально отражает звук. Для проверки качества отражения необходимо провести несколько испытаний, параметры которых нужно определить. Считатайте, что излучатель звука излучает на частоте $f = 10$ кГц, скорость звуковой волны равна $v = 340$ м/c, длительность импульса $\tau = 1$ мс, ребро $a = 1$ м.
\begin{enumerate}
\item Излучатель и приёмник звука поставили в одной точке на плоскости.  Отражатель расположили вокруг этой точки квадратом, в котором она является центром. Во сколько раз длительность принимаемого импульса больше, чем длительность испускаемого? Считывайте, что отражение звука целиком рассеянное, $а$ переотражением отражённого звука можно пренебречь. Длина стороны квадрата равна $а = 1$ м. Источник и приемник можно считать точечными. Распространением звука по высоте пренебрегите.
\item Излучатель и приёмник звука поставили в одной точке на плоскости. Придумайте такую конструкцию (включая ее геометрические размеры) из отражателя звука на плоскости, которая так отражает импульсы излучаемого звука, что отражённый звук звучит непрерывно, в предположении, что рассеянным отражением звука можно пренебречь (то есть отражение звука происходит в точности по закону зеркального отражения), а при отражении от поверхности его интенсивность не уменьшается. Излучатель испускает 1 импульс.  Источник и приемник можно считать точечными.
\item Точечный излучатель поставили в центр шара радиуса $R = 34$ см, поверхность которого покрыта отражателем звука. На каком минимальном расстоянии от центра шара нужно расположить приёмник звука, чтобы звук в приёмнике звучал без перерыва? Условия излучения и отражения такие же, как в пункте 2.
\item Точечный излучатель поставили в центр куба с ребром а, поверхность которого покрыта идеальным поглотителем звука, переизлучения не происходит. При поглощении звука поглотитель нагревается, тепло он не проводит, но охлаждается с поверхности за счёт теплопроводности. Оказалось, что в результате непрерывного излучения в разных точках поглотителя поддерживается постоянная во времени температура, большая окружающей среды и разная в различных точках поверхности куба. Максимальная температура оказалась на 10 К больше минимальной.  На сколько градусов по Кельвину минимальная температура больше температуры окружающей среды? Считать в расчётах, что излучатель испускает сферически симметричные волны звука.
\item Из отражателя изготовили пирамиду, боковые грани которой перпендикулярны друг другу, а боковые рёбра равны $a, 2a, 3a$ соответственно. Точечный излучатель и приёмник расположили в двух разных точках основания пирамиды так, чтобы импульс звука, отражённый ровно по одному разу от каждой боковой поверхности, вернулся за наименьшее время. Чему равно это время? Условия излучения и отражения такие же как в пункте \textbf{2}. 

\end{enumerate}
