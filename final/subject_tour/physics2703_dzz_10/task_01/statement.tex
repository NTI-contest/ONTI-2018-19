\assignementTitle{Космическая фотосъемка}{40}{1}

Спутник вращается по круговой орбите высотой $H=500$ км. Съёмка Земли выполняется с помощью оптико-электронной 
камеры. Фотоприёмник представляет собой матрицу размером 3000$\cdot$1800 пикселей, широкая сторона матрицы 
расположена поперёк направления полёта. Размер пикселя матрицы составляет $\delta=12$ мкм. Радиус Земли 6370 км, 
постоянная Планка $6.63\cdot 10^{-34}$ Дж$\cdot$с.

\begin{enumerate}
    \item Определить скорость подспутниковой точки («тени» спутника на поверхности Земли). 
    \item Какое максимальное время выдержки можно установить для камеры (при съёмке вертикально вниз), чтобы смаз 
    изображения, имеющего разрешение $L=3$ метра, был не более 1 пикселя? Разрешением изображения принят 
    размер проекции пикселя на поверхность Земли.
    \item Для обеспечения съёмки объектов, расположенных не под трассой полёта, спутник имеет 
    возможность отклоняться по крену. В этом случае оптическая ось объектива расположена не вертикально, 
    а под некоторым углом к вертикали. Отклонение производится только в плоскости, перпендикулярной 
    направлению полёта спутника (объектив смотрит немного «в бок»). Во сколько раз будет больше разрешение 
    снимков при съёмке с креном (отклонением оси объектива от вертикали) до $\theta=20$ градусов? Принять, 
    что разрешение снимка определяется как поперечный (по направлению поперёк полета) размер проекции пикселя, 
    соответствующего центру кадра. Учесть, что проекцией пикселя в этом случае является не квадрат, а 
    равнобедренная трапеция.
    \item Полярная орбита спутника такова, что она проходит над Москвой с юга на север. Центр Москвы имеет 
    координаты 55.75$^{\circ}$  с.ш. и 37.6$^{\circ}$  в.д Изначальн спутни производил съёмку (с креном) города 
    Калуги, имеющего координаты 54.5$^{\circ}$  с.ш. и 36.3$^{\circ}$ в.д. Съёмка Калуги производилась в момент, 
    когда спутник находился на широте Калуги и долготе Москвы. Какое угловое ускорение должен развивать 
    спутник при перенацеливании по крену, чтобы успеть после съёмки Калуги навестись по углу крена на город 
    Ростов Ярославской области, имеющий координаты 57.2$^{\circ}$ с.ш. и 39.4$^{\circ}$  в.д ? Съёмка Ростова 
    производилась в момент, когда спутник находился на широте Ростова и долготе Москвы. Отклонения оси 
    объектива по тангажу нет, то есть ось объектива всегда находится в плоскости, перпендикулярной направлению 
    полёта. Предельная угловая скорость спутника при перенацеливании очень большая. Замедление спутника при 
    остановке вращения по модулю равно угловому ускорению при разгоне. 
\end{enumerate}

\putImgWOCaption{14cm}{1}