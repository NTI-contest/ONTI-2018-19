\solutionSection

\begin{enumerate}
    \item Орбитальная скорость спутника $V_\text{орб}$
    $$V_\text{орб}=\sqrt\frac{G \cdot M_\text{зем}}{R_\text{зем}+H}$$
    Тогда скорость подспутниковой точки с учётом круглости Земли можно определить как:
    \begin{eqnarray}\nonumber
    V_\text{пст}=V_\text{орб}\cdot\frac{R_\text{зем}}{R_\text{зем}+H}=\frac{R_\text{зем}}{R_\text{зем}+H}\cdot\sqrt{\frac{G\cdot M_\text{зем}}{R_\text{зем}+H}}=
    \\\nonumber
    =\frac{6370\cdot 10^3}{6870\cdot 10^3}\cdot\sqrt{\frac{6,67\cdot10^{-11}\cdot 5.97\cdot 10^24}{6870\cdot 10^3}}=7059.18 \text{ м/с}
    \end{eqnarray}
    
    \answerMath {7059.18  м/с.}
    \item Сдвигу изображения на 1 пиксель соответствует перемещение подспутниковой точки на величину проекции пикселя. Тогда:
    $$V_\text{пст}\cdot T=L$$
    Отсюда находим время выдержки Т:
    $$T=\frac{L}{V_\text{пст}} =\frac{3}{7059.18}=0.425\text{ мс}$$
    \answerMath {0.425 мс.}
    \item Для обеспечения перекрытия нужно, чтобы период съёмки не превышал время пролёта подспутниковой точкой короткой стороны снимка (именно она расположена вдоль вектора скорости). Тогда:
    $$\frac{1}{\nu_\text{кадр}} =\frac{1800\cdot L}{V_\text{пст}}$$ 
    Для кадровой частоты получаем:
    $$\nu_\text{кадр}=\frac{V_\text{пст}}{1800\cdot L}=\frac{7059.18}{1800\cdot 3}=1.31 \text{ Гц}$$
    
    \answerMath {1.31 Гц.}
    \item Так как высота орбиты $H\ll R_\text{зем}$, а угол 20 градусов достаточно мал, то поверхность Земли можно считать плоской. \\
    Очевидно, что в случае съёмки с креном квадратному пикселю матрицы соответствует его трапециевидная проекция на поверхность Земли. Пусть Х – направление вдоль полёта, а У – поперёк полёта.  
    Тогда для центра поля зрения размеры трапеции (средняя линия $a_x$ и высота $a_y$) определяются через дальность наблюдения $L_\text{набл}$ как:
    $$a_x=\frac{L_\text{набл}}{f}\cdot\delta$$
    $$a_y=\frac{L_\text{набл}}{f}\cdot\delta\cdot\frac{1}{\cos\theta}$$ 
    
    \putImgWOCaption{12cm}{2}
    \begin{center}
        Поперечный разрез, спутник летит ОТ наблюдателя рисунка. \\
    \end{center}
    Так как размер пикселя $\delta\ll f$ (фокусное расстояние), то выполняется примерное равенство углов $\theta'=\theta$ (так как угловой размер пикселя очень мал, треугольники с зелёными сторонами в сильной степени остроугольные). \\
    Тогда $$a_y=\frac{L_\text{набл}}{f}\cdot\delta\cdot\frac{1}{\cos\theta'}\sim \frac{L_\text{набл}}{f}\cdot\delta\cdot\frac{1}{\cos\theta} =\frac{H}{f}\cdot\delta\cdot\frac{1}{(\cos\theta)^2}$$\\\\
    Интерес представляет $a_y$ , так как она больше $a_x$. Учтём, что разрешение снимка L при наблюдении в надир известно.
    $$a_y=\frac{L_\text{набл}}{f}\cdot\delta\cdot\frac{1}{\cos\theta}=\frac{H}{f}\cdot\delta\cdot\frac{1}{(\cos\theta)^2} =L\cdot\frac{1}{(\cos\theta)^2}$$ 
    Тогда отношение проекций пикселя для двух случаев (съёмка с креном и вертикально вниз)
    $$k=\frac{L\cdot\frac{1}{(\cos\theta)^2}}{L}=\frac{1}{(\cos\theta)^2}=\frac{1}{(0.9397)^2} =1.13$$
    
    \answerMath {в 1.13 раза.}
    \item Обозначим широты городов как s, а долготы как d. Характерные расстояния на местности составляют не более 300 км, значит, Землю можно принять плоской.
    Длина дуги 1 градуса географической параллели на широте Калуги составляет: 
    $$l_\text{калуга}=\frac{\pi}{180}\cdot R_\text{зем}\cdot\cos{s_\text{калуга}}=\frac{3.1416}{180}\cdot 6370\cdot 10^3 \cdot \cos 54.5 =64561\text{ м/градус}$$
    Аналогично для Ростова:
    $$l_\text{ростов}=\frac{\pi}{180}\cdot R_\text{зем}\cdot\cos{s_\text{ростов}}=\frac{3.1416}{180}\cdot 6370\cdot 10^3 \cdot \cos 57.2 =60226\text{ м/градус}$$
    Расстояние от подспутниковой точки до Калуги в момент съёмки этого города составит $l_\text{калуга}\cdot(d_\text{москвы}-d_\text{калуга})$ , 
    для Ростова аналогично в момент его съёмки $l_\text{ростов}\cdot(d_\text{ростов}-d_\text{москвы})$
    Тогда угол крена при съёмке Калуги:
    $$\theta_\text{калуга}=-\arctg \left( \frac{l_\text{калуга}\cdot(d_\text{москвы}-d_\text{калуга})}{H}\right)=$$
    $$=-\arctg\left(\frac{64.561\cdot(37.6-36.1)}{500}\right) =-10.961 \: \text{градус}$$
    Аналогично, угол крена при съёмке Ростова:
    \begin{eqnarray}\nonumber
    \theta_\text{ростов}=-\arctg \left( \frac{l_\text{ростов}\cdot(d_\text{ростов}-d_\text{москвы})}{H}\right)=
    \\\nonumber
    =-\arctg\left(\frac{60.226\cdot(39.4-37.6)}{500}\right)=-12.233 \text{градуса}
    \end{eqnarray}
    \textit{Разные знаки углов крена нужны, так как Калуга и Ростов расположены по разные стороны от трассы полёта спутника: Калуга – к западу, а Ростов~– к востоку.}\\
    При перенацеливании спутник должен повернуться вокруг своей продольной оси на угол $\Delta\theta$
    $$\Delta\theta=\theta_\text{ростов}-\theta_\text{калуга}=23.194 \text{ градуса}$$
    Максимально допустимое время перенацеливания определяется как время смещения подспутниковой точки с широты Калуги на широту Ростова:
    $$T_\text{пер}=\frac{\frac{\pi}{180}\cdot R_\text{зем}\cdot( s_\text{ростов}-s_\text{калуга})}{V_\text{пст}} =\frac{\frac{3.1416}{180}\cdot6370\cdot10^3\cdot(57.2-54.5)}{7059.18}$$
    $$=\frac{300179.88\text{ м}}{7059.18\text{ м/с}}=42.523\text{ с}$$
    Так как ограничения на угловую скорость спутника при повороте по крену нет, то первую половину времени перенацеливания он разгоняется, а вторую половину-замедляется. Тогда с учётом того, что угловое ускорение и замедление одинаковы по модулю:
    $$\Delta\theta=2\cdot\frac{\varepsilon\cdot\left(\frac{T_\text{пер}}{2}\right)^2}{2}=\varepsilon\cdot\left(\frac{T_\text{пер}}{2}\right)^2$$
    Тогда угловое ускорение $\varepsilon$ определяется как:
    $$\varepsilon=\frac{4\cdot\Delta\theta}{T_\text{пер}^2 }=\frac{4\cdot23.194}{42.523^2} =0.0513\text{ градус/с}^2$$
    
    \answerMath {0.0513 градус/с$^2$.}
\end{enumerate}