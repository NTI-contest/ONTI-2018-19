\solutionSection

\begin{enumerate}
    \item Орбитальная скорость спутника $V_\text{орб}$
    $$V_\text{орб}=\sqrt\frac{G \cdot M_\text{зем}}{R_\text{зем}+H}$$
    Тогда скорость подспутниковой точки с учётом круглости Земли можно определить как:
    \begin{eqnarray}\nonumber
    V_\text{пст}=V_\text{орб}\cdot\frac{R_\text{зем}}{R_\text{зем}+H}=\frac{R_\text{зем}}{R_\text{зем}+H}\cdot\sqrt{\frac{G\cdot M_\text{зем}}{R_\text{зем}+H}}=
    \\\nonumber
    =\frac{6370\cdot 10^3}{6870\cdot 10^3}\cdot\sqrt{\frac{6,67\cdot10^{-11}\cdot 5.97\cdot 10^24}{6870\cdot 10^3}}=7059.18 \text{ м/с}
    \end{eqnarray}
    
    \answerMath {7059.18  м/с.}
    \item Сдвигу изображения на 1 пиксель соответствует перемещение подспутниковой точки на величину проекции пикселя. Тогда:
    $$V_\text{пст}\cdot T=L$$
    Отсюда находим время выдержки Т:
    $$T=\frac{L}{V_\text{пст}} =\frac{3}{7059.18}=0.425\text{ мс}$$
    \answerMath {0.425 мс.}
    \item Для обеспечения перекрытия нужно, чтобы период съёмки не превышал время пролёта подспутниковой точкой короткой стороны снимка (именно она расположена вдоль вектора скорости). Тогда:
    $$\frac{1}{\nu_\text{кадр}} =\frac{1800\cdot L}{V_\text{пст}}$$ 
    Для кадровой частоты получаем:
    $$\nu_\text{кадр}=\frac{V_\text{пст}}{1800\cdot L}=\frac{7059.18}{1800\cdot 3}=1.31 \text{ Гц}$$
    
    \answerMath {1.31 Гц.}
    \item Так как высота орбиты $H\ll R_\text{зем}$, а угол 20 градусов достаточно мал, то поверхность Земли можно считать плоской. \\
    Очевидно, что в случае съёмки с креном квадратному пикселю матрицы соответствует его трапециевидная проекция на поверхность Земли. Пусть Х – направление вдоль полёта, а У – поперёк полёта.  
    Тогда для центра поля зрения размеры трапеции (средняя линия $a_x$ и высота $a_y$) определяются через дальность наблюдения $L_\text{набл}$ как:
    $$a_x=\frac{L_\text{набл}}{f}\cdot\delta$$
    $$a_y=\frac{L_\text{набл}}{f}\cdot\delta\cdot\frac{1}{\cos\theta}$$ 
    
    \putImgWOCaption{12cm}{2}
    \begin{center}
        Поперечный разрез, спутник летит ОТ наблюдателя рисунка. \\
    \end{center}
    Так как размер пикселя $\delta\ll f$ (фокусное расстояние), то выполняется примерное равенство углов $\theta'=\theta$ (так как угловой размер пикселя очень мал, треугольники с зелёными сторонами в сильной степени остроугольные). \\
    Тогда $$a_y=\frac{L_\text{набл}}{f}\cdot\delta\cdot\frac{1}{\cos\theta'}\sim \frac{L_\text{набл}}{f}\cdot\delta\cdot\frac{1}{\cos\theta} =\frac{H}{f}\cdot\delta\cdot\frac{1}{(\cos\theta)^2}$$\\\\
    Интерес представляет $a_y$ , так как она больше $a_x$. Учтём, что разрешение снимка L при наблюдении в надир известно.
    $$a_y=\frac{L_\text{набл}}{f}\cdot\delta\cdot\frac{1}{\cos\theta}=\frac{H}{f}\cdot\delta\cdot\frac{1}{(\cos\theta)^2} =L\cdot\frac{1}{(\cos\theta)^2}$$ 
    Тогда отношение проекций пикселя для двух случаев (съёмка с креном и вертикально вниз)
    $$k=\frac{L\cdot\frac{1}{(\cos\theta)^2}}{L}=\frac{1}{(\cos\theta)^2}=\frac{1}{(0.9397)^2} =1.13$$
    
    \answerMath {в 1.13 раза.}
\end{enumerate}