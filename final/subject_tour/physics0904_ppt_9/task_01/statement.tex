\assignementTitle{Задача 2. «Металл-органические каркасы»}{}{2}
\textbf{Магниторельсовый тормоз} — железнодорожный тормоз, тормозной эффект которого создаётся за счёт взаимодействия тормозной колодки непосредственно с рельсом; сила, с которой тормозная колодка прижимается к рельсу при этом образуется за счёт магнитного поля, создаваемого электромагнитами и притягивающего тормозную колодку и рельс друг к другу. Такой тормоз состоит из двух башмаков, каждый из которых расположен между колёсами, и четырёх цилиндров подвески башмаков, в которых размещены пружины, удерживающие башмаки на расстоянии 140—150 мм над головкой рельса.\\
Для характеристики работы такого тормоза можно ввести коэффициент торможения, это величина, показывающая во сколько раз магниторельсовый тормоз повышает силу трения колодок. Например, если коэффициент равен 200\%, то магниторельсовый тормоз увеличивает силу трения в 2 раза.
\begin{enumerate}
\item Для магниторельсового тормоза характерно резкое торможение – сила прижатия башмака к рельсу сразу принимает максимальное значение. Во сколько раз увеличится тормозной путь, если скорость перед началом торможения будет больше в 3 раза?
\item На сколько процентов уменьшится длина тормозного пути при применении магниторельсового тормоза, если коэффициент торможения равен 210\%? Считать, что сила прижатия башмака к рельсу сразу достигает максимального значения.
\item Чему равна магнитная сила прижатия, действующая на рельсы со стороны включенного магниторельсового тормоза, если полная сила давления рельса на башмак достигает 100 кН, а коэффициент торможения при этом равен 140\%?
\item Пусть коэффициент трения тормозной колодки равен 0.4, коэффициент торможения магниторельсового поезда 160\%, а магнитное поле прижимает с силой 70 кН. Какой массы должен быть поезд, чтобы его тормозной путь при движении со скоростью 120 км/ч был не больше 700 метров?
\item Тормозная система плавного торможения устроена так, что сила торможения увеличивается пропорционально уже пройденному тормозному пути, с постоянным коэффициентом пропорциональности. Во сколько раз увеличится тормозной путь, если скорость перед началом торможения будет больше в 2 раза?
\end{enumerate}