\solutionSection

\begin{tabular}{l|}
    Дано: \\
    $v = 1.5$ м/с \\
    $h = 50$ м \\
    \hline \\
    $v_\text{мин}$, $H$, $\Delta H$, $\Delta v_\text{мин}$ - ?
\end{tabular}

Минимальная скорость достигается в наивысшей точке траектории, $v_\text{мин} = 0$ м/с.

$$\left\{
  \begin{array}{c}
    \Delta h = \dfrac{v^2 - v_\text{мин}^2}{2g} \\
    H = h + \Delta h \\
  \end{array}
\right.$$

$$H=h+\frac{v^2 - v_\text{мин}^2}{2g}=50+\frac{12^2}{2 \cdot 9.8} \approx 57.35 \space \text{м}.$$

Учтем силу сопротивления воздуха.

Минимальная скорость также достигается в наивысшей точке траектории и равна нулю.

Следовательно $\Delta v_\text{мин} = 0$.

$$ma = mg + F_c; \space a = g + F_c/m = 9.8 + 2.2/1 = 12 \text{м/с}^2$$
$$H_1 = h + \frac{v^2 - v_\text{мин}^2}{2a} = 50 + \frac{12^2}{2 \cdot 12} \approx 56 \: \text{м}$$
$$\Delta H = H - H_1 = 1.35 \text{м}$$

\answerMath{$H=50.11\text{м}; \space v_\text{мин} = 0 \: \text{м/с}$. С учетом $F_c$ минимальная скорость не изменится, а высота уменьшится на $1.35$~м.}