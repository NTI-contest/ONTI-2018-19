\solutionSection
\begin{enumerate}
    \item [а)] Можно последовательно ставить точки внутри многоугольника, соединять со всеми вершинами этого многоугольника и считать сколько треугольников добавилось. Изначально треугольников не было. Поставим точку и соединим со всеми вершинами. Появятся 10 треугольников. Потом ставим следующую точку внутрь какого-то треугольника. Один треугольник вычитается и добавляется три, т.е. плюс два треугольника. Всего получается $10+2018\cdot2=4046.$
    
    \item [б)] Пусть при некоторой триангуляции получились $n$ треугольников. Сосчитаем сумму всех углов всех треугольников. При каждой точке внутри десятиугольника сумма углов равна $360^\g$, а сумма углов в вершинах равна сумме внутренних углов десятиугольника. Получаем: $360^\g\cdot 2019+180^\g\cdot (10-2)=180^\g\cdot n.$ Откуда $n=2\cdot2019+8=4046.$ 
\end{enumerate}



