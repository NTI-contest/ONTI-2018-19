\solutionSection

Число AR-архитекторов очевидно кратно 10, поскольку «каждый десятый 
AR-архитектор является 3D-дизайнером». Значит, их либо 10, либо 20. 
20 AR-архитекторов быть не может, поскольку такое их число будет противоречить 
условию, согласно которому только «каждый пятый 3D-дизайнер является 
AR-архитектором». Следовательно, AR-архитекторов – 10 человек.
Поскольку «каждый десятый AR-архитектор является 3D- дизайнером», 
число AR-архитекторов, которые являются 3D- дизайнерами $10:10 = 1$ человек. 
Так как «каждый пятый 3D-дизайнер является AR-архитектором», то в компании 
$1 \cdot 5 = 5$ 3D-дизайнеров.

\putImgWOCaption{6cm}{1}

Тогда 9 (чистых AR-архитекторов) + 1 (AR-архитектор и 3D-дизайнер) + 4 (чистых 3D-дизайнера) = 14 человек. 
Получается, что «чистых» программистов $20-14 = 6$.

Известно, однако, что не все программисты «чистые»: среди них треть является AR-архитекторами и треть – 3D-дизайнерами. Следовательно, общее число программистов должно быть кратно 3 и их число > 6 (поскольку только чистых программистов уже 6). Значит, их может быть 9, 12, 15 или 18 – последовательно рассмотрим все 4 варианта.

\begin{enumerate}
    \item Допустим, общее число программистов – 9. Треть от 9 равна 3, а значит, 3 программиста должны оказаться AR-архитекторами и 3 – 3D-дизайнерами. При этом мы знаем, что 6 программистов – «чистые», т.е. «нечистых» остается всего 3. Получается, что 3 программиста-AR-архитектора и 3 программиста-3D-дизайнера – это одни и те же люди
    \putImgWOCaption{6cm}{2}
    \item Допустим, общее число программистов – 12. Треть от 12 равна 4, а значит, 4 программиста должны оказаться AR-архитекторами и 4 – 3D-дизайнерами. Учитывая, что «чистых» программистов 6, «нечистых» в сумме должно быть тоже 6. Это возможно только если найдется 2 человека, которые будут одновременно программистами, 3D-дизайнерами и AR-архитекторами.
    \putImgWOCaption{6cm}{3}
    \item Допустим, общее число программистов – 15. Треть от 15 равна 5, а значит, 5 программистов должны оказаться AR-архитекторами и 5 – 3D-дизайнерами. При этом «чистых» программистов 6, и «нечистых» должно быть 9. Значит, должен быть один человек, который является одновременно программистом, AR-архитектором и 3D-дизайнером.
    \putImgWOCaption{6cm}{4}
    Это решение удовлетворяет условию задачи.
    \item Проверим последний случай, когда общее число программистов – 18. 18:3 = 6. Значит, среди программистов 6 – 3D-дизайнеры. Однако известно, что 3D-дизайнеров всего 5 (противоречие).
\end{enumerate}

\putImgWOCaption{6cm}{5}

