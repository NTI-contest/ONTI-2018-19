\assignementTitle{}{20}{1}

У правильного десятиугольника отметили все вершины и еще 2019 точек внутри. Некоторые из 2029 отмеченных точек соединили отрезками так, 
что исходный десятиугольник  оказался разбит на треугольники и каждая точка является вершиной хотя бы одного треугольника, причем любые два 
треугольника либо не имеют общих точек, либо имеют общую вершину, либо имеют общую сторону. (Такое разбиение фигур называется \textbf{триангуляцией}.) 

\begin{enumerate}
    \item[a)] Сколько треугольников получилось?
    \item[б)] Докажите, что количество треугольников не зависит от способа триангуляции.
\end{enumerate}