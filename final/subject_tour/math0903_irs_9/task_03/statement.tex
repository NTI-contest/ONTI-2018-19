\assignementTitle{}{50}{3}

На одном из полей $8\times8$ находится робот-разведчик, который каждую секунду перемещается в соседнее поле. Охранная 
система каждую секунду может проверить любые $n$ полей, есть ли там робот. Может ли охранная система за одну минуту наверняка обнаружить шпиона, если
\begin{enumerate}
    \item [а)] $n=15$;

    \item [б)] $n=9$;

    \item [в)] $n=8$?
\end{enumerate}
(Считается, что если у робота есть шанс не оказаться на проверяемом поле, то он не обнаружен.)