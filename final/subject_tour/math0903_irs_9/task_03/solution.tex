\solutionSection
\begin{enumerate}
    \item [а)] Приведем алгоритм проверки, для которого в некоторый момент робот окажется в проверямом поле. Пронумеруем поля как в таблице:
    \begin{center}
    \begin{tabular}{|c|c|c|c|c|c|c|c|}\hline
    % use packages: color,colortbl
     {1} &  {2} &  {3} &  {4} &  {5} &  {6} &  {7} &  {8}\\\hline
     {9} &  {10} &  {11} &  {12} &  {13} &  {14} &  {15} &  {16}\\\hline
     {17} &  {18} &  {19} &  {20} &  {21} &  {22} &  {23} &  {24}\\\hline
     {25} &  {26} &  {27} &  {28} &  {29} &  {30} &  {31} &  {32}\\\hline
     {33} &  {34} &  {35} &  {36} &  {37} &  {38} &  {39} &  {40}\\\hline
     {41} &  {42} &  {43} &  {44} &  {45} &  {46} &  {47} &  {48}\\\hline
     {49} &  {50} &  {51} &  {52} &  {53} &  {54} &  {55} &  {56}\\\hline
     {57} &  {58} &  {59} &  {60} &  {61} &  {62} &  {63} &  {64}\\\hline
    \end{tabular}
    \end{center}
    
    
    Сперва проверим все поля с 1 по 15. На следующей секунде проверим поля с 8 по 22, затем с 15 по 29. Каждый раз первые 7 проверяемых убираем и добавляем следующие 7. Таким образом добираемся до конца таблицы за 8 секунд. У шпиона нет возможности ``проскочить'' не оказавшись на проверяемом поле. Следовательно, охранная система в какой-то момент обнаружит шпиона не зависимо где он находился в начале и как передвигался.
    
    \item [б)] Алгоритм примерно такой же, только вместо 7 полей передвигаемся на 1 поле: на первой секунде проверяем с 1 по 9, на второй со 2 по 10, затем с 3 по 11 и т.д. Шпион будет обнаружен за не более чем 56 секунд.
    
    \item [в)] Раскрасим поля в шахматном порядке. Предположим, что шпион в первой секунде проверки оказался на черном поле. Тогда в нечетные секунды он будет в черных, а в четных секундах в белых полях. Начиная с угла пронумеруем диагонали в одном направлении с 1 по 15. Пусть угловая клетка (первая диагональ) будет черного цвета. Проверим 1 и 3 ряд клеток (черные) и еще какие-то. На второй секунде проверим 2-й и 4-й ряд (белые). Потом 5-й ряд, 6-й ряд, и т.д. по одному. Ряды 12 и 14 можем проверить вместе, так же как и ряд 13 с 15-м. Шпион не сможет пройти с непроверенных полей в проверенные, не оказавшись в проверяемом. Единственный момент, когда это можно сделать, только если предположение вначале неверно, т.е. на первой секунде шпион находился в белой клетке. Если охранная система проверит еще раз по тому же алгоритму с 16 по 30 секунду, то сможет поймать шпиона не более чем за 30 секунд.
\end{enumerate}

\additionalCriteria 
Только за пункт а) --- 10 баллов, за пункт б) --- 25 баллов (включает пункт а)), за пункт в) --- 50 баллов (включает первые два пункта).

\answerMath{а) да; б) да; в) да.}   