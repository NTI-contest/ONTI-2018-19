Считаем количество клеток на рисунке 1: 26 (клетки, пересекающие нижнюю и левую границы, не учитываются).

Считаем количество клеток на рисунке 2: 15 (клетки, пересекающие нижнюю и левую границы, не учитываются).

Формула из дополнительных материалов:

$$x=(\frac{a}{20})\cdot N\cdot k\cdot b$$
$a$ – число клеток в 20 больших квадратах

$x_1 = \left(26\cdot4\cdot\frac{20}{20}\right)\cdot225\cdot\left(\frac{1}{0.9\cdot10^{-3}}\right)\cdot10^{-3} = 26000$ – число клеток в 1 мл первого образца в 1й день \textit{(2.5 балла)}

$x_2 = \left(15\cdot4\cdot\frac{20}{20}\right)\cdot225\cdot\left(\frac{1}{0.9\cdot10^{-3}}\right)\cdot\frac{1}{25} = 600000$ – число клеток в 1 мл второго образца в 1й день \textit{(2.5 балла)}

$x_1\cdot5=26000\cdot5=\textbf{130000}$ – число клеток в 5 мл первого образца в первый день \textit{(2.5 балла)}

$x_2\cdot\frac{5}{2.5}=6000000\cdot\frac{5}{2.5}=1200000$ – число клеток в 5 мл второго образца в первый день \textit{(2.5 балла)}