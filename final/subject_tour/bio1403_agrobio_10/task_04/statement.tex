\assignementTitle{}{10}{4}

Перед вами 2 пробирки с 5 мл взвеси микроорганизмов. В первом образце количество микроорганизмов было посчитано на первый день с использованием камеры Горяева. Перед нанесением образца в камеру для взвеси из первой пробирки сделали 3 десятикратных разведения. Количество микроорганизмов во втором образце было посчитано на второй день, и перед нанесением была сделана серия из двух пятикратных разведений.

На рисунке 1 представлен фрагмент из 4х малых квадратов камеры Горяева с нанесённым первым образцом.

На рисунке 2 представлен аналогичный фрагмент камеры после нанесения второго образца.

\putImgWOCaption{8cm}{1}

Используя рисунки и дополнительные материалы «Подсчет клеток в счетной камере Горяева» рассчитайте количество клеток в каждой пробирки в первый день, если известно, что через каждые сутки количество микроорганизмов увеличивается в 2.5 раза.