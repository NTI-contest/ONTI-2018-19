\solutionSection
В профазе 1 мейоза \textit{(1.5 балла)}.

У животных происходит конъюгация хромосом, во время этого гомологичные хромосомы попарно временно сближаются, и в это время может произойти обмен генетическим материалом между гомологичными участками. \textit{(2.5 балла)}

У Инфузории \textit{(1.5 балла)}.

У инфузорий конъюгация - это половой процесс, при котором между двумя контактирующими клетками происходит перенос ядер.

Механизм этого процесса выглядит следующим образом:

\begin{enumerate}
\item 2 инфузории сближаются, между ними образуется цитоплазматический мостик
\item Микронуклеус делится мейозом, образуя 4 гаплоидных ядра, одно из которых затем делится митотически, образуя тем самым два гаплоидных ядра – пронуклеуса. 3 оставшихся ядра и макронуклеус дегенерируют.
\item Клетки обмениваются ядрами.
\item В каждой клетке пронуклеус сливается с исходным пронуклеусом, образуя диплоидный синкарион.
\item Цитоплазматический мостик рвётся, клетки расходятся. Таким образом, они обменялись генетическим материалом.
\item Синкарион затем делится несколько раз, а образовавшиеся ядра становятся макронуклеусом и микронуклеусом. \textit{(2.5 балла)}
\end{enumerate}
Ещё конъюгация есть у бактерий и водорослей \textit{(1 балл)}.