\assignementTitle{}{14}{5}

Изучая трансляцию эукариот, молекулярные биологи смогли отфильтровать не подошедшие к матрице транспортные 
РНК и установить набор подходящих тРНК в той последовательности, в которой они связывались с матричной РНК. 
Эти тРНК несут следующие антикодоны:

5'-UGA-3' 5'-AUU-3' 5'-GCG-3' 5'-UCC-3' 5'-AAC-3' 5'-GGA-3' 5'-GCA-3' 5'-CCC-3' 5'-AGG-3' 5'-GGC-3'

Установите участок последовательности белка, который был синтезирован в ходе данного эксперимента и запишите 
его последовательность аминокислот в виде однобуквенных обозначений.

Известно, что ген, кодирующий данный фрагмент белка, находится на прямой цепи ДНК, восстановите последовательность 
этого гена на ДНК, предполагая, что он содержит только экзоны. Укажите ' концы последовательности.

Сколько вообще в природе есть возможных вариантов последовательности мРНК для кодирования данного участка белка? 
Запишите число.
