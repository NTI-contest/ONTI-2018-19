\solutionSection
Уравнения:

$$N_2+8H^+ +8e \to 2NH_4^+ или N_2+8H+ \to 2NH_4^+ \textit{(1 балл)}$$
$$С_2H_2+2H^+\to C_2H_4 \textit{(1 балл)}$$

Выход реакции с ацетиленом:

Выход реакции $=\frac{c(\text{этилена})}{c(\text{этилена})+c(\text{ацетилена})} = \frac{7.5}{7.5+1.5}=0.83$

Для реакции с азотом $\frac{0.83}{3}=0.28$ \textit{(2 балла)}

$$c(\text{исходная азота})\cdot\text{выход}=c(\text{перешедшего в аммиак азота})=3.3\cdot0.28=0.924\text{ M}$$
$$n=V\cdot c = 5\cdot10^{-3}\text{ л}\cdot0.924 \:\text{ моль/л}=4.62\cdot10^{-3}\text{  моль}$$
$$m = M\cdot n=14\cdot2\cdot4.62\cdot10^{-3}=0.12936\text{ г}= 0.12936\cdot103\text{ мг в 5 г почвы} \: \textit{ (2 балла)}$$
$$0.12936\cdot10^{3*}\cdot\frac{1000}{5}=25872 \: \text{ — в кг почвы за 2 часа}\textit{ (2 балла)}$$
$\frac{25872}{2}=\textbf{12936} \: \text{ мг в кг почвы за 1 час — \textbf{верный ответ}}\textit{ (2 балла)}$