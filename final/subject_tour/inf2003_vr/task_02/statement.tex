\assignementTitle{Распознавание метки}{100}{}

Приложение дополненной реальности пытается найти на изображении, полученном с камеры телефона, метку, представляющую собой ярко раскрашенный прямоугольник. Благодаря контрастному цвету прямоугольника на этапе предварительной обработки изображение удалось преобразовать в массив из $W$ на $H$ элементов, каждый из которых равен либо '\#' (ASCII 35), если он принадлежит прямоугольнику, либо '.' (ASCII 46) в противном случае.

Алгоритм предварительной обработки неидеален, и мог совершить от $0$ до $N$ ошибок~--- выдать '\#' вместо '.' или наоборот.

В первом примере теста ошибочным может быть левый верхний элемент элемент непосредственно справа от него, элемент непосредственно снизу от него. Все остальные варианты из нуля или одной ошибки не дают прямоугольника в результате исправления.

Напишите программу, которая по данному изображению подсчитывает количество различных возможных положений метки.

\inputfmtSection

Первая строка входного файла содержит целые числа $W H N$. Следующие $H$ строк содержат по $W$ символов . и \# каждая~--
изображение после предварительной обработки.

Ограничения:

$1 \le W, H \le 100$, $0 \le N \le W \times H$

\outputfmtSection

Выходной файл должен содержать единственное целое число~--- количество возможных расположений метки.

\markSection

Решения работающие для $W, H \le 20$ оцениваются из 50 баллов. Баллы выставляются за каждый успешно пройденный тест.

\sampleTitle{1}

\begin{myverbbox}[\small]{\vinput}
    3 3 1
    .#.
    ##.
    ...
\end{myverbbox}
\begin{myverbbox}[\small]{\voutput}
    3
\end{myverbbox}
\inputoutputTable

\includeSolutionIfExistsByPath{final/subject_tour/inf2003_vr/task_02}
