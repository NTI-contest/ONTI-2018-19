\assignementTitle{Задача 2. Мортира}{100}{}

Вы участвуете в разработке игры в виртуальной реальности жанра
  <span lang="en">Tower Defense</span>.
  В играх подобного жанра, целью является защита основного
  замка от наступления вражеских юнитов.
</p>
<p>
  Одно из защитных орудий в игре~-- мортира.
  Это пушка которая может регулировать угол наклона в пределах от 45 до 90 градусов от земли.
  Сама себе пушка урон нанести не может.
</p>
<p>
  Как разработчику, вам поручили задание разработать искусственный интеллект для этого орудия.
  На начальных уровнях искусственный интеллект должен быть достаточно прост,
  вплоть до того, что он ничего не должен знать о параметрах орудия.
  Орудие имеет ограниченное количество боеприпасов~-- 20 зарядов.
  До того как боезапас иссякнет, пушка должна поразить хотя бы одну цель.
  Все, что может узнать ваш ИИ~-- это расстояние, на котором находится цель,
  и расстояние, на которое улетел снаряд после выстрела.
  Искусственный интеллект может управлять лишь углом наклона орудия.
</p>
<p>
  Ваши коллеги уже создали для вас простую тестовую сцену, и даже написали за вас API.
  Вам лишь необходимо реализовать интерфейс.
  Ссылка на репозиторий с проектом:
  <a href="https://github.com/Lukaviy/AI-for-defense.git">
      <code>https://github.com/Lukaviy/AI-for-defense.git</code></a>.
</p>
<p>
  Также, ваши коллеги еще не до конца уверены в выборе игрового движка.
  Возможно в скором будущем они перейдут на другой.
  Поэтому они убедительно просят вас воздержаться от использования функций,
  зависимых от <span cats-dict="1">Unity</span>.
  (Вместо <code>Mathf</code> использовать <code>System.Math</code>, и т.д.)
 </p>

 <p>Необходимо реализовать класс <code>CannonAI</code> со следующим интерфейсом:</p>
   <pre>
     <code>
        public class CannonAI : ICannonAI
        {
            // Расстояние на котором находится цель
            void SetTarget(double distance);
            // Угол наклона в градусах в который нужно установить пушку перед выстрелом
            double GetShootAngle();
            // Информация о дальности полета снаряда
            void FeedbackHitDistance(double distance);
        }
      </code>
   </pre>
</ProblemStatement>
<OutputFormat>
  <p>
     Файл с решением должен содержать только реализацию класса <code>CannonAI</code>.
     В качестве среды программирования необходимо выбирать <code>C#</code>.
  </p>
</OutputFormat>

<ProblemConstraints>
<p>
Гарантируется, что пушка всегда может достать до противника.
</p>
<p>
$45 \le Angle \le 90$
</p>
<p>
$10 \le Distance \le 10^6$
</p>
</ProblemConstraints>

<Explanation>
  <p>
    При изменении угла в диапазоне от 45 до 90 градусов расстояние от точки запуска
    до точки падения изменяется монотонно. Таким образом, данная задача
    решается стандартным алгоритмом бинарного поиска.
  </p>
  <p>
    Основной особенностью и целью задачи является знакомство участников с техникой
    реализации и отправки на проверку модулей Unity-проектов.

    
\inputfmtSection

В первой строке содержится количество запросов $n\space (1\leq n \leq 10^5 $

Затем в $n$ строках через пробел записываются два слова $command_i$ (code или decode) — команда (кодировать и декодировать соответственно) и $ bitcode_i\space (1\leq len(bitcode_i) \leq 100)$.

\outputfmtSection

Для каждого запроса в отдельной строке выведите последовательность битов с ведущими нулями — результат выполнения команды.

\explanationSection

Длина кодируемого сообщения всегда кратна 4, декодируемого — 7.

\markSection

Баллы за задачу будут начислены, если все тесты будут пройдены успешно.

\sampleTitle{1}

\begin{myverbbox}[\small]{\vinput}
    2
    decode 0101011
    code 0001
\end{myverbbox}
\begin{myverbbox}[\small]{\voutput}
    0001
    0001011
\end{myverbbox}
\inputoutputTable

\includeSolutionIfExistsByPath{final/subject_tour/inf2703_dzz/task_02}
