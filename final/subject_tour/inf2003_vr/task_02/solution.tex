\solutionSection

Прямоугольник, подходящий для того, чтобы быть меткой, должен обладать следующим свойством: количество точек внутри этого прямоугольника + количество решёток вне этого прямоугольника не должно превосходить $N$.

Так как $W, H \le 100$, можно перебрать все возможные прямоугольники и с помощью префиксных сумм (<code>.</code> = 1, <code>\#</code> = 0) \textit{(100 баллов)}  или двойного цикла \textit{(50 баллов)}, посчитать количество точек внутри прямоугольника. Это количество обозначим $P$. Пусть рассматриваемый прямоугольник имеет координаты $x1, y1$ ~--- левый верхний угол, $x2, y2$ ~--- правый нижний угол. Тогда общее количество клеток, входящих в рассматриваемый прямоугольник, \linebreak $S = (x2 - x1 + 1) \cdot (y2 - y1 + 1)$.

Общее количество решёток на изображении обозначим как $C$. Тогда количество решёток вне выбранного прямоугольника равно $K = C - (S - P)$. Следовательно, количество действий для исправления изображения равно $K + P$. Асимптотика решения с использованием префикс сумм $O(W^4)$, с использованием двойного цикла -- $O(W^6)$.

\codeExample

\inputCPPSource
%\inputminted[fontsize=\footnotesize, linenos]{python}{final/subject_tour/inf1303_bas/task_04/source.cpp}