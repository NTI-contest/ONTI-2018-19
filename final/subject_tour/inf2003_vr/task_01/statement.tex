\assignementTitle{Мишень}{100}{}

Вы реализуете игру "дартс" \: в виртуальной реальности. Игра устроена следующим образом. Дана мишень в форме круга радиуса $R$, расположенная в плоскости $XY$ с центром в координатах $(0, 0)$, разбитая на $n$ равных секторов. Сектора пронумерованы от $1$ до $n$ по часовой стрелке. В момент броска мишень находится в таком положении, что сегмент с номером 1 расположен в верхней полуплоскости справа от оси $Y$. Мишень закручивается вокруг оси $Z$ по часовой стрелке и приобретает постоянную угловую скорость $w$ градусов в секунду. 

Игрок, стоя лицом к мишени на расстоянии $z$, бросает дротик из координаты $(x, y)$ в плоскости $XY$ перпендикулярно мишени в её сторону. Дротик летит прямолинейно с постоянной скоростью $v$, не изменяя высоту полёта.

Требуется написать программу, которая определяет номер сектора, в который попадёт игрок.

\inputfmtSection

Входные данные содержат целые числа $R$, $w$, $x$, $y$, $z$, $v$, $n$. Гарантируется, что участник не попал в границу между секторами.

Ограничения:

$1 \le r, w, v, z \le 100$
$-100 \le x, y \le 100$
$2 \le n \le 100 $

\outputfmtSection

Выходные данные должны содержать единственное целое число -- номер сектора, в который попал участник, или 0, если участник не попал в мишень.

\sampleTitle{1}

\begin{myverbbox}[\small]{\vinput}
    10 20 10 0 100 100 100
\end{myverbbox}
\begin{myverbbox}[\small]{\voutput}
    20
\end{myverbbox}
\inputoutputTable

\includeSolutionIfExistsByPath{final/subject_tour/inf2003_vr/task_01}