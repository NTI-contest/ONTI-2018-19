\assignementTitle{Полоса препятствий}{100}{}

Одно из заданий на соревнованиях по подводной робототехнике — полоса препятствий. Полоса состоит из $n$ подряд идущих стен, высота $i$-й стены $h_i$. 

Цель робота — преодолеть все стены, достать флаг, который находится в конце полосы за всеми стенами, и вернуться обратно. Стену можно преодолеть двумя способами: подняться, проплыть над ней и опуститься обратно или сделать подкоп. Если стена была преодолена с помощью подкопа, то в при проходе в обратную сторону можно воспользоваться уже готовым подкопом. Всего разрешается сделать не более $m$ подкопов.

Робот имеет ограниченный заряд батареи. Изначально он способен подняться над стеной высотой $k$ или меньше, далее после каждого подъёма максимальная высота стены, которую может преодолеть робот, снижается на $1$. 

Робот поддерживает две команды: D -- идти через подкоп (если подкопа ещё нет, то робот выкапывает его), U -- проплыть над препятствием.

\inputfmtSection

Первая строка входного файла содержит целые числа $n m k$.
Следующая строка содержат $n$ целых чисел $h_i$ - высоты стен.

Ограничения:
$1 \le n, m \le 10^5$, $1 \le k, h_i \le 10^9$

\outputfmtSection

Выходной файл должен содержать строку NO, если задание выполнить невозможно. В противном случае -- две строки: строку YES и строку из $2n$ символов U и D -- последовательность команд робота, которая позволит ему преодолеть полосу препятствий и вернуться обратно. Первые $n$ символов содержат описание прохождение от стены с номером $1$ до стены с номером $n$, следующие $n$ символов содержат описание прохождения от стены с номером $n$ до стены с номером $1$.

\markSection

Баллы за каждую подзадачу начисляются только в случае, если все тесты этой
подзадачи и необходимых подзадач успешно пройдены.

\begin{tabular}{|c|c|c|c|c|}
Подзадача&Баллы&Дополнительные ограничения&Необходимые подзадачи&Информация о проверке \\\
1&40&$1 \le n, m \le 10^2$, $1 \le k, h_i \le 10^9$&-&полная\\\
2&60&$1 \le n, m \le 10^5$, $1 \le k, h_i \le 10^9$&1&полная\\\
\end{tabular}


\sampleTitle{1}

\begin{myverbbox}[\small]{\vinput}
    10 5 10
    10 1 4 2 4 3 4 4 8 9
\end{myverbbox}
\begin{myverbbox}[\small]{\voutput}
    YES
    DUDUDUUUDDDDUUUDUDUD
\end{myverbbox}
\inputoutputTable

\includeSolutionIfExistsByPath{final/subject_tour/inf2003_vr/task_03}