\solutionSection

\begin{enumerate}
\item Можно потратить все подкопы, если $n \le m$, иначе можно сделать $n$ подкопов и пройти маршрут только с помощью команд $D$. Далее всегда считаем, что $n \le m$.
\item Если стена преодолевается с помощью подкопа, то подкоп лучше сделать при первом проходе стены. Это позволит сэкономить один дополнительный прыжок.
\item Исходя из пункта 2, каждую стену, через которую следует преодолеть с помощью подъёма, мы переплываем дважды.
\item Теперь мы знаем, что всего следует сделать $2 \cdot n - 2 \cdot m$ подъёмов и высота последнего подъёма $H = k - (2 \cdot n - 2 \cdot m) + 1$.
\item Не важно, сможет ли робот переплыть препятствие при первом проходе, так как при обратном проходе максимальная высота подъёма робота будет меньше.
\item Используя пункты 1-5. Для решения задачи нам достаточно пробежать по всем стенам слева направо и действовать следующим образом: если высота стены меньше или равна $H$, то эту стену мы переплываем и увеличиваем максимальную возможную высоту подъёма $H$ на $1$. В противном случае эту стену следует преодолеть подкопом. Если уже робот уже смог преодолеть $n - m$ стен, то все оставшиеся стены проходим подкопом. Далее можно легко восстановить ответ, зная, какие стены как преодолеваются подкопом. Асимптотика решения $O(n)$.
\end{enumerate}

\codeExample

\inputCPPSource
%\inputminted[fontsize=\footnotesize, linenos]{python}{final/subject_tour/inf1303_bas/task_04/source.cpp}