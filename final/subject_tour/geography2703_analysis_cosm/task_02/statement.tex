\assignementTitle{}{10}{}

Опираясь на данные портала Всемирной лесной вахты (Global Forest Watch; \url{https://www.globalforestwatch.org/}), расположите реки из нижеприведённого списка по уменьшению доли (процента) потерь лесного покрова (tree cover loss) за период 2001-2017 гг. от общей площади всех сомкнутых лесов в пределах их водосборных бассейнов, согласно данным наблюдения из космоса.

Для целей данной задачи учитывайте в качестве сомкнутых лесов территории с древесной растительностью (как естественного, так и искусственного происхождения) с сомкнутостью древесного полога (tree cover) не менее 30\%.

Используйте показатель всех суммарных потерь древесного полога за указанный период в процентах от его общей площади по состоянию на 2000 год, включая его временные потери, компенсируемые восстановлением древесной растительности.

\begin{enumerate}
    \item Волга
    \item Миссисипи
    \item Лена
    \item Иравади
    \item Меконг
    \item Амур
    \item Амазонка
    \item Конго
\end{enumerate}

\explanationSection

Наборы пространственных данных о лесном покрове и его изменениях с 2000 года являются, во многом, ядром большого массива информации о лесах мира, собранных на портале «Всемирной лесной вахты» (\url{https://www.globalforestwatch.org/map}). Эти всемирные карты лесного покрова получены лабораторией GLAD (Global Land Analysis \& Discovery) Географического факультета Университета Мэриленда (США) путём обработки большого массива снимков со спутников серии Landsat. Эта база данных постоянно пополняется и в настоящий момент включает в себя данные по 2018 год.

Геопортал «Всемирной лесной вахты» обладает не только развитыми средствами визуализации различной картографической информации, но и достаточно развитыми средствами анализа. В частности, он позволяет подсчитывать статистику по лесному покрову и его изменениям по конкретным территориям, включая политические границы стран и регионов, экологические регионы и бассейны крупных рек, а также в пределах произвольного контура. Доступ к этим инструментам анализа осуществляется из закладки «ANALYSIS», находящейся рядом с закладкой с условными обозначениями (LEGEND). Хотя геопортал не имеет русскоязычного интерфейса, знания английского языка в рамках программы средней школы вполне достаточно, чтобы выбрать из списка территорий, доступных для анализа «в один клик», речные бассейны («river basins»).

После этого на карте появляются границы крупнейших речных бассейнов, при клике мышью на которых производится подсчет статистики по выбранному речному бассейну по наборам данных о лесном покрове и его изменениях. При этом используются те наборы данных и с теми параметрами, которые присутствуют в легенде в момент проведения анализа. Также высвечивается название реки, по бассейну которой ведётся подсчёт статистики. 30\% сомкнутости древесного полога является параметром по умолчанию. После того, как на портал добавили данные за 2018 год, для получения статистики за период 2001-2017 необходимо воспользоваться соответствующим «ползунком» в панели с условными обозначениями.

Для поиска на карте бассейнов рек из условия задачи достаточно знаний в объёме школьного курса географии (были выбраны только крупнейшие реки). Доля (процент) потерь лесного покрова для каждого речного бассейна из списка рассчитывается из цифр, выданных при анализе статистики (общей площади лесов и общей площади потерь лесного покрова).

\answerMath{5, 3, 2, 6, 7, 4, 1, 8.}