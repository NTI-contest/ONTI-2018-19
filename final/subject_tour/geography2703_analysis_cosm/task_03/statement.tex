\assignementTitle{}{10}{}

Определите страну по описанию.

Как и в большинстве бывших испанских колоний, абсолютное большинство граждан этой страны - католики по вероисповеданию. Однако, в отличии от большинства стран Латинской Америки, испанский язык не является здесь официальным, и владеет им лишь небольшой процент населения.

Входит в число самых населённых стран мира, с населенеим, превышающим 100 миллионов человек.

Согласно данным Всемирного банка (международной финансовой организации, акционерами которой являются правительства 189 стран мира, – \url{https://data.worldbank.org/}) за 2017 год, по своему валовому внутреннему продукту (ВВП) это государство опережает такие страны как Финляндия, Португалия, Египет, Новая Зеландия и Кувейт и продолжает достаточно быстрый экономический рост. Её экономика - одна из самых быстрорастущих в регионе.

Будучи молодой индустриальной экономикой, страна по-прежнему имеет достаточно большой аграрный сектор, дающий существенный вклад в ВВП страны и её экспортный потенциал. Страна является одним из крупнейших производителей сахара в мире. А по производству другого продукта плантационного хозяйства (но, в отличие от сахарного тростника, получаемого с древесных плантаций) - вообще является одним из двух мировых лидеров.

В то же время, вклад промышленности в экономику страны уже гораздо выше, чем сельского хозяйства. Среди важнейших отраслей промышленности - судостроение, по объёму которого страна, по некоторым данным, занимала в отдельные годы четвёртое место, сразу за тройкой мировых лидеров (Китай, Япония и Южная Корея). Более половины ВВП - вклад сферы услуг.

Согласно данным портала Всемирной лесной вахты (Global Forest Watch; \url{https://www.globalforestwatch.org/}), в 2017 году половина территории страны имела древесный покров сомкнутостью 30\% и более (включая не только леса, но и древесные плантации). По данным космической съёмки, суммарные потери лесного покрова страны с 2001 по 2017 год составили около 6\%. Примерно на 40\% этой площади потеря древесного полога оказалась безвозвратой и может рассматриваться как постоянное обезлесивание. Однако, и на тех территориях, где древесный полог восстановился, это могут быть не естественные леса, а заменившие их плантации древесных культур. Принимая во внимание значение плантационного хозяйства для экономики страны, это представляется распространённой ситуацией.

В сохранившихся лесах региона обитают прадставители одного из родов примитивных приматов, ранее относившихся к устаревшему отряду полуобезъян. Однако, по современной классификации, даный род относится к подотряду сухоносых приматов - к тому же, к которому относятся обезьяны и человек. Большинство других полуобезьян причисляются сегодня к подотряду мокроносых приматов.

\answerMath{Филиппины.}