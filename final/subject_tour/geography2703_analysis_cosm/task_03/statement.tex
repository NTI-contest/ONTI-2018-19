\assignementTitle{}{10}{}

Определите страну по описанию.

Как и в большинстве бывших испанских колоний, абсолютное большинство граждан этой страны - католики по вероисповеданию. Однако, в отличии от большинства стран Латинской Америки, испанский язык не является здесь официальным, и владеет им лишь небольшой процент населения.

Входит в число самых населённых стран мира, с населенеим, превышающим 100 миллионов человек.

Согласно данным Всемирного банка (международной финансовой организации, акционерами которой являются правительства 189 стран мира, – \url{https://data.worldbank.org/}) за 2017 год, по своему валовому внутреннему продукту (ВВП) это государство опережает такие страны как Финляндия, Португалия, Египет, Новая Зеландия и Кувейт и продолжает достаточно быстрый экономический рост. Её экономика - одна из самых быстрорастущих в регионе.

Будучи молодой индустриальной экономикой, страна по-прежнему имеет достаточно большой аграрный сектор, дающий существенный вклад в ВВП страны и её экспортный потенциал. Страна является одним из крупнейших производителей сахара в мире. А по производству другого продукта плантационного хозяйства (но, в отличие от сахарного тростника, получаемого с древесных плантаций) - вообще является одним из двух мировых лидеров.

В то же время, вклад промышленности в экономику страны уже гораздо выше, чем сельского хозяйства. Среди важнейших отраслей промышленности - судостроение, по объёму которого страна, по некоторым данным, занимала в отдельные годы четвёртое место, сразу за тройкой мировых лидеров (Китай, Япония и Южная Корея). Более половины ВВП - вклад сферы услуг.

Согласно данным портала Всемирной лесной вахты (Global Forest Watch; \url{https://www.globalforestwatch.org/}), в 2017 году половина территории страны имела древесный покров сомкнутостью 30\% и более (включая не только леса, но и древесные плантации). По данным космической съёмки, суммарные потери лесного покрова страны с 2001 по 2017 год составили около 6\%. Примерно на 40\% этой площади потеря древесного полога оказалась безвозвратой и может рассматриваться как постоянное обезлесивание. Однако, и на тех территориях, где древесный полог восстановился, это могут быть не естественные леса, а заменившие их плантации древесных культур. Принимая во внимание значение плантационного хозяйства для экономики страны, это представляется распространённой ситуацией.

В сохранившихся лесах региона обитают прадставители одного из родов примитивных приматов, ранее относившихся к устаревшему отряду полуобезъян. Однако, по современной классификации, даный род относится к подотряду сухоносых приматов - к тому же, к которому относятся обезьяны и человек. Большинство других полуобезьян причисляются сегодня к подотряду мокроносых приматов.

\explanationSection

Одного первого абзаца в условиях задачи практически достаточно, чтобы вычислить страну. Практически во всех бывших испанских колониях испанский язык имеет статус официального. (Кроме Филиппин, он не является таковым ещё только в Марокко, бывшей испанской колонией лишь частично). Это сразу исключает из рассмотрения все страны Латинской Америки, Экваториальную Гвинею и Западную Сахару. Про испанскую колониальную империю учащимся должно быть хорошо известно из школьного курса истории.

Можно также заподозрить, что речь идёт о Соединённых Штатах Америки, которые тоже частично были испанскими владениями. Хотя там испанский является официальным в ряде штатов, а испаноговорящее население многочисленно, можно счесть, что он не является официальным во всей стране (как, впрочем, и английский – де юре).

Следующий признак, население более ста миллионов человек, сразу исключает из рассмотрения все мелкие страны. Даже если участник не помнит всего списка самых крупных по населению стран, это резко суживает круг поиска. Из бывших испанских колоний под это условие подходит ещё только Мексика (испаноязычность которой хорошо известна). Ещё такую численность населения имеет Бразилия, но это – бывшая португальская колония. Под это условие также подходят США.

Следующий признак – молодая индустриальная экономика с большим аграрным сектором – однозначно исключает США из числа кандидатов (даже если считать эту страну частично бывшей испанской колонией).

Сочетание лидирующих роли страны в судостроении, производстве сахара и продуктов древесных плантаций (даже если не догадаться, что речь идёт о кокосовых орехах) – достаточно специфическое и не характерно для большого числа стран.

Указанные цифры потери лесного покрова также позволяют примерно вычислить, о какой стране идёт речь, с помощью портала «Всемирной лесной вахты», к которому участники имели доступ во время индивидуального тура.

Наконец, обитание на территории страны примитивных приматов исключает из рассмотрения и США. Более того, приматов, ранее относившихся к отряду полуобезьян, сравнительно немного и почти все их группы широко известны. Все лемурообразные живут только на Мадагаскаре. Лориобразные встречаются в ряде тропических стран Африки и Азии. Другая группа, широко известная хотя бы по научно-популярным фильмам о животных, – долгопяты. Именно эта небольшая группа (семейство) примитивных приматов относится по сегодняшней классификации к другому подотряду (сухоносых приматов), чем все перечисленные выше. Они живут только на островах Юго-Восточной Азии, в том числе, на Филиппинах.

\answerMath{Филиппины.}