На решение каждой из задач по географии дано ограниченное количество попыток (от 1 до 2).

В процессе решения разрешено пользоваться следующими сайтами и никакими другими (нарушение этого правила повлечет дисквалификацию). Рекомендации по применению конкретных из них указаны в текстах задач.
\begin{itemize}
    \item \url{www.globalforestwatch.org/}
    \item \url{data.worldbank.org/}
    \item \url{earthenginepartners.appspot.com/science-2013-global-forest}
\end{itemize}

Задания разнообразные, поэтому рекомендуем сначала просмотреть все и потом выбирать.
Не останавливайтесь подолгу на одной.

\subimport{final/subject_tour/geography2703_analysis_cosm/task_01/}{statement.tex}
\subimport{final/subject_tour/geography2703_analysis_cosm/task_02/}{statement.tex}
\subimport{final/subject_tour/geography2703_analysis_cosm/task_03/}{statement.tex}
\subimport{final/subject_tour/geography2703_analysis_cosm/task_04/}{statement.tex}
\subimport{final/subject_tour/geography2703_analysis_cosm/task_05/}{statement.tex}

Приведённые далее изображения озёр получены со спутников Landsat 7 и Landsat 8. Они не являются единовременным снимком, а получены с помощью синтеза ряда безоблачных снимков за 2017 год, обработанных таким образом, чтобы компенсировать всегда имеющиеся различия в положении солнца, положении спутника, прозрачности атмосферы и пр. Обработка проведена лабораторией GLAD (Global Land Analysis \& Discovery) Географического факультета Университета Мэриленда (США). Данное изображение сделано с помощью портала Глобальные изменения лесного покрова (\url{http://earthenginepartners.appspot.com/science-2013-global-forest}).

Изображение представляет интенсивность электромагнитного излучения, отражённого от поверхности земли и принятого аппаратурой спутников, в условных цветах: не видимый человеческим глазом коротковолновый инфракрасный свет представлен красным, ближний инфракрасный (также не различимый глазом) - зелёным, видимый красный свет - синим. В таком цветовом синтезе хорошо видны различия в растительности и влажности поверхности.

При этом чистые воды водоёмов и водотоков выглядят тёмными, почти чёрными, так как равномерно поглощают большую часть солнечных лучей во всех указанных участках спектра. Если в воде имеется взвесь минеральных частиц или высокая плотность планктонных организмов, такие акватории выглядят как оттенки синего и голубого тонов. Такой же цвет могут иметь и мелководные участки, где сквозь толщу воды проходит солнечный свет, отражённый от дна водоёма и/или от водной растительности.

Участки суши, покрытые сомкнутой растительностью, как правило, выглядят при данном цветовом синтезе как различные оттенки зелёного. При этом, более тёмные и насыщенные тона, как правило, соответствуют древесной растительности, а более светлые - травянистой и кустарниковой.

Открытый грунт, в зависимости о его состава и характера, отображается различными оттенками красного и фиолетового. Частично покрытые растительностью участки (имеющие несомкнутый растительный покров) могут иметь различные промежуточные оттенки (розоватые, бурые и пр.). Красноватые тона может иметь и болотная растительность. Голые скалы и песок, как правило, имеют различные оттенки голубого и синего. Солончаки, а также лёд и снег - практически белые.

Рассмотрите внимательно данные спутниковые изображения. Большинство озёр, которые вы видите, — крупнейшие в своих регионах и широко известные даже за их пределами. Возможно, Вы узнаете их по известным очертаниям из географических атласов. Это облегчит Вашу задачу.

По хорошо различимому на снимках рельефу местности, форме озера и другим признакам предположите происхождение озёрной котловины. По связанным водотокам сделайте выводы о проточном или бессточном характере озера. По окружающей растительности и другим признакам сделайте выводы о климатической и природной зонах, в которых находится каждое озеро. На основании анализа всех этих данных постройте предположения о солёности озёрных вод.

\subimport{final/subject_tour/geography2703_analysis_cosm/task_06/}{statement.tex}