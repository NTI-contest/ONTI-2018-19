\assignementTitle{}{14}{}

Поставьте в соответствие каждому виду животных из списка ниже страну, в которой этот вид обитает в естественных условиях. (Примечание: ареалы некоторых из приведённых ниже видов не ограничивается территорией только одной страны; но в таких случаях прочие страны, где встречаются такие виды, не входят в данный список.)

Животные:
\begin{enumerate}
    \item Большая панда (Ailuropoda melanoleuca)
    \item Виргинский опоссум (Didelphis virginiana)
    \item Какапо / Совиный попугай (Strigops habroptila)
    \item Карликовый бегемот (Choeropsis liberiensis)
    \item Карликовый шимпанзе / Бонобо (Pan paniscus)
    \item Кенгуру Дориа (Dendrolagus dorianus)
    \item Комодский варан (Varanus komodoensis)
    \item Морская игуана (Amblyrhynchus cristatus) 
    \item Обыкновенный слепыш (Spalax microphthalmus)
    \item Обыкновенный тенрек (Tenrec ecaudatus) 
    \item Пингвин Гумбольдта (Spheniscus humboldti)
    \item Пустынный хамелеон (Chamaeleo namaquensis)
    \item Шерстокрыл / Кагуан (Cynocephalus volans)
    \item Эму (Dromaius novaehollandiae)
\end{enumerate}

Страны:
\begin{enumerate}
    \item[А.] Австралия
    \item[Б.] Демократическая республика Конго
    \item[В.] Индонезия
    \item[Г.] Китай
    \item[Д.] Кот-д’Ивуар
    \item[Е.] Мадагаскар
    \item[Ж.] Мексика
    \item[З.] Намибия
    \item[И.] Новая Зеландия
    \item[К.] Папуа - Новая Гвинея
    \item[Л.] Россия 
    \item[М.] Филиппины
    \item[Н.] Чили
    \item[О.] Эквадор
\end{enumerate}

Введите последовательность из 14 заглавных русских букв (страны) без разделителей, соответствующих животным, указанным в пунктах 1-14. 

\explanationSection

Задача на общую эрудицию и логическое мышление. Все виды животных были подобраны таким образом, чтобы они встречались в природе только в одной стране из приведённого списка. Об этом прямо сказано в условии задачи.

Баллы начислялись пропорционально количеству правильно указанных пар «вид животных – страна».

Часть животных из списка достаточно широко известна, как и их распространение. Так, большая панда – общеизвестный символ Китая. А эму – крупнейшая птица Австралии и вторая по величине из ныне живущих, после африканского страуса. Многочисленные публикации и научно-популярные фильмы сделали достаточно широко известным бонобо (карликового шимпанзе) – один из четырёх видов человекообразных обезьян. В отличие от своих близких родичей, шимпанзе, они обитают только в бассейне Конго на территории Демократической республики Конго. Самый крупный вид ящерицы, комодский варан, также широко известен. Другое его популярное название – дракон с острова Комодо. Достаточно только помнить, что остров Комодо – часть индонезийского архимелага. Не менее известна и другая ящерица из списка – морская игуана с Галапагосских островов. Вспомнить, что Галапагоссы принадлежат Эквадору несколько сложнее, но тоже можно. Нелетающий новозеландский попугай какапо – также частый персонаж научно-популярных фильмов о природе.

Шерстокрыл несколько менее известен. Однако, тем, кто интересуется природой, он известен как одно из немногих млекопитающих, обладающих летательной перепонкой.

Само название виргинского опоссума намекает, что он обитает на территории Северной Америки. Опоссумы также являются популярными персонажами американского фольклора и героями ряда популярных мультфильмов (например, «Ледникового периода»). Поскольку США в списке отсутствует, даже не зная точно, можно предположить обитание опоссума в соседней Мексике.

Бегемоты общеизвестны как африканские животные. Они обитают сегодня только в Африке к югу от Сахары (кроме Мадагаскара). Остаётся выбрать из списка оставшихся африканских стран. Поскольку Демократическая республика Конго уже «занята» бонобо, остаются Намибия и Кот-д’Ивуар. Пустынная Намибия – явно не самое подходящее время для бегемотов, ведущих полуводный образ жизни.

Распределение по странам оставшихся животных несколько более сложно и содержит «ловушки» – намёки на простые решения, не являющиеся правильными. Так, хамелеоны широко представлены на острове Мадагаскар, многие их виды являются эндемиками этого острова (то есть обитают только там). Однако, распространены они шире, встречаясь преимущественно в Африке южнее Сахары, но также и в некоторых других регионах. Конкретно упомянутый в списке пустынный хамелеон – обитатель не Мадагаскара, а Намибии. Отчасти догадаться об этом поможет его название: в списке нет других африканских стран с пустынным климатом.

На Мадагаскаре же обитает другой зверь из списка – обыкновенный тенрек. Здесь уже привязка к Мадагаскару однозначная. Похожие внешне на ежей тенреки живут, в большинстве своём, именно на Мадагаскаре. Некоторые виды встречаются в Восточной и Центральной Африке и на соседних островах Индийского океана. Но типичный представитель семейства, обыкновенный тенрек, – эндемик Мадагаскара.

Другое животное с «неочевидной» страной обитания – кенгуру Дориа. То, что некоторые виды кенгуру обитают за пределами Автралии – малоизвестный факт. Попытка «поселить» данный вид кенгуру в эту страну оставила бы «бездомным» эму. Если участник достаточно эрудирован, чтобы «поселить» в Австралию, всё-таки, именно эму, то для кенгуру ему придётся искать одну из соседних стран – Новую Зеландию или Папуа – Новую Гвинею. Если при этом знать про родину какапо, то наиболее очевидный выбор для кенгуру Дориа – именно Папуа – Новая Гвинея.

Аналогичная ситуация с пингвином Гумбольта. Общеизвестно, что большинство пингвинов обитают в Антарктиде и на прилегающих островах. Однако, в реальности они распространены более широко, встречаясь на побережьях целого ряда стран Южного полушария, вплоть до Перу и тропических Галапагосских островов. Антарктиды в списке стран, по понятным причинам, нету. Поэтому выбирать приходится из других стран южного полушария, имеющих морское побережье. Из представленных в списке стран пингвины могут встречаться у побережья Австралии, Новой Зеландии, Намибии и Чили. Поскольку другие варианты «заняты» более очевидными местными обитателями, то Чили – единственный оставшийся вариант. Кроме того, на место может указывать название данного вида пингвинов. Всё-таки, распространение этих птиц связано, преимущественно с холодными водами. В низкие широты они захотят, как правило, в тех местах, где вдоль побережья проходят холодные океанские течения – Бенгальское, вдоль западного побережья Африки, и течение Гумбольта, текущее как раз на север вдоль западного побережья Южной Америки, и снабжающее прохладной водой одноименных пингвинов. Схема океанских течений разбирается в школьном курсе географии. Конкретные название крупнейших течений должны быть известны эрудированным ученикам.

Подобрать достаточно известный вид эндемичных животных, обитающих ТОЛЬКО в России (и при этом не мигрирующий в другие страны), оказалось не так уж просто. Даже известный пример такого рода, русская выхухоль, обитает и в соседних странах (в Белоруссии, Украине, Литве и Казахстане). Ареал распространения обыкновенного слепыша также заходит на территорию Украины, но этот вид роющих грызунов не встречается южнее Кавказского хребта и уж точно не обитает ни в одной другой стране из списка. Если участник никогда не слышал об этом животном, что вычислить соответствующую ему страну можно методом исключения. Тем не менее, это животное не является малоизвестным, особенно в степных регионах России, где он может вредить огородам и приусадебным участкам, поедая корнеплоды и луковичные (зверьки эти исключительно растительноядные).

Правильные сочетания пар даны ниже. (Зелёным выделены животные, достаточно широко известные, жёлтым – те, о стране обитания которых достаточно легко догадаться, голубым – более сложные случаи, которые требуют более глубоких знаний и логики, синим – пары, вычисляемые по остаточному принципу или известные эрудированным участникам.)

\begin{longtable}{|c|l|p{4cm}|}
    \hline
    \rowcolor{green} 1. & Большая панда (Ailuropoda melanoleuca) & Китай \\
    \hline
    \rowcolor{yellow} 2. & Виргинский опоссум (Didelphis virginiana) & Мексика \\
    \hline
    \rowcolor{green} 3. & Какапо / Совиный попугай (Strigops habroptila) & Новая Зеландия \\
    \hline
    \rowcolor{yellow} 4. & Карликовый бегемот (Choeropsis liberiensis) & Кот-д’Ивуар \\
    \hline
    \rowcolor{green} 5. & Карликовый шимпанзе / Бонобо (Pan paniscus) & Демократическая республика Конго \\
    \hline
    \rowcolor{cyan} 6. & Кенгуру Дориа (Dendrolagus dorianus) & Папуа - Новая Гвинея \\
    \hline
    \rowcolor{green} 7. & Комодский варан (Varanus komodoensis) & Индонезия \\
    \hline
    \rowcolor{green} 8. & Морская игуана (Amblyrhynchus cristatus) & Эквадор \\
    \hline
    \rowcolor{blue} 9. & Обыкновенный слепыш (Spalax microphthalmus) & Россия \\
    \hline
    \rowcolor{cyan} 10. & Обыкновенный тенрек (Tenrec ecaudatus) & Мадагаскар \\
    \hline
    \rowcolor{blue} 11. & Пингвин Гумбольдта (Spheniscus humboldti) & Чили \\
    \hline
    \rowcolor{cyan} 12. & Пустынный хамелеон (Chamaeleo namaquensis) & Намибия \\
    \hline
    \rowcolor{green} 13. & Шерстокрыл / Кагуан (Cynocephalus volans) & Филиппины \\
    \hline
    \rowcolor{green} 14. & Эму (Dromaius novaehollandiae) & Австралия \\
    \hline
\end{longtable}

\answerMath{ГЖ3ИДБКВОЛЕНЗМА.}