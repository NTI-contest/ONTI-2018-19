\assignementTitle{}{48}{}

На основании проведённого анализа для \textbf{каждого} из изображений (по 2 в каждом из следующих шагов):
\begin{itemize}
    \item Укажите название озера, его террирориальное расположение (1 балл)
    \item Выберите из приведённого ниже списка тот набор признаков, который, по Вашему мнению, в наилучшей степени характеризует данное озеро. Поставьте в соответствие каждому изображению озера его краткую характеристику: происхождение его котловины, солёность, проточность. (1 балл)
    \item Обоснуйте ответ. (2 балла)
\end{itemize}

Список признаков, которые соответствуют озёрам:
\begin{enumerate}
    \item Завально-запрудное, пресное, проточное
    \item Изначально - приморское, современная котловина - ледниково-тектоническая, пресное, проточное
    \item Кратерное (вулканическое), пресное, проточное
    \item Кратерное (метеоритное), пресное, проточное
    \item Ледниково-тектоническое, пресное, проточное
    \item Пойменное, пресное, проточное
    \item Приморская лагуна/лиман, от пресного до солёного в разных частях, проточное
    \item Тектоническое (грабен), солёное, бессточное
    \item Тектоническое (межгорная котловина и разлом), солоноватое, бессточное
    \item Тектоническое (тектонический прогиб), пресное, проточное
    \item Тектоническое (тектонический прогиб), солоноватое, бессточное
    \item Тектоническое (разлом), пресное, проточное
\end{enumerate}

Обратите внимание, что одно изображение соответствует одному описанию. Рекомендуем заранее просмотреть все изображения.

Озеро 1:
\putImgWOCaption{8cm}{1}
Озеро 2:
\putImgWOCaption{8cm}{2}
Озеро 3:
\putImgWOCaption{8cm}{3}
Озеро 4:
\putImgWOCaption{8cm}{4}
\newpage
Озеро 5:
\putImgWOCaption{8cm}{5}
Озеро 6:
\putImgWOCaption{8cm}{6}
Озеро 7:
\putImgWOCaption{8cm}{7}
\newpage
Озеро 8:
\putImgWOCaption{8cm}{8}
Озеро 9:
\putImgWOCaption{8cm}{9}
Озеро 10:
\putImgWOCaption{8cm}{10}
\newpage
Озеро 11:
\putImgWOCaption{8cm}{11}
Озеро 12:
\putImgWOCaption{8cm}{12}

\explanationSection

Эта задача, прежде всего, – на умение участников анализировать то, что они видят на космических снимках и сопоставлять с известными им фактами и знаниями из школьного курса физической географии.

Большинство озёр на приведённых в задаче космических снимках – крупные (часто крупнейшие в своих регионах) и широко известные. Они должны легко опознаваться участниками, знакомыми с географическими картами и работавшие в школе с контурными картами или их аналогами. Опознание название и расположения известного озера, как правило, делает несложным выбор для него основных параметров, поскольку они известны из школьного курса физической географии.

Однако, и без знания названия озера на приведённых космических снимках достаточно информации, чтобы, анализируя её, сделать правильный выбор. Для анализа также необходимы общие знания об озёрах, их типах и формировании из школьного курса физической географии. При оценке задачи значительное внимание уделялось именно умению участников анализировать указанную информацию и делать выводы. То есть участники должны были не просто выбрать подходящий для данного озера набор признаков, но и аргументировать свой выбор. За это начислялось до половины возможных баллов.

Основными признаками, по которым можно сделать вывод о характере озера и его происхождении, следующие.
\begin{itemize}
    \item Связь озера с речной сетью (водотоками). Наличие рек, как втекающих, так и вытекающих из озера свидетельствует о его проточном характере. При этом проточные озёра крайне редко бывают солёными – обычно они пресные или солоноватоводные.
    \item Связь с морским побережьем. Как правило, озёра, отделённые от моря небольшой полосой дюн, по своему происхождению являются бывшими морскими лагунами.
    \item Растительность вокруг и по берегам озера. Значительные площади без растительности или с редкой травянистой растительностью (если речь не идёт о высокогорьях), скорее всего, свидетельствуют о засушливости климата в регионе, где расположено озеро. Именно в засушливых(аридных) регионах чаще всего можно встретить солёные и солоноватоводные озёра. Напротив, наличие по берегам большого количества лесов и открытых болот, является свидетельством более влажного климата и является дополнительным аргументом в пользу пресноводности озера.
    \item По берегам сильно солёных озёр обычно имеются отложения соли. На космических снимках они обычно прекрасно видны и сильно отличаются от любого растительного покрова. На это прямо указывается в условии задачи
    \item Также контрастно выглядят на космических снимках ледники и снежники в горных регионах. Их наличие вблизи озера, скорее всего, свидетельствует о его ледниковом питании (а значит, о его пресноводности и, скорее всего, проточном характере).
    \item Рельеф района, где расположено озера, наличие горных систем. По этим признакам часто можно судить о происхождении озерной котловины. Озёра расположенные на плоской равнине, скорее всего, располагаются в тектоническом прогибе или имеют ледниковое или эоловое происхождение. Однако, тектоническое и ледниковое происхождение озёр возможно и в горах. Также на снимках видны случаи расположения озёр в межгорном прогибе. Расположение озера в узкой горной долине может свидетельствовать о его завальном происхождении.
    \item Форма озера также может многое сказать о происхождении его котловины. Озёра, расположенные в тектонической или ледниковой котловине, как правило, имеют более плавные береговые линии. Для кратерных (а также карстовых) озёр характерна круглая форма. Озёра, заполнившие тектонические разломы, как правило, имеют узкую вытянутую форму.
\end{itemize}

Несколько приведённых в задаче озёр практически с первого взгляда не оставляют сомнения о своём происхождении. Так, озеро номер 3 (Маникуаган, сегодня оно является водохранилищем) заполняет котловину, образованную ударным (метеоритным кратером). Даже если не знать об этом озере (а оно достаточно широко известно), об этом явно свидетельствует его кольцеобразная форма.

Также моно достаточно уверенно сказать о происхождении озера номер 5 (Сарезское). Оно является классическим примером запрудного (завального) по происхождению озера. Его обычно первым приводят во всех учебниках в числе примеров такого типа озёр. Но и по космическому снимку видно, что оно расположено в высокогорном районе (горные хребты вокруг, практически полное отсутствие растительности, ледники на вершинах соседних хребтов). Его вытянутая форма свидетельствует о расположении в горной долине. При этом хорошо видно, что ширина озера с одной стороны постепенно уменьшается, а самая широкая часть находится на одном из краёв сильно вытянутого вдоль ущелья озера. Видно русло вытекающего из этой части озера узкого водотока. Такая форма не может объясняться ничем иным, кроме наличия препятствия (запруды) в данном месте.

Озеро номер 10 (Малави (Ньяса)) не только очень известное, но и имеет характерную вытянутую форму, так как располагается в тектоническом разломе (рифте). Великая Восточно-Африканская рифтовая долина специально разбирается в школьном курсе физической географии. При этом обращается внимание на форму расположенных в ней озёр. (Другой случай подобного рода, известный из школьных учебников, – озеро Байкал.) На приведённом космическом снимке также прекрасно видны реки, втекающие в озеро и вытекающие из него, что не оставляет сомнения в его проточном и, скорее всего, пресноводном характере.

Озеро 11 (Лагуна Энзели (Анзали)) также не оставляет участникам широкого выбора. На космическим снимке хорошо видно его расположение вблизи побережья крупного водоёма (это Каспийское море), от которого оно отделено узкой полосой суши. Очевидно, что это озеро по своему происхождению является лагуной / лиманом. Это также заставляет предположить, что вода в нём, как и в расположенном рядом море (большом солёном озере), солёная. Однако, на снимке хорошо видна река, впадающее в море вблизи озера и соединяющаяся с ним протоками. Это позволяет допустить опреснение части водоёма.

Ещё одной озеро, по форме которого можно сделать однозначный выбор набора признаков, – озеро 11 (Лагу Сараэа). На снимке прекрасно видно, что озеро расположено в пойме крупной реки (это Амазонка), в месте впадения в неё ругой реки. Вокруг видны многочисленные старицы – озёра и протоки характерной формы, являющиеся частями старого русла реки. Изрезанные берега озера, многочисленные заливы и острова говорят о том, что оно расположено на почти совершенно плоском месте, то есть непосредственно в пойме видимой на снимке крупной реки. Также нет никаких сомнений, что данное озеро – проточное и пресноводное.

Озеро номер 7 – единственное в задаче, где хорошо видны солончаки (бело-голубые) по его берегам и вдоль бывших водотоков, а также красноваты, практически лишённые растительности земли вокруг и к западу от озера. Всё это не оставляет сомнений в то, что данное озеро расположено в засушливом климате и сильно солёное. Полностью солёное озеро в списке – единственное.

Ещё паре озёр подобрать набор признаков несколько сложнее. Однако, на снимках также имеются характерные признаки, позволяющие сделать практически однозначный выбор. Так, расположенное на острове Суматра озеро Тоба (номер  9), хоть и н имеет такой правильной круглой формы, как Маникуаган, также имеет форму кольца с центральным островом. Даже просто по аналогии можно сделать вывод о кратерном происхождении данного озера. Неправильная форма объясняется тем, что данный кратер – не ударный, а вулканический. Если участник также знает название и расположение озера (нашёл на карте, к которой во время предметного тура был доступ), то его нахождение в вулканически активном регионе добавит аргументов в пользу такого решения. Вытекающая из озеро река видна не очень отчётливо (хотя видна). Однако, расположенная по берегам буйная растительность, в том числе, – тёмно-зелёные сомкнутые леса, не оставляет сомнений, что данный регион не страдает от засухи. Нахождение здесь солёных озёр – маловероятно.

Ещё одно озеро, номер 8 (Иссык-Куль), очевидно, расположен в межгорной котловине между двумя хребтами. Это – единственный такой случай в данной задаче. Понять наличие там ещё и разлома по снимкам невозможно. Однако, межгорная котловина и хребты видны прекрасно. Также по снимку можно предположить бессточный характер озера, из которого не вытекает ни она река. В западной части озера рядом расположена река Чу, но она также не вытекает из озера. Это позволяет предположить, что озеро может быть солёным или солоноватоводным. В то же время, оно, очевидно, питается от расположенных на соседних горных хребтах ледников, то есть вряд ли может быть сильно солёным.

Из оставшихся трёх озёр (1, 2, 4 и 6) явно выделяется номер 6 (Чаны). По снимку хорошо видно, что оно бессточное расположено в довольно засушливом районе и, в отличии от трёх других оставшихся, не имеет стока. На снимке также виден ярд соседних, мелководных и пересыхающих озёр. Около некоторых видны отложения соли (хотя не такие большие и явные, как вокруг Большого Солёного озера). Можно сразу предположить, что озеро солёное или солоноватое. Все остальное оставшиеся озёра – пресные и проточные. Видно, что озеро неглубокое, довольно неправильной формы, что не противоречит предположению о том, что его котловина находится в тектоническом прогибе. (Участники должны были хорошо представлять себе, что это значит и не путать тектонически прогиб с тектоническим разломом.)

Озеро 1 (африканская Виктория) и озеро 2 (канадское Большое Невольничье) – крупнейшие и очень известные озёра характерной формы, легко узнаваемые на географических картах. (Виктория – крупнейшее по площади озеро на континенте и второе в мире, Большое Невольничье – пятое на континенте и десятое в мире.) Оба являются проточными, что хорошо видно на снимках, и, соответственно пресноводными. Форма озёр не позволяет неспециалисту однозначно сделать вывод о различиях в характере происхождения их котловин. Вывод об участии ледников в формировании котловины Большого Невольничьего озера можно сделать, зная их расположение. (Оба крупнейших озера можно было легко найти на доступном участникам геопортале Всемирной лесной вахты). Также вокруг Большого Невольничьего озера можно увидеть множество небольших озёр и болот характерной вытянутой формы, которые указывают на имеющиеся здесь ледниковые формы рельефа. Однако, в целом, для идентификации этих озёр стоило было полагаться на общую эрудицию участников и знание ими географических карт.

Наконец, озеро 4 (Титикака) – также крупнейшее озеро своего континента (по запасу воды, да и по площади тоже, если считать Марокайбо морским заливом). О его пресноводности и проточности можно судить по втекающим и вытекающим из него рекам. Однако, не зная заранее, сложно предположить, что оно когда-то было морским заливом. Поэтому набор признаков для этого озера приходится подбирать по остаточному принципу, либо полагаться на общую эрудицию участников, которые смогут опознать данное озеро по его форме или найти на геопортале.

Правильные соответствия даны ниже.