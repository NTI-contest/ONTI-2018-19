\assignementTitle{}{48}{}

На основании проведённого анализа для \textbf{каждого} из изображений (по 2 в каждом из следующих шагов):
\begin{itemize}
    \item Укажите название озера, его террирориальное расположение (1 балл)
    \item Выберите из приведённого ниже списка тот набор признаков, который, по Вашему мнению, в наилучшей степени характеризует данное озеро. Поставьте в соответствие каждому изображению озера его краткую характеристику: происхождение его котловины, солёность, проточность. (1 балл)
    \item Обоснуйте ответ. (2 балла)
\end{itemize}

Список признаков, которые соответствуют озёрам:
\begin{enumerate}
    \item Завально-запрудное, пресное, проточное
    \item Изначально - приморское, современная котловина - ледниково-тектоническая, пресное, проточное
    \item Кратерное (вулканическое), пресное, проточное
    \item Кратерное (метеоритное), пресное, проточное
    \item Ледниково-тектоническое, пресное, проточное
    \item Пойменное, пресное, проточное
    \item Приморская лагуна/лиман, от пресного до солёного в разных частях, проточное
    \item Тектоническое (грабен), солёное, бессточное
    \item Тектоническое (межгорная котловина и разлом), солоноватое, бессточное
    \item Тектоническое (тектонический прогиб), пресное, проточное
    \item Тектоническое (тектонический прогиб), солоноватое, бессточное
    \item Тектоническое (разлом), пресное, проточное
\end{enumerate}

Обратите внимание, что одно изображение соответствует одному описанию. Рекомендуем заранее просмотреть все изображения.

Озеро 1:
\putImgWOCaption{8cm}{1}
Озеро 2:
\putImgWOCaption{8cm}{2}
Озеро 3:
\putImgWOCaption{8cm}{3}
Озеро 4:
\putImgWOCaption{8cm}{4}
\newpage
Озеро 5:
\putImgWOCaption{8cm}{5}
Озеро 6:
\putImgWOCaption{8cm}{6}
Озеро 7:
\putImgWOCaption{8cm}{7}
\newpage
Озеро 8:
\putImgWOCaption{8cm}{8}
Озеро 9:
\putImgWOCaption{8cm}{9}
Озеро 10:
\putImgWOCaption{8cm}{10}
\newpage
Озеро 11:
\putImgWOCaption{8cm}{11}
Озеро 12:
\putImgWOCaption{8cm}{12}