\assignementTitle{}{8}{}

В Северной Африке, Аравии и Сирии "вади" называют сухие русла рек и речных долин, которые могут иногда заполняться водой (например, после сильных дождей). Это арабское слово часто встречается там в географических названиях. В пустынях Средней Азии такие сухие русла и мёртвые речные долины называют "узбой". (Это также и имя собственное для конкретной древней речной долины в Туркмении.)

А какое название используется для сухих русел древних рек в засушливых регионах Намибии и северо-запада Ботсваны?

\answerMath{Omuramba.}