\assignementTitle{}{8}{}

В Северной Африке, Аравии и Сирии "вади" называют сухие русла рек и речных долин, которые могут иногда заполняться водой (например, после сильных дождей). Это арабское слово часто встречается там в географических названиях. В пустынях Средней Азии такие сухие русла и мёртвые речные долины называют "узбой". (Это также и имя собственное для конкретной древней речной долины в Туркмении.)

А какое название используется для сухих русел древних рек в засушливых регионах Намибии и северо-запада Ботсваны?

\explanationSection

Задача чисто на знание. Слово «омурамба» (omuramba, мн.ч.: omiramba) используется для названия сухих русел рек в пустыне Калахари. Слово происходит из языка гереро. В истории эти географические объекты, в частности, получили печальную известность в связи с боями, которые велись колониальными войсками кайзеровской Германии против африканского народа гереро.

В 1904-1908 гг. входе жестокого подавления германскими войсками восстания племён гереро и нама (готтентотов) были истреблено, по некоторым данным, до 80\% гереро и около половины нама. После поражения в открытом сражении большинство гереро были оттеснены в пустыню Калахари за пределы тогдашней германской колонии Намибии, где многие из них погибли. Эти события рассматриваются сегодня как один из самых ранних случаев геноцида в ХХ веке. В 2004 году гоноцид гереро и нама был признан Германией.

Этот печальный эпизод упоминается и в школьном курсе истории Нового времени.
\answerMath{Omuramba.}