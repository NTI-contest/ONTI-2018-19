\assignementTitle{}{10}{}

Выберите страны-лидеры по производству электроэнергии и добыче полезных ископаемых по данным за 2017 год.

Поставьте в соответствие каждому параметру тройку стран, которые лидируют по этому параметру в мире.

\begin{enumerate}
    \item Общий объём электроэнергии, выработанной на атомных электростанциях
    \item Доля электроэнергии, выработанной на атомных электростанциях
    \item Общий объём электроэнергии, выработанной на гидроэлектростанциях
    \item Доля электроэнергии, выработанной на гидроэлектростанциях
    \item Добыча нефти
    \item Добыча газа
    \item Добыча железной руды
    \item Добыча бокситов (основного сырья для производства алюминия)
    \item Добыча медной руды
    \item Добыча никелевых руд
    \item Добыча урана
    \item Добыча алмазов
\end{enumerate}

\begin{enumerate}
    \item[a)] Россия, Саудовская Аравия, США
    \item[б)] Чили, Перу, Китай
    \item[в)] Китай, Канада, Бразилия
    \item[г)] Казахстан, Канада, Австралия
    \item[д)] Австралия, Бразилия, Китай
    \item[е)] Австралия, Китай, Гвинея
    \item[ж)] Россия, Ботсвана, Канада
    \item[з)] Франция, Украина, Словакия
    \item[и)] Индонезия, Филиппины, Канада/Франция (Новая Каледония)
    \item[к)] США, Россия, Иран
    \item[л)] Албания, Демократическая Республика Конго, Парагвай
    \item[м)] США, Франция, Китай
\end{enumerate} 

\explanationSection

Задача, во многом, аналогичная предыдущей, но требующая эрудиции в области экономической географии.

Часть данных (по выработке электроэнергии) можно найти на портале открытых данных Всемирного банка, доступ к которой был у участников тура: \url{https://data.worldbank.org/}. Портал не имеет русскоязычного интерфейса, но для поиска необходимой информации достаточно знаний английского языка в рамках программы средней школы. Портал позволяет искать существующие статистические данные по стране или по показателю (индикатору). Страны сгруппированы по алфавиту. Показатели сгруппированы по разделам, названия которых, в большинстве случаев, понятны при минимальном знании языка. Кроме того, портал предоставляет возможности поиска по конкретным странам и ключевым словам.

Большинство показателей, речь о которых идёт в условии задачи, сгруппированы в разделе «Energy \& Mining» (энергия и добыча полезных ископаемых). Надо только обратить внимание на закладку «All indicators» («Все показатели»), чтобы получить полный их список. Отдельные показатели за ряд лет по всем странам могут быть сгружены в формате электронной таблицы. В частности, через Портал открытых данных Всемирного банка могут быть найдены или вычислены из имеющихся следующие показатели:
\begin{itemize}
    \item «Доля электроэнергии, выработанной на атомных электростанциях»
(«Electricity production from nuclear sources (\% of total)»)
    \item «Доля электроэнергии, выработанной на гидроэлектростанциях»
(«Electricity production from hydroelectric sources (\% of total)»)
    \item «Общий объём электроэнергии, выработанной на гидроэлектростанциях»
(вычисляется с помощью «Electricity production from renewable sources, excluding hydroelectric (kWh)» и из «Electricity production from renewable sources, excluding hydroelectric (\% of total)», по которым можно рассчитать общий объём производства энергии)
    \item Общий объём электроэнергии, выработанной на атомных электростанциях
(вычисляется с помощью «Electricity production from renewable sources, excluding hydroelectric (kWh)» и из «Electricity production from renewable sources, excluding hydroelectric (\% of total)», по которым можно рассчитать общий объём производства энергии)
\end{itemize}

Показатели по добыче полезных ископаемых невозможно посмотреть напрямую. Это требует хороших знаний школьного курса экономической географии зарубежных стран, а также, по ряду вопросов, дополнительной общей эрудиции.

В задаче специально подобраны тройки стран-лидеров по каждому показателю а не отдельные страны. Лидеры вообще могут регулярно меняться. Также разные источники иногда отдают лидерство в тот или иной год разным странам. Однако, тройки лидеров более устойчивы и, кроме того, помогают участникам найти правильный ответ. Можно не помнить, какая именно страна добыла в 2017 году больше всех нефти, но тройка мировых лидеров общеизвестна и часто обсуждается в средствах массовой информации – Россия, Саудовская Аравия, США. Аналогично широко известно лидерство России и США в добыче природного газа. Значительная доля Ирана на мировом рынке газа также часто обсуждается в СМИ.

Показатели добычи рудных полезных ископаемых не так широко обсуждаются, как нефть и газ. Однако, ряд фактов известен достаточно широко:
\begin{itemize}
    \item большая роль Австралии в добыче ряда твёрдых полезных ископаемых, особенно руд металлов;
    \item роль Казахстана как источника урана для советских атомных проектов;
    \item ведущая роль Чили и Перу в мире в добыче медной руды и связанные с этим проблемы;
    \item ведущая роль нашей страны в добыче алмазов, а также важная (в том числе историческая) роль стран южной Африки.
\end{itemize}

Всё это, в значительной мере позволяет довольно однозначно определить тройку лидеров в добыче медной руды, урана и алмазов.

Немного труднее определить тройку лидеров в добыче железной руды и бокситов. В обоих случаях лидирующая роль принадлежит Австралии и очень значительна роль Китая. Решить этот вопрос можно, вспомнив, что добыча железной руды является крупнейшей отраслью промышленности также в Бразилии. Роль Гвинеи в добыче бокситов обычно менее известна.

Лидеры в добыче никелевых руд могут быть определены по остаточному принципу. Хотя роль Индонезии, Филиппин и Канады в добыче никелевых руд также известна тем, кто интересуется вопросами распределения природных ресурсов в мире.

Правильные соответствия даны ниже.


\answerMath{1 - м, 2 - з, 3 - в, 4 - л, 5 - а, 6 - к, 7 - д, 8 - е, 9 - б, 10 - и, 11 - г, 12 - ж.}