\solutionSection

\begin{enumerate}
    \item Исходя из данных задачи в реакцию ПЦР было внесено 60 нг ДНК ($6 \cdot 10^{-5}$ мг) \textit{1 балл}. 
    Если поделить полученное число на массу 46 хромосом, получим число 10000 – это количество наборов 
    хромосом в смеси \textit{1 балл}. Так как данных копий гена в каждом наборе два, то количество копий матричной 
    ДНК равно 20000 \textit{2 балла}. Количество получаемых после ПЦР копий можно посчитать следующим образом: 
    $20000 \cdot 2^{30}$ (справедливо, так как эффективность ПЦР 100\%, рост количества матрицы экспоненциальный), 
    что примерно равно $2,15 \cdot 10^{13}$ \textit{2 балла} за формулу ещё \textit{1 балл} за конечный ответ.
    \item 1500 пн соответствует 500 кодонам \textit{1 балл}. Значит белок состоит из 500 
    аминокислот, масса равна $500 \cdot 100 = 50000$ а.е.м. (50 кДа). \textit{1 балл}
    \item Так как указана матричная цепь ДНК, нужно по принципу комплементарности построить мРНК: 5’-AUGAGAAAUGUUGCGUAUAAACCUGAU-3’
    С данной мРНК будет транслироваться следующий пептид: мет-арг-асн-вал-ала-тир-лиз-про-асп \textit{2 балла}. При нейтральном рН положительный заряд несут арг и лиз, отрицательный – асп \textit{1 балл}. Суммарный заряд пептида будет положительным \textit{2 балла}.
\end{enumerate}