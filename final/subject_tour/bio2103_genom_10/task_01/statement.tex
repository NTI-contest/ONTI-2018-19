\assignementTitle{}{14}{1}

ПЦР (полимеразная цепная реакция) – это метод, позволяющий получить огромное количество копий определённого фрагмента ДНК. В ходе данной реакции используются ДНК-полимеразы, производящие синтез новых цепей ДНК на основе содержащихся в растворе матричных ДНК; исходя из свойств данного метода, после проведения каждого следующего цикла реакции количество ДНК увеличивается в 2 раза по сравнению с предыдущим циклом (при условии 100\% эффективности реакции).

\begin{enumerate}
    \item Была проведена ПЦР для амплификации определенного гена. В качестве матрицы использовали геномную ДНК человека. 
    Известно, что в реакцию внесли 2 мкл ДНК с концентрацией 30~нг/мкл, число циклов реакции было 30. 
    ДНК была взята от человека, гомозиготного по данному гену (число аллелей – два). Какое количество 
    гаплоидных наборов хромосом было внесено в реакцию ПЦР? Какое количество копий было получено по 
        окончании реакции? (эффективность ПЦР считать 100\%, масса 46 хромосом человека – $6 \cdot 10^{-9}$~мг).
    \item Известно, что длина кодирующей аминокислоты области данного гена составляет 1500 пар нуклеотидов. Определите ожидаемую молекулярную массу белка, который закодирован в этой последовательности. Считайте среднюю массу аминокислоты равной 100 г/моль.
    \item Ниже приведена последовательность начала кодирующей части матричной цепи ДНК. Оцените заряд белка, кодируемого данной последовательностью мРНК при нейтральном рН.
\end{enumerate}
 
\begin{center}
    3’-TACTCTTTACAACGCATATTTGGACTA-5’    
\end{center}


Для решения задачи используйте таблицу генетического кода и структуры аминокислот.

\putImgWOCaption{16cm}{1}

\putImgWOCaption{14cm}{2}