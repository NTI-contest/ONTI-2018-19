\solutionSection

Так как любая комбинация с мутацией гена W приводит к формированию белых цветков, то продукт этого гена 
первый в пути биосинтеза \textit{1 балл}. Так как любая комбинация с мутацией гена В (кроме комбинации с W) приводит 
к формированию синих цветков, то ген В следующий в данном пути \textit{1 балл}. Мутация по гену Y приводит к 
формированию жёлтых цветков в комбинации с мутациями по генам O и R, значит жёлтый пигмент образуется 
из синего, и ген Y следует в пути за геном В \textit{1 балл}, следующим является ген О (по такой же логике) \textit{1 балл} и 
финальный ген – R, так как нет комбинации двух мутаций, приводящих к формированию красных цветков, это говорит 
о том, что это предпоследняя стадия формирования розового пигмента \textit{2 балла}. 

Таким образом, порядок генов: W-B-Y-O-R \textit{2 балла}; порядок синтеза пигмента: белый, синий, жёлтый, оранжевый, 
красный, розовый \textit{2 балла}.
