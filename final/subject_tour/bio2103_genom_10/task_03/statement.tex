\assignementTitle{}{10}{3}

Представьте, что вы производитель новых сортов роз, которые получаете с помощью инструментов геномного 
редактирования. Внося мутации в гены, которые относятся к тому или иному этапу метаболизма растительного пигмента, 
вы получаете розы определённого цвета. Дикий тип розы – розовые цветки. Однако до сих пор вы вносили одну мутацию 
в случайный ген метаболизма пигмента и отбирали розы интересующего цвета. Были получены розы следующих цветов: 
белого ($W^-$), жёлтого ($Y^-$), красного ($R^-$), оранжевого ($О^-$) и синего ($B^-$).

Представим, что существует путь биосинтеза розового пигмента, в ходе которого каждый следующий пигмент 
преобразуется определённым ферментом и дает другой цвет. Для определения порядка действия этих ферментов 
Вы вносили единичные мутации в два гена ферментов розовой розы, приводящие к потере активности двух ферментов 
данного пути.

Ниже приведены цвета роз – двойных мутантов во всех возможных комбинациях двух внесенных мутаций.

\begin{tabular}{| l | l | l | l | l |}
    \hline
     & $B^-$ & $O^-$ & $Y^-$ & $R^-$\\
    \hline
    $W^-$ & белый & белый & белый & белый \\
    \hline
    $B^-$ & & синий & синий & синий\\
    \hline
    $O^-$ & & & жёлтый & оранжевый\\
    \hline
    $Y^-$ &  &  & & жёлтый\\
    \hline
\end{tabular}\\

Восстановите последовательность мутаций в пути биосинтеза розового пигмента и порядок действия генов в этом пути. Обоснуйте свои рассуждения.