\assignementTitle{}{10}{4}

Как известно, биосинтез ДНК, РНК и белков являются матричными процессами, при этом элонгация принципиально осуществляются по одному из двух механизмов (см. рис): вспомогательная группа либо освобождается от растущей цепи (тип 1 на рис.), либо отщепляется от единицы, добавляющейся к растущей цепи. 

\putImgWOCaption{9cm}{1}

\begin{enumerate}
    \item Укажите, по какому типу происходит элонгация при репликации, транскрипции и трансляции.
    \item Для каждого процесса укажите, что является вспомогательной группой. 
    \item Все матричные процессы происходят последовательно. На основании их существования и последовательности была сформулирована центральная догма молекулярной биологии. Схематично она выглядит так:
    \putImgWOCaption{5cm}{2}
\end{enumerate}

Напишите название процесса, соответствующего каждой цифре. Какие из этих процессов происходят не во всех клетках (и в каких случаях происходят)? Для каждого процесса укажите ключевой фермент (или клеточную структуру), осуществляющий элонгацию, и его субстрат.