\solutionSection

\begin{enumerate}
    \item Тип 2 – репликация и транскрипция \textit{2 балла (по 1 за каждый процесс)}, Тип 1 – трансляция \textit{1 балл}.
    \item Вспомогательная (уходящая) группа: транскрипция и репликация — пирофосфат \textit{2 балла}. Трансляция — тРНК \textit{1 балл}.
    \item 1 – репликация, 2 – транскрипция, 3 – обратная транскрипция, 4 – трансляция \textit{4 балла, по 1 за каждую цифру}. 
    Обратная транскрипция не является универсальным процессом \textit{1 балл}, происходит в заражённых ретровирусами 
    клетках (необходимо для получения ДНК по матрице геномной РНК), также  идет при удлинении теломер в 
    стволовых клетках, клетках зародышевой линии, в раковых клетках. \textit{1 балл}. Репликация ДНК также происходит 
    не во всех клетках: некоторые остаются в G0-фазе клеточного цикла \textit{1 балл}. Для репликации ключевым ферментом 
    элонгации является ДНК-зависимая ДНК-полимераза, её субстратом являются дизоксирибонуклеотидтрифосфаты (здесь 
    и далее \textit{1 балл} за указание фермента, \textit{1 балл} за указание субстрата, но только если указан фермент). 
    Для транскрипции ключевым ферментом элонгации является ДНК-зависимая РНК-полимераза, субстрат – 
    рибонуклеотидтрифосфаты. Для обратной транскрипции ключевым ферментом элонгации является РНК-зависимая 
    ДНК-полимераза (обратная транскриптаза, ревертаза, теломераза), субстрат – дизоксирибонуклеотидтрифосфаты. 
    В процессе трансляции элонгация обеспечивается рибосомами, субстратом являются аминоацил-тРНК.
\end{enumerate}