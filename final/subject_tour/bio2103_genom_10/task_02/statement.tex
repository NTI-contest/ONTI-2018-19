\assignementTitle{}{5}{2}

В биологических лабораториях используют много различных растворов. Известно, что многие из них могут «зарастать», со временем в них размножаются бактерии, и они становятся непригодными для работы. Ниже перечислены составы некоторых активно используемых растворов. Выберите из них те, которые потенциально могут быть пригодными для жизни бактерий. 

Предположите, для каких именно жизненных процессов данные вещества могут быть использованы.

\begin{enumerate}
    \item Раствор ТЕ – трис-ЭДТА
    \item Буфер ТАЕ – трис-уксусная кислота-ЭДТА
    \item Цитратный буфер (содержит лимонную кислоту)
    \item PBS – фосфатный солевой буфер (содержит соли ортофосфорной кислоты и хлорид натрия)
    \item Буфер TGB (содержит трис и глицин)
\end{enumerate}

Ниже приведены структуры компонентов буферных растворов.

\putImgWOCaption{10cm}{1}

\putImgWOCaption{10cm}{2}