\solutionSection

Цепь: \textbf{мелкие растения$\to$белки, зайцы$\to$совы, лисы, \\
мелкие растения$\to$олени} \textit{(2 балла)}

Нормальные условия:

На уровне мелких растений: $16\cdot228000=\textbf{3648000}$ ккал \textit{(1 балл)}

Пусть $X$ – уровень перехода, $Y$ – количество энергии у белок, зайцев, оленей, то есть суммарно на уровне консументов 1 порядка $3Y$, решим уравнения:

$$3648000\cdot X=3Y$$
$$2Y\cdot X=62259$$

$$Y=1216000X$$
$$2\cdot1216000X\cdot X=62259$$
$$X^2=0.0256$$
$$X=0.16=16\%$$
$\textbf{Y=194560}$ ккал — у каждого вида на втором уровне \textit{(3 балла)}

У совы и лисы по: \textbf{31129.5} ккал \textit{(2 балла)}

Вторая часть задачи:

$16-1.8 = 14.2$ га осталось

С него энергии: $14.2\cdot228000=3237600$ ккал, $3237600\cdot16\%=518016$ ккал переходит на следующий уровень.

Знаем, что оленям и белкам надо по 194560 ккал энергии, а зайцам теперь нужно: $194560\cdot65\%=126464$ ккал

Проверям, достаточно ли. Надо: $194560\cdot2+126464=515584$ ккал.

А есть: 518016 ккал $\Rightarrow$ На этом уровне всем хватает.

Лисам и совам переходит: $(194560+126464)\cdot16\%=51363.84$. Не хватает. \textbf{Лисы и совы} пострадают раньше всего. Им не хватает \textbf{10895.7} ккал

\textbf{Лисы могу начать питаться оленями}. 

Тогда они получат ещё $194560\cdot16\%=31129.6$. Это даже в избытке и им \textbf{хватит} для выживания. \textit{(5 баллов за вторую часть)}