\assignementTitle{}{9}{2}

Предположим, что растения умеют очень быстро адаптироваться к новым условия среды обитания как на клеточном, 
так и на морфологическом уровне. Если эпифитная орхидея экваториального леса попадёт в новые для неё условия 
обитания с экстремально низкими температурами, сильным ветром, отсутствием деревьев, мелких насекомых и птиц, 
какие морфологические изменения должны произойти в строении её органов, чтобы орхидея смогла существовать в 
таких условиях? Что изменится в её образе жизни? 

Чем опасны низкие температуры окружающей среды для клеток растений? Как клетки листьев могут адаптироваться к 
низким температурам, чтобы орхидея выжила в таких экстремальных условиях?