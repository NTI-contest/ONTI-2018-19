\explanationSection

Ответ в свободной форме, важно сохранить ключевые моменты.

Корни перестанут быть воздушными \textit{(1~балл)}. Орхидее нужно будет либо сбрасывать листья на зиму, либо использовать другие механизмы для уменьшения площади поверхности испарения воды (например, скручивать листья) \textit{(1~балл)}. Наземная часть растения должна накапливать сахара и масла, а подземная крахмал \textit{(1~балл)}.

Орхидее не нужен будет яркий и крупный околоцветник, т.к. опыление теперь будет ветром, а не насекомыми. Цветки станут мелкими и не яркими. \textit{(1~балл)}

Не сможет вести эпифитный образ жизни из-за отсутствия деревьев. \textit{(1~балл)}

Экстремально низкие температуры опасны тем, что в клетках образуется лёд \textit{(1~балл)}, его кристаллы разрушают клетку. \textit{(1~балл)}

Чтобы не было льда клетка должна иметь высоко проницаемые мембраны, для быстрого транспорта воды из клетки. В мембранах должно увеличиться количество ненасыщенных жирных кислот. Также накопление сахара увеличивает осмотическое давление в клетках, а масла вытесняют воду в вакуоли, защищая клетку от вымерзания. \textit{(2~балла)}