\solutionSection

Цепь: фитопланктон $\to$ минтай $\rightarrow$ чайки, люди, морские котики \textit{(2 балла)}

Переход между минтаем и консументами второго порядка $25-5=18\%$.

5\% минтая гибнет, значит остаётся $500\cdot95\% = 475$ кг на пищевую цепь.

Токсинов: 150 г в 3.85 кг$\cdot30\% = 1.155$ кг

$475\cdot30\%=142.5$ кг биомасса минтая

Значит в 142.5 кг будет $142.5\cdot\frac{0,15}{1.115} = 19.17$ кг токсинов в биомассе минтая

на следующий уровень передаётся: $19.17\cdot18\% = 3.45$ кг токсинов

Как распределяется между разными видами: $70+90+1 = 161$

Человеку: $\frac{70}{161}=0.435$, морскому котику: $\frac{90}{161}=0.559$, чайке: $\frac{1}{161}=0.006$

Люди получат: $3.45\cdot0.435 = 1.5$ кг, морские котики: $0.559\cdot3.45 = 1.9$ кг, чайки: $3.45\cdot0.006 = 0.02$ кг (\textit{6 баллов, по 2 за правильное количество для каждого вида}).

$\frac{1.5}{8} = 0.1875$ кг токсинов в неделю получает один человек. Это 0.27\%

$\frac{3}{0.27} = 11.1$ недель. Это 2.7 месяца \textit{(2 балла)}