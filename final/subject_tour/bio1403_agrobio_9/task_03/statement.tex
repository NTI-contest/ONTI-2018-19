\assignementTitle{}{10}{3}

Залив загрязняется вредными химическими отходами, которые с водой и фитопланктоном попадают в минтай и 
накапливаются в его мышечной ткани. 5\% минтая гибнет от отравления, выпадая таким образом из трофической цепи. 
1 минтай в среднем запасает 150~г токсинов в своей сухой биомассе, вместе с которой они передаются консументам 
следующего порядка. На одном трофическом уровне полученная биомасса распределяется между разными видами 
пропорционально среднему весу одной особи вида. В пищевой цепочке участвуют люди (средний вес 1 особи 70~кг), 
чайки (средний вес одной особи 1~кг), морские котики (средний вес одной особи 90~кг). Запишите пищевую цепь и 
вычислите, сколько токсинов попадает в каждый из видов в неделю, если известно, что вес обитающего в заливе минтая 
составляет 500~кг в неделю, средний вес одной рыбы 3.85~кг, минтай на 70\% состоит из воды. Переход биомассы между 
продуцентами и консументами первого порядка 23\%, между каждым последующим уровнем он уменьшается на 5\%.

Рыбой из залива питается 8 человек, критическая норма токсинов для человека~— 3\% от массы его тела. Через 
сколько месяцев отравится первый человек?
