\solutionSection 

Уравнение фотосинтеза: $6CO_2 + 6H_2O \to C_6H_{12}O_6+6O_2$ \textit{(1 балл)}

Уравнение нитрификации: $2NH_3 + 4O_2= 2NO_3^- + 2H^+ + 2H_2O$ \textit{(1 балл)}

$$1 \: \text{л} \: \text{воды} = 1 \: \text{кг} = 103 \: \text{г}$$
$$n\text{(воды)}= \frac{m}{M}=\frac{6\cdot10^3}{18}=333 \: \text{моль}$$

$n$(кислорода выделяемого): 333 моль

Бактерии используют 5\% - это $0.05\cdot333 = 16.65$ моль.

$n$(нитрата) $= \frac{16.65}{2} = 8.325$ моль \textit{(3 балла)}

$8.325\cdot800\cdot4=26640$ моль \textit{(2 балла)} с 8 соток в месяц.