\solutionSection
\begin{enumerate}

    \item[a)] Всего вариантов расставить двух различных королей ---
	$64 \cdot 63 = 4032$. Теперь рассмотрим варианты, когда они бьют друг друга.
	Таких вариантов \linebreak $4 \cdot 3 + 24 \cdot 5 + 36 \cdot 8 = 420$.
	Искомая вероятность будет равна $\frac{420}{4032} = \frac{5}{48}$
	
	\item[б)] Эти три фигуры могут попарно друг друга бить только в том случае, 
	если они стоят <<уголком>>, где король и ферзь стоят по диагонали, а 
	ладья соседствует с ними обоими. Тогда заметим, что всего подобных уголков 
	в каждом $2 \times 2$ квадратике можно сделать $8$ штук, а значит, так как
	в $8 \times 8$ есть $49$ различных (они накладываются друг на друга) квадратов
	$2 \times 2$, то всего вариантов, когда все они друг друга бьют --- $49 \cdot 8 = 392$.
	
	Следовательно, так как всего $64 \cdot 63 \cdot 62 = 249984$ вариантов, то
    искомая вероятность равна $\frac{392}{249984} = \frac{7}{4464}$.
    
\end{enumerate}

\answerMath{а) $\frac{5}{48}$; б) $\frac{7}{4464}$.}

\additionalCriteria

\begin{enumerate}
		
	\item[a)] \textit{(10 баллов)}

	$+10$ баллов за рассуждения, объясняющие найденную вероятность.	
	
	\item[б)] \textit{(15 баллов)}
	
	$+5$ баллов за рассуждения о возможных позициях фигур;
	
	$+10$ баллов за рассуждения, объясняющие найденную вероятность.

\end{enumerate}

\underline{Замечания} 
\begin{enumerate}
	\item Если в процессе вычислений вероятностей допущены арифметические ошибки,
	но все рассуждения правильны, то балл не снижается.
\end{enumerate}