\solutionSection

Решив первое уравнение каким-либо известным способом находим его корни	
$$x_1 = 7, x_2 = -6.$$
    
Аналогично находим корни второго уравнения: 
$$x_1 = -1, x_2 = 7, x_3 = -\frac{2}{3}.$$
    
Общим корнем будет $x = 7$.

\answerMath{ $7$.}

\additionalCriteria

$+4$ баллов за нахождение корней первого уравнения;

$+9$ баллов за нахождение корней второго уравнения;

$+2$ балла за выбор общего корня.
	
\underline{Замечания} 

\begin{enumerate}
	\item За нахождения общего корня перебором баллы снижаются
	только в случае, когда не была проведена проверка того, что найденное
	число является корнем \underline{обоих} уравнений.

	\item Если найденный общий корень не совпадает с ответом,
	то за задачу ставится $0$ баллов.	
	 
	\item Если при вычислении корней обоих уравнений были совершены 
	арифметические ошибки, то баллы за каждое из этих уравнений ставятся
	пропорционально числу правильно найденных корней.
\end{enumerate}