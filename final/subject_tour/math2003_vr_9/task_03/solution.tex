\solutionSection
\begin{enumerate}
    \item[a)] Заметим, что высота в такой трапеции
        будет равна $h = \sin{60^\circ} \cdot 6 = 3\sqrt{3}$. Тогда по
        формуле площади трапеции $S = \frac{1}{2}h(a+b) = 135$.
        
    \item[б)]    б) Так как $AB > CD$, то острый угол $B$~---~наименьший из углов трапеции.
        Опустим из вершины $C$ перпендикуляр $CE$ на $AB$. Его длина не больше длины $MN = 6$.
        Кроме того, из условия следует, что 
        $$BE = 18\sqrt{3} - 12\sqrt{3} = 6\sqrt{3}.$$
        
        Поэтому $\tg\angle B \leqslant \dfrac{1}{\sqrt{3}}$, откуда $\angle B \leqslant 30^\circ$.	

        Для того чтобы получить максимальное значение угла $B$, достаточно предположить, что $MN$~---~высота трапеции.   
\end{enumerate}

\answerMath{а) $135$.}

\additionalCriteria

\textit{(20 баллов)}
\begin{enumerate}
    \item[a)] \textit{(5 баллов)} 

	\underline{Общие критерии}

	$+5$ баллов за правильно найденную площадь.

	\item[б)] \textit{(15 баллов)} 
	
	\underline{Общие критерии}

	$+10$ баллов за доказанную оценку;
	
	$+5$ баллов за правильно подобранный пример с углами $30^{\circ}$ и $150^{\circ}$.
\end{enumerate}	
