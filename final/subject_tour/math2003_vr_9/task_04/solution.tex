\solutionSection
\begin{enumerate}
    \item[a)] Покажем, что Петя может всегда выиграть, как бы 
	Илья не ходил. Для этого Петя может всегда ходить противоположнвм образом,
	то есть если Илья поднял цену на $2000$ рублей, то Петя поднимет на $5000$
	рублей и наоборот. Тогда после $4$ таких ходов стоимость станет равной 
	$29000$ рублей и Петя заберет картину.
	
	\item[б)]  Рассмотрим игру с конца с позиции выигрышных и проигрышных позиций. Так как максимальная цена~--- $28200$ рублей, то
	$28200$~--- выигрышная, $28190$~--- выигрышная, \dots, $28160$~--- выигрышная, 
	$28150$~--- проигрышная, $28140$~--- проигрышная, $28130$~--- выигрышная,
	$28120$~--- выигрышная, $28110$~--- проигрышная, $28100$~--- выигрышная,
	$28090$~--- выигрышная, $28080$~--- проигрышная, $28070$~--- проигрышная, \dots
	
    Можно заметить, что в данном случае будут повторяться постоянно $7$ позиций:
    
    \begin{center}	
	    \dots П, П, В, В, П, В, В\dots 
    \end{center}

	Итак, получим, что $1050$ и $1020$ --- это проигрышные позиции, поэтому 
	как бы первый раз не сходил Илья он может потом продолжить играть так,
	чтобы выиграть.
\end{enumerate}	

\answerMath{а) Нет, у Пети есть выигрышная стратегия; б) Да, у Ильи есть выигрышная стратегия.}

\additionalCriteria

\begin{enumerate}
    \item[a)] \textit{(10 баллов)} 

	$+5$ баллов за нахождение стратегии для Пети;
	
	$+5$ баллов за рассуждения о том, что Илья проиграет, независимо от своей игры.

	\item[б)]  \textit{(15 баллов)} 
	
	$+10$ баллов за доказательство того факта, что позиции повторяются;
	
	$+5$ баллов за рассуждения о том, что Илья выиграет, независимо от игры оппонента.
\end{enumerate}
	
\underline{Замечания} 
\begin{enumerate}
	\item Если без каких-либо объяснений сказано, что позиции повторяются, то
	ставится $5$ вместо $10$ баллов.
	
	\item Если правильно найден вид периода позиций, но нет рассуждений, 
	почему, если рассматривать с конца, будут повторяться именно эти $7$ позиций,
	то ставится аналогично $5$ вместо $10$ баллов.
	
	\item Если все рассуждения о периодичности сделаны верно, но 
	для позиций $1050$ и $1020$ сказано, что хотя бы одна выигрышная,
	то снимается все $5$ баллов.  
\end{enumerate}
