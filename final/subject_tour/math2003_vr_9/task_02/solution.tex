\solutionSection

Найдем для начала хотя бы одно решение. Для этого переберем малые простые числа и убедимся, что $p = 5$ подходит.
	
Покажем, что других решений у уравнения нет. Для этого заметим, что если $p \neq 5$, то оно взаимно просто с пятью, а значит $p^4$ дает остаток 1 при делении на 5.

Тогда выражение $2p^4 + 46$ дает остаток $3$ при делении на $5$. Легко убедиться, что квадраты целых чисел могут давать только остатки $0, 1, 4$. Следовательно, больше нет решений в целых числах.

\answerMath{$p = 5$.}

\additionalCriteria

	$+5$ баллов за подобранный пример;
	
	$+10$ баллов за доказательство единственности.

	
