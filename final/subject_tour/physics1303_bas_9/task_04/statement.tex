\assignementTitle{}{}{4}

Участок электрической цепи робота-пылесоса представляет из себя медный провод длиной $l=10$~см и  площадью 
поперечного сечения $S=2$ мм$^2$. Определить количество теплоты, выделяемое этим участком цепи за время $t=30$ мин 
работы пылесоса при постоянной мощности всасывания, которой соответствует протекающий по проводу ток $I=0,5$ А.  
Во сколько раз  изменится количество теплоты, выделяемое этим участком цепи, если параллельно медному проводу  
подсоединить еще один такой же провод в два раза большей длины. Удельное сопротивление меди принять равным $\rho=0,0175$~Ом$\cdot$мм$^2$/м. 
Разность потенциалов на концах рассматриваемого участка цепи поддерживается постоянной. Ответ округлить до 
второго знака после запятой.