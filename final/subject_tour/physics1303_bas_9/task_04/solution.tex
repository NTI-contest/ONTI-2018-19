\solutionSection

\begin{tabular}{l|}
    Дано: \\
    $l = 10$ см \\
    $S = 2$ мм$^2$ \\
    $t = 30$ мин \\
    $I = 0.5$ А \\
    $\rho = 0.0175$ Ом $\cdot$ мм$^2$ / м \\
    \hline \\
    $Q$ - ?
\end{tabular}

$$R = \rho \frac{l}{S}$$

Количество теплоты, выделяемое медным проводником 

$$Q = I^2 R t = I^2 \rho \frac{l}{S} t = 0.5^2 \cdot 0.0175 \cdot 0.1 \cdot 30 \cdot 60 / 2 \approx 0.39 \: \text{Дж}$$
 
В результате подключения второго проводника суммарная сила тока на рассматриваемом участке изменится:

$$I_1 = \frac{U}{R}; \: I_2 = \frac{U}{2R}; \: I_{12} = I_1 + I_2 = \frac{3}{2} \frac{U}{R} = \frac{3}{2} I$$

Сопротивление участка после подключения второго проводника

$$\frac{1}{R_\text{н}} = \frac{1}{R} + \frac{1}{2R}; \: R_\text{н} = \frac{2}{3} R$$
 
Новое количество теплоты

$$Q_\text{н} = \left( \frac{3}{2} I \right)^2 \frac{2}{3}Rt = \frac{3}{2}I^2Rt$$
 
Тогда

$$\frac{Q_\text{н}}{Q} = \frac{\frac{3}{2}I^2Rt}{I^2Rt} = 1.5$$

\answerMath{$Q=0.39$ Дж; Количество теплоты увеличится в 1.5 раза.}
