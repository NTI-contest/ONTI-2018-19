\assignementTitle{}{}{2}

Сфероробот, представляющий из себя сферу диаметром $d=40$ см с подвижной массой внутри, движется по горизонтальной 
плоскости с постоянной скоростью $v=1$м/с без проскальзывания и пробуксовки. Сколько полных оборотов должен 
сделать робот для того, чтобы переместиться на расстояние $S=1$км. Определить ускорение точек боковой поверхности 
робота, касающихся горизонтальной плоскости, относительно центра сферы.  На сколько изменится скорость робота в 
условии задачи, если на всем пути $S$ имела место постоянная пробуксовка, в результате которой он сделал в 4 раза 
больше оборотов при тех же условиях работы его движителя.