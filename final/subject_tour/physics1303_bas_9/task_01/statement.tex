\assignementTitle{}{}{1}

По команде с пульта управления квадрокоптер массой $m = 1$ кг начинает подниматься вертикально вверх с постоянной 
скоростью $v=12$ м/с. На высоте $h=50$ м от поверхности Земли в результате сбоя в электрической цепи пропеллеры 
квадрокоптера останавливаются, и через некоторое время он падает на Землю. На какой высоте $H$ от поверхности 
Земли за время полета неисправного квадрокоптера его скорость  будет минимальной? Чему равна эта скорость?  
Ускорение свободного падения g принять равным 9,8 м/с$^2$. Сопротивлением воздуха пренебречь. Ответ округлить 
до второго знака после запятой. Как изменится результат, если учесть среднюю силу сопротивления воздуха $F_c = 2,2$~Н.