\assignementTitle{}{22}{3}

Элемент космического мусора относительно небольшого размера массой \linebreak $m=0.1$~кг попадает в обшивку 
научно-исследовательского спутника массой \linebreak $M=50$~кг и застревает в ней. Угол между скоростями спутника и элемента 
космического мусора перед столкновением составлял $\alpha=120^{\circ}$ (см. рисунок), скорость спутника перед 
столкновением $V=7$ км/с, скорость элемента космического мусора перед столкновением $v=10$ км/с. Определить 
изменение кинетической энергии спутника в результате столкновения (ответ дать в МДж и округлить до десятых долей). 
Спутник и элемент космического мусора считать материальными точками.

\putImgWOCaption{6cm}{1}