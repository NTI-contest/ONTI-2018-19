\assignementTitle{}{18}{5}

Навигатор путешественника, находящегося на южном полюсе Земли (90$^{\circ}$ южной широты) зафиксировал сигнал от 
Российского навигационного спутника системы ГЛОНАСС. Высота полета спутника над уровнем моря составляет $H=19400$ км, угол 
отклонения спутника от оси вращения Земли, выходящей из южного полюса, составляет $\alpha=20^{\circ}$ (см. рисунок). 
Определить расстояние $X$ от спутника до приемной антенны навигатора (результат округлить до километров), а 
также длину волны поступившего сигнала (результат округлить до сантиметров), если частота сигнала $v=1246$ МГц. 
Определить время задержки сигнала $\Delta t$ (это время, через которое сигнал, испущенный передающим устройством 
спутника, фиксируется навигатором), ответ дать в мс. Скорость распространения радиоволн принять равной $c=3\cdot 10^8$ м/с, 
радиус Земли $R_\text{З}=6400$ км.

\putImgWOCaption{6cm}{1}