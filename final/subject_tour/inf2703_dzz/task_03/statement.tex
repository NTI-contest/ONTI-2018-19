\assignementTitle{}{15}{}

Изучите расстановку группы геостационарных спутников некоторой компании. Напишите программу, которая по заданному списку координат геостационарных спутников определяет для каждого количество спутников в зоне видимости. Под зоной видимости понимается возможность прохождения сигнала между спутниками по прямолинейной траектории.

Если геостационарная орбита имеет радиус 42164км с центром, совпадающим с центром Земли, что соответствует высоте над уровнем моря 35786~км. То, радиус Земли равен 6378~км. Таким образом, Земля закрывает $\angle \alpha \approx 17.2034^{\circ}$ градусов обзора.

\putImgWOCaption{12cm}{1}

При этом у Земли есть толстые слои атмосферы, которые сильно искажают сигналы. В связи с этим,  $\angle \alpha = 20^{\circ}$,$\angle \alpha = 40^{\circ}$. Обратите внимание, что границы угла позволяют осуществлять передачу данных.

\inputfmtSection

В первой строке подается число $n\space (1\leq n \leq 100)$ — количество спутников у компании.

Далее в $n$ строках через пробел подаются $s_i (1 \leq s_i \leq 50)$ и $c_i (-180 < (c_i) \leq 180)$~— латинские названия спутников без разделителей и их целочисленные координаты (положительные — восточное полушарие, отрицательные — западное полушарие). В одной координате не располагается более одного спутника.

\outputfmtSection

Единственное число — ответ на задачу.

\markSection

Баллы за задачу будут начислены, если все тесты будут пройдены успешно.

\sampleTitle{1}

\begin{myverbbox}[\small]{\vinput}
    9
    NewStar-1 0
    NewStar-2 -40
    NewStar-3 160
    NewStar-4 80
    NewStar-5 40
    NewStar-6 -120
    NewStar-7 -160
    NewStar-8 120
    NewStar-9 -80
\end{myverbbox}
\begin{myverbbox}[\small]{\voutput}
    NewStar-7 -160 8
    NewStar-6 -120 8
    NewStar-9 -80 8
    NewStar-2 -40 8
    NewStar-1 0 8
    NewStar-5 40 8
    NewStar-4 80 8
    NewStar-8 120 8
    NewStar-3 160 8
\end{myverbbox}
\inputoutputTable

\sampleTitle{2}

\begin{myverbbox}[\small]{\vinput}
    18
    TVPoint-1 100
    TVPoint-2 -90
    TVPoint-3 80
    TVPoint-4 -80
    TVPoint-5 -110
    TVPoint-6 -115
    TVPoint-7 110
    TVPoint-8 120
    TVPoint-9 -100
    TVPoint-10 -105
    TVPoint-11 -120
    TVPoint-12 95
    TVPoint-13 -85
    TVPoint-14 90
    TVPoint-15 115
    TVPoint-16 85
    TVPoint-17 -95
    TVPoint-18 105
\end{myverbbox}

\begin{myverbbox}[\small]{\voutput}
    TVPoint-11 -120 17
    TVPoint-6 -115 16
    TVPoint-5 -110 15
    TVPoint-10 -105 14
    TVPoint-9 -100 13
    TVPoint-17 -95 12
    TVPoint-2 -90 11
    TVPoint-13 -85 10
    TVPoint-4 -80 10
    TVPoint-3 80 10
    TVPoint-16 85 10
    TVPoint-14 90 11
    TVPoint-12 95 12
    TVPoint-1 100 13
    TVPoint-18 105 14
    TVPoint-7 110 15
    TVPoint-15 115 16
    TVPoint-8 120 17
\end{myverbbox}
\inputoutputTable

\includeSolutionIfExistsByPath{final/subject_tour/inf2703_dzz/task_03}