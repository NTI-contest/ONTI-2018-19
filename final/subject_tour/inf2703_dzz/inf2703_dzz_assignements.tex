В системах цифровой связи используются различные алгоритмы кодирования передаваемых сообщений. При передаче данных, какими бы ни были совершенным устройства приема/передачи, изредка появляются ошибки. Поэтому очень часто последовательности передаваемых битов дополняют хитро сгенерированными битами таким образом, чтобы можно было обнаружить ошибку и даже исправить.

Одним из простейших кодов, позволяющих обнаруживать и исправлять единичные ошибки, является Код Хэмминга. Построение кода опирается на принцип четности суммы подпоследовательности битов конечного кода.

Рассмотрим классический Код Хэмминга. Для того, чтобы иметь возможность определить и исправить ошибку в одном из битов четырёхбитного двоичного слова $a_1a_2a_3a_4$, необходимо последовательно приписать после него 3 бита $b_1b_2b_3$, вычисленных по следующему принципу:

$$b_1 = a_1 \oplus a_2 \oplus a_3 $$

$$b_2 = a_2 \oplus a_3 \oplus a_4 $$

$$b_3 = a_1 \oplus a_2 \oplus a_4 $$

Знак $\oplus$ означает исключающее "или" (строгая дизъюнкция) и вычисляется как сложение битов по модулю 2.

Таким образом, при безошибочной передаче данных:

$$S_1 = a_1 \oplus a_2 \oplus a_3 \oplus b_1 = 0 $$

$$S_2 = a_2 \oplus a_3 \oplus a_4 \oplus b_2 = 0 $$

$$S_3 = a_1 \oplus a_2 \oplus a_4 \oplus b_3 = 0 $$

Если $S_1$, $S_2$, $S_3$ не нули, то из этого можно сделать следующие выводы:

\begin{tabular}{сccс}
    0 & 0 & 1 & b_3 \\
    0 & 1 & 0 & b_2 \\
    0 & 1 & 1 & a_4 \\
    1 & 0 & 0 & b_1 \\
    1 & 0 & 1 & a_1 \\
    1 & 1 & 0 & a_3 \\
    1 & 1 & 1 & a_2 \\
\end{tabular}

\subimport{task_01/}{statement.tex}
\subimport{task_02/}{statement.tex}

Геостационарный искусственный спутник Земли — искусственный спутник Земли, постоянно находящийся на геостационарной орбите (круговая орбита, расположенная в плоскости экватора Земли). Геостационарная орбита имеет радиус 42164 км с центром, совпадающим с центром Земли, что соответствует высоте над уровнем моря 35786 км. Период обращения спутника на данной орбите равен звёздным суткам (23 ч 56 мин 4 с), движется он в восточном направлении. Таким образом, спутник занимает постоянное положение относительно земной поверхности.

С геостационарного спутника Земля видна под углом 17°, что позволяет видеть с него примерно треть площади земной поверхности. Геостационарные спутники широко используются для ретрансляции радио- и телевизионных передач и радиосвязи между наземными станциями, расположенными за пределами прямой видимости друг друга. Они обеспечивают возможность ретрансляции сразу нескольких телевизионных программ и связи по нескольким тысячам телефонных каналов.

Каждый такой спутник находится над некоторой точкой, характеризующейся долготой (по широте она равна 0). 
Например, 11,0°W Экспресс-АМ44 (Express) находится в западном полушарии по координатам 0°с.ш. 11°з.д.

\subimport{task_03/}{statement.tex}
\subimport{task_04/}{statement.tex}

Вы руководите миссией по исследованию поверхности плонеты X. В вашем распоряжении суперсовременная ракета X-Mission, которая на борту имеет полный набор датчиков для измерения различных показателей, а также модуль для исследования грунта и осуществления других околоповерхностных измерений.

Спустя несколько долгожданных лет полета X-Mission достигла орбиты планеты. Ракета провела ряд измерений: в том числе, серии снимков поверхности планеты. Ваша задача на основе снимков составить карту высот, а затем спланировать посадку модуля для взятия проб грунта. 

\subimport{task_05/}{statement.tex}
\subimport{task_06/}{statement.tex}