\assignementTitle{Задача 2.2}{15}{}

После определения возможных направлений коммуникации между спутниками вспомним об их задачах. Предположим, назначением рассматриваемых спутников является трансляция телепередач (спутниковое телевидение). При коммуникации с Землей есть ряд ограничений, связанных с тем, что у планеты есть атмосфера, из-за которой, сигнал под углами, наиболее близкими к прямому, доходят с меньшим количеством помех. Из-за этого, каждый спутник может обслуживать только ближайшие сектора Земного шара определенного размера. 

\putImgWOCaption{12cm}{1}

Зная размеры секторов, определите, какое наименьшее количество спутников некоторой модели необходимо дополнительно вывести на орбиту, чтобы обеспечить максимально возможное покрытие. Обратите внимание, что не обязательно координата запускамого спутника должна быть целочисленной. 

\inputfmtSection

В первой строке подается число $ n\space(1 \leq n \leq 10^5) $, $ \theta\space (1 \leq \theta \leq 100) $ — количество спутников у компании и целочисленный размер сектора в градусах, доступный для обслуживания моделью спутников, планируемых к запуску.

Далее в $n$ строках через пробел подаются $ s_i\space(1\leq len(s_i)\leq 50)$, $ с_i\space(-180.0 \lt c_i \leq 180) $ и $ \gamma_i (1 \leq \gamma_i \leq 100)$  — латинские названия спутников на орбите без разделителей, их целочисленные координаты (положительные  восточное полушарие, отрицательные — западное полушарие) и целочисленный размер их зоны обслуживания в градусах.

\outputfmtSection

Единственное число — ответ на задачу.

\explanationSection

Зоны покрытия обозначены отрезками на следующем изображении, красным указаны размеры непокрытых отрезков.

\putImgWOCaption{12cm}{2}

\markSection

Баллы за задачу будут начислены, если все тесты будут пройдены успешно.

\sampleTitle{1}

\begin{myverbbox}[\small]{\vinput}
    9 20
    HelloWorld-1 -120 40
    HelloWorld-2 160 40
    HelloWorld-3 -160 20
    HelloWorld-4 40 10
    HelloWorld-5 0 10
    HelloWorld-6 -80 40
    HelloWorld-7 -40 40
    HelloWorld-8 80 20
    HelloWorld-9 120 40
    \end{myverbbox}
\begin{myverbbox}[\small]{\voutput}
    8
\end{myverbbox}
\inputoutputTable



\includeSolutionIfExistsByPath{final/subject_tour/inf2703_dzz/task_04}