\assignementTitle{}{20}{}

На основании измерений, произведенных с орбиты планеты X вы выявили некоторый прямоугольный участок:

\begin{itemize}
\item на котором минимальна вероятность бури и иных негативных погодных явлений,
\item обладают признаками наличия интересующих типов почв.
\end{itemize}

У вас есть серия снимков поверхности, для каждого участка которого вы можете при помощи специализированного программного обеспечения определить высоту. Однако, не в каждый момент времени все объекты на снимках хорошо видны: на планете X, как и на Земле, образуются облака. Ваша задача вопреки всем сложностям составить подробную карту высот на интересующем участке с целью дальнейшего использования для определения места посадки модуля-анализатора на поверхность планеты и построения для него маршрута. 

\inputfmtSection

В первой строчке через пробел подаются целые числа $a_x, a_y \space (-10^7 \leq a_x,$ \linebreak $a_y \leq 10^7)$, $s_x, s_y \space (1 \leq s_x, s_y \leq 10^3)$ и $l_x, l_y \space (1 \leq l_x, l_y \leq 10^3)$— координаты левого верхнего интересующего прямоугольника, размеры интересующего прямоугольника и размеры снимков ($x$ — строки, $y$ — столбцы, угол съемки не меняется).

В следующей строке подается целое число $n\space(1\leq n\leq 10^3)$ — количество отснятых снимков.

Далее будут описания снимков в формате:

\begin{itemize}
\item $b_x\space (a_x-l_x+1 \leq b_x\leq a_x+s_x-1), b_y \space (a_y-l_y+1 \leq b_y\leq a_y+s_y-1)$, — координаты верхнего левого угла снимка.
\item $l_x$ строк по $l_y$ значений, записанных через табуляцию, $f_{ij}\space (-10^4\leq f_{ij}\leq 10^4)$ — определенные значения высоты или латинская буква "x", если область закрыта облаками.
\end{itemize}

\outputfmtSection

$s_x$ строк по $s_y$ значений, записанных через табуляцию — карта высот интересующей местности. Укажите латинскую буква "x", если по указанному участку данные о высоте отсутствуют.

\markSection

Баллы за задачу будут начисляться пропорционально количеству успешно пройденных тестов.

Результат за успешное прохождение только первого теста (методом распечатки верного ответа или иным эквивалентным образом) засчитан не будет.

\sampleTitle{1}

\begin{myverbbox}[\small]{\vinput}
    -2 -4 6 4 2 3
    5
    -2 -5
    -5834	-4887	x
    3248	-8439	-8849
    3 -3
    8608	-7907	x
    6086	-206	7279
    1 -4
    628	9560	8310
    7148	1339	-7771
    -2 -1
    6546	7660	6677
    8941	8667	6376
    -3 -1
    -53	4960	-8522
    6546	7660	6677
\end{myverbbox}

\begin{myverbbox}[\small]{\voutput}
    -4887	x	x	6546	
    -8439	-8849	x	8941	
    x	x	x	x	
    628	9560	8310	x	
    7148	1339	-7771	x	
    x	8608	-7907	x	
\end{myverbbox}
\inputoutputTable


\includeSolutionIfExistsByPath{final/subject_tour/inf2703_dzz/task_05}