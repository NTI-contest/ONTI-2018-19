\assignementTitle{}{25}{}

Итак, вы получили снимок поверхности планеты, куда территориально планируется запустить модуль для взятия проб грунта. Увидев полученные данные, вы обратили внимание на то, что на планете преобладают сильно и глубоко расчленённые участки земной коры со складчатой или складчато-глыбовой структурой. Это означает, что поверхность характеризуется сильными перепадами высот. С одной стороны, это хорошо, так как можно попробовать оценит различные слои пород, не только поверхностный на глубине несколько метров. Однако, это создает ряд технических сложностей. Так, например, технические характеристики модуля:

\begin{itemize}
    \item позволяют высадиться в любой точке планеты,
    \item позволяют перемещаться по поверхности планеты только на соседние координаты (по горизонтали, вертикали и диагоналям),
    \item не позволяют осуществлять резкие спуски и подъемы (не более некоторого значения),
    \item позволяют отправить собранные образцы до ракеты из любой точки планеты.
\end{itemize}


Проведите ваши исследования наиболее эффективно! 

Ваша задача опустить модуль на поверхность планеты таким образом, чтобы можно было собрать образцы грунта с наибольшей возможной площади. Оцените наибольшую площадь, которую сможет обойти модуль, если его маршрут ограничен исключительно перепадами высот.

\inputfmtSection

В первой строке через пробел подаются три целых числа $x, y \space (1\leq x \cdot y \leq 10^7)$, $p\space (0 \leq p \leq 10^8)$ — размеры карты и предельная величина преодолимого перепада высот.

Далее в $x$ строках подаются $y$ значений, записанных через табуляцию, \linebreak $f_{ij}\space(-10^8 \leq f_{ij} \leq 10^8)$ — определенные значения высоты на поверхности платнеты~X.

\outputfmtSection

Единственное число — ответ на задачу.

\markSection

Баллы за задачу будут начисляться пропорционально количеству успешно пройденных тестов.

Результат за успешное прохождение только первого теста (методом распечатки верного ответа или иным эквивалентным образом) засчитан не будет.

\sampleTitle{1}

\begin{myverbbox}[\small]{\vinput}
    6 8 7
    -10	0	0	3	-7	-3	-9	-2
    -5	-5	6	0	-8	-10	-9	9
    5	9	-6	3	1	-8	-2	9
    -7	-4	-6	7	-2	4	9	-4
    -3	-7	4	8	5	0	6	-3
    -1	-8	-7	-2	6	-7	-5	-5
\end{myverbbox}

\begin{myverbbox}[\small]{\voutput}
    47	
\end{myverbbox}
\inputoutputTable


\includeSolutionIfExistsByPath{final/subject_tour/inf2703_dzz/task_06}