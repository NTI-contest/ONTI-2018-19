\assignementTitle{Списывание-2}{20∏}{}

Приближаются школьные экзамены. В одной из школ их проводят так.

Детей рассаживают в аудиторию, в которой $a$ рядов, в каждом из которых $b$ одноместных парт. Ряды нумеруются один за другим, как и парты в одном ряду (ряды имеют номера от $1$ до $a_1$, парты — от $1$ до $b$ в каждому ряду). 

Группа школьников решила попробовать списать на этом важном событии. Они решили, что будут передавать правильные ответы между собой, перекидываясь бумажками. Так как учителя не дремлют, то если школьник, сидящий в ряду $a_1$ за партой с номером $b_1$, кинет бумажку школьнику, сидящему в ряду $a_2$ за партой с номером $b_2$, то если $\left|a_1-a_2\right| + \left|b_1-b_2\right| > d$, то их заметят, и тогда план провалится. Так как это не первый экзамен в их жизни, многие пары уже были пойманы на списывании. За такими парами будет особый контроль, так что они не смогут перекидываться между собой. Для простоты вам будет дан список пар, которые еще не были замечены за списыванием друг у друга. Списывание удастся, если каждый сможет получить ответ от любого другого члена группы (возможно передавая через кого-либо). Смогут ли все школьники из группы всё-таки списать на экзамене?

\inputfmtSection

Первая строка содержит три целых числа $a, b, d\space (1\leq a, b\leq 1000,\space a\cdot b \leq 10^4,\space 0\leq d\leq 10^9)$.

Вторая строка содержит два целых числа $n\space(1\leq n \leq 10^5)$ — количество объединившихся для списывания человек, и $m\space\left(1\leq m \leq \min{\left(\dfrac{n(n-1)}{2}, 10^6\right)}\right)$ — количество пар, люди в которых еще не списывали друг у друга (которые могут перекидывать бумажки).

Следующие $n$ строк содержат по два целых числа: в $i$-й строке указаны $a_i, b_i\space(1\leq a_i\leq a,\space 1\leq b_i\leq b)$, соответствующие $i$-му школьнику из списывающей группы. Гарантируется, что каждая пара $\left(a_i, b_i\right)$ встречается не более $1$ раза.

Дальнейшие $m$ строк описывают пары, за которыми не будут повышенного надзора: в $i$-й строке содержатся два числа — $u_i, v_i\space (1\leq u_i, v_i \leq n, \space u_i \neq v_i)$ — номера школьников из списка выше. Гарантируется, что каждая пара школьников указывается во входных данных не более $1$ раза.

\outputfmtSection

Выведите "YES" без кавычек, если списывание удастся, иначе выведите "NO".

\explanationSection

В первом тесте каждый может перекидываться с каждым, но второй и третий находятся слишком далеко. Так как они могут передать ответы друг другу через второго, то каждый может передать каждому, ответ YES.

Во втором тесте первый находится слишком далеко от всех остальных, поэтому ответ NO.

\sampleTitle{1}

\begin{myverbbox}[\small]{\vinput}
    3 3 2
    3 3
    1 1
    1 3
    3 1
    1 2
    2 3
    1 3
\end{myverbbox}

\begin{myverbbox}[\small]{\voutput}
    YES
\end{myverbbox}
\inputoutputTable

\sampleTitle{2}

\begin{myverbbox}[\small]{\vinput}
    3 4 2
    3 3∏
    1 1
    1 4
    3 4
    1 2
    2 3
    1 3
\begin{myverbbox}[\small]{\voutput}
    NO
\end{myverbbox}
\inputoutputTable

\includeSolutionIfExistsByPath{final/subject_tour/inf1903_bigdata/task_07}