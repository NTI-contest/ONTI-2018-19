\solutionSection

Формализуем задачу: представим ее в виде графа на $n$ вершинах, где вершины соответствуют школьникам, а длина ребра - Манхэттенское расстояние между ними.

Для двух точек на плоскости считается как $d(x_1, y_1, x_2, y_2) = |x_2 - x_1| + |y_2 - y_1|$

Будем хранить граф в виде списков смежности, добавляя только те ребра, которые нам указали в условии, и длина которых меньше d. Тогда если граф связный, то ответ YES, иначе NO. Критерием связности будет достижимость всех вершин из какой-либо одной, что проверяется поиском в глубину.

\codeExample

\inputPythonSourceAdditional
%\inputminted[fontsize=\footnotesize, linenos]{python}{1st_tour/inf/try_1/task_03/source.py}