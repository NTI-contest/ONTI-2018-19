\solutionSection

Даны два натуральных числа $n$ и $m$. Надо найти число решений в натуральных числах уравнения:
$$x_1 + x_2 + \dots + x_n = m.$$

Выпишем в ряд $m$ единиц $\underbrace{111\dots 1}_m$.

Заметим, существует биекция из множества решений нашего уравнения в множество всевозможных способов поставить $n-1$ перегородку между этими единицами.

$$\underbrace{\underbrace{111\dots 1}_{x_1}|\underbrace{111\dots 1}_{x_2}|\dots|\underbrace{111\dots 1}_{x_n}}_m$$

Способов так расставить перегородки, очевидно $C_{m-1}^{n-1}$. Осталось научиться считать это число сочетаний.

Можно посчитать на Python, не забыв взять ответ по модулю $10^9+7$. Если вы попробуете написать аналогичное решение на C++, вы неизбежно столкнётесь с проблемой переполнения на больших тестах. Значит, надо брать по модулю на промежуточных шагах. $a/b = a\cdot b^{-1} = a\cdot b ^ {MOD - 2}$.

Сложность алгоритма: $O((m-n)(n\cdot\log_2{((10^9+7)-2)}))$.

\codeExample

\inputPythonSourceAdditional
%\inputminted[fontsize=\footnotesize, linenos]{python}{1st_tour/inf/try_1/task_03/source.py}