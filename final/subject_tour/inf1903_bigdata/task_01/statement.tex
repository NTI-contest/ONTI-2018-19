\assignementTitle{Подарок на 14 февраля}{10}{}

Саша и Лена живут в противоположных домах. Каждый вечер в каждом из этих домов загорается свет в некоторых окнах, в результате чего живущие напротив могут наблюдать некоторый узор из окон, в которых горит свет. Окна в домах Лены и Саши устроены стандартным образом — они образуют прямоугольную клетчатую сетку $N$ строк на $M$ столбцов.

Саша знает любимый узор Лены и хочет порадовать её. Для этого он может пойти к соседям по дому и попросить у них выключить свет на вечер. Саша хочет, чтобы в результате этого окна, оставшиеся гореть, образовывали любимый узор Лены и ничего более. 

Однако Саша подозревает, что он может сделать это не единственным образом. Ему стало интересно, сколькими способами он может попросить соседей выключить свет (обращаем ваше внимание на то, что просить соседей включать свет он не может~— тому есть свои причины), так чтобы горящие окна его дома обрадовали Лену. Порядок, в котором выключаются окна, неважен.

Поскольку вечер уже начался и ему пора начать обход соседей, Саша просит вас помочь ему с подсчётом.

\inputfmtSection

В первой строке вводятся $2$ целых числа $N, M \space (1\leq N, M \leq 10)$ — число этажей и число окон на каждом этаже в доме Саши. $i$-я из следующих $N$ строк содержит $M$ чисел ($1$ или~$0$) — горит ли соответствующее окно на $(N-i+1)$-м этаже или нет соответственно.

В $N+2$-й строке вводятся $2$ целых числа $H, W\space(1\leq H,W \leq 10)$ — высота и ширина любимого узора Лены. $i$-я из следующих $H$ строк содержит $W$ чисел ($1$~или~$0$) — цвет соответствующей ячейки в $(N-i+1)$-м ряду в любимом узоре Лены соответственно.

\outputfmtSection

Выведите в ответ одно целое число — количество способов, которыми Саша может попросить соседей выключить свет, так чтобы оставшиеся гореть окна образовывали только любимый узор Лены.

\explanationSection

В первом тесте любимый узор Лены можно нарисовать в любом месте Сашиного дома, всего таких вариантов $(N - 1) \cdot (M - 1) = 4 \cdot 5 = 20$.

\sampleTitle{1}

\begin{myverbbox}[\small]{\vinput}
    5 6
    1 1 1 1 1 1
    1 1 1 1 1 1
    1 1 1 1 1 1
    1 1 1 1 1 1
    1 1 1 1 1 1
    2 2
    1 0
    0 1
\end{myverbbox}

\begin{myverbbox}[\small]{\voutput}
    20
\end{myverbbox}
\inputoutputTable

\sampleTitle{2}

\begin{myverbbox}[\small]{\vinput}
    5 5
    1 1 0 0 0
    1 1 0 0 0
    0 0 0 0 0
    0 0 0 0 0
    0 0 0 0 0
    2 2
    1 0
    0 0
\end{myverbbox}

\begin{myverbbox}[\small]{\voutput}
    4
\end{myverbbox}
\inputoutputTable

\includeSolutionIfExistsByPath{final/subject_tour/inf1903_bigdata/task_01}