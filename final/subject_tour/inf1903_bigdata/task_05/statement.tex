\assignementTitle{Расписание переговорной-2}{20}{}

Переговорная комната "Яанровогереп" в компании "Скедня" каждый день открыта в течение $h$ часов. На завтра $n$ команд дали заявки на проведение в Яанровогерепе встреч. Для команды с номером  известна величина $t_i$ — целое количество минут, которое им необходимо для решения всех вопросов. В один момент времени две команды не могут быть в переговорной, заходят и выходят они мгновенно. Необходимо узнать, какое максимальное количество команд могут посетить переговорную в этот день.

\inputfmtSection

Первая строка содержит два целых числа $h\space(1\leq h\leq 24),\space n\space(1\leq n\leq 10^5)$ — время работы переговорки (в часах) и количество команд, желающих занять переговорку.

Вторая строка содержит $n$ целых чисел $t_i\leq(1\leq t_i\leq 24\cdot 60)$ — времена, на которые команды хотят занять переговорку (в минутах).

\outputfmtSection

Выведите максимальное число команд, которые смогут посетить переговорную в этот день

\explanationSection

В первом тесте переговорная будет свободна $2\cdot 60=120$ минут, что является верхней границей суммарного времени групп в комнате. Рассматривая все возможные случаи, получаем, что если мы запустим первую группу, то больше четырех групп мы запустить не сможем. Тогда рассматриваем только группы со $2$-й по $6$-ю, суммарное время которых равно $120$, то есть мы сможем их запустить, получив пять групп, что и будет ответом.

Во втором тесте единственная группа хочет занять комнату больше чем на то время, которое она будет открыта, поэтому никто не сможет зайти в нее.

В третьем тесте суммарное время, которое нужно всем группам равно 4 минутам, то есть мы сможем запустить все 4 группы.

\sampleTitle{1}

\begin{myverbbox}[\small]{\vinput}
    2 6
    70 20 30 40 10 20
\end{myverbbox}

\begin{myverbbox}[\small]{\voutput}
    5
\end{myverbbox}
\inputoutputTable

\sampleTitle{2}

\begin{myverbbox}[\small]{\vinput}
    1 1
    61
\begin{myverbbox}[\small]{\voutput}
    0
\end{myverbbox}
\inputoutputTable

\sampleTitle{3}

\begin{myverbbox}[\small]{\vinput}
    3 4
    1 1 1 1
\end{myverbbox}

\begin{myverbbox}[\small]{\voutput}
    4
\end{myverbbox}
\inputoutputTable

\includeSolutionIfExistsByPath{final/subject_tour/inf1903_bigdata/task_05}