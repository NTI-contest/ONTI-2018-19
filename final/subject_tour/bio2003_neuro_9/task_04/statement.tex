\assignementTitle{}{10}{4}

Гипоталамус — отдел промежуточного мозга, который отвечает за регулирование гомеостатических параметров, таких как температура тела и кислотность крови, а также необходим для формирования памяти и различных поведенческих реакций.

Рассмотрите поведение кошки после электрической стимуляции различных зон гипоталамуса — передней или латеральной.

Кошка до стимуляции (животное с электродом помечено звездочкой, справа — контрольное животное, которое необходимо для сравнения):

\putImgWOCaption{12cm}{1}

Кошка после стимуляции передней части гипоталамуса:

\putImgWOCaption{12cm}{2}
\putImgWOCaption{12cm}{3}

Кошка после стимуляции латеральной части гипоталамуса:

\putImgWOCaption{12cm}{4}
\putImgWOCaption{12cm}{5}
\putImgWOCaption{12cm}{6}

Какие утверждения можно считать верными, исходя из результатов данного эксперимента:

\begin{enumerate}
    \item стимуляция латерального гипоталамуса вызывает истинно агрессивное поведение
    \item стимуляция переднего гипоталамуса вызывает истинно агрессивное поведение
    \item результаты данного эксперимента позволяют заключить, что гипоталамус является единственной структурой, отвечающей за развитие агрессивного поведения
    \item после стимуляции латеральной части гипоталамуса животное демонстрирует исследовательское поведение
    \item результаты данного эксперимента позволяют заключить, что единственной функцией гипоталамуса является формирование агрессивного поведения
\end{enumerate}