\solutionSection
\begin{enumerate}
\item При росте карбонатного скелета происходит следующая реакция:
$$H_2O+CO_2+CaCO_3 \leftrightarrow Ca(HCO_3)_2.$$ Соответственно расти он может, только когда 
равновесие сдвинуто влево, в сторону осаждения $CaCO_3$. Чем холоднее вода, тем лучше в ней растворяются газы, 
в том числе и $CO_2$, в результате чего равновесие сдвигается вправо, в сторону растворения. Приблизительно 
при температуре 20$^{\circ}$С равновесие сдвигается в сторону растворения настолько, что скелет не 
нарастает. \textit{(10~баллов)}
\item В их клетках живут симбиотические простейшие (водоросли, диномонады~– тоже правильно), способные 
фотосинтезировать и имеющие зеленый цвет. \textit{(4~балла)} Кораллы получают часть продуктов фотосинтеза (которыми водоросли «оплачивают») свое проживание \textit{(3 балла)}. Водоросли изымают из тканей коралла часть $CO_2$, в результате чего равновесие, описанное в вопросе (пункте 1) смещается в сторону нерастворимого карбоната $Са$ и коралл может расти. \textit{(3 балла)}.
\end{enumerate}