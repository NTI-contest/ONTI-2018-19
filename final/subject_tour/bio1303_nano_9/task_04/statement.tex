\assignementTitle{Шлем и иглы}{20}{4}

У многих видов планктонных рачков дафний ярко выражен цикломорфоз. Так называют сезонные различия во внешнем строении 
разных поколений одного вида. Зимой и весной дафнии крупнее, голова у них круглая, глаз большой, а хвостовая игла 
(вырост на заднем конце панциря, покрывающего тело) часто довольно короткая. К началу или к середине лета размеры 
тела (длина створок панциря и масса) и диаметр глаза уменьшаются, длина хвостовой иглы увеличивается, а на голове 
появляется длинный вырост — шлем. 

\putImgWOCaption{6cm}{1}

\begin{centering}
    Рис. 1. Летняя самка Daphnia retrocurva.
\end{centering}

У одних видов он широкий (рис. 1), у других более узкий (рис. 2, верхний ряд), 
иногда похожий на иглу. (Эти признаки появляются у новых поколений — детей и внуков «весенних» дафний!) 
У осенних и зимних поколений (если дафнии зимуют в активном состоянии, а не в виде покоящихся яиц) 
шлем обычно уменьшается или исчезает.

\putImgWOCaption{12cm}{2}

\begin{centering}
    Рис. 2. Цикломорфоз у двух видов дафний — D. cucullata в одном из озер Дании и D. retrocurva в одном из озер США.
\end{centering}

\begin{enumerate}
    \item Как вы думаете, каков приспособительный смысл этих изменений?
    \item Почему шлем и длинная хвостовая игла не вырастают весной и зимой?
\end{enumerate}