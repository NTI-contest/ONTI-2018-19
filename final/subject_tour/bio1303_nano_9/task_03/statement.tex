\assignementTitle{Японский флот}{20}{3}

Во время Реставрации Мейдзи (передача власти в Японии от сегуна к императору, сопровождалась «открытием» страны для иностранцев) 
и последовавших за ней преобразований, был реорганизован, а фактически создан заново по английским и американским образцам 
военный флот Японии. Поскольку на него возлагали большие надежды, людей, служивших на нем считали практически героями, и 
выделяли для них все лучшее, в том числе и пищу – белый шлифованный рис и морскую рыбу. Вскоре моряков поразила странная 
болезнь, которой при этом не заражались люди в портах. 

Победить ее удалось только после того, как была полностью скопирована организация быта во флоте Англии, 
в том числе и рацион – солонина (мясо), ржаные сухари и темный эль.

\begin{enumerate}
    \item Чем была вызвана болезнь и почему она возникла
    \item Опишите ее предположительные симптомы и напишите название.
\end{enumerate}
