\assignementTitle{Окрас собак}{20}{5}

Локус Агути (А), отвечающий за распределение окраса по телу у собак, определяет следующие окрасы:

\begin{enumerate}
    \item Соболиный/олений (обозначается буквой у)
    \putImgWOCaption{7.5cm}{1}
    \item Чепрачный (обозначается буквой s) 
    \putImgWOCaption{7.5cm}{2}
    \item Подпалый (обозначается буквой t) 
    \putImgWOCaption{6cm}{3}
    \item Агути/зонарный (обозначается буквой w) \newline
    \putImgWOCaption{7cm}{4}
    \item Черный
    \putImgWOCaption{7cm}{5}
\end{enumerate}

При скрещивании двух собак чепрачного окраса (Пара А) $\frac{3}{4}$ потомков имели чепрачный окрас и $\frac{1}{4}$ зонарный. При скрещивании этих 
зонарных потомков между собой $\frac{3}{4}$ их потомков были зонарными и $\frac{1}{4}$ подпалыми. При скрещивании любых из этих потомков с черными собаками черные потомки не рождались никогда.

При скрещивании собак соболиного окраса и чепрачного окраса (Пара В) $\frac{1}{2}$ потомков имели соболиный и $\frac{1}{2}$ - чепрачный окрас. 
При скрещивании некоторых потомков соболиного и чепрачного окраса между собой в потомстве так же наблюдалась половина 
чепрачных и половина соболиных собак. При скрещивании других соболиного и чепрачного потомка пары В между собой $\frac{1}{2}$ 
потомков оказались чепрачными, $\frac{1}{4}$  - соболиными, а остальные – черными. 

При скрещивании соболиной собаки из пары В и зонарного потомка пары А $\frac{1}{2}$ потомков оказались соболиными, $\frac{1}{4}$ - зонарными и $\frac{1}{4}$ - подпалыми.

Установите отношения (порядок доминирования) между аллелями локуса А, а также генотип Пары А и Пары В.

