\solutionSection
Обозначим окрасы следующим образом:\\
$(A)y$ – соболиный\\
$(A)s$ – чепрачный\\
$(A)t$ – подпалый\\
$(A)w$ – зонарный\\
$(A)$ – черный\\
Пара А:\\
Р:$\hspace{150pt}$чепрачный $\times$ чепрачный
$$(A)?(A)?\times(A)?(A)?$$
F1$\hspace{137pt}\frac{3}{4}$ чепрачный $+$ $\frac{1}{4}$ зонарный
$$(A)?(A)?\hspace{34pt}(A)?(A)?$$
У чепрачных собак должна быть хотя бы  1 аллель, отвечающая за соответствующий окрас. Т.е. Родители и чепрачные потомки $(A)s(A)$? Расщепление в  1 поколении в соотношении 3к1 говорит о том, что оба родителя гетерозиготны, хотя бы один из них должен нести аллель зонарного окраса, и зонарный окрас рецессивен по отношению к чепрачному. Тогда один из пары А имеет генотип $(A)s(A)w$, зонарные потомки – $(A)$ $w$ $(A)$?\\
Р$\hspace{60pt}$зонарный (потомок пары А) $\times$ зонарный (потомок пары А) 
$$(A)\hspace{3pt}w\hspace{3pt}(A)?\hspace{34pt}(A)\hspace{3pt}w\hspace{3pt}(A)?$$
F1$\hspace{147pt}\frac{3}{4}$ зонарный $+$ $\frac{1}{4}$ подпалый\\
Рассуждения аналогичны рассуждениям для пары А. Оба зонарных потомка пары А гетерозиготны и  должны нести аллель подпалого окраса, который рецессивен относительно зонарного. Соответственно второй родитель из пары А должен был так же нести аллель подпалого окраса.\\
Таким образом генотип пары A$: Asat \times AsAw$ $\textit{(7 баллов)}$\\
(Тогда чепрачные потомки имеют генотипы $AsAs + Asat + AsAw$, зонарные – $Awat$, зонарные внуки пары $A\hspace{3pt}AwAw +2 Awat$, подпалые внуки $atat$).\\
Известно что при скрещивании всех упомянутых выше собак с черными черные потомки не рождались. Это может быть в том случае – если черный окрас рецессивен по отношению ко всем рассмотренным.\\
Пара В:\\
Р$\hspace{150pt}$соболиный $\times$ чепрачный
$$(A)y(A)?\hspace{34pt}(A)s(A)?$$
F1$\hspace{137pt}\frac{1}{2}$ соболиный $\times$ $\frac{1}{2}$чепрачный
$$(A)y(A)?\hspace{34pt}(A)s(A)?$$
\begin{enumerate}
\item F1$\hspace{125pt}$соболиный $\times$ чепрачный
$$(A)y(A)?\hspace{34pt}(A)s(A)?$$
F2$\hspace{120pt}\frac{1}{2}$ соболиный $+$ $\frac{1}{2}$ чепрачный
$$(A)y(A)?\hspace{34pt}(A)s(A)?$$
\item F1$\hspace{125pt}$соболиный $\times$ чепрачный
$$(A)y(A)?\hspace{34pt}(A)s(A)?$$
F2$\hspace{90pt}\frac{1}{2}$ чепрачный $+$ $\frac{1}{4}$ соболиный $+$ $\frac{1}{4}$ черный\\
\end{enumerate}
Из предидущего пункта условия нам известно, что черный окрас – рецессивный, по крайней мере по отношению к чепрачному. Исходя из того, что он проявился во втором поколении потомков пары В, один из родителей должен нести аллель этого окраса. При этом, поскольку от пары В не рождалось черных собак, можно предположить, что черный окрас рецессивен по отношению так же и к соболиному, и является самой рецессивной аллелью локуса А. Тогда черные собаки – рецессивные гомозиготы с генотипом $aa$\\
Поскольку разные пары потомков пары В дают разное расщепление по окрасам при скрещивании, их генотипы не однородны внутри фенотипа. Рассмотрим скрещивание 2). Черные потомки имеют генотип аа, который мог возникнуть, только если оба их родителя несут аллель а. Тогда:\\
\begin{enumerate}
\item [2.] F1$\hspace{125pt}$соболиный $\times$ чепрачный
$$(A)ya?\hspace{34pt}(A)sa?$$
F2$\hspace{127pt}AsAy+Asa+Aya + aa$\\
\end{enumerate}
Собаки имеющие генотип $aa$ – черные, $Aya$ – соболиные, $Asa$ – чепрачные. Чтобы получить наблюдаемое в потомстве расщепление по фенотипу, собаки $AsAy$ должны быть чепрачными. Таким образом $As\textgreater Ay$.\\
Вернемся к паре В\\
Р$\hspace{150pt}$соболиный $\times$ чепрачный
$$(A)y(A)?\hspace{34pt}(A)s(A)?$$
F1$\hspace{137pt}\frac{1}{2}$ соболиный $+$ $\frac{1}{2}$ чепрачный
$$(A)y(A)?(\text{но не }As)+Aya+\hspace{10pt}(A)s(A)?+Asa\hspace{47pt}$$
Получается в потомстве мы имеем генотипы $Aya$ и $Asa$, но аллель а несет только 1 родитель, иначе получались бы черные собаки. Для этого тот родитель, у которого нет а должен нести и $As$ и $Ay$, и такая собака будет чепрачной.\\
Таким образом генотип пары В$:Aya\times AsAy$ $\textit{(7 баллов)}$\\
Отношения между соболиным, зонарным и подпалым окрасом:\\
Р$\hspace{113pt}$соболиный (из В) $\times$ зонарный (F1 от А)
$$Aya\times Awat$$
F1$\hspace{90pt}\frac{1}{2}$ соболиный $+$ $\frac{1}{4}$ зонарный $+$ $\frac{1}{4}$ подпалый\\
Генотипы потомков от этого скрещивания:\\
$AyAw+Ayat+Awa+ata$. Как известно $a$ – рецессивен по отношению ко всем остальным аллелям, значит $Awa$ – зонарные, $ata$ – подпалые. Тогда для получения расщепления по фенотипу, наблюдавшегося в потомстве $AyAw+Ayat$ – должны быть соболиного окраса. Таким образом соболиный окрас доминантен по отношению к зонарному и подпалому, и соответственно порядок доминирования в локусе A$:As\textgreater Ay\textgreater Aw\textgreater at\textgreater a$, все доминирование полное. $\textit{(6 баллов)}$\\
\answerMath{\\Пара А$: Asat \times AsAw$ $\textit{(7 баллов)}$\\
Пара В$:Aya\times AsAy$ $\textit{(7 баллов)}$\\
Порядок доминирования в локусе А$:As\textgreater Ay\textgreater Aw\textgreater at\textgreater a$, все доминирование полное. $\textit{(6 баллов)}$}
