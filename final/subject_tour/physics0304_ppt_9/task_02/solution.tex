\solutionSection

\begin{enumerate}
    \item Свет проходит через волновод по сердцевине, значит для расчёта скорости света, проходящего через этот волновод, надо учитывать только показатель преломления сердцевины.  Следовательно, скорость света уменьшится в 1.01 раза, а значит идти он будет в 1.01 раза дольше, то есть на 1\%.
    
    \answerMath{1\%.}

    \markSection

    \begin{itemize}
        \item Учтено, что скорость света в среде связана с показателем преломления – 5 баллов
        \item Получен правильный численный ответ – 5 баллов
    \end{itemize}
    
    \item Наиболее быстро пройдёт луч, который идёт по прямой. Медленнее всего будет идти луч, который идёт под углом полного внутреннего отражения к оболочке, так как он будет преодолевать больший путь. Изобразим это на рисунке:
		
		\putImgWOCaption{7cm}{1}

		Из закона преломления следует, что предельный угол $\alpha$ определяется так:
		$$sin\alpha=\frac{1.474}{1.479}$$
		
		Длина ломаной больше в $1/(sin\alpha)$ раз. Следовательно, если луч по прямой идёт расстояние $L$, то он проходит его за время: $t_1= \frac{L \cdot 1.474}{c}$ . Если же он идёт по максимально длинному пути, то время прохождения равно: $t_2= \frac{L \cdot 1.479}{c}$.

		Чтобы импульсы не перекрывались, нужно, чтобы $\Delta t=t_2- t_1 \leq T$.
		$$\frac{L \cdot 0.005}{c} \leq T; \: L \leq \frac{c \cdot T}{0.005}=\frac{3 \cdot 10^8 \cdot 10^{-5}}{5 \cdot 10^{-3} \: \text{м}} =600 \: \text{км}$$

		\answerMath{$L= \frac{c \cdot T}{0.005}=600$ км.}

		\markSection

		\begin{itemize}
			\item Указано для каких лучей будет максимальное и минимальное значение времени – 2 балла
			\item Найден критический угол – 2 балла
			\item Найдена длина ломанной – 2 балла
			\item Найдено условие на длину оптоволоконного кабеля – 2 балла
			\item Получено правильно числовое значение – 2 балла
		\end{itemize}

		\item Минимальный радиус может быть найден из условия, что угол, под которым падает луч из сердцевины на оболочку, не может быть меньше угла полного внутреннего отражения. Изобразим это на рисунке:
		
		\putTwoImg{6cm}{2}{5cm}{3}

		Построим треугольник, вершинами которого являются центр окружности и два последовательных отражения от внутреннего и внешнего круга. Внешний угол этого треугольника не может быть больше 90 градусов (случай касательный), а угол к нормали внешней окружности не может быть меньше $\alpha$. Используя теорему синусов, запишем: 
		$$\frac{R+d/2}{1}=\frac{R-d/2}{sin\alpha}$$  
		
		Отсюда находим предельное значение радиуса:

		$$R_{min}=\frac{d}{2} \cdot \frac{1+sin\alpha}{1-sin\alpha}=\frac{9 \: \text{мкм}}{2} \cdot \frac{\frac{1+1.474}{1.479}}{\frac{1- 1.474}{1.479}}=4.5 \: \text{мкм} \cdot \frac{2.953}{0.005}=2657.7 \: \text{мкм}$$
		
		\answerMath{$R_{min}=2657.7$ мкм.}
		
		\markSection

		\begin{itemize} 
			\item Указано ограничение на минимальный радиус – 4 балла
			\item Приведены геометрические соображения для нахождения радиуса – 2 балла
			\item Получено выражение для минимального радиуса – 2 балла
			\item Получен правильной числовой ответ – 2 балла
		\end{itemize}

		\item Мощность сигнала пропорциональна квадрату амплитуды, значит в первом случае её уменьшение можно считать как умножение на $0.9743^2 = 0.94926$ на каждый километр. Возведём в десятую степень: $0.94926^{10} = 0.59409$, то есть КПД передачи 59.409\%. Посчитаем аналогично для второго случая: $0.9791^{20} = 0.65545$, то есть 65.545\%. Посчитаем отношение: $\gamma=\frac{65.545}{59.409} \approx 1.103$.
		
		\answerMath{КПД возрастёт в 1.103 раза.}

		\markSection

		\begin{itemize}
			\item Указана связь между амплитудой колебаний и мощности – 4 балла
			\item Сделана оценка затухания – 2 балла
			\item Сделан верный расчет затухания (экспоненциально падает) – 2 балла
			\item Получен правильной числовой ответ (в диапазоне) – 2 балла
		\end{itemize}

		\item Этот пункт аналогичен пункту 1, только вместо длины ломаной надо оценивать длину синусоиды. Так как граничные значения показателя преломления те же самые, максимальный наклон синусоиды определяется углом полного внутреннего отражения:
		
		\putImgWOCaption{7cm}{4}

		У функции $sin(x)$ максимальный наклон равен 1, следовательно, у функции $sin(x/k)$ он равен $1/k$, так как этот график получается растяжение в $k$ раз вдоль горизонтальной оси.
		
		Предельный наклон будет равен $$ctg(\alpha)=\sqrt{\frac{1}{sin^2\alpha}-1}=\sqrt{\left(\frac{1.479}{1.474}\right)^2-1}=0.0824364.$$

		Таким образом, надо оценить длину синусоиды $y=sin(0.0824364 \cdot x)$. Рассмотрим малый участок на графике $y=sin(0.0824364 \cdot x)$ от точки $x$ до точки $x+ \Delta x$. Заменим его прямым отрезком. Длина этого отрезка равна:

		$$\Delta L=\sqrt{\Delta x^2+(sin(0.0824364 \cdot (x+\Delta x))-sin(0.0824364 \cdot x) )^2}$$ 
		
		Считая, что  $\Delta x << 1$, получаем, что: $$\Delta L \approx \Delta x \cdot \sqrt{(1+0.0824364^2 \cdot cos^2(0.0824364 \cdot x))}$$
		$$\Delta L \approx \Delta x \cdot (1+0.5 \cdot 0.0824364^2 \cdot cos^2(0.0824364 \cdot x) )$$
		$$\Delta L \approx \Delta x \cdot (1+0.25 \cdot 0.0824364^2 \cdot (1+cos(0.5 \cdot 0.0824364 \cdot x)))$$

		При движении луча на расстояния, много большие периода этого косинуса, вклад периодического слагаемого равен нулю. Тогда получаем приближённую формулу:
		$$\Delta L \approx \Delta x \cdot (1+0.25 \cdot 0.0824364^2 )=\Delta x \cdot 1.0017$$

		Следовательно, если луч по прямой идёт расстояние $L$, то он проходит его за время: $t_1=\frac{L \cdot 1.474}{c}$. Если же он идёт по длинному пути, то время прохождения равно: $t_2= t_1 \cdot 1.0017$. Отсюда $0.0017 \cdot \frac{L \cdot 1.474}{c} \leq T;$ $$L \leq \frac{c \cdot T}{0.0017 \cdot 1.474} \approx \frac{3 \cdot 10^8 \cdot 10^{-5}}{0.0017 \cdot 1.474}=1197 \: \text{км}.$$

		\answerMath{$L \leq 1197$ км. В качестве правильной подойдёт любая оценка, близкая к 1200 км. Точный расчёт показывает, что получается примерно 1201 км.}

		\markSection

		\begin{itemize}
			\item Приведены соображения о том, какие лучи дадут максимальное и минимальное время – 2 балла
			\item Сделан вывод о необходимости оценки длины синусоиды – 2 балла
			\item Сделана оценка о длине синусоиды – 3 - балла
			\item Получен правильный числовой ответ  - 3 балла.
		\end{itemize}
\end{enumerate}