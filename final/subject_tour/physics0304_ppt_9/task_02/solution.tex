\solutionSection

\begin{tabular}{l|}
    Дано: \\
    $d = 40$ см \\
    $v = 1$ м/с \\
    $S = 1$ км \\
    \hline \\
    $N_n$, $a_\text{цс}$ - ?
\end{tabular}

Неподвижная система координат - Земля, подвижная - центр сферы.

Нижняя точка сферы в данный момент времени неподвижна относительно Земли.

\putImgWOCaption{6cm}{1}

По закону сложения скоростей скорость нижней точки сферы 
относительно Земли равна скорости этой точки относительно 
центра сферы (скорости вращения) плюс скорость центра сферы 
относительно Земли.

Проектируем на ось x:

$$v_\text{нижняя относительно Земли} = 0 = -v_\text{вращения} + v \Rightarrow v_\text{вращения}=v$$
$$a_\text{цс} = \frac{v_\text{вращения}^2}{r} = \frac{v^2}{d/2}=\frac{1^2}{0.2} = 5 \text{м/с}^2$$
$$N=\frac{S}{2 \cdot \pi \cdot r} = \frac{1000}{2 \cdot 3.14 \cdot 0.2} \approx 796.178 \Rightarrow N_n = 796 \: \text{оборотов}$$

Время движения робота без пробуксовки $t = \frac{S}{v}$ 

Время, затрачиваемое роботом на один полный оборот, $t_1 = \frac{t}{N} = \frac{S}{v \cdot N}$

Тогда время движения робота с пробуксовкой $t_\text{п} = t_1  \cdot  4N = \frac{S}{v \cdot N} \cdot 4N = 4 \cdot S/v$

Скорость робота при пробуксовке $v_\text{п} = \frac{S}{t_\text{п}} = \frac{Sv}{4S} = v/4 = 0.25$ м/с

Изменение скорости $\Delta v = 1 - 0.25 = 0.75$ м/с

\answerMath{$N_\text{п} = 796$ оборотов; $a_\text{цс} = 5$ м/с$^2$; скорость робота в результате пробуксовки 
уменьшится на 0.75 м/с.}
