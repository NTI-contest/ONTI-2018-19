\solutionSection

\begin{enumerate}
    \item Свет проходит через волновод по сердцевине, значит для расчёта скорости света, проходящего через этот волновод, надо учитывать только показатель преломления сердцевины.  Следовательно, скорость света уменьшится в 1.01 раза, а значит идти он будет в 1.01 раза дольше, то есть на 1\%.
    
    \answerMath{1\%.}

Критерии:
	Учтено, что скорость света в среде связана с показателем преломления – 5 баллов
	Получен правильный численный ответ – 5 баллов

	Наиболее быстро пройдёт луч, который идёт по прямой. Медленнее всего будет идти луч, который идёт под углом полного внутреннего отражения к оболочке, так как он будет преодолевать больший путь. Изобразим это на рисунке:
 
Из закона преломления следует, что предельный угол α определяется так:
sin⁡α=1.474/1.479
Длина ломаной больше в 1/(sin⁡α ) раз. Следовательно, если луч по прямой идёт расстояние L, то он проходит его за время:   t_1=  (L∙1.474)/c . Если же он идёт по максимально длинному пути, то время прохождения равно:  t_2=  (L∙1.479)/c .
Чтобы импульсы не перекрывались, нужно, чтобы ∆t=t_2- t_1≤T .
(L∙0.005)/c≤T;  L ≤  (c∙T)/0.005=(3∙〖10〗^8∙〖10〗^(-5))/(5∙〖10〗^(-3) ) м =600 км

Ответ: L=  (c∙T)/0.005=600 км

Критерии: 
	Указано для каких лучей будет максимальное и минимальное значение времени – 2 балла
	Найден критический угол – 2 балла
	Найдена длина ломанной – 2 балла
	Найдено условие на длину оптоволоконного кабеля – 2 балла
	Получено правильно числовое значение – 2 балла

	Минимальный радиус может быть найден из условия, что угол, под которым падает луч из сердцевины на оболочку, не может быть меньше угла полного внутреннего отражения. Изобразим это на рисунке:
  
Построим треугольник, вершинами которого являются центр окружности и два последовательных отражения от внутреннего и внешнего круга. Внешний угол этого треугольника не может быть больше 90 градусов (случай касательный) , а угол к нормали внешней окружности не может быть меньше α. Используя теорему синусов, запишем: 
(R+d/2)/1=(R-d/2)/sin⁡α  
Отсюда находим предельное значение радиуса:
R_min=d/2∙(1+sin⁡α)/(1-sin⁡α )=(9 мкм)/2∙(1+1.474/1.479)/(1- 1.474/1.479)=4.5 мкм ∙2.953/0.005=2657.7 мкм 
Ответ: R_min=2657.7 мкм 

Критерии: 
	Указано ограничение на минимальный радиус – 4 балла
	Приведены геометрические соображения для нахождения радиуса – 2 балла
	Получено выражение для минимального радиуса – 2 балла
	Получен правильной числовой ответ – 2 балла

	 Мощность сигнала пропорциональна квадрату амплитуды, значит в первом случае её уменьшение можно считать как умножение на 0.97432 = 0.94926 на каждый километр.
Возведём в десятую степень:  0.9492610 = 0.59409, то есть КПД передачи 59.409 %.
Посчитаем аналогично для второго случая:  0.979120 = 0.65545, то есть 65.545 %.
Посчитаем отношение:  γ=65.545/59.409≈1.103 .
Ответ: КПД возрастёт в 1.103 раза. 

Критерии: 
	Указана связь между амплитудой колебаний и мощности – 4 балла
	Сделана оценка затухания – 2 балла
	Сделан верный расчет затухания (экспоненциально падает) – 2 балла
	Получен правильной числовой ответ (в диапазоне) – 2 балла

 
	Этот пункт аналогичен пункту 1, только вместо длины ломаной надо оценивать длину синусоиды. Так как граничные значения показателя преломления те же самые, максимальный наклон синусоиды определяется углом полного внутреннего отражения:
 
У функции sin(x) максимальный наклон равен 1, следовательно, у функции sin(x/k) он равен 1/k, так как этот график получается растяжение в k раз вдоль горизонтальной оси.
Предельный наклон будет равен  ctg(α)=√(1/sin^2⁡α -1)=√((1.479/1.474)^2-1)=0.0824364 .
Таким образом, надо оценить длину синусоиды y=sin⁡(0.0824364∙x). Рассмотрим малый участок на графике y=sin⁡(0.0824364∙x) от точки х до точки х+Δx. Заменим его прямым отрезком. Длина этого отрезка равна:
∆L=√(〖∆x〗^2+(sin⁡(0.0824364∙(x+∆x))-sin⁡(0.0824364∙x) )^2 )  
Считая, что  ∆x << 1, получаем, что:   ∆L≈∆x∙√(1+〖0.0824364〗^2∙cos^2⁡(0.0824364∙x) ) 
∆L≈∆x∙(1+0.5∙〖0.0824364〗^2∙cos^2⁡(0.0824364∙x) )
∆L≈∆x∙(1+0.25∙〖0.0824364〗^2∙(1+cos⁡(0.5∙0.0824364∙x) ))
  При движении луча на расстояния, много большие периода этого косинуса, вклад периодического слагаемого равен нулю. Тогда получаем приближённую формулу:
∆L≈∆x∙(1+0.25∙〖0.0824364〗^2 )=∆x∙1.0017
Следовательно, если луч по прямой идёт расстояние L, то он проходит его за время:   t_1=  (L∙1.474)/c . Если же он идёт по длинному пути, то время прохождения равно:  t_2= t_1∙1.0017 .  Отсюда   0.0017∙(L∙1.474)/c≤T;L≤(c∙T)/(0.0017∙1.474)≈(3∙〖10〗^8∙〖10〗^(-5))/(0.0017∙1.474)=1197 км. 
Ответ: L≤1197 км . В качестве правильной подойдёт любая оценка, близкая к 1200 км. Точный расчёт показывает, что получается примерно 1201 км.

Критерии: 
	Приведены соображения о том, какие лучи дадут максимальное и минимальное время – 2 балла
	Сделан вывод о необходимости оценки длины синусоиды – 2 балла
	Сделана оценка о длине синусоиды – 3 - балла
	Получен правильный числовой ответ  - 3 балла.
