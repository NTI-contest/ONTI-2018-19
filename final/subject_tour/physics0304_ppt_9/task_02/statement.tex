\assignementTitle{Идеальный светово}{}{2}

\textbf{Оптическое волокно} — нить из оптически прозрачного материала, используемая для переноса света 
внутри себя посредством полного внутреннего отражения. Для обеспечения полного внутреннего отражения абсолютный 
показатель преломления сердцевины несколько выше показателя преломления оболочки. Сердцевина изготавливается из 
чистого материала (стекла или пластика) и имеет диаметр 9 мкм. Оболочка имеет диаметр 125 мкм и состоит из 
материала с легирующими добавками, изменяющими показатель преломления. Показатель преломления оболочки равен 1.474, 
показатель преломления сердцевины — 1.479. Луч света, направленный в сердцевину, будет распространяться по ней, 
многократно отражаясь от оболочки. Для эффективной волоконно-оптической связи сигналы передают импульсами света 
продолжительностью $\tau = 1$ мкс, с интервалом между импульсами $T = 10$~мкс.\\
Скорость света с принять равной 300 000 км/сек.
\begin{enumerate}
\item Показатель преломления зависит не только от свойств вещества, но и от частоты света. По данному волноводу пустили свет другой частоты. Для света с этой частотой оба показателя преломления больше в 1.01 раза. На сколько процентов время распространения света с такой частотой больше, чем у основного, для которого показатели преломления указаны в условии? Дайте ответ с точностью до сотых процента.
\item Пусть данный оптоволоконный кабель идеально прямой. Какой максимальной длины он может быть, чтобы соседние импульсы света не перекрывались? Задержкой света при отражении от оболочки пренебрегите. 
\item Оптоволоконный кабель, описанный в условии, проложили по окружности. Каким должен быть минимальный радиус этой окружности, чтобы свет мог по этому кабелю пройти? В расчётах считать, что толщина кабеля во много раз меньше этого радиуса.
\item У современных образцов оптоволокна амплитуда сигнала уменьшается на  2.57\% при прохождении 1 км (для длины волны 1.55 мкм). Развитие технологии производства волокна может снизить потери до 2.09\% при прохождении 1 км.  Во сколько раз при этом увеличится КПД передачи сигнала на расстояние 10 км?\\
Для электромагнитной волны амплитуда и мощность сигнала связаны также, как и амплитуда и энергия обычных механических колебаний. 
\item В градиентных волокнах показатель преломления сердцевины плавно возрастает от края к центру. Пусть 
градиентное волокно ($n_{max} = 1.479, n_{min} = 1.474$) устроено таким образом, что луч, идущий под 
максимальным допустимым углом к направлению волокна, при котором он распространяется дальше за счёт полного 
внутреннего отражения, двигается по синусоиде.  Кабель считайте идеально прямым. Оцените, какой максимальной 
длины этот кабель может быть, чтобы соседние импульсы света не перекрывались по длительности. В расчётах можно 
считать, что косинус угла полного внутреннего отражения $\ll 1$.

\textit{Примечание:} $\sin(x+y)=\sin{x}\cdot\cos{y}+\cos{x}\cdot\sin{y}, \cos(2x)=2\cdot\cos^2x-1$.
Для малых x$: \sin(x)\approx x,   (1+x)^n\approx1+n\cdot x$.

\end{enumerate}