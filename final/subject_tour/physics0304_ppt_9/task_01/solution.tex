\solutionSection

\begin{enumerate}
    \item Если сила прижатия к рельсу сразу принимает максимальное значение, то торможение происходит под действием постоянной силы трения. Это значит, что оно равноускоренное. Для равноускоренного движения верно: $v^2=2 \cdot a \cdot s$.
    Отсюда следует, что тормозной путь увеличится в 9 раз. 
    \answerMath{увеличится в 9 раз.}

    \markSection

    \begin{itemize}
        \item Приведен закон сохранения энергии или учтено, что движение будет равноускоренным – 3 балла
        \item Получено выражение для тормозного пути – 4 балла
        \item Получен правильный числовой результат – 3 балла
    \end{itemize}

    \item Если коэффициент торможения равен 210\%, то сила трения увеличится в 2.1 раза, а значит, ускорение торможения будет в 2.1 раза больше: $v^2=2 \cdot a \cdot s$.
    Это значит, что длина тормозного пути уменьшится в 2.1 раза, а значит будет составлять $\frac{100 \%}{2.1}=47.619\%$ от исходного. Следовательно, уменьшится на 52.381\%.
    \answerMath{уменьшится на 52.381\%.}

    \markSection

    \begin{itemize}
	    \item Рассчитан новый тормозной путь – 3 балла
	    \item Рассчитано изменение тормозного пути – 3 балла
	    \item Получен правильны й числовой ответ – 4 балла
    \end{itemize}

    \item Если полная сила 100 кН, то сила давления рельса на башмак без этого тормоза равна $\frac{100}{1.4} = 71 429$ Н.  Это значит, что магнитная сила прижатия равна 28 571 Н.
    
    \answerMath{28571 Н.} 

    \markSection

    \begin{itemize}
	    \item Рассчитана сила давления рельса на башмак без тормоза – 3 балла
	    \item Рассчитана магнитная сила прижатия – 4 балла
	    \item Получен правильный числовой ответ – 3 балла
    \end{itemize}

    \item Пусть коэффициент трения тормозной колодки равен 0.4, коэффициент торможения магниторельсового поезда 160\%, а магнитное поле прижимает с силой 70 кН. Какой массы должен быть поезд, чтобы его тормозной путь при движении со скоростью 120 км/ч был не больше 700 метров?
    
    Если магнитное поле прижимает с силой 70 кН, то можно найти полную силу прижатия $N$ рельса к башмаку из соотношения: $$\frac{N}{N-70}=1.6 \rightarrow N=\frac{70 \cdot 1.6}{0.6}= 186.67 \: \text{кН}.$$
    
    Тогда сила трения равна: $F=0.4 \cdot 186.67=74.67$ кН. Отсюда ускорение торможения равно: $a=\frac{F}{m}=\frac{74.67}{m}$.  Тормозной путь не больше 700 метров: $s=\frac{v^2}{2 \cdot a}=\frac{m \cdot v^2}{2 \cdot 74.67} \leq 700$.
    $$m \leq \frac{2 \cdot 74.67 \cdot 700}{\left(\frac{120}{3.6}\right)^2} =94.082 \: \text{тонн}.$$

    \answerMath{масса поезда должна не превышать 94.082 тонны.}

    \markSection

    \begin{itemize}
	    \item Найдена полная сила прижатия рельса к башмаку – 2 балла
	    \item Записан второй закон Ньютона – 2 балла
	    \item Найдена масса поезда -4 балла
	    \item Получен правильный числовой ответ – 2 балла
    \end{itemize}

    \item Тормозная система плавного торможения устроена так, что сила торможения увеличивается пропорционально уже пройденному тормозному пути, с постоянным коэффициентом пропорциональности. Во сколько раз увеличится тормозной путь, если скорость перед началом торможения будет больше в 2 раза?
    
    Запишем это в виде второго закона Ньютона в проекции на ось движения:
    $$m \cdot a=-k \cdot x \rightarrow a= -\frac{k}{m} \cdot x$$

    Это уравнение является уравнением колебаний, только здесь сила направлена всегда против скорости, а не против перемещения, поэтому автомобиль движется до остановки. 
    Закон движения можно описать как:
    $$ x(t)=\left\{
        \begin{aligned}
            x_{max} \cdot sin(\omega \cdot t), \: \text{до остановки};\\
            x_{max}, \: \text{после остановки};
        \end{aligned}
    \right. $$

    Остановка наступает через четверть периода этого синуса (скорость равна нулю). 
    Таким образом, $t=\frac{\pi}{2 \cdot \omega}=\frac{\pi}{2} \cdot \sqrt{\frac{m}{k}}$. Тормозной путь не зависит от начальной скорости.

    \answerMath{не изменится.}

    \markSection

    \begin{itemize}
        \item Записан второй закон Ньютона или выражение для энергии, из которого видно, что это уравнение колебаний. – 3 балла
        \item Выражена частота этих колебаний – 2 балла
        \item Указано, что поезд остановится через четверть периода – 2 балла
        \item Получен правильный ответ – 3 балла.
    \end{itemize}
\end{enumerate}