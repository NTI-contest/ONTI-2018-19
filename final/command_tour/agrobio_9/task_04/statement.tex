\assignementTitle{}{10}{}

Оцените важные для аквапонных установок физико - химические параметры воды и предложите методы их стабилизации без использования химических реагентов. Для работы может быть использовано все, что представлено в лаборатории.

Сделайте же это!  

Но только под контролем преподавателя!

\textit{Реализуйте идею и результат приведите в письменном виде в виде отчета. В отчете приведите расчеты, описание и схемы (если это необходимо). Листок с ответом подпишите (название команды) и проставьте дату.}

Максимальный балл - 10 баллов

\textbf{Задание выполняется “online”}

\markSection

Определены значения 8 возможных параметров (нитраты, нитриты, аммиак/аммоний, растворённый  углекислый газ, (ОВП) – редокс-потенциал, $рН$, $кН$, $gН$) – 0.625 баллов за каждый параметр; - макс. количество 5 баллов

Для каждого определённого параметра предложен свой вариант способа изменения параметра до оптимального значения (адекватного ситуации работы реальной аквапонной системы прикреплённой к команде) – 0.625 балла за каждый вариант оптимизации значения параметра; - макс. количество – 5 баллов.

\underline{Оптимизация параметров:}

Нитраты – увеличение биомассы водных растений или растений гидропонного модуля; частичная замена воды;

Нитриты – увеличение аэрации бактериального фильтра; увеличение площадки бакфильтра; частичная замена воды;

Аммиак/аммоний-ион – увеличение биомассы водных растений и растений гидропонного модуля; частичная замена воды; усиление аэрации бакфильтра, увеличение площади бакфильта;

Р-рёный углекислый газ – увеличение количества водных растений; подключение дополнительного растительноводного фильтра;

ОВП – добавить органики (торф низовой, фрагмент топляка); увеличить аэрацию; поставить оксигенатор; обработать УФ (увеличить количество образующихся перекисей в воде);

кН – отстаивание воды; кипячение перед добавлением в систему или частичной замене;

gН – добавить осадочные породы (известняк) для увеличения gН или топляк (коряги, сфагнум, торф) для уменьшения gН, кипячение перед частичной заменой, добавление дистиллированной воды;

рН – добавить осадочные породы (известняк) для увеличения рН или топляк (коряги, сфагнум, торф) для уменьшения рН