\subsection*{Требования к участникам}

\begin{itemize}
    \item Навыки проектирования, cборки и пусконаладки электронных схем и устройств.
    \item Программирование микроконтроллера Arduino.
    \item Умение работать с технической документацией.
    \item Владение английским для чтения технической документации по тематике.
    \item Навыки тестирования и отладки программ.
    \item Навыки тестирования и отладки электронного оборудования.
    \item Работа с датчиками освещённости, таймерами, геркона и др.
    \item Моделирование, конструирование, инженерные испытания.
    \item Основы веб-программирования (HTML/HTML5, Websockets)
    \item Работа в команде, распределение ролей.        
\end{itemize}

Ориентировочный состав команды — 5 человек:
\begin{itemize}
    \item 1 архитектор систем
    \item 1 электронщик/монтажник
    \item 1 программист электронных устройств (датчиков)
    \item 1 программист взаимодействий электронных устройств
    \item 1 тестировщик        
\end{itemize}

\subsection*{Возможный итоговый результат}

Следующим образом может выглядеть макет умного города:

\putImgWOCaption{12cm}{5}
\putImgWOCaption{12cm}{6}
\putImgWOCaption{12cm}{7}

Программная реализация умного города может состоять из 4 частей:

Релизация логики стенда:

\inputminted[fontsize=\footnotesize, linenos]{cpp}{final/command_tour/vrs/task_02/source_1.cpp}

Реализация логики беспроводного трансивера NRF24L01, который используется для общения с машиной:

\inputminted[fontsize=\footnotesize, linenos]{cpp}{final/command_tour/vrs/task_02/source_2.cpp}

Реализация логики машины:

\inputminted[fontsize=\footnotesize, linenos]{cpp}{final/command_tour/vrs/task_02/source_3.cpp}

Реализация логики трансивера машины:

\inputminted[fontsize=\footnotesize, linenos]{cpp}{final/command_tour/vrs/task_02/source_4.cpp}
