\subsection*{Этапы проведения}

\subsubsection*{Подготовка}

На этапе подготовки организаторы приводят помещение в состояние, необходимое для проведения олимпиады: расставляют столы, готовят маркерную доску и комплекты материалов, сырья, комплектующих и раздаточный материал для команд.

\subsubsection*{Техника безопасности}

Перед началом работы каждый участник команды должен пройти инструктаж по технике безопасности и поставить свою подпись в журнале. При построении макета участникам предстоит работать с острыми предметами, в связи с чем каждый член команды при работе с ножом, ножницами и прочими колюще-режущими предметами обязан использовать защитные перчатки во избежание возникновения порезов.

Несмотря на то, что вся электроника рассчитана на напряжение не выше 12 В, при сборке схем следует обращать внимание на следующие моменты:
\begin{itemize}
    \item полярность подключения линий питания
    \item правильность подключения информационных и питающих линий
    \item отсутствие короткого замыкания    
\end{itemize}

В процессе сборки схем в отдельных случаях может потребоваться пайка элементов и проводов. Для выполнения данного типа работы необходимо выполнить следующие правила и условия:
\begin{itemize}
    \item при работе с паяльником и сопутствующими инструменты необходимо надевать защитные очки и защитные перчатки
    \item проверить целостность шнура, штепсельной вилки и розетки
    \item пайка должна производиться на специально оборудованном и хорошо вентилируемом рабочем месте, с ярким освещением
    \item избегать прикосновения к жалу паяльника или паяльной станции.
    \item не работать вблизи горючих и легковоспламеняющихся предметов и на столах из горючих материалов без негорючей подставки.
    \item в перерывах между работой ставить паяльник только на подставку    
\end{itemize}

\subsubsection*{Организационный этап}

На организационном этапе участникам объясняют правила, ставят цель, которую должны достичь команды, и объясняют условия победы. Затем ведущие-консультанты описывают состав комплектов для команд и дают инструкции.

\subsubsection*{Работа над проектами}

Перед тем как приступить к заданию, каждая команда должна разработать концепцию своего проекта и согласовать с ведущим олимпиады в формате технического задания. После утверждения концепции каждой команде необходимо спланировать и распределить задачи между всеми участниками команды - в таком случае риск не уложиться в отведенное время существенно снижается.

В рамках направления конструирование командам требуется собрать макет из деталей и расходников, предоставляемых организаторами мероприятия. Собранный макет должен выполнять роль каркаса для набора электронных модулей системы умного города. Поскольку олимпиада не является архитектурным конкурсом, то основным критерием оценки качества макета выступает техническая эстетика, а именно:

\begin{itemize}
    \item На макете не должно быть внешних следов от связующих материалов (клея, скотча и т.д.);
    \item На макете по возможности должны быть максимально спрятаны из вида все соединительные провода;
    \item Ровные линии стыков углов макета;
    \item Отсутствие каких-либо внешних повреждений и общий аккуратный вид конструкции;
    \item Реализуемость Умного города, соответствующего макету по принципу работы и функционалу в реальной жизни;
    \item Соответствие задач макета реальным задачам, стоящим перед системами умного города;
    \item Опционально макет должен иметь возможность представления работоспособности всех систем наблюдателю, находящемуся в одной точке, без необходимости разрушения каких-либо элементов макета (быстросъемные элементы конструкции или другое конструктивное решение).        
\end{itemize}

В отдельных случаях организаторами может быть предложен дополнительный инструмент и\\или оборудования для дополнения макетов отдельными конструктивными и функциональными элементами из предоставленных командам материалов.

По направлению электроника каждой команде будет предложен набор электронных модулей и устройств. Каждая команда вправе использовать весь набор, однако нужно четко представлять принцип работы выбранного устройства и иметь понимание того, как оно будет интегрировано в общую архитектуру проекта.

Для успешного выполнения задания команда должна обладать следующими знаниями и навыками:

\begin{itemize}
    \item Основы электротехники, включая закон Ома, принципы действия базовых электронных компонентов;
    \item Навыки сборки электронных схем;
    \item Навыки проектирования плат;
    \item Навыки работы с паяльником и мультиметром.
\end{itemize}

Основными критериями по данному направлению являются:

\begin{itemize}
    \item Правильность подключения информационных и питающих линий;
    \item Нетривиальное использование компонентов;
    \item Количество задействованных элементов.
\end{itemize}

Направление Программирование подразумевает разработку двух составляющих проекта:

\begin{itemize}
    \item Управляющая программа для контроллера нижнего уровня
    \item Веб-панель управления, предоставляющая интерфейс пользователя системы        
\end{itemize}

Управляющая программа разрабатывается на базе микроконтроллерной платформы прототипирования Arduino Mega. Для успешного выполнения задания команда должна обладать следующими знаниями:

\begin{itemize}
    \item Разработка программных алгоритмов
    \item Основы языков программирования C/C++, в том числе базовые принципы объектно- ориентированного программирования
    \item Понимание принципов работы аппаратных интерфейсов UART, I2C, SPI, GPIO, АЦП, включая их инициализацию и имплементация программной логики на их основе
    \item Понимание принципа работы технологии беспроводной передачи данных LoRa, а также протокола LoRaWAN.
\end{itemize}

Выбор инструментов и методов разработки панели управления предоставляется сами участникам.

В обобщенном виде алгоритм выполнения задания может включать следующие шаги

\begin{enumerate}
    \item Изучение предметной области и объекта автоматизации
    \item Изучение состава комплекта программно-аппаратных средств, в том числе документации, предоставляемых организаторами
    \item Планирование работы, составление технического задания и распределение задач между членами команды
    \item Сборка макета и реализация функциональных требований
    \item Тестирование и отладка полученного решения
    \item Демонстрация законченного решения комиссии
\end{enumerate}

\subsubsection*{Взаимодействие с организаторами во время работы}

На протяжении всей работы команд, их сопровождают ведущие мероприятия. При возникновении каких-либо вопросов по организационной части, документации, или при вопросах, связанных с некорректной работой wifi-сети, оборудования и\\или его заменой, команды всегда могут обратиться к ведущим.

Также, при необходимости дополнительно установить на компьютеры, выданные участникам, каких-либо программ библиотек, драйверов или иного программного обеспечения, участники также могут попросить ведущих поставить им необходимое на компьютеры.

В ходе работы, если предусмотрены какие-либо дополнительные материалы, оборудование или ресурсы для участников, они могут их также получить через ведущих.

\subsubsection*{Перерывы и их проведение}

Если предварительно в программе проведения мероприятия, перерывы и/или централизованные приемы пищи всеми участниками не указаны, то это означает, что команды могут распоряжаются своим временем полностью по своему усмотрению и самостоятельно выбирают время для своего отдыха, сна, приемов пищи и т.д. Таким образом, если команда или ее участник в этой ситуации захотят сделать перерыв на еду, она может поесть в отведенной для этого зоне для кофе-брейков или, с разрешения организаторов (вожатых), пойти в буфет/столовую.

Во время перерывов участники не должны покидать здание, где проводится мероприятие, кроме случаев, когда получили на это разрешение от вожатых, т.к. они несут личную ответственность за всех участников. 

Если перерывы в программе мероприятия указаны, то участники на время перерывов останавливают все свои работы и отходят от столов, соответственно, время, отведенное командам на работу останавливается. После начала перерыва на прием пищи (завтрак/ обед/ ужин) участники завершают свои работы, организованно собираются около вожатых и централизованно идут в их сопровождении к месту приема пищи. В этом случае участники не могут остаться и продолжить работу или пойти куда-либо самостоятельно.

После завершения общих перерывов работа команд продолжается только после объявления ведущих. В противном случае, команды будут оштрафованы на то время, которое они потратили на работу. Это же относится к удаленной работе и/или в случае, когда участники взяли с собой и использовали технику для работы вне рабочей зоны.

Исключение правила запрета работы вне рабочей зоны, действующее для всех команд - это работа над презентацией и выступлением. Однако, даже эти действия рекомендуется выполнять либо в рабочей зоне, либо в месте размещения участников на время мероприятия (гостиница/ профилакторий).

Исключение из правила запрета любой работы во время общего перерыва, являющееся индивидуальным, это возможность для команд, которым было начислено дополнительное время по причине замены комплектующих, работать во время перерыва по своему желанию в течение добавленного им времени.

\subsubsection*{Регламент демонстрации разработанных решений} 

Для подведения итогов формируется экспертная комиссия, которая оценивает разработанные критерии согласно таблице 3. 

Оценивается теоретический концепт Умного города: его актуальность и позиционирование, цели и задачи команд при работе над проектом, реализуемость предлагаемой системы в реальной жизни с учетом и в соответствии с описанием системы, которую представляет команда, и т.д. Подробнее указано в критериях оценки.

\begin{itemize}
    \item Выступление команды проходит не более 5 минут в формате защиты подготовленного решения;
    \item За минуту до окончания времени (после 4 минут выступления) команду предупреждают о скором завершении защиты;
    \item Если команда не укладывается в отведенное регламентом время, ее прерывают без возможности продолжения защиты.     
\end{itemize}

\subsubsection*{Оценка практического представления работы команды}

Оценивается макет Умного города: его работоспособность, функционал, аккуратность сборки, его соответствие ТЗ написанного командами, программный код и его совершенство, панель управления, ее удобство и т.д. Подробнее указано в критериях оценки.
\begin{itemize}
    \item Выступление команды идет не более 13 минут в формате стендовой защиты. Роль стенда выполняет сам макет;
    \item За минуту до окончания времени выступления команду предупреждают о скором окончании защиты;
    \item Во время защиты жюри проверяет соответствие выполненной работы и ТЗ, а также качество реализации;
    \item Во время защиты жюри имеет право задавать вопросы по теме для проверки эрудиции команд;
    \item Время на переход и подготовку к выступлению следующих участников не должно превышать 2 минуты.         
\end{itemize}

\subsubsection*{Подведение итогов}

После завершения защиты всех команд в течение 1 часа жюри подводят итоги выступлений и награждают победителей и призеров. Также командам выдается сводная таблица результатов. После этого у команд есть возможность узнать свои сильные и слабые стороны, пообщаться с жюри и получить ответы на интересующие их вопросы.

\subsubsection*{Апелляция и решение спорных вопросов}

Строго говоря, апелляция по факту несогласия участников с итоговой оценкой их работы запрещена и невозможна. В то же время, если команда имеет какие-либо вопросы к жюри после завершения защиты, она может подойти и спросить, почему ее работа была оценена именно таким образом.

Изменение баллов команде после их окончательного выставления возможно только в случае ошибки при подсчетах итоговых баллов (что может быть проверено самими участниками по сводной таблице, выдаваемой им по завершению награждения) и/или грубого нарушения регламента организаторами.
