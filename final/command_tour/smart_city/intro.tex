\subsection*{Описание задачи}

Одним из направлений развития урбанистического общества является цифровизация и автоматизация городского пространства с помощью технических средств с целью оптимизации тех или иных процессов.

Внедрение подобных систем, как правило, осуществляется на нескольких уровнях: индивидуальное жилье, общегородская инфраструктура, промышленность. При этом оптимизация процессов в масштабе города и на предприятиях видится наиболее перспективной, поскольку влечет за собой существенные экономические и социальные эффекты, что ускоряет окупаемость затрат на разработку и развертывание системы. В то же время, автоматизация индивидуального жилья также актуальна, однако дороговизна компонентов “Умного дома” сдерживает массовое внедрение системы, несмотря на то, что производители и системные интеграторы уже давно представили на рынке множество решений.

В рамках Олимпиады по направлению “Умный город” командам предлагается решить задачи по следующим направлениям:

\begin{itemize}
    \item автоматизация процессов сбора показаний приборов учета и управления поставкой ресурсов
    \item обеспечение безопасности в детских учебных учреждениях
    \item организация автоматических частных пассажироперевозок    
\end{itemize}

Данные направления, какие-то в большей, какие-то в меньшей степени, уже представлены в крупных городах. В частности, автоматизация ЖКХ является одной из наиболее широко внедряемых технологии цифровизации города, что подтверждает актуальность направления.

Что касается безопасности в детский садах и школах, то существующие системы безопасности в виде систем видеонаблюдения решают задачу лишь отчасти, поскольку в случае возникновения ЧП позволяют лишь постфактум установить обстоятельства, при которых ребенок покинул территорию. В предлагаемом концепте новизна технического решения заключается в том, что оно позволяет реагировать мгновенно на события в режиме реального времени. Существующие браслеты на основе технологий геолокации благодаря поддержке функции построения геозон в целом решают задачу, но время автономной работы таких устройств очень ограничено, что накладывает существенные ограничения и создает неудобства в виде необходимости постоянно подзарядки..

Задача управления беспилотным транспортом представлена в несколько другом разрезе по сравнению с классической задачей автономности, где фокус делается в основном на цифровой обработке информации с датчиков и управлении автомобилем.. В нашем случае алгоритм автономности предполагает весьма упрощенный подход, и акцент сделан на интеграции автомобиля в общую инфраструктуру города, который функционирует в рамках единой информационной сети.

Информационная сеть, о которой сказано выше, является связующим звеном между всеми проектируемыми объектами города, и она также должна быть построена командами. Сделать они это должны будут на двух уровнях:
\begin{itemize}
    \item уровень датчиков и исполнительных механизмов: контроллер нижнего уровня, осуществляющий сбор данных и трансляцию команд через беспроводной трансивер
    \item уровень веб-панели управления, куда стекается информация об объектах автоматизации        
\end{itemize}

Базовой технологией обмена информацией, обеспечивающей взаимодействие всех элементов города с панелью управления, является стандарт беспроводной передачи данных LoRaWAN. Выбор данной технологии обусловлен тем, что данный стандарт проектировался целенаправленно для подобных задач автоматизации. Поскольку технология является относительно новой по сравнению со ставшими уже классическими технологиями доступа WiFi, Ethernet и GSM, то здесь подразумевается образовательный контекст, в рамках которого команды познакомятся со стандартом и практическими аспектами его работы, что создает существенный задел для тех школьников, кто намерен посвятить себе электронной инженерии. 

Командам будет предоставлена возможность осуществлять передачу данных через многоканальную базовую станцию. Серверное ПО также уже будет развернуто заранее, и командам будет представлено описание интерфейса доступа к данным (API) по протоколу Websockets, с помощью чего команды построят собственную панель управления.Веб-интерфейс может быть выполнен на отдельном хосте (ПК участников) в виде веб-страницы. Он должен отображать информацию о каждом объекте и предоставлять возможность отправить управляющие воздействия и/или задать параметры работы объекта. Ниже приведен предполагаемый формат веб-интерфейса для каждого отдельно взятого объекта:

\begin{table}[H]
    Таблица 1 — Краткое описание веб-интерфейса панели управления
    \begin{tabular} {|c|p{6cm}|p{8cm}|}
        \hline
        & Тема & Описание функций веб-интерфейса \\
        \hline
        1 & Автоматизация сбора информации ЖКХ & 
        \begin{itemize}
            \item Отображение расхода воды в см. куб
            \item Отображение расхода электричества
            \item Управление отключением подачи электричества
            \item Управление отключением подачи воды                
        \end{itemize} \\
        \hline
        2 & Беспилотный транспорт &
        \begin{itemize}
            \item Отображение текущего положения автомобиля
            \item Отправка команды вызова автомобиля    
        \end{itemize} \\
        \hline
        3 & Система оповещения для детских садов и школ	— Добавление и удаление пользователей (ФИО, адрес метки) &
        \begin{itemize}
            \item Отображение списка пользователей
            \item Отображение списка пользователей, находящихся за пределами разрешенной зоны                
        \end{itemize}\\
        \hline
    \end{tabular}
\end{table}

На рисунках 1-3 представлены структурные элементов Умного города. 

\putImgWOCaption{12cm}{1}

\centerline{Рисунок 1 — Основное оборудование стенда}

\putImgWOCaption{12cm}{2}

\centerline{Рисунок 2 — Беспилотный автомобиль}

\putImgWOCaption{12cm}{3}

\centerline{Рисунок 3 — Информационная инфраструктура города}

Ниже приведен возможный внешний вид макета города:

\putImgWOCaption{12cm}{3}

\centerline{Рисунок 4 — Примерный внешний вид макета}

Ориентировочный габарит макета составляет $80\times 80 \times 30$ см. Он содержит несколько элементов городской инфраструктуры: жилой дом, детский сад или школу, парковку, светофор и административное здание. В жилом доме предполагается симуляция системы сбора показания счетчиков воды и электричества, в детском саду (школе) устанавливается система контроля присутствия, административное здание выполняет декоративную функцию и будет, к примеру, содержать только элементы подсветки. Отдельная площадка выделена под парковку, на которой размещается беспилотный транспорт (выше не показан). При подаче команды вызова автомобиль по команде центрального контроллера начинает движение. При выезде с парковки автоматически по датчику открывается шлагбаум. Автомобиль следует вдоль линии, учитывая сигналы светофора, останавливаясь на маршрутных метках, после чего возвращается на парковку.

Ниже представлено предварительное описание задач в части видения организаторами функционала и состава, при этом каждая команда для получения максимальной оценки должна выполнить все проекты из таблицы 2.

\begin{table}[H]
    Таблица 2 — Перечень тематик на проработку в рамках олимпиады
    \begin{tabular}{|c|p{6cm}|p{8cm}|}
        \hline
        & Тема & Описание задачи \\
        \hline
        1 & Автоматизация сбора информации ЖКХ & 

        Команде предоставляется набор комплектующих для сборки стенда, состоящего следующих элементов:
        \begin{itemize}
            \item детали каркаса стенда
            \item электроника (датчики, исполнительные механизмы, контроллер, беспроводной приемопередатчик, соединительные провода, и т.д.)
            \item краткая инструкция по сборке стенда
            \item описание серверного протокола        
        \end{itemize}
        
        Функционально стенд должен выполнять следующие задачи:
        
        \begin{itemize}
            \item управление питанием датчика тока
            \item ручное регулирование током в цепи датчика тока
            \item ручное включение/отключение помпы
            \item съем показаний с датчика тока
            \item управление помпой
            \item съем показаний с датчика потока воды
            \item передача на сервер показаний датчика тока и потока воды в соответствии с протоколом
            \item чтение с сервера команды на отключение подачи электричества и воды
        \end{itemize} \\
        \hline

        2 & Беспилотный транспорт & 
        
        Команде предоставляется набор комплектующих для сборки стенда, состоящего следующих элементов: 

        \begin{itemize}
            \item детали каркаса стенда
            \item электроника (датчики, исполнительные механизмы, контроллер, беспроводной приемопередатчик, соединительные провода, и т.д.)
            \item краткая инструкция по сборке стенда
            \item описание серверного протокола                
        \end{itemize}
        
        Функционально стенд должен выполнять следующие задачи:
        
        \begin{itemize}
            \item автономное движение автомобиля вдоль линии
            \item чтение маршрутных меток 
            \item возможность остановки над маршрутными метками
            \item передача на сервер о текущей локации (над какой меткой расположен автомобиль) в соответствии с протоколом
            \item чтение с сервера команды подачи авто
            \item управление светофором
            \item управление шлагбаумом.    
        \end{itemize} \\
        \hline

        3 & Система оповещения для детских садов и школ & 
        
        Команде предоставляется набор комплектующих для сборки стенда, состоящего следующих элементов:
        \begin{itemize}
            \item детали каркаса стенда
            \item электроника (датчики, исполнительные механизмы, контроллер, беспроводной приемопередатчик, соединительные провода, и т.д.)
            \item краткая инструкция по сборке стенда
            \item описание серверного протокола        
        \end{itemize}
        
        Функционально стенд должен выполнять следующие задачи:
        
        \begin{itemize}
            \item управление системами оповещения (световая, звуковая и др.)
            \item сканирование радиомаяков iBeacon и сохранение их адресов
            \item отправка на сервер адресов радиомаяков в соответствии с серверным протоколом
            \item прием команды на включении систем оповещения о выходе объекта за пределы разрешенной зоны
            \item опционально: выдача пуш-уведомлений в собственное мобильное приложение о выходе объекта мониторинга за пределы разрешенной зоны
            \item опционально: маяк при выходе из разрешенной зоны должен отсылать на сервер информацию о геолокации.        
        \end{itemize} \\
        \hline
        
    \end{tabular}
\end{table}

\subsection*{Порядок проведения олимпиады}

Мероприятие проводится в хорошо вентилируемом помещении размером, достаточным для вмещения всех команд и оборудования, материалов и комплектующих. Каждой команде предоставляется отдельный стол с розеткой (удлинителем), за которым команда будет работать в ходе всего мероприятия, необходимый инвентарь и инструкции. Также у команды должен иметься в наличии минимум один ноутбук с установленной средой Arduino IDE и всеми необходимыми библиотеками для разработки программы, а также браузером, с возможностью доступа в интернет. В течение всего времени проведения участникам помогают ведущие-консультанты. Для подведения итогов формируется конкурсная комиссия, которая оценивает разработанные командами решения.

\subsection*{Инвентарь}

В таблице 3 представлен предварительный список оборудования для проведения олимпиады. Список будет пополняться в процессе уточнения требуемых комплектующих и элементов.


Таблица 3 — Оборудование и программное обеспечение для проведения мероприятия

\begin{tabular}{|p{10cm}|c|c|}
    \hline
    Общее для всех систем & Ед.изм & Кол-во \\
    \hline
    Arduino Mega & Шт. & 1 \\
    Трансивер LoRa 868 МГц (SPI) & Шт. & 1 \\
    Беспроводной трансивер 2.4 ГГц & Шт. & 1  \\
    Базовая станция LoRaWAN & Шт. & 1 \\
    Серверное ПО LoRaWAN IOT Vega Server (win) v1.2.1 & Шт. & 1 \\
    Сервер с виртуальной машиной Windows & Шт. & 1  \\
    Среда разработки Arduino IDE & Шт. & 1 \\
    Детали конструктора для сборки стенда & Набор & 1 \\
    - основание & - & - \\
    - стены	& - & - \\
    - окна & - & - \\
    - метизы & - & - \\
    Термоклеевые стержни 11 мм/270 мм & Стержень & 20 \\
    Дозаторы термоклея & Шт. & 1 \\
    Монтажное рабочее место & Шт. & 1 \\
    Отвертки & Набор & 1 \\ 
    \hline
    \hline
    Автоматизация ЖКХ & Ед.изм & Кол-во	 \\
    \hline
    Датчик потока воды & Шт. & 1 \\
    Датчик тока & Шт. & 1 \\
    Насос & Шт. & 1 \\
    Силовой ключ & Шт. & 2 \\
    Емкость для жидкости & Шт. & 2 \\
    Шланги & Набор & 1 \\
    Провода & Набор & 1 \\ 
    \hline            
\end{tabular}


\begin{tabular}{|p{10cm}|c|c|}   
    \hline
    Беспилотный транспорт & Ед.изм & Кол-во		 \\
    \hline
    Шасси автомобиля Amperka miniQ & Набор & 1 \\
    Драйвер двигателя (шилд) & Шт. & 1 \\
    Платформа Arduino & Шт. & 1 \\
    Беспроводной трансивер 2.4 ГГц & Шт. & 1 \\
    Датчик линии аналоговый & Шт. & 1 \\
    Плата расширения & Шт. & 1 \\
    Батарейный отсек & Шт. & 1 \\
    Провода & Набор & 1	 \\
    Светофор & Набор & 1 \\
    - стойка & - & - \\
    - светодиодная лента & - & - \\
    - провода & - & - \\
    Шлагбаум & Набор & 1 \\
    - корпус & - & - \\
    - привод & - & - \\
    \hline
    \hline
    Система оповещения для детских садов и школ	& Ед.изм & Кол-во		 \\
    \hline
    Модуль BLE HM-10 (UART RX/TX) & Шт. & 1 \\
    Радиомаяк iBeacon & Шт. & 1 \\
    Пьезоизлучатель & Шт. & 1 \\
    Светодиодный модуль & Шт. & 1 \\
    \hline

\end{tabular}

