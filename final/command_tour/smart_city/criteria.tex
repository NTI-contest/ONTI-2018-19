\subsection*{Критерии оценки результата}

Ниже приведена таблица соответствия реализованной функции и количество баллов:
\begin{table}[H]
    Таблица 4 — Критерии оценки выполнения задания
    \begin{tabular}{|p{14cm}|c|}
        \hline
        Описание функции&Баллы \\
        \hline
        \hline
        \textbf{Конструирование}	& \textbf{35} \\
        \hline
        Внешняя эстетика макета 

        \begin{itemize}
            \item ровные линии соединения 
            \item отсутствие явных загрязнений поверхности       
        \end{itemize}  & 25 \\
        \hline
        Реализуемость (соответствие макета возможности строительства 
        
        в реальной жизни) & 10 \\
        \hline
        \hline
        \textbf{Включение в логику следующих элементов:} & \textbf{72}  \\
        \hline
        Опрос датчика потока воды & 2 \\
        \hline
        Опрос датчик тока & 2 \\
        \hline
        Опрос BLE-сканера & 5 \\
        \hline
        Прием кода маршрутной метки от автомобиля & 4 \\
        \hline
        Установка сигнала светофора & 2 \\
        \hline
        Считывание показаний датчика света возле шлагбаума & 3 \\
        \hline
        Управление шлагбаумом 
        \begin{itemize}
            \item открыть            
            \item закрыть
        \end{itemize} & 2 \\
        \hline
        Передача статуса светофора автомобилю & 4 \\
        \hline
        Включение/отключение подачи воды & 2 \\
        \hline
        Включение/отключение подачу электричества & 2 \\
        \hline
        Отправка команды на старт движения автомобиля & 4 \\
        \hline
        Имплементация/запуск/портирование драйвера трансивера LoRa & 5 \\
        \hline
        Передача данных на сервер
        \begin{itemize}
            \item показания датчика потока воды 
            \item показания датчика тока 
            \item код маршрутной метки 
            \item UUID обнаруженного в радиоэфире маяка                 
        \end{itemize} & 10 \\
        \hline
        Прием данных с сервера
        \begin{itemize}
            \item включить/отключить подачу воды
            \item включить/отключить подачу электричества
            \item команда вызова автомобиля
            \item включить/отключить сканирование маяков                 
        \end{itemize} & 10 \\
        \hline
        Полуавтоматическое движение автомобиля
        \begin{itemize}
            \item старт движения
            \item движение вдоль линии
            \item остановка на маршрутных метках
            \item передача номера маршрутной метки
            \item остановка на светофоре
            \item остановка перед шлагбаумом
            \item остановка на парковке                         
        \end{itemize} & 15 \\
        \hline
        
    \end{tabular}
    
\end{table}

\begin{table}[H]
    \begin{tabular}{|p{14cm}|c|}
        \hline
        \textbf{Панель управления} & \textbf{25}  \\
        \hline
        Реализация веб-панели управления 
        \begin{itemize}
            \item реализация уровня обмена сообщениями с макетом 
            \item реализация интерфейса 
            \item реализация логики отображения информация и управления            
        \end{itemize} & 25 \\
        \hline
        \hline
        \textbf{ Междисциплинарные критерии} & \textbf{10}  \\
        \hline
        Команда смогла ответить на все вопросы по презентации & 10 \\
        \hline
        \hline
        \textbf{Штрафные баллы}	 & \\
        \hline
        Реализация дополнительных элементов или нестандартное использование имеющихся, описанных в техническом задании & -0,5xN \\
        \hline
        Отсутствие элементов, описанных в техническом задании & -0,5xN \\
        \hline
        Нарушение техники безопасности & -15хN \\
        \hline
        Нарушение регламента проведения мероприятия & -10хN \\
        \hline
        \hline
        \textbf{ИТОГОВАЯ СУММА БАЛЛОВ} & \textbf{142} \\
        \hline
    \end{tabular}
    
\end{table}
