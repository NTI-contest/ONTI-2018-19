\textit{Разработка AR-навигатора.}

Надо создать модуль приложения, отвечающий за отображение моделей достопримечательностей, а также дополнительных объектов взаимодействия. Модели отображаются в “реальном” мире, с учетом положения устройства.

\markSection
\begin{itemize}
    \item	Отображение различных моделей в режиме AR-браузера (т.е. в разных местах открываются разные модели) – 16 баллов.
    \item	3Д модели расположены на расстоянии примерно равном расстоянию до объекта на карте – 10 баллов.
\end{itemize}

На выбор отображение моделей на экране:
Прочее:
\begin{itemize}
    \item	Объекты располагаются на экране вне зависимости от положения устройства по центру.объекты. – 12 баллов.
    \item	Располагаются примерно на плоскости земли (с использованием датчиков акселерометра и/или магнитометра и/или гироскопа). – 24 балла.
    \item	Предыдущее + модель отображается только тогда, когда в её сторону направлена камера. – 36 баллов.
\end{itemize}

\solutionSection
Реализация отображает разные модели. Так же модели отображаются на расстоянии примерно равном расстоянию до объекта. 

Модель можно увидеть только если угол разворота пользователя совпадает с расположением виртуальной модели.

\putImgWOCaption{10cm}{1} 

\begin{center}
    Рис.4.7. Работа AR-браузера в режиме AR-навигатора и поиска артефактов
\end{center}

\textbf{ОЦЕНКА:}
\begin{itemize}
    \item	Отображаются различные модели в режиме AR-браузера (т.е. в разных местах открываются разные модели) – 16 баллов.
    \item	3Д модели расположены на расстоянии примерно равном расстоянию до объекта на карте – 10 баллов.
    \item	Модели располагаются примерно на плоскости земли (с использованием датчиков акселерометра и/или магнитометра и/или гироскопа). + модель отображается только тогда, когда в её сторону направлена камера. – 36 баллов.
\end{itemize}

\inputminted[fontsize=\footnotesize, linenos]{csharp}{final/command_tour/ar/task_05_1/source_1.cs}

\inputminted[fontsize=\footnotesize, linenos]{csharp}{final/command_tour/ar/task_05_1/source_2.cs}