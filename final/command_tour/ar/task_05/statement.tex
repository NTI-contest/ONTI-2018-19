\assignementTitle{Виртуальная карта проекта}{200}{}

Разработка программного компонента мобильного приложения для визуализации интерактивной карты территории, на которой объекты городской инфраструктуры представлены при помощи технологий дополненной реальности в виде трехмерных цифровых моделей, с возможностью отображения маршрута, соответствующего номеру группы, всех возможных маршрутов для заданного числа групп ( от 3 до 7), посетителей территории, находящихся на ней в настоящий момент.

\markSection
Динамическая сетка, Маркеры (2-4) задают границы сетки-карты. В зависимости от их расположения  меняется размер визуализации участка территории и отображаемых достопримечательностей .
\begin{itemize}
    \item Сетка подстраивается под маркеры – 12 баллов. 
    \item Количество ячеек подстраивается под JSON – 8 баллов.
    \item В соответвенных точках маршрута подсвечиваются соответственные артефакты. – 35 баллов
\end{itemize}

Построение маршрутов на выбор:
\begin{itemize} 
    \item	Маршрут статичный (векторы рисует на программа, а это заложенное заранее изображение). – 10.5 баллов.
    \item	Маршрут строится в приложении, т.е. линия между точками генерируется – 35 баллов.
    Прочее:
    \item	Меню для выбора маршрута, который нужно показать – 10 баллов.
    \item	Графы прохождения других групп – 35 баллов.
\end{itemize}

\solutionSection
Было разработано приложение с следующим интерфейсом:
\putImgWOCaption{8cm}{1}
Реализованное приложение перестраивает сетку под положение маркеров. Возможно задавать разные варианты JSON и приложение будет перестраивать сетку.
\putImgWOCaption{10cm}{2} 

\textbf{ОЦЕНКА:}
\begin{itemize}
    \item	Сетка подстраивается под маркеры – 12 баллов. 
    \item	Количество ячеек подстраивается под JSON – 8 баллов.
\end{itemize}
Из файла JSON происходит считывание и в нужных местах сетки подсвечиваются артефакты которые нужно собрать. Так же между артефактами натягиваются вектора-указатели показывающие в каком направлении нужно идти определенной группе.

\putImgWOCaption{10cm}{3} 
 
\textbf{ОЦЕНКА:}
\begin{itemize}
    \item	В соответвенных точках маршрута подсвечиваются соответственные 
артефакты. – 35 баллов
    \item	Маршрут строится в приложении, т.е. линия между точками 
генерируется – 35 баллов.
\end{itemize}

Разработано меню, в котором можно посмотреть маршруты прохода других групп. На изображении приведен пример работы приложения для квеста в котором участвуют 4 группы. Можно посмотреть маршрут каждой из них.

\putImgWOCaption{10cm}{4} 

\textbf{ОЦЕНКА:}
\begin{itemize}
    \item	Меню для выбора маршрута, который нужно показать – 10 баллов.
    \item	Графы прохождения других групп – 35 баллов.
\end{itemize}

\subsection*{Добавление геолокации}

\markSection
На выбор отображение групп:
    \begin{itemize}
        \item	Группа отображается одной точкой.  – 35 баллов
        \item	Отображается каждый участник группы с клеточной точностью. – 49 баллов
        \item	Отображается каждый участник группы с 1/10 * клеточной точностью. – 70 баллов
    Прочее:
        \item	Отображение участников других групп(баллы за отображение одной группы.
    \end{itemize}
\solutionSection
С помощью перевода из глобальных координат в глобальные ??? был реализован алгоритм который с 1/10 клеточной точностью отображает группы на карте. Точность можно менять. 
\putImgWOCaption{10cm}{5} 
\textbf{ОЦЕНКА:}
\begin{itemize}
    \item Отображается каждый участник группы с 1/10 * клеточной точностью – 70 баллов
\end{itemize}
В приложении есть настройки которые отображают расположение участников других групп.
\putImgWOCaption{10cm}{6} 

\subsection*{Разработка AR-браузера}

\markSection
\begin{itemize}
    \item	Отображение различных моделей в режиме AR-браузера (т.е. в разных местах открываются разные модели) – 16 баллов.
    \item	3Д модели расположены на расстоянии примерно равном расстоянию до объекта на карте – 10 баллов.
\end{itemize}
На выбор отображение моделей на экране:
\begin{itemize}
    \item	Объекты располагаются на экране вне зависимости от положения устройства по центру.объекты. – 12 баллов.
    \item	Располагаются примерно на плоскости земли (с использованием датчиков акселерометра и/или магнитометра и/или гироскопа). – 24 балла.
    \item	Предыдущее + модель отображается только тогда, когда в её сторону направлена камера. – 36 баллов.
\end{itemize}

\solutionSection
Реализация отображает разные модели. Так же модели отображаются на расстоянии примерно равном расстоянию до объекта. 
Модель можно увидеть только если угол разворота пользователя совпадает с расположением виртуальной модели.

\putImgWOCaption{10cm}{7} 

\textbf{ОЦЕНКА:}
\begin{itemize}
    \item	Отображаются различные модели в режиме AR-браузера (т.е. в разных местах открываются разные модели) – 16 баллов.
    \item	3Д модели расположены на расстоянии примерно равном расстоянию до объекта на карте – 10 баллов.
    \item	Модели располагаются примерно на плоскости земли (с использованием датчиков акселерометра и/или магнитометра и/или гироскопа). + модель отображается только тогда, когда в её сторону направлена камера. – 36 баллов.
\end{itemize}

