\solutionSection

\textbf{Задача 1.1 Дизайн (Множитель 20)}

\begin{itemize}
    \item На сайте присутствуют перечисленные блоки из "Структура лендинга" - 0.1 баллов за каждый реализованный блок 
    \putTwoImg{8cm}{1}{8cm}{2}
    \putTwoImg{8cm}{3}{8cm}{4}
    \putTwoImg{8cm}{5}{8cm}{6}
    \putImgWOCaption{8cm}{7}

    \textbf{Итого максимум:} 0.7 балла

    \item Контрастная цветовая гамма (основные элементы выделены яркими цветами) - 0.2 балла

    Контрастная цветовая гамма. Основой выбран белый цвет, кнопки и ключевые элементы выделяются оранжевым и темно коричневым. Используются яркие изображения той же цветовой гаммы.

    \putTwoImg{8cm}{8}{8cm}{9}

    \textbf{Итого максимум:} 0.2 балла

    \item Цвета выбраны не случайной, например, с помощью цветового круга - 0.3 балла

    Цветовая палитра сайта была создана на основе выбранного ранее изображения при помощи сайта: \url{https://www.imgonline.com.ua/get-dominant-colors.php}

    \putImgWOCaption{8cm}{10}
    
    \textbf{Итого максимум:} 0.2 балла

    \item Хорошая компоновка графики и текста - 0.4 балла
    \item Однородность текста и равномерное распределение свободного пространства - 0.2 балла
    \putImgWOCaption{10cm}{11}
    \putImgWOCaption{8cm}{12}
    \textbf{Итого максимум:} 0.6 балла
\end{itemize}

\textbf{Задача 1.2 Верстка (Множитель 10)}
\begin{itemize}
    \item Реализована адаптивная верстка (0.2 балла за каждый тип устройства: десктоп/планшет/мобильное устройство)

    Реализована адаптивная версия сайта для мобильного устройства, планшета и десктопа 

    \putTwoImg{6cm}{13}{10.3cm}{14}
    \putImgWOCaption{8cm}{15}
    \textbf{Итого максимум:} 0.6 балла

    \item Форма регистрации расположена на самой посадочной странице - 0.5 балла
    \putImgWOCaption{8cm}{16}
    \textbf{Итого максимум:} 0.5 балла
\end{itemize}

\textbf{Задача 2 (Множитель 10)}
\begin{itemize}
    \item По нажатию на пункт меню, страница перемещается к соответствующему блоку лендинга  - 0.25 балла (+0.25 за анимированную версию)
    \item Наличие навигационной системы, которая отображает просматриваемый блок- 0.5 балла
    \putImgWOCaption{8cm}{17}
    \putImgWOCaption{8cm}{18}
    \textbf{Итого максимум:} 1 балл

    \item Наличие анимации на блоках лендинга - 0.4 балла

    На сайте реализована анимация отдельных элементов страницы с помощью фреймворка wow.js
    \textbf{Итого максимум:} 0.4 балла
\end{itemize}

\textbf{Задача 3 (Множитель 20)}

\begin{itemize}
    \item Реализована проверка полученных данных на пустоту - 0.1 балла
    \item Реализована проверка полученных данных на валидность - 0.25 балла (+0.25 за использование regex)
    \item Реализовано сохранение полученных данных в базу данных - 0.5 балла
    \item Выводится сообщение об ошибке в полученных данных - 0.2 балла
    \item Данные после отправки формы и вывода ошибки не сбрасываются - 0.2 балла
    
    После получения данных на сервере происходит их проверка на пустоту и валидность (проверка на то, что пользователь не отправил пустую форму и не ввел лишних символов). После проверки данные сохраняются в СУБД MySQL на сервере, для связи с базой данных используется расширение PDO. 

    Если при проверке была обнаружена ошибка, она выводится пользователю, данные в базу данных не сохраняются, уже введенные данные в форму не сбрасываются.
    \putTwoImg{8cm}{19}{8cm}{20}
    \textbf{Итого максимум:} 1.5 балла
    
    \item Отправка данных на сервер - 0.5 балла (-0.25 за перезагрузку страницы)
    \item Выводится сообщение о успешной регистрации пользователя - 0.2 балла
    
    Отправка данных на сервер осуществляется без перезагрузки страницы с помощью Ajax. После успешной регистрации группы на экскурсию, выводится сообщение с указанием кода группы для мобильного приложения с дополненной реальностью.
    \putImgWOCaption{8cm}{21}
    \textbf{Итого максимум:} 0.7 балла
\end{itemize}

\textbf{Задача 4 (Множитель 15)}
\begin{itemize}
    \item Добавлено краткое и лаконичное описание реализованного проекта, раскрывающее его суть - 0.25 балла
    \item Соответствие контента выбранной тематике сайта - 0.2 балл
    
    Добавлено описание проекта и экскурсии, описывающее основную цель экскурсии и использование технологии дополненной реальности в данном проекте. 
    Весь контент (изображение, текст) соответствует  тематике сайта, то есть раскрывает тему экскурсии по старинным каменным усадьбам города Иркутска.
    
    \putImgWOCaption{8cm}{22}
    \textbf{Итого максимум:} 0.45 балла
    \item Наличие описания экскурсии/квеста/шоп-тура: маршрут следования (0.1 балла), время прохождения (0.1 балла)/легенда (0.1 балла), подсчет очков (0.1 балла)
    \item Отображение маршрута: без карты (штраф 0.5 балла), на статической карте (0.2 балла), на динамической карте с отрисованным маршрутом (0.4 балла)
    
    На сайте добавлено описание маршрута, перечислены все усадьбы и особняки, также добавлено время прохождения всего маршрута, так как это экскурсия, а не квест, то этого достаточно, легенда и подсчет очков здесь не нужны.
    
    На сайте также отображается маршрут на динамической карте с отрисованным маршрутом. Карта реализована при помощи конструткора карт от Яндекса (\url{https://yandex.ru/map-constructor/})
    \putImgWOCaption{8cm}{23}
    \textbf{Итого максимум:} 0.6 балла
    
    \item Добавлено описание объектов, соответствующих тематике лендинга  - 0.25 балла
    \item Добавлены изображения объектов, соответствующих тематике лендинга - 0.25 балла
    
    На сайт добавлены изображения и описания самых значимых и интересных достопримечательностей, которые будут посещены на экскурсии (соответствуют тематике сайта) 

    \putImgWOCaption{8cm}{24}
    \textbf{Итого максимум:} 0.5 балла
    
    \item Наличие графических и/или текстовых объектов, позволяющих усилить интерес пользователя предпринять необходимое действие (время акции, скидки, количество оставшихся товаров и т.п.) - 0.5 балла
    \putTwoImg{8cm}{25}{8cm}{26}
    
    \item Наличие сведений о команде разработчиков - 0.25 балла
    \item Наличие контактов и иконок социальных сетей - 0.1 балла
    
    На сайте добавлены сведения о команде разработчиков (фото, ФИО, роль в проекте)
    Также добавлены основные контакты команды разработчиков: адрес, телефон, почта, группа в социальной сети VK.

    \putTwoImg{8cm}{27}{8cm}{28}
    \textbf{Итого максимум:} 0.35 балла
\end{itemize}

\textbf{Задача 5 (Множитель 5)}

\begin{itemize}
    \item Разработан брендированный QR для скачивания приложения с посадочной страницы - 1 балл

    Был разработан брендированный QR-для скачивания приложения с помощью сайта https://www.qrcode-monkey.com/
    Отличительной особенностью QR являются цветовая палитра и логотип.
    
    \putImgWOCaption{5cm}{29}

    \textbf{Итого максимум:} 1 балл
\end{itemize}