\subsubsection*{Часть 2.  Добавление сведений о местонахождении экскурсионных групп}

Необходимо добавить в приложение фиксацию текущего положения участника, а также отображать положение своей и других групп с определенной точностью.

\markSection

На выбор отображение групп:
\begin{itemize}
\item Группа отображается одной точкой — 35 баллов
\item Отображается каждый участник группы с клеточной точностью — 49 баллов
\item Отображается каждый участник группы с 1/10 * клеточной точностью — 70 баллов
\end{itemize}

Прочее:
\begin{itemize}
    \item Отображение участников других групп (баллы за отображение одной группы) — 
\end{itemize}

\solutionSection

С помощью перевода из глобальных координат в локальные (относительно карты), был реализован алгоритм который с точностью в 1/10 клетки отображает положение группы на карте. Алгоритм заключается в линейном преобразовании координат, относительно верхней левой и нижней правой точек. На малых расстояниях (между крайними точками) он позволяет получать достаточно высокую точность.

\putImgWOCaption{10cm}{6}

\begin{center}
Рис.4.5. Визуализация AR-браузером положения одной экскурсионной группы и ее маршрута
\end{center}

\markSection

\begin{itemize}
\item Отображается каждый участник группы с 1/10 * клеточной точностью – 70 баллов
\end{itemize}

В приложении есть настройки которые отображают расположение участников других групп.

\putImgWOCaption{10cm}{7}

\begin{center}
Рис.4.6. Визуализация AR-браузером положения нескольких экскурсионных групп и их   маршрутов следования 
\end{center}

\subsubsection*{Приложение к задаче 4.1 Часть 2. Код-основа для добавления на виртуальную карту сведений о местонахождении групп экскурсантов.}

\inputminted[fontsize=\footnotesize, linenos]{csharp}{final/command_tour/ar/task_04/source2.cs}

\subsubsection*{Задача 4.2. Разработка AR-навигатора.}

Надо создать модуль приложения, отвечающий за отображение моделей достопримечательностей, а также дополнительных объектов взаимодействия. Модели отображаются в “реальном” мире, с учетом положения устройства.

\markSection

\begin{itemize}
\item Отображение различных моделей в режиме AR-браузера (т.е. в разных местах открываются разные модели) – 16 баллов.
\item 3Д модели расположены на расстоянии примерно равном расстоянию до объекта на карте – 10 баллов.
\end{itemize}

На выбор отображение моделей на экране:
\begin{itemize}
    \item Объекты располагаются на экране вне зависимости от положения устройства по центру — 12 баллов.
    \item Объекты располагаются примерно на плоскости земли (с использованием датчиков акселерометра и/или магнитометра и/или гироскопа) — 24 балла.
    \item Предыдущий пункт + модель отображается только тогда, когда в её сторону направлена камера — 36 баллов.
\end{itemize}

\solutionSection

Реализация отображает разные модели. Так же модели отображаются на расстоянии примерно равном расстоянию до объекта. 

Модель можно увидеть только если место находится в направлении поля зрения камеры пользователя.

\putImgWOCaption{8cm}{8}

\begin{center}
Рис.4.7. Работа AR-браузера в режиме AR-навигатора и поиска артефактов
\end{center}

\markSection

\begin{itemize}
    \item Отображаются различные модели в режиме AR-браузера (т.е. в разных местах открываются разные модели) — 16 баллов.
    \item 3Д модели расположены на расстоянии примерно равном расстоянию до объекта на карте — 10 баллов.
    \item Модели располагаются примерно на плоскости земли (с использованием датчиков акселерометра и/или магнитометра и/или гироскопа). + модель отображается только тогда, когда в её сторону направлена камера — 36 баллов.
\end{itemize}

\subsubsection*{Приложение к задаче 4.2 Код-основа AR-навигатора}

\inputminted[fontsize=\footnotesize, linenos]{csharp}{final/command_tour/ar/task_04/source3-1.cs}
\inputminted[fontsize=\footnotesize, linenos]{csharp}{final/command_tour/ar/task_04/source3-2.cs}


\subsubsection*{Примечания}

Для решения использованы предоставленные организаторами файлы, облегчающие работу с сетью — \url{http://bit.ly/ntiar-network}. Помимо этого, были использованы следующие алгоритмы:
\begin{itemize}
    \item Перевод из глобальной в локальную систему координат — \url{http://bit.ly/ntiar-geo}
    \item Определение расстояния между двумя точками на сфере — \url{http://bit.ly/ntiar-location-helper}
\end{itemize}