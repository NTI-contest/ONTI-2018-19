\subsubsection*{Приложение к задаче 4.1 Часть 1. Код-основа для формирования динамической AR-карты}

\inputminted[fontsize=\footnotesize, linenos]{csharp}{final/command_tour/ar/task_04/source.cs}

Разработано меню, в котором можно посмотреть маршруты прохода других групп. На изображении приведен пример работы приложения для квеста в котором участвуют 4 группы. Можно посмотреть маршрут каждой из них.

\putImgWOCaption{8cm}{5}

\begin{center}
Рис.4.4. Меню для выбора визуализации AR-браузером маршрутов и местонахождения для нескольких экскурсионных групп. 
\end{center}

\markSection

Меню для выбора маршрута, который нужно показать — 10 баллов.

Графы прохождения других групп — 35 баллов.

\subsubsection*{Часть 2.  Добавление сведений о местонахождении экскурсионных групп}

Необходимо добавить в приложение фиксацию текущего положения участника, а также отображать положение своей и других групп с определенной точностью.

\markSection

На выбор отображение групп:
Группа отображается одной точкой — 35 баллов
Отображается каждый участник группы с клеточной точностью — 49 баллов
Отображается каждый участник группы с 1/10 * клеточной точностью — 70 баллов
Прочее:
Отображение участников других групп (баллы за отображение одной группы) — 

\solutionSection

С помощью перевода из глобальных координат в локальные (относительно карты), был реализован алгоритм который с точностью в 1/10 клетки отображает положение группы на карте. Алгоритм заключается в линейном преобразовании координат, относительно верхней левой и нижней правой точек. На малых расстояниях (между крайними точками) он позволяет получать достаточно высокую точность.

\putImgWOCaption{8cm}{6}


Рис.4.5. Визуализация AR-браузером положения одной экскурсионной группы и ее маршрута
Оценка
Отображается каждый участник группы с 1/10 * клеточной точностью – 70 баллов
В приложении есть настройки которые отображают расположение участников других групп.

\putImgWOCaption{8cm}{7}


Рис.4.6. Визуализация AR-браузером положения нескольких экскурсионных групп и их   маршрутов следования 

Приложение к задаче 4.1 Часть 2. Код-основа для добавления на виртуальную карту сведений о местонахождении групп экскурсантов.


Задача 4.2. Разработка AR-навигатора.
Надо создать модуль приложения, отвечающий за отображение моделей достопримечательностей, а также дополнительных объектов взаимодействия. Модели отображаются в “реальном” мире, с учетом положения устройства.

Критерии:
Отображение различных моделей в режиме AR-браузера (т.е. в разных местах открываются разные модели) – 16 баллов.
3Д модели расположены на расстоянии примерно равном расстоянию до объекта на карте – 10 баллов.
На выбор отображение моделей на экране:
Объекты располагаются на экране вне зависимости от положения устройства по центру — 12 баллов.
Объекты располагаются примерно на плоскости земли (с использованием датчиков акселерометра и/или магнитометра и/или гироскопа) — 24 балла.
Предыдущий пункт + модель отображается только тогда, когда в её сторону направлена камера — 36 баллов.

Решение
Реализация отображает разные модели. Так же модели отображаются на расстоянии примерно равном расстоянию до объекта. 
Модель можно увидеть только если место находится в направлении поля зрения камеры пользователя.

\putImgWOCaption{8cm}{8}


Рис.4.7. Работа AR-браузера в режиме AR-навигатора и поиска артефактов
Оценка
Отображаются различные модели в режиме AR-браузера (т.е. в разных местах открываются разные модели) — 16 баллов.
3Д модели расположены на расстоянии примерно равном расстоянию до объекта на карте — 10 баллов.
Модели располагаются примерно на плоскости земли (с использованием датчиков акселерометра и/или магнитометра и/или гироскопа). + модель отображается только тогда, когда в её сторону направлена камера — 36 баллов.

Приложение к задаче 4.2 Код-основа AR-навигатора

\inputminted[fontsize=\footnotesize, linenos]{csharp}{final/command_tour/ar/task_04/source3.cs}


\textit{Примечания}

Для решения использованы предоставленные организаторами файлы, облегчающие работу с сетью — \url{http://bit.ly/ntiar-network}. Помимо этого, были использованы следующие алгоритмы:
\begin{itemize}
    \item Перевод из глобальной в локальную систему координат — \url{http://bit.ly/ntiar-geo}
    \item Определение расстояния между двумя точками на сфере — \url{http://bit.ly/ntiar-location-helper}
\end{itemize}