\assignementTitle{Создание 3D-моделей и их стилизация}{140}{}

Разработать 3Д-модели, имеющихся на карте объектов (достопримечательностей, виртуальных артефактов). Модели должны иметь сходство с реальными объектами, но быть стилизованы в соответствии легендой и тематикой цифровой территории. Вне зависимости от выбора типа проекта, необходимо предусмотреть наличие в нем артефактов: цифровых объектов, с которыми посетители взаимодействую находясь в той или иной точке территории.

Необходимо создать 10 и более низкополигональных стилизованных 3D-моделей достопримечательностей Иркутска. Команда должна определиться со стилем моделей и представить их в общей стилистике путем подбора гармоничных форм и цветовых сочетаний. Все модели должны иметь не более 3 тысяч полигонов и экспортированы в Unity.

\subsubsection*{1. Планирование}

Определение общего стиля, создание mood board'a, распределение времени на этапы, выбор достопримечательностей города для создания 3D-моделей зданий и виртуального артефакта

\putTwoImg{8cm}{1}{8cm}{2}

\putTwoImg{8cm}{3}{8cm}{4}

\putTwoImg{8cm}{5}{8cm}{6}

\putTwoImg{8cm}{7}{8cm}{8}

\putTwoImg{8cm}{9}{8cm}{10}

\subsection*{2. Создание моделей}

Создание моделей осуществляется по описанному ниже алгоритму и имеет общий подход в оценке выполнения.

Создание основных форм моделей. Обязательно наличие сходства созданной модели с оригиналом по форме. Добавление деталей оригинала для большего сходства, не превышая порог количества полигонов

\putImgWOCaption{8cm}{11}

\putImgWOCaption{8cm}{12}

\underline{Условия:} Каждая модель оценивается индивидуально, в общей сложности 11 моделей (10 зданий + артефакт), в качестве множителя выступает их количество.

\begin{table}[H]
    \begin{center}
        \begin{tabular}{|p{8cm}|c|c|c|}
            \hline
            Критерий оценивания	& Множитель	& Балл	& Итого \\
            \hline
            Детализация модели &	11 &	0.7 &	7.7 \\
            \hline
            Сходство с реальным объектом	& 11 &	0.7 &	7.7\\
            \hline 
            Отсутствие перевернутых нормалей (артефакт не учитывается) &	10	& 0.45 &	4.5 \\
            \hline
        \end{tabular} 
    \end {center} 
\end{table}         

\underline{Условия:} Оценивается детализация каждой модели, наличие несбалансированных деталей, которые обладают чрезмерным количеством полигонов; сходство форм с оригиналом, наличие узнаваемости. 

Просматриваются нормали на наличие ориентированных вовнутрь объекта. При наличии 2х и более перевернутых нормалей этот пункт оценивается в 0 баллов.

\textbf{Всего: 19.9 балла.}

\subsection*{3. Оптимизация моделей}

Предоставление модели с правильной топологией, оптимизация количества полигонов путем удаления ненужных.

\putImgWOCaption{8cm}{13}

\putImgWOCaption{12cm}{14}

\begin{table}[H]
    \begin{center}
        \begin{tabular}{|c|c|c|c|}
            \hline
            Критерий оценивания &	Множитель & Балл & Итого \\
            \hline
            Отсутствие неправильных полигонов & 11 & 0.6 & 6.6 \\
            \hline
            Количество полигонов не должно превышать 3k & 0.9 & 11 & 9.9\\
            \hline
        \end{tabular} 
    \end {center} 
\end{table}         
    

\underline{Условия:} Оценивается количество полигонов и их качество. Наличие полигонов с очень острыми углами (менее 30 градусов) считается недочетом для 4-х угольных полигонов; наличие 5-угольных полигонов и более также считается недочетом. Наличие несвязанных с объектом вершин или ребер, наличие двух и более полигонов на одних координатах считаются недочетом. 

Каждый из недочетов уменьшает балл на 0.2 для неправильных полигонов.

Если количество полигонов превышает 3 тысячи, но не превышает 3100 полигонов, ставится оценка равная половине оценочного балла за количество полигонов.

\textbf{Всего: 16.5 балла.}

\subsection*{4. Создание UV развёртки. }

Необходимо «развернуть» созданные модели для последующего наложения на них текстур. Важно создать правильную развертку, чтобы избежать искажений текстуры.

\putImgWOCaption{12cm}{15}

\begin{table}[H]
    \begin{center}
        \begin{tabular}{|c|c|c|c|}
            \hline
            Критерий оценивания &	Множитель &	Балл &	Итого \\
            \hline
            Наличие UV-развертки &	11 &	0.6	& 6.6 \\       
            \hline
        \end{tabular} 
    \end {center} 
\end{table} 

\underline{Условия:}Если созданная развертка чрезмерно искажает текстуру из-за неверного построения, такая развертка оценивается в половину оценочного балла.

\textbf{Всего: 6.6 балла.}

\subsection*{5. Создание текстур}

Наложение текстур на модель путем создания текстур из фотографий, либо с помощью рисования (Texture Paint).

\putTwoImg{4cm}{16}{4cm}{17}

\putImgWOCaption{12cm}{18}

\begin{table}[H]
    \begin{center}
        \begin{tabular}{|c|c|c|c|}
            \hline
            Критерий оценивания &	Множитель &	Балл &	Итого \\
            \hline
            Наличие UV-развертки &	11 &	2	& 22 \\       
            \hline
        \end{tabular} 
    \end {center} 
\end{table} 

\underline{Условия:} Текстура наложена ровно, материалы не заходят за границы друг друга и размещены на одной карте. При наличии материалов, но отсутствии текстуры - 0 баллов.

\textbf{Всего: 22 балла.}

\subsection*{6. Экспорт моделей}

\putImgWOCaption{12cm}{19}

\putImgWOCaption{8cm}{20}

\putImgWOCaption{12cm}{21}

\putImgWOCaption{12cm}{22}

\begin{table}[H]
    \begin{center}
        \begin{tabular}{|c|c|c|c|}
            \hline
            Критерий оценивания &	Множитель &	Балл &	Итого \\
            \hline
            Успешный экспорт в Unity & 15 & 1 &	15 \\      
            \hline
        \end{tabular} 
    \end {center} 
\end{table} 

\underline{Условия:} Модель экспортирована в формате .fbx или .obj, в Unity экспортирована текстура. Модель не имеет лишних неиспользуемых материалов в импорте, лампы и камеры. Модель находится в подпапке и имеет правильную размерность. Если из модели создан префаб - бонус в 7 пункте.

\subsection*{7. Общая оценка проекта}

Оценивание всех моделей, их сочетаемости по цветам в соответствии с правилами цветоведения; оценка общности их форм и их читаемости; оценка дополнительных моделей и текстур в случае предоставления их участниками. 

\begin{table}[H]
    \begin{center}
        \begin{tabular}{|c|c|c|c|}
            \hline
            Критерий оценивания &	Множитель &	Балл &	Итого \\
            \hline
            Бонусы за доп.модели и текстуры  & 20 & 1 & 20 \\
            \hline
            Единство стиля & 40	& 1 & 40 \\   
            \hline
        \end{tabular} 
    \end {center} 
\end{table}



\textbf{Всего: 60 баллов.}

\textbf{Общая сумма: 140 баллов.}

