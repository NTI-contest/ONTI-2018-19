\assignementTitle{Мобильное приложение – AR-браузер}{400}{}

Задача предполагает разработку и создание мобильного приложения AR-браузера, позволяющего рассматривать маршрут в виде виртуальной карты с отображением всех достопримечательностей, маршрутов, местонахождения экскурсионных групп на маршруте, а также работать с AR-навигатором, помогающим ориентироваться туристу, находящемуся на реальной территории, находить и собирать артефакты тематической экскурсии. Ее можно разделить на две крупных подзадачи: 
\begin{enumerate}
    \item[4.1] \textbf{Виртуальная AR-карта тематической экскурсии} (200 баллов). В эту задачу входит разработка программного компонента мобильного приложения для визуализации интерактивной карты территории, на которой объекты городской инфраструктуры представлены при помощи технологий дополненной реальности в виде трехмерных моделей, с возможностью отображения маршрута, соответствующего номеру группы, всех возможных маршрутов для заданного числа групп (от 3 до 7), посетителей территории, находящихся на ней в настоящий момент.
    \item[4.2] \textbf{AR-навигатор по территории проведения экскурсии} (200 баллов). Разработать программный компонент мобильного приложения - AR-навигатор, который визуализирует информацию в виде объектов дополненной реальности (артефактов) при нахождении в соответствующих точках гео-локации на территории.
\end{enumerate}

\subsubsection*{Виртуальная AR-карта тематической экскурсии (200 баллов)}

Необходимо создать модуль приложения, в котором будет отображаться сетка с динамически отображающим на ней маршрутом как своей группы, так и других. Помимо маршрута, есть возможность отображения положения экскурсионных групп. Для определения “своей” группы используется авторизация, связанная с сайтом. Сетка располагается между двумя-четырьмя маркерами, находящимися в поле зрения камеры.

\subsubsection*{Часть 1. Динамическая AR-карта с достопримечательностями и маршрутами}

\markSection

Динамическая сетка, Маркеры (2-4) задают границы сетки-карты. В зависимости от их расположения  меняется размер визуализации участка территории и отображаемых достопримечательностей.
\begin{itemize}
    \item Сетка подстраивается под маркеры – 12 баллов. 
    \item Количество ячеек подстраивается под JSON – 8 баллов.
    \item В соответвенных точках маршрута подсвечиваются соответственные артефакты. – 35 баллов
\end{itemize}

Построение маршрутов на выбор:
\begin{itemize} 
    \item	Маршрут статичный (векторы рисует на программа, а это заложенное заранее изображение). – 10.5 баллов.
    \item	Маршрут строится в приложении, т.е. линия между точками генерируется – 35 баллов.
\end{itemize}

Прочее:
\begin{itemize}
    \item	Меню для выбора маршрута, который нужно показать – 10 баллов.
    \item	Графы прохождения других групп – 35 баллов.
\end{itemize}

\solutionSection
Было разработано приложение с следующим интерфейсом:
\putImgWOCaption{10cm}{1}

\begin{center}
    Рис.4.1. Интерфейс мобильного приложения AR-браузера
\end{center}

Реализованное приложение перестраивает сетку под положение маркеров. Возможно задавать разные варианты JSON и приложение будет перестраивать сетку.
\putImgWOCaption{10cm}{2} 

\begin{center}
    Рис.4.2. Визуализация динамической виртуальной AR-карты тематической экскурсии
\end{center}

\markSection

\begin{itemize}
    \item	Сетка подстраивается под маркеры – 12 баллов. 
    \item	Количество ячеек подстраивается под JSON – 8 баллов.
\end{itemize}

Из файла JSON происходит считывание и в нужных местах сетки подсвечиваются артефакты которые нужно собрать. Так же между артефактами натягиваются вектора-указатели показывающие в каком направлении нужно идти определенной группе.

\putImgWOCaption{10cm}{3} 
 
\begin{center}
    Рис.4.3. Визуализация в AR-браузере маршрута экскурсии на виртуальной карте
\end{center}

\markSection

\begin{itemize}
    \item	В соответвенных точках маршрута подсвечиваются соответственные 
артефакты. – 35 баллов
    \item	Маршрут строится в приложении, т.е. линия между точками 
генерируется – 35 баллов.
\end{itemize}

\subsubsection*{Приложение к задаче 4.1 Часть 1. Код-основа для формирования динамической AR-карты}

\inputminted[fontsize=\footnotesize, linenos]{csharp}{final/command_tour/ar/task_04/source1-1.cs}
\inputminted[fontsize=\footnotesize, linenos]{csharp}{final/command_tour/ar/task_04/source1-2.cs}
\inputminted[fontsize=\footnotesize, linenos]{csharp}{final/command_tour/ar/task_04/source1-3.cs}

Разработано меню, в котором можно посмотреть маршруты прохода других групп. На изображении приведен пример работы приложения для квеста в котором участвуют 4 группы. Можно посмотреть маршрут каждой из них.

\putImgWOCaption{10cm}{5}

\begin{center}
Рис.4.4. Меню для выбора визуализации AR-браузером маршрутов и местонахождения для нескольких экскурсионных групп. 
\end{center}

\markSection

\begin{itemize}
\item Меню для выбора маршрута, который нужно показать — 10 баллов.
\item Графы прохождения других групп — 35 баллов.
\end{itemize}
