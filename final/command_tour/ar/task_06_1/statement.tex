\assignementTitle{Демонстрация работоспособности мобильного приложения, защита готового проекта}{135}{}

Заключительной задачей финала ОНТИ по профилю “Разработка приложений виртуальной и дополненной реальности. Технологии дополненной реальности” является демонстрация работоспособности готового мобильного приложения. Решение всех задач оценивалось ранее автономно, каждой по отдельности. Теперь участники должны продемонстрировать все реализованные функциональные возможности мобильного приложения в комплексе. Кроме качества решения всего комплекса задач, оценивается сама презентация (питчинг): умение за отведенное ограниченное время представить  сильные стороны продукта. 

Критерии оценки и рекомендации.
\begin{enumerate}
    \item Продемонстрировано, как пользователь будет взаимодействовать с продуктом от посадочной страницы до визуализации виртуальных артефактов  на маршруте:
    \begin{itemize}
        \item продемонстрирована работа всех компонентов веб-ресурса тематической экскурсии - 10 баллов
        \item продемонстрирована форма регистрации экскурсионной группы с последующим получением идентификатора группы для мобильного приложения  - 25 баллов
        \begin{tabular}{|p{7cm}|l|}
            \hline
            Критерий & Оценка \\
            \hline
            продемонстрирована форма регистрации & 5 \\
            \hline
            продемонстрировано, что форма регистрации связана с сервером & 10 \\
            \hline
            в результате регистрации генерируется идентификатор экскурсионной группы для использования в мобильном приложении & 10 \\
            \hline
        \end{tabular}
        \item продемонстрирована возможность загрузки AR-браузера через QR-код на мобильное устройство - 15 баллов
        \begin{tabular}{|p{7cm}|l|}
            \hline
            Критерий & Оценка \\
            \hline
            на веб-ресурсе размещен QR-код & 5 \\
            \hline
            QR-код стилизован под тематическую экскурсию & 5 \\
            \hline
            по QR-коду можно загрузить AR-браузер тематической экскурсии & 5 \\
            \hline
            загрузка приложения осуществляется не по QR-коду, а каким-либо иным образом & 3 \\
            \hline
        \end{tabular}
        \item продемонстрирована работа AR-браузера в режиме виртуальной карты -35 баллов
        \begin{tabular}{|p{7cm}|l|}
            \hline
            Критерий & Оценка \\
            \hline
            виртуальная карта генерируется при распознавании маркеров, маркеры задают размер карты и отображаемых на ней объектов & 5 \\
            \hline
            на AR-карте визуализируется маршрут конкретной группы в соответствии с идентификатором & 5 \\
            \hline
            есть возможность отобразить все маршруты, артефакты маршрута подсвечиваются & 5 \\
            \hline
            продемонстрирована возможность отображения одной группы на маршруте (использован сервис подмены гео-координат) & 10 \\
            \hline
            продемонстрирована возможность отображения разного количества групп на маршруте (использован сервис подмены гео-координат), в зависимости от выбора пользователя AR-браузера & 10 \\
            \hline
        \end{tabular}
        \item продемонстрирована работа AR-браузера в режиме AR-навигатора -25 баллов.
        \begin{tabular}{|p{7cm}|l|}
            \hline
            Критерий & Оценка \\
            \hline
            AR-браузер определяет местонахождение пользователя на выбранном участке территории & 5 \\
            \hline 
            AR-браузер определяет местонахождение пользователя на выбранном участке территории в соответствии с этим на экран мобильного устройства генерируется изображение артефакта & 10 \\
            \hline
            Есть возможность собирать артефакты и накапливать баллы за прохождение тематической экскурсии & 10 \\
            \hline
        \end{tabular}
    \end{itemize}
    \item Качество презентации продукта
    \begin{tabular}{|p{7cm}|p{4cm}|}
        \hline
        Критерий & Оценка \\
        \hline
        Продемонстрированы все функции продукта, нет превышение регламента (уложились за три минуты) & 15 \\
        \hline
        Продемонстрированы все функции продукта, было превышение регламента демонстрации & -1 балл за каждую лишнюю минуту \\
        \hline
        Ответы на вопросы экспертов & 10 - уверенное владение материалом
        
        -1 балл за отсутствие ответа \\
        \hline
    \end{tabular}
\end{enumerate}
    
    