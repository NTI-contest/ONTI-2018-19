\assignementTitle{Проектирование и расчет маршрутов}{135}{}

Пятая задача проекта включает в себя проектирование и расчет нескольких моделей маршрутов для мультигрупп с разными вариациями локаций. Возможность выгрузки маршрута в  AR-браузер для отображения маршрута конкретной группы или всех групп на виртуальной AR-карте.

Задачу можно разбить на части:
\begin{enumerate}
    \item Для разного кол-ва участвующих в игре групп (для 3, для 4, для 5, для 6, для 7) выберите достопримечательности, которые нужно пройти. Обновите JSON (\url{https://drive.google.com/open?id=1cnqAJA9O1woMV3vyQJy0ZMEhu9VDaVEi})
    \item Для разного кол-ва участвующих в игре групп (для 3, для 4, для 5, для 6, для 7) с помощью сервиса \url{http://table.nti-ar.ru} составьте граф следования для каждой группы. Выгрузите JSON.
\end{enumerate}

\markSection
\begin{itemize}
    \item 	1 конфигурация графа - 15.625 баллов. Максимально кол-во графов за все конфигурации 125 баллов. Т.е. если каждая разработанная конфигурация 15.625 баллов, то для максимального оценивания достаточно 8 конфигураций. Если же максимальный балл каждую конфигурацию набрать не получается, то увеличением числа конфигураций можно набрать максимальную оценку. 
    \item Валидно составлен JSON по формату 3 (пример - \url{https://drive.google.com/open?id=13WOLWes08m5z1rJxOwQE9dz-UtUqlm-j}  в приложении) - 10 баллов
\end{itemize}

\solutionSection

Реализовано 4 конфигурации, каждая из которых примерно оценивается в 10, 720 баллов. (формат 3: \url{https://drive.google.com/open?id=13WOLWes08m5z1rJxOwQE9dz-UtUqlm-j} ). При разработке еще 8 таких же конфигураций можно получить максимальный балл. 

Фрагмент одной из разработанных конфигураций (входная для отправки на сервер):
\url{https://drive.google.com/open?id=1fm3D9w6QDfAB8LcO2xAEBl35qAFxQs4f}

\inputminted[fontsize=\footnotesize, linenos]{json}{final/command_tour/ar/task_06/source_1.json}

Фрагмент - конфигурация выходная с сервера:
\url{https://drive.google.com/open?id=1-GMPWVK7We9NQMHlBFJuSe6L7tKC0K53}

\inputminted[fontsize=\footnotesize, linenos]{json}{final/command_tour/ar/task_06/source_2.json}

Скриншоты работы в сервисе : 

\putImgWOCaption{12cm}{1}

\textbf{Оценка:}
\begin{itemize}
    \item	Одна конфигурация оценивается в 10, 720 баллов. При наличии 14 конфигураций работа получает 125 баллов.
    \item	Предоставленный JSON валиден, т.е. проходит проверку в JSON-валидаторе(\url{https://jsonlint.com}) и формат совпадает с форматом-3 (\url{https://drive.google.com/open?id=13WOLWes08m5z1rJxOwQE9dz-UtUqlm-j} ) - 10 баллов
\end{itemize}

\textit{Примечаниe}

Для автоматической оценки представляемых участниками конфигураций маршрутов был разработан и использован  специальный веб-сервис \url{http://table.nti-ar.ru}.

При загрузке конфигурации в веб сервис производится симуляция обхода графа, где определяются общее время, за которое все группы обойдут все точки, и итоговое время ожидания для каждой точки. Исходя из полученных данных, вычисляется итоговый коэффициент.

Итоговый коэффициент за граф вычисляется по формуле: 100 - (O$\cdot$0.4 + E$\cdot$0.6), где O - коэффициент загруженности и E - коэффициент неэффективности использования графа. Т.к. коэффициент принимает значение от 0 до 100, то его значение приводится к диапазону от 0 до максимального балла за граф посредством деления.

Коэффициент O находится следующим образом: для каждой точки находится время, которое другие группы простояли в очереди в данной точке. Затем это время делится на максимальное время, на которое точка может быть загружена. Так находится коэффициент перегруженности точки. Затем находится сумма данных коэффициентов и умножается на 100 / n, где n - кол-во точек.

Максимальное время, которое группы могут простоять в точке, равно сумме времен ожидания групп, при условии что все они пришли в точку в один момент. Т.е. для первой группы время ожидания равно 0, для следующей времени работы точки, для следующей времени работы точки умноженному на 2 и т.д.

Коэффициент E находится как 1 - (время работы точки) $\cdot$ (кол-во точек) / (общее время работы маршрута).
Для примера решения ответ 64 получается следующим образом: находится итоговый коэффициент равный 36, получившийся по формуле из коэффициентов O = 0 и F = 60. Первый коэффициент показывает, что ни в какой точке не находились несколько групп сразу. Второй коэффициент показывает насколько долго в точках не было ни одной группы.  На рисунке показаны коэффициенты составляющие F для каждой точки. Отсюда видно, что наименьшее значение получилось для точек “merchant house” и “manor” - это значит что данные точки меньше всего простаивают по времени.

\putImgWOCaption{6cm}{2}

Далее для получения реальных баллов 64 умножается на 0.15625 для перевода во вторичные баллы. В результате выходит 10, 720 баллов. 
