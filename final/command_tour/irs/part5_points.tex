
\begin{enumerate}
    \item В качестве задачи для симулятора участникам необходимо выполнить следующее:
    \begin{enumerate}
        \item Робот устанавливается в модели логистического центра в заранее неизвестном
        секторе. При этом структура логистического центра известна заранее.
        Задача робота локализоваться и доехать в точку финиша, чьи
        координаты заданы изображением ARTag маркера, заранее считанным с камеры реального
        устройства. На финише необходимо вывести на экран слово \texttt{finish}.

        \textit{Входные данные:} через файл \texttt{input.txt} управляющей программе передаются:
        \begin{enumerate}
            \item[-] В первой строке $19200$, разделенных пробелом, целых чисел $P_{1,i}$
            ($0 \le P_{1,i} \le 2^{24}$) --- изображение ARTag маркера;
        \end{enumerate}
        Каждое число в маркере --- точка, закодированная в формате RGB, т.е. строка с изображением
        маркера эквивалентна снимку разрешением $160 \times 120$ точек. Один маркер кодирует
        координаты для первого робота, в данной ситуации являющегося единственным.

        \textit{Ожидаемый результат:} После запуска программы робот перемещается в сектор финиша.
        После остановки на экран робота выведено \texttt{finish}.

        \item Имя файла с управляющей программой для проверки решения в симуляторе: \texttt{sim\_part5.qrs}.
    \end{enumerate}
    \item Командам будет предоставлено две попытки на демонстрацию решения задачи на реальном
    роботе.
    \item Все попытки осуществляются 10 марта.
    \item За 15 мин до сдачи в карантин судья определяет сектора старта
        и направления в секторах старта двух роботов. Данные значения команда должна внести в программу перед тем,
        как сдать робота в карантин.
    \item После сдачи в карантин судья определяет сектор старта
        и направление в секторе старта для оставшегося  робота.
    \item В начале соревновалельного дня определяются сектора сервисного обслуживания (сектора, из которых необходимо сканировать ARTag метки) и
    направление расположения ARTag меток ($0$ - маркеры находятся на ``верхнем'' стеллаже, $1$ - на ``правом'' стеллаже
    и т.д. по часовой стрелке). Данные значения команда должна внести в программу самостоятельно.
    \item Сектора сервисного обслуживания являются одинаковыми для всех попыток.
    \item Сектора сервисного обслуживания: первый --- ``$(2,1)~3$'', второй --- ``$(2,5)~3$'', третий --- ``$(5,2)~1$''.
    \item Сектора старта могут быть различными в разных попытках.
    \item Правила именования файлов с программой управления:
    \begin{enumerate}
        \item для первой попытки: \texttt{part5\_1.js}.
        \item для второй попытки: \texttt{part5\_2.js}.
    \end{enumerate}
    \item Максимальное время выполнения одной попытки - 5 минут.
    \item Баллы за решение задач этапа:
        \begin{enumerate}
            \item \textbf{Симулятор:}
            \begin{enumerate}
                \item Робот проехал из точки старта в точку финиша и вывел \texttt{finish} на всех проверочных полигонах --- 8 баллов.
            \end{enumerate}
            \item \textbf{Реальный робот:} Роботы располагаются в секторах старта, задача определить своё местоположение,
                    распознать arTag метки расположенные с секторах сервисного обслуживания, и доехать до финиша.
            \begin{enumerate}
                \item Все роботы определили своё местоположение, остановившись, издали звуковой сигнал и
                        вывели на экран свои координаты в формате $(X;Y)$ и через 10 секунд продолжили движение --- 14 баллов.
                \item Один из роботов смог распознать один из arTag маркеров, издал звуковой сигнал, находясь в секторе
                        сервисного обслуживания, вывел на экран номер робота для которого предназначется данный сектор финиша
                        и его координаты в формате $N~(X,Y)$,
                        где $N$ --- номер робота, и через 10 секунд продолжил движение --- 10 баллов.
                \item Все arTag маркеры были распознаны согласно предыдущему пункту --- 14 баллов.
                \item Один из роботов доехал до сектора финиша и вывел \texttt{finish} --- 10 баллов.
                \item Все роботы достигли соотвествующух секторов финиша и вывели \texttt{finish} --- 12 баллов.
            \end{enumerate}
        \end{enumerate}
    \item За подзадачу начисляется только половина возможных баллов, если не было засчитано решение подзадачи в симуляторе.
    \item Баллы за все попытки в каждой подзадаче суммируются.
    \item Выполнение всех критериев в каждой из двух попыток всех трех подзадач дает дополнительные 4 балла.
    \item Максимальное количество баллов за этап --- 132.
\end{enumerate}