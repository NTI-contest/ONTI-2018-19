Недалёкоое будущее, в автоматических логистических центрах практически нет людей, всю работу
выполняют роботы-погрузчики, которыми управляет интеллектуальное ПО по распределению задач.

Несмотря на отсутствие человеческого фактора, не следует думать, что в таких
центрах никогда не будет проблем. Форс-мажор может произойти в любой момент и система
из роботов и ПО должна уметь справляться с такими ситуациями.

Поэтому для финальной задачи профиля <<Интеллектуальные робототехнические системы>>
предлагается рассмотреть следующий эпизод: в логистическом центре произошла перезагрузка
всех систем, что привело к сбросу информации о местоположениях работающих в данный момент роботов-погрузчиков.

В начале выполнения задания считается, что робототехнические устройства активируются
в каких-то секторах (каждый в своём) логистического центра.
При этом 2 из них имеют информацию о своём местоположении,
а у 3-го была повреждена память и по этому эта информация была потеряна.
Структура логистического центра известна заранее всем устройствам.
Роботы-погрузчики должны переместиться в сектор своей приписки.
Координаты секторов сервисного обслуживания известны заранее.
Информацию о местоположении необходимого сектора приписки можно узнать в
секторе сервисного обслуживания. В нём располагается ARTag маркер, в котором закодирован номер робота и
координаты сектора его приписки. При перемещении роботы не должны сталкиваться, а
также повреждать логистический центр.


Задача участников Олимпиады --- разработать программу управления несколькими робототехническими
устройствами для выполнения задания описанного выше.

\putImgForRef{16cm}{sources/NTI_IRS_2019_field.jpg}
{Полигон для запуска робототехнических устройств на финале Олимпиады НТИ}{fig:NTI-IRS-2019-field}
