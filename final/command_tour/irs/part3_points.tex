
\begin{enumerate}
    \item Командам необоходимо подготовить две задачи для симулятора:
    \begin{enumerate}
        \item В качестве первой задачи участникам необходимо выполнить следующее:
        \begin{enumerate}
            \item Робот устанавливается в модели логистического центра в заранее неизвестном секторе.
                Задача робота проехать из точки старта в точку финиша.

            \textit{Входные данные:} через файл \texttt{input.txt} управляющей программе передаются:
            \begin{enumerate}
                \item В первой строке через пробел --- координаты сектора старта $X_s$, $Y_s$ и направление старта
                    $D_s$ (направление робота в секторе старта от 0 до 3 начиная с направления вверх и дальше по часовой стрелке), $0 \le X_s, Y_s \le 7$;
                \item В второй строке через пробел --- координаты сектора финиша $X_f$, $Y_f$, $0 \le X_f, Y_f \le 7$.
            \end{enumerate}

            \textit{Ожидаемый результат:} После запуска программы робот перемещается в сектор
            финиша по оптимальному пути. Оптимальным является путь, длина строки маршрута, которого наименьшая.
            Для маршрута используется следующая нотация действий без пробелов:
            \begin{enumerate}
                \item $F$ --- проезд в следующий сектор по ходу движения;
                \item $L$ --- поворот налево в данном секторе;
                \item $R$ --- поворот направо в данном секторе.
            \end{enumerate}
            После остановки на экран робота выведен путь в описанной нотации выше.
        \end{enumerate}
        \item В качестве второй задачи:
        \begin{enumerate}
            \item Роботы устанавливаются в модели логистического центра в заранее неизвестном секторе.
            Задача роботов проехать из точки старта в точки финиша.

            \textit{Входные данные:} через файл \texttt{input.txt} управляющей программе передаются:
            \begin{enumerate}
                \item В первой строке через пробел --- координаты сектора старта первого робота $X_{1s}$, $Y_{1s}$,
                        направление старта первого робота $D_{1s}$ (направление робота в секторе старта от 0 до 3
                        начиная с направления вверх и дальше по часовой стрелке) и координаты сектора
                        финиша первого робота $X_{1f}$, $Y_{1f}$,  $0 \le X_{1s}, Y_{1s}, X_{1f}, Y_{1f} \le 7$;
                \item Во второй строке через пробел --- координаты сектора старта второго робота $X_{2s}$, $Y_{2s}$,
                        направление старта второго робота $D_{2s}$ (направление робота в секторе старта от 0 до 3
                        начиная с направления вверх и дальше по часовой стрелке) и координаты сектора
                        финиша вторго робота $X_{2f}$, $Y_{2f}$,  $0 \le X_{2s}, Y_{2s}, X_{2f}, Y_{2f} \le 7$;
                \item В третьей строке через пробел --- координаты сектора старта третьего робота $X_{3s}$, $Y_{3s}$,
                        направление старта третьего робота $D_{3s}$ (направление робота в секторе старта от 0 до 3
                        начиная с направления вверх и дальше по часовой стрелке) и координаты сектора
                        финиша третьего робота $X_{3f}$, $Y_{3f}$,  $0 \le X_{3s}, Y_{3s}, X_{3f}, Y_{3f} \le 7$;
            \end{enumerate}

            \textit{Ожидаемый результат:} После запуска программы происходит вычисление планируемых маршрутов для каждого
                    робота. После вычисления в файл ``output.txt'' записана одна строка следующего вида:
                    $1P_{1}2P_{2}3P_3$, без пробелов, где $1P_{1}$, $P_{2}$ и $P_3$ -- маршруты соответсвующих роботов.
                    Для вывода маршрута используется следующая нотация без пробелов:
                    \begin{itemize}
                        \item $F$ --- проезд в следующий сектор по ходу движения;
                        \item $L$ --- поворот налево в данном секторе;
                        \item $R$ --- поворот направо в данном секторе;
                        \item $S$ --- остановка в данном секторе.
                    \end{itemize}
                    Роботы движутся без столкновений и так, что команды выполняются паралельно
                    и любое действие занимает одинаковое количество времени.
        \end{enumerate}


        \item Правила именования файлов с управляющей программой для проверки решений в симуляторе:
        \begin{enumerate}
            \item Для первой подзадачи: \texttt{sim\_part3\_1.js};
            \item Для второй подзадачи: \texttt{sim\_part3\_2.js}.
        \end{enumerate}

    \end{enumerate}
    \item Команде необходимо будет подготовить решения для двух разных подзадач для реальных роботов. На демонстрацию каждого решения предоставляется 2 попытки.
    \item Все попытки осуществляются 8 марта.
    \item За 15 минут до времени сдачи роботов в карантин для данных подзадач судья определяет сектора старта и финиша, а также направление робота в секторе старта.
            Данные значения команда должна внести в программу перед тем, как сдать робота в карантин.
    \item Секторы старта и финиша могут быть различными в разных попытках.
    \item Правила именования файлов с программой управления:
    \begin{enumerate}
        \item для первой попытки первой подзадачи: \texttt{part3\_1\_1.js};
        \item для второй попытки первой подзадачи: \texttt{part3\_1\_2.js};
        \item для первой попытки второй подзадачи: \texttt{part3\_2\_1.js};
        \item для второй попытки второй подзадачи: \texttt{part3\_2\_2.js}.
    \end{enumerate}
    \item Максимальное время выполнения одной попытки - 5 минут.
    \item Баллы за решение задач этапа:
    \begin{enumerate}
        \item \textbf{Первая задача в симуляторе}: робот проехал из точки старта в точку финиша и вывел на экран маршрут робота
                на всех проверочных полигонах --- 6 баллов.
        \item \textbf{Вторая задача в симуляторе}: В консоль выведены верные маршруты перемещения из точки старта в точку финиша для
                всех роботов на всех проверочных полигонах.
            \begin{enumerate}
                \item Маршруты позволяют роботам доехать из начальных в конечные координаты -- 10 баллов.
                \item В маршруте кажного робота было использовано не более 3х команд $S$  -- 4 балла.
            \end{enumerate}
        \item \textbf{Первая подзадача на реальном роботе:} Робот доехал до сектора финиша,
                остановился и вывел \texttt{finish} --- 8 баллов.
        \item \textbf{Вторая подзадача на реальном роботе:} Роботы располагается с случайных секторах
        робототехнического полигона.
        \begin{enumerate}
            \item Первый робот доехал до сектора финиша не задев остальных роботов, остановился и вывел \texttt{finish} --- 10 баллов.
            \item Второй робот доехал до сектора финиша не задев остальных роботов, остановился и вывел \texttt{finish} --- 12 баллов.
            \item Третий робот доехал до сектора финиша не задев остальных роботов, остановился и вывел \texttt{finish} --- 12 баллов.
        \end{enumerate}
    \end{enumerate}
    \item За первую подзадачу начисляется только половина возможных баллов, если не было засчитано решение первой задачи в симуляторе
    \item За вторую подзадачу начисляется только половина возможных баллов, если не были засчитаны решения за все задачи в симуляторе.
    \item Баллы за все попытки в каждой подзадаче суммируются.
    \item Выполнение всех критериев в каждой из двух попыток всех трех подзадач дает дополнительные 4 балла.
    \item Максимальное количество баллов за этап --- 108.
\end{enumerate}