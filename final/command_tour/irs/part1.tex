% \noindent
% \textbf{24-25 февраля}

\textit{Задача:} три робототехнических устройства располагаются в модели логистического центра.
Им необходимо обменяться данными о наличии или отсутствии стен слева от робота, перед собой и справа от робота,
где 1 --- присутствует стена, 0 --- отсутствует. Выводить показация следует на экран всех роботов непрерывно.
Данные, принятые с каждого робота, необходимо вывести на разных строках при этом показания с одного робота
следует выводить через пробел в порядке указаном выше.

\putImgForRef{10cm}{final/command_tour/irs/sources/many_robots_ex}
{Пример расположения роботов и показания выведенные на экран}{fig:part1_ex}

\textit{Включая содержательные задачи:}
\begin{itemize}
    \item Нахождение порогового значения для датчиков расстояния;
    \item Реализация коммуникации между робототехническими устройствами.
\end{itemize}

