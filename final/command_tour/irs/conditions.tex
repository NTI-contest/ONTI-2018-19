\begin{enumerate}
    \item Из полученного набора датчиков команды могут выбирать те, с помощью которых,
            по мнению участников, можно решить задачу наиболее эффективным способом.
    \item Команды могут вносить любые изменения в мобильные наземные платформы.
    \item Участники во время командного этапа финального тура могут использовать интернет и заранее подготовленные библиотеки для решения задачи.
    \item Участники не могут использовать помощь тренера, сопровождающего лица или привлекать третьих лиц для решения задачи.
    \item Финальная задача формулируется участникам в первый день финального тура, но участники выполняют решение
            задачи поэтапно. Критерии прохождения каждого этапа формулируются для каждого дня финального тура.
            За подзадачи, решенные в конкретном этапе начисляются баллы. Баллы за подзадачи можно получить только в день,
            закрепленный за конкретным этапом.
    \item Некоторые подзадачи строго требуют выполнение каких-то предыдущих подзадач. Выполнение данных подзадач
            без выполнения предыдущих допускается, однако данная попытка будет оценена в 0 баллов.
    \item При выполнении подзадачи в виртуальной среде полное количество баллов можно получить лишь при первой попытке.
            При второй попытке сдать подзадачу может быть начислено лишь половина баллов за данную задачу.
            При последующих попытках баллы не начисляются.
    \item Во время рабочего времени команды могут проводить испытания на полигоне. Количество 
            подходов, которое может сделать команда может быть ограничено в зависимости ограничений,
            накладываемых расписанием финального этапа Олимпиады.
    \item Испытания на полигоне должны осуществляться так, чтобы не мешать другим командам, 
            проводящим в это время свои испытания на полигоне. Для этого всем командам может
            быть назначено ограничение по времени, которое они могут тратить на одно испытание. После
            истечения этого времени, команда должна дать возможность проводить испытания следующей команде.
    \item Часть подзадач необходимо будет решить в симуляторе TRIK Studio: команда получает 3 
            тестовых виртуальных полигона с соответствующими наборами входные данных для подготовки
            решения, в то время как приемка решения происходит на расширенном наборе полигонов для
            проверки универсальности управляющей программы. Начисление баллов за подзадачи может
            происходить только в тот этап, в котором данные подзадачи сформулированы.
    \item У команды есть не более двух попыток для сдачи решения подзадач в симуляторе:
    \begin{enumerate}
        \item Решения принимаются на проверку до истечения первых 6 часов работы в
                соответствующий соревновательный день. Может меняться в зависимости от дня.
        \item До истечения 6 часов, команда должна загрузить свое решение на \textit{Google Drive}
                в каталог, доступ к которому участники получат в начале дня.
                Участники команды ответственны за то, что ссылка на каталог с их решениями не
                попадет участникам других команд.
        \item До истечения указанного времени команды могут изменять файл с решением
                сколько угодно раз. Проверяться будет всегда только последняя доступная версия.
        \item Если решение отправлено на проверку в течение первых 4 часов работы
                в соответствующий соревновательный день, то команда имеет право на вторую попытку,
                если результаты проверки решения ее не устраивают.
        \item Если команда хочет воспользоваться правом проверки решения до истечения
                4 часов, то она должна загрузить в каталог на \textit{Google Drive} программу со своим
                решением и сообщить об этом судьям.
    \end{enumerate}
    \item Часть подзадач для реальных роботов может быть запрещена к приемке без успешного прохождения $60\%$ всех тестов,
        предназначенных для проверки решения соответствующей подзадачи в симуляторе.
    \item Каждый день финального тура за 2 часа (может варьироваться в зависимости от расписания) до конца выделенного
            рабочего времени команды должны сдать роботов в зону карантина. Время сдачи роботов в карантин может изменяться
            и зависит от количества команд и сложности подзадач, принимаемых в конкретный этап.
    \item Перед сдачей робота в карантин команды должны загрузить на роботов управляющие
            программы, подготовленные для демонстрации решения задачи, а также ее копию в
            \textit{Google Drive} в каталог, доступ к которому участники получат в начале соревновательного дня.
            Без программы, загруженной в каталог \textit{Google Drive}, команды
            не допускаются до проверки решения на реальном роботе.
    \item После момента, когда все роботы сданы в карантин, судьи по одной вызывают команды
            для приемки решения подзадач, закрепленных за этапом конкретного дня финального тура.
    \item Может быть предусмотрено до двух попыток сдачи решения одной и той же подзадачи на
            реальных роботах. Конкретное количество попыток определяется в конкретных подзадачах.
    \item После прохождения приемочных запусков, баллы набранные командой заносятся судьями в протокол.
            Один из участников команды расписывается за набранный результат, подтверждая согласие команды с оценкой проведенных запусков.
    \item Роботы должны выполнять задание полностью автономно. Удаленное управление не допускается.
            Касание какого-либо робота участником команды после его старта во время приемочных запусков не допускается.
            Алгоритм, реализующий систему управления группой роботов, должен планировать свое выполнение, полагаясь только на информацию с датчиков.
    \item Введение данных в программу до старта устройства (например, координат робота в начале работы) разрешается только
            для тех задач, где это явно прописано. Во всех других случаях введение данных в программу роботов перед запуском запрещено.
    \item Для всех роботов программа должна быть одинаковой, допускается отличие лишь в разрешенных входных данных.
    \item Если какая-то подзадача подразумевает считывание информации с элементов, расположенных на полигоне,
            запрещается при запуске роботов вводить информацию о положении этих элементов или значениях, которые данные элементы определяют.
    \item Если во время приемочных запусков у судьи возникли сомнения о том, что задачи подэтапа решены корректно
            (роботы не выполняют задачу полностью автономно, участник вводит значения в каких-либо роботов перед запуском),
            то он вправе провести инспекцию кода. По результатам инспекции, судья вправе снять с команды баллы, набранные за данный этап.
    \item Если во время приемочных запусков у судьи возникает ситуация, когда он не может однозначно решить
            выполняются ли критерии решения подзадачи, он вправе принять решение не в пользу команды.
    \item Команда вправе обсуждать с судьей результаты приемочных запусков до вызова следующей команды,
            но финальное решение остается о начислении баллов остается за судьей.
\end{enumerate}