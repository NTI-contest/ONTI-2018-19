

\begin{enumerate}
    \item В качестве задачи для симулятора участникам необходимо выполнить следующее:
    \begin{enumerate}
        \item Робот устанавливается в модели логистического центра в случайных
        секторе. При этом структура логистического центра известна заранее.
        Задача робота проехать из точки старта в точку финиша, чьи
        координаты заданы изображением ARTag маркера, заранее считанным с камеры реального
        устройства. На финише необходимо вывести на экран слово \texttt{finish}.

        \textit{Входные данные:} через файл \texttt{input.txt} управляющей программе передаются:
        \begin{enumerate}
            \item[-] В первой строке через пробел --- координаты сектора старта $X_s$, $Y_s$ и направление старта $D_s$
                    (направление робота в секторе старта от 0 до 3 начиная с направления вверх и дальше
                    по часовой стрелке), $0 \le X_s, Y_s \le 7$;
            \item[-] Во второй строке $19200$, разделенных пробелом, целых чисел $P_{1,i}$
                    ($0 \le P_{1,i} \le 2^{24}$) --- изображение ARTag маркера;
        \end{enumerate}
        Каждое число в маркере --- точка, закодированная в формате RGB, т.е. строка с изображением
        маркера эквивалентна снимку разрешением $160 \times 120$ точек. Один маркер кодирует 
        координаты для первого робота, в данной ситуации являющегося единственным.
        
        \textit{Ожидаемый результат:} После запуска программы робот перемещается в сектор финиша.
                После остановки на экран робота выведено \texttt{finish}.

        \item Имя файла с управляющей программой для проверки решения в симуляторе: \texttt{sim\_part4.js}.
    \end{enumerate}
    \item Команде необходимо будет подготовить решения для двух разных подзадач для реального робота. На демонстрацию каждого решения предоставляется 2 попытки.
    \item Все попытки осуществляются 9 марта.
    \item За 15 минут до времени сдачи роботов в карантин для 2ой подзадачи судья определяет сектор старта и направление робота в секторе старта.
            Данные значения команда должна внести в программу перед тем, как сдать робота в карантин.
    \item За 15 минут до времени сдачи роботов в каранин судья определяет сектор сервисного обслуживания
        (сектор, из которого необходимо сканировать ARTag метку) и направление расположения ARTag метки ($0$ -
        маркеры находятся на ``верхнем'' стеллаже, $1$ - на ``правом'' стеллаже и т.д. по часовой стрелке).
        Данные значения команда должна внести в программу перед тем, как сдать робота в карантин.
    \item Cектор старта и сектор сервисного обслуживания может быть разным для каждой попытки.
    \item Правила именования файлов с программой управления:
    \begin{enumerate}
        \item для первой подзадачи: \texttt{part4\_1.js};
        \item для первой попытки второй подзадачи: \texttt{part4\_2\_1.js};
        \item для второй попытки второй подзадачи: \texttt{part4\_2\_2.js}.
    \end{enumerate}
    \item Максимальное время выполнения одной попытки - 3 минуты.
    \item Баллы за решение задач этапа:
    \begin{enumerate}
        \item \textbf{Симулятор}: робот проехал из точки старта в точку финиша на всех проверочных полигонах --- 6 баллов.
        \item \textbf{Первая подзадача на реальном роботе:} Робот верно распознал ARTag метку, которая установлена прямо
                перед камерой, и вывел на экран верное значение, закодированное на метке --- 2 балла.
        \item \textbf{Вторая подзадача на реальном роботе:} Робот располагается за два сектора до сектора сервисного обслуживания:
            \begin{enumerate}
                \item Робот доехал до сектора сервисного обслуживания, распределения задач, остановился, издал звуковой сигнал,
                    вывел на экран верное ARTag метки --- 4 балла.
                \item Робот доехал до указаных в метке координат, остановился, издал сигнал и вывел на экран
                    \texttt{finish}--- 8 баллов.
            \end{enumerate}

    \end{enumerate}
    \item За вторую подзадачу начисляется только половина возможных баллов, если не было засчитано решение в симуляторе, а также если не сдана
        первая подзадача.
    \item Выводить значение метки и \texttt{finish} следует не менее 10 секунд.
    \item Баллы за все попытки в каждой подзадаче суммируются.
    \item Выполнение всех критериев в каждой из двух попыток всех трех подзадач дает дополнительные 4 балла.
    \item Максимальное количество баллов за этап --- 38.
\end{enumerate}