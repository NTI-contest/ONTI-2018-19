% Первый день

\begin{enumerate}
    \item Командам необоходимо подготовить две задачи для симулятора:
    \begin{enumerate}
        \item В качестве первой задачи участникам необходимо выполнить следующее:
        \begin{enumerate}
            \item Робот устанавливается в модели логистического центра в заранее неизвестном секторе.
                Необоходимо в процессе движения опредлить своё местоположение, остановиться и
                вывести на экран координаты робота в формате ``$(X,Y)$''~ без пробелов и без кавычек.

            \textit{Ожидаемый результат:} После запуска программы робот определил своё местоположение, остановился и
            вывел на экран координаты своего местоложожения в формате ``$(X,Y)$''~ без пробелов и без кавычек.
        \end{enumerate}
        \item В качестве второй задачи:
        \begin{enumerate}
            \item Робот устанавливается в модели логистического центра в заранее неизвестном секторе.
                Известно, что два сектора данного логистического центра заблокированы другими роботами.
                Координаты этих секторов передаются через входной файл.
                Необоходимо в процессе движения опредлить своё местоположение, остановиться и
                вывести на экран координаты робота в формате ``$(X,Y)$''~ без пробелов и без кавычек.

            \textit{Входные данные:} через файл \texttt{input.txt} управляющей программе передаются:
            \begin{enumerate}
                \item В первой строке через пробел --- координаты одного заблокированного сектора $X_{1}$, $Y_{1}$,
                    $0 \le X_{1}, Y_{1} \le 7$;
                \item Во второй строке через пробел --- координаты другого заблокированного сектора $X_{2}$, $Y_{2}$,
                    $0 \le X_{2}, Y_{2} \le 7$;
            \end{enumerate}

            \textit{Ожидаемый результат:} После запуска программы робот определил своё местоположение, остановился и
            вывел на экран координаты своего местоложожения в формате ``(X,Y)'' без пробелов и без кавычек.

        \end{enumerate}

        \item Правила именования файлов с управляющей программой для проверки решений в симуляторе:
        \begin{enumerate}
            \item Для первой задачи: \texttt{sim\_part2\_1.qrs};
            \item Для второй задачи: \texttt{sim\_part2\_2.qrs}.
        \end{enumerate}
    \end{enumerate}
    \item Команде необходимо будет подготовить решения для двух разных подзадач для реальных роботов.
            На демонстрацию каждого решения предоставляется 2 попытки.
    \item Все попытки осуществляются 7 марта.
    \item После сдачи в карантин для 1ой подзадачи судья определяет сектор старта
            и направление робота в секторе старта.
    \item За 5 мин до сдачи в карантин для 2ой подзадачи судья определяет сектора старта
        и направления в секторах старта двух роботов. Данные значения команда должна внести в программу перед тем,
        как сдать робота в карантин.
    \item После сдачи в карантин для 2ой подзадачи судья определяет сектор старта
        и направление в секторе старта для оставшегося  робота.
    \item Секторы старта могут быть различными в разных попытках.
    Данные значения команда должна внести в программу перед тем, как сдать робота в карантин.
    \item Правила именования файлов с программой управления:
    \begin{enumerate}
        \item для первой попытки первой подзадачи: \texttt{part2\_1\_1.js};
        \item для второй попытки первой подзадачи: \texttt{part2\_1\_2.js};
        \item для первой попытки второй подзадачи: \texttt{part2\_2\_1.js};
        \item для второй попытки второй подзадачи: \texttt{part2\_2\_2.js}.
    \end{enumerate}
    \item Максимальное время выполнения одной попытки - 5 минут.
    \item Баллы за решение задач этапа:
        \begin{enumerate}
            \item \textbf{Первая задача в симуляторе}: робот смог определить своё местоположение на всех проверочных полигонах --- 12 баллов.
            \item \textbf{Вторая задача в симуляторе}: робот смог определить своё местоположение на всех проверочных полигонах --- 14 баллов.
            \item \textbf{Первая подзадача на реальном роботе:} Робот располагается с случайном секторе
            робототехнического полигона. Робот смог определить своё местоположение на поле, остановился,
            издал звуковой сигнал, вывел на экран свои координаты в формате ``$(X,Y)$'' --- 14 баллов.
            \item \textbf{Вторая подзадача на реальном роботе:} Роботы располагается с случайных секторах
                робототехнического полигона. Роботы смогли определить своё местоположение на поле, остановились,
                издали звуковой сигнал, вывел на экран свои координаты в формате ``$(X,Y)$'', а также вывели \texttt{finish} --- 18 баллов.
        \end{enumerate}
    \item Баллы за первую подзадачу не начисляются, если не было частично засчитано решение первой задачи в симуляторе
    \item Баллы за вторую подзадачу не начисляются, если не были частично засчитаны решения за все задачи в симуляторе.
    \item Баллы за все попытки в каждой подзадаче суммируются.
    \item Выполнение всех критериев в каждой из двух попыток всех двух подзадач дает дополнительные 4 балла.
    \item Максимальное количество баллов за этап --- 94.
\end{enumerate}