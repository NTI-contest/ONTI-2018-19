\assignementTitle{}{5}{}

Реализуйте систему мониторинга состояния водителя, которая включит в себя решения задач 2 и 3, а именно, система должна одновременно выполнять следующие функции:
\begin{itemize}
\item (1 балл) На экран должно выводиться изображение с веб-камеры в режиме реального времени;
\item (3 балла) В случае, если человек, на которого направлена камера дольше чем на 1 секунду, закрыл глаза, или смотрит на свой мобильный (который лежит на уровне коленей водителя), или смотрит на автомагнитолу, необходимо активировать пьезо-элемент, подключенный к плате Arduino.
\item (1 балл) На экране в режиме реального времени поверх изображения с камеры должны выводится следующие параметры:
\begin{itemize}
    \item отношение амплитуды альфа-ритма (8-13 Гц включительно) к амплитуде бета-ритма (15-30 Гц включительно) для ЭЭГ. Временное окно для вычисления амплитуд составляет 1 секунду;
    \item число сердечных сокращений в минуту, вычисляемое по 10 последним ударам сердца;
    \item значение сигнала с модуля кожно-гальванической реакции (в вольтах).
\end{itemize}
\end{itemize}

\markSection

Испытуемый по команде от члена жюри поочередно реализует вышеописанные ситуации. Баллы за первый пункт выставляются в случае вывода видео на экран. Баллы за второй пункт выставляются за корректную активацию пьезо-элемента: по 1 баллу за каждую из трёх ситуаций. Балл за третий пункт выставляется при корректном изменении параметров при совершении действий по команде члена жюри. 

Число попыток: одна попытка 22 марта.
