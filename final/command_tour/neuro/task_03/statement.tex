\assignementTitle{}{30}{}

\section*{Подзадача 1 (6 баллов)}

Напишите программу (скетч) для платы Arduino, которая будет осуществлять запись данных от подключенных к плате Arduino сенсоров в последовательный порт согласно протоколу, описанному в Приложении 5, с частотой 250 Гц.

\markSection

Ставится по 2 балла за каждый корректно выведенный сигнал в программе BiTronics Studio.

Число попыток: 3 попытки за всё время.

Срок сдачи: 22 марта (до конца рабочего дня, согласно расписанию).

\section*{Подзадача 2 (24 балла)}

Напишите программу для ПК, которая в режиме реального времени будет выводить поверх изображения с веб-камеры следующие параметры сигналов с сенсоров:
\begin{itemize}
    \item (10 баллов) отношение амплитуды альфа-ритма (8-13 Гц включительно) к амплитуде бета-ритма (15-30 Гц включительно) для сигнала ЭЭГ. Временное окно для вычисления амплитуд составляет 1 секунду;
    \item (10 баллов) число сердечных сокращений в минуту, вычисляемое по 10 последним ударам сердца;
    \item (4 балла) значение сигнала с модуля кожно-гальванической реакции (в вольтах).
\end{itemize}

\markSection

Корректность расчета вышеперечисленных величин будет проверяться путём подключения к вашему компьютеру устройства, передающего сигналы с заранее известными организаторам параметрами. За каждый верно выведенный параметр ставятся баллы, указанные в скобках выше. 

Число попыток: 3 попытки за всё время.

Срок сдачи: 22 марта (до конца рабочего дня, согласно расписанию).
