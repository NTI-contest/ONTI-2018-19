\solutionSection

В данной задаче командам необходимо было разобраться, как выводить оцифрованные значения с третьего аналогового входа платы Arduino поверх изображения с камеры, а также реализовать распознавание различных положений водителя за рулем. 

Для проверки данной задачи использовалось контрольное видео (участники в первый раз его видели при сдаче): \url{https://drive.google.com/file/d/17EHTJLTJGGk1VVMJ3AfoqtjRiMEvCD6U/view?usp=sharing} (сокращенная ссылка: \url{https://clck.ru/FVfGL})

На данном видео были представлены следующие ситуации (1 – «не опасная» ситуация, 2,3,4 – «опасные» ситуации):

\begin{table}[H]
    \begin{tabular}{|l|c|c|}
        \hline
        Момент времени & Тип ситуации & Описание ситуации \\
        на контрольном видео & & \\
        \hline
        00:00-00:06 & 1 & Смотрит прямо/в зеркало \\
        00:06-00:10 & 3 & Смотрит на панель \\
        00:10-00:17 & 1 & Смотрит прямо/в зеркало \\
        00:17-00:20 & 2 & Закрыты глаза \\
        00:20-00:27 & 1 & Смотрит прямо/в зеркало \\
        00:28-00:32 & 2 & Закрыты глаза \\
        00:32-00:37 & 1 & Смотрит прямо/в зеркало \\
        00:37-00:43 & 3 & Смотрит на панель \\
        00:43-00:48 & 1 & Смотрит прямо/в зеркало \\
        00:48-00:53 & 2 & Закрыты глаза \\
        00:53-00:58 & 1 & Смотрит прямо/в зеркало \\
        00:59-01:04 & 4 & Смотрит на колено \\
        01:04-01:13 & 1 & Смотрит прямо/в зеркало \\
        01:13-01:17 & 2 & Закрыты глаза \\
        01:17-01:25 & 1 & Смотрит прямо/в зеркало \\
        01:25-01:32 & 3, 2 & Смотрит на панель, \\
        & & затем закрывает глаза \\
        01:38-01:46 & 1 & Смотрит прямо/в зеркало \\
        01:46-01:53 & 3 & Смотрит на панель \\
        01:54-01:59 & 1 & Смотрит прямо/в зеркало \\
        02:00-02:03 & 2 & Закрыты глаза \\
        \hline
    \end{tabular}
\end{table}

За каждую верно определенную «опасную» ситуацию можно было получить по 2 балла, в сумме не более 20 баллов.

Пример скриншота работающей программы для данной задачи:

\putImgWOCaption{15cm}{1}

\subsubsection*{Скетч для платы Arduino}

Передает результаты оцифровки третьего аналогового входа по Serial и принимает команды на включение и выключение пьезоэлемента: 

\begin{minted}[fontsize=\footnotesize, linenos]{cpp}
    const int buzzerPin = 11,
      	potPin = 3;

    void setup() {
    Serial.begin(115200);
    pinMode(buzzerPin, OUTPUT);
    }

    void loop() {
    if (Serial.available()) {
        int cmd = Serial.read();
        if (cmd == 49)
        tone(buzzerPin, 500);
        if (cmd == 48)
        noTone(buzzerPin);
        int potVal = analogRead(potPin) / 4;
        Serial.write(potVal);
    }
    delay(1);
    }
\end{minted}

\subsubsection*{Программа для обработки видео}


\begin{minted}[fontsize=\footnotesize, linenos]{python}
    import cv2
    import dlib
    import numpy as np
    from numpy.linalg import norm
    from imutils import face_utils
    from serial import Serial


    class DriverControl:
        """Класс для определения направления поворота головы. Подробности по адресу
        www.learnopencv.com/head-pose-estimation-using-opencv-and-dlib/"""
        def __init__(self, size):
            """Аргументы:
                size - разрешение изображения
            Переменные экземпляра класса:
                maxFrames - Число последовательных кадров на видео, для которых
                            должны выполняться условия cрабатывания
                            пьезоэлемента, преждем чем его активировать.
                earThold -  порог ширины открытия глаз выше ниже которого глаза
                            считаются закрытыми
            """
            self.size = size
            self.model_points = np.array([
                                    (0.0, 0.0, 0.0),
                                    (0.0, -330.0, -65.0),
                                    (-225.0, 170.0, -135.0),
                                    (225.0, 170.0, -135.0),
                                    (-150.0, -150.0, -125.0),
                                    (150.0, -150.0, -125.0)
                                ])
            self.focal_length = size[1]
            self.center = (size[1]/2, size[0]/2)
            self.camera_matrix = np.array([
                                    [self.focal_length, 0, self.center[0]],
                                    [0, self.focal_length, self.center[1]],
                                    [0, 0, 1]
                                ], dtype='double')
            self.dist_coeffs = np.zeros((4, 1))
            self.nFrames = 0
            self.maxFrames = 16
            self.earThold = 0.3

        def getNosePoint(self, shape):
            """Вычисляет направление поворота головы.
            Аргументы:
                shape - массив ключевых точек лица
            Возвращаемые значения:
                nosePoint - координаты конца вектора, указывающего направление
                            головы. Начало вектора - ключевая точка лица
                            с индексом 30.
            """
            if shape is None:
                return (0, 0)
            image_points = np.array([
                                        shape[30], 	# Nose tip
                                        shape[8],  	# Chin
                                        shape[45], 	# Left eye left corner
                                        shape[36], 	# Right eye right corne
                                        shape[54], 	# Left Mouth corner
                                        shape[48]  	# Right mouth corner
                                    ], dtype="double")

            (success, rotation_vector,
            translation_vector) = cv2.solvePnP(self.model_points,
                                                image_points,
                                                self.camera_matrix,
                                                self.dist_coeffs,
                                                flags=cv2.SOLVEPNP_ITERATIVE)
            (nose_end_point2D, jacobian) = cv2.projectPoints(np.array([(0.0, 0.0,
                                                                        1000.0)]),
                                                            rotation_vector,
                                                            translation_vector,
                                                            self.camera_matrix,
                                                            self.dist_coeffs)
            nosePoint = (int(nose_end_point2D[0][0][0]),
                        int(nose_end_point2D[0][0][1]))
            return nosePoint

        def getEAR(self, shape):
            """Вычисляет ширину открытия глаз.
            Аргументы:
                shape - массив ключевых точек лица
            """
            leftH = (norm(shape[37]-shape[41]) + norm(shape[38]-shape[40])) / 2
            leftW = norm(shape[36]-shape[39])
            leftEAR = leftH / leftW
            rightH = (norm(shape[43]-shape[47]) + norm(shape[44]-shape[46])) / 2
            rightW = norm(shape[42]-shape[45])
            rightEAR = rightH / rightW
            EAR = (leftEAR + rightEAR) / 2
            return EAR

        def checkDriver(self, shape):
            """Возвращает True если человек дольше чем на maxFrames кадров закрыл
            глаза, смотрит на свой мобильный, который лежит у него на колене или
            смотрит на автомагнитолу; иначе - False.
            Аргументы:
                shape - массив ключевых точек лица
            """
            # Получаем координату конца вектора направления поворота головы
            nosePoint = self.getNosePoint(shape)
            # Вычисляем ширину открытия глаз
            ear = self.getEAR(shape)
            # Проверяем не ниже ли порога ширина открытия глаз и не опустился ли
            # конец вектора направления поворота головы ниже верхней губы
            if ear < self.earThold or nosePoint[1] > shape[62][1]:
                self.nFrames += 1
                if self.nFrames >= self.maxFrames:
                    return False
            else:
                self.nFrames = 0
                return True


    class LandmarkDetector:
        """Класс для поиска ключевых точек лица на изображении"""
        def __init__(self, modelPath='shape_predictor_68_face_landmarks.dat',
                    showLandmarks=False):
            """Аргументы:
                modelPath - путь к файлу с предобученной моделью для поиска
                            ключевых точек.
                showLandmarks - отмечать ли ключевые точки на лице.
            """
            self.detector = dlib.get_frontal_face_detector()
            self.predictor = dlib.shape_predictor(modelPath)
            self.showLandmarks = showLandmarks

        def getLandmarks(self, image):
            """Возвращает массив с координатами ключевых точек.
            Аргументы:
                image - изображение для поиска ключевых точек
            """
            gray = cv2.cvtColor(image, cv2.COLOR_BGR2GRAY)
            rects = self.detector(gray, 0)
            shape = None
            for rect in rects:
                (bX, bY, bW, bH) = face_utils.rect_to_bb(rect)
                cv2.rectangle(image, (bX, bY), (bX + bW, bY + bH),
                            (0, 255, 0), 1)
                shape = self.predictor(gray, rect)
                shape = face_utils.shape_to_np(shape)
                if self.showLandmarks is True:
                    for (x, y) in shape:
                        cv2.circle(image, (x, y), 1, (0, 0, 255), -1)
            return shape


    def launchVideo(videoSource, arduinoPort):
        """Воспроизводит видео, подаёт команды на Arduino. Для завершения программы
        нажать q.
        Аргументы:
            videoSource - источник видеопотока.
            arduinoPort - COM-порт к которому подключается Arduino"""
        landDetector = LandmarkDetector()
        cap = cv2.VideoCapture(videoSource)
        res = np.array((cap.get(4), cap.get(3)), dtype=int)
        dc = DriverControl(res)
        ser = Serial(arduinoPort, 115200, timeout=0)
        voltageA3 = 0.0
        while True:
            # Считываем кадр
            ret, frame = cap.read()
            if ret:
                # Ищем ключевые точки лица
                shape = landDetector.getLandmarks(frame)
                if shape is not None:
                    # Проверяем не выполняется ли одно из условий задачи.
                    driverOK = dc.checkDriver(shape)
                    if driverOK is False:
                        # Если выполняется, шлём на Arduino символ "1"
                        ser.write(b'1')
                    else:
                        ser.write(b'0')
                # Считываем с Arduino результат оцифровки сигнала с потенциометра
                pot_val = ser.read()
                if pot_val != b'':
                    # Переводим значение АЦП в напряжение
                    voltageA3 = int.from_bytes(pot_val, 'big') * 5.0 / 255
                # Выводим напряжение на изображение
                cv2.putText(frame, 'Value from A3 = {:.2f}'.format(voltageA3),
                            (20, 20), cv2.FONT_HERSHEY_SIMPLEX,
                            0.5, (0, 255, 0), 1)
                cv2.imshow("Output", frame)
                if cv2.waitKey(1) & 0xFF == ord('q'):
                    break

        cap.release()
        ser.close()
        cv2.destroyAllWindows()


    if __name__ == '__main__':
        launchVideo('test.mp4', 'COM13')

\end{minted}