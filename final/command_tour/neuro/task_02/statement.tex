\assignementTitle{}{20}{}

Напишите программу, которая будет выполнять следующие действия:

\begin{enumerate}
    \item Считывать видео с веб-камеры в режиме реального времени и выводить его на экран. На изображении при этом лицо должно быть выделено прямоугольником. 
    В случае, если человек, на которого направлена камера, дольше, чем на 1 секунду:
    \begin{itemize}
        \item закрыл глаза;
        \item смотрит на свой мобильный, который лежит у него на колене;
        \item смотрит на автомагнитолу,
    \end{itemize}
    необходимо активировать пьезоэлемент, подключенный к плате Arduino. При этом пьезоэлемент не должен работать дольше трех секунд за раз. Примеры вышеперечисленных ситуаций представлены на скриншотах в Приложении 3. Водитель во время движения может смотреть в зеркала заднего и бокового вида, что не должно вызывать активацию пьезоэлемента.
    \item Одновременно с этим (вместе с п. 2.1 и 2.2) необходимо в режиме реального времени выводить на изображение значение напряжения на 3 аналоговом входе платы Arduino, к которому подключен потенциометр (см. схему в Приложении 4).
\end{enumerate}

\markSection

Решение этой задачи будет оцениваться путем запуска контрольного видео (см. отрывок видео \url{https://clck.ru/FVTCA}), на котором водитель определенное количество раз будет повторять вышеописанные ситуации. Отрывок видео имеет продолжительность 5 секунд, следует откалибровать скорость воспроизведения видео вашей программой по нему.

Баллы за решение данной задачи будут выставляться по следующей формуле:
$$Result=(\frac{K_{ok}}{K_{sit}} — 0.5\cdot K_{wrong})\cdot (1 - D),$$
где: 
$K_ok$ — количество верных срабатываний (т.е. пьезоэлемент просигналил в те моменты, когда водитель находился в одной из ситуаций, описанных в п. 2.2 данной задачи);
$К_{sit}$ — количество ситуаций на видео, описанных в п. 2.2 данной задачи.
$K_{wrong}$ — количество неверных срабатываний пьезоэлемента после трех неверных срабатываний. Например, если было два неверных срабатывания, то $K_{wrong} = 0$. Если было 4 неверных срабатывания, то $K_{wrong} = 1$.
$D$ — коэффициент дисконтирования. Значение коэффициента дисконтирования в зависимости от даты сдачи представлен в таблице ниже:

\begin{table}[H]
    \caption{Таблица 2 — Значение коэффициента дисконтирования}
    \begin{tabular}{|l|c|}
    \hline
    Дата сдачи задания & Коэффициент дисконтирования (D) \\
    \hline
    19 марта & 0\% \\
    20 марта & 20\% \\
    21 марта & 30\% \\
    22 марта & 50\% \\
    \hline
    \end{tabular}
\end{table}

В случае, если $\frac{K_{wrong}}{K_{sit}} \cdot 20 \leq 0.5\cdot K_{wrong}$, то за задачу ставится 0 баллов.

Например:  
Если вы сдали задачу 21 марта полностью правильно, то вы получите $20 \cdot 0.7 = 14$ баллов за данную задачу. 
Если вы сдали задачу 22 марта и вы набрали при сдаче 10 баллов, то с учетом дисконтирования вы получите: $10 \cdot 0.5 = 5$ баллов за задачу.

Число попыток: 3 попытки за всё время сдачи данного задания.

Срок сдачи: 13:00, 22 марта 2019 г.