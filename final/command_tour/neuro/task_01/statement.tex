\assignementTitle{}{35}{}

Первое задание будет проходить в формате соревнования на платформе kaggle.com. Пригласительную ссылку на соревнование вы получите от судей на вашей площадке.
Вам дана запись трёх сгенерированных сигналов, подобных тем, которые могут быть сняты с водителя большегрузного транспорта во время движения по дороге общего пользования. Ниже перечислены эти сигналы:

\begin{itemize}
    \item одноканальная электроэнцефалограмма (ЭЭГ), зарегистрированная с затылочной части головы водителя;
    \item кожно-гальваническая реакция (КГР), электроды были закреплены на руке водителя;
    \item электрокардиограмма (ЭКГ), снятая в первом отведении. 
\end{itemize}

Частота оцифровки данных сигналов составляет 250 Гц. Сигналы записаны в файлы train.csv и test.csv. Данные поделены на равные временные отрезки. 

В первый столбец обоих файлов записаны идентификаторы временных отрезков данных. Во второй столбец train.csv записано число 1, если человек способен быстро реагировать на изменения обстановки во время движения, и число 0, если он находится в состоянии со сниженным уровнем внимания (или засыпает). Сигналы с датчиков ЭЭГ, КГР и ЭКГ записаны, соответственно, в третий, четвёртый и пятый столбцы файла train.csv и во второй, третий и четвёртый столбцы файла test.csv.

Постройте модель для определения состояния человека с точки зрения безопасности для остальных участников дорожного движения, проанализировав данные из файла train.csv. Воспользовавшись получившейся моделью, спрогнозируйте состояние водителя, основываясь на данных из файла test.csv. Свой ответ запишите в csv-файл, в первом столбце которого должны быть записаны идентификаторы отрезков данных из test.csv, а во втором столбце свои прогнозы состояния водителя соответствующие этим идентификаторам.

Разработчики данного задания при построении модели руководствовались статьями, перечисленными в Приложении 1.

Описание файлов: 

Файлы можно скачать по ссылке: \url{https://drive.google.com/file/d/1ktSb7mZk} \linebreak \url{Anekri6SVGfyuA3PUKYqy5oi/view?usp=sharing}  (укороченная ссылка: \url{https://clck.ru/FVT9v})

Данные разделены на две группы:
\begin{itemize}
\item Обучающая выборка (train.csv)
\item Тестовая выборка (test.csv)
\end{itemize}

Обучающая выборка используется для построения ваших моделей. Для каждой строки из обучающей выборки мы даём вам информацию о том, способен ли водитель быстро  реагировать на изменение окружающей обстановки во время движения или нет. Ваша модель может использовать сигналы ЭЭГ, КГР и ЭКГ для определения состояния водителя. Частота оцифровки сигналов 250 Гц.

Тестовая выборка применяется для оценки качества ваших моделей. Для примеров из тестовой выборки мы даём вам информацию о том, способен ли водитель быстро реагировать на изменение окружающей обстановки во время движения или нет. Примените свою модель для того чтобы определить состояние водителя на каждом из временных отрезков тестовой выборки.

Кроме того, вам дан пример решения sample.csv, для которого значения столбца target были сгенерированны случайным образом.

Поля данных:
\begin{itemize}
    \item ID — идентификатор временного отрезка данных;
    \item target — способен ли водитель быстро реагировать на изменение окружающей обстановки во время движения: 1 - способен, 0 - нет;
    \item eeg — одноканальная электроэнцефалограмма (ЭЭГ), зарегистрированная с затылочной части головы водителя;
    \item gsr — кожно-гальваническая реакция (КГР), электроды были закреплены на руке водителя;
    \item cg — электрокардиограмма (ЭКГ), снятая в первом отведении.
\end{itemize}

Формат ответа: csv-файл, где в первом столбце находятся идентификаторы временных отрезков данных из test.csv, а во втором столбце - состояние человека (0 или 1).

Перед отправкой решения в систему Kaggle, необходимо передать код представителю жюри в аудитории. Решения без исходного кода засчитываться не будут.

\markSection
Проверка точности решения осуществляется автоматически в системе Kaggle (подробнее см. Приложение 2). Оценка ставится пропорционально числу правильно определенных состояний, округляется до сотых по правилам математического округления. Например, если вы правильно определили состояние водителя на всех временных отрезках тестовых данных, вы получите 35 баллов. Если состояние водителя определено правильно только для половины отрезков, вы получите 17,5 баллов.

Число попыток: 3 попытки в сутки. Число попыток обновляется в 03:00 по московскому времени. Неиспользованные попытки сгорают.

Срок сдачи: 13:00, 22 марта 2019 г.

