\assignementTitle{}{10}{}

Опишите свою физиологическую модель, использовавшуюся для решения задачи 1. Для этого сформулируйте в текстовом виде  (формат .txt) закономерности, выявленные в  анализируемых сигналах, в соответствии с которыми выносилось  решение о состоянии водителя (состояния “0” или “1” в задаче 1). Решение необходимо сдать представителю жюри. 

\markSection

На базе статей, которые были предоставлены для подготовки к финалу (Приложение 1), были выделены паттерны сигналов, характерные для разных состояний водителя. На основании этих данных были сгенерированы данные для задачи 1. За каждый верно определенный паттерн выставляется по 2 балла. В сумме не более 10 баллов.

Число попыток: 1 попытка. 

Срок сдачи: Данное задание можно сдавать 21 марта (весь день) и 22 марта (до 12-00), дается всего 1 попытка. 
