\solutionSection

При генерации данных использовались следующие паттерны сигналов, характерные для разных состояний водителя (на основе статей из Приложения №1):
\begin{enumerate}
\item Временной интервал между локальными минимумами сигнала КГР изменяется пропорционально уровню внимания водителя. Чем ниже уровень внимания, тем больше временные промежутки. Сигнал был искажён симуляцией механических артефактов и изменения температуры в кабине автомобиля.
\item При снижении уровня внимания водителя частота сердечных сокращений падает от величин порядка 90 ударов в минуту до порядка 60.
\item Со снижением внимания амплитуда альфа-ритма относительно амплитуды бета-ритма сигнала ЭЭГ растёт.
\item Уровни альфа- и бета- ритмов в сигнале ЭЭГ периодически усиливаются и ослабляются на промежутках времени порядка 30-60 с. При этом длительность периодов усиления альфа-ритма со снижением уровня внимания растёт. Длительность периодов при этом колеблется.
\end{enumerate}

Дополнительным паттерном, имеющим корреляцию с уровнем внимания водителя, является отношение спектральных компонент кардиоритмограммы (кардиоинтервалограммы): низкочастотной (LF) к высокочастотной (HF). При появлении сонливости наблюдается рост LF/ HF. 
Данный признак не был использован при генерации данных для задачи 1, но оценивается наравне с вышеупомянутыми признаками.

За каждый верно определенный паттерн ставилось 2 балла. В сумме – не более 10 баллов.

Комментарий: обратите внимание, что при регистрации КГР измеряется сопротивление, а не проводимость.

