\section*{Приложение 5 — Протокол передачи данных}

Загруженный на плату Arduino скетч должен опрашивать три аналоговых входа платы  (A0, A1, A2) и отправлять результаты оцифровки с данных входов на последовательный (Serial) порт компьютера. Для того чтобы программа на компьютере смогла различать данные с разных входов, необходимо их разделить. Для разделения используются строки с названиями портов: “А0”, “А1”, “А2”. Таким образом, на последовательный порт в одном сообщении мы отправляем 9 байт информации: “A0XA1YA2Z”.

\begin{enumerate}
\item А0, А1, А2 — три строки, каждая из которых состоит из двух символов. Символ занимает 1 байт. 
\item X, Y, Z — результаты оцифровки сигналов с аналоговых входов А0, А1, А2. Эти данные предварительно нужно привести к размеру 8 бит каждое. Плата Arduino имеет 10 битный аналогово-цифровой преобразователь, поэтому функция analogRead возвращает значение от 0 до 1023 ($1023 = 2^{10}-1$). Для приведения считанного значения к 8-битному диапазону можно использовать функцию map, деление на 4 или побитовый сдвиг вправо на две позиции.
\end{enumerate}

Частота обмена данными с COM-портом должна составлять 115200 бит/с. На компьютер должно передаваться 250 описанных выше сообщений в секунду. Равные временные промежутки между циклами оцифровки сигнала можно формировать при помощи функции delay или средствами библиотеки TimerOne.