\section*{Приложение 1 — Список литературы к задаче №1 и №4}

\begin{enumerate}
    \item Регуляция цикла бодрствование-сон, Ковальзон В.М., Долгих В.В. Неврологический журнал, №6, 2016 стр.316-322. 
    \url{http://www.sleep.ru/lib/Nevrol_6-2016.pdf}
    \item Нейрофизиология и нейрохимия сна, Ковальзон В.М. Сомнология и медицина сна. Национальное руководство памяти А.М.Вейна и ЯЯ.И.Левина/ Ред. М.Г. Полуэктов. М.: «Медфорум». 2016. С.11 – 55. 
    \url{http://www.sleep.ru/lib/Medforum_2016_1.pdf}
    \item Прогнозирование моментов критического снижения уровня бодрствования по показателям зрительно-моторной координации, Г.Н. Арсеньев, О.Н.Ткаченко, Ю.В. Украинцева, В.Б.Дорохов
    Журнал высшей нервной деятельности, 2014, том 64, №1, с. 64-76. 
    \url{http://www.sleep.ru/lib/arsenev2014_ru.pdf}
    \item Сомнология и безопасность профессиональной деятельности, В.Б.Дорохов. 
    Журнал высшей нервной деятельности, 2013, том 63, №1, с. 33 – 47
    \url{http://www.sleep.ru/lib/Dorokhov_ed_2013JourVND.pdf}
    \item Психомоторный тест для исследования зрительно-моторной координации при выполнении монотонной деятельности по прослеживанию цели, В.Б.Дорохов, Г.Н.Арсеньев, О.Н.Ткаченко, Д.В.Захарченко, Т.П.Лаврова, В.В.Дементиенко. 
    Журнал высшей нервной деятельности, 2011, том 61 №4, с. 476 – 484
    \url{http://www.sleep.ru/lib/VND0476.pdf}
    \item Биоматематическая модель процесса засыпания человека-оператора, 
    
    В.В.Дементиенко, В.Б.Дорохов, С.В.Герус, Л.Г.Коренева, А.Г.Марков, 
    
    В.М.Шахнарович. Физиология человека, 2008, том 34, №4, с. 1 – 10
    \url{http://www.sleep.ru/lib/Dementienko_Dor_Rus.pdf}
    \item Альфа-активность ЭЭГ при дремоте, как необходимое условие эффекторного взаимодействия с внешним миром, Дорохов В.Б. Электронный журнал «Исследовано в России», 2003.
    \url{http://www.sleep.ru/download/192.pdf}
    \item Альфа-веретена и К-комплекс – фазические активационные паттерны при спонтанном восстановлении нарушений психомоторной деятельности на разных стадиях дремоты, В.Б.Дорохов
    Журнал высшей нервной деятельности, 2003, том 53, №4, с.502 – 511
    \url{http://www.sleep.ru/lib/Dorokhov-K-compl.pdf}
    \item Электродермальные показатели субъективного восприятия ошибок в деятельности при дремотных изменениях сознания, В.Б.Дорохов, 
    
    В.В.Дементиенко, Л.Г.Коренева, А.Г.Марков, В.М.Шахнарович
    Журнал высшей нервной деятельности, 2000., том 50, №2, с. 206 – 218
    
    http://www.sleep.ru/lib/Dorokhov2000.pdf 
    \item Анализ психофизиологических механизмов нарушения деятельности при дремотных изменениях сознания, В.Б.Дорохов. Вестник РГНФ, 2003, с. 137 – 144
    \url{http://www.sleep.ru/download/Dorohov_04.pdf}
    \item Обзор систем бдительности водителя, перевод и публикация с разрешения Совета по Безопасности и Стандартам на Железных Дорогах Великобритании (RSSB), 2002
    \url{http://www.neurocom.ru/pdf/press/report_rssb_russian.pdf}
    \item Hypothesis about the nature of electrodermal reactions, Dementienko V.V., 
    
    Koreneva L.G., Tarasov A.V., Shakhnarovitch V.M. Journal of Psychophysiology, 
    
    1998.V.30/1-2, p.267
    \url{http://www.neurocom.ru/pdf/press/edrhype.pdf}
    \item Удаленный контроль состояния водителя как средство повышения безопасности перевозок, Тагиров М.К., Юров А.П., Иванов И.И., Макаев Д.В. Материалы II Международной научно-практической конференции «Научно-\linebreak технические аспекты комплексного развития транспортной отрасли» в рамках Международного форума Донецкой Народной Республики 25-26 мая 2016 года. Секция «Информационные процессы, технологии и системы на транспорте».
    \url{http://www.diat.edu.ua/files/2016.pdf#page=52}
    \item Программно-аппаратная реализация бортовых оперативно-советующих экспертных систем на транспорте, Н.В.Корнеев, А.В. Гребенников. Известия Самарского научного центра Российской академии наук, т.16, №4, 2014, с. 116-122
    
    \url{https://cyberleninka.ru/article/n/programmno-apparatnaya-realizatsi}\linebreak \url{ya-bortovyh-operativno-sovetuyuschih-ekspertnyh-sistem-na-transporte}
    \item Вариабельность сердечного ритма во время сна у здоровых людей, 
    
    И.М.Воронин, Е.В.Бирюкова Вестник аритмологии, №30, 2002, с. 68 – 71
    \url{http://www.vestar.ru/atts/2965/2965voronin.pdf}
    \item Эффективность систем мониторинга водителя, В.В.Дементиенко, 
    
    В.Б.Дорохов, С.В.Герус, А.Г.Марков, В.М.Шахнарович. Журнал технической физики, 2007, том 77, вып. 6, с. 103 – 108.
    \url{https://journals.ioffe.ru/articles/viewPDF/9153}
    \item Оценка эффективности систем контроля уровня бодрствования человека-оператора с учетом вероятностной природы возникновения ошибок при засыпании, В.В.Дементиенко, В.Б.Дорохов
    Журнал высшей нервной деятельности, 2013, том 63, №1, с. 24 – 32
    \url{http://www.sleep.ru/lib/JourVND_63_01.pdf}
    \item Система анализа физиологического состояния водителей транспортных \linebreak средств, Щербакова Т.Ф., Култынов Ю.И., Осипова О.С. Актуальные проблемы и достижения в медицине/ Сборник научных трудов по итогам международной научно-практической конференции. №2. Самара, 2015.Секция №21. Медицина труда.
    \url{http://izron.ru/articles/aktualnye-problemy-i-dostizheniya-v-meditsine-sbornik-nauchnykh-trudov-po-itogam-mezhdunarodnoy-nauch/sektsiya-21-meditsina-truda-spetsialnost-14-02-04/sistema-analiza-}\linebreak \url{fiziologicheskogo-sostoyaniya-voditeley-transportnykh-sredstv/}
    \item Бортовая система мониторинга функционального состояния оператора транспортного средства, В.В.Савченко. Механика машин, механизмов и материалов, 2012, №1,  с. 20 – 25
    \url{http://mmmm.by/pdf/ru/2012/1_2012/3.pdf}
    \item Электроэнцефалографические показатели дремотного состояния при выполнении монотонной операторской деятельности, О.Н.Ткаченко, А.А.Фролов. Труды МФТИ, 2010, том 2, №2, с. 41 – 45
    \url{https://mipt.ru/science/trudy/2_6/41-45-arphcxl1tgs.pdf}
    \item Основы сомнологии: физиология и нейрохимия цикла «Бодрствование - сон», В.М.Ковальзон – М.: БИНОМ. Лаборатория знаний, 2012. – 239 с.
    \url{https://docplayer.ru/26078038-V-m-kovalzon-osnovy-somnologii.html} 
\end{enumerate}