На этом этапе участникам необходимо было разработать систему оценки состояния водителя за рулем, определяющая опасные ситуации на дороге: сонливость и потеря внимания водителя. Для решения данной задачи участникам необходимо было использовать методы машинного обучения, компьютерного зрения и анализ биосигналов человека (кожно-гальваническая реакция, электрокардиограмма, электроэнцефалограмма).

Участникам было предложено решить 5 задач. Для решения задач необходимо было распределить между участниками команды следующие роли: физиолог, специалист по машинному обучению, специалист по компьютерному зрению, инженер-электронщик. Один человек может взять на себя несколько ролей. 

Максимальное количество баллов — 100 баллов. В таблице ниже представлен график открытия задач и количество баллов, которые можно будет получить за задачи.

\begin{table}[H]
    Таблица 1 - Баллы за задачи, график и роли
    
    \begin{tabular}{|l|l|l|l|l|}
    \hline
    № &	Кол-во баллов & Дата & Срок сдачи & Требуемые роли \\
                    & & открытия & (время местное) &  \\            
    \hline
    1 & 35 баллов & 19 марта & 13:00, 22 марта & Физиолог, \\
                                             & & & & Специалист по машинному \\
                                             & & & & обучению \\
    \hline
                                             2 & 20 баллов & 19 марта & 13:00, 22 марта & Физиолог, \\
                                              & & & & Специалист по \\ 
                                              & & & & компьютерному зрению, \\
                                              & & & & инженер электронщик \\
    \hline
    3 & 30 баллов & 20 марта & 13:00, 22 марта & Все \\
    \hline
    4 & 10 баллов & 20 марта & 12:00, 22 марта & Физиолог \\
    \hline
    5 & 5 баллов & 22 марта & 13:00, 22 марта & Все \\
    \hline
    \end{tabular}
\end{table}

Задачи можно выполнять параллельно и сдавать в любом порядке. Внимательно ознакомьтесь с критериями оценивания задач (указаны после каждой задачи), количеством попыток и сроками для сдачи заданий.