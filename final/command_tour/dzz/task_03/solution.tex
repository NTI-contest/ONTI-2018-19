\solutionSection

\subsubsection*{Интеграция СУПН и спутника}

Интеграция заключается в адаптации написанных ранее решений и сборке устройств в единую сеть.

\begin{enumerate}
    \item Код для БКУ:
    
    \inputminted[fontsize=\footnotesize, linenos]{python}{final/command_tour/dzz/task_03/source_1.py}

    \item Код для Arduino можно оставить таким же, как и в предыдущем пункте.
    \item Пример кода для Raspberry:
    
    \inputminted[fontsize=\footnotesize, linenos]{python}{final/command_tour/dzz/task_03/source_2.py}
\end{enumerate}

Связка из трёх устройств будет работать, если одновременно запустить на исполнение код БКУ и код на Raspberry.

\subsubsection*{Полная интеграция}

На данном этапе остаётся собрать все устройства в единую сеть, осуществить крепление оптической системы к спутнику и произвести юстировку. 

\subsubsection*{Проектирование группировки}

Даны области съёмки:
\begin{enumerate}
    \item 68-69, 73-74
    \item 70-71, 73-74
    \item 73-74, 70-72
    \item 71-72, 64-65
    \item 70-71, 57-59
    \item 70-71, 48-49
    \item 69-70, 43-44
    \item 69-70, 37-38
    \item 68-69, 33-35
\end{enumerate}

И камеры с соответствующими характеристиками:

\begin{table}[H]
    \begin{center}
        \begin{tabular}{|p{2cm}|p{2cm}|p{3cm}|p{3cm}|p{3cm}|}
            \hline Тип и диапазон &Диапазон съемки & Ширина полосы при размещении на высоте 500 км, в км & Пространственное разрешение на высоте 500 км, в м & Условная стоимость аппарата с камерой этого типа \\
            \hline
            HR1 / ВД & ВД & 44 & 8 & 2 \\
            \hline
            HR2 / ВД & ВД & 23 & 4 & 2 \\
            \hline
            HR3 / ВД & ВД & 16 & 3 & 3 \\
            \hline
            HR4 / ВД & ВД & 88 & 8 & 4 \\
            \hline
            HR5 / ВД & ВД & 47 & 4 & 4 \\
            \hline
            HR6 / ВД & ВД & 32 & 3 & 6 \\ 
            \hline
        \end{tabular}
    \end{center}
\end{table}

\textit{Пример решения:}

\putImgWOCaption{13cm}{1}

\putImgWOCaption{13cm}{2}

\putImgWOCaption{13cm}{3}

\putImgWOCaption{13cm}{4}

\putImgWOCaption{13cm}{5}
