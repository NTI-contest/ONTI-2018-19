Сегодня последний и самый важный день. Вам нужно будет откалибровать свои устройства в подготовленном помещении. Делать это нужно аккуратно, проверив все моменты, так как на саму защиту у вас будет очень мало времени (около 15 минут).

Как будет выглядеть финальное задание:
\begin{enumerate}
    \item[0.] В затемненном коридоре будет стоять специально укрепленный подвес, на расстоянии 4-5 метров от него - экран со снимком поверхности Земли из космоса в высоком разрешении.
    \item[1.] После подвеса спутника вы настраиваете оптику;
    \item[2.] Далее подходите к жюри со своими ПК для демонстрации запуска программы управления;
    \item[3.] Необходимо открыть папку, в которую сохраняются фото;
    \item[4.] Запустить работу устройств для получения снимков (как это было нужно сделать в последнем задании этого дня), дождаться выполнения программы;
    \item[5.] Скинуть лучший снимок на флешку жюри
    \item[6.] Оперативно снять спутник и отнести его в 308.
\end{enumerate}

Требования к снимку:
\begin{itemize}
    \item разрешение $640 \times 480$
    \item максимально детализированное изображение
\end{itemize}

\markSection

\begin{table}[H]
    \begin{center}
        \begin{tabular}{|p{8cm}|c|c|}
            \hline
            Задача & Макс. балл & Балл команды \\
            \hline
            \multicolumn{3}{|c|}{Интеграция СУПН и спутника} \\
            \hline
            Настроен обмен данными между Raspberry и БКУ&	10& \\
            \hline
            По сигналу от БКУ Raspberry делает фото и передаёт его на ПК&	20	& \\
            \hline
            \multicolumn{3}{|c|}{Полная интеграция} \\
            \hline
            Осуществлена полная сборка компонентов, компоненты запитаны и находятся в работоспособном состоянии	&15& \\
            \hline
            Решение комплексной задачи:
            \begin{itemize}
                \item спутник ориентируется с оптической системой на заданный угол
                \item после ориентации спутник делает снимок
                \item снимок передаётся на ПК
                \item полученный снимок не является размытым
            \end{itemize}	& 50 & \\
            \hline	
        \end{tabular}
    \end{center}
\end{table}

Критерии оценки для комиссии жюри.

\putImgWOCaption{16.5cm}{1}

Ниже представлено изображение с миррой, по которому оценивалось качество съёмки.

\putImgWOCaption{16cm}{2}