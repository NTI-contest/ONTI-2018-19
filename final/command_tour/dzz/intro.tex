Проход морских судов по акватории Арктики играет важную роль в экономике приарктических государств, в т.ч. России, Норвегии, Швеции, Канады и США. Оперативное наблюдение ледовой обстановки не только способствует безопасности судоходства, но и позволяет решать задачи точного планирования отгрузок.

\putImgWOCaption{10cm}{1}

Контроль движения судов и мониторинг ледовой обстановки осуществляется, в том числе, следующими инструментами:
\begin{itemize}
    \item AIS (Аutomatic Identification System) - международная система определения положения судов, состоящая из бортовых станций на кораблях, наземных опорных станций и станций космического базирования.
    \item Данными космической съемки спутниками ДЗЗ, в т.ч.
    \begin{itemize}
        \item Данные радарной съемки (в т.ч. например канадского RadarSat-2)
        \item Обзорные данные с оптических мультиспектральных камер низкого разрешения (в т. ч. например аппаратов Aqua и Terra)
        \item Данные с оптических камер (в первую очередь видимого и инфракрасного диапазона) высокого разрешения.
    \end{itemize}
\end{itemize}

Участникам предлагаются компоненты задачи разработки проекта созвездия  спутников ДЗЗ с оптическими и мультиспектральными камерами, предназначенным для мониторинга ледовой обстановки Северного Морского Пути.
