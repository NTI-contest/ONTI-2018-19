\solutionSection

\subsubsection*{Спутник}

\begin{enumerate}
    \item Проверка работоспособности устройств:
    
    ДУС:

\inputminted[fontsize=\footnotesize, linenos]{python}{final/command_tour/dzz/task_01/source_1.py}

    Магнитометр:

\inputminted[fontsize=\footnotesize, linenos]{python}{final/command_tour/dzz/task_01/source_2.py}

    Маховик:

\inputminted[fontsize=\footnotesize, linenos]{python}{final/command_tour/dzz/task_01/source_3.py}

    \item Стабилизация

    Пример кода для стабилизации:

\inputminted[fontsize=\footnotesize, linenos]{python}{final/command_tour/dzz/task_01/source_4.py}

\end{enumerate}

\subsubsection*{СУПН}

Первоначально подключение Raspberry к ПК осуществляется через патч-корд. Ниже приведен алгоритм действий как можно подключить Raspberry к ПК с GNU/Linux-системой.

\begin{enumerate}
    \item Можно установить пакет, который поможет осуществить подключение Raspberry к вашему ПК. Например, это может быть dnsmasq. Для этого наберите в терминале:
    \putImgWOCaption{15cm}{1}
    \item Далее не лишним будет создать ethernet-соединение, позволяющее подключаться устройствам к вашему ПК, используя диспетчер сети.
    \item Используя dnsmasq можно определить IP, который имеет Raspberry в локальном подключении:
    \putImgWOCaption{15cm}{2}    
    IP вашей Raspberry в данном случае - 10.42.0.90
    \item Далее подключимся к Raspberry по ssh:
    \putImgWOCaption{15cm}{3}
\end{enumerate}

Пароль по умолчанию: raspberry

Если всё прошло успешно имя пользователя сменится на pi@raspberrypi.

Теперь у нас есть доступ к терминалу Raspberry, и мы можем ей управлять.
 
