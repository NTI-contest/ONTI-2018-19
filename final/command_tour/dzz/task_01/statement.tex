\assignementTitle{}{35}{}

Сегодня вашей задачей станет предварительная подготовка к большой комплексной задаче. Ниже задачи разделены на блоки и даны небольшие инструкции как их выполнить. Вы можете выполнять задачи любым способом, не обязательно следовать им от и до, если вы знаете решение.

\subsection*{Спутник}

Необходимо решить базовые задачи: 
\begin{itemize}
    \item собрать КА;
    \item проверить работу его устройств, прежде всего:
    \begin{itemize}
        \item вывод данных в веб-консоль;
        \item исправная работа маховика;
        \item исправная работа датчика угловой скорости (ДУС);
        \item исправная работа магнитометра.
    \end{itemize}
    \item застабилизировать космический аппарат. 
\end{itemize}

Рекомендуем использовать для программирования язык С.

В данных задачах вам поможет сайт: \url{http://www.orbicraft.sputnix.ru/doku.php?id=stabilization}

Исходник кода по стабилизации не возбраняется брать и адаптировать с сайта.

\subsubsection*{Система управления полезной нагрузкой}

Для получения снимков с камеры рекомендуем использовать Raspberry. Микрокомпьютер уже имеет карту памяти с предустановленной ОС. Также установлен пакет Motion, который имеет возможность транслировать потоковое видео с камеры, а также задавать конфигурации камеры. Однако для работы необязательно использовать именно его.

Для начала подключите Raspberry к ПК и попробуйте получить пару снимков с камеры.

Инструкция по подключению:
\begin{enumerate}
    \item[1.] Сейчас вам понадобится сканер сетевых подключений. Для упрощения и ускорения работы с ним советуем отключить wifi.
    \begin{enumerate} 
        \item[1.1.] Откройте командную строку через комбинацию клавиш "window + r" и введите в поиске "cmd"
        \item[1.2.] В командной строке введите команду "ipconfig"
        \item[1.3.] Найдите строку с упоминанием  ip4, здесь будет указан локальный ip, вашего компьютера, он пригодится вам далее.
    \end{enumerate}
    \item[2.] Скачайте, установите и откройте сканер сетевых подключений. Наиболее популярен "advansed ip", но в некоторых случаях куда быстрее работает "10-страйк сканирование сети". Далее будет приведена инструкция к последней утилите.
    \begin{enumerate}
        \item[2.1.] Выберете самое левое изображение с лупой - "сканирование сети".
        \item[2.2.] Выберете "сканирование диапазона".
        \item[2.3.] В графе "интерфейс" вам нужно указать устройство с  ip4 адрес из "ipconfig", который вы нашли ранее.
        \item[2.4.] В графе " диапазон" вам нужно добавить диапазон, включающий ip4 вашего компьютера.
        
        Например для ip4 169.254.29.10 диапазон будет 169.254.0.0 - 169.254.255.255. Затем нужно убрать галочки или удалить все лишние диапазоны.
        \item[2.5.] В следующем окне вам нужно оставить галочку только на "icmp пинге", другие убрать, цифры не трогать.
        \item[2.6.] Запускайте сканирование и ждите появления ip4 raspberry. Обычно он находится до 40\% загрузки. Если нет - обратитесьd за помощью.
    \end{enumerate}
    \item[3.] Скачайте, установите и откройте Putty.
    \begin{enumerate}
        \item[3.0.] Для упрощения перезагрузки программы после ошибки советуем закрепить её в стартовой панели.
        
        В графе host Name укажите pi\@<адрес ip4 из сканера>
        
        connection type: ssh

        Далее нажмите open.
    \end{enumerate}
    \item[4.] Далее должна открыться unix-авая командная строка raspberry.
    
    Пароль: raspberry
    Это нормально, что при вводе вы не видите пароль.
\end{enumerate}

\subsection*{Приготовления к оптике}

Изначально оптическая система будет настраиваться по источнику света. У вас есть необходимые детали для сборки и установки источника:
\begin{itemize}
    \item корпус с 3д-паззлом
    \item патрон для лампочки
    \item вилка для подключения к сети
\end{itemize}

Необходимо собрать и установить источник на рабочий стол напротив.

\markSection

В первый день начисления баллов не происходит, так как рабочего командного времени крайне мало. Однако есть чек-лист, который может помочь вам скоординировать свою работу.

\begin{table}[H]
    \center
    \begin{tabular}{|p{10cm}|c|c|}
        \hline
        Задача & 	Отметка о выполнении\\
        \hline
        \multicolumn{2}{|c|}{Спутник} \\
        \hline
        Спутник собран и соблюдена развесовка (при подвесе нет наклона спутника)& \\
        \hline	
        Данные выводятся в web-консоль& 	\\
        \hline
        ДУС работоспособен и способен снимать показания& 	\\
        \hline
        Магнитометр работоспособен и способен снимать показания& 	\\
        \hline
        Маховик работоспособен& 	\\
        \hline
        Спутник осуществляет стабилизацию& 	\\
        \hline
        \multicolumn{2}{|c|}{СУПН} \\
        \hline
        Осуществлено подключение ПК к Raspberry. Есть возможность войти в терминал или графический интерфейс.& \\
        \hline	
    \end{tabular}
\end{table}