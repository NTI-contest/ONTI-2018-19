\assignementTitle{}{35}{}

Спроектируйте систему управления углом курса БПЛА через крен на симуляторе полета БПЛА в системе MATLAB. Система управления должна удовлетворять требованиям:
\begin{itemize}
	\item время переходного процесса в линейной зоне – 10 сек;
	\item перанный угол крена не более 30 градусов по модулю;
	\item разерегулирование не более 5\%;
	\item задворот должен проходить по кратчайшему пути.
\end{itemize}
\solutionSection
Управление по курсу осуществляется через изменение угла крена БПЛА. Другими словами, входом системы управления является рассогласование $\psi-\psi^\text{зад}$, где $\psi$ – текущий угол курса, $\psi^\text{зад}$ – требуемый угол курса, а выходной величиной – заданный угол крена $γ^\text{зад}$. 
Подходящим контроллером является пропорциональный регулятор (П-регулятор):
$$u=K_\psi \epsilon,$$
где $\epsilon$ – сигнал ошибки, $u$ – управляющий сигнал, $K_\psi$ – коэффициент усиления.
Оба угла $\psi$ и $\psi^\text{зад}$ изменяются в пределах $[-180^\circ, +180^\circ]$. Для обеспечения кратчайшего разворота сигнал ошибки $\epsilon$ рассчитывается как
$$\epsilon=atan2(sin⁡(\psi-\psi^\text{зад} ),cos⁡(\psi-\psi^\text{зад} ) )$$.
Чтобы заданный угол крена не превышал по модулю 30 градусов, управляющий сигнал регулятора ограничивается следующим образом:
\begin{displaymath}
    \gamma^\text{зад}= \left\{ \begin{array}{ll}
     u , |u| \leq 30^\circ\\
    30^\circ, u > 30^\circ \\
    -30^\circ,u < -30^\circ
    \end{array} \right.
\end{displaymath}
Коэффициент $K_\psi$ подбирается таком малом сигнале рассогласования, что $|\gamma^\text{зад} <30^\circ$. Увеличение коэффициента ускоряет переходный процесс, однако, при больших значениях $K_\psi$ регулятор курса может работать быстрее, чем система стабилизации крена, что приведет к перерегулированию и колебательности. Удовлетворительное качество процесса регулирования достигается при $K_\psi=8.14$.
\putImgWOCaption{10cm}{2}
\begin{center}
    Рис. 2 Структурная схема системы управления углом курса.
  \end{center} 
\includeSolutionIfExistsByPath{}
