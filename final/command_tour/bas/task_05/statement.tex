\assignementTitle{}{35}{}

Напишите программу полета БПЛА на заданную точку. Полет на заданную точку осуществить при помощи регулятора курса, разработанного в задаче 4. Решение задачи проверить в ходе летных испытаний.

\solutionSection
\begin{enumerate}
    \item Рассчитать курс на требуемую точку. используя измерения текущего местоположения.     
    \item  Рассчитать расстояние до требуемой точки, используя измерения текущего местоположения.
    \item  Если расстояние до заданной точки меньше заданного (5 м), подать сигнал на цифровую ножку для включения камеры.
    \item  Провести расчет заданного угла курса для стабилизации курса на точку аналогично Задаче 4. Полученный ответ перевести из градусов в отсчеты ШИМ и ограничить.
    \item Вывести полученное значение ШИМ сигнала в канал элеронов. 
    Параметры системы управления:
    \begin{itemize}
        \item	Коэффициент пропорциональности регулятора: 4;
        \item	Ограничения ШИМ в канале элеронов: 1350… 1650;
        \item	Коэффициент пересчета градусы – ШИМ: 2.
    \end{itemize}
\end{enumerate}
Переменные
\begin{minted}[fontsize=\footnotesize, linenos]{cpp}
float presetCoordinates[2] = {56.097714, 35.878395};	// координаты метки
#define SETPOINT_RADIUS 5	// Окрестность точки, в которой провести съемку
#define HEADING_P       4	// Коэффициент П регулятора
\end{minted}
Код в цикле
\begin{minted}[fontsize=\footnotesize, linenos]{cpp}
// Курс на метку
double courseToPreset = TinyGPSPlus::courseTo(latitude, longitude, 
					presetCoordinates[0], presetCoordinates[1]);

// Расстояние до метки
unsigned long distanceToPreset = 
	(unsigned long) TinyGPSPlus::distanceBetween(latitude, longitude, 
		presetCoordinates[0], presetCoordinates[1]);
		
	//Выдача сигнала в заданном радиусе до точки
	if(distanceToPreset <= SETPOINT_RADIUS) digitalWrite(4, 1);
	else digitalWrite(4, 0);

	// Стабилизатор курса
	double set_roll = constrain(1500 + 
		HEADING_P*(courseToPreset - heading), 1350, 1650);

	// Расчет управления на стабилизатор крена
	double Koeff = 2.0;		// коэффициент перевода градусов в ШИМ
	pwmAil = constrain(1500 + 
		Koeff*(map(set_roll, 1000, 2000, -20, 20)), 1350, 1650);

	// Выдача ШИМ
	AIL.writeMicroseconds(pwmAil);

\end{minted}

\includeSolutionIfExistsByPath{}
