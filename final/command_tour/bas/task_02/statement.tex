\assignementTitle{}{35}{}
Напишите программу определения угла крена комплексируя измерения акселерометра и гироскопа. Проведите предварительную ручную калибровку акселерометра и гироскопа. Калибровку акселерометра проведите с учетом сдвига нуля и масштабного коэффициента, при калибровке гироскопа учтите только сдвиг нуля. Выведите на один график оценку угла крена по акселерометру и комплексированную. Подберите коэффициент фильтрации так. чтобы в измерения не вносилось запаздывание, но точность улучшилась.

\solutionSection
\subsubsection*{Этап 1. Программа калибровки датчиков}

Модели измерений датчиков имеют вид:
$$\omega_i=\omega_i^\text{изм}-b_i^\text{гир}-\xi_i,$$
$$n_i=k_i^\text{акс}(n_i^\text{изм}-b_i^\text{акс})-\eta_i,$$
где $\omega_i$ – истинное значение угловой скорости вращения вокруг оси $i={x,y,z}$, $\omega_i^\text{изм}$ – измеренное значение $\omega_i$, $b_i^\text{гир}$ – сдвиг нуля гироскопа, $n_i$ – истинное значение кажущегося ускорения по оси $i$, $n_i^\text{изм}$ – измеренное значение $n_i$, $b_i^\text{акс}$ – сдвиг нуля акселерометра, $k_i^\text{акс}$ – масштабный коэффициент акселерометра, $\xi_i$, $\eta_i$ – случайные погрешности гироскопа и акселерометра соответственно.
\begin{enumerate}
    \item Калибровка гироскопа. Оставить датчик неподвижно – истинная угловая скорость в этом случае принимается за 0. Для каждой оси найти значение масштабного коэффициента $b_\i^\text{гир$       как среднее $N$ замеров $\omega_{i,k}^\text{изм}$ ($N>100,k$ – номер замера):
           $$b_i^\text{гир}=\frac{1}{N}\sum_{k=1}^N \omega_{i,k}^\text{изм}$$

    \item Калибровка акселерометра. Для каждой оси акселерометра вычислить следующие два средних значения:
        $$\mu_j=frac{1}{N}\sum_{k=1}^N n_{i,k}^\text{изм},j={1,2},$$
        причем значение $\mu_1$ считается, когда соответствующая ось акселерометра направлена вверх, а значение $\mu_2$ когда ось направлена вниз. За истинные значения $\mu_j$ принимаются соответственно $+g$ и $-g$, где $g\cong 9.81$ м/с$^2$. Используя модель измерений акселерометра и пренебрегая случайной погрешностью, получим следующую систему двух уравнений для каждой оси акселерометра:
        $$g=k_i^\text{акс} (\mu_1-b_i^\text{акс} ),$$
        $$-g=k_i^\text{акс} (\mu_2-b_i^\text{акс} )$$.
        Решив данную систему относительно $k_i^\text{акс}$ и $b_i^\text{акс}$  получим следующий результат:
        $$k_i^\text{акс}=frac{2g}{\mu_2-\mu_1},$$
        $$b_i^\text{акс}=frac{\mu_1+\mu_2}{2}$$.
        В программе выбор оси для снятия замеров осуществляется вводом соответствующего символа с клавиатуры.


\end{enumerate}
\subsubsection*{Этап 2. Программа расчета угла крена}

\begin{enumerate}

    \item Задать найденные в прошлом этапе калибровочные коэффициенты в класс, осуществляющий получение данных с датчика MPU9250 с помощью функций setAccelCal и setGyroBias\_rads.
    \item Найти угловую скорость крена на текущем шаге $k: w_k=-\omega_x$.
    \item Найти измерение угла крена по акселерометру: $z_k=atan2(n_{y'}-n_z)$.
    \item Провести комплексирование измерений по формуле:
        $$\hat x_k=Kz_k+(1-K)(x_k+w_k \Delta t)$$,
        где $\hat x_k$- оценка значения угла крена на текущем шаге, $\Delta t$ – шаг дискретизации, $K$ – коэффициент усиления. Подбор коэффициента усиления осуществляется следующим образом. Коэффициент увеличивают (при подавляется шум измерений) пока алгоритм комплексирования не начнет вносить запаздывания в систему измерений на существенных частотах, т.е. запаздывание не заметно при обычных маневрах самолета. Например, для $\Delta t=20$ мс хорошие результаты дает значение коэффициента $K=0.07-0.13.$
    \item Вывести в последовательный порт значения $z_k$ и $\hat x_k$.

\includeSolutionIfExistsByPath{}
