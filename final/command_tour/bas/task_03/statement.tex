\assignementTitle{}{35}{}
Напишите программу взаимодействия с органами управления БПЛА. Требуется запустить двигатель на 3 сек. и выключить его, отклонить руль высоты в крайнее верхнее положение на 3 сек., затем в крайнее нижнее на 3 сек. и вернуть в нейтральное. Последнее повторить для руля направления и элеронов. Для элеронов первое отклонение должно обеспечить положительный крен. Для руля направления – положительное изменение курса.
\solutionSection 
\begin{enumerate}
    \item Конфигурировать цифровые ножки контроллера согласно следующей таблице.
    \begin{table}
        \center
        \caption{Таблица соответствий каналов управления и цифровых выводов контроллера.}
        \begin{tabular}{|c|c|}
            \hline
            Канал управления&	Цифровой вывод\\
            \hline
            Элероны&	2\\
            \hline
            Руль высоты&	3\\
            \hline
            Двигатель&	5\\
            \hline
            Руль направления&	6\\
        \end{tabular}
    \end{table}
    \item Запустить двигатель на 3 сек. Для этого требуется экспериментально определить значение ШИМ на котором двигатель запускается (1200). С целью уменьшения нагрузки на двигатель, ШИМ увеличивается линейно со временем в течение 3х секунд с стартового значения до максимально разрешенного (1500), после чего ШИМ устанавливается в значение 1000, что соответствует выключению двигателя.
    \item Элероны и рули высоты и направления управляются однотипно. Крайние положения соответствуют значениям ШИМ 1000 и 2000. Значения ШИМ для положительного отклонения управляющих поверхностей определяются экспериментально. В данном случае положительные отклонения соответствуют значению 2000. 
    \item Для перевода поверхностей в нейтральное положение значению ШИМ присваивается 1500.
\includeSolutionIfExistsByPath{}
