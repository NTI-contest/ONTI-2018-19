\section{Введение}
\textit{«Если в мире когда-нибудь появится триллионер, то это будет человек, который эксплуатирует природные ресурсы на астероидах. В них находятся безграничные запасы энергии и ресурсов.» }

— Нил Деграсс Тайсон, астрофизик

Астероиды это остатки от раннего образования Солнечной системы или обломки от разрушения планеты. Десятки тысяч астероидов вращаются вокруг Солнца. В нашей солнечной системе большинство астероидов группируется внутри пояса астероидов, между орбитами Марса и Юпитера, где насчитывается более 1 млн астероидов, около 200 из которых имеют диаметр более 100 км.

Добыча полезных ископаемых на астероидах с помощью роботов — критически важная задача для реализации планов длительных космических полетов на астероиды, Луну и Марс. Хотя внедрение автоматизации в наземной добыче полезных ископаемых шло медленно из-за технических проблем, несколько крупных горнодобывающих компаний, например Rio Tinto, BHP Billiton используют автономное или полуавтономное оборудование и технологии удаленного виртуального управления, которые позволяют горнорабочим управлять оборудованием, находясь на расстоянии тысячи миль от него, что, в принципе, также применимо в космосе (источник: NASA). Кроме того, в добычу полезных ископаемых в космосе вовлечены не только компании. По словам консультанта Navitas, страны Ближнего Востока разрабатывают космические программы и вкладывают средства в зарождающиеся частные космические инициативы по добыче сырья. Это нацелено на то, чтобы дать им опору для создания внеземных запасов воды — вещества, способного служить топливом для космических путешествий, — и других ресурсов, которые могут быть использованы для производства чего-либо в космосе. У ОАЭ и Саудовской Аравии есть космические программы. Саудовская Аравия подписала в 2015 году договор с Россией о сотрудничестве в области освоения космоса. Абу-Даби является инвестором предприятия космического туризма Ричарда Брэнсона, Virgin Galactic. Помимо денег Ближний Восток также обладает выгодным расположением, находясь близко к экватору, что снижает затраты на запуск космических аппаратов из-за вращения земли. Navitas ожидает, что компании запустят спутники, ищущие редкие газы и металлы на астероидах, в ближайшие пять лет, а фактическая добыча произойдет в ближайшие восемь лет (источник: Bloomberg).

По некоторым оценкам минералы астероидного пояса между Марсом и Юпитером могут стоить 700 квинтиллионов долларов США, что составляет 100 миллиардов долларов США для каждого из 7 миллиардов человек на Земле согласно текущим ценам. Джон С. Льюис, автор «Mining the Sky», утверждает, что астероид диаметром 1 км будет иметь массу около 2 миллиардов тонн. В Солнечной системе возможно находится около миллиона астероидов такого размера. По словам Льюиса, один из этих астероидов будет содержать 30 миллионов тонн никеля, 1,5 миллиона тонн металлического кобальта и 7500 тонн платины, причем стоимость одной только платины составляет более 150 миллиардов долларов США (источник: Льюис 1996, Биггс 2013, NASA).

Астероиды содержат воду, которая может стать ключом к тому, чтобы добраться до Марса и исследовать дальний космос. Вода является бесценным товаром в космосе, учитывая потенциальные трудности добычи льда на Марсе и/или возможности привезти астероид на Землю.