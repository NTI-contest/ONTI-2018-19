\subimport{final/command_tour/aero_sys/}{intro.tex}
\subimport{final/command_tour/aero_sys/}{task.tex}
\subimport{final/command_tour/aero_sys/}{infrastructure.tex}
\subimport{final/command_tour/aero_sys/}{realization.tex}

\section{Возможный план работ}

Данный вариант плана работ предлагается организаторами для реализации участниками, однако команда вольна выбирать свою последовательность решения задач. Для реализации задания организаторами предполагается, что участники команды обладают независимыми компетенциями в следующих дисциплинах:
\begin{enumerate}
    \item Электроника - чтение и расчет электронных схем, подключение, пайка, диагностика электронных компонентов и схем; 
    \item Программирование - настройка базового ПО (Windows 10 (license)
    
    Ubuntu 18.04 LTS (freeware), написание и модификация программ под специализированным ПО (ROS Melodic Morena (freeware), Arduino IDE (freeware).
    \item Конструирование - проектирование и расчет конструкций, 3D моделирование, учет компоновки механизмов на готовом изделии.
    \item Сборка - сборка компонентов ровера, доработка конструкционных решений.
\end{enumerate}
Номера в ячейках таблицы №1 соответствуют номерам заданий из состава работ пункта 3. Выполнение задания. Весь день условно разбит на 4 часа. В течение которых предполагается выполнение данных задач.

\begin{center}
    Таблица. №1 Примерный план работ
\end{center}
\begin{longtable}{|p{3.5cm}|p{2.5cm}|p{2.5cm}|p{2.5cm}|p{2.5cm}|}
    \hline
    \textbf{Компетенция} & \multicolumn{4}{|c|}{\textbf{День 1}} \\
    \hline
    Электроника	& 3.1.1 & 3.1.1 & 3.1.2 & 3.1.3 \\
    \hline
    Программирование & 3.2.1 & 3.2.2/3.2.3 & 3.3.1/3.3.2/3.3.3 & 3.3.4 \\
    \hline
    Конструирование & 3.4.1 & 3.4.1 & 3.4.1 & 3.4.1 \\
    \hline
    Сборка & 3.1.1 & 3.1.1 & 3.1.1 & 3.1.4/3.1.5 \\
    \hline
    \textbf{Компетенция}& \multicolumn{4}{|c|}{\textbf{День 2}} \\
    \hline
    Электроника & 3.4.5 & 3.4.5 & 3.4.5 & 3.4.5 \\
    \hline
    Программирование & 3.3.4 & 3.3.5/3.3.6 & 3.4.7 & 3.5.1 \\
    \hline
    Конструирование & 3.4.2 & 3.4.2 & 3.4.3 & 3.4.3 \\
    \hline
    Сборка & 3.4.4 & 3.4.4 & 3.4.4 & 3.4.4 \\
    \hline
    \textbf{Компетенция}&\multicolumn{4}{|c|}{\textbf{День 3}} \\
    \hline
    Электроника & 3.4.6 & 3.4.6 & 3.4.6 & 3.4.6 \\
    \hline
    Программирование & 3.5.1 & 3.5.1 & 	3.5.2 & 	3.5.2 \\
    \hline 
    Конструирование & 3.4.4 & 3.4.4 & 3.4.4 & 3.4.4 \\
    \hline
    Техника & 3.4.4 & 3.4.4 & 3.4.4 & 3.4.4 \\
    \hline
    \textbf{Компетенция}& \multicolumn{4}{|c|}{\textbf{День 4}} \\
    \hline
    Электроника & \multicolumn{4}{l|}{\multirow{4}{*}{\begin{tabular}[c]{@{}l@{}}Проведение испытаний на полигоне и доработка\\ выявленных недостатков в режиме реального времени.\end{tabular}}} \\ \cline{1-1}
        Программирование & \multicolumn{4}{l|}{} \\ \cline{1-1}
        Конструирование  & \multicolumn{4}{l|}{}  \\ \cline{1-1}
        Сборка           & \multicolumn{4}{l|}{} \\
    \hline
\end{longtable}

\section{Наземные испытания (НИ)}

После сборки и тестирования работоспособности силами команды ровер предоставляется организаторам для наземных испытаний. По готовности команда Участников заявляет готовность к наземным испытаниям Организаторам в устной форме. НИ проводятся для всех участников последовательно в порядке живой очереди. Организаторы прекращают прием заявок на НИ за 30 минут до окончания работ каждого дня. Организаторы проверяют работоспособность ровера в соответствии с таблицей (см. ниже) и выставляют оценки каждой из команд. По результату проверки узла организаторы выставляют баллы за работоспособность узла. За успешную проверку баллы начисляются, за непрохождение попытки проверки узла баллы не начисляются. Однако, не все узлы нужно предоставлять на НИ одновременно. Участники сами заявляют те узлы ровера которые они хотят предоставить на проверку. Для участия в этапе НИ любой участник команды устно подает заявку организаторам и в порядке общей очереди предоставляет ровер на НИ. Прохождение этапа НИ не является необходимым для дальнейшего выполнения задания, но дает возможность заработать баллы для победы. Команда Участников не сдавшая подзадачи НИ в день сдачи данного набора подзадач (см. таблицу №2 ниже) может сдать их в следующие дни, но потеряет при этом 30\% баллов полученных от успешной сдачи подзадачи. 

\begin{center}
    Таблица. №2 Подзадачи этапа Наземные испытания
\end{center}
\begin{longtable}{|p{0.5cm}|p{3.5cm}|p{10.5cm}|}
    \hline
    № & Название Подзадачи & Задачи \\
    \hline
    1. & Сборка ровера и Наземные испытания 27-28.03.2019 &	Пройти НИ узлов в формате демонстрации работоспособности соответствующего узла Организаторам:
    \begin{itemize}
        \item Демонстрация связи с ровером (Команда демонстрирует возможность зайти на Raspberiy ровера по SSH)
        \item Демонстрация работы джойстика с тестовой прошивкой (Команда демонстрирует работоспособность джойстика в любой конфигурации, т.е. нажатия/перемещения любого количества кнопок/слайдеров/джойстиков джойстика передаются на компьютер и там детектируются)
        \item Демонстрация работы камеры (Команда демонстрирует изображение с камеры на компьютере посредством какого-либо модуля ROS)
        \item Демонстрация работы лидара (Команда демонстрирует изображение с лидара на компьютере посредством какого-либо модуля ROS)
    \end{itemize} \\
    \hline
    2. & Настройка ровера и НИ 28-29.03.2019 & 
    \begin{itemize}
        \item Демонстрация движения и поворота ровера. (Команда демонстрирует движение ровера по прямой на расстояние не менее 1м. и поворот на месте или в движении на угол не менее 25 градусов)
        \item Демонстрация работы джойстика с написанной командой Участников прошивкой (Команда демонстрирует работу джойстика передавая и обрабатывая нажатия/перемещения кнопок/слайдеров/джойстиков джойстика в ROS)
        \item Демонстрация движения и поворота ровера по командам с джойстика (Команда демонстрирует движение ровера по прямой на расстояние не менее 1м. и поворот на месте или в движении на угол не менее 25 градусов по командам джойстика путем передачи команд в ROS ровера)
    \end{itemize} \\
    \hline
    3. & Сборка полезной нагрузки и НИ 29.03.2019 & 
    \begin{itemize} 
        \item Демонстрация работы полезной нагрузки “бурение”. (Команда должна продемонстрировать бурение отверстия глубиной не менее 20 мм, при помощи полезной нагрузки зафиксированной и смонтированной на ровере в такой же конфигурации, в которой команда планирует выполнять испытания на полигоне)
        \item Демонстрация работы полезной нагрузки “забор жидкости” (Команда должна продемонстрировать забор жидкости из открытого контейнера на глубине не менее 40 мм. от поверхности в объеме не менее 10 и не более 50 мл., при помощи полезной нагрузки зафиксированной и смонтированной на ровере в такой же конфигурации, в которой команда планирует выполнять испытания на полигоне)
        \item Демонстрация работы полезной нагрузки “система распределения и хранения жидкости” (Команда должна продемонстрировать распределение и хранение жидкости между двумя контейнерами в объеме не менее 10 и не более 50 мл, при помощи полезной нагрузки зафиксированной и смонтированной на ровере в такой же конфигурации, в которой команда планирует выполнять испытания на полигоне)
    \end{itemize} \\
    \hline
\end{longtable}

\section{Испытаний на полигоне}

После истечения времени на лабораторную часть задания или досрочно по решению команды участников, команды переходят к этапу испытаний на полигоне. 

\begin{enumerate}
    \item[6.1] Автостарт из модуля
    \begin{enumerate}
        \item[6.1.1] Команда участников вместе с организаторами устанавливает ровер в спускаемый модуль и переносит его в специально отведенное место на полигоне.   
        \item[6.1.2] Все представители команды участников покидают полигон и переходят в огороженную зону из которой осуществляется управление  ровером.
        \item[6.1.3] Судья на полигоне открывает аппарель спускаемого модуля после чего ровер в автономном режиме должен определить, что аппарель открылась и выехать по ней из модуля. Выезд из модуля засчитывается если ровер пересек обоими осями задних колес линию №1 нанесенную на поверхности полигона перед аппарелью
        \item[6.1.4] После пересечения линии №1 ровер должен самостоятельно остановиться и дождаться перехода на ручное управление командой.
        \item[6.1.5] Дальнейшие передвижения по полигону ровер должен производить по командам Участников опираясь на данные камер/датчиков/лидара.
        \item[6.1.6] Задача команды Участников запрограммировать ровер на автономное принятие решение на старт и движение до остановки. 
        \item[6.1.7] В зоне высадки ровер управляется в режиме телеуправления (на джойстике)
    \end{enumerate}
    \item[6.2] Движение в зоне навигации
    \begin{enumerate} 
	    \item[6.2.1] В зоне навигации эмулируется реальное телеуправление планетоходом из центра управления (ЦУ) на Земле. Т.е. движение ровера осуществляется по пакету команд на движение сформированному ЦУ в рамках одного сеанса связи.
	    \item[6.2.2]  Отправка каждого пакета команд на движение, фиксируется судьей в ЦУ.
        \item[6.2.3] Отправка нового пакета команд на движение возможна, только после полного выполнения предыдущего пакета команд или в случае аварийной остановки при столкновении с препятствием. 
        \item[6.2.4] Задача команды Участников провести ровер через зону навигации за возможно меньшее количество сеансов связи без столкновения с препятствиями.
        \item[6.2.5] Подзадача считается выполненной при остановке ровера в зоне бурения всеми колесами.
        \item[6.2.6] По решению команды Участников можно отказаться от прохождения зоны навигации и проходить ее в режиме телеуправления (на джойстике). При этом баллы за подзадачу “Проезд зоны навигации” не начисляются.
    \end{enumerate}
    \item[6.3] Бурение
    \begin{enumerate}
        \item[6.3.1] Бурение должно производится при помощи разработанного командой участников оборудования полезной нагрузки.
        \item[6.3.2] Бурение должно осуществляться внутри зоны бурения.
        \item[6.3.3] Бур должен попадать в точку бурения. Диаметр точки 100 мм.
        \item[6.3.4] Задача команды Участников пробурить отверстие над контейнером с жидкостью для последующего забора жидкости.
    \end{enumerate}
    \item[6.4] Сбор образцов жидкости
    \begin{enumerate}
        \item[6.4.1] Сбор жидкости производится из 2-х точек. 
        \item[6.4.2] Жидкости в разных точках имеют разные физико/химические характеристики и их смешивание не допускается
        \item[6.4.3] Объем жидкости необходимый для сбора не менее 10 мл и не более 50 мл. из каждой точки.
    \end{enumerate} 
    \item[6.5] Сохранение образцов жидкости
    \begin{enumerate} 
        \item[6.5.1] Организаторы дают команде Участников 2 контейнера для хранения жидкости. Создание системы распределения жидкости по контейнерам в процессе бурения и крепление контейнеров на ровере в процессе возврата к модулю определяется конструкторским решением команды участников. 
        \item[6.5.2] Система забора и хранения жидкости должна исключать возможность пролития жидкости на пол полигона и поверхности ровера
        \item[6.5.3] Пролитие жидкости вне системы забора и хранения - пенализируется
        \item[6.5.4] После возвращения ровера к посадочному модулю, организаторы проверяют достаточность собранных объемов жидкости см п.п. 6.4.3
    \end{enumerate}
\end{enumerate}

\subsubsection*{Оценка испытаний на полигоне}

\begin{center}
    Таблица. №3 Оценка испытаний на полигоне
\end{center}
\begin{tabular}{|p{0.5cm}|p{3.5cm}|p{10.5cm}|}
    \hline
    № & Подзадача & Задачи \\
    \hline
    2. & Автостарт из модуля & После открывания аппарели посадочного модуля, ровер должен начать выполнение задания. Задание считается выполненным если ровер пересекает линию №1.
    Касание бортов посадочного модуля - штраф 
    Пересечение бокового торца аппарели - штраф Касание ограждения полигона - штраф
    Выезд за пределы полигона осью хотя бы одного колеса - незачет попытки \\
    \hline
    3. & Проезд зоны навигации & Преодоление зоны навигации в режиме пакетной загрузки команд на движение.

    Загрузка каждого пакета команд на движение ровера в ровер - штраф.
    Касание препятствий в зоне навигации - штраф за каждое касание
    Сбитие препятствий в зоне навигации - штраф за каждое сбитие
    Касание границы зоны навигации - штраф
    Выезд за пределы полигона осью хотя бы одного колеса - незачет попытки
    Движение в зоне навигации в режиме телеуправления - не получение баллов за подзадачу “Проезд зоны навигации” \\
    \hline
    4. & Бурение & 	В ручном режиме довести ровер от края зоны навигации до точки бурения и пробурить поверхность. 

    Бурение вне выделенной точки для бурения - штраф 
    Бурение вне выделенной зоны для бурения - штраф
    Отсутствие образца жидкости в контейнере на ровере из пробуренного отверстия - незачет подзадачи “Бурение” \\
    \hline
            5. & Сбор образцов
    (за каждую точку) & Забрать образцы жидкости в ровер

    Забор только одного образца - штраф
    Отсутствие обоих образцов воды на финише - незачет подзадачи “Сбор образцов” \\
    \hline
\end{tabular}

\subsubsection*{Время на выполнение заданий}

\begin{longtable}{|p{0.5cm}|p{3.5cm}|p{3.5cm}|p{6.8cm}|}
    \hline
    & Название подзадачи & Время & Комментарии \\
    \hline
    1. & Наземные испытания & Вечером каждого дня см. таблицу раздела Наземные испытания & В случае более ранней готовности команда может перейти к каждому следующему под-этапу НИ проверки заранее \\
    \hline
    2. & Автостарт из модуля & 1 минута с момента открытия аппарели & Не выезд из модуля в течение 1-й минуты после открытия аппарели - незачет попытки. После выезда ровера из модуля и перехода на ручное управление начинается отсчет времени на следующие задания \\
    \hline
    3. & \begin{tabular}[c]{@{}l@{}}Проезд\\ зоны навигации\end{tabular} & \multirow{4}{*}{10 минут} & \multirow{4}{*}{\begin{tabular}[c]{@{}l@{}}На все подзадачи время общее. По\\ истечении 10 минут засчитываются \\ баллы за полностью законченные\\ задачи. За задачи законченные не\\ полностью баллы не засчитываются,\\ но и не отнимаются.\end{tabular}} \\ \cline{1-2}
    4. & Бурение                                                         &                           & \\ \cline{1-2}
    5. & Сбор образцов                                                   &                           & \\ \cline{1-2}
    6. & Возврат к посадочному модулю через зону навигации               &                           & \\ \hline
\end{longtable}

\subsubsection*{Подведение итогов}

Итоговый результат определяется суммированием лучших попыток Испытаний на полигоне и наземных испытаний. Количество зачетных попыток ограничено (не более 2-х), и при использовании каждой последующей попытки, результат данной попытки уменьшается на 10\%. 

\subsubsection*{Общие правила}

Для всех задач и подзадач используются общие дополнительные правила работы на полигоне

\textit{Тренировки и тестирование}

Как и в случае всей робототехники космического назначения, команда НЕ может производить тестирование оборудования и испытания на полигоне, однако приветствуется использования любых других доступных способов тестирования не нарушающих общественный порядок.

\textit{Дополнительные требования по технике безопасности} 
Модель ровера  не должна рассыпаться в руках (выдерживать падение с высоты 10 см на поверхность пола)

На ровере должен быть установлен аварийный выключатель, он может использоваться как основной. Расположение аварийного выключателя должно быть выбрано с учетом его легкой доступности в нештатной ситуации.

\textit{Общие штрафы}

\begin{center}
    \begin{tabular}{|p{0.5cm}|p{8.5cm}|p{3.5cm}|}
        \hline
        № & Штрафы & Баллы - Штрафы \\
        \hline
        1 & Разрядка аккумулятора во время сдачи задачи & 	Незачет попытки \\
        \hline
        2 & Ровер не начал выполнение задания после открытия аппарели посадочного модуля & Незачет попытки \\
        \hline
        3 & Не уложились в контрольное время выполнения попытки & Незачет попытки \\
        \hline
        4 & Бурение вне зоны бурения & 30 баллов \\
        \hline
    \end{tabular}
\end{center}

\subimport{final/command_tour/aero_sys/}{solution.tex}