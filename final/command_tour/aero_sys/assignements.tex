\subimport{final/command_tour/aero_sys/}{intro.tex}
\subimport{final/command_tour/aero_sys/}{task.tex}
\subimport{final/command_tour/aero_sys/}{infrastructure.tex}


\section{Выполнение задания}

В данном разделе приведены общие рекомендации по выполнению задания. Сборка ровера и конструирование полезной нагрузки должны производиться каждой командой участников индивидуально без посторонней помощи.

\textbf{Проверка ровера и джойстика}

При проверке работоспособности, настройке и программировании ровера рекомендуется пользоваться инструкцией на плату управления:
\url{https://voltbro.gitbook.io/turtlebro/}

При проверке работоспособности, настройке и программировании джойстика рекомендуется пользоваться инструкцией на джойстик:
\url{https://github.com/voltbro/BroJoystick}

Общие рекомендации по управлению и программированию ровера:
\url{http://docs.voltbro.ru/starting-ros/}

\putImgWOCaption{12cm}{2}

\textbf{Конструирование полезной нагрузки}

При конструировании полезной нагрузки необходимо учитывать следующие параметры:
\begin{enumerate}
    \item Бурение осуществляется сверлом Д3мм с зенкером Д16мм
    \item Глубина бурения не менее 20 мм. от поверхности.
    \item Глубина опускания среза шланга Д4мм для забора воды не менее 40 мм. от поверхности.
    \item Расположение точек крепления полезной нагрузки возможно на передней панели ровера в пределах от 125 мм до 235 мм от поверхности. И не более 100 мм вперед от переднего среза рамы.
    \item Располагать точки крепления полезной нагрузки можно как на самой передней панели, так и на раме ровера. Крепление к панели может осуществляться винтами М4 на М4 гайку через отверстия в панели. Крепление к раме осуществляется винтами М4х8 или М4х10 на Т-гайку М4.
    \item Чертежи для лазерной резки принимаются в формате .DXF (Autodesk Drawing eXchange Format)
    \item Чертежи для 3D-принтеров принимаются в формате в формате .STL (“StereoLithography”)
    \item Имеет смысл взять с собой и использовать для моделирования и черчения собственные ноутбуки с предустановленными и удобными для использования командой Участников программами моделирования.
\end{enumerate}

\textbf{Программирование и алгоритмизация}

При разработки модели управления ровером необходимо учитывать что движение в зоне навигации происходит в пакетном режиме передачи команд. Под пакетным режимом подразумевается передача последовательности команд на движение в рамках одной группы. Группы команды на движение могут передаваться через консоль Ubuntu, через консоль ROS, при помощи встроенных пакетов автонавигации ROS или посредством выполнения программы на Питоне. Одна группа команд на движение считается выполненной с момента передачи ее описанными выше способами на ровер до завершения ровером движения.  Реализация той или иной формы пакетного режима выбирается командой Участников исходя из собственных предпочтений и удобства.

При программировании джойстика и размещении датчиков и камер необходимо учитывать, что в режиме телеуправления оператор ровера не видит полигон и пользуется только данными с камер, датчиков и лидара.

\subsubsection*{}{Состав работ}

\begin{enumerate}
    \item Сборка ровера
    \begin{enumerate}
        \item Изготовление кабелей подключения моторов
        \item Изготовление кабелей подключения лидара
        \item Изготовление кабелей питания
        \item Установка выключателя
        \item Установка колес
    \end{enumerate}
    \item Проверка работоспособности ровера как ноды Robot Operating System (ROS)
    \begin{enumerate}
        \item Установка связи с ровером 
        \item Проверка передачи команд движения и остановки 
        \item Проверка передачи видео 
    \end{enumerate}
    \item Настройка пульта ДУ
    \begin{enumerate}
        \item Изучение документации
        \item Загрузка тестовой прошивки
        \item Проверка работоспособности
        \item Разработка прошивки для использования в качестве управляющего элемента ROS
        \item Подключение к ПК как устройство ввода для ROS
        \item Проверка работы функций 
    \end{enumerate}
    \item Конструирование полезной нагрузки (ПН)
    \begin{enumerate}
        \item Проектирование ПН позволяющей выполнить задачу бурения
        \item Проектирование ПН позволяющей выполнить задачу забора образцов жидкости
        \item Проектирование ПН позволяющей отделять образцы собранные в разных точках по разным контейнерам
        \begin{itemize}
            \item Изготовление чертежей деталей в соответствии с доступными технологиями изготовления
            \item Предзащита конструкции полезной нагрузки перед организаторами, для проверки ее на безопасность и реализуемость. Предзащита конструкции ПН проходит по предоставленным командой Участников эскизам и чертежам конструкции. Организаторы вправе отказать команде Участников в изготовлении деталей по чертежам или выдаче оборудования для реализации ПН, если по мнению организаторов предложенная командой Участников ПН не соответствует требованиям безопасности. 
            \item Печать деталей полезной нагрузки на 3D-принтере
            \textbf{Внимание! В течение одного дня каждая команда имеет возможность пользоваться 3D-принтером не более 2-х часов в день. Неиспользованное время в рамках одного дня на следующий день не переносится. Очередность использования командами Участников 3D-принтеров будет определена графиком.}
            \item Сдача чертежей для лазерной резки на изготовление
            \textbf{Внимание! После сдачи чертежей на изготовление, детали будут изготовлены в течение 12 часов.} Таким образом все чертежи на резку желательно сдавать до вечера второго дня (27.03.2019), чтобы у команды Участников было время на внесение изменений в проекты в случае неудачи.
        \end{itemize}
        \item Сборка ПН
        \item Проектирование схем подключения ПН
        \item Монтаж электрической схемы подключения ПН
        \item Написание ПО для оперирования ПН
    \end{enumerate}
    \item Разработка модели управления ровером
    \begin{enumerate}
	    \item Разработка алгоритма создания и загрузки пакета команд движения ровером или настройка системы автономной навигации
	    \item По желанию команды Участников разработка дополнительных алгоритмов управления ровером: предотвращения столкновения с препятствием, автономной навигации, логирования пройденного пути, автоматического возврата по своему следу и пр. 
    \end{enumerate}
\end{enumerate}

\section{Возможный план работ}

Данный вариант плана работ предлагается организаторами для реализации участниками, однако команда вольна выбирать свою последовательность решения задач. Для реализации задания организаторами предполагается, что участники команды обладают независимыми компетенциями в следующих дисциплинах:
\begin{enumerate}
    \item Электроника - чтение и расчет электронных схем, подключение, пайка, диагностика электронных компонентов и схем; 
    \item Программирование - настройка базового ПО (Windows 10 (license)
    
    Ubuntu 18.04 LTS (freeware), написание и модификация программ под специализированным ПО (ROS Melodic Morena (freeware), Arduino IDE (freeware).
    \item Конструирование - проектирование и расчет конструкций, 3D моделирование, учет компоновки механизмов на готовом изделии.
    \item Сборка - сборка компонентов ровера, доработка конструкционных решений.
\end{enumerate}
Номера в ячейках таблицы №1 соответствуют номерам заданий из состава работ пункта 3. Выполнение задания. Весь день условно разбит на 4 часа. В течение которых предполагается выполнение данных задач.

\begin{table}
    \caption{Примерный план работ}
    \begin{longtable}
Компетенция	День 1
Электроника	3.1.1	3.1.1	3.1.2	3.1.3
Программирование	3.2.1	3.2.2/3.2.3	3.3.1 /3.3.2 /3.3.3	3.3.4
Конструирование	3.4.1	3.4.1	3.4.1	3.4.1
Сборка	3.1.1	3.1.1	3.1.1	3.1.4/3.1.5
Компетенция	День 2
Электроника	3.4.5	3.4.5	3.4.5	3.4.5
Программирование	3.3.4	3.3.5/3.3.6	3.4.7	3.5.1
Конструирование	3.4.2	3.4.2	3.4.3	3.4.3
Сборка	3.4.4	3.4.4 	3.4.4	3.4.4 
Компетенция	День 3
Электроника	3.4.6	3.4.6	3.4.6	3.4.6
Программирование	3.5.1	3.5.1		3.5.2		3.5.2
Конструирование	3.4.4	3.4.4	3.4.4	3.4.4
Техника	3.4.4	3.4.4	3.4.4	3.4.4
Компетенция	День 4
Электроника	Проведение испытаний на полигоне и доработка выявленных недостатков в режиме реального времени.
Программирование	
Конструирование	
Сборка	
Таблица. №1 Примерный план работ
5. Наземные испытания (НИ)
После сборки и тестирования работоспособности силами команды ровер предоставляется организаторам для наземных испытаний. По готовности команда Участников заявляет готовность к наземным испытаниям Организаторам в устной форме. НИ проводятся для всех участников последовательно в порядке живой очереди. Организаторы прекращают прием заявок на НИ за 30 минут до окончания работ каждого дня. Организаторы проверяют работоспособность ровера в соответствии с таблицей (см. ниже) и выставляют оценки каждой из команд. По результату проверки узла организаторы выставляют баллы за работоспособность узла. За успешную проверку баллы начисляются, за непрохождение попытки проверки узла баллы не начисляются. Однако, не все узлы нужно предоставлять на НИ одновременно. Участники сами заявляют те узлы ровера которые они хотят предоставить на проверку. Для участия в этапе НИ любой участник команды устно подает заявку организаторам и в порядке общей очереди предоставляет ровер на НИ. Прохождение этапа НИ не является необходимым для дальнейшего выполнения задания, но дает возможность заработать баллы для победы. Команда Участников не сдавшая подзадачи НИ в день сдачи данного набора подзадач (см. таблицу №2 ниже) может сдать их в следующие дни, но потеряет при этом 30% баллов полученных от успешной сдачи подзадачи. 

№	Название Подзадачи	Задачи
1.	Сборка ровера и Наземные испытания
27-28.03.2019	Пройти НИ узлов в формате демонстрации работоспособности соответствующего узла Организаторам:
    \item Демонстрация связи с ровером (Команда демонстрирует возможность зайти на Raspberiy ровера по SSH)
    \item Демонстрация работы джойстика с тестовой прошивкой (Команда демонстрирует работоспособность джойстика в любой конфигурации, т.е. нажатия/перемещения любого количества кнопок/слайдеров/джойстиков джойстика передаются на компьютер и там детектируются)
    \item Демонстрация работы камеры (Команда демонстрирует изображение с камеры на компьютере посредством какого-либо модуля ROS)
    \item Демонстрация работы лидара (Команда демонстрирует изображение с лидара на компьютере посредством какого-либо модуля ROS)
2.	Настройка ровера и НИ
28-29.03.2019 	    \item Демонстрация движения и поворота ровера. (Команда демонстрирует движение ровера по прямой на расстояние не менее 1м. и поворот на месте или в движении на угол не менее 25 градусов)
    \item Демонстрация работы джойстика с написанной командой Участников прошивкой (Команда демонстрирует работу джойстика передавая и обрабатывая нажатия/перемещения кнопок/слайдеров/джойстиков джойстика в ROS)
    \item Демонстрация движения и поворота ровера по командам с джойстика (Команда демонстрирует движение ровера по прямой на расстояние не менее 1м. и поворот на месте или в движении на угол не менее 25 градусов по командам джойстика путем передачи команд в ROS ровера)
3.	Сборка полезной нагрузки и НИ
29.03.2019 	    \item Демонстрация работы полезной нагрузки “бурение”. (Команда должна продемонстрировать бурение отверстия глубиной не менее 20 мм, при помощи полезной нагрузки зафиксированной и смонтированной на ровере в такой же конфигурации, в которой команда планирует выполнять испытания на полигоне)
    \item Демонстрация работы полезной нагрузки “забор жидкости”
(Команда должна продемонстрировать забор жидкости из открытого контейнера на глубине не менее 40 мм. от поверхности в объеме не менее 10 и не более 50 мл., при помощи полезной нагрузки зафиксированной и смонтированной на ровере в такой же конфигурации, в которой команда планирует выполнять испытания на полигоне)
    \item Демонстрация работы полезной нагрузки “система распределения и хранения жидкости” (Команда должна продемонстрировать распределение и хранение жидкости между двумя контейнерами в объеме не менее 10 и не более 50 мл, при помощи полезной нагрузки зафиксированной и смонтированной на ровере в такой же конфигурации, в которой команда планирует выполнять испытания на полигоне)
 Таблица. №2 Подзадачи этапа Наземные испытания
6. Испытаний на полигоне
После истечения времени на лабораторную часть задания или досрочно по решению команды участников, команды переходят к этапу испытаний на полигоне. 

6.1 Автостарт из модуля
6.1.1 Команда участников вместе с организаторами устанавливает ровер в спускаемый модуль и переносит его в специально отведенное место на полигоне.
6.1.2 Все представители команды участников покидают полигон и переходят в огороженную зону из которой осуществляется управление  ровером.
6.1.3 Судья на полигоне открывает аппарель спускаемого модуля после чего ровер в автономном режиме должен определить, что аппарель открылась и выехать по ней из модуля. Выезд из модуля засчитывается если ровер пересек обоими осями задних колес линию №1 нанесенную на поверхности полигона перед аппарелью
6.1.4 После пересечения линии №1 ровер должен самостоятельно остановиться и дождаться перехода на ручное управление командой.
6.1.5 Дальнейшие передвижения по полигону ровер должен производить по командам Участников опираясь на данные камер/датчиков/лидара.
6.1.6 Задача команды Участников запрограммировать ровер на автономное принятие решение на старт и движение до остановки. 
6.1.7 В зоне высадки ровер управляется в режиме телеуправления (на джойстике)

6.2 Движение в зоне навигации
	6.2.1 В зоне навигации эмулируется реальное телеуправление планетоходом из центра управления (ЦУ) на Земле. Т.е. движение ровера осуществляется по пакету команд на движение сформированному ЦУ в рамках одного сеанса связи.
	6.2.2  Отправка каждого пакета команд на движение, фиксируется судьей в ЦУ.
6.2.3 Отправка нового пакета команд на движение возможна, только после полного выполнения предыдущего пакета команд или в случае аварийной остановки при столкновении с препятствием. 
6.2.4 Задача команды Участников провести ровер через зону навигации за возможно меньшее количество сеансов связи без столкновения с препятствиями.
6.2.5 Подзадача считается выполненной при остановке ровера в зоне бурения всеми колесами.
6.2.6 По решению команды Участников можно отказаться от прохождения зоны навигации и проходить ее в режиме телеуправления (на джойстике). При этом баллы за подзадачу “Проезд зоны навигации” не начисляются.
6.3 Бурение
6.2.1 Бурение должно производится при помощи разработанного командой участников оборудования полезной нагрузки.
6.2.2 Бурение должно осуществляться внутри зоны бурения.
6.2.3 Бур должен попадать в точку бурения. Диаметр точки 100 мм.
6.2.4 Задача команды Участников пробурить отверстие над контейнером с жидкостью для последующего забора жидкости.
6.3 Сбор образцов жидкости
6.3.1 Сбор жидкости производится из 2-х точек. 
6.3.2 Жидкости в разных точках имеют разные физико/химические характеристики и их смешивание не допускается
6.3.3 Объем жидкости необходимый для сбора не менее 10 мл и не более 50 мл. из каждой точки.
6.4 Сохранение образцов жидкости
6.4.1 Организаторы дают команде Участников 2 контейнера для хранения жидкости. Создание системы распределения жидкости по контейнерам в процессе бурения и крепление контейнеров на ровере в процессе возврата к модулю определяется конструкторским решением команды участников. 
6.4.2 Система забора и хранения жидкости должна исключать возможность пролития жидкости на пол полигона и поверхности ровера
6.4.3 Пролитие жидкости вне системы забора и хранения - пенализируется
6.4.4 После возвращения ровера к посадочному модулю, организаторы проверяют достаточность собранных объемов жидкости см п.п. 6.3.3
Оценка испытаний на полигоне
№	Подзадача	Задачи
2.	Автостарт из модуля	После открывания аппарели посадочного модуля, ровер должен начать выполнение задания. Задание считается выполненным если ровер пересекает линию №1.

Касание бортов посадочного модуля - штраф 
Пересечение бокового торца аппарели - штраф Касание ограждения полигона - штраф
Выезд за пределы полигона осью хотя бы одного колеса - незачет попытки
3.	Проезд зоны навигации	Преодоление зоны навигации в режиме пакетной загрузки команд на движение.

Загрузка каждого пакета команд на движение ровера в ровер - штраф.
Касание препятствий в зоне навигации - штраф за каждое касание
Сбитие препятствий в зоне навигации - штраф за каждое сбитие
Касание границы зоны навигации - штраф
Выезд за пределы полигона осью хотя бы одного колеса - незачет попытки
Движение в зоне навигации в режиме телеуправления - не получение баллов за подзадачу “Проезд зоны навигации”
4.	Бурение
	В ручном режиме довести ровер от края зоны навигации до точки бурения и пробурить поверхность. 

Бурение вне выделенной точки для бурения - штраф 
Бурение вне выделенной зоны для бурения - штраф
Отсутствие образца жидкости в контейнере на ровере из пробуренного отверстия - незачет подзадачи “Бурение”
5.	Сбор образцов
(за каждую точку)	Забрать образцы жидкости в ровер

Забор только одного образца - штраф
Отсутствие обоих образцов воды на финише - незачет подзадачи “Сбор образцов”
Таблица. №3 Оценка испытаний на полигоне
Время на выполнение заданий
 	Название подзадачи	Время	Комментарии
1.	Наземные испытания	Вечером каждого дня см. таблицу раздела Наземные испытания	В случае более ранней готовности команда может перейти к каждому следующему под-этапу НИ проверки заранее
2.	Автостарт из модуля	1 минута с момента открытия аппарели	Не выезд из модуля в течение 1-й минуты после открытия аппарели - незачет попытки. После выезда ровера из модуля и перехода на ручное управление начинается отсчет времени на следующие задания
3.	Проезд зоны навигации
	10 минут	На все подзадачи время общее.По истечении 10 минут засчитываются баллы за полностью законченные задачи. За задачи законченные не полностью баллы не засчитываются, но и не отнимаются.  

4.	Бурение		
5.	Сбор образцов		
6.	Возврат к посадочному модулю через зону навигации		

Подведение итогов
Итоговый результат определяется суммированием лучших попыток Испытаний на полигоне и наземных испытаний. Количество зачетных попыток ограничено (не более 2-х), и при использовании каждой последующей попытки, результат данной попытки уменьшается на 10%. 
Общие правила
Для всех задач и подзадач используются общие дополнительные правила работы на полигоне
Тренировки и тестирование
Как и в случае всей робототехники космического назначения, команда НЕ может производить тестирование оборудования и испытания на полигоне, однако приветствуется использования любых других доступных способов тестирования не нарушающих общественный порядок.
Дополнительные требования по технике безопасности 
Модель ровера  не должна рассыпаться в руках (выдерживать падение с высоты 10 см на поверхность пола)

На ровере должен быть установлен аварийный выключатель, он может использоваться как основной. Расположение аварийного выключателя должно быть выбрано с учетом его легкой доступности в нештатной ситуации.

Общие штрафы

№	Штрафы	Баллы - Штрафы
1	Разрядка аккумулятора во время сдачи задачи 	Незачет попытки
2	Ровер не начал выполнение задания после открытия аппарели посадочного модуля	Незачет попытки

3	Не уложились в контрольное время выполнения попытки	Незачет попытки
4	Бурение вне зоны бурения	30 баллов

