\section{Инфраструктура}

Задача решается на искусственном полигоне имитирующем поверхность условной планеты. Во время решения задачи команда не имеет визуального контакта с полигоном. Подготовка к решению задач ведется в лабораториях.

Организаторы создают три зоны: 

\begin{enumerate}
    \item Лабораторная зона
    \item Зона командного пункта
    \item Зона полигона
\end{enumerate}

\subsubsection*{Базовый набор для каждой команды-участника}
\begin{itemize}
    \item Набор для сборки ходовой платформы ровера-планетохода и система управления на основе  Robot Operating System (ROS)
    \item Механические и электронные компоненты для конструирования полезной нагрузки (сервоприводы, насос, элементы крепления, крепеж, провода, контейнеры для хранения жидкости и т.д.)
    \item Паяльное оборудование
    \item Ноутбук с операционной системой Ubuntu и установленной ROS и необходимыми пакетами (freeware)*
    \item Ноутбук с операционной системой Windows и установленным Arduino IDE (freeware), Autodesk Inventor (Educational license), Adobe Acrobat (freeware).*
    \item Набор ручного инструмента (бокорезы, пинцет, кусачки, плоскогубцы, макетный нож, набор отверток)
    \item Комплект расходных материалов (флюс, припой, изолента, монтажные, провода)
\end{itemize}

* Команда может использовать свои ноутбуки без ограничений.

Оборудование общего пользования на площадке
\begin{itemize}
    \item Элементы питания 18650 не менее 12 шт на команду
    \item Зарядные устройства не менее одного на 3 команды
    \item Расширенный набор ручного инструмента 
    \item Шуруповерт не менее 1шт на 5 команд
    \item Набор сверл не менее 1шт на 5 команд
    \item Комплект расходных материалов 
    \item Комплект крепежа
    \item 3д принтеры (1 шт на 2 команды), с ПО и ПК
    \item Станок лазерной резки (1 шт на 10 команд) с ПО и ПК
    \item Удлинители не менее 1 на команду
    \item Локальная сеть с интернетом
\end{itemize}

Программное обеспечение

Перечислено только специфическое программное обеспечение, предполагая наличие стандартного ПО:
\begin{itemize}
    \item Windows 10 (license)
    \item Ubuntu 18.04 LTS (freeware)
    \item ROS Melodic Morena (freeware)
    \item Autodesk Inventor 18 (Education license or trial)
    \item Adobe Acrobat (freeware)
    \item Arduino IDE (freeware)
\end{itemize}

Материалы и оборудование выдаваемое командам Участников по их требованию в случае успешного прохождения защиты проектов Полезной нагрузки (на одну команду):
\begin{itemize}
    \item сервопривод малый - 3 шт.
    \item сервопривод большой - 4шт.
    \item насос перистальтический 12 в - 1 шт.
    \item трубка для воды - 1 шт
    \item уголки металлические - 2 шт.
    \item листы металлические перфорированные - 3 шт.
    \item оргстекло 3мм
    \item мотор привода сверла - 1шт.
    \item сверло с зенкером - 1шт.
\end{itemize}

\subsubsection*{Полигон}

\putImgWOCaption{12cm}{1}

Полигон представляет собой искусственное покрытие с разметкой по зонам и местам для бурения

Описание ключевых зон полигона:
\begin{itemize}
    \item \textbf{Зона высадки} - Перед выполнением задания ровер участников загружается в посадочный модуль и выставляется в заранее известной точке на полигоне. Ориентация аппарели посадочного модуля одинакова для всех участников.
    \item \textbf{Зона навигации} - большая часть полигона представляет собой имитацию инопланетного грунта. В этой зоне движение ровера происходит только в режиме пакетной загрузки команд управления или в режиме автонавигации. В зоне навигации присутствуют препятствия. Касание/сбитие препятствия пенализируется.
    \item \textbf{Зона бурения} - помеченная зона на полигоне. Отдельно контрастной краской выделена область бурения и забора воды внутри зоны. Диаметр области бурения 100 мм. Бурение вне зоны бурения пенализируется. Бурение в зоне бурения, но вне точки бурения не пенализируется, но бессмысленно. На каждой из точек забора воды присутствует контейнер с жидкостью, с разным качеством воды. В первой точке забор жидкости возможен без бурения, т.к. контейнер открыт сверху. Во второй точке контейнер закрыт пластом породы, которую необходимо пробурить для осуществления забора жидкости.
\end{itemize}