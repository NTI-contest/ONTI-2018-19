\section{Выполнение задания}

В данном разделе приведены общие рекомендации по выполнению задания. Сборка ровера и конструирование полезной нагрузки должны производиться каждой командой участников индивидуально без посторонней помощи.

\textbf{Проверка ровера и джойстика}

При проверке работоспособности, настройке и программировании ровера рекомендуется пользоваться инструкцией на плату управления:
\url{https://voltbro.gitbook.io/turtlebro/}

При проверке работоспособности, настройке и программировании джойстика рекомендуется пользоваться инструкцией на джойстик:
\url{https://github.com/voltbro/BroJoystick}

Общие рекомендации по управлению и программированию ровера:
\url{http://docs.voltbro.ru/starting-ros/}

\putImgWOCaption{12cm}{2}

\textbf{Конструирование полезной нагрузки}

При конструировании полезной нагрузки необходимо учитывать следующие параметры:
\begin{enumerate}
    \item Бурение осуществляется сверлом Д3мм с зенкером Д16мм
    \item Глубина бурения не менее 20 мм. от поверхности.
    \item Глубина опускания среза шланга Д4мм для забора воды не менее 40 мм. от поверхности.
    \item Расположение точек крепления полезной нагрузки возможно на передней панели ровера в пределах от 125 мм до 235 мм от поверхности. И не более 100 мм вперед от переднего среза рамы.
    \item Располагать точки крепления полезной нагрузки можно как на самой передней панели, так и на раме ровера. Крепление к панели может осуществляться винтами М4 на М4 гайку через отверстия в панели. Крепление к раме осуществляется винтами М4х8 или М4х10 на Т-гайку М4.
    \item Чертежи для лазерной резки принимаются в формате .DXF (Autodesk Drawing eXchange Format)
    \item Чертежи для 3D-принтеров принимаются в формате в формате .STL \linebreak (“StereoLithography”)
    \item Имеет смысл взять с собой и использовать для моделирования и черчения собственные ноутбуки с предустановленными и удобными для использования командой Участников программами моделирования.
\end{enumerate}

\textbf{Программирование и алгоритмизация}

При разработки модели управления ровером необходимо учитывать что движение в зоне навигации происходит в пакетном режиме передачи команд. Под пакетным режимом подразумевается передача последовательности команд на движение в рамках одной группы. Группы команды на движение могут передаваться через консоль Ubuntu, через консоль ROS, при помощи встроенных пакетов автонавигации ROS или посредством выполнения программы на Питоне. Одна группа команд на движение считается выполненной с момента передачи ее описанными выше способами на ровер до завершения ровером движения.  Реализация той или иной формы пакетного режима выбирается командой Участников исходя из собственных предпочтений и удобства.

При программировании джойстика и размещении датчиков и камер необходимо учитывать, что в режиме телеуправления оператор ровера не видит полигон и пользуется только данными с камер, датчиков и лидара.

\subsubsection*{Состав работ}

\begin{enumerate}
    \item[3.1.] Сборка ровера
    \begin{enumerate}
        \item[3.1.1] Изготовление кабелей подключения моторов
        \item[3.1.2] Изготовление кабелей подключения лидара
        \item[3.1.3] Изготовление кабелей питания
        \item[3.1.4] Установка выключателя
        \item[3.1.5] Установка колес
    \end{enumerate}
    \item[3.2.] Проверка работоспособности ровера как ноды Robot Operating System (ROS)
    \begin{enumerate}
        \item[3.2.1] Установка связи с ровером 
        \item[3.2.2] Проверка передачи команд движения и остановки 
        \item[3.2.3] Проверка передачи видео 
    \end{enumerate}
    \item[3.3.] Настройка пульта ДУ
    \begin{enumerate}
        \item[3.3.1] Изучение документации
        \item[3.3.2] Загрузка тестовой прошивки
        \item[3.3.3] Проверка работоспособности
        \item[3.3.4] Разработка прошивки для использования в качестве управляющего элемента ROS
        \item[3.3.5] Подключение к ПК как устройство ввода для ROS
        \item[3.3.6] Проверка работы функций 
    \end{enumerate}
    \item[3.4.] Конструирование полезной нагрузки (ПН)
    \begin{enumerate}
        \item[3.4.1] Проектирование ПН позволяющей выполнить задачу бурения
        \item[3.4.2] Проектирование ПН позволяющей выполнить задачу забора образцов жидкости
        \item[3.4.3] Проектирование ПН позволяющей отделять образцы собранные в разных точках по разным контейнерам
        \begin{itemize}
            \item Изготовление чертежей деталей в соответствии с доступными технологиями изготовления
            \item Предзащита конструкции полезной нагрузки перед организаторами, для проверки ее на безопасность и реализуемость. Предзащита конструкции ПН проходит по предоставленным командой Участников эскизам и чертежам конструкции. Организаторы вправе отказать команде Участников в изготовлении деталей по чертежам или выдаче оборудования для реализации ПН, если по мнению организаторов предложенная командой Участников ПН не соответствует требованиям безопасности. 
            \item Печать деталей полезной нагрузки на 3D-принтере
            \textbf{Внимание! В течение одного дня каждая команда имеет возможность пользоваться 3D-принтером не более 2-х часов в день. Неиспользованное время в рамках одного дня на следующий день не переносится. Очередность использования командами Участников 3D-принтеров будет определена графиком.}
            \item Сдача чертежей для лазерной резки на изготовление
            \textbf{Внимание! После сдачи чертежей на изготовление, детали будут изготовлены в течение 12 часов.} Таким образом все чертежи на резку желательно сдавать до вечера второго дня (27.03.2019), чтобы у команды Участников было время на внесение изменений в проекты в случае неудачи.
        \end{itemize}
        \item[3.4.4] Сборка ПН
        \item[3.4.5] Проектирование схем подключения ПН
        \item[3.4.6] Монтаж электрической схемы подключения ПН
        \item[3.4.7] Написание ПО для оперирования ПН
    \end{enumerate}
    \item[3.5.] Разработка модели управления ровером
    \begin{enumerate}
	    \item[3.5.2] Разработка алгоритма создания и загрузки пакета команд движения ровером или настройка системы автономной навигации
	    \item[3.5.1] По желанию команды Участников разработка дополнительных алгоритмов управления ровером: предотвращения столкновения с препятствием, автономной навигации, логирования пройденного пути, автоматического возврата по своему следу и пр. 
    \end{enumerate}
\end{enumerate}