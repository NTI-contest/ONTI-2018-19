\section{Описание задачи и системы оценки}

\subsubsection*{Краткое описание задачи командного тура}

Задача командного тура финала профиля «Композитные технологии» Олимпиады Национальной технологической инициативы состоит в разработке конструкции и изготовлении сегмента крыла среднемагистрального самолёта из композиционных материалов, выполненного по кессонной схеме. За три дня участники должны разработать конструктивно-компоновочную схему своей конструкции, определиться с применяемыми материалами и изготовить изделие, полностью отвечающее требованиям технического задания.

\subsubsection*{Описание задачи}

Одной из актуальных задач отечественной промышленности на сегодняшний день является разработка среднемагистрального пассажирского самолёта, элементы крыла которого изготавливают из углепластика, композиционного материала, не имеющего аналогов по своим удельным характеристикам. Углепластик до восьми раз легче стали при аналогичной прочности, таким образом силовая структура крыла из композита значительно легче аналогичной, выполненной из металлических сплавов. Высокие удельные характеристики материала сказывается на стоимости полёта, чем легче самолёт – тем больше его грузоподъемность и меньше расход топлива. Известны зарубежные самолёты, более чем на половину выполненные из композиционных материалов – Boeing 787 Dreamliner и Airbus A350. Перспективный отечественный самолёт – МС-21, разрабатываемый НПК «Иркут» и ОКБ им. А. С. Яковлева, также будет иметь крылья, силовые элементы которых будут выполнены из углепластика.

Проектирование композитного изделия – многоступенчатый процесс, сопряженный с рядом трудностей и ограничений, вызванных нетрадиционной формой производства и общими ограничениями этих материалов. Конструктору необходимо уделять внимание всем факторам проектирования и производства, и его работа носит междисциплинарных характер.

Задачей участников станет проектирование и изготовление элемента крыла из углепластика, что погрузит их в конструкторскую и производственную деятельность современного авиационного предприятия. Участникам, как и инженерам- конструкторам, работающим с композитными изделиями, необходимо будет учесть все факторы, оказывающие влияния на конечное изделие, от конструктивно-компоновочной схемы до экономики и культуры производства.

\subsubsection*{Основное конкурсное задание}

Конструкция кессона крыла показана на рисунке 1. Кессон состроит минимум из четырёх структурных элементов – обшивки, состоящей из двух плоских листов, и лонжеронов, состоящих двух П-образных профилей. Размер обшивки – 600мм на 300мм, высота не менее 80 мм. Внутренняя структура не ограничена. Команды сами проектируют количество и расположение лонжеронов, можно ввести ребра жесткости, выполненные из трёхслойных композитных конструкций. Команды разрабатывают уникальную конструкцию, основываясь на собственных знаниях, навыках, умениях. Не существует единственно верного варианта выполнения задания.
 
\putImgWOCaption{13cm}{final/command_tour/composite/1.png}

\begin{center}
    Рис.1. Структура кессона крыла
\end{center}

На выполнения задание дается 3 дня. Метод изготовления изделия – формование с использованием гибкого пуансона (вакуумная инфузия). Форма и размеры кессона крыла должны соответствовать заданным критериям. Изделие, не отвечающее требованиям по габаритным размерам, дисквалифицируется. Кессон крыла должен соответствовать требованиям качества и выдерживать заданную нагрузку. При этом скорость выполнения изделия не контролируется. Каждая команда должна представить готовую к испытаниям, собранную деталь.

Оргкомитет финала Олимпиады предоставит технологические оснастки, средства индивидуальной защиты, предоставит доступ к одинаковым материалам и инструменты для всех команд. Перед соревнованиями участникам озвучены правила, критерии, техника безопасности и общий инструктаж.
Изделие должно быть изготовлено с максимальной экономической и технологической эффективностью.

Для оценки экономической эффективности изготовления экспертное жюри ведёт учёт расхода материалов по условным тарифам.

\subsubsection*{Система оценки}

При оценке командного этапа финала командам проставляются баллы по итогам изготовления изделия. Победителем считается команда, набравшая максимальное количество баллов. Максимально возможно набрать 100 баллов.
Результаты работы команд оцениваются по следующим критериям:
\begin{table}[H]
    \caption{Критерии оценки команд}
    \begin{center}
        \begin{tabular}{|p{0.3cm}|p{13cm}|p{1.3cm}|}
        \hline
        \multicolumn{2}{|c|}{Критерии} & Max баллы \\
        \hline
        1 & Для производства изделия команда затратила наименьшее количество средств. Баллы проставляются пропорционально, команда с самым дешевым изделием получает максимальное количество баллов, команда с самым дорогим изделием получает один балл, остальные команды получают баллы пропорционально стоимости относительно самого дешевого изделия.	& 25 \\
        \hline
        2 & Технологические ткани уложены в порядке, соответствующем технологии вакуумной инфузии. & 2 \\
        \hline
        3 & Команда дегазировала смолу, путем помещения ее в вакуумную ловушку на некоторое время. & 4\\
        \hline
        4 & Вакуумный пакет герметичен и не имеет подтеков воздуха извне. & 2 \\
        \hline
        5 & Рисунок на лицевой ткани имеет равномерную структуру. Отсутствуют посторонние нити (-1 балл за каждую), нет нарушений плетения ткани (-1 балл за каждое) и просветов (-2 балла за каждый). & 2 \\
        \hline
        6 & Лицевая поверхность изделия гладкая, ровная, стеклообразная, без микропор. & 3 \\
        \hline
        7 & На лицевой поверхности изделия отсутствуют раковины и каверны (-2 балла за каждый дефект более $1 \times 1$ мм$^2$) & 2 \\
        \hline
        8 & Отсутствуют подтёки клея, клеевой слой ровный. (-1 балл за каждый подтёк) & 3\\
        \hline
        9 & Изделие команды показало наименьший прогиб. Баллы проставляются пропорционально, команда с самым малым прогибом изделия получает максимальное количество баллов, команда с самым большим прогибом изделия получает один балл, остальные команды получают баллы пропорционально прогибу относительно самого жесткого изделия. & 25 \\
        \hline
        10 & Изделие команды получилось самым лёгким. Баллы проставляются пропорционально, команда с самым лёгким изделием получает максимальное количество баллов, команда с самым тяжелым изделием получает один балл, остальные команды получают баллы пропорционально массе относительно самого легкого изделия. & 25 \\
        \hline
        11 & Требования техники безопасности соблюдаются в полном объеме   (-1 балл за каждое нарушение) & 7 \\
        \hline
        \multicolumn{2}{|r|}{Всего:} &	100 \\
        \hline
        \end{tabular}
    \end{center}
\end{table}

\section{Критерии оценки}

\subsubsection*{Для производства изделия команда затратила наименьшее количество средств}

Максимальная оценка: 25 баллов.

Эксперты фиксируют и заносят в ведомость затраты на производство изделия. Расчет затрат производится в условных единицах.

В ведомости затрат на производство деталей учитывается стоимость конструкционных и вспомогательных материалов, т.к. общая условная стоимость материалов, переданных команде. Баллы проставляются пропорционально, команда с самым дешевым изделием получает максимальное количество баллов, команда с самым дорогим изделием получает один балл, остальные команды получают баллы пропорционально стоимости относительно самого дешевого изделия. В таблице приведен список материалов, доступный командам для закупки в течение конкурса, с указанием цены и единицы измерения.

\begin{table}[H]
    \caption{Лист материалов}
    \begin{center}
        \begin{tabular}{|p{0.5cm}|p{7cm}|c|c|c|}
            \hline
            \multicolumn{5}{|c|}{\textbf{Материалы}} \\
            \hline
            №& Материал & Цена & Количество & Ед. изм. \\
            \hline
            1 & Смола EC157 & 1410 & 1 & кг \\
            \hline
            2 &	Отвердитель для смолы HR 15 мин & 1410 & 1 & кг \\
            \hline
            3 & Отвердитель для смолы MLR 90 мин & 1410 & 1 & кг \\
            \hline
            4 & Клей-спрей & 1200 & 1	& шт. \\
            \hline
            5 & Пленка вакуумная & 170	& 1	& м.п. \\
            \hline
            6 & Вакуумная трубка 6/8 мм & 75	&1	& м.п. \\
            \hline
            7 & Трубка спиральная & 75&1	& м.п. \\
            \hline
            8 & Ткань ветошь & 35&	1	& м.п. \\
            \hline
            9 & Углеткань 200 г/м$^2$ & 1590 &1	& м.п. \\
            \hline
            10 & Стеклоткань 160 г/м$^2$ & 70	&1	& м.п. \\
            \hline
            11& Стеклоткань 360 г/м$^2$	& 75 &1	& м.п. \\
            \hline
            12 & Базальтовая ткань 270 г/м$^2$ &126	&1	& м.п. \\
            \hline
            13 & Soric 3 мм & 870 & 1 & м.п. \\
            \hline
            14 & Герможгут & 790 & 1 & рулон \\
            \hline
            15 & Жертвенная ткань & 210 & 1	& м.п. \\
            \hline
            16 & Смолопроводящая сетка & 290 & 1 & м.п. \\
            \hline
            17 & Переходники Т типа 6 мм & 59 & 1 & шт. \\
            \hline
            18 & Наполнитель уголь & 390 & 0.1 & л \\
            \hline
            19 & Разделитель & 170 & 0.05 & л \\
            \hline
            20 & Пенопласт 20 мм & 70 & 1 & шт. \\
            \hline
            21 & Лента малярная & 103 & 1 & шт. \\
            \hline
            22 & Кисть & 35 & 1	& шт. \\
            \hline
            23 & Ацетон & 59 & 0.5 & л \\
            \hline
        \end{tabular}
    \end{center}
\end{table}
\subsubsection*{Технологические ткани уложены в порядке, соответствующем технологии вакуумной инфузии}

Максимальная оценка: 2 балла.

Оценивается порядок укладки основных и вспомогательных материалов при изготовлении изделий методом вакуумной инфузии.

Верный порядок следующий (от технологической оснастки):
\begin{enumerate}
    \item Герметизирующий жгут;
    \item Армирующий наполнитель (стеклянная, базальтовая или углеродная ткань);
    \item Жертвенный слой (ткань Peel-ply);
    \item Смолопроводящая сетка;
    \item Трубка подачи свящующего (трубка спиральная);
    \item Трубка откачки воздуха (трубка спиральная);
    \item Плёнка вакуумная.
\end{enumerate}

При соблюдении верного порядка сборки вакуумного пакета команда получает 2 балла. При несоблюдении 0 баллов.

\subsubsection*{Команда дегазировала смолу, путем помещения ее в вакуумную ловушку на некоторое время}

Максимальная оценка: 4 балла.

Оценивается работа с эпоксидным связующем. При верном приготовлении связующего (замешивание смолы и отвердителя в верных пропорциях) и дегазации связующего команда получает 4 балла. При неверном приготовлении связующего или отсутствии дегазации команда получает 0 баллов.

\subsubsection*{Вакуумный пакет герметичен и не имеет подтеков воздуха извне}

Максимальная оценка: 2 балла.

Оценивается герметичность вакуумного пакета. При его герметичности команда получает 2 балла. При негерметичности команда получает 0 баллов.

\subsubsection*{Рисунок на лицевой ткани имеет равномерную структуру}

Максимальная оценка: 2 балла.

Оценивается качество драпировки армирующего наполнителя (стеклянной, базальтовой, углеродной ткани). На лицевой поверхности не должно быть посторонних нитей (минус 1 балл за каждую), плетение ткани не должно быть нарушено (минус 1 балл за каждое нарушение), не должно быть просветов ткани (минус 2 балла за каждый). На рисунке 2 представлено изделие, удовлетворяющее этому критерию.

\subsubsection*{Лицевая поверхность изделия гладкая, ровная, стеклообразная, без микропор}

Максимальная оценка: 3 балла.

Оценивается качество пропитки преформы (сухой заготовки из армирующей ткани) связующим. В случае, когда лицевая поверхность гладкая, ровная, стеклообразная, без микропор, команда получает 3 балла. При наличии микропор, шероховатости, искривлений команда получает 0 баллов. На рисунке 2 представлено изделие, удовлетворяющее этому критерию.

\subsubsection*{На лицевой поверхности изделия отсутствуют раковины и каверны}

Максимальная оценка: 2 балла.

Оценивается качество пропитки преформы (сухой заготовки из армирующей ткани) связующим. В случае, когда отсутствуют раковины и каверны команда получает 3 балла. При наличии раковин и каверн команда получает 0 баллов. На рисунке 2 представлено изделие, удовлетворяющее этому критерию.

\putImgWOCaption{11cm}{final/command_tour/composite/2.jpg}

\begin{center}
    Рис. 2. Изделие, отвечающее требованиям критерием 5, 6, 7. Равномерный рисунок армирующей ткани, гладкая, стеклообразная поверхность, отсутствие пор. Края не обработаны
\end{center}

\subsubsection*{Отсутствуют подтёки клея, клеевой слой ровный}

Максимальная оценка: 3 балла.

Оценивается качество склеивания составных элементов кессона. Если команда выполнила ровный клеевой слой без подтеков – команда получает 3 балла. За каждый подтек клея команда получает на балл меньше, вплоть до 0 баллов.

\subsubsection*{Изделие команды показало наименьший прогиб}

Максимальная оценка: 25 баллов.

Оценивается жесткость конструкций, полученных командами. Оценка проводится по методике трёхточечного изгиба. Минимальная нагрузка, которую должно выдержать изделие – 100 кг. Максимальная прикладываемая нагрузка – 200 кг. Баллы проставляются пропорционально, команда с самым малым прогибом изделия получает максимальное количество баллов, команда с самым большим прогибом изделия получает один балл, остальные команды получают баллы пропорционально прогибу относительно самого жесткого изделия.

\subsubsection*{Изделие команды получилось самым лёгким}

Максимальная оценка: 25 баллов.

Оценивается масса конструкций, полученных командами. Баллы проставляются пропорционально, команда с самым лёгким изделием получает максимальное количество баллов, команда с самым тяжелым изделием получает один балл, остальные команды получают баллы пропорционально массе относительно самого легкого изделия.

\subsubsection*{Требования техники безопасности соблюдаются в полном объеме}

Максимальная оценка: 7 баллов.

Команды должны поддерживать чистоту рабочих мест и инструмента, иметь опрятный вид и соблюдать требования техники безопасности и служебного этике во время соревнований. Команды должны строго соблюдать требования техники безопасности, с которыми участники знакомятся перед началом соревнований. За каждое нарушение команды получают на один балл меньше по данному критерию.

\section{Конкурсные требования}

Каждая команда представляет на конкурсе:
\begin{itemize}
    \item Готовое изделие – кессон крыла в сборе.
\end{itemize} 

\subsection*{Мероприятия, проводимые до начала конкурса}
\begin{itemize}
    \item Командам озвучиваются требования по технике безопасности и охраны труда. Каждый участник финала подписью заверяет то, что он ознакомлен с правилами.
    \item Командам предоставляются материалы в том виде, в каком они поставляются заводом-изготовителем с указанием их характеристик. Участники на основании ознакомления могут предварительно подобрать технологию изготовления и исходные компоненты. Материалы, которые были взяты на пробу, использовать во время соревнования запрещено.
    \item Команды могут ознакомиться с оборудованием, предоставленным организаторами.
\end{itemize}

\subsubsection*{Мероприятия, проводимые во время конкурса}
\begin{itemize}
    \item Изготовление всех компонентов изделия, кроме предоставленных организаторами.
    \item Сборка изделия.
\end{itemize}

\section{Оборудование, предоставляемое организаторами}
\begin{itemize}
    \item Оборудование и материалы: установка для вакуумной инфузии, формообразующие оснастки, струбцины, средства измерения (линейки, измерительные рулетки, пирометры), оборудование для механической обработки, материалы согласно инфраструктурному листу.
    \item Испытательное оборудование.
\end{itemize}

\section{Одежда и средства защиты}

Каждой команде предоставляется комплект средств индивидуальной защиты: индивидуальные комбинезоны, очки, перчатки, средства защиты органов дыхания, бахилы), а также одежда с символикой Олимпиады НТИ.

Одежда и средства индивидуальной защиты должны строго соответствовать требованиям безопасности для соответствующих технологий и производственных операций. Не допускается нарушение техники безопасности, выполнение технологических и производственных операций без соответствующих средств индивидуальной защиты.

На команду, участники которой не используют средства индивидуальной защиты (предусмотренные инструкцией по безопасности), требуемые при данном виде работ, налагается штраф в соответствии с критерием, изложенным в п. 2.11 «Требования техники безопасности соблюдаются в полном объеме».

\section{Содержимое инструментальных наборов}

Команды несут ответственность за собственное обеспечение всеми инструментами для изготовления изделия, не перечисленными в инфраструктурном листе. В набор инструментов команд могут входить следующие компоненты (список неокончательный):
\begin{itemize}
    \item Зажимы, крепления, захваты;
    \item Расходный инструмент, необходимый для изготовления компонентов изделия;
    \item Ручной инструмент для подготовки исходных компонентов для изготовления изделия;
    \item Ручной, режущий и измерительный инструмент;
    \item Ручной инструмент для сборки.
\end{itemize}

\section{Недопустимые материалы и оборудование}

Недопустимо использование участниками следующих материалов и оборудования во время проведения финала Олимпиады:
\begin{itemize}
    \item Ноутбуки;
    \item Планшеты, мобильные телефоны, коммуникаторы и т.п.;
    \item Устройства хранения цифровой информации (флеш-накопители, жесткие диски, компакт-диски и т.д.);
    \item Электронные средства связи и устройства с памятью;
    \item Устройства беспроводной передачи данных;
    \item Любое дополнительное программное обеспечение, за исключением предоставленного организаторами, если иное не разрешено экспертным жюри;
    \item Оборудование аналогичное, или выполняющие аналогичные функции, что и поставленное организаторами.
\end{itemize}

В процессе конкурса не допускается вынос из рабочей зоны команд инструмента, оборудования, компонентов, руководств, чертежей или устройств хранения данных.
