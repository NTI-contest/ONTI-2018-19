\assignementTitle{}{10}{}

Для правильного проектирования, балансирования и работы с аквапонными установками вам необходимо учесть все организмы, которые в ней присутствуют и понять, какие процессы происходят в каждом из модулей. 
 
\begin{enumerate}
    \item Внимательно изучите предложенные вам аквапонные установки и определите все биокомпоненты, задействованные в ней.
    \item Опишите все процессы, которые могут влиять на стабильность аквапонной установки, происходящие в ее модулях (биофильтрационный, аквакультурный и гидропонный).
\end{enumerate}
  
Результат приведите в письменном виде. Листок с ответом подпишите (название команды) и проставьте дату.

\textbf{Задание выполняется “offline”}

\markSection

Определены (поименованы):

\begin{enumerate}
    \item Салат (салат латук);
    \item Карп (кои и/или серебристый – равнозначно для д. сл.), стерлядь;
    \item Перловица и шаровка (до рода);
    \item Баксообщество (равнозначно бактерии).
    \item Сделано предположение о наличии протистов в системе.
\end{enumerate}

За каждый правильный ответ по 1 баллу.  – 5 баллов

Определены высшие растения, НЕ подключённого растительного фильтра (кабомба и элодея) – 1 балл, т.к. данный блок НЕ входит в систему (в задании сказано про аквапонные системы, а не все биокомпоненты, которые есть в комнате).

Выявлены процессы:

\begin{enumerate}
    \item Газообмена кислород-углекислый газ
    \item Нитрификации (аммоний-нитрит-нитрат)
    \item Выделение аммония рыбой
    \item Выделение рыбой не р-римой органики (слизь, кал)
    \item Поглощение нитратов, нитритов, аммония
\end{enumerate}

Плюс/либо замена на один параметров: фотосинтез растений, выделение корневой системой салата, скорость циркуляции воды в системе

Макс. значение баллов – 5 баллов