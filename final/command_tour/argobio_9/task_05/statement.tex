\assignementTitle{}{5}{}

Для уравновешивания параметров системы вам предлагается ввести в аквапонную установку дополнительный модуль с растениями. 

Для этого вам необходимо продумать следующие вопросы:
\begin{enumerate}
    \item Определите очередность подключения блока к уже существующим модулям.
    \item Определите способ создания единого потока воды в расширенной системе. (при помощи чего вы будете его создавать?)
    \item Оцените риски подключения дополнительного модуля. Сколько может вытечь воды, если отсоединится шланг / выключится электричество?
\end{enumerate}
  
Результат приведите в письменном виде. Листок с ответом подпишите (название команды) и проставьте дату.

\textiti{Реализуйте идею и результат приведите в письменном виде в виде отчета. В отчете приведите расчеты, описание и схемы (если это необходимо). Листок с ответом подпишите (название команды) и проставьте дату.}

Максимальный балл - 5 баллов

\markSection

\begin{enumerate}
    \item 	Очередность подключения блока вне зависимости от места подключения аргументировано – 1 балл

    Похожее задание было разобрано во втором туре Олимпиады. Правильным расположением может быть один из следующих вариантов: 
        
        \begin{itemize}
            \item Гидропонный > Растительноводный > Аквакультурный  > Бактериальный фильтрационный
            \item Гидропонный >  Аквакультурный  > Бактериальный фильтрационный > Растительноводный
        \end{itemize}    
    Таким образом, модуль должен быть подключен или до, или после гидропонного блока. 
    
    \item  	Выполнена схема подключения растительного фильтра и общая схема аквапонной установки. Указаны уровни воды, переливы, положение заборных патрубков – 1 балл

    \item  	Проведено расчётное определение скоростей работы водных помп в системе – 1 балл
            \begin{enumerate}
                \item Скорость работы зависит от типа и регулировки насоса. Участники должны измерить ее при помощи секундомера и емкости известного объема
            \end{enumerate}

    \item 	Скорости потоков воды уравновешены любым способом, рассчитана ошибка определения скорости потоков – 1 балл (без расчёта ошибки – 0.5 баллов)
    
    \item   Определён объём воды, способный вылиться из системы в случае откл.электричества или превышении скоростей заполнения с блоков системы – 1 балл 
            \begin{itemize}
                
                \item Объем воды, которая может вылиться из системы в случае поломки определяется для каждой конкретной установки (схемы подключения, модификаций, объемов воды..) отдельно.
            
                \item  При расчете должна быть нарисована схема системы с точным размещением по высоте приливных и сливных отверстий в разноуровневых ёмкостях
            
                \item должны быть определены максимальные объёмы каждой из отдельных ёмкостей в случае остановки циркуляции и самотечного слива. При этом важно учитывать, как будет происходить осушение: управляемо с использованием ограничителей минимального уровня воды, или по закону “сообщающихся сосудов”. 
                
                \item Построена модель водообмена для каждого критического случая (отсоединился шланг/выключилось электричесвто);
                
                \item Произведён рассчёт первого и второго варианта по потере воды системой.
                
            
            \end{itemize}    

\end{enumerate}

В начальных аквапнных установках (без модификаций, которые могли ввести участники) неправильное подключение дополнительного модуля, при котором скорости насосов (на вход и выход) не были сбалансированы по расчетам мог произойти  перелив воды по уровню заборного фильтра в биофильтрационном модуле или модуле с высшими водными растениями (в зависимости от схемы подключения). 