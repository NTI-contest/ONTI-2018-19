\assignementTitle{Создание биофильтра}{25}{1}

Приставка «био» всегда обозначает, что в процессе участвуют живые микроорганизмы - бактерии, поглощающие аммиак, от которого страдают обитатели аквариума, превращая его в нитриты и затем в нитраты.
Это жизненно необходимая составляющая здорового аквариума, поскольку буквально все органические соединения разлагаются, образуя вредоносный аммиак. Достаточное количество полезных бактерий контролирует содержание аммиака в воде. Дело остается за малым, создать для бактерий место обитания и комфортную среду. 
**Важно помнить, что «работать» должны бактерии, осевшие на фильтре, а не свободно плавающие!!

\textbf{Основные компоненты} (их можно конкретизировать для вашего фильтра): ёмкости, наполнители, источники микроорганизмов.

\textbf{Ожидаемый результат:} собранный биофильтр работает эффективно (определите показатели назначения – эффективность работы, проведите их мониторинг – в динамике, с шагом около 12/24 часа).

\textbf{Требования к результату:} результат должен быть оформлен в таблице, данные лучше визуализировать в виде графика. Дать критический анализ работы созданного биофильтра (почему выбраны те или иные компоненты, в чём плюсы/ минусы, идеи для улучшения конструкции/ условий).
Для этой задачи можно использовать оборудование: датчики рН, температуры, электропроводности.

\textbf{Расходные материалы:} тест-системы для определения аммония, нитрата, рН, наполнители (биокерамика, песок, гравий, поролон,….), соли аммония,..

\textbf{Ограничение:} объём фильтра ограничен ёмкостью 1 литр

\begin{center}
    \textbf{Лабораторный журнал по задаче:}
\end{center}

\begin{table}[H]
    \begin{center}
        \begin{tabular}{|p{3cm}|c|c|c|c|c|c|}
        \hline
        & \multicolumn{6}{c|}{Результаты измерений} \\
        \cline{2-7}
        \raisebox{1.5ex}[0cm][0cm]{Дата, время}
        & $NH_4^+$, \dots & $NO_3^-$, \dots & $NO_3^-$, \dots & $O_2$, \dots & $T$, \dots & pH\\
        \hline
         &  &  & & & & \\
        \hline
        &  &  & & & & \\
        \hline
        &  &  & & & & \\
        \hline
        \end{tabular}
    \end{center}
\end{table}

\markSection 

Правильное и своевременное измерение показателей в течение всего финального тура олимпиады. Максимум - 15 баллов.

Построение графиков - максимум 3 балла.

Эффективность работы биофильтра - 7 баллов.

Эффективность работы биофильтра проверяется по следующим параметрам: происходит снижение концентрации аммиака и ионов аммония, происходит увеличение содержания нитрат-ионов, концентрация кислорода в системе не уменьшается.