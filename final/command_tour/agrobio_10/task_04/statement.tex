\assignementTitle{}{10}{}

\begin{enumerate}
    \item Ознакомьтесь с материалами - Из источника \url{https://www.ncbi.nlm.nih.gov/pmc/articles/PMC5946678/} 
 
    \putImgWOCaption{5.5cm}{1}
    
    «Gut microbial communities of fish in other trophic levels have less characteristic dominance when compared to herbivores. However, one study comparing \linebreak phylogenetically similar benthivore and planktivore freshwater species showed they contained different unique intestinal bacterial communities (Uchii et al., 2006). In general, within the marine environment, Proteobacteria, rather than Firmicutes, is often the dominant phylum at the non-herbivorous trophic levels (Miyake et al., 2015). Vibrionaceae, Aeromonas and Pseudomonas are all frequently reported in carnivores, omnivores and (zoo-) planktivores.»

    \underline{Какие роды бактерий являются наиболее традиционными обитателями} 

    \underline{кишечника рыб?}

    \answerMath{Pseudomonas spp., Vibrio spp, относящиеся к типу протеобактерии (Proteobacteria). (участники могли выяснить это из статьи, на которую дана ссылка)}

    \underline{Какую бактерию вам предложили для культивирования?}-\textbf(2 балла)

    \answerMath{Pseudomonas fluorescens}

    \item Укажите, в каких пропорциях и какие препараты вы интродуцировали в своих фильтры? \textbf{(2 балла)}
    
    \explanationSection

    Участники должны сами определиться с пропорциями, которые приведены ниже, и которые выдавались через установленное время . 

    Внизу пояснения – в пробирку №1 добавили \dots, №2 \dots, №3 \dots

    \item Возьмите 3 центрифужных пробирки объёмом 15 мл, подпишите их (№1, 2, 3) 
    
    В пробирки добавьте:
    
    В №1 -  10 мл фосфатного буфера (рН 7.5). Внесите туда бактерию Pseudomonas fluorescens Встряхните.
    
    В №2 – то, что вносили в фильтры (объёмом 10 мл, сохраняя ваши пропорции)
    
    В №3 – 10 мл пробы из аквапонной установки

    \item Используйте тест-систему Biolog-Eco для анализа ваших проб.
    
    Ознакомиться с системой можно тут: \url{https://openwetware.org/wiki/M465:Biolog_Ecoplates}

    В каждую лунку вносим 0,2 мл тестируемого образца.

    Занесите все действия в лабораторный журнал.

    \item Подпишите (сбоку аккуратно) вашу планшетуку. Поместите в термостат $Т=27^\circ C$
    
    Наблюдайте изменения окраски лунок на первые сутки, на вторые \textbf{(2 балла)}

    \putImgWOCaption{7cm}{2}

    \item Интерпретируйте полученные результаты и занесите в журнал \textbf{(4 балла)}
    
    \answerMath

    В качестве результата можно ожидать классификацию трёх объектов на основе разного спектра и активности потребления субстратов. Участники могли выдвинуть гипотезу, что коммерческий препарат, если представляет собой только нитрификаторов, то штаммы, входящие в состав препарата, не используют органические субстраты. Участники могли предложить разного рода визуализации полученных результатов, вплоть до кластерного анализа.
\end{enumerate}