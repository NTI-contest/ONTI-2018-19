\assignementTitle{Как антибиотики воздействуют на полезную микрофлору?}{20}{}

«Антибиотики присутствуют в большом количестве в продукции аквакультуры, что становится проблемой, к такому выводу пришли ученые из университета Аризона в США. Анализ проб из различных видов рыбы и морепродуктов, выращенных на фермах в США, продемонстрировал наличие пяти различных видов антибиотиков, которые достаточно часто встречались в чрезмерных концентрациях. Об этом сообщает foodcontrol.ru.

При этом анализ импорта продемонстрировал еще более угрожающие данные. Исследователи обнаружили следы 47 различных видов антибиотиков в креветках, лососе, соме, форели и тилапии, ввезенных из 11 стран. И опять же достаточно часто антибиотики встречались в больших концентрациях. Результаты работы были опубликованы в Journal of Hazardous Materials.

Как сообщает kombi-korma.ru, агентства по контролю за оборотом пищевой продукции в США практически не обращают внимание на эту проблему, точно также, как и общественность, которая традиционно беспокоится о наличии антибиотиков в мясе сельскохозяйственных животных. В то же время об аналогичной проблеме в продукции аквакультуры вообще мало кто говорит, отмечают специалисты».

\begin{enumerate}
    \item Обсудите в команде, какие эффекты может иметь использование антибиотиков/ пестицидов в агропроизводстве. Укажите все заслуживающие внимание негативные последствия. (4 балла)
    
    \answerMath{
        \begin{itemize}
            \item снижение биоразнообразия микробных сообществ почв – нарушение естественного биоконтроля патогенов (супрессивности), присущего природным ландшафтам
            \item дисбактериоз - в животноводстве, снижение иммунитета, делающее объекты животноводства более уязвимыми инфекциям
            \item аккумуляция в цепях питания – потребление с/х продукции, содержащей антибиотические вещества – дисбактериозы у потребителей
            \item резистентность патогенных микроорганизмов (в результате горизонтального переноса генов в природной среде)
            
            ( – балл за указание одного корректного эффекта, максимум 4)
        \end{itemize}
    }

    \item В каких случаях антибиотические вещества могут попадать в водоёмы, которые являются базой для аквакультуры? \textbf{(2 балла)}
    
    \answerMath{
        \begin{itemize}
            \item с кормами (производитель стремится увеличить сроки хранения кормов)
            \item стоки с животноводческих ферм (экскременты содержат антибиотики в случае их использования даже при профилактике заболеваний молодняка)
            \item стоки с полей, обработанных пестицидами  
            
            (балл за указание одного корректного примера, максимум 2)
        \end{itemize}
    }

    \item Какие способы снижения объёмов использования таких веществ можно применять без ущерба для здоровья среды? Предложите способы, покажите, в чём их положительные стороны. \textbf{(4 балла)}
    
    \answerMath{
        \begin{itemize} 
            \item севооборот в растениеводстве – позволяет существенно снизить численность фитопатогенов (как бактерий, так и грибов – микромицетов), увеличивает разнообразие микроорганизмов (ризосфера разных культур отличается по составу), восстановить баланс биогенных элементов 
            \item комплексные посадки, включающие растения, которые являются источниками репеллентов нежелательных популяций
            \item биоконтроль патогенов, в тч использование суперпаразитов
            
            (за каждый корректный способ – 1 балл, за его аргументацию +1 балл, максимум – 4 балла)
        \end{itemize}
    }

    \item Возьмите предложенные штаммы микроорганизмов (полезные!). Предложите дизайн эксперимента для определения минимальной действующей концентрации антибиотика/ пестицида.
    
    Для определения активности штамма используйте методику гидролиза диацетата флуоресцеина. \textbf{(6 баллов)}

    \answerMath{
        Для работы использовались пекарские дрожжи - Saccharomyces cerevisiae

        В качестве антибиотика был взят нистатин. За обоснованный дизайн эксперимента (схема, логические пояснения) – 4 балла, за аккуратное исполнение методики, работу в лаборатории - + 2 балла
        }

    \item Проанализируйте полученные данные. Сделайте вывод. \textbf{(4 балла)}
    
    \answerMath{
        \begin{itemize}
            \item в случае активности дрожжей – в среду высвобождался флуоресцеин, относительное содержание которого регистрировали на спектрофотометре. 
            \item за аккуратность работы на спектрофотометре – 2 балла, корректную интерпретацию (супернатант зелёный – нет ингибирования, нет зелёной окраски – есть ингибирование) +2 балла
        \end{itemize}
    }
\end{enumerate}