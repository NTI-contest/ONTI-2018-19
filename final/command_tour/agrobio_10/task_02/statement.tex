\assignementTitle{Культивирование бактерий}{30}{2}

\begin{enumerate}
    \item Перечислите необходимые компоненты для роста бактерий. 
    (4 балла)
    \answerMath{Основными биогеннными элементами являются: С, H, N, O, P, S, К; микроэлементы (многие ионы металлов, являясь составной частью активного центра ферментов или участвуя в поддержании их пространственной структуры, обеспечивают обмен веществ микроорганизмов; наибольшее влияние на рост и развитие микроорганизмов оказывают ионы железа, меди, марганца, цинка, бора, молибдена, кобальта и ряда других металлов).
    
    Таким образом, для культивирования и роста бактерий необходимы следующие компоненты:
    \begin{itemize}
        \item Источники углерода – органические/ неорганические ($СО_2$, $СO$)
        \item Источники азота – аммонийные, нитратные формы, аминокислоты, белки
        \item Минеральная основа – соли, традиционно присутствующие в природной воде/ почвенном растворе
        \item Микроэлементы 
        \item Факторы роста - витамины, пурины, пиримидины и аминокислоты, дрожжевой экстракт, дрожжевой автолизат (комплексный источник витаминов), возможны растительные экстракты 
    \end{itemize}
    }
    \item Из предложенного списка веществ выберите те, которые могут быть источником питания. Определите их в соответствующую колонку. \textbf{(6 баллов)} 
    
    \textit{Количество строк в таблице приведено примерное. Можете оставлять какие-то пустыми, или добавить необходимые.}
    
    $NаНСO_3$, $СаСO_3$, $NH_4Cl$, лизин, $(NH_4)
    _2SO_4$, $КNО_3$, Тиомолибдат аммония, $NaNО_3$, гистидин, гидролизат белка, пептон, $N_2$, $Na_2S$, треонин, $HCN$, $N_2O$, $C_2H_2$ , лизин, сукцинат, ацетат, КOH, декстран, полиэтиленгликоль, $С_2H_5OH$, аргинин, $NaH_2PO_4$, холекальциферол, триптофан, $СаСО_3$, сахароза, $Na_2HPO_4$ , $КH_2PO_4$ , $NaСl$, $FeSO4$, тиамин,$CuSO_4$, никотиновая кислота, пантотеновая кислота, рибофлавин, $MoS_2$ пиридоксин, биотин, дианкобаламин, $Fe(MoO_4)_3 \cdot nH_2O$  параамино-бензойная кислота, $H_4Мo_8O_{26}$ фолиевая кислота, холин;  $CoSe_2$ аденин, $SiO_2$, гуанин, цитозин, $SeSO_3$ тимин, урацил; триптофан, холин, глутаминовая кислота , метионин, валин, лейцин, фенилаланин, $NH_2CONH_2$ , гистидин.

    \begin{table}[H]
        \begin{center}
            \begin{tabular}{|p{2.3cm}|p{2.3cm}|p{2.5cm}|p{3cm}|p{2cm}|}
                \hline
                Источники С	&Источники N	&Минеральные компоненты	&Микроэлементы	&Факторы роста\\
                \hline
                & & & & \\
                \hline
                & & & & \\
                \hline
                & & & & \\
                \hline
                & & & & \\
                \hline
            \end{tabular}
        \end{center}
    \end{table}

    \textit{Верная таблица с ответами:}

    \begin{longtable}{|p{2.3cm}|p{2.3cm}|p{2.5cm}|p{3cm}|p{4cm}|}
        \hline
        Источники С	&Источники N	&Минеральные компоненты	&Микроэлементы	&Факторы роста\\
        \hline
        гидролизат белка, $C_2H_2$, сукцинат, ацетат, декстран, полиэтиленгликоль, $С_2H_5OH$, холекальциферол, сахароза, аминокислоты & $NH_4Cl$, $(NH_4)_2SO_4$, $КNО_3$, Тиомолибдат аммония, $NaNО_3$, гидролизат белка, пептон, $N_2$,  аминокислоты, $NH_2CONH_2$, $N_2O$& $NаНСO_3$, $СаСO_3$, $NaH_2PO_4$, $Na_2HPO_4$, $КH_2PO_4$, $NaСl$, $Na_2S$, $NaNО_3$ $FeSO_4$, $СаСО_3$& $Fe(MoO_4)_3 \cdot nH_2O$, $H_4Мo_8O_{26}$, $CoSe_2$, $SiO_2$, $SeSO_3$, $CuSO_4$, $MoS_2$, Тиомолибдат аммония& треонин, тимин, урацил; триптофан, холин, глутаминовая кислота, метионин, валин, лизин, гистидин, лейцин, фенилаланин, гистидин, аденин, гуанин, цитозин, никотиновая кислота, пантотеновая кислота, рибофлавин, биотин, дианкобаламин, параамино-бензойная кислота, фолиевая кислота, тиамин, пиридоксин, лизин, аргинин, триптофан\\
        \hline
    \end{longtable}

	\item Перед вами стоит задача культивировать азотфиксаторов, какую среду будете использовать? (предложите состав и укажите, для чего используется каждый компонент). \textbf{(6 баллов)}
    
    \answerMath{Необходимы источники углерода, в качестве источника азота – атмосферный $N_2$ (т.к. в случае использования нитратов/ аммония или органического источника – азотфиксаторы предпочтут их вместо того, чтобы фиксировать молекулярный азот. При этом селекции азотфиксаторов не произойдёт). Необходимы микроэлементы (молибден, железо, ванадий), которые могут выступать в качестве активного центра нитрогеназы (фермента, ответственного за восстановление $N_2$). Важно поддержание нейтральной по кислотности среды, тк при низких значения рН ингибируется нитрогеназа.}

    \item Вам предлагают культивировать следующию бактерию. Предложите оптимальный состав питательной среды для бактерии. Какие дополнительные условия необходимо выполнить для успешного роста (рН, ОВП – окислительно-восстановительный потенциал, температура, \dots)? \textbf{(14 баллов)}
    
    \textit{Бактерия Х.}

    \textit{Облигатный аэроб, температурный оптимум роста 28-30; растет в пределах рН среды от 5.2 до 8.0. В дополнительных факторах роста не нуждается (прототроф). Обладает лецитиназной активностью, желатину разжижает. Крахмал не гидролизует. В качестве источника углерода использует глюкозу, триптофан, глицерин, этиловый спирт, слабо усваивает мальтозу. Не усваивает лактозу, дульцит, рамнозу, сорбит. Использует аммиачные и нитратные соли азота в качестве единственного источника азота.}

    \answerMath{В качестве примера бактерии может быть приведена Pseudomonas spp 

    Для контроля организаторы подготовили минимальную синтетическую среду следующего состава: $К_2НРO_4$ - 7 г/л, $KH_4PO_4$ - 3 г/л, $NH_4Cl$ - 1 г/л, глюкоза - 2 г/л. Задача участников состояла подобрать на своё усмотрение минимальный достаточный состав, на котором штамм Псевдомонады мог образовать биомассу. Также, необходимо было указать физико-химические условия культивирования - температуру, кислотность, аэрация (это всё указано в описании бактерии). 

    Лучшим результатом становился тот, где бактерия показывала максимальный рост. В обеих командах рост был ниже контроля, и были отличия в количестве биомассы между командами. Причём отличия были зафиксированы в 3-й повторности.}
\end{enumerate}