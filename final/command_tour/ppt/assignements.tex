Участникам нужно разработать и построить модель автономного мостового крана для сортировочной станции железной дороги. Кран должен двигаться по крайним путям сортировочной станции (на полигоне - по специальным направляющим) и перемещать маркированные контейнеры заданным образом между платформами.  Для определения типов контейнеров и позиций их перемещения должно использоваться "машинное зрение", с камерой, размещенной на каретке мостового крана.
Участникам предоставляется полигон с направляющими, макетами железнодорожных платформ и грузовых контейнеров. На решение задачи выделено 3 дня (всего 24 астрономических часа), которые включают в себя:
\begin{itemize}
    \item Создание 3D-модели прототипа в САПР, 
    \item Изготовление деталей изделия на лазерном станке или 3D-принтере, 
    \item Сборку конструкции и ее оснащение электроникой, 
    \item Программирование и отладку,
    \item Прохождение тестовых испытаний
\end{itemize}

\section{Описание полигона}

Испытательный стенд (Полигон) представляет собой прямоугольный стол размером $1200 \times 600$~мм. Вдоль длинных сторон стола закреплены 4 ряда "рельсовых путей", на равных расстояниях друг от друга.  На рельсовые пути устанавливаются "железнодорожные платформы", а на них - "контейнеры".  Рельсы не обеспечивают легкое перемещение платформ, а только их точное позиционирование по координате $Y$ (в дальнейшем, координатой обозначаем X положение вдоль рельсов, координатой $Y$ - положение поперек рельсов. Благодаря рельсам, координата $Y$ может принимать только несколько заранее известных, дискретных значений).  Платформы могут быть сцеплены друг с другом, образуя "железнодорожный состав", с фиксированным шагом между платформами.  

Размер контейнера (Д $\times$ Ш $\times$ В) составляет $105 \times 50 \times 40$~мм. По бокам крышки контейнера имеются выступы шириной 1.5-2 мм для более удобного захвата манипулятором. 

Также вдоль длинных сторон стола закреплены линейные направляющие длиной не менее 1000 мм, на расстоянии 480-580 мм друг от друга.  Конструкция направляющих различается на "ведущей" и на "ведомой" стороне полигона. 

На ведущей стороне установлена заранее изготовленная каретка (платформа) с шаговым двигателем и концевым выключателем и натянут зубчатый ремень. Мостовой кран, разрабатываемый командой, должен опираться на эту платформу и крепиться к ней в краевые пазы, быстросъемными винтами. Чертеж каретки, с габаритными и посадочными размерами, приведен в приложениях.

На ведомой стороне, к столу прикреплен алюминиевый профиль $20 \times 40$~мм. Команда должна самостоятельно сконструировать передвижную опору моста с этой сторону, используя в качестве колес, например, шарикоподшипники.

Участникам предоставляется один полигон на несколько команд, поэтому испытания на полигоне производятся поочередно, а конструкция мостового крана должна обеспечивать быструю установку и снятие с полигона.

\textbf{Размещение управляющей электроники и ПО.} Контроллер и все сопутствующие модули должны быть размещены на мостовом кране. От готовой каретки, к контроллеру подключается 4-контактный кабель шагового двигателя и 2-контактный кабель концевого выключателя.  Питание на мостовой кран подается по проводу от сетевого блока питания (12V 2A).  Обработка видеопотока и управление высокого уровня выполняется на ноутбуке.  От мостового крана может выходить не более 3 кабелей:  (1) USB-кабель от камеры к ноутбуку, (2) USB-кабель от ноутбука к контроллеру и (3) кабель питания 12V от сетевого блока питания, с максимальным током до 2A.  

\textbf{Маркировка.} На верхние грани каждой платформы и контейнера наклеена пленка белого цвета, с визуальными маркерами.  Цвет остальных поверхностей не регламентируется.  Маркеры:

\begin{center}
    \begin{tabular}{|p{7cm}|p{7cm}|}
        \hline
        \textbf{На платформе}&	\textbf{На контейнере}\\
        \hline
        прямоугольная рамка \textbf{черного} цвета, по размеру контейнера, толщина линии 5 мм & прямоугольная рамка \textbf{зеленого} цвета, по размеру контейнера, толщина линии 5 мм \\
        \hline
        QR-код, \textbf{черного} цвета, со стороной квадрата не менее 20 мм, содержащий название \textbf{места назначения} & QR-код \textbf{зеленого} цвета, содержащий \textbf{название груза} \\
        \hline
    \end{tabular}
\end{center}

QR-коды на платформе и на контейнере расположены полностью внутри рамки, но их смещение и ориентация не регламентируется (не обязательно по центру).

\section{Основная задача}

На 2-х путях сортировочной станции стоят составы с контейнерами.  На двух других путях стоят пустые составы. Командам выдается "сортировочный лист" - файл данных в формате CSV, определяющий, контейнеры с какими грузами должны отправиться в какие города (поля "код груза", "код города", "кол-во").   Портальный кран должен, в автономном режиме, откалиброваться, отсканировать полигон, распознавая и запоминая положения позиционных меток и содержание QR-кодов, а затем переставить контейнеры в соответствии с "сортировочным листом".  

Программа, анализирующая изображения и управляющая логикой работы устройства, находится в ноутбуке,  для низкоуровневого управления устройством используется "бутерброд" на базе  Arduino UNO, CNC-shield и драйверов шаговых двигателей.  Обмен происходит по последовательному порту, за счет GCODE-подобных команд.

\section{Промежуточные задачи} 


В течение 4-х дней соревнования участники имеют возможность заработать дополнительные баллы, до указанных дедлайнов демонстрируя работоспособность отдельных узлов устройства. Список промежуточных задач показан ниже.  В последний день промежуточные тесты (помеченные звездочкой*) принимаются только в том случае, если команда не готова сдавать ни один из интегральных тестов.

\begin{center}
    \small
    \begin{longtable}{|p{2.8cm}|p{6.2cm}|p{1.1cm}|p{1.1cm}|p{1.1cm}|p{1.1cm}|}
        \hline
        \multirow{2}{*}{Название теста} & \multirow{2}{*}{Методика проверки} & \multicolumn{4}{l|}{Начисляемые баллы (по дням)} \\ 
        \cline{3-6} 
        &  & Д1 & Д2 & Д3 & Д4 \\ 
        \hline
        \multicolumn{6}{|c|}{Конструирование и сборка} \\
        \hline
        Схват работает &	Узел схвата собран, подключен к контроллеру, любым образом (например, с кнопки) переключается состояние схватить/отпустить, показан захват и удержание макета контейнера & 3 & 2 & 1 & 1* \\
        \hline
        Ось X работает & Мост полностью собран (можно без каретки), установлен на тестовый стенд, привод оси X подключен к контроллеру, показано перемещение моста двигателем по всей длине путей. Управлять можно любым образом (например, с потенциометра) &4 &3 &2 &2*\\
        \hline
        Калибровка X работает&	Как в предыдущем тесте, но показана калибровка оси X по концевому выключателю. &1 &1 &0.5 &0.5*\\
        \hline
        Ось Y работает &	Каретка собрана, механизм ее перемещения собран, подключен к контроллеру. Показано, что диапазон перемещения каретки достаточен для позиционирования к каждому из рельсовых путей. Наличие собранного моста подразумевается, но движение по X не требуется.  Наличие подъемного узла (ось Z) не требуется. & 4 & 3 & 2 & 2* \\
        \hline
        Калибровка Y работает &Как в предыдущем тесте, но показана калибровка оси Y по концевому выключателю. &1 &1 &0.5 &0.5* \\
        \hline
        Ось Z работает & Мост смонтирован, захват смонтирован, подъемный узел каретки собран, подключен, показан подъем/опускание (достаточно 2-х положений), показано, что высота в нижнем положении достаточна для захвата контейнера, а в верхнем - для перемещения контейнера над другими контейнерами (подъем не менее чем на 60 мм). &	4 &3 &2 &2* \\
        \hline
        \multicolumn{6}{|c|}{Программирование (любой демонстрируемый код должен быть показан}\\
        \multicolumn{6}{|c|}{с исходниками). Демонстрируемая функциональность может быть в одном} \\
        \multicolumn{6}{|c|}{приложении или в нескольких отдельных} \\
        \hline
        Чтение QR-кода & Показано приложение на ПК, которое распознает и выводит любой QR-код с контейнера и с платформы.  &2 &2 &1 &1* \\
        \hline
        Распознавание рамки& Показано приложение на ПК, которое распознает маркеры-рамки и выводит непрерывно обновляемые координаты их положения в кадре &2 &2 &1 &1*\\
        \hline
        Распознавание платформа/груз &Показано приложение на ПК, которое распознает (по цвету маркеров или другим признакам), находится ли в кадре платформа или контейнер. &2 &2 &1 &1* \\
        \hline
        \multicolumn{6}{|c|}{Электроника и прошивка} \\
        \hline
        Драйвера откалиброваны по току & Показано, что каждый из используемых шаговых двигателей потребляет ток не более 400-500 мА (рекомендовано 200-300 мА для оси Y).  Обязательный тест, без его прохождения команда не допускается к дальнейшим испытаниям. & 2 & 1 & 0 & 1* \\
        \hline
        Показана система команд & Показано, что прошивка контроллера умеет отрабатывать набор текстовых команд, достаточный для выполнения задания (калибровка, перемещение, подъем/опускание, захват/отпускание). Разрешается показывать на моторах "россыпью" (вне зависимости от степени готовности конструкции), команды выдаются вручную с терминала. & & 5 & 4 & 4* \\
        \hline
        \multicolumn{6}{|c|}{Интегральное тестирование} \\
        \hline
        Готовность у-ва (ручное управление) & Портальный кран полностью собран, подключен, показано перемещение по всем осям и работа схвата (любой способ управления контроллером) & & & 5 & 4 \\
        \hline
        Позиционирование по камере & Устройство находит выставленный на полигон одиночный контейнер и позиционируется над ним, с ошибкой не более 5 мм (смещение позиции схвата относительно центра контейнера). & & & 5 & 4 \\
        \hline
        Автоматический захват & Будучи установленным над контейнером, устройство по команде захватывает и поднимает его, а затем снова ставит на место. Повторить 5 раз, оценка умножается на долю удачных попыток.  Выскальзывание контейнера, сброс контейнера с высоты, смещение более чем на 5 мм относительно исходной позиции после отпускания, переворот контейнера, считаются неудачными попытками. & & & 5 & 4 \\
        \hline
        Ручная сортировка & На полигон ставится $N=10$ пустых платформ и столько же платформ с контейнерами. Управляя краном вручную, надо за 3 минуты переставить максимальное количество контейнеров на свободные платформы. Перестановка всегда выполняется с изменением обеих координат (не на соседнюю платформу).  Баллы начисляются за контейнер, горизонтально стоящий между подпорками-ограничителями.  $N$ - количество выполненных перестановок & & & $1.5 \cdot N$ & $N$ \\
        \hline
        Сканирование грузов & На полигон ставится 5 пустых платформ и столько же платформ с контейнерами. Действуя автономно, устройство перемещается над путями, сканируя расположение платформ и контейнеров.  Результаты выводятся в любом понятном текстовом формате, для каждого объекта должны отображаться тип объекта (платформа или контейнер), значение QT-кода, номер пути и позиция на пути (в мм). Максимальный балл умножается на долю правильно определенных объектов. & & & 15 & 10 \\
        \hline
    \end{longtable}
\end{center}

\section{Финальный тест} 

Команде выдается (в виде текстового файла или распечатанный) "лист сортировки". На полигон в произвольном порядке ставится 5 пустых платформ и 5 платформ с контейнерами. Действуя автономно, устройство перемещается над путями, сканируя расположение платформ и контейнеров, а затем перемещает контейнеры на нужные платформы в соответствии с листом сортировки. При начальной расстановке гарантируется, что все целевые платформы свободны.  

Баллы начисляются по формуле  $5 \cdot N + 3 \cdot P -3 \cdot F-2 \cdot M$,  где $N$ - число правильно переставленных контейнеров, $P$ - число неточно переставленных контейнеров (стоит на нужной платформе, сдвинут более чем на 5 мм, не горизонтален),  $F$ - число "испорченных" контейнеров (т.е. упавших, перевернутых, поставленных мимо платформы), $M$ - число контейнеров, переставленных не на ту платформу. Если контейнер переставлен не на ту платформу, да еще и "испорчен", то он засчитывается и как $F$ и как $M$  (т.е. -5 баллов).  Максимальная оценка - 25 баллов.

В том случае, если остается время до конца соревнований, организаторы имеют право назначить, в течение последнего дня, до 3-х временных слотов для выполнения финального теста. Команды, не успевшие подготовить устройство для финального теста к очередному временному слоту, теряют свою попытку. Командам, успевшим провести финальный тест несколько раз, засчитывается лучший результат.

Промежуточные тесты, за которые в данный день засчиталось бы 0 баллов, не проводятся.  В последний день оставшиеся промежуточные тесты проводятся только между временными слотами для финальных тестов, либо при наличии свободных полигонов.