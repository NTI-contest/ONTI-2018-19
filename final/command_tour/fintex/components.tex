Решение представляет из себя набор python скриптов (компонент), каждый
из который ответственен за определенный функционал. 

Управление скриптами происходит через параметры командной строки и
через конфигурационные файлы. Параметры командной строки описываются 
тдельно в разделах, где описывается функциональность каждого скрипта.

\twoitems{Конфигурационные файлы - файлы в формате JSON, должны читаться
из текущей директории, где происходит вызов скрипта и могут быть двух типов}
{конфигурационный файл с настройками для работы с \textit{Microsoft Face API}}
{конфигурационный файл с настройками для работы с Ethereum узлом}

\subsubsection*{Конфигурационный файл с настройками для MS Face API}

Имя: \texttt{faceapi.json}

\threeitems{Содержит следующую информацию}
{\texttt{key} --- ключ подписки для использования сервиса}
{\texttt{serviceUrl} --- URL для доступа к сервису}
{\texttt{groupId} -- группа пользователей (в терминах \textit{MS Face API}),
с которой работает в данный момент скрипт}

Пример файла:
\begin{minted}[fontsize=\footnotesize]{json}
{"key": "563879b61984550e40cbbe8d3039523c",
 "serviceUrl": "https://westeurope.api.cognitive.microsoft.com/face/v1.0/",
 "groupId": "fintech-01"}
\end{minted}

\subsubsection*{Конфигурационный файл с настройками для работы с узлом Ethereum}

Имя: \texttt{network.json}

Содержит следующую информацию:
\begin{itemize}
  \item \texttt{rpcUrl} --- URL для доступа к узлу по JSON RPC;
  
  \item \texttt{privKey} --- приватный ключ аккаунта для подписи транзакций,
  отправляемых на узел Ethereum, и для локальных вызовов;
  
  \item \texttt{gasPriceUrl} --- URL для доступа к сервису, предоставляющему
  цену за газ. Данный сервис возвращает JSON, из которого нужно взять
  значение (в \textit{gwei}) из поля \texttt{fast}:
    \begin{minted}[fontsize=\footnotesize]{json}
{"block_time":19.91, "fast":10.0, "instant":25.0, "block_number":7240426,
 "standard":5.0, "health":true, "slow":3.0}
    \end{minted}
  
  \item \texttt{defaultGasPrice} --- цена за газ в wei, использующаяся,
  если нельзя получить цену за газ из сервиса.

\end{itemize}

Пример файла:
\begin{minted}[fontsize=\footnotesize]{json}
{"rpcUrl": "https://sokol.poa.network", 
 "privKey": "c5d2460186f7233c927e7db2dcc703c0e500b653ca82273b7bfad8045d85a470", 
 "gasPriceUrl": "https://gasprice.poa.network/", 
 "defaultGasPrice": 2000000000}
\end{minted}

Если это явно не прописано в описании функционала соответствующей компоненты,
она не должна оставлять после себя никаких файлов-данных. Аналогично с
получением данных - чтение данных из файла должно происходить только в
нескольких случаях:
\begin{itemize}
  \item для получения настроек для работы с \textit{MS Face API};
  
  \item для получения видеопотока, эмулирующего камеру;
  
  \item для получения настроек для работы с Ethereum узлом;
  
  \item для получения Application Binary Interface контракта или его байткода;
  
  \item или если это явно указано в описании компоненты.

\end{itemize}
в остальных случаях данные должны быть получены из сервиса
\textit{MS Face API} или блокчейн сети. 

Все компоненты, работающие с сервисом \textit{MS Face API},
должны уметь обрабатывать ситуации, когда:
\begin{itemize}
  \item недоступно подключение к сервису (сообщение об ошибке
  \texttt{''No connection to MS Face API provider''});

  \item используется некорректный ключ подписки (сообщение об
  ошибке \texttt{''Incorrect subscription key''}).

\end{itemize}

Компоненты, отправляющие транзакции в блокчейн, должны уметь обрабатывать
ситуации, когда:
\begin{itemize}
  \item недоступно подключение к узлу Ethereum (сообщение об ошибке
  \texttt{''No connection to node''});

  \item на счету аккаунта, от чьего имени выполняется транзакция,
  недостаточно средств, чтобы покрыть все затраты по использованному
  газу (сообщение об ошибке \texttt{''No enough funds to send transaction''});

  \item транзакция может не включаться в блок продолжительное время в
  зависимости от нагрузки на сеть (сообщение об ошибке
  \texttt{''Transaction is not validated too long''}).

\end{itemize}

При возникновении данных событий в терминал должно выдаваться сообщение
понятное пользователю, а не stack trace.

Возможны также другие конфигурационные файлы, необходимые для работы
тех или иных подсистем решения. Данные файлы будут отдельно описываться
в соответствующих разделах. 

\subsection*{Настройка сервиса KYC}


Развертывание и настройка в блокечейн сети регистра соответствий аккаунтов и телефонных номеров (Know Your Customer, KYC) выполняется компанией производителем программного обеспечения (тем, кто разрабатывает данный сервис). Тоже самое относится к обработчику сертификатов на получение средств. Предполагается, что в потенциальные пользователи сервиса (пользователи, регистрирующие соответствия, владельцы платежных терминалов) могут ознакомиться с опубликованным в открытом доступе исходным кодом контрактов, принять решение на основе этого, доверять ли этому разработчику и начать пользоваться сервисом.

Фиксация контрактов в блокчейн гарантирует, что приозводитель не сможет изменить логику работы приложения позднее, т.е. условия участия в системе, с которыми ознакомились пользователи, не поменяется.

Функционал сервиса KYC должен быть доступен сразу, минуя фазы настройки, когда контракт зарегистрирован в блокчейн, но еще находится в нерабочем состоянии. Поэтому первичная настройка базовых параметров по максимуму должна выполняться вовремя инициализации контракта во время его регистрации (deploy).

\begin{myverbbox}{\scriptFile}
setup.py <command> [options]
\end{myverbbox}
\scriptTitle


%-----------------------------------------------------
%US-001
\newuserstory{Регистрация контракта}



\begin{myverbbox}[\small]{\cmdLine}
$ setup.py --deploy
\end{myverbbox}
\scriptExample{
Подключается к узлу Ethereum и регистрирует следующие контракты
\begin{itemize}
  \item регистр соответствий аккаунтов и телефонных номеров в сети блокчейн. 
  \item обработчик сертификатов на получение средств.
\end{itemize}

На терминал выводятся адреса зарегистрованных контрактов. В текущей рабочей директории создается JSON файл \texttt{registrar.json}, содержащий адреса зарегистрированных контрактов.

}

% AC-001-01
\begin{myverbbox}[\small]{\output}
$ cat registrar.json
cat: registrar.json: No such file or directory
$ setup.py --deploy
KYC Registrar: 0x00360d2b7D240Ec0643B6D819ba81A09e40E5bCd
Payment Handler: 0x95426f2bC716022fCF1dEf006dbC4bB81f5B5164
$ cat registrar.json
{"registrar": {"address": "0x00360d2b7D240Ec0643B6D819ba81A09e40E5bCd", "
startBlock": 123456}, "payments": {"address": "0x95426f2bC716022fCF1dEf00
6dbC4bB81f5B5164", "startBlock": 123457}}
\end{myverbbox}
\acceptanceCriteria{В блокчейн сеть отправляются транзакции для регистрации и первичной настройке контрактов. Транзакции успешно включены в один
или несколько блоков. Для проведения транзакции выбрана цена из значения \texttt{fast}, возвращенного сервисом \texttt{https://gasprice.poa.network}. Контракты по адресу, выведенным в терминале, созданы одной из отправленных транзакций, и могут быть просмотрены с помощью браузера блоков.
}

% AC-001-02
\begin{myverbbox}[\small]{\output}
$ cat registrar.json
{"registrar": {"address": "0x00360d2b7D240Ec0643B6D819ba81A09e40E5bCd", "
startBlock": 123456}, "payments": {"address": "0x95426f2bC716022fCF1dEf00
6dbC4bB81f5B5164", "startBlock": 123457}}
$ setup.py --deploy
KYC Registrar: 0x23B40E5bCd06D819ba81A09e0340Ec06460d2b7D
Payment Handler: 0xE797A1da01eb0F951E0E400f9343De9d17A06bac
$ cat registrar.json
{"registrar": {"address": "0x23B40E5bCd06D819ba81A09e0340Ec06460d2b7D", "
startBlock": 456123}, "payments": {"address": "0xE797A1da01eb0F951E0E400f
9343De9d17A06bac", "startBlock": 456125}}
\end{myverbbox}
\acceptanceCriteria{В блокчейн сеть отправляются транзакции для регистрации и первичной настройке контрактов. Транзакции успешно включены в один
или несколько блоков. Для проведения транзакции выбрана цена из значения \texttt{defaultGasPrice} из файла \texttt{network.json}.
}

%-----------------------------------------------------
%US-002
\newuserstory{Вывод владельца контракта регистра соответствий}



\begin{myverbbox}[\small]{\cmdLine}
$ setup.py --owner registrar
\end{myverbbox}
\scriptExample{
Подключается к узлу Ethereum и получает адрес аккаунта, имеющего полномочия выполнять действия по подверждению запросов на регистрацию и удалению соответствий аккаунтов и телефонных номеров в сети блокчейн. 

На терминал выводится адрес аккаунта.

}

% AC-002-01
\begin{myverbbox}[\small]{\output}
$ cat network.json | python -mjson.tool | grep privKey 
    "privKey": "c5d2460186f7233c927e7db2dcc703c0e500b653ca82273b7bfad8045
d85a470",
$ setup.py --deploy
KYC Registrar: 0x9FdddF5bf10c65221da0a78ADAFec1D8E9EF0A7D
Payment Handler: 0xD79A8FDB771Ea12359270aD7020bcCB328C9f5f7
$ setup.py --owner registrar
Admin account: 0x9cce34F7aB185c7ABA1b7C8140d620B4BDA941d6
\end{myverbbox}
\acceptanceCriteria{Транзакции в блокчейн сеть не отправляются.
}

% AC-002-02
\begin{myverbbox}[\small]{\output}
$ cat network.json | python -mjson.tool | grep privKey 
    "privKey": "64e604787cbf194841e7b68d7cd28786f6c9a0a3ab9f8b0a0e87cb438
7ab0107",
$ cat registrar.json
{"registrar": {"address": "0x9FdddF5bf10c65221da0a78ADAFec1D8E9EF0A7D", "
startBlock": 234156}, "payments": {"address": "0xD79A8FDB771Ea12359270aD7
020bcCB328C9f5f7", "startBlock": 451247}}
$ setup.py --owner registrar
Admin account: 0x9cce34F7aB185c7ABA1b7C8140d620B4BDA941d6
\end{myverbbox}
\acceptanceCriteria{Транзакции в блокчейн сеть не отправляются.
}

%-----------------------------------------------------
%US-003
\newuserstory{Изменение владельца контракта регистра соответствий}



\begin{myverbbox}[\small]{\cmdLine}
$ setup.py --chown registrar <address>
\end{myverbbox}
\scriptExample{
Подключается к узлу Ethereum и отправляет транзакцию на изменение аккаунта, имеющего полномочия выполнять действия по подверждению запросов на регистрацию и удалению соответствий аккаунтов и телефонных номеров в сети блокчейн. Только аккаунт, в текущее время обладающий полномочиями на выполнение вышеуказанных действий, имеет возможность производить данное изменение.

На терминал выводится адрес нового аккаунта.

}

% AC-003-01
\begin{myverbbox}[\small]{\output}
$ cat network.json | python -mjson.tool | grep privKey 
    "privKey": "c5d2460186f7233c927e7db2dcc703c0e500b653ca82273b7bfad8045
d85a470",
$ setup.py --owner registrar
Admin account: 0x9cce34F7aB185c7ABA1b7C8140d620B4BDA941d6
$ setup.py --chown registrar 0x6455f1445c72ba9460c6f6ab364d3935a0ad4559
New admin account: 0x6455F1445c72BA9460C6f6AB364d3935a0AD4559
$ setup.py --owner registrar
Admin account: 0x6455F1445c72BA9460C6f6AB364d3935a0AD4559
\end{myverbbox}
\acceptanceCriteria{В блокчейн сеть отправляется транзакция для изменения аккаунта. Транзакция успешно верифицирована и включена в блок. Для проведения транзакции выбрана цена из значения \texttt{fast}, возвращенного сервисом \texttt{https://gasprice.poa.network}.
}

% AC-003-02
\begin{myverbbox}[\small]{\output}
$ cat network.json | python -mjson.tool | grep privKey 
    "privKey": "c5d2460186f7233c927e7db2dcc703c0e500b653ca82273b7bfad8045
d85a470",
$ setup.py --owner registrar
Admin account: 0x6455F1445c72BA9460C6f6AB364d3935a0AD4559
$ setup.py --chown registrar 0x11460ff94ca4212d9d02b5bea1766dd099e0a9df
Request cannot be executed
$ setup.py --owner registrar
Admin account: 0x6455F1445c72BA9460C6f6AB364d3935a0AD4559
\end{myverbbox}
\acceptanceCriteria{Транзакции в блокчейн сеть не отправляются.
}

% AC-003-03
\begin{myverbbox}[\small]{\output}
$ cat network.json | python -mjson.tool | grep gasPriceUrl 
    "gasPriceUrl": "https://gasprice.poa.network/",
$ curl https://gasprice.poa.network/
curl: (6) Could not resolve host: gasprice.poa.network
$ setup.py --chown registrar 0x11460ff94ca4212d9d02b5bea1766dd099e0a9df
New admin account: 0x11460ff94cA4212d9D02b5bEA1766Dd099E0A9DF
\end{myverbbox}
\acceptanceCriteria{В блокчейн сеть отправляется транзакция для изменения аккаунта. Транзакция успешно верифицирована и включена в блок. Для проведения транзакции выбрана цена из значения \texttt{defaultGasPrice} из файла \texttt{network.json}.
}

% AC-003-04
\begin{myverbbox}[\small]{\output}
$ cat network.json | python -mjson.tool | grep privKey 
    "privKey": "c5d2460186f7233c927e7db2dcc703c0e500b653ca82273b7bfad8045
d85a470",
$ setup.py --owner registrar
Admin account: 0x9cce34F7aB185c7ABA1b7C8140d620B4BDA941d6
$ setup.py --chown registrar 0x6455f1445c72ba9460c6f6ab364d3935a0ad4559
New admin account: 0x6455F1445c72BA9460C6f6AB364d3935a0AD4559
$ setup.py --owner registrar
Admin account: 0x6455F1445c72BA9460C6f6AB364d3935a0AD4559
\end{myverbbox}
\acceptanceCriteria{Если из транзакции, которая была отправлена в результате команды \texttt{setup.py --chown}, извлечь поле \texttt{input} и отправить его в новой транзакции снова в поле \texttt{input} на адрес контракта регистра соответствий, то эта транзакция будет включена в блок, но статус ее исполнения будет - ошибка, поскольку новый аккаунт, обладающий полномочиями, был изменен транзакцией, посланной командо \texttt{setup.py --chown}. Следовательно, изменение аккаунта с полномочиями выполнять действия по подверждению запросов на регистрацию и удалению соответствий аккаунтов и телефонных номеров не происходит. Статус можно подтвердить для данной транзакции в браузере блоков.
}


\subsection*{Подотовка сервиса идентификации}


Сервис идентификации человека по лицу перед полноценной работой требует предварительной настройки. Первое, что должно быть сделано - в сервис необходимо добавить лица людей, которых в дальнейшем необходимо идентифицировать. Поскольку система автоматического тестирования не может работать с камерой, то изображения человека будут передаваться через видео-файл, передаваемый в сервис через параметры командной строки.

Администратор свервиса должен иметь возможность просматривать список добавленных пользователей, удалять пользователя (и его изображения) из системы. 

Как только необходимое количество лиц зарегистрировано в сервисе, администратор может запустить обучение нейронной сети.

\begin{myverbbox}{\scriptFile}
face-management.py <command> [options]
\end{myverbbox}
\scriptTitle


%-----------------------------------------------------
%US-004
\newuserstory{Простое добавление пользователя в сервис индентификации}


Сервис должен быть устроен так, что добавление изображений человека в систему происходит анонимно, т.е. имя при добавлении не указывается. 


\begin{myverbbox}[\small]{\cmdLine}
$ face-management.py --simple-add <path to video file>
\end{myverbbox}
\scriptExample{
При использовании команды простого добавления пользователя из видео-потока извлекается 5 кадров с изображением лица человека. При этом подразумевается, что все кадры в видео приндалежат одному и тому же человеку. 

Если в видео-потоке недостаточно кадров с изображением человека, то обработка такого видео должно приводить к ошибке.

}

% AC-004-01
\begin{myverbbox}[\small]{\output}
$ cat faceapi.json | python -mjson.tool | grep groupId
    "groupId": "fintech-01",
$ curl -X GET "https://<datacenter url>/face/v1.0/persongroups/fintech-01
" -H "Content-Type: application/json" -H "Ocp-Apim-Subscription-Key: 0000
00000000000000000000000000000" 
{"error":{"code":"PersonGroupNotFound","message":"Person group is not fou
nd.\r\nParameter name: personGroupId"}}
$ face-management.py --simple-add /path/to/video.avi
5 frames extracted
PersonId: 419e345a-e6d6-4d9c-953d-667787b8d52e
FaceIds
=======
e27558b9-812d-41c3-b114-8e434b8f4602
44c350f2-6653-4616-a1b7-e0fe9b481b6b
f945a3be-4b20-4049-b080-4142a55e4f93
855ab7c2-9bb3-49ed-8cac-1366c0274b08
9c4af288-54cd-4375-8eef-f8c29ed56685
\end{myverbbox}
\acceptanceCriteria{Требуемый \texttt{personGroupId} не существовал до этого в сервисе \textit{Microsoft Face API}. После добавления \texttt{personGroupId} добавляется новый \texttt{personId}, с которым ассоциируется 5 изображений лица (\texttt{persistedFaceId}).
}

% AC-004-02
\begin{myverbbox}[\small]{\output}
$ cat faceapi.json | python -mjson.tool | grep groupId
    "groupId": "fintech-01",
$ curl -X GET "https://<datacenter url>/face/v1.0/persongroups" -H "Conte
nt-Type: application/json" -H "OcApim-Subscription-Key: 00000000000000000
0000000000000000" 
[{"personGroupId":"fintech-01","name":"fintech-01","userData":null}]
$ face-management.py --simple-add /path/to/video1.avi
5 frames extracted
PersonId: 37da04e7-f471-49c7-a54c-a08f05950fc5
FaceIds
=======
1d499868-3d01-487c-8bab-626dc562e4e8
27dadf08-bc60-4a29-82a7-7d21ea7f40af
b8cf9c2f-a606-4f21-851d-26e0a0dc8a74
bf4806de-8c4b-4a12-8495-002f43dba797
ff79486f-15ac-43be-9c6c-b2840f8c8d22
\end{myverbbox}
\acceptanceCriteria{Требуемый \texttt{personGroupId} существует в \textit{Microsoft Face API}. В данную группу добавляется новый \texttt{personId}, с которым ассоциируется 5 изображений лица (\texttt{persistedFaceId}).
}

% AC-004-03
\begin{myverbbox}[\small]{\output}
face-management.py --simple-add /path/to/video2.avi
Video does not contain any face
\end{myverbbox}
\acceptanceCriteria{Выдается ошибка при попытке обработать видео, в котором либо нет кадров с лицом пользователя, либо в видео содержится меньше 5 кадров. Группа с \texttt{personGroupId} не создается, новый \texttt{personId} не добавляется.
}

% AC-004-04
\begin{myverbbox}[\small]{\output}
$ face-management.py --simple-add /path/to/video1.avi
5 frames extracted
PersonId: 52865cde-3af8-443d-b260-9319c2cb1788
FaceIds
=======
cdb6227e-7453-4057-b4fa-79660914e597
6976d3c2-dee5-4f24-8950-f38ff10c70ad
fae15e55-6639-42a4-a954-731c33310e41
15092567-5765-49ed-ac63-94bc5fa08d17
a77f1f0a-aa95-4bd1-9826-6b453aec42b2
$ face-management.py --simple-add /path/to/video31.avi
5 frames extracted
PersonId: 9fa0a99b-8e76-474d-8223-dea217c2c19b
FaceIds
=======
b552ef11-a162-4a7d-9047-ccfc84a07043
90c0815a-ecce-45c6-8107-ced7ef29a249
fde35dba-505d-4a62-ac5a-c6ae4c89128e
6c6910b4-0ab5-4eb4-9e53-95b1929f9867
fdb9d352-65b0-41a2-a1be-03ea5b543160
$ face-management.py --simple-add /path/to/video41.avi
5 frames extracted
PersonId: f290ecb9-bfab-46f7-b623-45140d730628
FaceIds
=======
e5735ecd-ca09-4fd4-bfd3-8ace67702ab0
9e1bbdee-5981-4f6b-aba5-03be57e5e910
3120ef58-8d53-4558-8b84-784ba338f621
8fceb9c7-f029-4326-9703-6749005674fa
8ea02a3b-7dc0-455a-858c-67251b0ca3b4
\end{myverbbox}
\acceptanceCriteria{Несколько добавлений пользователя проходят успешно. Для каждого видео добавляется новый \texttt{personId} добавляется.
}

%-----------------------------------------------------
%US-005
\newuserstory{Улучшенное добавление пользователя в сервис индентификации}


Наличие в наборе изображений одного и того же человека, но у которых есть различия в мимике, положении головы и освещенности, влияет на качество обучения нейронной сети сервиса, поэтому при сборе данных для сервиса идентификации важно собирать разные изображения.


\begin{myverbbox}[\small]{\cmdLine}
$ face-management.py --add <path to video file 1> [ <path to video file 2
> [ <path to video file 3> [ <path to video file 4> [ <path to video file
 5> ] ] ] ]
\end{myverbbox}
\scriptExample{
Команда ожидает 5 видео файлов. Требования к каждому из видео файлов:
\begin{itemize}
  \item В первом по списку видеофайле лицо человека должно быть неподвижно (допускаются небольшие повороты)
  \item Во втором по списку видеофайле должны быть зафиксированы наклоны головы влево и вправо (\textit{roll})
  \item В третьем по списку видеофайле должны быть зафиксированы повороты головы влево и вправо (\textit{yaw})
  \item В четвертом видеофайле должен быть зафиксирован открытый рот
  \item В пятом - должно быть зафиксировано закрытие глаз
\end{itemize}

}

% AC-005-01
\begin{myverbbox}[\small]{\output}
$ cat faceapi.json | python -mjson.tool | grep groupId
    "groupId": "fintech-01",
$ curl -X GET "https://<datacenter url>/face/v1.0/persongroups/fintech-01
" -H "Content-Type: application/json" -H "Ocp-Apim-Subscription-Key: 0000
00000000000000000000000000000" 
{"error":{"code":"PersonGroupNotFound","message":"Person group is not fou
nd.\r\nParameter name: personGroupId"}}
$ face-management.py --add /path/to/video1.avi
5 frames extracted
PersonId: ddaa7036-cab0-4d8f-9b36-18f20e294c51
FaceIds
=======
5754a049-0de7-4ee5-99ba-45d0d3398645
56c0aa34-50c4-4d0b-bcf0-c86e9c9dec52
f626c3d7-1126-401c-b125-16b6c65e6ed8
6d617030-941d-4f6c-9d05-b714b4e2b504
1087c13c-09d7-4053-8bc6-9381c48081e2
\end{myverbbox}
\acceptanceCriteria{Требуемый \texttt{personGroupId} не существовал до этого в сервисе \textit{Microsoft Face API}. После добавления \texttt{personGroupId} добавляется новый \texttt{personId}, с которым ассоциируется 5 изображений лица (\texttt{persistedFaceId}). Изображение лица человека не должно значительно отличаться от кадра к кадру (допускаются повороты до 5 градусов).
}

% AC-005-02
\begin{myverbbox}[\small]{\output}
$ cat faceapi.json | python -mjson.tool | grep groupId
    "groupId": "fintech-01",
$ curl -X GET "https://<datacenter url>/face/v1.0/persongroups" -H "Conte
nt-Type: application/json" -H "OcApim-Subscription-Key: 00000000000000000
0000000000000000" 
[{"personGroupId":"fintech-01","name":"fintech-01","userData":null}]
$ face-management.py --add /path/to/video1.avi
5 frames extracted
PersonId: 6dc70d27-3bad-400a-983f-fec6591cff6b
FaceIds
=======
08609f0a-55ee-46ed-899b-5f7d66f62cce
2bb6d3a9-1c52-4a70-ac62-44881f2aed29
4309d3b7-5250-47e3-bc7e-f3d1da8badd1
76b6a516-c378-4d86-b4cc-03b60b5c2d2c
85f2af43-ca0e-4dd1-a400-9780d7b7a8f5
\end{myverbbox}
\acceptanceCriteria{В требуемую \texttt{personGroupId} добавляется новый \texttt{personId}, с которым ассоциируется 5 изображений лица (\texttt{persistedFaceId}). Изображение лица человека не должно значительно отличаться от кадра к кадру (допускаются повороты до 5 градусов).
}

% AC-005-03
\begin{myverbbox}[\small]{\output}
$ face-management.py --add /path/to/video2.avi
Base video does not follow requirements
\end{myverbbox}
\acceptanceCriteria{Изображение лица человека в видео значительно отличаться от кадра к кадру (повороты больше чем на 5 градусов либо есть открый рот, либо есть закрытые глаза). Группа с \texttt{personGroupId} не создается, новый \texttt{personId} не добавляется.
}

% AC-005-04
\begin{myverbbox}[\small]{\output}
$ face-management.py --add /path/to/video1.avi /path/to/video2.avi
10 frames extracted
PersonId: 8d0b7195-c172-4d1a-8588-063f3e64c0c0
FaceIds
=======
08609f0a-55ee-46ed-899b-5f7d66f62cce
2bb6d3a9-1c52-4a70-ac62-44881f2aed29
4309d3b7-5250-47e3-bc7e-f3d1da8badd1
76b6a516-c378-4d86-b4cc-03b60b5c2d2c
85f2af43-ca0e-4dd1-a400-9780d7b7a8f5
300c161f-7154-43c4-80bc-34d1fd9ffb1d
4f57f616-23ea-4be7-b977-8e9b14a192b4
52c9b5c4-6cb7-4aae-b9d0-206d6db90510
8f9a5e7f-c2c4-4d92-908b-6f220b4d3b32
d45f9324-a845-4db6-abc6-88214726bdcc
\end{myverbbox}
\acceptanceCriteria{В требуемую \texttt{personGroupId} добавляется новый \texttt{personId}, с которым ассоциируется 10 изображений лица (\texttt{persistedFaceId}). Помимо 5 изображений лиц из первого видео, добавляется 5 - из второго.

Лицо человека во втором видео должно наклонятся из крайнего положения влево к крайнему положению вправо. Зафиксирован максимальный наклон в каждую из сторон не меньше 30 градусов. Из второго видео выбрано 5 кадров с наклонами головы из следующего списка: 30 градусов влево, 15 градусов влево, 0 градусов, 15 градусов вправо, 30 градусов вправо. Допускается отклонение от перечисленных значений +/- 3 градуса.
}

% AC-005-05
\begin{myverbbox}[\small]{\output}
$ face-management.py --add /path/to/video1.avi /path/to/video3.avi
Roll video does not follow requirements
\end{myverbbox}
\acceptanceCriteria{Во втором по счету видео лицо человека не доходит до крайних положений. Т.е. нельзя зафиксировать максимальный наклон в каждую из сторон от 30 градусов и более. Группа с \texttt{personGroupId} не создается, новый \texttt{personId} не добавляется.
}

% AC-005-06
\begin{myverbbox}[\small]{\output}
$ face-management.py --add /path/to/video1.avi /path/to/video2.avi /path/
to/video3.avi
15 frames extracted
PersonId: a25ddbf2-aabf-41c7-b9a9-d69f1c055761
FaceIds
=======
0ca5925b-55eb-44fd-9624-39fbaa33a5c5
31091841-2e3c-43c2-a14c-497fd8137d25
3f46e69d-33d7-4474-97a1-230046dc9cfa
f1752749-f602-4151-b3bb-7cec627de3df
f556db9a-55cd-4542-bf0f-a5848376ba66
08609f0a-55ee-46ed-899b-5f7d66f62cce
2bb6d3a9-1c52-4a70-ac62-44881f2aed29
4309d3b7-5250-47e3-bc7e-f3d1da8badd1
76b6a516-c378-4d86-b4cc-03b60b5c2d2c
85f2af43-ca0e-4dd1-a400-9780d7b7a8f5
300c161f-7154-43c4-80bc-34d1fd9ffb1d
4f57f616-23ea-4be7-b977-8e9b14a192b4
52c9b5c4-6cb7-4aae-b9d0-206d6db90510
8f9a5e7f-c2c4-4d92-908b-6f220b4d3b32
d45f9324-a845-4db6-abc6-88214726bdcc
\end{myverbbox}
\acceptanceCriteria{В требуемую \texttt{personGroupId} добавляется новый \texttt{personId}, с которым ассоциируется 15 изображений лица (\texttt{persistedFaceId}). Помимо 10 изображений лиц из первого и второго видео, добавляется 5 - из третьего.

Лицо человека в третьем видео должно поворачиваться из крайнего положения влево к крайнему положению вправо. Зафиксирован максимальный поворот в каждую из сторон не меньше 20 градусов. Из третьего видео выбрано 5 кадров с наклонами головы из следующего списка: 20 градусов влево, 10 градусов влево, 0 градусов, 10 градусов вправо, 20 градусов вправо. Допускается отклонение от перечисленных значени +/- 3 градуса.
}

% AC-005-07
\begin{myverbbox}[\small]{\output}
$ face-management.py --add /path/to/video1.avi /path/to/video2.avi /path/
to/video4.avi
Yaw video does not follow requirements
\end{myverbbox}
\acceptanceCriteria{Во третьем по счету видео лицо человека не доходит до крайних положений. Т.е. нельзя зафиксировать максимальный поворот в каждую из сторон от 20 градусов и более. Группа с \texttt{personGroupId} не создается, новый \texttt{personId} не добавляется.
}

% AC-005-08
\begin{myverbbox}[\small]{\output}
$ face-management.py --add /path/to/video1.avi /path/to/video2.avi /path/
to/video3.avi /path/to/video4.avi
16 frames extracted
PersonId: e3152bc7-431c-4215-a19b-d128f2768438
FaceIds
=======
100d7ff0-0e95-4998-a385-d65a69cbbab4
1ca71243-1370-4c7f-b457-ac53ade0b59a
25f51952-880c-435e-9afb-3ca8b2b981d4
cf17ee9e-140b-444b-a87a-35209a631aa1
f0ae9fd1-81bd-4fcc-853c-107e228be383
2307cf52-687b-4d51-9861-728b6297aa7d
34ac19bb-4018-498b-bb13-02098aaac75c
7e01b585-78b1-42a0-8e48-4b1ef199dde7
99943bc7-f92e-49fb-91b4-06b123343982
ae8b5917-ee88-4499-b262-8bfbfda687ca
8195743c-ae84-48fe-b381-1eb73c82f5c1
87ad1c82-0ffd-4b39-8b72-fa6323257802
9cb6fc25-f32e-49d7-b6de-352bf2d911fc
bff40657-1ea4-4667-b2aa-d8b480f527be
d4b57970-6390-4c23-a366-0c51fdc17f9b
1f41144e-eee5-4210-8a7d-ffa4adc3ba21
\end{myverbbox}
\acceptanceCriteria{В требуемую \texttt{personGroupId} добавляется новый \texttt{personId}, с которым ассоциируется 16 изображений лица (\texttt{persistedFaceId}). Помимо 15 изображений лиц из первого-третьего видео, добавляется 1 изображение - из четвертого.

На серии кадров человека в четвертом видео рот человека должен быть открыт. Один из кадров с открытым ртом отправляется в \textit{Microsoft Face API}.
}

% AC-005-09
\begin{myverbbox}[\small]{\output}
$ face-management.py --add /path/to/video1.avi /path/to/video2.avi /path/
to/video3.avi /path/to/video5.avi
Video to detect open mouth does not follow requirements
\end{myverbbox}
\acceptanceCriteria{Во четвертом по счету видео не найдены кадры с открытым ртом. Группа с \texttt{personGroupId} не создается, новый \texttt{personId} не добавляется.
}

% AC-005-10
\begin{myverbbox}[\small]{\output}
$ face-management.py --add /path/to/video1.avi /path/to/video2.avi /path/
to/video3.avi /path/to/video4.avi /path/to/video5.avi
18 frames extracted
PersonId: 588e76bc-a0ce-4066-94d2-96620d105401
FaceIds
=======
220dd1ea-3392-466b-ab50-fbb465a513ff
4977b87d-3c36-45d7-b827-8907aa66b918
29e7d8fd-f065-41b6-9fd0-b191fb3dde78
d0e269f1-199a-459e-9fbf-1c24e015fdec
94d0887a-4e0a-4b86-9e6e-8f6fdd45f1ec
12938747-8530-4bcc-a811-b890aa852175
2d486b65-ed15-4313-836c-2cd4ca7e02cb
9a1ab5a7-9e46-4566-8d85-54bbcd021307
7a6f7afd-a1f6-4d7a-9e6f-c5e0daf1501a
105dfcaa-a25d-45ff-ba38-48de04a735e6
090707ff-91b8-4647-afbd-a80f1976c10d
65f53802-dfa0-46a0-b9e5-c4780f41d580
8c485998-ea43-439b-a84e-606433232133
7b0166db-f7a2-477c-b3f6-bbf3833ea770
191f33fc-501f-4d1a-8b5f-a4f9917f311a
44c79f24-5f87-48c6-af1f-f85e0fca521e
cdd4e100-217d-4163-8f51-4c8f82282f46
e417cb0c-ad18-47eb-8ffa-d850f3aaa8ec
\end{myverbbox}
\acceptanceCriteria{В требуемую \texttt{personGroupId} добавляется новый \texttt{personId}, с которым ассоциируется 18 изображений лица (\texttt{persistedFaceId}). Помимо 16 изображений лиц из первого-четвертого видео, добавляется два изображения - из пятого.

На серии кадров человека в пятом видео был закрыт сначала левый глаз, потом правый. По одному кадру с каждым из закрытых глаз отправляется в \textit{Microsoft Face API}.
}

% AC-005-11
\begin{myverbbox}[\small]{\output}
$ face-management.py --add /path/to/video1.avi /path/to/video2.avi /path/
to/video3.avi /path/to/video4.avi /path/to/video6.avi
18 frames extracted
PersonId: 5d5d6e92-f8f0-47cf-8e9b-b4455092603e
FaceIds
=======
0e5edd8e-7756-470e-a26c-f54ab70a5524
73b4caea-c85c-4594-9153-198351937d94
2ae3c850-190a-4cd0-afd2-8efa341767ba
adb7744e-b2fb-4a58-b3a8-f22c3143a8cd
7e0f87d6-14ad-4ec6-971f-e4c2771cc267
47c8c307-1f90-473c-b48c-792d5b5a1241
33e35a22-358a-4847-bcdd-4a5bdfd5eb69
2c7b951b-126a-4eec-9e22-1d9d65ece9e3
bcaf7f6a-1c44-476e-a51f-9dde2744ade0
1c9c0134-ada0-47c2-848c-d4424b232f04
b79945e1-de47-4011-98f2-e7e8f0d9f3ef
f972b6ac-078b-4e3d-a779-4cde8ea28f36
fbeb7aa1-f66e-4412-ad07-b61103e8af0b
9c762e64-075b-4fa4-8e39-0d83df86428b
9794e349-e4e4-44ad-951c-610b4890b16e
2e93da04-8f31-4a27-bd9e-5768335447a5
b9227512-dfa8-4a17-94e2-be520b8975a9
b67c1f86-2e3d-45a2-a7fb-57bbbff9cd8c
\end{myverbbox}
\acceptanceCriteria{В требуемую \texttt{personGroupId} добавляется новый \texttt{personId}, с которым ассоциируется 18 изображений лица (\texttt{persistedFaceId}). Помимо 16 изображений лиц из первого-четвертого видео, добавляется два изображения - из пятого.

На серии кадров человека в пятом видео был закрыт сначала правый глаз потом левый. По одному кадру с каждым из закрытых глаз отправляется в \textit{Microsoft Face API}.
}

% AC-005-12
\begin{myverbbox}[\small]{\output}
$ face-management.py --add /path/to/video1.avi /path/to/video2.avi /path/
to/video3.avi /path/to/video4.avi /path/to/video7.avi
Video to detect closed eyes does not follow requirements
\end{myverbbox}
\acceptanceCriteria{Во пятом по счету видео не найдены кадры с закрытыми глазами. Причем видео считается не удовлетворяющим требованиям, если только один глаз был закрыт. Группа с \texttt{personGroupId} не создается, новый \texttt{personId} не добавляется.
}

% AC-005-13
\begin{myverbbox}[\small]{\output}
$ face-management.py --add /path/to/video11.avi /path/to/video12.avi /pat
h/to/video13.avi /path/to/video14.avi /path/to/video15.avi
Roll video does not follow requirements
\end{myverbbox}
\acceptanceCriteria{В каком-то видео не найдены кадры, удовлетворяющие требованиям. Группа с \texttt{personGroupId} не создается, новый \texttt{personId} не добавляется.
}

%-----------------------------------------------------
%US-006
\newuserstory{Получение всех пользователей из сервиса идентификации}


Администратор сервиса может получить список всех добавленных пользователей. 


\begin{myverbbox}[\small]{\cmdLine}
$ face-management.py --list
\end{myverbbox}
\scriptExample{


}

% AC-006-01
\begin{myverbbox}[\small]{\output}
$ cat faceapi.json | python -mjson.tool | grep groupId
    "groupId": "fintech-01",
$ curl -X GET "https://<datacenter url>/face/v1.0/persongroups/fintech-01
" -H "Content-Type: application/json" -H "Ocp-Apim-Subscription-Key: 0000
00000000000000000000000000000" 
{"error":{"code":"PersonGroupNotFound","message":"Person group is not fou
nd.\r\nParameter name: personGroupId"}}
$ face-management.py --list
The group does not exist
\end{myverbbox}
\acceptanceCriteria{Требуемый \texttt{personGroupId} не существовует в сервисе \textit{Microsoft Face API} ни до, ни после выполнения команды.
}

% AC-006-02
\begin{myverbbox}[\small]{\output}
$ cat faceapi.json | python -mjson.tool | grep groupId
    "groupId": "fintech-01",
$ curl -X GET "https://<datacenter url>/face/v1.0/persongroups" -H "Conte
nt-Type: application/json" -H "OcApim-Subscription-Key: 00000000000000000
0000000000000000" 
[{"personGroupId":"fintech-01","name":"fintech-01","userData":null}]
$ face-management.py --list
Persons IDs:
27dadf08-bc60-4a29-82a7-7d21ea7f40af
b8cf9c2f-a606-4f21-851d-26e0a0dc8a74
bf4806de-8c4b-4a12-8495-002f43dba797
ff79486f-15ac-43be-9c6c-b2840f8c8d22
\end{myverbbox}
\acceptanceCriteria{Требуемый \texttt{personGroupId} существует в \textit{Microsoft Face API}.
}

% AC-006-03
\begin{myverbbox}[\small]{\output}
$ cat faceapi.json | python -mjson.tool | grep groupId
    "groupId": "fintech-01",
$ curl -X GET "https://<datacenter url>/face/v1.0/persongroups" -H "Conte
nt-Type: application/json" -H "OcApim-Subscription-Key: 00000000000000000
0000000000000000" 
[{"personGroupId":"fintech-01","name":"fintech-01","userData":null}]
$ face-management.py --list
No persons found
\end{myverbbox}
\acceptanceCriteria{Требуемый \texttt{personGroupId} существует в \textit{Microsoft Face API}, но в ней нет пользователей.
}

%-----------------------------------------------------
%US-007
\newuserstory{Удаление пользователя из сервиса идентификации}


Администратор сервиса может удалить пользователя сервиса по его идентификатору. 


\begin{myverbbox}[\small]{\cmdLine}
$ face-management.py --del <person id>
\end{myverbbox}
\scriptExample{


}

% AC-007-01
\begin{myverbbox}[\small]{\output}
$ cat faceapi.json | python -mjson.tool | grep groupId
    "groupId": "fintech-01",
$ curl -X GET "https://<datacenter url>/face/v1.0/persongroups/fintech-01
" -H "Content-Type: application/json" -H "Ocp-Apim-Subscription-Key: 0000
00000000000000000000000000000" 
{"error":{"code":"PersonGroupNotFound","message":"Person group is not fou
nd.\r\nParameter name: personGroupId"}}
$ face-management.py --del 27dadf08-bc60-4a29-82a7-7d21ea7f40af
The group does not exist
\end{myverbbox}
\acceptanceCriteria{Требуемый \texttt{personGroupId} не существовует в сервисе \textit{Microsoft Face API} ни до, ни после выполнения команды.
}

% AC-007-02
\begin{myverbbox}[\small]{\output}
$ cat faceapi.json | python -mjson.tool | grep groupId
    "groupId": "fintech-01",
$ curl -X GET "https://<datacenter url>/face/v1.0/persongroups" -H "Conte
nt-Type: application/json" -H "OcApim-Subscription-Key: 00000000000000000
0000000000000000" 
[{"personGroupId":"fintech-01","name":"fintech-01","userData":null}]
$ face-management.py --del 27dadf08-bc60-4a29-82a7-7d21ea7f40af
Person deleted
\end{myverbbox}
\acceptanceCriteria{Требуемый \texttt{personGroupId} существует в \textit{Microsoft Face API}. Пользователь с заданным ID удаляется из сервиса.
}

% AC-007-03
\begin{myverbbox}[\small]{\output}
$ face-management.py --del bdaf190a-4805-4e2c-95af-1afa8b2623df
The person does not exist
\end{myverbbox}
\acceptanceCriteria{Пользователь с данным ID не существет в требуемой \texttt{personGroupId}.
}

%-----------------------------------------------------
%US-008
\newuserstory{Запуск обучения сервиса индентификации }


Администратор сервиса может запустить обучение нейронной сети сервиса \textit{Microsoft Face API} для возможности дальнейшего распознавания человека по лицу.

Обучение должно запускаться только если до этого происходило добавление или удаление нового человека.


\begin{myverbbox}[\small]{\cmdLine}
$ face-management.py --train
\end{myverbbox}
\scriptExample{


}

% AC-008-01
\begin{myverbbox}[\small]{\output}
$ cat faceapi.json | python -mjson.tool | grep groupId
    "groupId": "fintech-01",
$ curl -X GET "https://<datacenter url>/face/v1.0/persongroups/fintech-01
" -H "Content-Type: application/json" -H "Ocp-Apim-Subscription-Key: 0000
00000000000000000000000000000" 
{"error":{"code":"PersonGroupNotFound","message":"Person group is not fou
nd.\r\nParameter name: personGroupId"}}
$ face-management.py --train
There is nothing to train
\end{myverbbox}
\acceptanceCriteria{Требуемый \texttt{personGroupId} не существовует в сервисе \textit{Microsoft Face API} ни до, ни после выполнения команды. Тренировка сервиса не запускается.
}

% AC-008-02
\begin{myverbbox}[\small]{\output}
$ cat faceapi.json | python -mjson.tool | grep groupId
    "groupId": "fintech-01",
$ curl -X GET "https://<datacenter url>/face/v1.0/persongroups" -H "Conte
nt-Type: application/json" -H "OcApim-Subscription-Key: 00000000000000000
0000000000000000" 
[{"personGroupId":"fintech-01","name":"fintech-01","userData":null}]
$ face-management.py --train
There is nothing to train
\end{myverbbox}
\acceptanceCriteria{Требуемый \texttt{personGroupId} существует в \textit{Microsoft Face API}, но в группе нет ни одного добавленного пользователя. Тренировка сервиса не запускается.
}

% AC-008-03
\begin{myverbbox}[\small]{\output}
$ cat faceapi.json | python -mjson.tool | grep groupId
    "groupId": "fintech-01",
$ curl -X GET "https://<datacenter url>/face/v1.0/persongroups" -H "Conte
nt-Type: application/json" -H "OcApim-Subscription-Key: 00000000000000000
0000000000000000" 
[{"personGroupId":"fintech-01","name":"fintech-01","userData":null}]
$ face-management.py --simple-add /path/to/video1.avi
5 frames extracted
PersonId: 37da04e7-f471-49c7-a54c-a08f05950fc5
FaceIds
=======
1d499868-3d01-487c-8bab-626dc562e4e8
27dadf08-bc60-4a29-82a7-7d21ea7f40af
b8cf9c2f-a606-4f21-851d-26e0a0dc8a74
bf4806de-8c4b-4a12-8495-002f43dba797
ff79486f-15ac-43be-9c6c-b2840f8c8d22
$ face-management.py --train
Training successfully started
\end{myverbbox}
\acceptanceCriteria{Требуемый \texttt{personGroupId} существует в \textit{Microsoft Face API}. Поскольку запуску команды тренировки сервиса предшествует команда добавления пользователя, то обучение запускается.
}

% AC-008-04
\begin{myverbbox}[\small]{\output}
$ face-management.py --add /path/to/video1.avi /path/to/video2.avi /path/
to/video3.avi /path/to/video4.avi /path/to/video6.avi
18 frames extracted
PersonId: 5d5d6e92-f8f0-47cf-8e9b-b4455092603e
FaceIds
=======
0e5edd8e-7756-470e-a26c-f54ab70a5524
73b4caea-c85c-4594-9153-198351937d94
2ae3c850-190a-4cd0-afd2-8efa341767ba
adb7744e-b2fb-4a58-b3a8-f22c3143a8cd
7e0f87d6-14ad-4ec6-971f-e4c2771cc267
47c8c307-1f90-473c-b48c-792d5b5a1241
33e35a22-358a-4847-bcdd-4a5bdfd5eb69
2c7b951b-126a-4eec-9e22-1d9d65ece9e3
bcaf7f6a-1c44-476e-a51f-9dde2744ade0
1c9c0134-ada0-47c2-848c-d4424b232f04
b79945e1-de47-4011-98f2-e7e8f0d9f3ef
f972b6ac-078b-4e3d-a779-4cde8ea28f36
fbeb7aa1-f66e-4412-ad07-b61103e8af0b
9c762e64-075b-4fa4-8e39-0d83df86428b
9794e349-e4e4-44ad-951c-610b4890b16e
2e93da04-8f31-4a27-bd9e-5768335447a5
b9227512-dfa8-4a17-94e2-be520b8975a9
b67c1f86-2e3d-45a2-a7fb-57bbbff9cd8c
$ face-management.py --train
Training successfully started
\end{myverbbox}
\acceptanceCriteria{Поскольку запуску команды тренировки сервиса предшествует команда добавления пользователя, то обучение запускается.
}

% AC-008-05
\begin{myverbbox}[\small]{\output}
$ face-management.py --del 27dadf08-bc60-4a29-82a7-7d21ea7f40af
Person deleted
$ face-management.py --train
Training successfully started
\end{myverbbox}
\acceptanceCriteria{Поскольку запуску команды тренировки сервиса предшествует команда удаления пользователя, то обучение запускается.
}

% AC-008-06
\begin{myverbbox}[\small]{\output}
$ face-management.py --train
Training successfully started
$ face-management.py --del bdaf190a-4805-4e2c-95af-1afa8b2623df
The person does not exist
$ face-management.py --train
Already trained
\end{myverbbox}
\acceptanceCriteria{Поскольку после предыдущего запуска команды тренировки сервиса изменений в списке пользователей не происходило, то обучение не запускается.
}

%-----------------------------------------------------
%US-009
\newuserstory{Обнаружение уже добавленного пользователя }


Администратор сервиса не сможет добавить пользователя в сервис \textit{Microsoft Face API}, если изображения лица данного пользователя уже были добавлены в систему, и эти изображения были использованы для обучения сервиса.


% AC-009-01
\begin{myverbbox}[\small]{\output}
$ face-management.py --simple-add /path/to/video1.avi
5 frames extracted
PersonId: 37da04e7-f471-49c7-a54c-a08f05950fc5
FaceIds
=======
1d499868-3d01-487c-8bab-626dc562e4e8
27dadf08-bc60-4a29-82a7-7d21ea7f40af
b8cf9c2f-a606-4f21-851d-26e0a0dc8a74
bf4806de-8c4b-4a12-8495-002f43dba797
ff79486f-15ac-43be-9c6c-b2840f8c8d22
$ face-management.py --train
Training successfully started
$ face-management.py --simple-add /path/to/video21.avi
The same person already exists.
\end{myverbbox}
\acceptanceCriteria{Первое и второе видео содержат кадры лица одного и того же человека. Идентификация человека на втором видео происходит успешно, поскольку пять разных кадров из видео указывают на одного и того же человека с высокой степенью (не менее 50\%) уверенности определения. Добавление пользователя в систему не происходит.
}

% AC-009-02
\begin{myverbbox}[\small]{\output}
$ face-management.py --add /path/to/video1.avi /path/to/video2.avi /path/
to/video3.avi /path/to/video4.avi /path/to/video6.avi
18 frames extracted
PersonId: 5d5d6e92-f8f0-47cf-8e9b-b4455092603e
FaceIds
=======
0e5edd8e-7756-470e-a26c-f54ab70a5524
73b4caea-c85c-4594-9153-198351937d94
2ae3c850-190a-4cd0-afd2-8efa341767ba
adb7744e-b2fb-4a58-b3a8-f22c3143a8cd
7e0f87d6-14ad-4ec6-971f-e4c2771cc267
47c8c307-1f90-473c-b48c-792d5b5a1241
33e35a22-358a-4847-bcdd-4a5bdfd5eb69
2c7b951b-126a-4eec-9e22-1d9d65ece9e3
bcaf7f6a-1c44-476e-a51f-9dde2744ade0
1c9c0134-ada0-47c2-848c-d4424b232f04
b79945e1-de47-4011-98f2-e7e8f0d9f3ef
f972b6ac-078b-4e3d-a779-4cde8ea28f36
fbeb7aa1-f66e-4412-ad07-b61103e8af0b
9c762e64-075b-4fa4-8e39-0d83df86428b
9794e349-e4e4-44ad-951c-610b4890b16e
2e93da04-8f31-4a27-bd9e-5768335447a5
b9227512-dfa8-4a17-94e2-be520b8975a9
b67c1f86-2e3d-45a2-a7fb-57bbbff9cd8c
$ face-management.py --train
Training successfully started
$ face-management.py --add /path/to/video1.avi /path/to/video2.avi /path/
to/video3.avi /path/to/video4.avi /path/to/video6.avi
The same person already exists.
\end{myverbbox}
\acceptanceCriteria{Поскольку для добавления пользователя использовалось одно и то же видео, то повторное добавление пользователя в систему не происходит.
}

% AC-009-03
\begin{myverbbox}[\small]{\output}
$ face-management.py --simple-add /path/to/video1.avi
5 frames extracted
PersonId: 37da04e7-f471-49c7-a54c-a08f05950fc5
FaceIds
=======
1d499868-3d01-487c-8bab-626dc562e4e8
27dadf08-bc60-4a29-82a7-7d21ea7f40af
b8cf9c2f-a606-4f21-851d-26e0a0dc8a74
bf4806de-8c4b-4a12-8495-002f43dba797
ff79486f-15ac-43be-9c6c-b2840f8c8d22
$ face-management.py --train
Training successfully started
$ face-management.py --simple-add /path/to/video22.avi
5 frames extracted
PersonId: 31e1fc90-84ff-498a-833c-1730aa00a310
FaceIds
=======
21a12afe-6aab-4593-ac77-d20d2bea7e8b
e2f4e1d8-ba2d-4c13-80b7-791f10949143
8c366814-3e8a-441e-86f2-c9edf967c04e
8510de4b-9a70-4d62-823d-471107f838da
c159bd2e-2f71-4c0d-8c85-179d76d96953
\end{myverbbox}
\acceptanceCriteria{Первое и второе видео содержат кадры лица разных людей. Человек из второго видео не был добавлен до этого в сервис \textit{Microsoft Face API}. Кадры лица человека из второго видео добавляются в сервис.
}

% AC-009-04
\begin{myverbbox}[\small]{\output}
$ face-management.py --simple-add /path/to/video1.avi
5 frames extracted
PersonId: 52865cde-3af8-443d-b260-9319c2cb1788
FaceIds
=======
cdb6227e-7453-4057-b4fa-79660914e597
6976d3c2-dee5-4f24-8950-f38ff10c70ad
fae15e55-6639-42a4-a954-731c33310e41
15092567-5765-49ed-ac63-94bc5fa08d17
a77f1f0a-aa95-4bd1-9826-6b453aec42b2
$ face-management.py --simple-add /path/to/video31.avi
5 frames extracted
PersonId: 9fa0a99b-8e76-474d-8223-dea217c2c19b
FaceIds
=======
b552ef11-a162-4a7d-9047-ccfc84a07043
90c0815a-ecce-45c6-8107-ced7ef29a249
fde35dba-505d-4a62-ac5a-c6ae4c89128e
6c6910b4-0ab5-4eb4-9e53-95b1929f9867
fdb9d352-65b0-41a2-a1be-03ea5b543160
$ face-management.py --simple-add /path/to/video41.avi
5 frames extracted
PersonId: f290ecb9-bfab-46f7-b623-45140d730628
FaceIds
=======
e5735ecd-ca09-4fd4-bfd3-8ace67702ab0
9e1bbdee-5981-4f6b-aba5-03be57e5e910
3120ef58-8d53-4558-8b84-784ba338f621
8fceb9c7-f029-4326-9703-6749005674fa
8ea02a3b-7dc0-455a-858c-67251b0ca3b4
$ face-management.py --train
Training successfully started
$ face-management.py --simple-add /path/to/video22.avi
5 frames extracted
PersonId: 31e1fc90-84ff-498a-833c-1730aa00a310
FaceIds
=======
21a12afe-6aab-4593-ac77-d20d2bea7e8b
e2f4e1d8-ba2d-4c13-80b7-791f10949143
8c366814-3e8a-441e-86f2-c9edf967c04e
8510de4b-9a70-4d62-823d-471107f838da
c159bd2e-2f71-4c0d-8c85-179d76d96953
\end{myverbbox}
\acceptanceCriteria{Даже если обучение сервиса не происходило возможно добавлять несколько разных пользователей в сервис.
}


\subsection*{Взаимодействие с пользователем}


%-----------------------------------------------------
%US-010
\newuserstory{Идентификация пользователя}


Для идентификации пользователя, компонента использует кадры с лицом из видео-потока и отправляет их в сервис \textit{Microsoft Face API}. 


\begin{myverbbox}[\small]{\cmdLine}
$ faceid.py --find <path to video file>
\end{myverbbox}
\scriptExample{
В случае, когда в текущей директории нет файла \texttt{actions.json}, то запускается простой (небезопасный) способ аутентификации. При этом из видео-потока извлекается 5 кадров с изображением лица человека. Подразумевается, что все кадры в видео приндалежат одному и тому же человеку. Идентификация человека происходит успешно, если кадры из видео указывают на одного и того же человека с высокой степенью (не менее 50\%) уверенности определения.

Если в видео-потоке недостаточно кадров с изображением человека или на изображении невозможно определить лицо, то обработка такого видео должно приводить к ошибке. 

После успешной идентификации в текущей директории создается (если файл уже существовал, то он пересоздается) файл \texttt{person.json}, в котором указывается идентификатор, возвращенный сервисом \textit{Microsoft Face API}. Пример файла:
\texttt{\string{"id": "37da04e7-f471-49c7-a54c-a08f05950fc5"\string}}

}

% AC-010-01
\begin{myverbbox}[\small]{\output}
$ face-management.py --simple-add /path/to/video1.avi
5 frames extracted
PersonId: 37da04e7-f471-49c7-a54c-a08f05950fc5
FaceIds
=======
1d499868-3d01-487c-8bab-626dc562e4e8
27dadf08-bc60-4a29-82a7-7d21ea7f40af
b8cf9c2f-a606-4f21-851d-26e0a0dc8a74
bf4806de-8c4b-4a12-8495-002f43dba797
ff79486f-15ac-43be-9c6c-b2840f8c8d22
$ face-management.py --train
Training successfully started
$ cat person.json
cat: person.json: No such file or directory
$ faceid.py --find /path/to/video21.avi
37da04e7-f471-49c7-a54c-a08f05950fc5 identified
$ cat person.json
{"id": "37da04e7-f471-49c7-a54c-a08f05950fc5"}
\end{myverbbox}
\acceptanceCriteria{Первое и второе видео содержат кадры лица одного и того же человека. Идентификация человека на втором видео происходит успешно, поскольку пять разных кадров из видео указывают на одного и того же человека с высокой степенью (не менее 50\%) уверенности определения. В текущей директории создан файл \texttt{person.json}
}

% AC-010-02
\begin{myverbbox}[\small]{\output}
$ face-management.py --simple-add /path/to/video1.avi
5 frames extracted
PersonId: 37da04e7-f471-49c7-a54c-a08f05950fc5
FaceIds
=======
1d499868-3d01-487c-8bab-626dc562e4e8
27dadf08-bc60-4a29-82a7-7d21ea7f40af
b8cf9c2f-a606-4f21-851d-26e0a0dc8a74
bf4806de-8c4b-4a12-8495-002f43dba797
ff79486f-15ac-43be-9c6c-b2840f8c8d22
$ face-management.py --train
Training successfully started
$ face-management.py --simple-add /path/to/video2.avi
5 frames extracted
PersonId: d9e01953-7bc1-4079-96af-3f3a26cf4b1d
FaceIds
=======
3c8195bb-49e7-4b45-b7ae-d44e60310599
d01523ce-cd83-4605-8298-afd4bb8d9e81
c0facdb1-704f-44f2-b76f-4e7298c476be
dcfa8e7b-f8d4-4567-80c5-55f3c5e13d85
a6872fc7-e45b-467f-b9a7-e18935d057da
$ cat person.json
{"id": "33fc4c4a-911a-4ab0-9535-f588d47c3a60"}
$ faceid.py --find /path/to/video21.avi
The service is not ready
$ cat person.json
cat: person.json: No such file or directory
\end{myverbbox}
\acceptanceCriteria{Первое и третье видео содержат кадры лица одного и того же человека. Но поскольку после добавления нового пользователя не происходило повторное обучение сервиса, идентификация человека невозможна. Существовавший в текущей директории файл \texttt{person.json} удаляется.
}

% AC-010-03
\begin{myverbbox}[\small]{\output}
$ cat faceapi.json | python -mjson.tool | grep groupId
    "groupId": "fintech-01",
$ curl -X GET "https://<datacenter url>/face/v1.0/persongroups/fintech-01
" -H "Content-Type: application/json" -H "Ocp-Apim-Subscription-Key: 0000
00000000000000000000000000000" 
{"error":{"code":"PersonGroupNotFound","message":"Person group is not fou
nd.\r\nParameter name: personGroupId"}}
$ faceid.py --find /path/to/video21.avi
The service is not ready
$ cat person.json
cat: person.json: No such file or directory
\end{myverbbox}
\acceptanceCriteria{Поскольку не существует группы на момент запуска команды идентификации пользователя, то идентификация человека невозможна. Файл \texttt{person.json} не создается.
}

% AC-010-04
\begin{myverbbox}[\small]{\output}
$ face-management.py --add /path/to/video1.avi /path/to/video2.avi /path/
to/video3.avi /path/to/video4.avi /path/to/video5.avi
18 frames extracted
PersonId: 5d5d6e92-f8f0-47cf-8e9b-b4455092603e
FaceIds
=======
0e5edd8e-7756-470e-a26c-f54ab70a5524
73b4caea-c85c-4594-9153-198351937d94
2ae3c850-190a-4cd0-afd2-8efa341767ba
adb7744e-b2fb-4a58-b3a8-f22c3143a8cd
7e0f87d6-14ad-4ec6-971f-e4c2771cc267
47c8c307-1f90-473c-b48c-792d5b5a1241
33e35a22-358a-4847-bcdd-4a5bdfd5eb69
2c7b951b-126a-4eec-9e22-1d9d65ece9e3
bcaf7f6a-1c44-476e-a51f-9dde2744ade0
1c9c0134-ada0-47c2-848c-d4424b232f04
b79945e1-de47-4011-98f2-e7e8f0d9f3ef
f972b6ac-078b-4e3d-a779-4cde8ea28f36
fbeb7aa1-f66e-4412-ad07-b61103e8af0b
9c762e64-075b-4fa4-8e39-0d83df86428b
9794e349-e4e4-44ad-951c-610b4890b16e
2e93da04-8f31-4a27-bd9e-5768335447a5
b9227512-dfa8-4a17-94e2-be520b8975a9
b67c1f86-2e3d-45a2-a7fb-57bbbff9cd8c
$ face-management.py --train
Training successfully started
$ faceid.py --find /path/to/video22.avi
The person was not found
$ cat person.json
cat: person.json: No such file or directory
\end{myverbbox}
\acceptanceCriteria{Видео, использовавшееся при добавлении человека, и видео, использующееся при распознавании, содержат кадры лица разных людей. Человек, чье лицо изображено на видео, использующееся при распозновании, не был добавлен до этого в сервис \textit{Microsoft Face API}. Кадры лица человека из второго видео не добавляются в сервис \textit{Microsoft Face API}. Файл \texttt{person.json} не создается.
}

% AC-010-05
\begin{myverbbox}[\small]{\output}
$ cat faceapi.json | python -mjson.tool | grep groupId
    "groupId": "fintech-01",
$ curl -X GET "https://<datacenter url>/face/v1.0/persongroups/fintech-01
" -H "Content-Type: application/json" -H "Ocp-Apim-Subscription-Key: 0000
00000000000000000000000000000" 
{"error":{"code":"PersonGroupNotFound","message":"Person group is not fou
nd.\r\nParameter name: personGroupId"}}
$ faceid.py --find /path/to/video100.avi
The video does not follow requirements
$ cat person.json
cat: person.json: No such file or directory
$ face-management.py --simple-add /path/to/video1.avi
5 frames extracted
PersonId: 37da04e7-f471-49c7-a54c-a08f05950fc5
FaceIds
=======
1d499868-3d01-487c-8bab-626dc562e4e8
27dadf08-bc60-4a29-82a7-7d21ea7f40af
b8cf9c2f-a606-4f21-851d-26e0a0dc8a74
bf4806de-8c4b-4a12-8495-002f43dba797
ff79486f-15ac-43be-9c6c-b2840f8c8d22
$ face-management.py --train
Training successfully started
$ faceid.py --find /path/to/video100.avi
The video does not follow requirements
$ cat person.json
cat: person.json: No such file or directory
\end{myverbbox}
\acceptanceCriteria{Видео, использующееся при распознавании, либо содержит меньше 5 кадров, либо не содержит кадров с лицом пользователя. Файл \texttt{person.json} не создается.
}

% AC-010-06
\begin{myverbbox}[\small]{\output}
$ faceid.py --find /path/to/video101.avi
The person was not found
$ cat person.json
cat: person.json: No such file or directory
\end{myverbbox}
\acceptanceCriteria{На видео, использующееся при распознавании, как минимум один из пяти разных кадров указывает на человека, отличающегося от человека в других кадрах. Степень уверенности определения не менее 50\%. Файл \texttt{person.json} не создается.
}

%-----------------------------------------------------
%US-011
\newuserstory{Запроса действий на безопасную идентификацию пользователя}


Для обеспечения безопасной идентфикации сервис может сформировать запрос к пользователю на выполнение определенных действий. Совершив данные действия пользователь продемонстрирует, что является живым человеком. Таким образом достигается дополнительная безопасность.


\begin{myverbbox}[\small]{\cmdLine}
$ faceid.py --actions
\end{myverbbox}
\scriptExample{
Команда генерирует JSON \texttt{actions.json}, содержащий описание набора действий, которые должен совершить пользователь.

Действия выбираются из следующего списка:
\begin{itemize}
  \item YawRight - поворот головы направо на 15 градусов
  \item YawLeft - поворот головы налево на 15 градусов
  \item RollRight - наклон головы вправо на 20 градусов
  \item RollLeft - наклон головы влево на 20 градусов
  \item CloseRightEye - прищурить правый глаз на 2 секунды
  \item CloseLeftEye - прищурить левый глаз на 2 секунды
  \item OpenMouth - открыть широко рот
\end{itemize}

В сформированных действиях не должно быть 2 поворота и/или наклона. Всего действий должно быть 3 или 4.

}

% AC-011-01
\begin{myverbbox}[\small]{\output}
$ faceid.py --actions
$ cat actions.json
{"actions": ["OpenMouth", "YawLeft", "CloseLeftEye"]}
\end{myverbbox}
\acceptanceCriteria{Сформирован файл \texttt{actions.json} c тремя действиями. В сформированных действиях - один запрос на открытие рта, один запрос на закрытие глаза, один запрос на изменение положения головы.
}

% AC-011-02
\begin{myverbbox}[\small]{\output}
$ faceid.py --actions
$ cat actions.json
{"actions": ["CloseRightEye", "RollRight", "OpenMouth", "CloseLeftEye"]}
\end{myverbbox}
\acceptanceCriteria{Сформирован файл \texttt{actions.json} c тремя действиями. В сформированных действиях - один запрос на открытие рта, два запроса (разные глаза) на закрытие глаза, один запрос на изменение положения головы.
}

% AC-011-03
\begin{myverbbox}[\small]{\output}
$ faceid.py --actions
$ cat actions.json
{"actions": ["CloseLeftEye", "CloseRightEye", "RollRight"]}
\end{myverbbox}
\acceptanceCriteria{Сформирован файл \texttt{actions.json} c тремя действиями. В сформированных действиях - два запроса на закрытие глаз (разные глаза), один запрос на изменение положения головы.
}

%-----------------------------------------------------
%US-012
\newuserstory{Безопасная идентификация пользователя}


Должен быть action JSON, сохраняет PersonId в person JSON.


\begin{myverbbox}[\small]{\cmdLine}
$ faceid.py --find /path/to/video.avi
\end{myverbbox}
\scriptExample{


}

%-----------------------------------------------------
%US-013
\newuserstory{Получение баланса идентифицированного пользователя}


После идентификации пользователь может получить баланс на соответствующем данному пользователю аккаунте в сети блокчейн.


\begin{myverbbox}[\small]{\cmdLine}
$ faceid.py <PIN code>
\end{myverbbox}
\scriptExample{
Используя идентификатор, содержащийся в \texttt{person.json}, и PIN-код скрипт генерирует приватный ключ пользователя. В блокчейн сеть отправляется запрос на баланс аккаунта, полученного из приватного ключа.

Приватный ключ генерируется по правилу:

$$K = keccak256(keccak256(keccak256(keccak256(keccak256(''), I, P_1), I, P_2), I,$$
$$P_3), I, P_4)$$

где $''$ - 'пустая' последовательность байт, $I$ - это идентификатор, который возвращает система распознавания по лицу, приведенный к длине в 16 байт, а ($P_1$, $P_2$, $P_3$, $P_4$) - четыре цифры PIN-кода, где каждая цифра представлена целым числом длиной 1 байт, $P_1$ - цифра самого старшего разряда в PIN-коде (первая цифра), а $P_4$ - цифра самого младшего разряда в PIN-коде (последняя цифра).

Например, идентификатору \texttt{37da04e7-f471-49c7-a54c-a08f05950fc5} при применении PIN-кода \texttt{1234} соответствует приватный ключ \texttt{6be7217f318a6409ba8e87e42ce600e14153647a816f7f2d43d244d1a00ed3df}.

При выводе баланса должно быть автомтическое масштабирование суммы - нормализация к одному из возможных значений: \textit{poa}, \textit{finney}, \textit{szabo}, \textit{gwei}, \textit{mwei}, \textit{kwei}, \textit{wei}. Происходит следующим образом:
\begin{itemize}
  \item Используется минимально возможное нормализованное значение, чья целая часть больше ноля;
  \item Дробная часть записывается с округлением до $10^{-6}$;
  \item Завершающие нули в дробной части, полученные полсе округления, не записываются (например, 1.3, но не 1.300000).
\end{itemize}

}

% AC-013-01
\begin{myverbbox}[\small]{\output}
$ cat person.json
{"id": "37da04e7-f471-49c7-a54c-a08f05950fc5"}
$ faceid.py --balance 4590
Your balance is 2.5 poa
\end{myverbbox}
\acceptanceCriteria{Баланс пользователя получен от узла блокчейн сети, доступ к которому получен через JSON RPC.
}

% AC-013-02
\begin{myverbbox}[\small]{\output}
$ cat person.json
{"id": "a9d0f1d5-1359-43ca-bcc3-cc7c1e314b86"}
$ faceid.py --balance 1864
Your balance is 84.000138 szabo
\end{myverbbox}
\acceptanceCriteria{Баланс пользователя получен от узла блокчейн сети, доступ к которому получен через JSON RPC.
}

% AC-013-03
\begin{myverbbox}[\small]{\output}
$ cat person.json
{"id": "37da04e7-f471-49c7-a54c-a08f05950fc5"}
$ faceid.py --balance 6879
Your balance is 0 poa
\end{myverbbox}
\acceptanceCriteria{Неправильно задан PIN-код, поэтому баланс запрашивается у несуществующего аккаунта.
}

% AC-013-04
\begin{myverbbox}[\small]{\output}
$ cat person.json
cat: person.json: No such file or directory
$ faceid.py --balance 6879
ID is not found
\end{myverbbox}
\acceptanceCriteria{Выдается ошибка, если в текущей директории не существует файл \texttt{person.json}.
}

%-----------------------------------------------------
%US-014
\newuserstory{Отправка запроса на регистрацию соответствия}


После идентификации пользователь может отправить запрос на регистрацию соответствия между телефоном и своим аккаунтом.


\begin{myverbbox}[\small]{\cmdLine}
$ faceid.py --add <pin code> <phone number>
\end{myverbbox}
\scriptExample{
Используя идентификатор, содержащийся в \texttt{person.json}, и PIN-код скрипт генерирует приватный ключ пользователя. В блокчейн сеть отправляется транзакция к контракту регистра соответствий с запросом регистрации. В терминал выводится хэш транзакции, в рамках которой в контракт добавлен запрос регистрации.

}

% AC-014-01
\begin{myverbbox}[\small]{\output}
$ cat network.json
{"rpcUrl": "https://sokol.poa.network", "gasPriceUrl": "https://gasprice.
poa.network/", "defaultGasPrice": 2000000000}
$ cat person.json
{"id": "37da04e7-f471-49c7-a54c-a08f05950fc5"}
$ faceid.py --add 4590 +79991234567
Registration request sent by 0xa7f3239715ff731a3d6fc477b18e35b9b0a9e1ede8
4bca5e91517e8e5bf1cc69
\end{myverbbox}
\acceptanceCriteria{В блокчейн сеть отправляется транзакция с запросом регистрации соответствия аккаунта, ассоциированного с аккаунтом, указанным в файле \texttt{person.json} и номером телефона, указанным в параметре командной строки. При обработке запроса, контракт производить событие (\texttt{event}) \texttt{RegistrationRequest}, в с указанием аккаунта, отправившего запрос:

\texttt{event RegistrationRequest(address indexed sender);}


Для проведения транзакции выбрана цена из значения \texttt{fast}, возвращенного сервисом \texttt{https://gasprice.poa.network}.
}

% AC-014-02
\begin{myverbbox}[\small]{\output}
$ cat person.json
{"id": "37da04e7-f471-49c7-a54c-a08f05950fc5"}
$ faceid.py --add 4590 +79991234567
Registration request sent by 0xa7f3239715ff731a3d6fc477b18e35b9b0a9e1ede8
4bca5e91517e8e5bf1cc69
$ faceid.py --add 4590 +79991234567
Registration request already sent
$ faceid.py --add 4590 +79991239812
Registration request already sent
\end{myverbbox}
\acceptanceCriteria{Повторный запрос на регистрацию соответствия аккаунта не отправляется в блокчейн сеть, поскольку данный аккаунт уже посылал запрос на регистрацию соответствий. Первичный запрос еще не был обработан. В терминал выводится сообщение об ошибке. Транзакция в блокчейн сеть не отправляется.
}

% AC-014-03
\begin{myverbbox}[\small]{\output}
$ cat person.json
cat: person.json: No such file or directory
$ faceid.py --add 4590 +79991234567
ID is not found
\end{myverbbox}
\acceptanceCriteria{Выдается ошибка, если в текущей директории не существует файл \texttt{person.json}. Транзакция в блокчейн сеть не отправляется.
}

% AC-014-04
\begin{myverbbox}[\small]{\output}
$ cat person.json
{"id": "37da04e7-f471-49c7-a54c-a08f05950fc5"}
$ faceid.py --add 1234 +79991234567
No funds to send the request
\end{myverbbox}
\acceptanceCriteria{Поскольку ситуация, когда неправильный приватный ключ сформирован из-за некорретного PIN-кода, неотличима от ситуации, когда на аккаунте нет средств для оплаты комиссии на обработку транзакции, в терминал выводится сообщение об ошибке. Транзакция в блокчейн сеть не отправляется.
}

% AC-014-05
\begin{myverbbox}[\small]{\output}
$ cat registrar.json
cat: registrar.json: No such file or directory
$ faceid.py --add 4590 +79122229016
No contract address
\end{myverbbox}
\acceptanceCriteria{Выдается сообщение об ошибке, если в текущей директории нет файла \texttt{registrar.json}, содержащего адрес контракта регистра соответствий. Транзакция в блокчейн сеть не отправляется.
}

% AC-014-06
\begin{myverbbox}[\small]{\output}
$ cat registrar.json
{"registrar": {"address": "0x340Ec06460d9b2b7D23B40E5bCd0a81A09e06D81", "
startBlock": 456123}, "payments": {"address": "0x81A09e06D81797AE2b7D23B4
0E5bCd0a1da01eb0F951x", "startBlock": 456125}}
$ faceid.py --add 4590 +79122229016
Seems that the contract address is not the registrar contract
\end{myverbbox}
\acceptanceCriteria{Выдается сообщение об ошибке, если в адрес контракта, указанного в файле \texttt{registrar.json}, не принадлежит контракту регистра соответствий.
}

% AC-014-07
\begin{myverbbox}[\small]{\output}
$ faceid.py --add 4590 +79122
Incorrect phone number
$ faceid.py --add 4590 adad
Incorrect phone number
$ faceid.py --add 4590 +73431231543543534653
Incorrect phone number
$ faceid.py --add 4590
Incorrect phone number
\end{myverbbox}
\acceptanceCriteria{Выдается сообщение об ошибке, если номер телефона указан некорректно. Корректный номер содержит 11 цифр и начинается со знака +.
}

% AC-014-08
\begin{myverbbox}[\small]{\output}
$ cat person.json
{"id": "37da04e7-f471-49c7-a54c-a08f05950fc5"}
$ faceid.py --add 4590 +79991234567
Registration request sent by 0xa7f3239715ff731a3d6fc477b18e35b9b0a9e1ede8
4bca5e91517e8e5bf1cc69
$ faceid.py --find /path/to/video23.avi
168f2e61-089f-88a9-53b2-3b3d0c497704 identified
$ faceid.py --add 5981 +79037518950
Registration request sent by 0x6a8a46c4e005b9e8bea97aa58c5839787b5689e9e8
785ea9e38330395582f78f
\end{myverbbox}
\acceptanceCriteria{В блокчейн сеть отправляется транзакция с запросом регистрации соответствия аккаунта даже если в контракте регистра уже учтены запросы на регистрацию соответствий других аккаунтов. При обработке запроса, контракт производит событие (\texttt{event}) \texttt{RegistrationRequest}.
}

% AC-014-09
\begin{myverbbox}[\small]{\output}
$ cat network.json | python -mjson.tool | grep gasPriceUrl 
    "gasPriceUrl": "https://gasprice.poa.network/",
$ curl https://gasprice.poa.network/
curl: (6) Could not resolve host: gasprice.poa.network
$ faceid.py --add 5981 +79037518950
Registration request sent by 0x6a8a46c4e005b9e8bea97aa58c5839787b5689e9e8
785ea9e38330395582f78f
\end{myverbbox}
\acceptanceCriteria{В блокчейн сеть отправляется транзакция c запросом регистрации соответствия аккаунта. Транзакция успешно верифицирована и включена в блок. Для проведения транзакции выбрана цена из значения \texttt{defaultGasPrice} из файла \texttt{network.json}.
}

% AC-014-10
\begin{myverbbox}[\small]{\output}
$ cat person.json
{"id": "37da04e7-f471-49c7-a54c-a08f05950fc5"}
$ faceid.py --add 4590 +79991234567
Registration request sent by 0x73aad4ff595a8813cc7d440d244545017c77098528
f010a7caaa7d74c382f6c5
\end{myverbbox}
\acceptanceCriteria{Если из транзакции c идентификатором \texttt{0x73aad4ff595a8813cc7d440d244545017c77098528f010a7caaa7d74c382f6c5} извлечь поле \texttt{input} и отправить его в новой транзакции с того же аккаунта снова в поле \texttt{input} на адрес контракта регистра соответствий, то эта транзакция будет включена в блок, но статус ее исполнения будет - ошибка, поскольку такой запрос регистрации соответствия уже был послан в контракт. Статус можно подтвердить для данной транзакции в браузере блоков.
}

% AC-014-11
\begin{myverbbox}[\small]{\output}
$ cat person.json
{"id": "37da04e7-f471-49c7-a54c-a08f05950fc5"}
$ faceid.py --add 4590 +79991234567
Registration request sent by 0xa7f3239715ff731a3d6fc477b18e35b9b0a9e1ede8
4bca5e91517e8e5bf1cc69
\end{myverbbox}
\acceptanceCriteria{Если из транзакции с идентификатором \texttt{0xa7f3239715ff731a3d6fc477b18e35b9b0a9e1ede84bca5e91517e8e5bf1cc69} извлечь поле \texttt{input}, изменить в нем те байты, которые кодируют номер телефона так, чтобы передаваемый номер телефона содержал количество цифр отличное от 11, либо содержал буквы, и отправить получившийся набор байт в новой транзакции с того же аккаунта снова в поле \texttt{input} на адрес контракта регистра соответствий, то эта транзакция будет включена в блок, но статус ее исполнения будет - ошибка, поскольку такой номер телефона некорректный. Статус можно подтвердить для данной транзакции в браузере блоков.
}

% AC-014-12
\begin{myverbbox}[\small]{\output}
$ cat person.json
{"id": "37da04e7-f471-49c7-a54c-a08f05950fc5"}
$ faceid.py --add 4590 +79991234567
Registration request sent by 0x73aad4ff595a8813cc7d440d244545017c77098528
f010a7caaa7d74c382f6c5
\end{myverbbox}
\acceptanceCriteria{Если из транзакции c идентификатором \texttt{0x73aad4ff595a8813cc7d440d244545017c77098528f010a7caaa7d74c382f6c5} извлечь поле \texttt{input} и отправить его в новой транзакции с аккаунта, отличающегося от \texttt{from} в упомянутой выше транзакции, на адрес контракта регистра соответствий, то эта транзакция будет включена в блок, но статус ее исполнения будет - успешно, поскольку такой изначальный запрос регистрации соответствия еще не был подтвержден. Статус транзакции можно подтвердить для данной транзакции в браузере блоков.
}

%-----------------------------------------------------
%US-015
\newuserstory{Отправка запроса на удаление соответствия}


После идентификации пользователь может отправить запрос на удаление соответствия между телефоном и своим аккаунтом.


\begin{myverbbox}[\small]{\cmdLine}
$ faceid.py --del <pin code>
\end{myverbbox}
\scriptExample{
Используя идентификатор, содержащийся в \texttt{person.json}, и PIN-код скрипт генерирует приватный ключ пользователя. В блокчейн сеть отправляется транзакция к контракту регистра соответствий с запросом удаления. В терминал выводится хэш транзакции, в рамках которой в контракт добавлен запрос удаления.

}

% AC-015-01
\begin{myverbbox}[\small]{\output}
$ cat network.json
{"rpcUrl": "https://sokol.poa.network", "gasPriceUrl": "https://gasprice.
poa.network/", "defaultGasPrice": 2000000000}
$ cat person.json
{"id": "37da04e7-f471-49c7-a54c-a08f05950fc5"}
$ faceid.py --del 4590
Unregistration request sent by 0x58aff6987a78c9109ff3b99b0e1cf333faba313b
c8002d4b333653be45d7e1d8
\end{myverbbox}
\acceptanceCriteria{В блокчейн сеть отправляется транзакция с запросом удаления ранее зарегистрированного соответствия аккаунта, ассоциированного с аккаунтом, указанным в файле \texttt{person.json} и номера телефона. При обработке запроса, контракт производить событие (\texttt{event}) \texttt{UnregistrationRequest}, в с указанием аккаунта, отправившего запрос:

\texttt{event UnregistrationRequest(address indexed sender);}


Для проведения транзакции выбрана цена из значения \texttt{fast}, возвращенного сервисом \texttt{https://gasprice.poa.network}.
}

% AC-015-02
\begin{myverbbox}[\small]{\output}
$ cat person.json
{"id": "37da04e7-f471-49c7-a54c-a08f05950fc5"}
$ faceid.py --del 4590
Unregistration request sent by 0x58aff6987a78c9109ff3b99b0e1cf333faba313b
c8002d4b333653be45d7e1d8
$ faceid.py --del 4590
Unregistration request already sent
\end{myverbbox}
\acceptanceCriteria{Повторный запрос на удаление соответствия аккаунта не отправляется в блокчейн сеть, поскольку данный аккаунт уже посылал запрос на удаление соответствий. Первичный запрос еще не был обработан. В терминал выводится сообщение об ошибке. Транзакция в блокчейн сеть не отправляется.
}

% AC-015-03
\begin{myverbbox}[\small]{\output}
$ cat person.json
{"id": "da04e377-47f1-c749-54ca-0fc5a08f0595"}
$ faceid.py --add 6104 +79220012534
Registration request sent by 0xc95d677eb6f3fcb55e08274ae0eed1970391e637f1
062426a3406fd7d4cfcfcb
$ faceid.py --del 6104
Account is not registered yet
\end{myverbbox}
\acceptanceCriteria{Запрос на удаление соответствия аккаунта не отправляется в блокчейн сеть, поскольку данный аккаунт не зарегистрирован в регистре соответствий. В терминал выводится сообщение об ошибке. Транзакция в блокчейн сеть не отправляется.
}

% AC-015-04
\begin{myverbbox}[\small]{\output}
$ cat person.json
cat: person.json: No such file or directory
$ faceid.py --del 4590
ID is not found
\end{myverbbox}
\acceptanceCriteria{Выдается ошибка, если в текущей директории не существует файл \texttt{person.json}. Транзакция в блокчейн сеть не отправляется.
}

% AC-015-05
\begin{myverbbox}[\small]{\output}
$ cat person.json
{"id": "37da04e7-f471-49c7-a54c-a08f05950fc5"}
$ faceid.py --del 1234
No funds to send the request
\end{myverbbox}
\acceptanceCriteria{Поскольку ситуация, когда неправильный приватный ключ сформирован из-за некорретного PIN-кода, неотличима от ситуации, когда на аккаунте нет средств для оплаты комиссии на обработку транзакции, в терминал выводится сообщение об ошибке. Транзакция в блокчейн сеть не отправляется.
}

% AC-015-06
\begin{myverbbox}[\small]{\output}
$ cat registrar.json
cat: registrar.json: No such file or directory
$ faceid.py --del 4590
No contract address
\end{myverbbox}
\acceptanceCriteria{Выдается сообщение об ошибке, если в текущей директории нет файла \texttt{registrar.json}, содержащего адрес контракта регистра соответствий. Транзакция в блокчейн сеть не отправляется.
}

% AC-015-07
\begin{myverbbox}[\small]{\output}
$ cat registrar.json
{"registrar": {"address": "0x340Ec06460d9b2b7D23B40E5bCd0a81A09e06D81", "
startBlock": 456123}, "payments": {"address": "0x81A09e06D81797AE2b7D23B4
0E5bCd0a1da01eb0F951x", "startBlock": 456125}}
$ faceid.py --del 4590
Seems that the contract address is not the registrar contract.
\end{myverbbox}
\acceptanceCriteria{Выдается сообщение об ошибке, если в адрес контракта, указанного в файле \texttt{registrar.json}, не принадлежит контракту регистра соответствий.
}

% AC-015-08
\begin{myverbbox}[\small]{\output}
$ cat person.json
{"id": "37da04e7-f471-49c7-a54c-a08f05950fc5"}
$ faceid.py --del 4590
Unregistration request sent by 0x58aff6987a78c9109ff3b99b0e1cf333faba313b
c8002d4b333653be45d7e1d8
$ faceid.py --find /path/to/video23.avi
168f2e61-089f-88a9-53b2-3b3d0c497704 identified
$ faceid.py --del 5981
Unregistration request sent by 0xda0e2a124e6d37080b539f1bb0dc4c698b8025b6
2dcc56d75efc1115d87381aa
\end{myverbbox}
\acceptanceCriteria{В блокчейн сеть отправляется транзакция с запросом удаления ранее зарегистрированного соответствия аккаунта даже если в контракте регистра уже учтены запросы на удаление соответствий других аккаунтов. При обработке запроса, контракт производить событие (\texttt{event}) \texttt{UnregistrationRequest}.
}

% AC-015-09
\begin{myverbbox}[\small]{\output}
$ cat network.json | python -mjson.tool | grep gasPriceUrl 
    "gasPriceUrl": "https://gasprice.poa.network/",
$ curl https://gasprice.poa.network/
curl: (6) Could not resolve host: gasprice.poa.network
$ faceid.py --del 5981
Unregistration request sent by 0xda0e2a124e6d37080b539f1bb0dc4c698b8025b6
2dcc56d75efc1115d87381aa
\end{myverbbox}
\acceptanceCriteria{В блокчейн сеть отправляется транзакция c запросом удаления соответствия аккаунта. Транзакция успешно верифицирована и включена в блок. Для проведения транзакции выбрана цена из значения \texttt{defaultGasPrice} из файла \texttt{network.json}.
}

% AC-015-10
\begin{myverbbox}[\small]{\output}
$ cat person.json
{"id": "37da04e7-f471-49c7-a54c-a08f05950fc5"}
$ faceid.py --del 4590
Unregistration request sent by 0x58aff6987a78c9109ff3b99b0e1cf333faba313b
c8002d4b333653be45d7e1d8
\end{myverbbox}
\acceptanceCriteria{Если из транзакции с идентификатором \texttt{0x58aff6987a78c9109ff3b99b0e1cf333faba313bc8002d4b333653be45d7e1d8} извлечь поле \texttt{input} и отправить его в новой транзакции с того же аккаунта снова в поле \texttt{input} на адрес контракта регистра соответствий, то эта транзакция будет включена в блок, но статус ее исполнения будет - ошибка, поскольку такой запрос удаления соответствия уже был послан в контракт. Статус можно подтвердить для данной транзакции в браузере блоков.
}

%-----------------------------------------------------
%US-016
\newuserstory{Отмена запроса на регистрацию или удаление соответствия}


После идентификации пользователь может отправить отмену для запроса на добавление или удаление соответствия между телефоном и своим аккаунтом.


\begin{myverbbox}[\small]{\cmdLine}
$ faceid.py --cancel <pin code>
\end{myverbbox}
\scriptExample{
Используя идентификатор, содержащийся в \texttt{person.json}, и PIN-код скрипт генерирует приватный ключ пользователя. В блокчейн сеть отправляется транзакция к контракту регистра соответствий на отмену запроса добавления или удаления соответствия. В терминал выводится хэш транзакции, в рамках которой просиходит отмена.

}

% AC-016-01
\begin{myverbbox}[\small]{\output}
$ cat network.json
{"rpcUrl": "https://sokol.poa.network", "gasPriceUrl": "https://gasprice.
poa.network/", "defaultGasPrice": 2000000000}
$ cat person.json
{"id": "37da04e7-f471-49c7-a54c-a08f05950fc5"}
$ faceid.py --add 4590 +79991234567
Registration request sent by 0xa7f3239715ff731a3d6fc477b18e35b9b0a9e1ede8
4bca5e91517e8e5bf1cc69
$ faceid.py --cancel 4590
Registration canceled by 0x99a7e5f28b653d63b5c0bbddebc91678530ee461e4fa21
c9ab2f281d28b8da5e
\end{myverbbox}
\acceptanceCriteria{В блокчейн сеть отправляется транзакция с отменой запроса регистрации соответствия аккаунта, ассоциированного с аккаунтом, указанным в файле \texttt{person.json}. При обработке запроса, контракт производить событие (\texttt{event}) \texttt{RegistrationСanceled}, в с указанием аккаунта, отправившего запрос:

\texttt{event RegistrationСanceled(address indexed sender);}


Для проведения транзакции выбрана цена из значения \texttt{fast}, возвращенного сервисом \texttt{https://gasprice.poa.network}.
}

% AC-016-02
\begin{myverbbox}[\small]{\output}
$ cat network.json
{"rpcUrl": "https://sokol.poa.network", "gasPriceUrl": "https://gasprice.
poa.network/", "defaultGasPrice": 2000000000}
$ cat person.json
{"id": "37da04e7-f471-49c7-a54c-a08f05950fc5"}
$ faceid.py --add 4590 +79991234567
Registration request sent by 0x20d2ac2a28641786ba03eff53facea687b35e08e68
0a1b71922c6fc1ed1f2735
$ faceid.py --cancel 4590
Registration canceled by 0x566674f71911cd3b7cbc2fd66353711d31aaa96f269440
d186374a017f53f324
\end{myverbbox}
\acceptanceCriteria{
}

% AC-016-03
\begin{myverbbox}[\small]{\output}
$ cat person.json
{"id": "37da04e7-f471-49c7-a54c-a08f05950fc5"}
$ faceid.py --del 4590
Unregistration request sent by 0x58aff6987a78c9109ff3b99b0e1cf333faba313b
c8002d4b333653be45d7e1d8
$ faceid.py --cancel 4590
Unregistration canceled by 0x84a9548edfa9ce5d05bb7c88702873196f7537e06061
9d4238e7f23ba8dfe3bd
\end{myverbbox}
\acceptanceCriteria{В блокчейн сеть отправляется транзакция с отменой запроса удаления ранее зарегистрированного соответствия аккаунта, ассоциированного с аккаунтом, указанным в файле \texttt{person.json}. При обработке запроса, контракт производить событие (\texttt{event}) \texttt{UnregistrationCanceled}, в с указанием аккаунта, отправившего запрос:

\texttt{event UnregistrationCanceled(address indexed sender);}
}

% AC-016-04
\begin{myverbbox}[\small]{\output}
$ cat person.json
{"id": "37da04e7-f471-49c7-a54c-a08f05950fc5"}
$ faceid.py --del 4590
Unregistration request sent by 0xbf08d8a2d7c7780024ba6e6dbe12d3d5a5a747ea
3071fca63d8a2ede50d3472a
$ faceid.py --cancel 4590
Registration canceled by 0x2946d6e84f7bc52619746d71da01ef04a31855a8deb6db
ea29310ab3b0f6201e
\end{myverbbox}
\acceptanceCriteria{
}

% AC-016-05
\begin{myverbbox}[\small]{\output}
$ faceid.py --add 4590 +79991234567
Registration request sent by 0xa7f3239715ff731a3d6fc477b18e35b9b0a9e1ede8
4bca5e91517e8e5bf1cc69
$ faceid.py --cancel 4590
Registration canceled by 0x99a7e5f28b653d63b5c0bbddebc91678530ee461e4fa21
c9ab2f281d28b8da5e
$ faceid.py --add 4590 +79991234567
Registration request sent by 0x4ae610dabaaddd16fe0097641e0e78b5423f891083
bb81266b228a49db54e2ff
\end{myverbbox}
\acceptanceCriteria{После отмены запроса регистрации соответствия, в блокчейн сеть снова можно отправить запрос на регистрацию соответствия для того же аккаунта.
}

% AC-016-06
\begin{myverbbox}[\small]{\output}
$ faceid.py --del 4590
Unregistration request sent by 0x58aff6987a78c9109ff3b99b0e1cf333faba313b
c8002d4b333653be45d7e1d8
$ faceid.py --cancel 4590
Unregistration canceled by 0x84a9548edfa9ce5d05bb7c88702873196f7537e06061
9d4238e7f23ba8dfe3bd
$ faceid.py --del 4590
Unregistration request sent by 0x03a8233da62730a9e5ac69d78c3fc3cf02b04ecb
995f5a9f9eb3b17c01692824
\end{myverbbox}
\acceptanceCriteria{После отмены запроса на удаление соответствия, в блокчейн сеть снова можно отправить отмену запроса удаления ранее зарегистрированного соответствия аккаунта.
}

% AC-016-07
\begin{myverbbox}[\small]{\output}
$ faceid.py --add 4590 +79991234567
Registration request sent by 0x4ae610dabaaddd16fe0097641e0e78b5423f891083
bb81266b228a49db54e2ff
$ faceid.py --cancel 4590
Registration canceled by 0x3d61f45ffa5ac772fc53d36fcc89bacaee425152d65c7f
d9a506ba40aa0ed6e1
$ faceid.py --cancel 4590
No requests found
\end{myverbbox}
\acceptanceCriteria{Выдается ошибка, если нет активных запросов на регистрацию или удаления соответствия.
}

% AC-016-08
\begin{myverbbox}[\small]{\output}
$ faceid.py --del 4590
Unregistration request sent by 0x03a8233da62730a9e5ac69d78c3fc3cf02b04ecb
995f5a9f9eb3b17c01692824
$ faceid.py --cancel 4590
Unregistration canceled by 0x41c2cd6cce43fddcd3c1489e1b2f40ad13b635bb0fce
22f2ea66a97f15b6c84b
$ faceid.py --cancel 4590
No requests found
\end{myverbbox}
\acceptanceCriteria{Выдается ошибка, если нет активных запросов на регистрацию или удаления соответствия.
}

% AC-016-09
\begin{myverbbox}[\small]{\output}
$ cat person.json
{"id": "37da04e7-f471-49c7-a54c-a08f05950fc5"}
$ faceid.py --cancel 4590
Registration canceled by 0x41c2cd6cce43fddcd3c1489e1b2f40ad13b635bb0fce22
f2ea66a97f15b6c84b
$ faceid.py --find /path/to/video23.avi
168f2e61-089f-88a9-53b2-3b3d0c497704 identified
$ faceid.py --cancel 5981
No requests found
\end{myverbbox}
\acceptanceCriteria{Выдается ошибка, если нет активных запросов на регистрацию или удаления соответствия.
}

% AC-016-10
\begin{myverbbox}[\small]{\output}
$ cat person.json
cat: person.json: No such file or directory
$ faceid.py --cancel 4590
ID is not found
\end{myverbbox}
\acceptanceCriteria{Выдается ошибка, если в текущей директории не существует файл \texttt{person.json}. Транзакция в блокчейн сеть не отправляется.
}

% AC-016-11
\begin{myverbbox}[\small]{\output}
$ cat person.json
{"id": "37da04e7-f471-49c7-a54c-a08f05950fc5"}
$ faceid.py --cancel 1234
No funds to send the request
\end{myverbbox}
\acceptanceCriteria{Поскольку ситуация, когда неправильный приватный ключ сформирован из-за некорретного PIN-кода, неотличима от ситуации, когда на аккаунте нет средств для оплаты комиссии на обработку транзакции, в терминал выводится сообщение об ошибке. Транзакция в блокчейн сеть не отправляется.
}

% AC-016-12
\begin{myverbbox}[\small]{\output}
$ cat registrar.json
cat: registrar.json: No such file or directory
$ faceid.py --cancel 4590
No contract address
\end{myverbbox}
\acceptanceCriteria{Выдается сообщение об ошибке, если в текущей директории нет файла \texttt{registrar.json}, содержащего адрес контракта регистра соответствий. Транзакция в блокчейн сеть не отправляется.
}

% AC-016-13
\begin{myverbbox}[\small]{\output}
$ cat registrar.json
{"registrar": {"address": "0x340Ec06460d9b2b7D23B40E5bCd0a81A09e06D81", "
startBlock": 456123}, "payments": {"address": "0x81A09e06D81797AE2b7D23B4
0E5bCd0a1da01eb0F951x", "startBlock": 456125}}
$ faceid.py --cancel 4590
Seems that the contract address is not the registrar contract.
\end{myverbbox}
\acceptanceCriteria{Выдается сообщение об ошибке, если в адрес контракта, указанного в файле \texttt{registrar.json}, не принадлежит контракту регистра соответствий.
}

% AC-016-14
\begin{myverbbox}[\small]{\output}
$ cat network.json | python -mjson.tool | grep gasPriceUrl 
    "gasPriceUrl": "https://gasprice.poa.network/",
$ curl https://gasprice.poa.network/
curl: (6) Could not resolve host: gasprice.poa.network
$ faceid.py --cancel 4590
Registration canceled by 0x41c2cd6cce43fddcd3c1489e1b2f40ad13b635bb0fce22
f2ea66a97f15b6c84b
\end{myverbbox}
\acceptanceCriteria{В блокчейн сеть отправляется транзакция c отменой запросом регистрации или удаления соответствия аккаунта. Транзакция успешно верифицирована и включена в блок. Для проведения транзакции выбрана цена из значения \texttt{defaultGasPrice} из файла \texttt{network.json}.
}

%-----------------------------------------------------
%US-017
\newuserstory{Отправка средств }


После идентификации пользователь может отправить часть средств, которые числятся на его балансе другому пользователю системы, указав его номер телефона.


\begin{myverbbox}[\small]{\cmdLine}
$ faceid.py --send <pin code> <phone number> <value>
\end{myverbbox}
\scriptExample{
Через RPC узел блокчейн сети происходит обращение к контракту регистрации соответствий на получение аккаунта соответствующего данному номеру телефона. После этого используя идентификатор, содержащийся в \texttt{person.json}, и PIN-код скрипт генерирует приватный ключ пользователя, в блокчейн сеть отправляется транзакция на перевод средств с баланса пользователя на аккаунт пользователя, ассоциированного с номером телефона.

}

% AC-017-01
\begin{myverbbox}[\small]{\output}
$ cat network.json | python -mjson.tool | grep gasPriceUrl 
    "gasPriceUrl": "https://gasprice.poa.network/",
$ cat person.json
{"id": "37da04e7-f471-49c7-a54c-a08f05950fc5"}
$ faceid.py --balance 1234
Your balance is 500 finney
$ faceid.py --send 1234 +79873344556 10000000000000000
Payment of 10 finney to +79873344556 scheduled
Transaction Hash: 0x27c9181caeb55d37e1105fa1a8648db7fe50f79064b98e56b8e85
4e3abb43728
$ faceid.py --balance 1234
Your balance is 489.860605 finney
\end{myverbbox}
\acceptanceCriteria{Средства успешно доставлены.
}

% AC-017-02
\begin{myverbbox}[\small]{\output}
$ cat person.json
{"id": "37da04e7-f471-49c7-a54c-a08f05950fc5"}
$ faceid.py --balance 1234
Your balance is 90 finney
$ faceid.py --send 1234 +79873344556 10000000000000000
No funds to send the payment
$ faceid.py --balance 1234
Your balance is 90 finney
\end{myverbbox}
\acceptanceCriteria{Транзакция в блокчейн сеть не отправляется.
}

% AC-017-03
\begin{myverbbox}[\small]{\output}
$ cat person.json
{"id": "37da04e7-f471-49c7-a54c-a08f05950fc5"}
$ faceid.py --send 1234 +79873312356 10000000000000000
No account with the phone number +79873312356
\end{myverbbox}
\acceptanceCriteria{Транзакция в блокчейн сеть не отправляется.
}

% AC-017-04
\begin{myverbbox}[\small]{\output}
$ cat person.json
{"id": "37da04e7-f471-49c7-a54c-a08f05950fc5"}
$ faceid.py --send 1114 +79873312356 10000000000000000
No funds to send the payment
\end{myverbbox}
\acceptanceCriteria{Поскольку ситуация, когда неправильный приватный ключ сформирован из-за некорретного PIN-кода, неотличима от ситуации, когда на аккаунте нет средств для перевода, в терминал выводится сообщение об ошибке. Транзакция в блокчейн сеть не отправляется.
}

% AC-017-05
\begin{myverbbox}[\small]{\output}
$ faceid.py --send 1234 +79122 10000000000000000
Incorrect phone number
$ faceid.py --send 1234 adad 10000000000000000
Incorrect phone number
$ faceid.py --send 1234 +73431231543543534653 10000000000000000
Incorrect phone number
\end{myverbbox}
\acceptanceCriteria{Выдается сообщение об ошибке, если номер телефона указан некорректно. Корректный номер содержит 11 цифр и начинается со знака +.
}

%-----------------------------------------------------
%US-018
\newuserstory{Генерация сертификата на получение средств}


После идентификации пользователь может отправить запрос на создание сертификата на получение определенного количества средств. Созданный сертификат впоследствии может быть использован любым пользователем до истечения срока действия. Сертификат имеет следующий формат: первые 32 байта - цифровой идентификатор сертификата, следующие 65 байт - цифровая подпись цифрового идентификатора (\texttt{r}, \texttt{s}, \texttt{v}).


\begin{myverbbox}[\small]{\cmdLine}
$ faceid.py --gift <pin code> <value> <expire date>
\end{myverbbox}
\scriptExample{
Используя идентификатор, содержащийся в \texttt{person.json}, и PIN-код скрипт генерирует приватный ключ пользователя, в блокчейн сеть отправляется транзакция к контракту управления сертификатами на создание нового сертификата на указанную сумму в \texttt{wei}, действующий до указанной даты в формате \texttt{HH:MM DD.MM.YYYY}. В терминал выводится созданный цифровой сертификат.

}

% AC-018-01
\begin{myverbbox}[\small]{\output}
$ cat network.json | python -mjson.tool | grep gasPriceUrl 
    "gasPriceUrl": "https://gasprice.poa.network/",
$ cat person.json
{"id": "37da04e7-f471-49c7-a54c-a08f05950fc5"}
$ faceid.py --balance
Your balance is 500 finney
$ faceid.py --gift 4590 10000000000000000 "15:40 08.03.2019"
aee79c36a8aff107f836a382d175b2e7cd86c34dae48a3288eb38b72da955d609e1c1f49a
a566e305bf444120af2f65923315026b0a03bcde4b139571752c0421a368cc1dc7171c6e8
08ba1fcb4cd7f3c034c64853dbd9a91bfc12ef9eece1e21c
$ faceid.py --balance
Your balance is 489.860605 finney
\end{myverbbox}
\acceptanceCriteria{В блокчейн сеть отправляется транзакция на создание сертификата. При обработке запроса, контракт производит событие (\texttt{event}) \texttt{CertificateCreated} c указанием идентификатора сертификата внутри контракта:

\texttt{event CertificateCreated(bytes32 indexed id);}


Для проведения транзакции выбрана цена из значения \texttt{fast}, возвращенного сервисом \texttt{https://gasprice.poa.network}. Баланс отправителя уменьшился на сумму указанную в сертификате.
}

% AC-018-02
\begin{myverbbox}[\small]{\output}
$ cat person.json
{"id": "37da04e7-f471-49c7-a54c-a08f05950fc5"}
$ faceid.py --balance
Your balance is 90 finney
$ faceid.py --gift 4590 100000000000000000 "15:40 08.03.2019"
No funds to create a certificate
$ faceid.py --balance
Your balance is 90 finney
\end{myverbbox}
\acceptanceCriteria{На балансе пользователя недостаточно средств чтобы создать сертификат с указанной суммой. Транзакция в блокчейн сеть не отправляется. Баланс пользователя не уменьшается. В терминал выводится сообщение об ошибке.
}

% AC-018-03
\begin{myverbbox}[\small]{\output}
$ cat person.json
{"id": "37da04e7-f471-49c7-a54c-a08f05950fc5"}
$ date
Wed Oct  21 07:28:00 MSK 2015
$ faceid.py --gift 4590 1000 "09:00 26.10.1985"
Expiration date is invalid
\end{myverbbox}
\acceptanceCriteria{Указанная дата годности сертификата уже истекла. Транзакция в блокчейн сеть не отправляется. Баланс пользователя не уменьшается. В терминал выводится сообщение об ошибке.
}

% AC-018-04
\begin{myverbbox}[\small]{\output}
$ cat person.json
cat: person.json: No such file or directory
$ faceid.py --gift 1234 1000 "15:40 08.03.2019"
ID is not found
\end{myverbbox}
\acceptanceCriteria{Выдается ошибка, если в текущей директории не существует файл \texttt{person.json}. Транзакция в блокчейн сеть не отправляется.
}

% AC-018-05
\begin{myverbbox}[\small]{\output}
$ cat person.json
{"id": "37da04e7-f471-49c7-a54c-a08f05950fc5"}
$ faceid.py --gift 1234 1000 "15:40 08.03.2019"
No funds to create a certificate
\end{myverbbox}
\acceptanceCriteria{Указан неверный пинкод. Поскольку ситуация, когда неправильный приватный ключ сформирован из-за некорретного PIN-кода, неотличима от ситуации, когда на аккаунте нет средств для оплаты транзакции, в терминал выводится сообщение об ошибке. Транзакция в блокчейн сеть не отправляется.
}

% AC-018-06
\begin{myverbbox}[\small]{\output}
$ cat network.json | python -mjson.tool | grep gasPriceUrl 
    "gasPriceUrl": "https://gasprice.poa.network/",
$ cat person.json
{"id": "37da04e7-f471-49c7-a54c-a08f05950fc5"}
$ faceid.py --balance
Your balance is 485.992209 finney
$ faceid.py --gift 4590 10000000000000000 "15:40 08.03.2019"
e4473f3c3e3f3803f5c640a93b44b9a35968df4e7d4a907cab99926f4af694a7053ac2253
149e7a5212f816cd367a471acc97c67cf38b7b3cf0d59ff8fbcf6d86a8d179e6e74d103f9
e7e423630454b06ffe5dd2806e3bc0551971675619b31a1b
$ faceid.py --balance
Your balance is 475.852814 poa
$ faceid.py --gift 4590 10000000000000000 "15:40 08.03.2019"
73bd8c3b60a59a3ccb37764f87e62dafab374d3c4885edd21192f060b83a47e616dcc4fd2
2eac78fdc3e023fa60ac7863a4111f245556b3966b2d960b104c53f5d7a2c67b787abdd86
2826708737d425a5d0e774d055a996cc853591113afc881c
$ faceid.py --balance
Your balance is 465.713419 poa
$ cat person.json
{"id": "da04e377-47f1-c749-54ca-0fc5a08f0595"}
$ faceid.py --balance
Your balance is 1.5 poa
$ faceid.py --gift 6104 10000000000000000 "15:40 08.03.2019"
b6f4176f9fe77211717fb3f7f41995b29cc04bd888d24eaa8d398fd57b8fc2153c352e75a
dbbb101570a58056ae00a0c6fb4c176f7e91f3489ae4bc1104f857e551bd96f1f7223ad9e
735f1544cc0eb89f2ed703dc4243cd1a2d96a47b91d0f41c
$ faceid.py --balance
Your balance is 1.48983 poa
\end{myverbbox}
\acceptanceCriteria{В блокчейн сеть отправляется несколько транзакции на создание сертификатов c нескольких аккаунтов. Для проведения транзакции выбрана цена из значения \texttt{fast}, возвращенного сервисом \texttt{https://gasprice.poa.network}. Баланс отправителей уменьшился на суммы указанные в сертификатах.
}

% AC-018-07
\begin{myverbbox}[\small]{\output}
$ cat network.json | python -mjson.tool | grep gasPriceUrl 
    "gasPriceUrl": "https://gasprice.poa.network/",
$ curl https://gasprice.poa.network/
curl: (6) Could not resolve host: gasprice.poa.network
$ cat person.json
{"id": "37da04e7-f471-49c7-a54c-a08f05950fc5"}
$ faceid.py --balance
Your balance is 500 finney
$ faceid.py --gift 4590 10000000000000000 "15:40 08.03.2019"
c265d15e7a89cab7a9e30a3287d0be6d8b2ab3e635ede29129514002b5bf4cdf7f880a32d
623da84c78d89b7814d5a0d41fc249d55902c7835c0cd2fa978cd077b4483d7d828a63d1c
bec57e5ebc65fb262c7d3c2fc3f3c7afc0b599fee1371b1c
$ faceid.py --balance
Your balance is 489.860605 finney
\end{myverbbox}
\acceptanceCriteria{В блокчейн сеть отправляется транзакция на создание сертификата. Транзакция успешно верифицирована и включена в блок. Для проведения транзакции выбрана цена из значения \texttt{defaultGasPrice} из файла \texttt{network.json}.
}

% AC-018-08
\begin{myverbbox}[\small]{\output}
$ cat registrar.json
cat: registrar.json: No such file or directory
$ faceid.py --gift 4590 10000000000000000 "15:40 08.03.2019"
No contract address
\end{myverbbox}
\acceptanceCriteria{Выдается сообщение об ошибке, если в текущей директории нет файла \texttt{registrar.json}, содержащего адрес контракта управления сертификатами. Транзакция в блокчейн сеть не отправляется.
}

% AC-018-09
\begin{myverbbox}[\small]{\output}
$ cat registrar.json
{"registrar": {"address": "0x340Ec06460d9b2b7D23B40E5bCd0a81A09e06D81", "
startBlock": 456123}, "payments": {"address": "0x81A09e06D81797AE2b7D23B4
0E5bCd0a1da01eb0F951x", "startBlock": 456125}}
$ faceid.py --gift 4590 10000000000000000 "15:40 08.03.2019"
Seems that the contract address is not the certificates contract.
\end{myverbbox}
\acceptanceCriteria{Выдается сообщение об ошибке, если в адрес контракта, указанного в файле \texttt{registrar.json}, не принадлежит контракту управления сертификатами.
}

%-----------------------------------------------------
%US-019
\newuserstory{Использование сертификата на получение средств }


После идентификации пользователь может отправить запрос на получение средств с уже созданного сертификата. Сертификат может быть использован только один раз и только до истечения указанного при создании срока годности.


\begin{myverbbox}[\small]{\cmdLine}
$ faceid.py --receive <pin code> <certificate>
\end{myverbbox}
\scriptExample{
Используя идентификатор, содержащийся в \texttt{person.json}, и PIN-код скрипт генерирует приватный ключ пользователя. В блокчейн сеть отправляется транзакция к контракту управления сертификатами на получение средств c сертификата.

}

% AC-019-01
\begin{myverbbox}[\small]{\output}
$ cat network.json | python -mjson.tool | grep gasPriceUrl 
    "gasPriceUrl": "https://gasprice.poa.network/",
$ cat person.json
{"id": "37da04e7-f471-49c7-a54c-a08f05950fc5"}
$ faceid.py --balance
Your balance is 500 finney
$ faceid.py --receive 4590 aee79c36a8aff107f836a382d175b2e7cd86c34dae48a3
288eb38b72da955d609e1c1f49aa566e305bf444120af2f65923315026b0a03bcde4b1395
71752c0421a368cc1dc7171c6e808ba1fcb4cd7f3c034c64853dbd9a91bfc12ef9eece1e2
1c
Received funds from the certificate
$ faceid.py --balance
Your balance is 510 finney
\end{myverbbox}
\acceptanceCriteria{В блокчейн сеть отправляется транзакция на получение средств. При обработке запроса, контракт производит событие (\texttt{event}) \texttt{CertificateUsed} с указанием идентификатора сертификата внутри контракта:

\texttt{event CertificateUsed(bytes32 indexed id);}


Для проведения транзакции выбрана цена из значения \texttt{fast}, возвращенного сервисом \texttt{https://gasprice.poa.network}. Баланс отправителя увеличился на сумму указанную в сертификате.
}

% AC-019-02
\begin{myverbbox}[\small]{\output}
$ cat person.json
{"id": "37da04e7-f471-49c7-a54c-a08f05950fc5"}
$ faceid.py --balance
Your balance is 510 finney
$ faceid.py --receive 4590 aee79c36a8aff107f836a382d175b2e7cd86c34dae48a3
288eb38b72da955d609e1c1f49aa566e305bf444120af2f65923315026b0a03bcde4b1395
71752c0421a368cc1dc7171c6e808ba1fcb4cd7f3c034c64853dbd9a91bfc12ef9eece1e2
1c
Cannot receive funds from the certificate
$ faceid.py --balance
Your balance is 510 finney
\end{myverbbox}
\acceptanceCriteria{Нельзя использовать сертификат, так как он уже был использован ранее. Транзакция в блокчейн сеть не отправляется. Баланс пользователя не изменяется. В терминал выводится сообщение об ошибке.
}

% AC-019-03
\begin{myverbbox}[\small]{\output}
$ cat person.json
{"id": "37da04e7-f471-49c7-a54c-a08f05950fc5"}
$ faceid.py --receive 4590 e4473f3c3e3f3803f5c640a93b44b9a35968df4e7d4a90
7cab99926f4af694a7053ac2253149e7a5212f816cd367a471acc97c67cf38b7b3cf0d59f
f8fbcf6d86a8d179e6e74d103f9e7e423630454b06ffe5dd2806e3bc0551971675619b31a
1b
Cannot receive funds from the certificate
\end{myverbbox}
\acceptanceCriteria{Нельзя использовать сертификат, так как указанная дата годности сертификата уже истекла. Транзакция в блокчейн сеть не отправляется. Баланс пользователя не изменяется. В терминал выводится сообщение об ошибке.
}

% AC-019-04
\begin{myverbbox}[\small]{\output}
$ cat person.json
cat: person.json: No such file or directory
$ faceid.py --receive 1234 e4473f3c3e3f3803f5c640a93b44b9a35968df4e7d4a90
7cab99926f4af694a7053ac2253149e7a5212f816cd367a471acc97c67cf38b7b3cf0d59f
f8fbcf6d86a8d179e6e74d103f9e7e423630454b06ffe5dd2806e3bc0551971675619b31a
1b
ID is not found
\end{myverbbox}
\acceptanceCriteria{Выдается ошибка, если в текущей директории не существует файл \texttt{person.json}. Транзакция в блокчейн сеть не отправляется.
}

% AC-019-05
\begin{myverbbox}[\small]{\output}
$ cat person.json
{"id": "37da04e7-f471-49c7-a54c-a08f05950fc5"}
$ faceid.py --receive 4590 0123456789abcdef0123456789abcdef0123456789abcd
ef0123456789abcdef0123456789abcdef0123456789abcdef0123456789abcdef0123456
789abcdef0123456789abcdef0123456789abcdef0123456789abcdef0123456789abcdef
01
Cannot receive funds from the certificate
\end{myverbbox}
\acceptanceCriteria{Указан несуществующий сертификат, который невозможно использовать. Транзакция в блокчейн сеть не отправляется. Баланс пользователя не изменяется. В терминал выводится сообщение об ошибке.
}

% AC-019-06
\begin{myverbbox}[\small]{\output}
$ cat network.json | python -mjson.tool | grep gasPriceUrl 
    "gasPriceUrl": "https://gasprice.poa.network/",
$ cat person.json
{"id": "37da04e7-f471-49c7-a54c-a08f05950fc5"}
$ faceid.py --gift 4590 1000000000000000 "15:40 08.03.2019"
e4473f3c3e3f3803f5c640a93b44b9a35968df4e7d4a907cab99926f4af694a7053ac2253
149e7a5212f816cd367a471acc97c67cf38b7b3cf0d59ff8fbcf6d86a8d179e6e74d103f9
e7e423630454b06ffe5dd2806e3bc0551971675619b31a1b
$ faceid.py --gift 4590 10000000000000000 "15:40 08.03.2019"
73bd8c3b60a59a3ccb37764f87e62dafab374d3c4885edd21192f060b83a47e616dcc4fd2
2eac78fdc3e023fa60ac7863a4111f245556b3966b2d960b104c53f5d7a2c67b787abdd86
2826708737d425a5d0e774d055a996cc853591113afc881c
$ faceid.py --balance
Your balance is 465.713419 poa
$ faceid.py --receive 4590 73bd8c3b60a59a3ccb37764f87e62dafab374d3c4885ed
d21192f060b83a47e616dcc4fd22eac78fdc3e023fa60ac7863a4111f245556b3966b2d96
0b104c53f5d7a2c67b787abdd862826708737d425a5d0e774d055a996cc853591113afc88
1c
Received funds from the certificate
$ faceid.py --balance
Your balance is 475.713419 poa
$ faceid.py --receive 4590 e4473f3c3e3f3803f5c640a93b44b9a35968df4e7d4a90
7cab99926f4af694a7053ac2253149e7a5212f816cd367a471acc97c67cf38b7b3cf0d59f
f8fbcf6d86a8d179e6e74d103f9e7e423630454b06ffe5dd2806e3bc0551971675619b31a
1b
Received funds from the certificate
$ faceid.py --balance
Your balance is 476.713419 poa
\end{myverbbox}
\acceptanceCriteria{В блокчейн сеть отправляется несколько транзакции на создание сертификатов c нескольких аккаунтов, а затем несколько транзакций на получение средств. Для проведения транзакций выбрана цена из значения \texttt{fast}, возвращенного сервисом \texttt{https://gasprice.poa.network}. Баланс отправителей корректно изменяется на суммы указанные в сертификатах.
}

% AC-019-07
\begin{myverbbox}[\small]{\output}
$ cat network.json | python -mjson.tool | grep gasPriceUrl 
    "gasPriceUrl": "https://gasprice.poa.network/",
$ curl https://gasprice.poa.network/
curl: (6) Could not resolve host: gasprice.poa.network
$ cat person.json
{"id": "37da04e7-f471-49c7-a54c-a08f05950fc5"}
$ faceid.py --balance
Your balance is 489.860605 finney
$ faceid.py --receive 4590 c265d15e7a89cab7a9e30a3287d0be6d8b2ab3e635ede2
9129514002b5bf4cdf7f880a32d623da84c78d89b7814d5a0d41fc249d55902c7835c0cd2
fa978cd077b4483d7d828a63d1cbec57e5ebc65fb262c7d3c2fc3f3c7afc0b599fee1371b
1c
Received funds from the certificate
$ faceid.py --balance
Your balance is 499.860605 finney
\end{myverbbox}
\acceptanceCriteria{В блокчейн сеть отправляется транзакция на получение средств с сертификата. Транзакция успешно верифицирована и включена в блок. Для проведения транзакции выбрана цена из значения \texttt{defaultGasPrice} из файла \texttt{network.json}.
}

% AC-019-08
\begin{myverbbox}[\small]{\output}
$ cat registrar.json
cat: registrar.json: No such file or directory
$ faceid.py --receive 4590 c265d15e7a89cab7a9e30a3287d0be6d8b2ab3e635ede2
9129514002b5bf4cdf7f880a32d623da84c78d89b7814d5a0d41fc249d55902c7835c0cd2
fa978cd077b4483d7d828a63d1cbec57e5ebc65fb262c7d3c2fc3f3c7afc0b599fee1371b
1c
No contract address
\end{myverbbox}
\acceptanceCriteria{Выдается сообщение об ошибке, если в текущей директории нет файла \texttt{registrar.json}, содержащего адрес контракта управления контрактами. Транзакция в блокчейн сеть не отправляется.
}

% AC-019-09
\begin{myverbbox}[\small]{\output}
$ cat registrar.json
{"registrar": {"address": "0x340Ec06460d9b2b7D23B40E5bCd0a81A09e06D81", "
startBlock": 456123}, "payments": {"address": "0x81A09e06D81797AE2b7D23B4
0E5bCd0a1da01eb0F951x", "startBlock": 456125}}
$ faceid.py --receive 4590 c265d15e7a89cab7a9e30a3287d0be6d8b2ab3e635ede2
9129514002b5bf4cdf7f880a32d623da84c78d89b7814d5a0d41fc249d55902c7835c0cd2
fa978cd077b4483d7d828a63d1cbec57e5ebc65fb262c7d3c2fc3f3c7afc0b599fee1371b
1c
Seems that the contract address is not the certificates contract.
\end{myverbbox}
\acceptanceCriteria{Выдается сообщение об ошибке, если в адрес контракта, указанного в файле \texttt{registrar.json}, не принадлежит контракту управления сертификатами.
}

% AC-019-10
\begin{myverbbox}[\small]{\output}
$ faceid.py --balance
Your balance is 500 finney
$ faceid.py --receive 4590 aee79c36a8aff107f836a382d175b2e7cd86c34dae48a3
288eb38b72da955d609e1c1f49aa566e305bf444120af2f65923315026b0a03bcde4b1395
71752c0421a368cc1dc7171c6e808ba1fcb4cd7f3c034c64853dbd9a91bfc12ef9eece1e2
1c
Received funds from the certificate
$ faceid.py --balance
Your balance is 510 finney
\end{myverbbox}
\acceptanceCriteria{Если из транзакции, которая была отправлена в результате команды \texttt{faceid.py --receive}, извлечь поле input и отправить его в новой транзакции снова в поле input на адрес контракта управления сертификатами, то эта транзакция будет включена в блок, но статус ее исполнения будет - ошибка по причине того, что сертификат уже был использован ранее. Следовательно, повторно сертификат не должен быть использован.
}

%-----------------------------------------------------
%US-020
\newuserstory{Вернуть средства из неиспользованных сертификатов }


После идентификации пользователь может отправить запрос на возврат средств с уже созданного сертификата после истечения срока годности. Возврат может быть осуществлен только один раз и только с аккаунта создателя сертификата. 


\begin{myverbbox}[\small]{\cmdLine}
$ faceid.py --withdraw <pin code>
\end{myverbbox}
\scriptExample{
Используя идентификатор, содержащийся в \texttt{person.json}, и PIN-код скрипт генерирует приватный ключ пользователя. В блокчейн сеть отправляется транзакция к контракту управления сертификатами на возврат средств c сертификата.

}

% AC-020-01
\begin{myverbbox}[\small]{\output}
$ cat network.json | python -mjson.tool | grep gasPriceUrl 
    "gasPriceUrl": "https://gasprice.poa.network/",
$ cat person.json
{"id": "37da04e7-f471-49c7-a54c-a08f05950fc5"}
$ faceid.py --balance
Your balance is 500 finney
$ faceid.py --withdraw 4590
Withdrew funds from the certificate
$ faceid.py --balance
Your balance is 510 finney
\end{myverbbox}
\acceptanceCriteria{В блокчейн сеть отправляется транзакция на возврат средств. При обработке запроса, контракт производит событие (\texttt{event}) \texttt{CertificateWithdrew} с указанием идентификатора сертификата внутри контракта:

\texttt{event CertificateWithdrew(bytes32 indexed id);}


Для проведения транзакции выбрана цена из значения \texttt{fast}, возвращенного сервисом \texttt{https://gasprice.poa.network}. Баланс отправителя увеличился на сумму указанную в сертификате.
}

% AC-020-02
\begin{myverbbox}[\small]{\output}
$ cat person.json
{"id": "37da04e7-f471-49c7-a54c-a08f05950fc5"}
$ faceid.py --balance
Your balance is 510 finney
$ faceid.py --withdraw 4590
Cannot withdraw funds from the certificate
$ faceid.py --balance
Your balance is 510 finney
\end{myverbbox}
\acceptanceCriteria{Нельзя вернуть средства, так как аналогичный запрос уже был выполнен. Транзакция в блокчейн сеть не отправляется. Баланс пользователя не изменяется. В терминал выводится сообщение об ошибке.
}

% AC-020-03
\begin{myverbbox}[\small]{\output}
$ cat person.json
{"id": "37da04e7-f471-49c7-a54c-a08f05950fc5"}
$ faceid.py --withdraw 4590
Cannot withdraw funds from the certificate
\end{myverbbox}
\acceptanceCriteria{Нельзя вернуть средства, так как указанная дата годности сертификата еще не истекла. Транзакция в блокчейн сеть не отправляется. Баланс пользователя не изменяется. В терминал выводится сообщение об ошибке.
}

% AC-020-04
\begin{myverbbox}[\small]{\output}
$ cat person.json
cat: person.json: No such file or directory
$ faceid.py --withdraw 1234
ID is not found
\end{myverbbox}
\acceptanceCriteria{Выдается ошибка, если в текущей директории не существует файл \texttt{person.json}. Транзакция в блокчейн сеть не отправляется.
}

% AC-020-05
\begin{myverbbox}[\small]{\output}
$ cat person.json
{"id": "37da04e7-f471-49c7-a54c-a08f05950fc5"}
$ faceid.py --withdraw 4590
Cannot withdraw funds from the certificate
\end{myverbbox}
\acceptanceCriteria{Указан несуществующий сертификат, который невозможно использовать. Транзакция в блокчейн сеть не отправляется. Баланс пользователя не изменяется. В терминал выводится сообщение об ошибке.
}

% AC-020-06
\begin{myverbbox}[\small]{\output}
$ cat network.json | python -mjson.tool | grep gasPriceUrl 
    "gasPriceUrl": "https://gasprice.poa.network/",
$ cat person.json
{"id": "37da04e7-f471-49c7-a54c-a08f05950fc5"}
$ faceid.py --gift 4590 1000000000000000 "15:40 08.03.2019"
e4473f3c3e3f3803f5c640a93b44b9a35968df4e7d4a907cab99926f4af694a7053ac2253
149e7a5212f816cd367a471acc97c67cf38b7b3cf0d59ff8fbcf6d86a8d179e6e74d103f9
e7e423630454b06ffe5dd2806e3bc0551971675619b31a1b
$ faceid.py --gift 4590 10000000000000000 "15:40 08.03.2019"
73bd8c3b60a59a3ccb37764f87e62dafab374d3c4885edd21192f060b83a47e616dcc4fd2
2eac78fdc3e023fa60ac7863a4111f245556b3966b2d960b104c53f5d7a2c67b787abdd86
2826708737d425a5d0e774d055a996cc853591113afc881c
$ faceid.py --balance
Your balance is 465.713419 poa
$ faceid.py --withdraw 4590
Withdrew funds from the certificate
$ faceid.py --balance
Your balance is 476.713419 poa
\end{myverbbox}
\acceptanceCriteria{В блокчейн сеть отправляется несколько транзакции на создание сертификатов c нескольких аккаунтов, а затем транзакция на возврат средств. Для проведения транзакций выбрана цена из значения \texttt{fast}, возвращенного сервисом \texttt{https://gasprice.poa.network}. Баланс отправителей корректно изменяется на суммы указанные в сертификатах.
}

% AC-020-07
\begin{myverbbox}[\small]{\output}
$ cat network.json | python -mjson.tool | grep gasPriceUrl 
    "gasPriceUrl": "https://gasprice.poa.network/",
$ curl https://gasprice.poa.network/
curl: (6) Could not resolve host: gasprice.poa.network
$ cat person.json
{"id": "37da04e7-f471-49c7-a54c-a08f05950fc5"}
$ faceid.py --balance
Your balance is 489.860605 finney
$ faceid.py --withdraw 4590
Withdrew funds from the certificate
$ faceid.py --balance
Your balance is 499.860605 finney
\end{myverbbox}
\acceptanceCriteria{В блокчейн сеть отправляется транзакция на получение средств с сертификата. Транзакция успешно верифицирована и включена в блок. Для проведения транзакции выбрана цена из значения \texttt{defaultGasPrice} из файла \texttt{network.json}.
}

% AC-020-08
\begin{myverbbox}[\small]{\output}
$ cat registrar.json
cat: registrar.json: No such file or directory
$ faceid.py --withdrew 4590
No contract address
\end{myverbbox}
\acceptanceCriteria{Выдается сообщение об ошибке, если в текущей директории нет файла \texttt{registrar.json}, содержащего адрес контракта управления контрактами. Транзакция в блокчейн сеть не отправляется.
}

% AC-020-09
\begin{myverbbox}[\small]{\output}
$ cat registrar.json
{"registrar": {"address": "0x340Ec06460d9b2b7D23B40E5bCd0a81A09e06D81", "
startBlock": 456123}, "payments": {"address": "0x81A09e06D81797AE2b7D23B4
0E5bCd0a1da01eb0F951x", "startBlock": 456125}}
$ faceid.py --withdrew 4590
Seems that the contract address is not the certificates contract.
\end{myverbbox}
\acceptanceCriteria{Выдается сообщение об ошибке, если в адрес контракта, указанного в файле \texttt{registrar.json}, не принадлежит контракту управления сертификатами.
}

% AC-020-10
\begin{myverbbox}[\small]{\output}
$ faceid.py --balance
Your balance is 500 finney
$ faceid.py --withdrew 4590
Withdrew funds from the certificate
$ faceid.py --balance
Your balance is 510 finney
\end{myverbbox}
\acceptanceCriteria{Если из транзакции, которая была отправлена в результате команды \texttt{faceid.py --receive}, извлечь поле input и отправить его в новой транзакции снова в поле input на адрес контракта управления сертификатами, то эта транзакция будет включена в блок, но статус ее исполнения будет - ошибка по причине того, что возврат средств уже был осуществлен ранее. Следовательно, повторно возврат не должен быть произведен.
}

%-----------------------------------------------------
%US-021
\newuserstory{Получение истории платежей }


После индентификации пользователь может получить список платежей, связанных с аккаунтом данного пользователя.


\begin{myverbbox}[\small]{\cmdLine}
$ faceid.py --ops 1234
\end{myverbbox}
\scriptExample{
Используя идентификатор, содержащийся в \texttt{person.json}, и PIN-код скрипт генерирует приватный ключ пользователя. В блокчейн сеть отправляется запрос на баланс аккаунта, полученного из приватного ключа.

История платежей отображается с момента первого подтверждения регистрации соответствия аккаунта, для которого отображаются платежи. В истории 

Сумма в каждой строчке, обозначающей платеж, всегда отображается в poa c точностью до 6 знака.

}

% AC-021-01
\begin{myverbbox}[\small]{\output}
$ cat person.json
{"id": "37da04e7-f471-49c7-a54c-a08f05950fc5"}
$ faceid.py --ops 1234
Operations:
13:01:42 04.03.2019 FROM: +79418552734 2.5 poa
18:31:52 08.03.2019 FROM: +79019594479 0.5 poa
18:32:02 08.03.2019   TO: +79418552734 0.15 poa
18:33:37 08.03.2019 FROM: +79114505724 0.3 poa
18:34:17 08.03.2019   TO: +79286975842 1.3 poa
$ faceid.py --balance 1234
Your balance is 2.349641 poa
\end{myverbbox}
\acceptanceCriteria{Поскольку на данном аккаунте был ненулевой баланс до подтверждения, поэтому итоговый баланс аккаунта больше суммы всех входящих платежей за вычетом всех исходящих платежей.
}

% AC-021-02
\begin{myverbbox}[\small]{\output}
$ cat person.json
{"id": "81e1dbe6-e0e0-4cf6-af4f-ff8341833b55"}
$ faceid.py --ops 6801
No operations found
$ faceid.py --balance 1234
Your balance is 598458 finney
\end{myverbbox}
\acceptanceCriteria{Поскольку на данном аккаунте не было операций, то выводится сообщение об ошибке.
}

% AC-021-03
\begin{myverbbox}[\small]{\output}
$ cat person.json
{"id": "37da04e7-f471-49c7-a54c-a08f05950fc5"}
$ faceid.py --ops 1234
Operations:
13:01:42 04.03.2019 FROM: +79010451275 2.5 poa
18:31:52 08.03.2019 FROM: +79019594479 0.5 poa
18:32:02 08.03.2019   TO: +79010451275 0.15 poa
18:33:37 08.03.2019 FROM: +79114505724 0.3 poa
18:34:17 08.03.2019   TO: +79286975842 1.3 poa
$ kyc.py --list del
0x32276b955E7dCBeA1C97fe8f06053E760E739e8d: +79010451275
$ kyc.py --confirm 0x32276b955E7dCBeA1C97fe8f06053E760E739e8d
Confirmed by 0xbdf07588e6652aaee28b29156c2ae266c9ba46e253ddd99b466cce31f7
7c51a3
$ faceid.py --ops 1234
Operations:
13:01:42 04.03.2019 FROM: +79010451275 2.5 poa
18:31:52 08.03.2019 FROM: +79019594479 0.5 poa
18:32:02 08.03.2019   TO: +79010451275 0.15 poa
18:33:37 08.03.2019 FROM: +79114505724 0.3 poa
18:34:17 08.03.2019   TO: +79286975842 1.3 poa
\end{myverbbox}
\acceptanceCriteria{Несмотря на то, что соответствие между аккаунтом и телефонным номером было удалено, в списке платежей выводится номер телефона, соответствовавший аккаунту в момент проведения конкретных платежей.
}

% AC-021-04
\begin{myverbbox}[\small]{\output}
$ cat person.json
{"id": "37da04e7-f471-49c7-a54c-a08f05950fc5"}
$ faceid.py --ops 1234
Operations:
13:01:42 04.03.2019 FROM: +79010451275 2.5 poa
18:31:52 08.03.2019 FROM: +79019594479 0.5 poa
18:32:02 08.03.2019   TO: +79010451275 0.15 poa
18:33:37 08.03.2019 FROM: +79114505724 0.3 poa
18:34:17 08.03.2019   TO: +79286975842 1.3 poa
$ kyc.py --list del
0xEF8eedf35C2D212bf70389ecA193622e833C3652: +79010451275
$ kyc.py --confirm 0xEF8eedf35C2D212bf70389ecA193622e833C3652
Confirmed by 0xbdf07588e6652aaee28b29156c2ae266c9ba46e253ddd99b466cce31f7
7c51a3
$ cat person.json
{"id": "81e1dbe6-e0e0-4cf6-af4f-ff8341833b55"}
$ faceid.py --add 6801 +79991234567
Registration request sent by 0xa7f3239715ff731a3d6fc477b18e35b9b0a9e1ede8
4bca5e91517e8e5bf1cc69
$ kyc.py --confirm 0xEF8eedf35C2D212bf70389ecA193622e833C3652
Confirmed by 0xbdf07588e6652aaee28b29156c2ae266c9ba46e253ddd99b466cce31f7
7c51a3
$ faceid.py --send 6801 +79873344556 15026871000000000000
Payment of 10 finney to +79873344556 scheduled
Transaction Hash: 0x27c9181caeb55d37e1105fa1a8648db7fe50f79064b98e56b8e85
4e3abb43728
$ cat person.json
{"id": "37da04e7-f471-49c7-a54c-a08f05950fc5"}
$ faceid.py --ops 1234
Operations:
13:01:42 04.03.2019 FROM: +79010451275 2.5 poa
18:31:52 08.03.2019 FROM: +79019594479 0.5 poa
18:32:02 08.03.2019   TO: +79010451275 0.15 poa
18:33:37 08.03.2019 FROM: +79114505724 0.3 poa
18:34:17 08.03.2019   TO: +79286975842 1.3 poa
18:39:34 08.03.2019 FROM: +79991234567 1.502687 poa
\end{myverbbox}
\acceptanceCriteria{Несмотря на то, что соответствие между аккаунтом и телефонным номером было изменено, в списке платежей выводятся оба номера телефона, соответствовавшие аккаунту в момент проведения конкретных платежей.
}

% AC-021-05
\begin{myverbbox}[\small]{\output}
$ cat person.json
{"id": "81e1dbe6-e0e0-4cf6-af4f-ff8341833b55"}
$ faceid.py --ops 1010
No operations found
\end{myverbbox}
\acceptanceCriteria{Поскольку ситуация, когда неправильный приватный ключ сформирован из-за некорретного PIN-кода, неотличима от ситуации, когда нет операций ассоциированных с аккаунтом, в терминал выводится сообщение об ошибке.
}

% AC-021-05
\begin{myverbbox}[\small]{\output}
$ cat registrar.json
cat: registrar.json: No such file or directory
$ faceid.py --ops 6801
No contract address
\end{myverbbox}
\acceptanceCriteria{Выдается сообщение об ошибке, если в текущей директории нет файла registrar.json, содержащего адрес контракта регистра соответствий.
}

% AC-021-06
\begin{myverbbox}[\small]{\output}
$ cat registrar.json
{"registrar": {"address": "0x340Ec06460d9b2b7D23B40E5bCd0a81A09e06D81", "
startBlock": 456123}, "payments": {"address": "0x81A09e06D81797AE2b7D23B4
0E5bCd0a1da01eb0F951x", "startBlock": 456125}}
$ faceid.py --ops 6801
Seems that the contract address is not the certificates contract.
\end{myverbbox}
\acceptanceCriteria{Выдается сообщение об ошибке, если в адрес контракта, указанного в файле registrar.json, не принадлежит контракту регистра соответствий.
}

%-----------------------------------------------------
%US-022
\newuserstory{Получение истории платежей, включая использование сертификатов}



\begin{myverbbox}[\small]{\cmdLine}
$ faceid.py --opsall 1234
\end{myverbbox}
\scriptExample{


}


\subsection*{Администрирование сервиса KYC}


Управление сервисом соответствий блокчейн аккаунтов и номеров телефонов происходить с использованием отдельной компоненты. Эта компонента позволяет администратору сервиса подтерждать запросы на регистрацию соответствий и запросы на удаление соответствий. 

\begin{myverbbox}{\scriptFile}
kyc.py <command> [options]
\end{myverbbox}
\scriptTitle


В зависимости от команды, через RPC узел блокчейн сети будет отправляться либо запрос на информацию, либо транзакция к контракту регистра соответствий. 

%-----------------------------------------------------
%US-023
\newuserstory{Получение всех запросов на регистрацию соответствий}


Любой пользователь сервиса,  может получить список всех запросов на регистрацию соответствий. 


\begin{myverbbox}[\small]{\cmdLine}
$ kyc.py --list add
\end{myverbbox}
\scriptExample{
Через RPC узел блокчейн сети отправляется запрос к контракту, адрес которого указан в поле \texttt{registrar} файла \texttt{registrar.json}, на получение всех неотменненных запросов на регистрацию соответствий. Запросы выводятся в формате \texttt{отправитель запроса: номер телефона}

}

% AC-023-01
\begin{myverbbox}[\small]{\output}
$ setup.py --deploy
KYC Registrar: 0x00360d2b7D240Ec0643B6D819ba81A09e40E5bCd
Payment Handler: 0x95426f2bC716022fCF1dEf006dbC4bB81f5B5164
$ cat registrar.json
{"registrar": {"address": "0x00360d2b7D240Ec0643B6D819ba81A09e40E5bCd", "
startBlock": 123456}, "payments": {"address": "0x95426f2bC716022fCF1dEf00
6dbC4bB81f5B5164", "startBlock": 123457}}
$ kyc.py --list add
No KYC registration requests found
\end{myverbbox}
\acceptanceCriteria{Поскольку в контракте, адрес которого указан в файле \texttt{registrar.json}, не было зарегистрировано ни одного запроса регистрации соответствия аккаунта, то выдается соответствующее сообщение. Транзакции в сеть не отправляются.
}

% AC-023-02
\begin{myverbbox}[\small]{\output}
$ cat network.json
{"rpcUrl": "https://sokol.poa.network", "gasPriceUrl": "https://gasprice.
poa.network/", "defaultGasPrice": 2000000000}
$ kyc.py --list add
No KYC registration requests found
$ cat person.json
{"id": "37da04e7-f471-49c7-a54c-a08f05950fc5"}
$ faceid.py --add 4590 +79991234567
Registration request sent by 0xa7f3239715ff731a3d6fc477b18e35b9b0a9e1ede8
4bca5e91517e8e5bf1cc69
$ cat person.json
{"id": "da04e377-47f1-c749-54ca-0fc5a08f0595"}
$ faceid.py --add 6104 +79220012534
Registration request sent by 0xc95d677eb6f3fcb55e08274ae0eed1970391e637f1
062426a3406fd7d4cfcfcb
$ rm person.json
$ kyc.py --list add
0x5bAD5c60781111094C247F81792eDDE9bb38818A: +79220012534
0xFCE1151f31065913F124917E0F2Ba5a6e29D6426: +79991234567
\end{myverbbox}
\acceptanceCriteria{Все два запроса на регистрацию соответствия, отправленные в контракт, адрес которого указан в файле \texttt{registrar.json}, отображены в выводе команды \texttt{kyc.py --list}. Сортировка вывода происходит по номеру телефона. Транзакции этой командой в сеть не отправляются.
}

% AC-023-03
\begin{myverbbox}[\small]{\output}
$ cat network.json
{"rpcUrl": "https://sokol.poa.network", "gasPriceUrl": "https://gasprice.
poa.network/", "defaultGasPrice": 2000000000}
$ kyc.py --list add
No KYC registration requests found
$ cat person.json
{"id": "37da04e7-f471-49c7-a54c-a08f05950fc5"}
$ faceid.py --add 4590 +79991234567
Registration request sent by 0xa7f3239715ff731a3d6fc477b18e35b9b0a9e1ede8
4bca5e91517e8e5bf1cc69
$ cat person.json
{"id": "da04e377-47f1-c749-54ca-0fc5a08f0595"}
$ faceid.py --add 6104 +79991234567
Registration request sent by 0xc95d677eb6f3fcb55e08274ae0eed1970391e637f1
062426a3406fd7d4cfcfcb
$ rm person.json
$ kyc.py --list add
0x5bAD5c60781111094C247F81792eDDE9bb38818A: +79991234567
0xFCE1151f31065913F124917E0F2Ba5a6e29D6426: +79991234567
\end{myverbbox}
\acceptanceCriteria{Два запроса на регистрацию соответствия, отправленные в контракт, адрес которого указан в файле \texttt{registrar.json}, приняты контрактом даже если они - для регистрации одного и того же номера телефона. Оба запроса отображены в выводе команды \texttt{kyc.py --list}. Сортировка вывода происходит сначала по номеру телефона, затем по адресу аккаунта. Транзакции этой командой в сеть не отправляются.
}

% AC-023-04
\begin{myverbbox}[\small]{\output}
$ cat network.json
{"rpcUrl": "https://sokol.poa.network", "gasPriceUrl": "https://gasprice.
poa.network/", "defaultGasPrice": 2000000000}
$ kyc.py --list add
No KYC registration requests found
$ cat person.json
{"id": "5069b18c-9f6d-42f9-aa5f-a5f6b924d87e"}
$ faceid.py --add 1096 +79125812224
Registration request sent by 0x01bef0afddd3ebc79e80c0385bf5e0bfea871b97d9
37c74462a3221094b44c1c
$ cat person.json
{"id": "37da04e7-f471-49c7-a54c-a08f05950fc5"}
$ faceid.py --add 4590 +79991234567
Registration request sent by 0xa7f3239715ff731a3d6fc477b18e35b9b0a9e1ede8
4bca5e91517e8e5bf1cc69
$ cat person.json
{"id": "da04e377-47f1-c749-54ca-0fc5a08f0595"}
$ faceid.py --add 6104 +79220012534
Registration request sent by 0xc95d677eb6f3fcb55e08274ae0eed1970391e637f1
062426a3406fd7d4cfcfcb
$ cat person.json
{"id": "5069b18c-9f6d-42f9-aa5f-a5f6b924d87e"}
$ faceid.py --cancel 1096
Registration canceled by 0x99a7e5f28b653d63b5c0bbddebc91678530ee461e4fa21
c9ab2f281d28b8da5e
$ rm person.json
$ kyc.py --list add
0x5bAD5c60781111094C247F81792eDDE9bb38818A: +79220012534
0xFCE1151f31065913F124917E0F2Ba5a6e29D6426: +79991234567
\end{myverbbox}
\acceptanceCriteria{Только два запроса на регистрацию соответствия, отправленные в контракт, адрес которого указан в файле \texttt{registrar.json}, отображены в выводе команды \texttt{kyc.py --list}. Отмененный запрос не отображается. Транзакции этой командой в сеть не отправляются.
}

% AC-023-05
\begin{myverbbox}[\small]{\output}
$ cat network.json
{"rpcUrl": "https://sokol.poa.network", "gasPriceUrl": "https://gasprice.
poa.network/", "defaultGasPrice": 2000000000}
$ kyc.py --list add
No KYC registration requests found
$ cat person.json
{"id": "37da04e7-f471-49c7-a54c-a08f05950fc5"}
$ faceid.py --add 4590 +79991234567
Registration request sent by 0xa7f3239715ff731a3d6fc477b18e35b9b0a9e1ede8
4bca5e91517e8e5bf1cc69
$ cat person.json
{"id": "da04e377-47f1-c749-54ca-0fc5a08f0595"}
$ faceid.py --add 6104 +79220012534
Registration request sent by 0xc95d677eb6f3fcb55e08274ae0eed1970391e637f1
062426a3406fd7d4cfcfcb
$ faceid.py --cancel 6104
Registration canceled by 0x99a7e5f28b653d63b5c0bbddebc91678530ee461e4fa21
c9ab2f281d28b8da5e
$ cat person.json
{"id": "37da04e7-f471-49c7-a54c-a08f05950fc5"}
$ faceid.py --cancel 4590
Registration canceled by 0xc90366c708b1c1e8462da76682338d5262a785d592918e
d915e1292ac0d6178a
$ rm person.json
$ kyc.py --list add
No KYC registration requests found
\end{myverbbox}
\acceptanceCriteria{Поскольку все запросы на регистрацию соответствий, оптравленные в контракт, адрес которого указан в файле \texttt{registrar.json}, были отменены, то в выводе команды \texttt{kyc.py --list} выдается соответствующее сообщение. Транзакции этой командой в сеть не отправляются.
}

% AC-023-06
\begin{myverbbox}[\small]{\output}
$ cat network.json
{"rpcUrl": "https://sokol.poa.network", "gasPriceUrl": "https://gasprice.
poa.network/", "defaultGasPrice": 2000000000}
$ kyc.py --list add
No KYC registration requests found
$ cat person.json
{"id": "37da04e7-f471-49c7-a54c-a08f05950fc5"}
$ faceid.py --add 4590 +79991234567
Registration request sent by 0xa7f3239715ff731a3d6fc477b18e35b9b0a9e1ede8
4bca5e91517e8e5bf1cc69
$ faceid.py --find /path/to/video23.avi
168f2e61-089f-88a9-53b2-3b3d0c497704 identified
$ faceid.py --del 5981
Unregistration request sent by 0xda0e2a124e6d37080b539f1bb0dc4c698b8025b6
2dcc56d75efc1115d87381aa
$ rm person.json
$ kyc.py --list add
0xFCE1151f31065913F124917E0F2Ba5a6e29D6426: +79991234567
\end{myverbbox}
\acceptanceCriteria{Только запросы на регистрацию соответствий, оптравленные в контракт, адрес которого указан в файле \texttt{registrar.json}, отображаются в выводе команды \texttt{kyc.py --list}. Транзакции этой командой в сеть не отправляются.
}

% AC-023-07
\begin{myverbbox}[\small]{\output}
$ cat registrar.json
cat: registrar.json: No such file or directory
$ kyc.py --list add
No contract address
\end{myverbbox}
\acceptanceCriteria{Выдается сообщение об ошибке, если в текущей директории нет файла \texttt{registrar.json}, содержащего адрес контракта регистра соответствий. Транзакция в блокчейн сеть не отправляется.
}

% AC-023-08
\begin{myverbbox}[\small]{\output}
$ cat registrar.json
{"registrar": {"address": "0x340Ec06460d9b2b7D23B40E5bCd0a81A09e06D81", "
startBlock": 456123}, "payments": {"address": "0x81A09e06D81797AE2b7D23B4
0E5bCd0a1da01eb0F951x", "startBlock": 456125}}
$ kyc.py --list add
Seems that the contract address is not the registrar contract
\end{myverbbox}
\acceptanceCriteria{Выдается сообщение об ошибке, если в адрес контракта, указанного в файле \texttt{registrar.json}, не принадлежит контракту регистра соответствий.
}

%-----------------------------------------------------
%US-024
\newuserstory{Получение всех запросов на удаление соответствий}


Любой пользователь сервиса,  может получить список всех запросов на удаление соответствий. 


\begin{myverbbox}[\small]{\cmdLine}
$ kyc.py --list del
\end{myverbbox}
\scriptExample{
Через RPC узел блокчейн сети отправляется запрос к контракту, адрес которого указан в поле \texttt{registrar} файла \texttt{registrar.json}, на получение всех неотменненных запросов на удаление соответствий. Запросы выводятся в формате \texttt{отправитель запроса: номер телефона}

}

% AC-024-01
\begin{myverbbox}[\small]{\output}
$ setup.py --deploy
KYC Registrar: 0x00360d2b7D240Ec0643B6D819ba81A09e40E5bCd
Payment Handler: 0x95426f2bC716022fCF1dEf006dbC4bB81f5B5164
$ cat registrar.json
{"registrar": {"address": "0x00360d2b7D240Ec0643B6D819ba81A09e40E5bCd", "
startBlock": 123456}, "payments": {"address": "0x95426f2bC716022fCF1dEf00
6dbC4bB81f5B5164", "startBlock": 123457}}
$ kyc.py --list del
No KYC unregistration requests found
\end{myverbbox}
\acceptanceCriteria{Поскольку в контракте, адрес которого указан в файле \texttt{registrar.json}, не было зарегистрировано ни одного запроса удаления соответствия аккаунта, то выдается соответствующее сообщение. Транзакции в сеть не отправляются.
}

% AC-024-02
\begin{myverbbox}[\small]{\output}
$ cat network.json
{"rpcUrl": "https://sokol.poa.network", "gasPriceUrl": "https://gasprice.
poa.network/", "defaultGasPrice": 2000000000}
$ kyc.py --list del
No KYC unregistration requests found
$ cat person.json
{"id": "37da04e7-f471-49c7-a54c-a08f05950fc5"}
$ faceid.py --del 4590
Unregistration request sent by 0x64e604787cbf194841e7b68d7cd28786f6c9a0a3
ab9f8b0a0e87cb4387ab0107
$ cat person.json
{"id": "da04e377-47f1-c749-54ca-0fc5a08f0595"}
$ faceid.py --del 6104
Unregistration request sent by 0xac09810740600c31fa69f9db79ed6fc3e3281f75
8a950fe1fb254a3a3ae571b6
$ rm person.json
$ kyc.py --list del
0x5bAD5c60781111094C247F81792eDDE9bb38818A: +79220012534
0xFCE1151f31065913F124917E0F2Ba5a6e29D6426: +79991234567
\end{myverbbox}
\acceptanceCriteria{Все два запроса на удаление соответствия, отправленные в контракт, адрес которого указан в файле \texttt{registrar.json}, отображены в выводе команды \texttt{kyc.py --list}. Сортировка вывода происходит по номеру телефона. Транзакции этой командой в сеть не отправляются.
}

% AC-024-03
\begin{myverbbox}[\small]{\output}
$ cat network.json
{"rpcUrl": "https://sokol.poa.network", "gasPriceUrl": "https://gasprice.
poa.network/", "defaultGasPrice": 2000000000}
$ kyc.py --list del
No KYC unregistration requests found
$ cat person.json
{"id": "5069b18c-9f6d-42f9-aa5f-a5f6b924d87e"}
$ faceid.py --del 1096
Unregistration request sent by 0x631897788617bc63fb0292c44f57a6c3824e4264
bc88fb4e3c08b910f3f417fc
$ cat person.json
{"id": "37da04e7-f471-49c7-a54c-a08f05950fc5"}
$ faceid.py --del 4590
Unregistration request sent by 0x7dbef654fd5c6145f82b72cd5dbbfc6d48fd276b
8a7a4371a61d55985b9d6d8b
$ cat person.json
{"id": "da04e377-47f1-c749-54ca-0fc5a08f0595"}
$ faceid.py --del 6104
Unregistration request sent by 0xf65e3ff0de05ca77aa8c820bd528facc139ee4b0
986f7c350f3198928bd2c72b
$ cat person.json
{"id": "5069b18c-9f6d-42f9-aa5f-a5f6b924d87e"}
$ faceid.py --cancel 1096
Unregistration canceled by 0x24036ccf201a67256250eabe66d3f9fd72f9c4d022f2
25a8ee964be060dcb993
$ rm person.json
$ kyc.py --list del
0x5bAD5c60781111094C247F81792eDDE9bb38818A: +79220012534
0xFCE1151f31065913F124917E0F2Ba5a6e29D6426: +79991234567
\end{myverbbox}
\acceptanceCriteria{Только два запроса на удаление соответствия, отправленные в контракт, адрес которого указан в файле \texttt{registrar.json}, отображены в выводе команды \texttt{kyc.py --list}. Отмененный запрос не отображается. Транзакции этой командой в сеть не отправляются.
}

% AC-024-04
\begin{myverbbox}[\small]{\output}
$ cat network.json
{"rpcUrl": "https://sokol.poa.network", "gasPriceUrl": "https://gasprice.
poa.network/", "defaultGasPrice": 2000000000}
$ kyc.py --list del
No KYC unregistration requests found
$ cat person.json
{"id": "37da04e7-f471-49c7-a54c-a08f05950fc5"}
$ faceid.py --del 4590
Unregistration request sent by 0x535306ee4b42c92aecd0e71fca98572064f049c2
babb2769faa3bbd87d67ec2d
$ cat person.json
{"id": "da04e377-47f1-c749-54ca-0fc5a08f0595"}
$ faceid.py --del 6104
Unregistration request sent by 0x425a3d65e6c6cbbf01507814af1ac7512f93cceb
8411381e1d5fb7adbfd44881
$ faceid.py --cancel 6104
Unregistration canceled by 0xd772c1d26a64e4114af99655ec353dc3dba1b12ba483
c0fc852e01d0432d3aa1
$ cat person.json
{"id": "37da04e7-f471-49c7-a54c-a08f05950fc5"}
$ faceid.py --cancel 4590
Unregistration canceled by 0xe7fcf89c34605f23590ff58513f1080dc375dd06cc5e
256320151057827a258a
$ rm person.json
$ kyc.py --list del
No KYC unregistration requests found
\end{myverbbox}
\acceptanceCriteria{Поскольку все запросы на удаление соответствий, оптравленные в контракт, адрес которого указан в файле \texttt{registrar.json}, были отменены, то в выводе команды \texttt{kyc.py --list} выдается соответствующее сообщение. Транзакции этой командой в сеть не отправляются.
}

% AC-024-05
\begin{myverbbox}[\small]{\output}
$ cat network.json
{"rpcUrl": "https://sokol.poa.network", "gasPriceUrl": "https://gasprice.
poa.network/", "defaultGasPrice": 2000000000}
$ kyc.py --list del
No KYC unregistration requests found
$ faceid.py --find /path/to/video23.avi
168f2e61-089f-88a9-53b2-3b3d0c497704 identified
$ faceid.py --del 5981
Unregistration request sent by 0xa413d075ec804ca282f1637fedbe82ecda199209
7d14e4d900d254e0f5fe523d
$ faceid.py --find /path/to/video24.avi
37da04e7-f471-49c7-a54c-a08f05950fc5 identified
$ faceid.py --add 4590 +79991234567
Registration request sent by 0x2f5ccb30fa919306cd2899a81c7670a9b1be09e8c1
5a0b78b0fbbc777d15e97b
$ rm person.json
$ kyc.py --list del
0x7F3faAe3f2238439d6ea5A320caB969aC62b68e3: +79870194581
\end{myverbbox}
\acceptanceCriteria{Только запросы на удаление соответствий, оптравленные в контракт, адрес которого указан в файле \texttt{registrar.json}, отображаются в выводе команды \texttt{kyc.py --list}. Транзакции этой командой в сеть не отправляются.
}

% AC-024-06
\begin{myverbbox}[\small]{\output}
$ cat registrar.json
cat: registrar.json: No such file or directory
$ kyc.py --list del
No contract address
\end{myverbbox}
\acceptanceCriteria{Выдается сообщение об ошибке, если в текущей директории нет файла \texttt{registrar.json}, содержащего адрес контракта регистра соответствий. Транзакция в блокчейн сеть не отправляется.
}

% AC-024-07
\begin{myverbbox}[\small]{\output}
$ cat registrar.json
{"registrar": {"address": "0x340Ec06460d9b2b7D23B40E5bCd0a81A09e06D81", "
startBlock": 456123}, "payments": {"address": "0x81A09e06D81797AE2b7D23B4
0E5bCd0a1da01eb0F951x", "startBlock": 456125}}
$ kyc.py --list del
Seems that the contract address is not the registrar contract
\end{myverbbox}
\acceptanceCriteria{Выдается сообщение об ошибке, если в адрес контракта, указанного в файле \texttt{registrar.json}, не принадлежит контракту регистра соответствий.
}

%-----------------------------------------------------
%US-025
\newuserstory{Подтверждение запросов на регистрацию или удаление соответствий}


Только администратор KYC сервиса имеет полномочия подтверждать запросы на регистрацию или удаление соответствий аккаунтов пользователей их телефонным номерам. 


\begin{myverbbox}[\small]{\cmdLine}
$ kyc.py --confirm <address>
\end{myverbbox}
\scriptExample{
В блокчейн сеть отправляется транзакция к контракту регистра соответствий. Контракт проверяет обладает ли отправитель транзакции полномочиями по подтверждению запросов, после чего происходит изменение статуса запроса:
\begin{itemize}
  \item если запрос был на регистрацию соответствия между аккаунтом и телефонным номером, то это соответствие становится доступным для других пользователей;
  \item если запрос был на удаление соответствия, то соответствие перестает существовать - по номеру телефона, который указывался в соответствии, больше нельзя будет получить адрес аккаунта.
\end{itemize}

}

% AC-025-01
\begin{myverbbox}[\small]{\output}
$ cat network.json | python -mjson.tool | grep privKey 
    "privKey": "c5d2460186f7233c927e7db2dcc703c0e500b653ca82273b7bfad8045
d85a470",
$ setup.py --owner registrar
Admin account: 0x9cce34F7aB185c7ABA1b7C8140d620B4BDA941d6
$ kyc.py --list add
0xFCE1151f31065913F124917E0F2Ba5a6e29D6426: +79991234567
$ kyc.py --confirm 0xFCE1151f31065913F124917E0F2Ba5a6e29D6426
Confirmed by 0x9960ed3041b9945289da338fa273462c820f58427408d57789c0e1400a
d5c9bb
$ kyc.py --list add
No KYC registration requests found
\end{myverbbox}
\acceptanceCriteria{Поскольку аккаунт, чей приватный ключ указан в \texttt{network.json}, обладает полномочиями на подтверждение запросов на регистрацию соответствий, посланная транзакция включается в блок со статусом успешного исполнения. Статус можно подтвердить для данной транзакции в браузере блоков.

При подтверждении контракт производит событие (\texttt{event}) \texttt{RegistrationConfirmed}, в с указанием аккаунта, отправившего запрос:

\texttt{event RegistrationConfirmed(address indexed sender);}


Для проведения транзакции выбрана цена из значения \texttt{fast}, возвращенного сервисом \texttt{https://gasprice.poa.network}.
}

% AC-025-02
\begin{myverbbox}[\small]{\output}
$ cat network.json | python -mjson.tool | grep privKey 
    "privKey": "6fadbaf86d2aa9c9e34a85df385a2c9afa9509e09756634e72bbcf94a
5ceb213",
$ setup.py --owner registrar
Admin account: 0x5e4d710a4995bFA3F6560effcad36C10ebac998C
$ kyc.py --list add
0xFCE1151f31065913F124917E0F2Ba5a6e29D6426: +79991234567
$ kyc.py --confirm 0xFCE1151f31065913F124917E0F2Ba5a6e29D6426
Failed but included in 0x9960ed3041b9945289da338fa273462c820f58427408d577
89c0e1400ad5c9bb
$ kyc.py --list add
0xFCE1151f31065913F124917E0F2Ba5a6e29D6426: +79991234567
\end{myverbbox}
\acceptanceCriteria{Поскольку аккаунт, чей приватный ключ указан в \texttt{network.json}, не обладает полномочиями на подтверждение запросов на регистрацию соответствий, посланная транзакция включается в блок, но статус ее исполнения - ошибка. Статус можно подтвердить для данной транзакции в браузере блоков.
}

% AC-025-03
\begin{myverbbox}[\small]{\output}
$ cat network.json | python -mjson.tool | grep privKey 
    "privKey": "c5d2460186f7233c927e7db2dcc703c0e500b653ca82273b7bfad8045
d85a470",
$ setup.py --owner registrar
Admin account: 0x9cce34F7aB185c7ABA1b7C8140d620B4BDA941d6
$ kyc.py --list del
0x32276b955E7dCBeA1C97fe8f06053E760E739e8d: +79010451275
$ kyc.py --confirm 0x32276b955E7dCBeA1C97fe8f06053E760E739e8d
Confirmed by 0xbdf07588e6652aaee28b29156c2ae266c9ba46e253ddd99b466cce31f7
7c51a3
$ kyc.py --list del
No KYC unregistration requests found
\end{myverbbox}
\acceptanceCriteria{Поскольку аккаунт, чей приватный ключ указан в \texttt{network.json}, обладает полномочиями на подтверждение запросов на удаление соответствий, посланная транзакция включается в блок со статусом успешного исполнения. Статус можно подтвердить для данной транзакции в браузере блоков.

При подтверждении контракт производит событие (\texttt{event}) \texttt{UnregistrationConfirmed}, в с указанием аккаунта, отправившего запрос:

\texttt{event UnregistrationConfirmed(address indexed sender);}


Для проведения транзакции выбрана цена из значения \texttt{fast}, возвращенного сервисом \texttt{https://gasprice.poa.network}.
}

% AC-025-04
\begin{myverbbox}[\small]{\output}
$ cat network.json | python -mjson.tool | grep privKey 
    "privKey": "6fadbaf86d2aa9c9e34a85df385a2c9afa9509e09756634e72bbcf94a
5ceb213",
$ setup.py --owner registrar
Admin account: 0x5e4d710a4995bFA3F6560effcad36C10ebac998C
$ kyc.py --list del
0x32276b955E7dCBeA1C97fe8f06053E760E739e8d: +79010451275
$ kyc.py --confirm 0x32276b955E7dCBeA1C97fe8f06053E760E739e8d
Failed but included in 0x360f73c63c063fae5d1e9ead454a4836ec470b54de0960f0
5909b7f8c7c140ba
$ kyc.py --list del
0x32276b955E7dCBeA1C97fe8f06053E760E739e8d: +79010451275
\end{myverbbox}
\acceptanceCriteria{Поскольку аккаунт, чей приватный ключ указан в \texttt{network.json}, не обладает полномочиями на подтверждение запросов на удаление соответствий, посланная транзакция включается в блок, но статус ее исполнения - ошибка. Статус можно подтвердить для данной транзакции в браузере блоков.
}

% AC-025-05
\begin{myverbbox}[\small]{\output}
$ kyc.py --list add
0xFCE1151f31065913F124917E0F2Ba5a6e29D6426: +79991234567
$ faceid.py --cancel 6104
Registration canceled by 0x26a975632f70533c9512e57b565b8ce8e2befee6b08dfb
c27c4a897a20e2c1d6
$ kyc.py --list add
No KYC registration requests found
$ kyc.py --confirm 0xFCE1151f31065913F124917E0F2Ba5a6e29D6426
Failed but included in 0xd534ca6cd7b201fa32702a1b5b6a38881f8a63e70f9d3e1a
8a953d61c6524405
\end{myverbbox}
\acceptanceCriteria{Транзакция, посланная в блокчейн сеть, на подвтерждение запроса регистрации соответствия включается в блок со статусом успешного исполнения. Статус можно подтвердить для данной транзакции в браузере блоков. Поскольку успешного подтверждения запрос теряет актуальность, то его нельзя в дальнейшем отменить.
}

% AC-025-07
\begin{myverbbox}[\small]{\output}
$ kyc.py --list del
0xFCE1151f31065913F124917E0F2Ba5a6e29D6426: +79991234567
$ faceid.py --cancel 6104
Unregistration canceled by 0xcfe93e00cbaaf37bad5f0b2ff3ab98072b4f3a2656e1
48b3bcccb553f2fa678f
$ kyc.py --list del
No KYC unregistration requests found
$ kyc.py --confirm 0xFCE1151f31065913F124917E0F2Ba5a6e29D6426
Failed but included in 0x890f693d9ea1b595bb4a33efee4a21d60a63ffd2ec40c378
83b92f82fc5da52b
\end{myverbbox}
\acceptanceCriteria{Аккаунт, чей приватный ключ указан в \texttt{network.json}, обладает полномочиями на подтверждение запросов на удаление соответствий. А поскольку запрос на удаление соответствия был отменен, посланная транзакция на подтерждение удаления включается в блок, но статус ее исполнения - ошибка. Статус можно подтвердить для данной транзакции в браузере блоков.
}

% AC-025-08
\begin{myverbbox}[\small]{\output}
$ kyc.py --list del
0xFCE1151f31065913F124917E0F2Ba5a6e29D6426: +79991234567
$ kyc.py --confirm 0xFCE1151f31065913F124917E0F2Ba5a6e29D6426
Confirmed by 0x7b0ac78d4efde931c083b037c1c7f91b4cc8f27a159c1264ae53aa473f
3e4e6a
$ faceid.py --cancel 6104
No requests found
\end{myverbbox}
\acceptanceCriteria{Транзакция, посланная в блокчейн сеть, на подвтерждение запроса удаления соответствия включается в блок со статусом успешного исполнения. Статус можно подтвердить для данной транзакции в браузере блоков. Поскольку успешного подтверждения запрос теряет актуальность, то его нельзя в дальнейшем отменить.
}

% AC-025-09
\begin{myverbbox}[\small]{\output}
$ kyc.py --list add
0xFCE1151f31065913F124917E0F2Ba5a6e29D6426: +79991234567
$ kyc.py --confirm 0xFCE1151f31065913F124917E0F2Ba5a6e29D6426
Confirmed by 0x2b2a51e6e75ece6beb4c5ed9d51d381a6766e4929f34c8c545b0c0f22a
890e3d
$ faceid.py --del 6104
Unregistration request sent by 0x5a10742dc65b7a28255fe7b4d2b9f290786892e1
e437654e2ccb53bed1bfe01a
\end{myverbbox}
\acceptanceCriteria{Транзакция, посланная в блокчейн сеть, на подвтерждение запроса регистрации соответствия включается в блок со статусом успешного исполнения. Статус можно подтвердить для данной транзакции в браузере блоков. Теперь для установленного соответствия есть возможность отправить запрос на его удаление.
}

% AC-025-10
\begin{myverbbox}[\small]{\output}
$ kyc.py --list del
0xFCE1151f31065913F124917E0F2Ba5a6e29D6426: +79991234567
$ kyc.py --confirm 0xFCE1151f31065913F124917E0F2Ba5a6e29D6426
Confirmed by 0xe3cb0ca89f217511efd4af6caea80048f9921bac3b9063844e65469cf5
15f138
$ faceid.py --add 6104 +79991234567
Registration request sent by 0x41c2cc0c9e11501a3eec0956812949b1b9b2d14d9d
c7118df2ef1aac71f18767
\end{myverbbox}
\acceptanceCriteria{Транзакция, посланная в блокчейн сеть, на подвтерждение запроса удаления соответствия включается в блок со статусом успешного исполнения. Статус можно подтвердить для данной транзакции в браузере блоков. Поскольку у данного аккаунта больше нет актуального соответствия, то есть возможность отправить запрос на создание нового соответствия.
}

% AC-025-11
\begin{myverbbox}[\small]{\output}
$ cat person.json
{"id": "37da04e7-f471-49c7-a54c-a08f05950fc5"}
$ faceid.py --add 4590 +79991234567
Registration request sent by 0xb004fbd84afb38927636fa378c0a62a86a02b00862
b6ff80fef4d6e948c0571d
$ kyc.py --confirm 0x5bAD5c60781111094C247F81792eDDE9bb38818A
Confirmed by 0x06a79e67d636d828aa58884bc7fed897698bb40922db4eadf708235cd9
c2de2c
$ cat person.json
{"id": "da04e377-47f1-c749-54ca-0fc5a08f0595"}
$ faceid.py --add 6104 +79991234567
Such phone number already registered
\end{myverbbox}
\acceptanceCriteria{При запросе на регистрацию соответствия выводится сообщение об ошибке, поскольку такой номер телефона уже зарегистрирован в системе. Транзакция в блокчейн не отправляется.
}

% AC-025-12
\begin{myverbbox}[\small]{\output}
$ cat person.json
{"id": "37da04e7-f471-49c7-a54c-a08f05950fc5"}
$ faceid.py --add 4590 +79991234567
Registration request sent by 0xb004fbd84afb38927636fa378c0a62a86a02b00862
b6ff80fef4d6e948c0571d
$ kyc.py --confirm 0x5bAD5c60781111094C247F81792eDDE9bb38818A
Confirmed by 0x06a79e67d636d828aa58884bc7fed897698bb40922db4eadf708235cd9
c2de2c
\end{myverbbox}
\acceptanceCriteria{Если из транзакции c идентификатором \texttt{0xb004fbd84afb38927636fa378c0a62a86a02b00862b6ff80fef4d6e948c0571d} извлечь поле \texttt{input} и отправить его в поле \texttt{input} новой транзакции с аккаунта, который отличется от аккаунта, отправившего эту транзакцию, на адрес контракта регистра соответствий, то эта транзакция будет включена в блок, но статус ее исполнения будет - ошибка, поскольку запрашиваемый в данных телефонный номер уже зарегистрирован в системе. Статус можно подтвердить для данной транзакции в браузере блоков.
}

% AC-025-13
\begin{myverbbox}[\small]{\output}
$ cat person.json
{"id": "37da04e7-f471-49c7-a54c-a08f05950fc5"}
$ faceid.py --add 4590 +79991234567
Registration request sent by 0xb004fbd84afb38927636fa378c0a62a86a02b00862
b6ff80fef4d6e948c0571d
$ cat person.json
{"id": "229b3307-fe57-4ddc-b7e7-ac9f1de7abda"}
$ faceid.py --add 3928 +79991234567
Registration request sent by 0x0330c0d84464f890f3c81afbba7a3a05e0e864ec2e
31676dedcb075b2bf855a0
$ kyc.py --confirm 0x5bAD5c60781111094C247F81792eDDE9bb38818A
Confirmed by 0x06a79e67d636d828aa58884bc7fed897698bb40922db4eadf708235cd9
c2de2c
\end{myverbbox}
\acceptanceCriteria{Если из транзакции, которая была отправлена в результате команды \texttt{kyc.py --confirm}, извлечь поле \texttt{input} и отправить его в новой транзакции снова в поле \texttt{input} на адрес контракта регистра соответствий, то эта транзакция будет включена в блок, но статус ее исполнения будет - ошибка по причине того, что такое соответствие уже подтверждено. Статус можно подтвердить для данной транзакции в браузере блоков.
}

% AC-025-14
\begin{myverbbox}[\small]{\output}
$ kyc.py --list add
0xFf8E5C14d240d76EAe66aEF1A3a99895e08a0c3F: +79271173070
0xa898240Ba2D5C0C2537BC187CaC5B70843EAC8dD: +79397411449
0x60861DE06626f60d89f5795AD950B502CE00A8B0: +79017422129
$ kyc.py --list del
0x5bAD5c60781111094C247F81792eDDE9bb38818A: +79220012534
0xFCE1151f31065913F124917E0F2Ba5a6e29D6426: +79991234567
$ kyc.py --confirm 0x60861DE06626f60d89f5795AD950B502CE00A8B1
Failed but included in 0x18e28784ed3043bccd713a5f99352b74c3ebc4ca469216a6
86040f025571a184
\end{myverbbox}
\acceptanceCriteria{Поскольку запрос, который пытается подтвердиться, не существует, то посланная транзакция включается в блок, но статус ее исполнения - ошибка. Статус можно подтвердить для данной транзакции в браузере блоков.
}

% AC-025-15
\begin{myverbbox}[\small]{\output}
$ cat person.json
{"id": "da04e377-47f1-c749-54ca-0fc5a08f0595"}
$ faceid.py --add 6104 +79220012534
Registration request sent by 0xa65d03add95baf22e482ebcd3423aa9bf9d8b0ec3f
ca2831960080b84536d360
$ kyc.py --confirm 0x5bAD5c60781111094C247F81792eDDE9bb38818A
Confirmed by 0x05dee4192aaad4acb149f9615e8a72a761fc8cdb1d8b4e9c00fe66665f
5a8184
$ faceid.py --del 6104
Unregistration request sent by 0x1b6edb65b6c2611a7285cc668f4110f574a7bc3b
0d28690b40a16f141eada39a
$ kyc.py --confirm 0x5bAD5c60781111094C247F81792eDDE9bb38818A
Confirmed by 0x1c6c2dd4b412e3c82ad1f10cef9a1a4112bc1a128dc4d12242d781aa03
ccbf2c
\end{myverbbox}
\acceptanceCriteria{Если из транзакции с идентификатором \texttt{0x1b6edb65b6c2611a7285cc668f4110f574a7bc3b0d28690b40a16f141eada39a}, извлечь поле \texttt{input} и отправить его в новой транзакции снова в поле \texttt{input} на адрес контракта регистра соответствий, то эта транзакция будет включена в блок, но статус ее исполнения будет - ошибка поскольку данный аккаунт больше не зарегистрирован в регистре соответствий. Статус можно подтвердить для данной транзакции в браузере блоков.
}

% AC-025-16
\begin{myverbbox}[\small]{\output}
$ cat registrar.json
cat: registrar.json: No such file or directory
$ kyc.py --confirm 0x60861DE06626f60d89f5795AD950B502CE00A8B0
No contract address
\end{myverbbox}
\acceptanceCriteria{Выдается сообщение об ошибке, если в текущей директории нет файла \texttt{registrar.json}, содержащего адрес контракта регистра соответствий. Транзакция в блокчейн сеть не отправляется.
}

% AC-025-17
\begin{myverbbox}[\small]{\output}
$ cat registrar.json
{"registrar": {"address": "0x340Ec06460d9b2b7D23B40E5bCd0a81A09e06D81", "
startBlock": 456123}, "payments": {"address": "0x81A09e06D81797AE2b7D23B4
0E5bCd0a1da01eb0F951x", "startBlock": 456125}}
$ kyc.py --confirm 0x60861DE06626f60d89f5795AD950B502CE00A8B0
Seems that the contract address is not the registrar contract
\end{myverbbox}
\acceptanceCriteria{Выдается сообщение об ошибке, если в адрес контракта, указанного в файле \texttt{registrar.json}, не принадлежит контракту регистра соответствий.
}

% AC-025-18
\begin{myverbbox}[\small]{\output}
$ cat network.json
{"rpcUrl": "https://sokol.poa.network", "gasPriceUrl": "https://gasprice.
poa.network/", "defaultGasPrice": 2000000000}
$ kyc.py --confirm 0x60861DE06626f60d89f5795AD950B502CE00A8B0
No admin account found
\end{myverbbox}
\acceptanceCriteria{Выдается сообщение об ошибке, если в \texttt{network.json} нет приватного ключа аккаунта, от имени которого выполнялось бы подтверждение.
}

%-----------------------------------------------------
%US-026
\newuserstory{Получение аккаунта по номеру телефона}


Любой пользователь по номеру телефона может получить аккаунт соответствующий номеру телефона, если такое соответствие было зарегистрировано.


\begin{myverbbox}[\small]{\cmdLine}
$ kyc.py --get <phone number>
\end{myverbbox}
\scriptExample{
Через RPC узел блокчейн сети отправляется запрос к контракту, адрес которого указан в поле \texttt{registrar} файла \texttt{registrar.json}, на получение соответствия. Если соответствие не зарегистрировано, контракт возвращает \texttt{0x000000000000000000000000000000000000000}.

}

% AC-026-01
\begin{myverbbox}[\small]{\output}
$ setup.py --deploy
KYC Registrar: 0x23B40E5bCd06D819ba81A09e0340Ec06460d2b7D
Payment Handler: 0xE797A1da01eb0F951E0E400f9343De9d17A06bac
$ kyc.py --get +79017422129
Correspondence not found
\end{myverbbox}
\acceptanceCriteria{Поскольку в контракте не было зарегистированных соответствий, то соответствующее сообщение выводится на экран.
}

% AC-026-02
\begin{myverbbox}[\small]{\output}
$ cat person.json
{"id": "37da04e7-f471-49c7-a54c-a08f05950fc5"}
$ faceid.py --add 4590 +79991234567
Registration request sent by 0xb004fbd84afb38927636fa378c0a62a86a02b00862
b6ff80fef4d6e948c0571d
$ kyc.py --confirm 0x5bAD5c60781111094C247F81792eDDE9bb38818A
Confirmed by 0x06a79e67d636d828aa58884bc7fed897698bb40922db4eadf708235cd9
c2de2c
$ cat person.json
{"id": "229b3307-fe57-4ddc-b7e7-ac9f1de7abda"}
$ faceid.py --add 3928 +79418552734
Registration request sent by 0x0330c0d84464f890f3c81afbba7a3a05e0e864ec2e
31676dedcb075b2bf855a0
$ kyc.py --confirm 0xdCaBFD5c76D3717567568710a735717bf0C792De
Confirmed by 0x840881693295f1213bc621234bc18aa1727c90c9e1416799df22e18f9c
7c029b
$ rm person.rm
$ cat network.json
{"rpcUrl": "https://sokol.poa.network", "gasPriceUrl": "https://gasprice.
poa.network/", "defaultGasPrice": 2000000000}
$ kyc.py --get +79991234567
Registered correspondence: 0x5bAD5c60781111094C247F81792eDDE9bb38818A
\end{myverbbox}
\acceptanceCriteria{Для одно из зарегистированных соответствий выводится аккаунт, с которым связан указанных номер телефона. Транзакция в сеть не отправляется.
}

% AC-026-03
\begin{myverbbox}[\small]{\output}
$ cat person.json
{"id": "37da04e7-f471-49c7-a54c-a08f05950fc5"}
$ faceid.py --add 4590 +79991234567
Registration request sent by 0xb004fbd84afb38927636fa378c0a62a86a02b00862
b6ff80fef4d6e948c0571d
$ kyc.py --confirm 0x5bAD5c60781111094C247F81792eDDE9bb38818A
Confirmed by 0x06a79e67d636d828aa58884bc7fed897698bb40922db4eadf708235cd9
c2de2c
$ kyc.py --get +79991234567
Registered correspondence: 0x5bAD5c60781111094C247F81792eDDE9bb38818A
$ faceid.py --del 4590
Unregistration request sent by 0xc7d00ca82740de2eef417cfa47c2970dbdec49b2
a6d15c64e85f83cf97939ca5
$ kyc.py --confirm 0x5bAD5c60781111094C247F81792eDDE9bb38818A
Confirmed by 0x0697183ae8eedd43b56f90cfa450ea6e3ac27fb69f1056ee33ccc5a958
7bcd2a
$ kyc.py --get +79991234567
Correspondence not found
\end{myverbbox}
\acceptanceCriteria{В первом случае, соответствие зарегистрировано и поэтому вызов команды позволяет получить аккаунт по номеру телефона. К моменты вызова команды второй раз, соответствие удалено, поэтому выводится сообщение об ошибке.
}

% AC-026-04
\begin{myverbbox}[\small]{\output}
$ cat person.json
{"id": "37da04e7-f471-49c7-a54c-a08f05950fc5"}
$ faceid.py --add 4590 +79991234567
Registration request sent by 0xb004fbd84afb38927636fa378c0a62a86a02b00862
b6ff80fef4d6e948c0571d
$ kyc.py --get +79991234567
Correspondence not found
\end{myverbbox}
\acceptanceCriteria{Хотя регистрация соответствия только была запрошена, оно не было подтверждено, поэтому выводится сообщение об ошибке.
}

% AC-026-05
\begin{myverbbox}[\small]{\output}
$ cat person.json
{"id": "37da04e7-f471-49c7-a54c-a08f05950fc5"}
$ faceid.py --add 4590 +79991234567
Registration request sent by 0xb004fbd84afb38927636fa378c0a62a86a02b00862
b6ff80fef4d6e948c0571d
$ kyc.py --confirm 0x5bAD5c60781111094C247F81792eDDE9bb38818A
Confirmed by 0x06a79e67d636d828aa58884bc7fed897698bb40922db4eadf708235cd9
c2de2c
$ faceid.py --del 4590
Unregistration request sent by 0xc7d00ca82740de2eef417cfa47c2970dbdec49b2
a6d15c64e85f83cf97939ca5
$ kyc.py --get +79991234567
Registered correspondence: 0x5bAD5c60781111094C247F81792eDDE9bb38818A
\end{myverbbox}
\acceptanceCriteria{Поскольку удаление соответствия не было было подтверждено, поэтому выводится информация об аккаунте.
}

% AC-026-06
\begin{myverbbox}[\small]{\output}
$ cat person.json
{"id": "37da04e7-f471-49c7-a54c-a08f05950fc5"}
$ faceid.py --add 4590 +79991234567
Registration request sent by 0xb004fbd84afb38927636fa378c0a62a86a02b00862
b6ff80fef4d6e948c0571d
$ cat person.json
{"id": "229b3307-fe57-4ddc-b7e7-ac9f1de7abda"}
$ faceid.py --add 3928 +79991234567
Registration request sent by 0x0330c0d84464f890f3c81afbba7a3a05e0e864ec2e
31676dedcb075b2bf855a0
$ kyc.py --confirm 0x5bAD5c60781111094C247F81792eDDE9bb38818A
Confirmed by 0x06a79e67d636d828aa58884bc7fed897698bb40922db4eadf708235cd9
c2de2c
$ kyc.py --get +79991234567
Registered correspondence: 0x5bAD5c60781111094C247F81792eDDE9bb38818A
\end{myverbbox}
\acceptanceCriteria{Регистрация соответствия первого аккаунта была подтверждена, регистрация второго аккаунта не была подтверждена, поэтому выводится информация об аккаунте, отправившем первый запрос на регистрацию.
}

% AC-026-07
\begin{myverbbox}[\small]{\output}
$ cat registrar.json
cat: registrar.json: No such file or directory
$ kyc.py --get +79991234567
No contract address
\end{myverbbox}
\acceptanceCriteria{Выдается сообщение об ошибке, если в текущей директории нет файла \texttt{registrar.json}, содержащего адрес контракта регистра соответствий. Транзакция в блокчейн сеть не отправляется.
}

% AC-026-08
\begin{myverbbox}[\small]{\output}
$ cat registrar.json
{"registrar": {"address": "0x340Ec06460d9b2b7D23B40E5bCd0a81A09e06D81", "
startBlock": 456123}, "payments": {"address": "0x81A09e06D81797AE2b7D23B4
0E5bCd0a1da01eb0F951x", "startBlock": 456125}}
$ kyc.py --get +79991234567
Seems that the contract address is not the registrar contract
\end{myverbbox}
\acceptanceCriteria{Выдается сообщение об ошибке, если в адрес контракта, указанного в файле \texttt{registrar.json}, не принадлежит контракту регистра соответствий.
}