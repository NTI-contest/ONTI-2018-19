Решение командной задачи разбито на подзадачи, сгруппированные в 3 набора.

Каждая подзадача (\textit{user story}) формулирует необходимый функционал,
который должен быть реализован командой, а также набор приемочных
тестов (\textit{acceptance criteria}), позволяющих проверить в полном ли
объеме решена данная подзадача.

Каждый набор подзадач предлагается решать в отдельном этапе (\textit{итерации}).
Подзадачи в первом наборе позволяют школьникам продемонстрировать базовое понимание
принципов решения данной задачи. Они могут быть решены с использованием тех разработок,
котороые были реализованы участниками финала на учебно-тренировочных сборах.
Поздадачи второй итерации позволяют реализовать минимально рабочий продукт.
Третья итерация состоит из подзадач повышенной сложности, которые требуют от
старшеклассников продемонстрировать более глубокое понимание принципов написания
контрактов для сети блокчейн, а также использовать знания из олимпиады по
программированию для оптимизации работы с видео-потоком.

На каждом этапе решение подзадач проверялось с помощью системы автоматической
системы оценивания, которая запускала решение участников с теми или иными
параметрами в соответствии с приемочными тестами. 

\subsection*{Первая итерация}

Участникам необходимо разработать базовый функционал для скриптов \texttt{setup.py}, 
\texttt{face-management.py} и \texttt{faceid.py}, реализовать соответствующие
Ethereum-контракты, тем самым решая следующие подзадачи:

\footnotesize
\begin{center}
\begin{tabular}{ |l|c|  }
 \hline
  \thead{user story (*)} & \thead{acceptance \\ criteria} \\
 \hline

US-001 Регистрация контракта & все \\
US-004 Простое добавление пользователя в сервис индентификации & все \\
US-013 Получение баланса идентифицированного пользователя & все \\

 \hline
\end{tabular}
\end{center}
\begin{flushright}
\textit{(*) формулировки задач приведены в разделе ''Подробное описание подзадач''.}
\end{flushright}
\normalsize

Решение подзадач было направлено на проверку следующих компетенций:
\begin{itemize}
    \item инициализация аккаунтов тестовой публичной сети Ethereum для тестирования
    решения в ходе процесса разработки;
    \item настройка окружения разработки программного обеспечения для использования
    системы ведения версий \textit{Git}, Python-библиотек для взаимодействия с
    узлами блокчейн платформы Ethereum и \textit{Microsoft Face API};
    \item написание и отладка программ на языке Python, работающих с параметрами
    командной строки;
    \item обеспечение отправки JSON-RPC запросов (платежные операции) из программ
    на языке Python к провайдерам \texttt{Web3};
    \item написание и отладка Ethereum контрактов на языке Solidity;
    \item подготовка тестовых данных --- видео-файлов для работы с сервисом
    распознавания человека по лицу;
    \item получение отдельных кадров из видео-потока; 
    \item обеспечение отправки REST запросов к сервису \textit{Microsoft Face API}
    из программ на языке Python.
\end{itemize}

\subsection*{Вторая итерация}

Участникам необходимо завершить работу над минимально достаточным прототипом
программного обеспечения банкомата, который бы позволял
\begin{itemize}
    \item идентифицировать человека по лицу;
    \item управлять соответствиями между аккаунтом пользователя и его телефонным номером;
    \item отправлять средства другому зарегистрированному в системе пользователю.
\end{itemize}
Данный функционал покрывался следующими подзадачами:

\footnotesize
\begin{center}
\begin{tabular}{ |l|c|  }
 \hline
  \thead{user story (*)} & \thead{acceptance \\ criteria} \\
 \hline

US-002 Вывод владельца контракта регистра соответствий & все \\
US-003 Изменение владельца контракта регистра соответствий & все \\
US-006 Получение всех пользователей из сервиса идентификации & все \\
US-007 Удаление пользователя из сервиса идентификации & все \\
US-008 Запуск обучения сервиса индентификации & \makecell{все, кроме \\ AC-008-04} \\
US-010 Идентификация пользователя & \makecell{все, кроме \\ AC-010-04} \\
US-014 Отправка запроса на регистрацию соответствия & все \\
US-015 Отправка запроса на удаление соответствия & все \\
US-016 Отмена запроса на регистрацию или удаление соответствия & все \\
US-017 Отправка средств & все \\
US-021 Получение истории платежей & все \\
US-023 Получение всех запросов на регистрацию соответствий & все \\
US-024 Получение всех запросов на удаление соответствий & все \\
US-025 Подтверждение запросов на регистрацию или удаление соответствий & все \\
US-026 Получение аккаунта по номеру телефона & все \\

 \hline
\end{tabular}
\end{center}
\begin{flushright}
\textit{(*) формулировки задач приведены в разделе ''Подробное описание подзадач''.}
\end{flushright}
\normalsize

Решение подзадач было направлено на проверку следующих компетенций:
\begin{itemize}
    \item взаимодействие с сервисом распознавания человека по лицу для наполнения
    его необходимыми данными, обучения и выполнения операций распознавания; 
    \item разработка контракта с простейшей моделью разграничения доступа;
    \item разработка схемы взаимодействия структур данных для контракта с
    заданными характеристиками;
    \item обеспечение отправки JSON-RPC запросов (транзакции на вызов методов
    Ethereum контрактов, получение данных из блокчейн сети) из программ на языке Python
    к провайдерам \texttt{Web3};
    \item выстраивание процесса интеграционного тестирования компонент приложения,
    работающего с сервисом распознавания лица и узлами блокчейн сети.
\end{itemize}

\subsection*{Третья итерация}

Работа участников должна быть направлена на наращивание функционала прототипа
программного обеспечения банкомата. В ходе третьей итерации должна была появиться
защита от мошенничества при операциях распознавания по лицу и возможность
формировать отложенные платежи.
После заверешения итерации у участников должно быть 3 Python-скрипта и
2 Ethereum-контракта, закрывающие последнии набор подзадач:

\footnotesize
\begin{center}
\begin{tabular}{ |l|  }
 \hline
  \thead{user story (*), все acceptance criteria} \\
 \hline

US-005 Улучшенное добавление пользователя в сервис индентификации \\
US-008 Запуск обучения сервиса индентификации \\
US-009 Обнаружение уже добавленного пользователя \\
US-010 Идентификация пользователя \\
US-011 Запрос действий на безопасную идентификацию пользователя \\
US-012 Безопасная идентификация пользователя \\
US-018 Генерация сертификата на получение средств \\
US-019 Использование сертификата на получение средств \\
US-020 Вернуть средства из неиспользованных сертификатов \\
US-022 Получение истории платежей, включая использование сертификатов \\

 \hline
\end{tabular}
\end{center}
\begin{flushright}
\textit{(*) формулировки задач приведены в разделе ''Подробное описание подзадач''.}
\end{flushright}
\normalsize

Решение подзадач было направлено на проверку следующих компетенций:
\begin{itemize}
    \item оптимизация работы c видео-потоком для уменьшения количества
    обрабатываемых кадров;
    \item проверки истинности информации с использованием цифровой подписи.
\end{itemize}

