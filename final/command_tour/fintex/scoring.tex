Для каждого дня соревнований (для каждой итерации) справедлива следующая процедура
приемочного тестирования:
\begin{itemize}
    \item В конце каждого дня финального этапа команды должны сформировать запрос
    на слияние (\textit{Merge Request}) из своей ветки исходного кода в основную ветку
    (\textit{master}) в Git-репозитории.
    \item Команда ответственна за то, чтобы в запросе на слияние не должно быть
    конфликтов. Запрос на слияние с конфликтами может не рассматриваться жюри для
    выполнения приемочного тестирования.
    \item После того, как все команды отправили запросы на слияние, жюри одобряет
    все запросы и приступает к приемочному тестированию, для тех подзадач, которые
    входят в соответствующую итерацию. Для этого исходный код приложения команды
    загружается в систему автоматического тестирования (поддержка которой осуществляется
    функциональность GitLab CI/CD), где запускаются автоматические тесты на соответствие
    решения участников требованиям к приемочным тестам (\textit{acceptance criteria}).
    \item Если все приемочные тесты для данной подзадачи пройдены успешно,
    команда получает баллы за данную подзадачу. Если хотя бы один тест не проходит,
    то баллы за данное подзадачу не начисляются.
\end{itemize}

Приемочные тесты для каждой подзадачи описаны в разделе ''Подробное описание подзадач''.

Дальше перечислены баллы, которые получает команда за решение подзадач в каждой итерации.

Максимальное количество баллов, которое может набрать команда за решение всех 
подзадач --- 400. 

\subsection*{Первая итерация}

\footnotesize
\begin{center}
\begin{tabular}{ |l|c|  }
 \hline
  \thead{user story} & \thead{баллы} \\
 \hline

US-001 Регистрация контракта & 10 \\
US-004 Простое добавление пользователя в сервис индентификации & 20 \\
US-013 Получение баланса идентифицированного пользователя & 10 \\

 \hline
\end{tabular}
\end{center}
\normalsize

Максимальное количество баллов за итерацию --- 40.

\subsection*{Вторая итерация}

\footnotesize
\begin{center}
\begin{tabular}{ |l|c|  }
 \hline
  \thead{user story} & \thead{баллы} \\
 \hline

US-002 Вывод владельца контракта регистра соответствий & 10 \\
US-003 Изменение владельца контракта регистра соответствий & 10 \\
US-006 Получение всех пользователей из сервиса идентификации & 10 \\
US-007 Удаление пользователя из сервиса идентификации & 10 \\
US-008 Запуск обучения сервиса индентификации (кроме AC-008-04) & 18 \\
US-010 Идентификация пользователя (кроме AC-010-04) & 23 \\
US-014 Отправка запроса на регистрацию соответствия & 10 \\
US-015 Отправка запроса на удаление соответствия & 10 \\
US-016 Отмена запроса на регистрацию или удаление соответствия & 15 \\
US-017 Отправка средств & 10 \\
US-021 Получение истории платежей & 20 \\
US-023 Получение всех запросов на регистрацию соответствий & 10 \\
US-024 Получение всех запросов на удаление соответствий & 10 \\
US-025 Подтверждение запросов на регистрацию или удаление соответствий & 15 \\
US-026 Получение аккаунта по номеру телефона & 10 \\

 \hline
\end{tabular}
\end{center}
\normalsize

Максимальное количество баллов за итерацию --- 191.

\subsection*{Третья итерация}

\footnotesize
\begin{center}
\begin{tabular}{ |l|c|  }
 \hline
  \thead{user story} & \thead{баллы} \\
 \hline

US-005 Улучшенное добавление пользователя в сервис индентификации & 50 \\
US-008 Запуск обучения сервиса индентификации (полностью) & 2 \\
US-009 Обнаружение уже добавленного пользователя & 20 \\
US-010 Идентификация пользователя (полностью) & 2 \\
US-011 Запрос действий на безопасную идентификацию пользователя & 5 \\
US-012 Безопасная идентификация пользователя & 35 \\
US-018 Генерация сертификата на получение средств & 10 \\
US-019 Использование сертификата на получение средств & 15 \\
US-020 Вернуть средства из неиспользованных сертификатов & 20 \\
US-022 Получение истории платежей, включая использование сертификатов & 10 \\

 \hline
\end{tabular}
\end{center}
\normalsize

Максимальное количество баллов за итерацию --- 169.

\subsection*{Критерии определения команды-победителя командного тура}

\begin{itemize}
    \item Сумма баллов, набранных за решения подзадач командного тура финального этапа, определяет итоговую результативность команды (измеряемую в баллах).
    \item Команды ранжируются по результативности.
    \item Команда победитель определяется, как команда с максимальной результативностью.
\end{itemize}
