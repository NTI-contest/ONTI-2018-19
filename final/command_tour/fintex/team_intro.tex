Современные банки в погоне за привлечением клиентов разрабатывают и вводят в
эксплуатацию все более комфортные сервисы. 

\twoitems{Например, российские банки Сбербанк и
Тинькофф-банк вводят в строй банкоматы, способные идентифицировать человека по 
лицу}{\url{https://tass.ru/ekonomika/4402287}}
{\url{https://www.banki.ru/news/lenta/?id=10740603}}

Банк ''Ак Барс'' планирует использовать технологию \textit{Face2Pay} в
образовательных учреждениях для оплаты школьниками питания:
\url{https://tass.ru/ekonomika/5607346}.

Не только в России (\url{https://www.banki.ru/news/lenta/?id=10670764}), но и в других
странах мира (\url{https://www.vesti.ru/doc.html?id=3126288}) разрабатываются системы
для оплаты общественного транспорта с помощью технологий распознавания человека по лицу.

Также стремясь увеличить скорость перемещения денежных средств (в частности для
межбанковских и международных переводов) банки рассматривают альтернативные платежные
системы, основанные на технологии блокчейн:
\begin{itemize}
  \item \url{https://www.jpmorgan.com/global/news/digital-coin-payments}
  \item \url{http://coinlog.ru/6-mirovyh-bankov-zapuskayut-stejblkoiny-v-obshhej-blokchejn-seti-ibm/}
  \item \url{https://bits.media/pyat-bankov-yaponii-obedinilis-dlya-sozdaniya-blokcheyn-resheniy/}
  \item \url{https://cryptoratings.ru/news/alfa-bank-podklyuchilsya-k-blokchejn-seti-marco-polo/}
\end{itemize}

Поскольку перечисленные выше технологии становятся трендом в современном финансовом
секторе, то участниками профиля ''Программная инженерия финансовых технологий''
предлагается разработать прототип программного обеспечения банкомата, проводящего операции
в блокчейн сети, и идентифицирующего пользователя по лицу.

При этом разработка модели машинного обучения для распознавания и идентификации человека по лицу
остается за рамками данного прототипа. Как и в реальных индустриальных проектах, на стадии
прототипа вместо разработки собственной модели будет применяться уже готовое решение, а именно
сервис \textit{Microsoft Face API}, наиболее удовлетворяющий нужды данного проекта.

В общих чертах работа банкомата описывается следующим образом:
\begin{itemize}
  \item Пользователь идентифицируется банкоматом по лицу.
  \item В качестве дополнительной меры безопасности банкомат может запросить от пользователя выполнить определенные действия, для проверки, что перед камерой банкомата не ''плоское'' изображение пользователя и не маникен.
  \item Поскольку блокчейн сеть авторизует пользовательские операции посредством приватного ключа пользователя, ключ генерируется на основании идентификатора пользователя, возвращенного сервисом идентификации человека по лицу, и PIN-кодом, введенным человеком. Таким образом, происходит защита от перехвата индентификатора человека.
  \item После получения приватного ключа возможны операции получения баланса, проведение платежей, получения истории платежей.
  \item Платежи другим пользователям осуществляются по их номеру телефона  
\end{itemize}

Следовательно, чтобы ПО банкомата могло работать необходимо реализовать дополнительные сервисные функции, осуществляемые при регистрации нового пользователя в систему (например, в офисе банка):
\begin{itemize}
  \item Управление пользователями в сервисе распознования по лицу: добавление и удаление пользователя и изображений его лица.
  \item Управление соответствями между пользователем и его номером телефона: регистрация и удаление соответствия.
  \item Для избежания мошенничеств регистрация и удаление соответствия должны подтверждаться авторизованным лицом (администратором в банке).
\end{itemize}
