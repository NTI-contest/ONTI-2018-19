\subsection*{Подотовка сервиса идентификации}


Сервис идентификации человека по лицу перед полноценной работой требует предварительной настройки. Первое, что должно быть сделано - в сервис необходимо добавить лица людей, которых в дальнейшем необходимо идентифицировать. Поскольку система автоматического тестирования не может работать с камерой, то изображения человека будут передаваться через видео-файл, передаваемый в сервис через параметры командной строки.

Администратор свервиса должен иметь возможность просматривать список добавленных пользователей, удалять пользователя (и его изображения) из системы. 

Как только необходимое количество лиц зарегистрировано в сервисе, администратор может запустить обучение нейронной сети.

\begin{myverbbox}{\scriptFile}
face-management.py <command> [options]
\end{myverbbox}
\scriptTitle


%-----------------------------------------------------
%US-004
\newuserstory{Простое добавление пользователя в сервис индентификации}


Сервис должен быть устроен так, что добавление изображений человека в систему происходит анонимно, т.е. имя при добавлении не указывается. 


\begin{myverbbox}[\small]{\cmdLine}
$ face-management.py --simple-add <path to video file>
\end{myverbbox}
\scriptExample{
При использовании команды простого добавления пользователя из видео-потока извлекается 5 кадров с изображением лица человека. При этом подразумевается, что все кадры в видео приндалежат одному и тому же человеку. 

Если в видео-потоке недостаточно кадров с изображением человека, то обработка такого видео должно приводить к ошибке.

}

% AC-004-01
\begin{myverbbox}[\small]{\output}
$ cat faceapi.json | python -mjson.tool | grep groupId
    "groupId": "fintech-01",
$ curl -X GET "https://<datacenter url>/face/v1.0/persongroups/fintech-01
" -H "Content-Type: application/json" -H "Ocp-Apim-Subscription-Key: 0000
00000000000000000000000000000" 
{"error":{"code":"PersonGroupNotFound","message":"Person group is not fou
nd.\r\nParameter name: personGroupId"}}
$ face-management.py --simple-add /path/to/video.avi
5 frames extracted
PersonId: 419e345a-e6d6-4d9c-953d-667787b8d52e
FaceIds
=======
e27558b9-812d-41c3-b114-8e434b8f4602
44c350f2-6653-4616-a1b7-e0fe9b481b6b
f945a3be-4b20-4049-b080-4142a55e4f93
855ab7c2-9bb3-49ed-8cac-1366c0274b08
9c4af288-54cd-4375-8eef-f8c29ed56685
\end{myverbbox}
\acceptanceCriteria{Требуемый \texttt{personGroupId} не существовал до этого в сервисе \textit{Microsoft Face API}. После добавления \texttt{personGroupId} добавляется новый \texttt{personId}, с которым ассоциируется 5 изображений лица (\texttt{persistedFaceId}).
}

% AC-004-02
\begin{myverbbox}[\small]{\output}
$ cat faceapi.json | python -mjson.tool | grep groupId
    "groupId": "fintech-01",
$ curl -X GET "https://<datacenter url>/face/v1.0/persongroups" -H "Conte
nt-Type: application/json" -H "OcApim-Subscription-Key: 00000000000000000
0000000000000000" 
[{"personGroupId":"fintech-01","name":"fintech-01","userData":null}]
$ face-management.py --simple-add /path/to/video1.avi
5 frames extracted
PersonId: 37da04e7-f471-49c7-a54c-a08f05950fc5
FaceIds
=======
1d499868-3d01-487c-8bab-626dc562e4e8
27dadf08-bc60-4a29-82a7-7d21ea7f40af
b8cf9c2f-a606-4f21-851d-26e0a0dc8a74
bf4806de-8c4b-4a12-8495-002f43dba797
ff79486f-15ac-43be-9c6c-b2840f8c8d22
\end{myverbbox}
\acceptanceCriteria{Требуемый \texttt{personGroupId} существует в \textit{Microsoft Face API}. В данную группу добавляется новый \texttt{personId}, с которым ассоциируется 5 изображений лица (\texttt{persistedFaceId}).
}

% AC-004-03
\begin{myverbbox}[\small]{\output}
face-management.py --simple-add /path/to/video2.avi
Video does not contain any face
\end{myverbbox}
\acceptanceCriteria{Выдается ошибка при попытке обработать видео, в котором либо нет кадров с лицом пользователя, либо в видео содержится меньше 5 кадров. Группа с \texttt{personGroupId} не создается, новый \texttt{personId} не добавляется.
}

% AC-004-04
\begin{myverbbox}[\small]{\output}
$ face-management.py --simple-add /path/to/video1.avi
5 frames extracted
PersonId: 52865cde-3af8-443d-b260-9319c2cb1788
FaceIds
=======
cdb6227e-7453-4057-b4fa-79660914e597
6976d3c2-dee5-4f24-8950-f38ff10c70ad
fae15e55-6639-42a4-a954-731c33310e41
15092567-5765-49ed-ac63-94bc5fa08d17
a77f1f0a-aa95-4bd1-9826-6b453aec42b2
$ face-management.py --simple-add /path/to/video31.avi
5 frames extracted
PersonId: 9fa0a99b-8e76-474d-8223-dea217c2c19b
FaceIds
=======
b552ef11-a162-4a7d-9047-ccfc84a07043
90c0815a-ecce-45c6-8107-ced7ef29a249
fde35dba-505d-4a62-ac5a-c6ae4c89128e
6c6910b4-0ab5-4eb4-9e53-95b1929f9867
fdb9d352-65b0-41a2-a1be-03ea5b543160
$ face-management.py --simple-add /path/to/video41.avi
5 frames extracted
PersonId: f290ecb9-bfab-46f7-b623-45140d730628
FaceIds
=======
e5735ecd-ca09-4fd4-bfd3-8ace67702ab0
9e1bbdee-5981-4f6b-aba5-03be57e5e910
3120ef58-8d53-4558-8b84-784ba338f621
8fceb9c7-f029-4326-9703-6749005674fa
8ea02a3b-7dc0-455a-858c-67251b0ca3b4
\end{myverbbox}
\acceptanceCriteria{Несколько добавлений пользователя проходят успешно. Для каждого видео добавляется новый \texttt{personId} добавляется.
}

%-----------------------------------------------------
%US-005
\newuserstory{Улучшенное добавление пользователя в сервис индентификации}


Наличие в наборе изображений одного и того же человека, но у которых есть различия в мимике, положении головы и освещенности, влияет на качество обучения нейронной сети сервиса, поэтому при сборе данных для сервиса идентификации важно собирать разные изображения.


\begin{myverbbox}[\small]{\cmdLine}
$ face-management.py --add <path to video file 1> [ <path to video file 2
> [ <path to video file 3> [ <path to video file 4> [ <path to video file
 5> ] ] ] ]
\end{myverbbox}
\scriptExample{
Команда ожидает 5 видео файлов. Требования к каждому из видео файлов:
\begin{itemize}
  \item В первом по списку видеофайле лицо человека должно быть неподвижно (допускаются небольшие повороты)
  \item Во втором по списку видеофайле должны быть зафиксированы наклоны головы влево и вправо (\textit{roll})
  \item В третьем по списку видеофайле должны быть зафиксированы повороты головы влево и вправо (\textit{yaw})
  \item В четвертом видеофайле должен быть зафиксирован открытый рот
  \item В пятом - должно быть зафиксировано закрытие глаз
\end{itemize}

}

% AC-005-01
\begin{myverbbox}[\small]{\output}
$ cat faceapi.json | python -mjson.tool | grep groupId
    "groupId": "fintech-01",
$ curl -X GET "https://<datacenter url>/face/v1.0/persongroups/fintech-01
" -H "Content-Type: application/json" -H "Ocp-Apim-Subscription-Key: 0000
00000000000000000000000000000" 
{"error":{"code":"PersonGroupNotFound","message":"Person group is not fou
nd.\r\nParameter name: personGroupId"}}
$ face-management.py --add /path/to/video1.avi
5 frames extracted
PersonId: ddaa7036-cab0-4d8f-9b36-18f20e294c51
FaceIds
=======
5754a049-0de7-4ee5-99ba-45d0d3398645
56c0aa34-50c4-4d0b-bcf0-c86e9c9dec52
f626c3d7-1126-401c-b125-16b6c65e6ed8
6d617030-941d-4f6c-9d05-b714b4e2b504
1087c13c-09d7-4053-8bc6-9381c48081e2
\end{myverbbox}
\acceptanceCriteria{Требуемый \texttt{personGroupId} не существовал до этого в сервисе \textit{Microsoft Face API}. После добавления \texttt{personGroupId} добавляется новый \texttt{personId}, с которым ассоциируется 5 изображений лица (\texttt{persistedFaceId}). Изображение лица человека не должно значительно отличаться от кадра к кадру (допускаются повороты до 5 градусов).
}

% AC-005-02
\begin{myverbbox}[\small]{\output}
$ cat faceapi.json | python -mjson.tool | grep groupId
    "groupId": "fintech-01",
$ curl -X GET "https://<datacenter url>/face/v1.0/persongroups" -H "Conte
nt-Type: application/json" -H "OcApim-Subscription-Key: 00000000000000000
0000000000000000" 
[{"personGroupId":"fintech-01","name":"fintech-01","userData":null}]
$ face-management.py --add /path/to/video1.avi
5 frames extracted
PersonId: 6dc70d27-3bad-400a-983f-fec6591cff6b
FaceIds
=======
08609f0a-55ee-46ed-899b-5f7d66f62cce
2bb6d3a9-1c52-4a70-ac62-44881f2aed29
4309d3b7-5250-47e3-bc7e-f3d1da8badd1
76b6a516-c378-4d86-b4cc-03b60b5c2d2c
85f2af43-ca0e-4dd1-a400-9780d7b7a8f5
\end{myverbbox}
\acceptanceCriteria{В требуемую \texttt{personGroupId} добавляется новый \texttt{personId}, с которым ассоциируется 5 изображений лица (\texttt{persistedFaceId}). Изображение лица человека не должно значительно отличаться от кадра к кадру (допускаются повороты до 5 градусов).
}

% AC-005-03
\begin{myverbbox}[\small]{\output}
$ face-management.py --add /path/to/video2.avi
Base video does not follow requirements
\end{myverbbox}
\acceptanceCriteria{Изображение лица человека в видео значительно отличаться от кадра к кадру (повороты больше чем на 5 градусов либо есть открый рот, либо есть закрытые глаза). Группа с \texttt{personGroupId} не создается, новый \texttt{personId} не добавляется.
}

% AC-005-04
\begin{myverbbox}[\small]{\output}
$ face-management.py --add /path/to/video1.avi /path/to/video2.avi
10 frames extracted
PersonId: 8d0b7195-c172-4d1a-8588-063f3e64c0c0
FaceIds
=======
08609f0a-55ee-46ed-899b-5f7d66f62cce
2bb6d3a9-1c52-4a70-ac62-44881f2aed29
4309d3b7-5250-47e3-bc7e-f3d1da8badd1
76b6a516-c378-4d86-b4cc-03b60b5c2d2c
85f2af43-ca0e-4dd1-a400-9780d7b7a8f5
300c161f-7154-43c4-80bc-34d1fd9ffb1d
4f57f616-23ea-4be7-b977-8e9b14a192b4
52c9b5c4-6cb7-4aae-b9d0-206d6db90510
8f9a5e7f-c2c4-4d92-908b-6f220b4d3b32
d45f9324-a845-4db6-abc6-88214726bdcc
\end{myverbbox}
\acceptanceCriteria{В требуемую \texttt{personGroupId} добавляется новый \texttt{personId}, с которым ассоциируется 10 изображений лица (\texttt{persistedFaceId}). Помимо 5 изображений лиц из первого видео, добавляется 5 - из второго.

Лицо человека во втором видео должно наклонятся из крайнего положения влево к крайнему положению вправо. Зафиксирован максимальный наклон в каждую из сторон не меньше 30 градусов. Из второго видео выбрано 5 кадров с наклонами головы из следующего списка: 30 градусов влево, 15 градусов влево, 0 градусов, 15 градусов вправо, 30 градусов вправо. Допускается отклонение от перечисленных значений +/- 3 градуса.
}

% AC-005-05
\begin{myverbbox}[\small]{\output}
$ face-management.py --add /path/to/video1.avi /path/to/video3.avi
Roll video does not follow requirements
\end{myverbbox}
\acceptanceCriteria{Во втором по счету видео лицо человека не доходит до крайних положений. Т.е. нельзя зафиксировать максимальный наклон в каждую из сторон от 30 градусов и более. Группа с \texttt{personGroupId} не создается, новый \texttt{personId} не добавляется.
}

% AC-005-06
\begin{myverbbox}[\small]{\output}
$ face-management.py --add /path/to/video1.avi /path/to/video2.avi /path/
to/video3.avi
15 frames extracted
PersonId: a25ddbf2-aabf-41c7-b9a9-d69f1c055761
FaceIds
=======
0ca5925b-55eb-44fd-9624-39fbaa33a5c5
31091841-2e3c-43c2-a14c-497fd8137d25
3f46e69d-33d7-4474-97a1-230046dc9cfa
f1752749-f602-4151-b3bb-7cec627de3df
f556db9a-55cd-4542-bf0f-a5848376ba66
08609f0a-55ee-46ed-899b-5f7d66f62cce
2bb6d3a9-1c52-4a70-ac62-44881f2aed29
4309d3b7-5250-47e3-bc7e-f3d1da8badd1
76b6a516-c378-4d86-b4cc-03b60b5c2d2c
85f2af43-ca0e-4dd1-a400-9780d7b7a8f5
300c161f-7154-43c4-80bc-34d1fd9ffb1d
4f57f616-23ea-4be7-b977-8e9b14a192b4
52c9b5c4-6cb7-4aae-b9d0-206d6db90510
8f9a5e7f-c2c4-4d92-908b-6f220b4d3b32
d45f9324-a845-4db6-abc6-88214726bdcc
\end{myverbbox}
\acceptanceCriteria{В требуемую \texttt{personGroupId} добавляется новый \texttt{personId}, с которым ассоциируется 15 изображений лица (\texttt{persistedFaceId}). Помимо 10 изображений лиц из первого и второго видео, добавляется 5 - из третьего.

Лицо человека в третьем видео должно поворачиваться из крайнего положения влево к крайнему положению вправо. Зафиксирован максимальный поворот в каждую из сторон не меньше 20 градусов. Из третьего видео выбрано 5 кадров с наклонами головы из следующего списка: 20 градусов влево, 10 градусов влево, 0 градусов, 10 градусов вправо, 20 градусов вправо. Допускается отклонение от перечисленных значени +/- 3 градуса.
}

% AC-005-07
\begin{myverbbox}[\small]{\output}
$ face-management.py --add /path/to/video1.avi /path/to/video2.avi /path/
to/video4.avi
Yaw video does not follow requirements
\end{myverbbox}
\acceptanceCriteria{Во третьем по счету видео лицо человека не доходит до крайних положений. Т.е. нельзя зафиксировать максимальный поворот в каждую из сторон от 20 градусов и более. Группа с \texttt{personGroupId} не создается, новый \texttt{personId} не добавляется.
}

% AC-005-08
\begin{myverbbox}[\small]{\output}
$ face-management.py --add /path/to/video1.avi /path/to/video2.avi /path/
to/video3.avi /path/to/video4.avi
16 frames extracted
PersonId: e3152bc7-431c-4215-a19b-d128f2768438
FaceIds
=======
100d7ff0-0e95-4998-a385-d65a69cbbab4
1ca71243-1370-4c7f-b457-ac53ade0b59a
25f51952-880c-435e-9afb-3ca8b2b981d4
cf17ee9e-140b-444b-a87a-35209a631aa1
f0ae9fd1-81bd-4fcc-853c-107e228be383
2307cf52-687b-4d51-9861-728b6297aa7d
34ac19bb-4018-498b-bb13-02098aaac75c
7e01b585-78b1-42a0-8e48-4b1ef199dde7
99943bc7-f92e-49fb-91b4-06b123343982
ae8b5917-ee88-4499-b262-8bfbfda687ca
8195743c-ae84-48fe-b381-1eb73c82f5c1
87ad1c82-0ffd-4b39-8b72-fa6323257802
9cb6fc25-f32e-49d7-b6de-352bf2d911fc
bff40657-1ea4-4667-b2aa-d8b480f527be
d4b57970-6390-4c23-a366-0c51fdc17f9b
1f41144e-eee5-4210-8a7d-ffa4adc3ba21
\end{myverbbox}
\acceptanceCriteria{В требуемую \texttt{personGroupId} добавляется новый \texttt{personId}, с которым ассоциируется 16 изображений лица (\texttt{persistedFaceId}). Помимо 15 изображений лиц из первого-третьего видео, добавляется 1 изображение - из четвертого.

На серии кадров человека в четвертом видео рот человека должен быть открыт. Один из кадров с открытым ртом отправляется в \textit{Microsoft Face API}.
}

% AC-005-09
\begin{myverbbox}[\small]{\output}
$ face-management.py --add /path/to/video1.avi /path/to/video2.avi /path/
to/video3.avi /path/to/video5.avi
Video to detect open mouth does not follow requirements
\end{myverbbox}
\acceptanceCriteria{Во четвертом по счету видео не найдены кадры с открытым ртом. Группа с \texttt{personGroupId} не создается, новый \texttt{personId} не добавляется.
}

% AC-005-10
\begin{myverbbox}[\small]{\output}
$ face-management.py --add /path/to/video1.avi /path/to/video2.avi /path/
to/video3.avi /path/to/video4.avi /path/to/video5.avi
18 frames extracted
PersonId: 588e76bc-a0ce-4066-94d2-96620d105401
FaceIds
=======
220dd1ea-3392-466b-ab50-fbb465a513ff
4977b87d-3c36-45d7-b827-8907aa66b918
29e7d8fd-f065-41b6-9fd0-b191fb3dde78
d0e269f1-199a-459e-9fbf-1c24e015fdec
94d0887a-4e0a-4b86-9e6e-8f6fdd45f1ec
12938747-8530-4bcc-a811-b890aa852175
2d486b65-ed15-4313-836c-2cd4ca7e02cb
9a1ab5a7-9e46-4566-8d85-54bbcd021307
7a6f7afd-a1f6-4d7a-9e6f-c5e0daf1501a
105dfcaa-a25d-45ff-ba38-48de04a735e6
090707ff-91b8-4647-afbd-a80f1976c10d
65f53802-dfa0-46a0-b9e5-c4780f41d580
8c485998-ea43-439b-a84e-606433232133
7b0166db-f7a2-477c-b3f6-bbf3833ea770
191f33fc-501f-4d1a-8b5f-a4f9917f311a
44c79f24-5f87-48c6-af1f-f85e0fca521e
cdd4e100-217d-4163-8f51-4c8f82282f46
e417cb0c-ad18-47eb-8ffa-d850f3aaa8ec
\end{myverbbox}
\acceptanceCriteria{В требуемую \texttt{personGroupId} добавляется новый \texttt{personId}, с которым ассоциируется 18 изображений лица (\texttt{persistedFaceId}). Помимо 16 изображений лиц из первого-четвертого видео, добавляется два изображения - из пятого.

На серии кадров человека в пятом видео был закрыт сначала левый глаз, потом правый. По одному кадру с каждым из закрытых глаз отправляется в \textit{Microsoft Face API}.
}

% AC-005-11
\begin{myverbbox}[\small]{\output}
$ face-management.py --add /path/to/video1.avi /path/to/video2.avi /path/
to/video3.avi /path/to/video4.avi /path/to/video6.avi
18 frames extracted
PersonId: 5d5d6e92-f8f0-47cf-8e9b-b4455092603e
FaceIds
=======
0e5edd8e-7756-470e-a26c-f54ab70a5524
73b4caea-c85c-4594-9153-198351937d94
2ae3c850-190a-4cd0-afd2-8efa341767ba
adb7744e-b2fb-4a58-b3a8-f22c3143a8cd
7e0f87d6-14ad-4ec6-971f-e4c2771cc267
47c8c307-1f90-473c-b48c-792d5b5a1241
33e35a22-358a-4847-bcdd-4a5bdfd5eb69
2c7b951b-126a-4eec-9e22-1d9d65ece9e3
bcaf7f6a-1c44-476e-a51f-9dde2744ade0
1c9c0134-ada0-47c2-848c-d4424b232f04
b79945e1-de47-4011-98f2-e7e8f0d9f3ef
f972b6ac-078b-4e3d-a779-4cde8ea28f36
fbeb7aa1-f66e-4412-ad07-b61103e8af0b
9c762e64-075b-4fa4-8e39-0d83df86428b
9794e349-e4e4-44ad-951c-610b4890b16e
2e93da04-8f31-4a27-bd9e-5768335447a5
b9227512-dfa8-4a17-94e2-be520b8975a9
b67c1f86-2e3d-45a2-a7fb-57bbbff9cd8c
\end{myverbbox}
\acceptanceCriteria{В требуемую \texttt{personGroupId} добавляется новый \texttt{personId}, с которым ассоциируется 18 изображений лица (\texttt{persistedFaceId}). Помимо 16 изображений лиц из первого-четвертого видео, добавляется два изображения - из пятого.

На серии кадров человека в пятом видео был закрыт сначала правый глаз потом левый. По одному кадру с каждым из закрытых глаз отправляется в \textit{Microsoft Face API}.
}

% AC-005-12
\begin{myverbbox}[\small]{\output}
$ face-management.py --add /path/to/video1.avi /path/to/video2.avi /path/
to/video3.avi /path/to/video4.avi /path/to/video7.avi
Video to detect closed eyes does not follow requirements
\end{myverbbox}
\acceptanceCriteria{Во пятом по счету видео не найдены кадры с закрытыми глазами. Причем видео считается не удовлетворяющим требованиям, если только один глаз был закрыт. Группа с \texttt{personGroupId} не создается, новый \texttt{personId} не добавляется.
}

% AC-005-13
\begin{myverbbox}[\small]{\output}
$ face-management.py --add /path/to/video11.avi /path/to/video12.avi /pat
h/to/video13.avi /path/to/video14.avi /path/to/video15.avi
Roll video does not follow requirements
\end{myverbbox}
\acceptanceCriteria{В каком-то видео не найдены кадры, удовлетворяющие требованиям. Группа с \texttt{personGroupId} не создается, новый \texttt{personId} не добавляется.
}

%-----------------------------------------------------
%US-006
\newuserstory{Получение всех пользователей из сервиса идентификации}


Администратор сервиса может получить список всех добавленных пользователей. 


\begin{myverbbox}[\small]{\cmdLine}
$ face-management.py --list
\end{myverbbox}
\scriptExample{


}

% AC-006-01
\begin{myverbbox}[\small]{\output}
$ cat faceapi.json | python -mjson.tool | grep groupId
    "groupId": "fintech-01",
$ curl -X GET "https://<datacenter url>/face/v1.0/persongroups/fintech-01
" -H "Content-Type: application/json" -H "Ocp-Apim-Subscription-Key: 0000
00000000000000000000000000000" 
{"error":{"code":"PersonGroupNotFound","message":"Person group is not fou
nd.\r\nParameter name: personGroupId"}}
$ face-management.py --list
The group does not exist
\end{myverbbox}
\acceptanceCriteria{Требуемый \texttt{personGroupId} не существовует в сервисе \textit{Microsoft Face API} ни до, ни после выполнения команды.
}

% AC-006-02
\begin{myverbbox}[\small]{\output}
$ cat faceapi.json | python -mjson.tool | grep groupId
    "groupId": "fintech-01",
$ curl -X GET "https://<datacenter url>/face/v1.0/persongroups" -H "Conte
nt-Type: application/json" -H "OcApim-Subscription-Key: 00000000000000000
0000000000000000" 
[{"personGroupId":"fintech-01","name":"fintech-01","userData":null}]
$ face-management.py --list
Persons IDs:
27dadf08-bc60-4a29-82a7-7d21ea7f40af
b8cf9c2f-a606-4f21-851d-26e0a0dc8a74
bf4806de-8c4b-4a12-8495-002f43dba797
ff79486f-15ac-43be-9c6c-b2840f8c8d22
\end{myverbbox}
\acceptanceCriteria{Требуемый \texttt{personGroupId} существует в \textit{Microsoft Face API}.
}

% AC-006-03
\begin{myverbbox}[\small]{\output}
$ cat faceapi.json | python -mjson.tool | grep groupId
    "groupId": "fintech-01",
$ curl -X GET "https://<datacenter url>/face/v1.0/persongroups" -H "Conte
nt-Type: application/json" -H "OcApim-Subscription-Key: 00000000000000000
0000000000000000" 
[{"personGroupId":"fintech-01","name":"fintech-01","userData":null}]
$ face-management.py --list
No persons found
\end{myverbbox}
\acceptanceCriteria{Требуемый \texttt{personGroupId} существует в \textit{Microsoft Face API}, но в ней нет пользователей.
}

%-----------------------------------------------------
%US-007
\newuserstory{Удаление пользователя из сервиса идентификации}


Администратор сервиса может удалить пользователя сервиса по его идентификатору. 


\begin{myverbbox}[\small]{\cmdLine}
$ face-management.py --del <person id>
\end{myverbbox}
\scriptExample{


}

% AC-007-01
\begin{myverbbox}[\small]{\output}
$ cat faceapi.json | python -mjson.tool | grep groupId
    "groupId": "fintech-01",
$ curl -X GET "https://<datacenter url>/face/v1.0/persongroups/fintech-01
" -H "Content-Type: application/json" -H "Ocp-Apim-Subscription-Key: 0000
00000000000000000000000000000" 
{"error":{"code":"PersonGroupNotFound","message":"Person group is not fou
nd.\r\nParameter name: personGroupId"}}
$ face-management.py --del 27dadf08-bc60-4a29-82a7-7d21ea7f40af
The group does not exist
\end{myverbbox}
\acceptanceCriteria{Требуемый \texttt{personGroupId} не существовует в сервисе \textit{Microsoft Face API} ни до, ни после выполнения команды.
}

% AC-007-02
\begin{myverbbox}[\small]{\output}
$ cat faceapi.json | python -mjson.tool | grep groupId
    "groupId": "fintech-01",
$ curl -X GET "https://<datacenter url>/face/v1.0/persongroups" -H "Conte
nt-Type: application/json" -H "OcApim-Subscription-Key: 00000000000000000
0000000000000000" 
[{"personGroupId":"fintech-01","name":"fintech-01","userData":null}]
$ face-management.py --del 27dadf08-bc60-4a29-82a7-7d21ea7f40af
Person deleted
\end{myverbbox}
\acceptanceCriteria{Требуемый \texttt{personGroupId} существует в \textit{Microsoft Face API}. Пользователь с заданным ID удаляется из сервиса.
}

% AC-007-03
\begin{myverbbox}[\small]{\output}
$ face-management.py --del bdaf190a-4805-4e2c-95af-1afa8b2623df
The person does not exist
\end{myverbbox}
\acceptanceCriteria{Пользователь с данным ID не существет в требуемой \texttt{personGroupId}.
}

%-----------------------------------------------------
%US-008
\newuserstory{Запуск обучения сервиса индентификации }


Администратор сервиса может запустить обучение нейронной сети сервиса \textit{Microsoft Face API} для возможности дальнейшего распознавания человека по лицу.

Обучение должно запускаться только если до этого происходило добавление или удаление нового человека.


\begin{myverbbox}[\small]{\cmdLine}
$ face-management.py --train
\end{myverbbox}
\scriptExample{


}

% AC-008-01
\begin{myverbbox}[\small]{\output}
$ cat faceapi.json | python -mjson.tool | grep groupId
    "groupId": "fintech-01",
$ curl -X GET "https://<datacenter url>/face/v1.0/persongroups/fintech-01
" -H "Content-Type: application/json" -H "Ocp-Apim-Subscription-Key: 0000
00000000000000000000000000000" 
{"error":{"code":"PersonGroupNotFound","message":"Person group is not fou
nd.\r\nParameter name: personGroupId"}}
$ face-management.py --train
There is nothing to train
\end{myverbbox}
\acceptanceCriteria{Требуемый \texttt{personGroupId} не существовует в сервисе \textit{Microsoft Face API} ни до, ни после выполнения команды. Тренировка сервиса не запускается.
}

% AC-008-02
\begin{myverbbox}[\small]{\output}
$ cat faceapi.json | python -mjson.tool | grep groupId
    "groupId": "fintech-01",
$ curl -X GET "https://<datacenter url>/face/v1.0/persongroups" -H "Conte
nt-Type: application/json" -H "OcApim-Subscription-Key: 00000000000000000
0000000000000000" 
[{"personGroupId":"fintech-01","name":"fintech-01","userData":null}]
$ face-management.py --train
There is nothing to train
\end{myverbbox}
\acceptanceCriteria{Требуемый \texttt{personGroupId} существует в \textit{Microsoft Face API}, но в группе нет ни одного добавленного пользователя. Тренировка сервиса не запускается.
}

% AC-008-03
\begin{myverbbox}[\small]{\output}
$ cat faceapi.json | python -mjson.tool | grep groupId
    "groupId": "fintech-01",
$ curl -X GET "https://<datacenter url>/face/v1.0/persongroups" -H "Conte
nt-Type: application/json" -H "OcApim-Subscription-Key: 00000000000000000
0000000000000000" 
[{"personGroupId":"fintech-01","name":"fintech-01","userData":null}]
$ face-management.py --simple-add /path/to/video1.avi
5 frames extracted
PersonId: 37da04e7-f471-49c7-a54c-a08f05950fc5
FaceIds
=======
1d499868-3d01-487c-8bab-626dc562e4e8
27dadf08-bc60-4a29-82a7-7d21ea7f40af
b8cf9c2f-a606-4f21-851d-26e0a0dc8a74
bf4806de-8c4b-4a12-8495-002f43dba797
ff79486f-15ac-43be-9c6c-b2840f8c8d22
$ face-management.py --train
Training successfully started
\end{myverbbox}
\acceptanceCriteria{Требуемый \texttt{personGroupId} существует в \textit{Microsoft Face API}. Поскольку запуску команды тренировки сервиса предшествует команда добавления пользователя, то обучение запускается.
}

% AC-008-04
\begin{myverbbox}[\small]{\output}
$ face-management.py --add /path/to/video1.avi /path/to/video2.avi /path/
to/video3.avi /path/to/video4.avi /path/to/video6.avi
18 frames extracted
PersonId: 5d5d6e92-f8f0-47cf-8e9b-b4455092603e
FaceIds
=======
0e5edd8e-7756-470e-a26c-f54ab70a5524
73b4caea-c85c-4594-9153-198351937d94
2ae3c850-190a-4cd0-afd2-8efa341767ba
adb7744e-b2fb-4a58-b3a8-f22c3143a8cd
7e0f87d6-14ad-4ec6-971f-e4c2771cc267
47c8c307-1f90-473c-b48c-792d5b5a1241
33e35a22-358a-4847-bcdd-4a5bdfd5eb69
2c7b951b-126a-4eec-9e22-1d9d65ece9e3
bcaf7f6a-1c44-476e-a51f-9dde2744ade0
1c9c0134-ada0-47c2-848c-d4424b232f04
b79945e1-de47-4011-98f2-e7e8f0d9f3ef
f972b6ac-078b-4e3d-a779-4cde8ea28f36
fbeb7aa1-f66e-4412-ad07-b61103e8af0b
9c762e64-075b-4fa4-8e39-0d83df86428b
9794e349-e4e4-44ad-951c-610b4890b16e
2e93da04-8f31-4a27-bd9e-5768335447a5
b9227512-dfa8-4a17-94e2-be520b8975a9
b67c1f86-2e3d-45a2-a7fb-57bbbff9cd8c
$ face-management.py --train
Training successfully started
\end{myverbbox}
\acceptanceCriteria{Поскольку запуску команды тренировки сервиса предшествует команда добавления пользователя, то обучение запускается.
}

% AC-008-05
\begin{myverbbox}[\small]{\output}
$ face-management.py --del 27dadf08-bc60-4a29-82a7-7d21ea7f40af
Person deleted
$ face-management.py --train
Training successfully started
\end{myverbbox}
\acceptanceCriteria{Поскольку запуску команды тренировки сервиса предшествует команда удаления пользователя, то обучение запускается.
}

% AC-008-06
\begin{myverbbox}[\small]{\output}
$ face-management.py --train
Training successfully started
$ face-management.py --del bdaf190a-4805-4e2c-95af-1afa8b2623df
The person does not exist
$ face-management.py --train
Already trained
\end{myverbbox}
\acceptanceCriteria{Поскольку после предыдущего запуска команды тренировки сервиса изменений в списке пользователей не происходило, то обучение не запускается.
}

%-----------------------------------------------------
%US-009
\newuserstory{Обнаружение уже добавленного пользователя }


Администратор сервиса не сможет добавить пользователя в сервис \textit{Microsoft Face API}, если изображения лица данного пользователя уже были добавлены в систему, и эти изображения были использованы для обучения сервиса.


% AC-009-01
\begin{myverbbox}[\small]{\output}
$ face-management.py --simple-add /path/to/video1.avi
5 frames extracted
PersonId: 37da04e7-f471-49c7-a54c-a08f05950fc5
FaceIds
=======
1d499868-3d01-487c-8bab-626dc562e4e8
27dadf08-bc60-4a29-82a7-7d21ea7f40af
b8cf9c2f-a606-4f21-851d-26e0a0dc8a74
bf4806de-8c4b-4a12-8495-002f43dba797
ff79486f-15ac-43be-9c6c-b2840f8c8d22
$ face-management.py --train
Training successfully started
$ face-management.py --simple-add /path/to/video21.avi
The same person already exists.
\end{myverbbox}
\acceptanceCriteria{Первое и второе видео содержат кадры лица одного и того же человека. Идентификация человека на втором видео происходит успешно, поскольку пять разных кадров из видео указывают на одного и того же человека с высокой степенью (не менее 50\%) уверенности определения. Добавление пользователя в систему не происходит.
}

% AC-009-02
\begin{myverbbox}[\small]{\output}
$ face-management.py --add /path/to/video1.avi /path/to/video2.avi /path/
to/video3.avi /path/to/video4.avi /path/to/video6.avi
18 frames extracted
PersonId: 5d5d6e92-f8f0-47cf-8e9b-b4455092603e
FaceIds
=======
0e5edd8e-7756-470e-a26c-f54ab70a5524
73b4caea-c85c-4594-9153-198351937d94
2ae3c850-190a-4cd0-afd2-8efa341767ba
adb7744e-b2fb-4a58-b3a8-f22c3143a8cd
7e0f87d6-14ad-4ec6-971f-e4c2771cc267
47c8c307-1f90-473c-b48c-792d5b5a1241
33e35a22-358a-4847-bcdd-4a5bdfd5eb69
2c7b951b-126a-4eec-9e22-1d9d65ece9e3
bcaf7f6a-1c44-476e-a51f-9dde2744ade0
1c9c0134-ada0-47c2-848c-d4424b232f04
b79945e1-de47-4011-98f2-e7e8f0d9f3ef
f972b6ac-078b-4e3d-a779-4cde8ea28f36
fbeb7aa1-f66e-4412-ad07-b61103e8af0b
9c762e64-075b-4fa4-8e39-0d83df86428b
9794e349-e4e4-44ad-951c-610b4890b16e
2e93da04-8f31-4a27-bd9e-5768335447a5
b9227512-dfa8-4a17-94e2-be520b8975a9
b67c1f86-2e3d-45a2-a7fb-57bbbff9cd8c
$ face-management.py --train
Training successfully started
$ face-management.py --add /path/to/video1.avi /path/to/video2.avi /path/
to/video3.avi /path/to/video4.avi /path/to/video6.avi
The same person already exists.
\end{myverbbox}
\acceptanceCriteria{Поскольку для добавления пользователя использовалось одно и то же видео, то повторное добавление пользователя в систему не происходит.
}

% AC-009-03
\begin{myverbbox}[\small]{\output}
$ face-management.py --simple-add /path/to/video1.avi
5 frames extracted
PersonId: 37da04e7-f471-49c7-a54c-a08f05950fc5
FaceIds
=======
1d499868-3d01-487c-8bab-626dc562e4e8
27dadf08-bc60-4a29-82a7-7d21ea7f40af
b8cf9c2f-a606-4f21-851d-26e0a0dc8a74
bf4806de-8c4b-4a12-8495-002f43dba797
ff79486f-15ac-43be-9c6c-b2840f8c8d22
$ face-management.py --train
Training successfully started
$ face-management.py --simple-add /path/to/video22.avi
5 frames extracted
PersonId: 31e1fc90-84ff-498a-833c-1730aa00a310
FaceIds
=======
21a12afe-6aab-4593-ac77-d20d2bea7e8b
e2f4e1d8-ba2d-4c13-80b7-791f10949143
8c366814-3e8a-441e-86f2-c9edf967c04e
8510de4b-9a70-4d62-823d-471107f838da
c159bd2e-2f71-4c0d-8c85-179d76d96953
\end{myverbbox}
\acceptanceCriteria{Первое и второе видео содержат кадры лица разных людей. Человек из второго видео не был добавлен до этого в сервис \textit{Microsoft Face API}. Кадры лица человека из второго видео добавляются в сервис.
}

% AC-009-04
\begin{myverbbox}[\small]{\output}
$ face-management.py --simple-add /path/to/video1.avi
5 frames extracted
PersonId: 52865cde-3af8-443d-b260-9319c2cb1788
FaceIds
=======
cdb6227e-7453-4057-b4fa-79660914e597
6976d3c2-dee5-4f24-8950-f38ff10c70ad
fae15e55-6639-42a4-a954-731c33310e41
15092567-5765-49ed-ac63-94bc5fa08d17
a77f1f0a-aa95-4bd1-9826-6b453aec42b2
$ face-management.py --simple-add /path/to/video31.avi
5 frames extracted
PersonId: 9fa0a99b-8e76-474d-8223-dea217c2c19b
FaceIds
=======
b552ef11-a162-4a7d-9047-ccfc84a07043
90c0815a-ecce-45c6-8107-ced7ef29a249
fde35dba-505d-4a62-ac5a-c6ae4c89128e
6c6910b4-0ab5-4eb4-9e53-95b1929f9867
fdb9d352-65b0-41a2-a1be-03ea5b543160
$ face-management.py --simple-add /path/to/video41.avi
5 frames extracted
PersonId: f290ecb9-bfab-46f7-b623-45140d730628
FaceIds
=======
e5735ecd-ca09-4fd4-bfd3-8ace67702ab0
9e1bbdee-5981-4f6b-aba5-03be57e5e910
3120ef58-8d53-4558-8b84-784ba338f621
8fceb9c7-f029-4326-9703-6749005674fa
8ea02a3b-7dc0-455a-858c-67251b0ca3b4
$ face-management.py --train
Training successfully started
$ face-management.py --simple-add /path/to/video22.avi
5 frames extracted
PersonId: 31e1fc90-84ff-498a-833c-1730aa00a310
FaceIds
=======
21a12afe-6aab-4593-ac77-d20d2bea7e8b
e2f4e1d8-ba2d-4c13-80b7-791f10949143
8c366814-3e8a-441e-86f2-c9edf967c04e
8510de4b-9a70-4d62-823d-471107f838da
c159bd2e-2f71-4c0d-8c85-179d76d96953
\end{myverbbox}
\acceptanceCriteria{Даже если обучение сервиса не происходило возможно добавлять несколько разных пользователей в сервис.
}
