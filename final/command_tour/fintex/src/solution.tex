\textit{\textbf{./api.py}}
\begin{minted}[fontsize=\footnotesize]{python}
from config import get
import cognitive_face as cf
from detector_util import FileLike, image_to_jpeg
from output_util import handle_error, print_error

KEY = get('faceapi.key')
cf.Key.set(KEY)

BASE_URL = get('faceapi.serviceUrl')
cf.BaseUrl.set(BASE_URL)

CONFIDENCE_THRESHOLD = 0.5

GROUP_ID = get('faceapi.groupId')


def create_group_if_not_exists():
    try:
        cf.person_group.create(GROUP_ID)
    except cf.CognitiveFaceException:
        pass


def list_groups():
    return handle_error(cf.person_group.lists())


def list_people_ids():
    try:
        return list(map(lambda person: person['personId'], cf.person.lists(GROUP_ID)))
    except cf.CognitiveFaceException as err:
        if err.code == 'PersonGroupNotFound':
            print_error('The group does not exist')


def is_group_up_to_date():
    try:
        response1 = cf.person_group.get_status(GROUP_ID)
        response2 = cf.person_group.get(GROUP_ID)

        return response1['status'] == 'succeeded' and response2['userData'] == 'ok'
    except cf.CognitiveFaceException:
        return False


def check_same_person(face_id):
    try:
        response = cf.face.identify([face_id], person_group_id=GROUP_ID, threshold=CONFIDENCE_THRESHOLD)
        candidates = response[0]['candidates']

        return len(candidates) != 0
    except cf.CognitiveFaceException:
        return False


def identify_person_id(faces_ids):
    if not is_group_up_to_date():
        print_error('The service is not ready')

    responses = cf.face.identify(faces_ids, person_group_id=GROUP_ID, threshold=CONFIDENCE_THRESHOLD)
    person_id = ''
    for response in responses:
        candidates = response['candidates']
        if len(candidates) != 1 or person_id not in ['', candidates[0]['personId']]:
            return None
        person_id = candidates[0]['personId']
    return person_id


def identify_person_id_old(image):
    response = cf.face.detect(FileLike(image_to_jpeg(image)), True, False)
    if len(response) != 1:
        print_error('There should be exactly 1 person on image')
    face_id = response[0]['faceId']
    response = cf.face.identify([face_id], person_group_id=GROUP_ID, threshold=CONFIDENCE_THRESHOLD)
    candidates = response[0]['candidates']

    print(candidates)

    if len(candidates) == 0:
        print_error('Cannot identify face')

    return candidates[0]['personId']


# Return face if there is exactly one face on the image
def detect(image):
    faces = cf.face.detect(FileLike(image_to_jpeg(image)), True, True, 'headPose')
    if len(faces) == 1:
        return faces[0]

    return None


def mark_group(status):
    cf.person_group.update(GROUP_ID, user_data=status)


def create_person(name, data=None):
    try:
        response = cf.person.create(GROUP_ID, name, data)
        mark_group('dirty')
        return response['personId']
    except cf.CognitiveFaceException as err:
        if err.code == 'PersonGroupNotFound':
            print_error('The group does not exist')


def delete_person(person_id):
    try:
        cf.person.delete(GROUP_ID, person_id)
        mark_group('dirty')
    except cf.CognitiveFaceException as err:
        if err.code == 'PersonGroupNotFound':
            print_error('The group does not exist')
        if err.code == 'PersonNotFound':
            print_error('The person does not exist')


def add_face(image, person_id):
    return cf.person.add_face(FileLike(image_to_jpeg(image)), GROUP_ID, person_id)['persistedFaceId']


def update_person_data(person_id, name, address):
    cf.person.update(GROUP_ID, person_id, name, address)
    print('Updated person %s data, name %s, address %s' % (person_id, name, address))


def get_person_address(person_id):
    response = cf.person.get(GROUP_ID, person_id)
    return response['userData']


def train():
    try:
        if is_group_up_to_date():
            print_error('Already trained')
        if len(cf.person.lists(GROUP_ID)) == 0:
            print_error('There is nothing to train')
        cf.person_group.train(GROUP_ID)
        mark_group('ok')
    except cf.CognitiveFaceException as err:
        if err.code == 'PersonGroupNotFound':
            print_error('There is nothing to train')


# Returns 0 <= confidence <= 1
def verify_person(person_id, face_id):
    response = cf.face.verify(face_id, person_group_id=GROUP_ID, person_id=person_id)
    return response['confidence']
\end{minted}

\textit{\textbf{./blockchain.py}}
\begin{minted}[fontsize=\footnotesize]{python}
import requests

from time import sleep
from ethereum.utils import privtoaddr
from checkers import check_pin_code
from config import get
from subprocess import check_output
import re
from web3 import Web3

from output_util import print_error

RPC_URL = get('network.rpcUrl')

web3 = Web3(Web3.HTTPProvider(RPC_URL))

try:
    PRIVATE_KEY = bytes.fromhex(get('network.privKey'))
except:
    PRIVATE_KEY = bytes(1)

DEFAULT_GAS = 100000

EMPTY_ADDRESS = '0x0000000000000000000000000000000000000000'

try:
    GAS_PRICE = int(1000000000 * requests.get(get('network.gasPriceUrl')).json()['fast'])
except requests.exceptions.RequestException:
    GAS_PRICE = get('network.defaultGasPrice')

REGISTRAR_ADDRESS = get('registrar.registrar.address')
CERTIFICATES_ADDRESS = get('registrar.payments.address')

compiled_registrar = check_output(["solc", "--optimize", "--bin", "--abi", "./Registrar.sol"]).decode()
compiled_registrar_abi = re.findall("ABI \\n(.*?)\\n", compiled_registrar)[0]
compiled_registrar_bytecode = re.findall("Binary: \\n(.*?)\\n", compiled_registrar)[0]

compiled_certificates = check_output(["solc", "--optimize", "--bin", "--abi", "Certificates.sol"]).decode()
compiled_certificates_abi = re.findall("ABI \\n(.*?)\\n", compiled_certificates)[0]
compiled_certificates_bytecode = re.findall("Binary: \\n(.*?)\\n", compiled_certificates)[0]

registrar = web3.eth.contract(address=REGISTRAR_ADDRESS,
                              abi=compiled_registrar_abi)
certificates = web3.eth.contract(address=CERTIFICATES_ADDRESS,
                                 abi=compiled_certificates_abi)


def check_registrar():
    if REGISTRAR_ADDRESS is None:
        print_error('No contract address')

    b1 = web3.eth.getCode(REGISTRAR_ADDRESS).hex()[2:]
    b2 = compiled_registrar_bytecode
    if b1 != b2[-len(b1):]:
        print_error('Seems that the contract address is not the registrar contract')


def check_certificates():
    if CERTIFICATES_ADDRESS is None:
        print_error('No contract address')

    b1 = web3.eth.getCode(CERTIFICATES_ADDRESS).hex()[2:]
    b2 = compiled_certificates_bytecode
    if b1 != b2[-len(b1):]:
        print_error('Seems that the contract address is not the certificates contract')


def private_key_to_address(private_key):
    return web3.toChecksumAddress(privtoaddr(private_key).hex())


def build_and_send_tx(private_key, to='', data='', value=0, nonce=None, gas_estimation=True, message=None):
    address = private_key_to_address(private_key)
    if nonce is None:
        nonce = web3.eth.getTransactionCount(address)
    tx = {
        'from': address,
        'to': to,
        'nonce': nonce,
        'data': data,
        'value': value,
        'gasPrice': GAS_PRICE
    }

    tx['gas'] = web3.eth.estimateGas(tx) if gas_estimation else DEFAULT_GAS

    signed = web3.eth.account.signTransaction(tx, private_key)

    tx_hash = web3.eth.sendRawTransaction(signed.rawTransaction)

    if message is not None:
        print(message)

    return wait_tx_receipt(tx_hash)


def encode_int(x, base=10):
    if type(x) == str:
        x = int(x, base)
    return x.to_bytes(32, byteorder='big')


# Waits until transaction is mined
def wait_tx_receipt(tx_hash, sleep_interval=0.5):
    while True:
        tx_receipt = web3.eth.getTransactionReceipt(tx_hash)
        if tx_receipt:
            return tx_receipt
        sleep(sleep_interval)


def get_owner_nonce():
    return web3.eth.getTransactionCount(private_key_to_address(PRIVATE_KEY))


# Deploys compiled contract, returns its address
def deploy_contract(contract_name, nonce):
    bytecode = ''
    if contract_name == 'registrar':
        bytecode = compiled_registrar_bytecode
    elif contract_name == 'certificates':
        bytecode = compiled_certificates_bytecode

    tx_receipt = build_and_send_tx(PRIVATE_KEY, data=bytecode, nonce=nonce)

    return {"address": tx_receipt['contractAddress'], "startBlock": tx_receipt['blockNumber']}


def get_owner():
    return registrar.functions.owner().call()


def get_requests(filter_type):
    requests = []

    head_addr = registrar.functions.headAddr().call()
    tail_addr = registrar.functions.tailAddr().call()
    node = registrar.functions.requests(head_addr).call()
    while node[1] != tail_addr:
        address = node[1]
        node = registrar.functions.requests(address).call()
        if filter_type == node[2]:
            requests += [(node[3], address)]
    return requests


def get_reg_requests():
    return get_requests(3)


def get_del_requests():
    return get_requests(4)


def change_owner(new_owner):
    if get_owner() != private_key_to_address(PRIVATE_KEY):
        print_error('Request cannot be executed')

    build_and_send_tx(PRIVATE_KEY, REGISTRAR_ADDRESS,
                      web3.keccak(text='transferOwnership(address)')[:4] + encode_int(new_owner[2:], 16))


def get_request_type(address):
    return registrar.functions.requests(address).call()[2]


def register_request(user_private_key, phone_number):
    try:
        tx_receipt = build_and_send_tx(user_private_key, REGISTRAR_ADDRESS,
                                       web3.keccak(text='registerRequest(uint256)')[:4] + encode_int(phone_number))

        return tx_receipt['transactionHash'].hex()
    except ValueError as err:
        if err.args[0]['code'] == -32016:
            request_type = get_request_type(private_key_to_address(user_private_key))
            if request_type == 3:
                print_error('Registration request already sent')
            else:
                print_error('Such phone number already registered')
        if err.args[0]['code'] == -32010:
            print_error('No funds to send the request')


def delete_request(user_private_key):
    request_type = get_request_type(private_key_to_address(user_private_key))
    try:
        tx_receipt = build_and_send_tx(user_private_key, REGISTRAR_ADDRESS, web3.keccak(text='deleteRequest()')[:4])

        return tx_receipt['transactionHash'].hex()
    except ValueError as err:
        if err.args[0]['code'] == -32016:
            if request_type == 4:
                print_error('Unregistration request already sent')
            else:
                print_error('Account is not registered yet')
        if err.args[0]['code'] == -32010:
            print_error('No funds to send the request')


def cancel_request(user_private_key):
    request_type = get_request_type(private_key_to_address(user_private_key))
    try:
        tx_receipt = build_and_send_tx(user_private_key, REGISTRAR_ADDRESS, web3.keccak(text='cancelRequest()')[:4])

        return request_type, tx_receipt['transactionHash'].hex()
    except ValueError as err:
        if err.args[0]['code'] == -32016:
            if request_type == 0:
                print_error('No requests found')
        if err.args[0]['code'] == -32010:
            print_error('No funds to send the request')


def confirm(address):
    if get('network.privKey') is None:
        print_error('No admin account found')

    try:

        tx_receipt = build_and_send_tx(PRIVATE_KEY,
                                       REGISTRAR_ADDRESS,
                                       web3.keccak(text='confirm(address)')[:4] + encode_int(address[2:], 16),
                                       gas_estimation=False)

        return tx_receipt['status'], tx_receipt['transactionHash'].hex()
    except ValueError as err:
        if err.args[0]['code'] == -32010:
            print_error('No funds to confirm the request')


def create_approval(user_private_key, value, time_to_expire):
    try:
        tx_receipt = build_and_send_tx(user_private_key, CERTIFICATES_ADDRESS,
                                       web3.keccak(text='approve(uint256)')[:4] + encode_int(time_to_expire),
                                       value)

        id_hex = tx_receipt['logs'][0]['topics'][1]

        message = web3.eth.account.signHash(id_hex, user_private_key)

        return id_hex.hex()[2:] + message['signature'].hex()[2:]
    except ValueError as err:
        if err.args[0]['code'] == -32010:
            print_error('No funds to create a certificate')


def use_approval(user_private_key, message):
    id = message[:64]
    r = message[64:128]
    s = message[128:192]
    v = message[192:]

    build_and_send_tx(user_private_key, CERTIFICATES_ADDRESS,
                      web3.keccak(text='useApproval(bytes32,uint8,bytes32,bytes32)')[:4]
                      + bytes.fromhex(id)
                      + encode_int(v, 16)
                      + bytes.fromhex(r)
                      + bytes.fromhex(s))

    return id


def cancel_approvals(user_private_key):
    build_and_send_tx(user_private_key, CERTIFICATES_ADDRESS,
                      web3.keccak(text='cancel()')[:4])


def send_transaction(user_private_key, to, value):
    try:
        addr = phone_to_address(to)
        if addr is None:
            print_error('No account with the phone number +%s' % to)
        tx_receipt = build_and_send_tx(user_private_key, to=addr, value=value,
                                       message='Payment of %s to +%s scheduled' % (normalize_value(value), to))
        return tx_receipt['transactionHash'].hex()
    except ValueError as err:
        if err.args[0]['code'] == -32010:
            print_error('No funds to send the payment')


def pin_code_to_private_key(pin_code):
    pin_code = check_pin_code(pin_code)
    person_id = get('person.id')
    if person_id is None:
        print_error('ID is not found')
    return get_private_key(person_id, pin_code)


# Evaluate private key by id and pin_code
def get_private_key(person_id, pin_code):
    id = bytes.fromhex(person_id.replace('-', ''))

    a = web3.keccak(bytes(0))
    b = web3.keccak(a + id + bytes([int(pin_code[0])]))
    c = web3.keccak(b + id + bytes([int(pin_code[1])]))
    d = web3.keccak(c + id + bytes([int(pin_code[2])]))
    return web3.keccak(d + id + bytes([int(pin_code[3])]))


def normalize_value(value, custom_currency=None):
    if custom_currency is not None:
        return ('%.6f' % web3.fromWei(value, custom_currency)).rstrip('0').rstrip('.') \
               + ' %s' % (custom_currency if custom_currency != 'ether' else 'poa')

    for currency in ['ether', 'finney', 'szabo', 'gwei', 'mwei', 'kwei', 'wei']:
        if web3.fromWei(value, currency) >= 1:
            return ('%.6f' % web3.fromWei(value, currency)).rstrip('0').rstrip('.') \
                   + ' %s' % (currency if currency != 'ether' else 'poa')
    return '0 poa'


def phone_to_address(phone):
    address = registrar.functions.db(int(phone)).call()
    if address == EMPTY_ADDRESS:
        return None
    return address


def address_to_phone(address, block='latest'):
    phone = registrar.functions.db_rev(web3.toChecksumAddress(address)).call(block_identifier=block)

    if phone == 0:
        return 'UNKNOWN'
    return '+%d' % phone


def get_balance(address):
    return web3.eth.getBalance(address)


def check_address(address):
    if not web3.isChecksumAddress(address):
        print_error('Wrong address format')
    return address


def get_tx_list(address):
    return requests.get(
        'https://blockscout.com/poa/sokol/api?module=account&action=txlist&address=%s&startblock=%s' % (
            address, get('registrar.registrar.startBlock'))
    ).json()['result']
\end{minted}

\textit{\textbf{./checkers.py}}
\begin{minted}[fontsize=\footnotesize]{python}
from output_util import print_error
from datetime import datetime as dt


def check_pin_code(pin_code):
    if len(pin_code) != 4 or not pin_code.isdecimal():
        print_error('Pin code should have length 4 and contain only digits')
    return pin_code


def check_name(name):
    if len(name) > 128 or not name.isalpha():
        print_error('Name should have length at most 128 and contain only letters')
    return name


def check_phone_number(phone_number):
    if len(phone_number) != 12 or not phone_number[1:].isdecimal() or phone_number[0] != '+':
        print_error('Incorrect phone number')
    return phone_number[1:]


def check_date(date_str):
    try:
        date = dt.strptime(date_str, '%H:%M %d.%m.%Y')
    except ValueError:
        print_error('Expiration date is invalid')

    delta = int((date - dt.now()).total_seconds())
    if delta <= 0:
        print_error('Expiration date is invalid')
    return delta


def check_ticket(ticket):
    return ticket


def check_message(message):
    if len(message) != 140:
        print_error('Message should have length 140')
    return message
\end{minted}

\textit{\textbf{./config.py}}
\begin{minted}[fontsize=\footnotesize]{python}
import json

# key format a.b.c => ./a.json -> "b" -> "c"
import os


def get(key):
    try:
        keys = key.split('.')
        config = json.load(open('./%s.json' % keys[0]))
        cur = config
        for k in keys[1:]:
            cur = cur[k]

        return cur
    except (IOError, json.JSONDecodeError, KeyError):
        return None


def set(key, value):
    keys = key.split('.')
    try:
        config = json.load(open('./%s.json' % keys[0]))
    except (IOError, json.JSONDecodeError, KeyError):
        config = dict()

    cur = config
    for k in keys[1:-1]:
        cur = cur[k]
    cur[keys[-1]] = value
    open('./%s.json' % keys[0], 'w+').write(json.dumps(config))


def clear(key):
    try:
        os.remove('./%s.json' % key)
    except OSError:
        pass
\end{minted}

\textit{\textbf{./contracts/Certificates.sol}}
\begin{minted}[fontsize=\footnotesize]{javascript}
pragma solidity ^0.5.2;


contract Certificates {
    event CertificateCreated(bytes32 indexed id);
    event CertificateUsed(bytes32 indexed id);
    event CertificateWithdrew(bytes32 indexed id);

    struct Approval {
        address sender;
        uint value;
        uint expire_time;
        bytes32 prev;
        bytes32 next;
    }

    mapping(address => uint) public lastN;

    // sha(sender, id) => approval
    mapping(bytes32 => Approval) public approvals;

    // address => last approval ticket
    mapping(address => bytes32) public lastTicket;

    function approve(uint time) public payable {
        require(msg.value > 0);

        bytes32 prevId = lastTicket[msg.sender];
        if (prevId == bytes32(0))
            prevId = keccak256(abi.encodePacked(msg.sender, uint(0)));

        uint n = ++lastN[msg.sender];

        bytes32 id = keccak256(abi.encodePacked(msg.sender, n));
        lastTicket[msg.sender] = id;

        approvals[prevId].next = id;
        approvals[id] = Approval(msg.sender, msg.value, now + time, prevId, bytes32(0));

        emit CertificateCreated(id);
    }

    function cancel() public {
        bytes32 id = keccak256(abi.encodePacked(msg.sender, uint(0)));
        while (approvals[id].next != bytes32(0)) {
            id = approvals[id].next;
            if (now > approvals[id].expire_time)
                cancelApproval(id);
        }
    }

    function cancelApproval(bytes32 ticket) public {
        require(approvals[ticket].value > 0);
        require(approvals[ticket].sender == msg.sender);
        require(now > approvals[ticket].expire_time);

        approvals[approvals[ticket].prev].next = approvals[ticket].next;
        if (approvals[ticket].next != bytes32(0))
            approvals[approvals[ticket].next].prev = approvals[ticket].prev;
        else
            lastTicket[msg.sender] = approvals[ticket].prev;

        msg.sender.transfer(approvals[ticket].value);

        delete approvals[ticket];

        emit CertificateWithdrew(ticket);
    }

    function useApproval(bytes32 ticket, uint8 v, bytes32 r, bytes32 s) public {
        require(approvals[ticket].value > 0);
        require(ecrecover(ticket, v, r, s) == approvals[ticket].sender);
        require(now <= approvals[ticket].expire_time);

        approvals[approvals[ticket].prev].next = approvals[ticket].next;
        if (approvals[ticket].next != bytes32(0))
            approvals[approvals[ticket].next].prev = approvals[ticket].prev;
        else
            lastTicket[msg.sender] = approvals[ticket].prev;

        msg.sender.transfer(approvals[ticket].value);

        delete approvals[ticket];

        emit CertificateUsed(ticket);
    }
}
\end{minted}

\textit{\textbf{./contracts/Registrar.sol}}
\begin{minted}[fontsize=\footnotesize]{javascript}
pragma solidity ^0.5.2;


contract Registrar {
    event Register(uint indexed phone_number, address addr);
    event Unregister(uint indexed phone_number);
    event RegistrationRequest(address indexed sender);
    event UnregistrationRequest(address indexed sender);
    event RegistrationCanceled(address indexed sender);
    event UnregistrationCanceled(address indexed sender);
    event RegistrationConfirmed(address indexed sender);
    event UnregistrationConfirmed(address indexed sender);

    struct Account {
        uint phone;
        address addr;
    }

    enum NodeType {
        NONE, HEAD, TAIL, REG, DEL
    }

    struct Node {
        address prev;
        address next;
        NodeType t;
        uint data;
    }

    // Double linked list
    address public headAddr = address(0);
    address public tailAddr = address(2 ** 160 - 1);
    // Sender => Request node
    mapping(address => Node) public requests;

    address public owner;

    // Phone number => Address
    mapping(uint => address) public db;

    // Address => Phone number
    mapping(address => uint) public db_rev;

    constructor() public {
        owner = msg.sender;
        // head
        requests[headAddr] = Node(headAddr, tailAddr, NodeType.HEAD, 0);
        // tail
        requests[tailAddr] = Node(headAddr, tailAddr, NodeType.TAIL, 0);
    }

    // Only contract owner can call these methods
    modifier onlyOwner() {
        require(msg.sender == owner);
        _;
    }

    function transferOwnership(address _owner) onlyOwner public {
        owner = _owner;
    }


    function registerRequest(uint phone) public {
        require(phone <= 1e12);
        require(db[phone] == address(0));
        require(requests[msg.sender].t == NodeType.NONE);

        requests[msg.sender].t = NodeType.REG;
        requests[msg.sender].data = phone;

        requests[msg.sender].next = tailAddr;
        requests[requests[tailAddr].prev].next = msg.sender;
        requests[msg.sender].prev = requests[tailAddr].prev;
        requests[tailAddr].prev = msg.sender;

        emit RegistrationRequest(msg.sender);
    }

    function deleteRequest() public {
        require(db_rev[msg.sender] != 0);
        require(requests[msg.sender].t == NodeType.NONE);

        requests[msg.sender].t = NodeType.DEL;

        requests[msg.sender].next = tailAddr;
        requests[requests[tailAddr].prev].next = msg.sender;
        requests[msg.sender].prev = requests[tailAddr].prev;
        requests[tailAddr].prev = msg.sender;

        emit UnregistrationRequest(msg.sender);
    }

    function confirm(address addr) onlyOwner public {
        require(requests[addr].t == NodeType.REG || requests[addr].t == NodeType.DEL);

        if (requests[addr].t == NodeType.REG) {
            db_rev[addr] = requests[addr].data;
            db[db_rev[addr]] = addr;
            emit RegistrationConfirmed(addr);
        }
        else {
            delete db[db_rev[addr]];
            delete db_rev[addr];
            emit UnregistrationConfirmed(addr);
        }
        requests[addr].t = NodeType.NONE;
        requests[requests[addr].prev].next = requests[addr].next;
        requests[requests[addr].next].prev = requests[addr].prev;
    }

    function cancelRequest() public {
        require(requests[msg.sender].t != NodeType.NONE);

        if (requests[msg.sender].t == NodeType.REG)
            emit RegistrationCanceled(msg.sender);
        if (requests[msg.sender].t == NodeType.DEL)
            emit UnregistrationCanceled(msg.sender);

        requests[msg.sender].t = NodeType.NONE;
        requests[requests[msg.sender].prev].next = requests[msg.sender].next;
        requests[requests[msg.sender].next].prev = requests[msg.sender].prev;
    }
}
\end{minted}

\textit{\textbf{./detector\_util.py}}
\begin{minted}[fontsize=\footnotesize]{python}
import cv2
import os
import dlib

from output_util import print_error

PREDICTOR_PATH = '/opt/shape_predictor_68_face_landmarks.dat' if os.path.exists(
    '/opt/shape_predictor_68_face_landmarks.dat') else './shape_predictor_68_face_landmarks.dat'
EYE_THRESHOLD = 0.22
MOUTH_THRESHOLD = 0.3

detector = dlib.get_frontal_face_detector()
predictor = dlib.shape_predictor(PREDICTOR_PATH)


class FileLike:
    def __init__(self, image):
        self.image = image

    def read(self):
        return self.image


def shape_to_list(shape):
    coords = [None] * 68

    for i in range(0, 68):
        coords[i] = (shape.part(i).x, shape.part(i).y)

    return coords


def right_eye_open(points):
    eye = points[36:42]
    return aspect_ratio(eye) > EYE_THRESHOLD


def left_eye_open(points):
    eye = points[42:48]
    return aspect_ratio(eye) > EYE_THRESHOLD


def mouth_open(points):
    mouth = [points[60], points[61], points[63], points[64], points[65], points[67]]
    return aspect_ratio(mouth) > MOUTH_THRESHOLD


def aspect_ratio(eye):
    return (dist(eye[1], eye[5]) + dist(eye[2], eye[4])) / (2 * dist(eye[0], eye[3]))


def dist(a, b):
    return ((a[0] - b[0]) ** 2 + (a[1] - b[1]) ** 2) ** 0.5


def landmarks(image):
    gray = cv2.cvtColor(image, cv2.COLOR_BGR2GRAY)

    dets, scores, idx = detector.run(gray, 0, -1)

    if len(dets) == 0:
        return None

    return shape_to_list(predictor(image, dets[0]))


def image_to_jpeg(image):
    status, encoded_image = cv2.imencode('.jpeg', image)
    if not status:
        print_error('Failed to encode frame into jpeg')
    return encoded_image.tostring()

\end{minted}

\textit{\textbf{./face\_conditions.py}}
\begin{minted}[fontsize=\footnotesize]{python}
import cv2

from detector_util import landmarks, left_eye_open, right_eye_open, mouth_open

DEFAULT_ANGLE_DELTA = 3


class FaceConditions:
    def __init__(self, label, yaw=None, roll=None, delta=DEFAULT_ANGLE_DELTA,
                 left_eye=None, right_eye=None, mouth=None):
        self.label = label
        self.completed = False

        self.yaw = yaw
        self.yaw_delta = delta
        self.roll = roll
        self.roll_delta = delta
        self.left_eye = left_eye
        self.right_eye = right_eye
        self.mouth = mouth

    def check_face(self, image, face_api_response=None):
        if self.completed:
            return

        if self.yaw is not None and self.roll is not None and face_api_response is not None:
            yaw_value = face_api_response['faceAttributes']['headPose']['yaw']
            roll_value = face_api_response['faceAttributes']['headPose']['roll']

            if self.yaw is not None and (yaw_value < self.yaw - self.yaw_delta or yaw_value > self.yaw + self.yaw_delta):
                return
            if self.roll is not None and (
                    roll_value < self.roll - self.roll_delta or roll_value > self.roll + self.roll_delta):
                return

        points = landmarks(image)
        if points is None:
            return False

        if self.left_eye is not None and self.left_eye != left_eye_open(points):
            return
        if self.right_eye is not None and self.right_eye != right_eye_open(points):
            return
        if self.mouth is not None and self.mouth != mouth_open(points):
            return

        self.completed = True

        return True
\end{minted}

\textit{\textbf{./faceid.py}}
\begin{minted}[fontsize=\footnotesize]{python}
from argparse import ArgumentParser
from datetime import datetime

from blockchain import register_request, delete_request, cancel_request, get_balance, \
    private_key_to_address, send_transaction, create_approval, use_approval, cancel_approvals, pin_code_to_private_key, \
    normalize_value, check_registrar, check_certificates, get_tx_list, address_to_phone
from config import get, set, clear
from output_util import print_error
from validators import PINPhoneValidate, TransactValidate, ApproveValidate, PINTicketValidate
from video_detector import simple_identify, identify

parser = ArgumentParser(prog='Faceid service')

parser.add_argument('--find',
                    nargs=1,
                    metavar='VIDEO',
                    help='Identifies person, stores person id in person.json')
parser.add_argument('--actions',
                    action='store_true',
                    help='Generates needed face actions')
parser.add_argument('--add',
                    nargs=2,
                    metavar=('PIN_CODE', 'PHONE_NUMBER'),
                    action=PINPhoneValidate,
                    help='Sends register request')
parser.add_argument('--del',
                    nargs=1,
                    metavar='PIN_CODE',
                    type=pin_code_to_private_key,
                    dest='delete',
                    help='Sends delete request')
parser.add_argument('--cancel',
                    nargs=1,
                    metavar='PIN_CODE',
                    type=pin_code_to_private_key,
                    help='Cancels register or unregister request')
parser.add_argument('--balance',
                    nargs=1,
                    metavar='PIN_CODE',
                    type=pin_code_to_private_key,
                    help='Prints user balance')
parser.add_argument('--send',
                    nargs=3,
                    metavar=('PIN_CODE', 'PHONE_NUMBER', 'VALUE'),
                    action=TransactValidate,
                    help='Sends money to another user')
parser.add_argument('--gift',
                    nargs=3,
                    metavar=('PIN_CODE', 'VALUE', 'EXPIRE_DATE'),
                    action=ApproveValidate,
                    help='Creates approval ticket')
parser.add_argument('--receive',
                    nargs=2,
                    metavar=('PIN_CODE', 'TICKET'),
                    action=PINTicketValidate,
                    help='Receives money via given ticket')
parser.add_argument('--withdraw',
                    nargs=1,
                    metavar='PIN_CODE',
                    type=pin_code_to_private_key,
                    help='Withdraw money from unused ticket')
parser.add_argument('--ops',
                    nargs=1,
                    metavar='PIN_CODE',
                    type=pin_code_to_private_key,
                    help='Prints all operations')
parser.add_argument('--opsall',
                    nargs=1,
                    metavar='PIN_CODE',
                    type=pin_code_to_private_key,
                    help='Prints all operations including certificates')

args = parser.parse_args()

if args.find:
    clear('person')
    video = args.find[0]

    actions = get('actions')
    if actions is None:
        person_id = simple_identify(video)
        set('person.id', person_id)
        print('%s identified' % person_id)
    else:
        person_id = identify(video, actions)
        set('person.id', person_id)
        print('%s identified' % person_id)

if args.actions:
    import json, random

    actions = random.choice([
        {'actions': ["OpenMouth", "YawLeft", "CloseLeftEye"]},
        {'actions': ["OpenMouth", "YawLeft", "CloseRightEye"]},
        {'actions': ["OpenMouth", "CloseRightEye", "YawLeft"]},
        {'actions': ["OpenMouth", "YawRight", "CloseRightEye"]}
    ])
    with open('actions.json', 'w') as out:
        json.dump(actions, out)

if args.add:
    check_registrar()
    private_key = args.add[0]
    phone = args.add[1]

    tx_hash = register_request(private_key, phone)

    print('Registration request sent by %s' % tx_hash)

if args.delete:
    check_registrar()
    private_key = args.delete[0]

    tx_hash = delete_request(private_key)

    print('Unregistration request sent by %s' % tx_hash)

if args.cancel:
    check_registrar()
    private_key = args.cancel[0]

    type, tx_hash = cancel_request(private_key)

    if type == 3:
        print('Registration canceled by %s' % tx_hash)
    else:
        print('Unregistration canceled by %s' % tx_hash)

if args.balance:
    private_key = args.balance[0]
    address = private_key_to_address(private_key)

    print('Your balance is %s' % normalize_value(get_balance(address)))

if args.send:
    check_registrar()
    private_key = args.send[0]
    to = args.send[1]
    value = args.send[2]

    tx_hash = send_transaction(private_key, to, value)

    print('Transaction Hash: %s' % tx_hash)

if args.gift:
    check_certificates()
    private_key = args.gift[0]
    value = args.gift[1]
    expire_time = args.gift[2]

    certificate = create_approval(private_key, value, expire_time)

    print(certificate)

if args.receive:
    check_certificates()
    private_key = args.receive[0]
    certificate = args.receive[1]

    ticket = use_approval(private_key, certificate)

    print(ticket)

if args.withdraw:
    check_certificates()
    private_key = args.withdraw[0]

    cancel_approvals(private_key)

    print('Cancelled approvals')

if args.ops:
    check_registrar()
    check_certificates()

    address = private_key_to_address(args.ops[0]).lower()

    tx_list = get_tx_list(address)

    if len(tx_list) == 0:
        print_error('No operations found')

    print('Operations:')
    for tx in tx_list:
        date = datetime.fromtimestamp(int(tx['timeStamp']))

        if int(tx['blockNumber']) >= get('registrar.registrar.startBlock'):
            if tx['to'] == address:
                from_phone = address_to_phone(tx['from'], int(tx['blockNumber']))
                if from_phone != 'UNKNOWN':
                    print('%s FROM: %s %s' % (
                    date.strftime('%H:%M:%S %d.%m.%Y'), from_phone, normalize_value(int(tx['value']), 'ether')))
            else:
                to_phone = address_to_phone(tx['to'], int(tx['blockNumber']))
                if to_phone != 'UNKNOWN':
                    print('%s   TO: %s %s' % (
                    date.strftime('%H:%M:%S %d.%m.%Y'), to_phone, normalize_value(int(tx['value']), 'ether')))
\end{minted}

\textit{\textbf{./face-management.py}}
\begin{minted}[fontsize=\footnotesize]{python}
from argparse import ArgumentParser
from threading import Thread
from time import sleep

from api import list_people_ids, delete_person, train, create_person, add_face, create_group_if_not_exists
from output_util import print_error
from video_detector import detect_all, detect_middle_faces, detect_roll_faces, detect_yaw_faces, \
    detect_open_mouth_faces, detect_open_eyes_faces

parser = ArgumentParser(prog='Faces management')

parser.add_argument('--simple-add',
                    nargs=1,
                    metavar='VIDEO',
                    help='Adds 5 faces from video')
parser.add_argument('--add',
                    nargs='+',
                    metavar='VIDEO',
                    help='Adds only needed faces from 1-5 videos')
parser.add_argument('--list',
                    action='store_true',
                    help='Prints all registered person ids')
parser.add_argument('--del',
                    nargs=1,
                    metavar='ID',
                    dest='delete',
                    help='Deletes person')
parser.add_argument('--train',
                    action='store_true',
                    help='Starts training')

args = parser.parse_args()

if args.simple_add:
    video = args.simple_add[0]

    faces = detect_all(video)
    if faces is None:
        print_error('Video does not contain any face')

    create_group_if_not_exists()

    print('%d frames extracted' % len(faces))

    person_id = create_person('1')

    print('PersonId: %s' % person_id)
    print('FaceIds')
    print('=======')

    for face in faces:
        print(add_face(face, person_id))

if args.add:
    person_id = ''
    faces = []

    done = 0
    results = [0, 0, 0, 0, 0]
    if len(args.add) > 0:
        def target(video):
            global results, done
            results[0] = detect_middle_faces(video)
            done += 1

        video = args.add[0]

        t = Thread(target=target, args=(video,))
        t.run()

    if len(args.add) > 1:
        def target(video):
            global results, done
            results[1] = detect_roll_faces(video)
            done += 1

        video = args.add[1]

        t = Thread(target=target, args=(video,))
        t.run()

    if len(args.add) > 2:
        def target(video):
            global results, done
            results[2] = detect_yaw_faces(video)
            done += 1

        video = args.add[2]

        t = Thread(target=target, args=(video,))
        t.run()

    if len(args.add) > 3:
        def target(video):
            global results, done
            results[3] = detect_open_mouth_faces(video)
            done += 1

        video = args.add[3]

        t = Thread(target=target, args=(video,))
        t.run()

    if len(args.add) > 4:
        def target(video):
            global results, done
            results[4] = detect_open_eyes_faces(video)
            done += 1

        video = args.add[4]

        t = Thread(target=target, args=(video,))
        t.run()

    while len(args.add) > done:
        sleep(0.5)

    if len(args.add) > 0:
        if results[0] is None:
            print_error('Base video does not follow requirements')

        faces += results[0]

    if len(args.add) > 1:
        if results[1] is None:
            print_error('Roll video does not follow requirements')

        faces += results[1]

    if len(args.add) > 2:
        if results[2] is None:
            print_error('Yaw video does not follow requirements')

        faces += results[2]

    if len(args.add) > 3:
        if results[3] is None:
            print_error('Video to detect open mouth does not follow requirements')

        faces += results[3]

    if len(args.add) > 4:
        if results[4] is None:
            print_error('Video to detect open eyes does not follow requirements')

        faces += results[4]

    create_group_if_not_exists()
    person_id = create_person('1')

    print('%d frames extracted' % len(faces))
    print('PersonId: %s' % person_id)
    print('FaceIds')
    print('=======')

    for face in faces:
        print(add_face(face, person_id))

if args.list:
    list_ids = list_people_ids()

    if len(list_ids) == 0:
        print_error('No persons found')

    print('Persons IDs:')

    for id in list_ids:
        print(id)

if args.delete:
    id = args.delete[0]

    delete_person(id)

    print('Person deleted')

if args.train:
    train()

    print('Training successfully started')
\end{minted}

\textit{\textbf{./kyc.py}}
\begin{minted}[fontsize=\footnotesize]{python}
from argparse import ArgumentParser

from blockchain import get_reg_requests, get_del_requests, confirm, phone_to_address, check_address, check_registrar
from checkers import check_phone_number
from output_util import print_error

parser = ArgumentParser(prog='KYC')

parser.add_argument('--list',
                    nargs=1,
                    metavar='REQUEST_TYPE',
                    help='Prints out all requests of given type')
parser.add_argument('--confirm',
                    nargs=1,
                    metavar='ADDRESS',
                    type=check_address,
                    help='Confirms request related to the given address')
parser.add_argument('--get',
                    nargs=1,
                    type=check_phone_number,
                    metavar='PHONE',
                    help='Prints address associated with given phone number')

args = parser.parse_args()

if args.list:
    check_registrar()
    request_type = args.list[0]

    lst = []
    if request_type == 'add':
        lst = sorted(get_reg_requests())
        if len(lst) == 0:
            print_error('No KYC registration requests found')
    elif request_type == 'del':
        lst = sorted(get_del_requests())
        if len(lst) == 0:
            print_error('No KYC unregistration requests found')

    for phone, address in lst:
        print('%s: +%s' % (address, str(phone)))

if args.confirm:
    check_registrar()
    address = args.confirm[0]

    status, tx_hash = confirm(address)

    if status == 1:
        print('Confirmed by %s' % tx_hash)
    else:
        print('Failed but included in %s' % tx_hash)

if args.get:
    check_registrar()
    phone = args.get[0]

    addr = phone_to_address(phone)
    if addr is None:
        print_error('Correspondence not found')
    print('Registered correspondence: %s' % phone_to_address(phone))
\end{minted}

\textit{\textbf{./output\_util.py}}
\begin{minted}[fontsize=\footnotesize]{python}
from json import dumps

def print_error(message):
    print(message)
    quit(0)

def print_json(obj):
    print(dumps(obj, indent=4))

def handle_error(response):
    if type(response) == dict and response.get('error'):
        print_error(response['error'])
    return response
\end{minted}

\textit{\textbf{./setup.py}}
\begin{minted}[fontsize=\footnotesize]{python}
from time import sleep

from argparse import ArgumentParser

from blockchain import deploy_contract, get_owner, change_owner, get_owner_nonce
from config import set

parser = ArgumentParser(prog='Setup')

parser.add_argument('--deploy',
                    action='store_true',
                    help='Deploys registrar and certificates contracts to the blockchain')
parser.add_argument('--owner',
                    nargs=1,
                    metavar='CONTRACT',
                    help='Prints current owner of the contract')
parser.add_argument('--chown',
                    nargs=2,
                    metavar=('CONTRACT', 'NEW_OWNER'),
                    help='Changes owner of the specified contract')


args = parser.parse_args()

if args.deploy:
    nonce = get_owner_nonce()
    registrar = deploy_contract('registrar', nonce)
    print('KYC Registrar: %s' % registrar['address'])
    set('registrar.registrar', registrar)

    certificates = deploy_contract('certificates', nonce + 1)
    print('Payment Handler: %s' % certificates['address'])
    set('registrar.payments', certificates)

if args.owner:
    contract = args.owner[0]

    if contract == 'registrar':
        print('Admin account: %s' % get_owner())

if args.chown:
    contract = args.chown[0]
    new_owner = args.chown[1]

    if contract == 'registrar':
        change_owner(new_owner)

        print('New admin account: %s' % new_owner)
\end{minted}

\textit{\textbf{./validators.py}}
\begin{minted}[fontsize=\footnotesize]{python}
from argparse import Action

from checkers import check_pin_code, check_phone_number, check_date, check_ticket

from blockchain import pin_code_to_private_key, phone_to_address


# Substitutes pin with private_key
class PINPhoneValidate(Action):
    def __call__(self, parser, args, values, option_string=None):
        pin_code, phone_number = values
        res = [pin_code_to_private_key(pin_code), check_phone_number(phone_number)]
        setattr(args, self.dest, res)


# Substitutes pin with private_key
class PINTicketValidate(Action):
    def __call__(self, parser, args, values, option_string=None):
        pin_code, ticket = values
        res = [pin_code_to_private_key(pin_code), check_ticket(ticket)]
        setattr(args, self.dest, res)


# Substitutes pin with private_key
# Substitutes phone with address
class PersonValidate(Action):
    def __call__(self, parser, args, values, option_string=None):
        name, phone_number, pin_code = values
        res = [pin_code_to_private_key(name),
               phone_to_address(check_phone_number(phone_number)),
               check_pin_code(pin_code)]
        setattr(args, self.dest, res)


# Substitutes pin with private_key
class TransactValidate(Action):
    def __call__(self, parser, args, values, option_string=None):
        pin_code, phone_number, amount = values
        res = [pin_code_to_private_key(pin_code),
               check_phone_number(phone_number),
               int(amount)]
        setattr(args, self.dest, res)


# Substitutes pin with private_key
class ApproveValidate(Action):
    def __call__(self, parser, args, values, option_string=None):
        pin_code, value, date_to_expire = values
        res = [pin_code_to_private_key(pin_code), int(value), check_date(date_to_expire)]
        setattr(args, self.dest, res)
\end{minted}

\textit{\textbf{./video\_detector.py}}
\begin{minted}[fontsize=\footnotesize]{python}
import cv2
import imutils

from config import get, set, clear
from api import detect, verify_person, check_same_person, identify_person_id
from face_conditions import FaceConditions
from output_util import print_error

WIDTH = 1000
CONDITIONS_TO_VERIFY = 3

def detect_all(video):
    faces_images = []

    cap = cv2.VideoCapture(video)
    success, frame = cap.read()
    count = 0

    while success:
        if count % 5 == 0:
            frame = imutils.resize(frame, width=WIDTH)

            face_api_response = detect(frame)

            if face_api_response is not None:
                if check_same_person(face_api_response['faceId']):
                    print_error('The same person already exists')

                faces_images.append(frame)

                if len(faces_images) == 5:
                    return faces_images

        count += 1
        success, frame = cap.read()

    return None


def detect_by_conditions(video, conditions, face_api=True):
    cur = 0
    faces_images = []

    cap = cv2.VideoCapture(video)
    success, frame = cap.read()
    count = 0

    while success:
        if count % 3 == 0:
            frame = imutils.resize(frame, width=WIDTH)

            face_api_response = detect(frame) if face_api else 0
            if face_api_response is not None and conditions[cur].check_face(frame, face_api_response):
                if face_api_response != 0 and check_same_person(face_api_response['faceId']):
                    print_error('The same person already exists')

                faces_images.append(frame)
                cur += 1

                if cur == len(conditions):
                    return faces_images

        count += 1
        success, frame = cap.read()

    return None


def detect_by_conditions_in_any_order(video, conditions, face_api=True):
    completed = 0
    faces_images = []

    cap = cv2.VideoCapture(video)
    success, frame = cap.read()

    while success:
        frame = imutils.resize(frame, width=WIDTH)
        face_api_response = detect(frame) if face_api else 0
        if face_api_response is not None:
            for condition in conditions:
                if condition.check_face(frame, face_api_response):
                    faces_images.append(frame)
                    completed += 1
                    break

            if completed == len(conditions):
                return faces_images

        success, frame = cap.read()

    return None


def detect_middle_faces(video):
    return detect_by_conditions(video, [
        FaceConditions('middle_face1', yaw=0, roll=0, delta=5),
        FaceConditions('middle_face2', yaw=0, roll=0, delta=5),
        FaceConditions('middle_face3', yaw=0, roll=0, delta=5),
        FaceConditions('middle_face4', yaw=0, roll=0, delta=5),
        FaceConditions('middle_face5', yaw=0, roll=0, delta=5)
    ])


def detect_roll_faces(video):
    return detect_by_conditions(video, [
        FaceConditions('roll_left_30', roll=30),
        FaceConditions('roll_left_15', roll=15),
        FaceConditions('roll_middle_0', roll=0),
        FaceConditions('roll_right_15', roll=-15),
        FaceConditions('roll_right_30', roll=-30)
    ])


def detect_yaw_faces(video):
    return detect_by_conditions(video, [
        FaceConditions('yaw_left_20', yaw=20),
        FaceConditions('yaw_left_10', yaw=10),
        FaceConditions('yaw_middle', yaw=0),
        FaceConditions('yaw_right_10', yaw=-10),
        FaceConditions('yaw_right_20', yaw=-20)
    ])


def detect_open_mouth_faces(video):
    return detect_by_conditions(video, [
        FaceConditions('open_mouth', mouth=True)
    ], face_api=False)


def detect_open_eyes_faces(video):
    return detect_by_conditions_in_any_order(video, [
        FaceConditions('closed_left_eye', left_eye=False, right_eye=True),
        FaceConditions('closed_right_eye', left_eye=True, right_eye=False),
    ], face_api=False)


def simple_identify(video):
    cap = cv2.VideoCapture(video)
    success, frame = cap.read()
    count = 0

    faces_ids = []

    while success:
        if count % 5 == 0:
            frame = imutils.resize(frame, width=WIDTH)

            face_api_response = detect(frame)

            if face_api_response is not None:
                faces_ids += [face_api_response['faceId']]

                if len(faces_ids) == 5:
                    break

        count += 1
        success, frame = cap.read()

    if len(faces_ids) < 5:
        print_error('The video does not follow requirements')

    person_id = identify_person_id(faces_ids)

    if person_id is None:
        print_error('The person was not found')

    return person_id


def action_to_face_condition(action):
    return action


def identify(video, actions):
    cap = cv2.VideoCapture(video)
    success, frame = cap.read()

    actions = list(map(action_to_face_condition, actions))

    count = 0
    cur = 0

    faces_ids = []

    while success:
        if count % 5 == 0:
            frame = imutils.resize(frame, width=WIDTH)

            face_api_response = detect(frame)

            if face_api_response is not None and actions[cur].check(frame, face_api_response):
                faces_ids += [face_api_response['faceId']]

                cur += 1
                if cur == len(actions):
                    break

        count += 1
        success, frame = cap.read()

    if cur < len(actions):
        print_error('The video does not follow requirements')

    person_id = identify_person_id(faces_ids)

    if person_id is None:
        print_error('The person was not found')

    return person_id
\end{minted}

