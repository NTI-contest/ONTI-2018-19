В заключительном этапе олимпиады баллы участника складываются
из двух частей: он получает баллы за индивидуальное решение задач по
предметам (математика и информатика) и за командное решение
практической задачи.

\begin{center}
    $S = S_i + S_m + S_t$
\end{center}

где $S_i$ --- сумма баллов по информатике, набранная в рамках индивидуального
тура заключительного этапа (максимум 300 баллов), поделённая на 3; $S_i$ ---
сумма баллов о математике, набранная в рамках индивидуального тура
заключительного этапа (максимум 100 баллов); $S_t$ --- количество баллов,
набранное в рамках командного тура заключительного этапа (максимум
400 баллов).

Критерий определения победителей и призеров:

\begin{center}
\begin{tabular}{ |c|c|c| }
 \hline
  & \textbf{Призеры} & \textbf{Победители} \\
 \hline
 Набранные баллы & от $191$ до $240$ баллов & от $240$ баллов и выше \\
 \hline
\end{tabular}
\end{center}
