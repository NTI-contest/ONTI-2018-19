\begin{itemize}
    \item В первые день соревнований все члены команд получают ноутбук со следующим набором
    установленного программного обеспечения:
    \begin{itemize}
        \item набор Python инструментария Anaconda c установленными библиотеками \texttt{web3} (\texttt{v5.0.0a3}), \texttt{cognitive\_face}, \texttt{opencv-python};
        \item среды разработки PyCharm и WingIDE;
        \item компилятор языка Solidity \texttt{solc} (\texttt{v0.5.4});
        \item клиент системы ведения версий \texttt{git};
        \item Интернет-браузер Chrome.
    \end{itemize}
    Каждый ноутбук имеет возможность выхода в сеть Интернет. Команды могут использовать
    собственные ноутбуки.

    \item Команды могут устанавливать дополнительное программное обеспечение, но после
    согласования с членами жюри.

    \item Участники во время командного этапа финального тура могут использовать интернет
    и заранее подготовленные библиотеки для решения задачи.

    \item Участники должны использовать язык программирования Python для написания
    программ, использующих командную строку. Для написания Ethereum контрактов
    участники могут использовать любой язык программирования.

    \item Для работы с сервисом \textit{Microsoft Face API} участникам предоставляется
    один ключ подписки на команду, а также базовый URL для доступа к REST API. При
    доступе к сервису с помощью данного ключа действует ограничение на 10 запросов в
    секунду. Участники одной команды должны сами заботиться о том, чтобы держать ключ
    в тайне от других команд. Ключ не должен передаваться третьим лицам. Если ресурс
    ключа подписки (примерно 60000 запросов) полностью используется командой, то
    организаторы вправе не предоставлять команде другой ключ.

    \item Участники не могут использовать помощь тренера, сопровождающего лица или
    привлекать третьих лиц для решения задачи.

    \item Финальная задача формулируется участникам в первый день финального тура,
    но участники выполняют решение  задачи поэтапно. Критерии прохождения каждого
    этапа формулируются для каждого дня финального тура. За подзадачи, решенные в
    конкретном этапе начисляются, баллы. Баллы за подзадачи можно получить только
    в день, закрепленный за конкретным этапом.

    \item В начале первого дня состязаний участники каждой команды получают доступ 
    к репозиторию на серверах GitLab.com. Каждая команда имеет свой собственный 
    репозиторий. Члены других команд не имеют доступ к чужим репозиториям.

    \item В течение дня не ведется учет количества изменений, которые команды
    регистрируют в Git-репозитории.

    \item В конце каждого дня финального этапа жюри проверяет решение участников
    на соответствие приемочным тестам для каждой подзадачи, входящей в набор
    для соответствующей итерации.

    \item Баллы за все подзадачи, для которых прошло приемочное тестирование,
    определяют баллы, набранные командой в данный день соревнований.

    \item Система автоматического тестирования имеет следующую конфигурацию:
    \begin{itemize}
        \item OS Linux
        \item Python \texttt{v3.6}
        \item Python модули (перечислены) и соответствующие зависимости
        (не перечислены): \texttt{web3} (\texttt{v5.0.0a3}), \texttt{opencv-python}, 
        \texttt{cognitive\_face}, \texttt{dlib},  \texttt{imutils}, 
        \texttt{ethereum}
        \item \texttt{/usr/local/bin/solc} (\texttt{v0.5.4})
        \item \texttt{/opt/shape\_predictor\_68\_face\_landmarks.dat}
    \end{itemize}

    \item После выставления баллов, командам предоставляется доступ к системе
    автоматического тестирования, ответственной за проведение приемочных тестов в
    конкретный день состязаний, так что члены команды могут ознакомиться с 
    логикой проверки и подать апелляцию, если не согласны с корректностью проведения
    тестов.

    \item После рассмотрения сути апелляции, жюри вправе провести тестирование
    вручную и назначить команде баллы за соответствующие подзадачи.

    \item В начале следующего дня состязаний жюри выдает всем командам свое решение
    подзадач предыдущего дня, которое команды могут использовать для того, чтобы
    решать следующий набор подзадач.

    \item Описанные выше условия могут быть изменены членами жюри. Все изменения в
    условиях объявляются участникам перед началом каждого дня состязаний.
\end{itemize}