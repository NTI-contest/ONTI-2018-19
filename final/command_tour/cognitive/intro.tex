\section{Описание задачи}

На заключительном этапе участник предлагается задача дистанционного («мысленного») управления объектом операторами по заданной траектории маршрута, имеющего несколько сегментов.

Задача предусматривает:
\begin{itemize}
    \item создание ИМК (интерфейс мозг-компьютер), включая его тестирование;
    \item использование ИМК для подготовки операторов (диагностика нужных качеств и обучение членов команды);
    \item отбор оператора с нужными качествами;
    \item управление объектом в «реальных» условиях.
\end{itemize}

Требуемые для решения задачи прохождения маршрута психофизиологические качества оцениваются при помощи психодиагностических методик, а также непосредственно в процессе обучения работе в ИМК. Для итогового задания (прохождения маршрута) следует выбрать оператора с наилучшей комбинацией качеств..
 
Для достижения результата участники решают следующие подзадачи:
\begin{enumerate}
    \item Сборка биоусилителя иустановка электродов для получения качественного сигнала ЭЭГ.
    \item Разработка программы для работы биоусилителя (регистрации электроэнцефалограммы).
    \item Тестирование различных алгоритмов машинного обучения для достижения наибольшей скорости и устойчивости работы ИМК (для управления объектом с помощью показателей электрической активности мозга).
    \item Проведение психодиагностической оценки каждого члена команды с целью определения участников с наиболее подходящими показателями для решения задачи.
    
    Участникам предлагается определить члена команды, обладающего наилучшей комбинацией показателей, используя психодиагностические процедуры и методы машинного обучения.

    Для решения поставленной подзадачи участникам предлагается модель маршрута, работая на которой участники смогут выбрать подходящего оператора (операторов) на основе  его когнитивных показателей, определенных психометрическими тестами, и результатов работы методов  машинного обучения, которые будут задействованы в дальнейшей работе ИМК.
    \item На последнем этапе происходит соревнование выбранных участников команд по прохождению неизвестного им ранее маршрута путем передвижения объекта при помощи нейроуправления. Побеждает команда, прошедшая маршрут с наибольшей скоростью и наименьшим количеством ошибок, что означает, что они смогли настроить алгоритм машинного обучения, корректно применить методы психодиагностики и в результате этого делегировать участника, оптимально подходящего для выполнения этой задачи.
\end{enumerate}

\subsubsection*{Сценарий работы над задачей}

На первом этапе осуществляется сборка усилителя биопентациалов, подключение к нему модулей активных электродов, расположение электродов на голове. Результат данного этапа – качественная регистрация ЭЭГ, что включает в себя стабильную работу усилителя, низкий уровень помех в ЭЭГ.

На втором этапе осуществляется  настройка визуальной части ИМК и методов машинного обучения, которые будут задействованы в ИМК. На данном этапе необходимо выбрать настройки визуальной части, способствующие наибольшей эффективности оператора, и настройки машинного обучения, обеспечивающие наиболее надёжную работу ИМК (определение элемента на экране, на котором было сфокусировано внимание оператора).

На третьем этапе проводятся психодиагностические тесты, направленные на оценку качеств, необходимых для эффективной работы в ИМК. Кроме этого потенциальные операторы проходят тестовую работу в ИМК для оценки их эффективности непосредственно в ИМК. В итоге выбирается оператор с наилучшей комбинацией качеств.

На четвертом этапе команда выставляет своего оператора, работающего с настроенным этой командой ИМК для прохождения финального задания.