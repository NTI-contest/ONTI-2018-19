\section{Решение задач первого дня}

\subsubsection*{Получение сигнала ЭЭГ с одноплатного компьютера}

Для получения сигнала ЭЭГ с одноплатного компьютера необходимо подключить модули аналогового усиления (ЭЭГ-модули от BiTronics Lab) к плате Arduino Due. После этого подключить ЭЭГ-электроды к усилительным модулям, а сами электроды установить в шлем для регистрации ЭЭГ, который надет на одного из членов команды. Для минимизации помех ЭЭГ-электроды должны быть свиты. После установки электродов подобрать такой уровень усиление на ЭЭГ-модулях, чтобы на экране была хорошо видна сама ЭЭГ, а также типичные артефакты в ЭЭГ – всплески при морганиях и напряжении жевательных мышц.

Команда, получившая сигнал необходимого качества, первой получает 10 баллов. Остальные – по убывающей в соответствии с временем выполнения.

\subsubsection*{Психодиагностика и выявление наилучшего оператора}

Проводятся психодиагностические тесты: опросник Русалова для определения показателей темперамента и корректурная проба Бурдона для определения показателей внимания (переключаемость и концентрация). Наилучшим оператором для финальной задачи будет тот участник команды, у которого будут самые высокие показатели внимания, а также выраженная эмоциональность.

Критерий оценки – корректная обработка результатов тестирование и выбор соответствующего оператора, 10 баллов.