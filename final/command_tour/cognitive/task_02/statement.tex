\section{Решение задач второго дня}

\subsubsection*{Настройка параметров стимуляции на экране и параметров записи ЭЭГ}

Подбираются такие параметры стимуляции (длительность миганий и пауз между ними), чтобы обеспечить максимальный комфорт и стабильность внимания оператора. Настройка осуществляется в соответствии с индивидуально-психологическими особенностями оператора. Показатель качества настройки данных параметров – построение эффективного классификатора, способного по характеристикам ЭЭГ, определить, на каком экранном элементе было сфокусировано внимание оператора. Таким образом выполняется тестирование нескольких вариантов стимульных характеристик, чтобы определить наиболее подходящий вариант для данного оператора.

Критерий оценки – выбор оптимального режима для оператора команды, 10 баллов.

\subsubsection*{Прохождение финального маршрута}

После тестирование различных стимульных сред команда переходит к финальному заданию: прохождение маршрута. Предварительно обученный оператор выполняет 10 ходов экранным объектом, выбирая направление движения при помощи фокусировки внимания на подсветках соответствующего экранного элемент (стрелка). Успешная попытка – перемещение экранного объекта в направлении, соответствующем стрелке, на которой был сфокусирован оператор. Побеждает команда с наибольшим количеством успешных ходов – 20 баллов. Остальные команды – по убывающей в соответствии с количеством успешных ходов.

\solutionSection

\subsubsection*{Настройка параметров стимуляции на экране и параметров записи ЭЭГ}

Подбираются такие параметры стимуляции (длительность миганий и пауз между ними), чтобы обеспечить максимальный комфорт и стабильность внимания оператора. Настройка осуществляется в соответствии с индивидуально-психологическими особенностями оператора. Показатель качества настройки данных параметров – построение эффективного классификатора, способного по характеристикам ЭЭГ, определить, на каком экранном элементе было сфокусировано внимание оператора. Таким образом выполняется тестирование нескольких вариантов стимульных характеристик, чтобы определить наиболее подходящий вариант для данного оператора.

Критерий оценки – выбор оптимального режима для оператора команды, 10 баллов.

\subsubsection*{Прохождение финального маршрута}

После тестирование различных стимульных сред команда переходит к финальному заданию: прохождение маршрута. Предварительно обученный оператор выполняет 10 ходов экранным объектом, выбирая направление движения при помощи фокусировки внимания на подсветках соответствующего экранного элемент (стрелка). Успешная попытка – перемещение экранного объекта в направлении, соответствующем стрелке, на которой был сфокусирован оператор. Побеждает команда с наибольшим количеством успешных ходов – 20 баллов. Остальные команды – по убывающей в соответствии с количеством успешных ходов.