\section*{Приложение А  - Описание стенда трека}

Стенд трека состоит из двух основных частей:

\begin{itemize}
    \item имитатор некоторого механизма (макет с шестеренками), расположенного на удаленном объекте,
    \item имитатор канала связи с удаленным объектом (макет системы «спутник – радар»).
\end{itemize}

В качестве механизма, расположенного на некотором удаленном объекте, был создан макет с системой шестеренок, приводимых в движение двигателем и оборудованный тремя оптопарами. Этот макет позволяет, при включении, регистрировать оптический сигнал, проходящий через прорези в шестеренках и записывать его в файл. Оптический сигнал считывается системой регистрации макета (на базе RaspberryPi и STM32L432) через равные промежутки времени (1 мс). Сформированный таким образом файл представляет собой последовательность из ‘0’ и ‘1’ (двоичный код) с каждой шестерни макета и в дальнейшем передается на компьютер управления, для последующего анализа участниками трека.

Имитатор канала связи по легенде трека назван системой «радар-спутник» и состоит из подвижной каретки «спутник» и вращающейся вокруг вертикальной оси каретки «радар». Каретка «спутник» может передвигаться вдоль рельсы (ось X) и имеет инфракрасный (ИК) передатчик. Управление осуществляется с помощью компьютера RaspberryPi, который получает необходимые для передачи данные от компьютера управления и передает их во время движения. Блок «Радар» оборудован компьютером RaspberryPi и цифровой видеокамерой, с помощью библиотеки OpenCV он распознает положение графического маркера «спутника» в поле зрения камеры и передает положение маркера на управляющий компьютер. Одновременно с этим ИК-приемник «радара» принимает поток данных, передаваемый «спутником». Скорость передачи данных ограничена на аппаратном уровне и составляет 115.2 кбит/с.

\putImgWOCaption{13cm}{1}

\begin{center}
    Рисунок 1. Общий вид стенда:

    1 – подвижная передающая каретка «спутник» с графическим маркером, 2 – блок управления, 3 – подвижная приемная каретка «радар» с видеокамерой, 4 – макет с шестеренками, 5 – роутер, 6 – компьютер управления стендом.
\end{center}

На рисунке 1 представлены основные части стенда:

1 – подвижная передающая каретка («спутник»), с помощью двигателя и приводного ремня может перемещаться вдоль оси X влево-вправо по программно-задаваемому закону движения. Имеет на борту ИК-передатчик, передающий, во время движения, данные телеметрии бортового механизма.

2 – блок управления, содержит управляющий компьютер «спутника» на базе RaspberryPi, блоки питания и логику управления моторами.

3 – подвижная приемная каретка («радар») имеет ИК-приемник, для приема данных со «спутника», видеокамеру для отслеживания «спутника» и двигатель для вращения каретки вокруг вертикальной оси.

4 – макет с шестеренками, позволяет регистрировать в цифровом виде сигналы с оптических пар, установленных на каждой шестеренке и записывать их в файлы, для управления данной системой используется компьютер RaspberryPi и вспомогательный микропроцессор STM32L432.

5 – роутер, объединяет все компьютеры стенда (3 штуки RaspberryPi и управляющий ноутбук) в локальную сеть, а также обеспечивает дальнейшее соединение с локальной сетью организаторов для удаленного управления и настройки стенда.

6 – компьютер управления стендом (ноутбук), предоставляет участникам трека web-интерфейс к задачам, управляет всеми основными блоками стенда, а также позволяет создавать и загружать программное обеспечение для решения задач на языках C, C++, Python, Java.

На рисунке 2 приведена схема взаимодействия всех блоков стенда между собой, а также адреса в локальной сети. Каждый компьютер стенда имеет фиксированный IP-адрес TCP/IP протокола, через который происходит управление, прием и передача данных для выполнения задач.