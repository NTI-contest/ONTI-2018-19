\solutionSection

\textit{Код решения задачи (блочное кодирование):}

Программа осуществляетблочное кодирование - входной файл разбивается на блоки с четко заданной структурой. В принятом файле блоки оказываются перемешанными из-за периодического попадания спутника в «слепые» зоны радара. Программа декодер – находит целые блоки и далее выстраивает из них исходное сообщение, выставляя блоки согласно их номерам заданном на этапе кодирования.

Программа блочного кодерования:

\inputminted[fontsize=\footnotesize, linenos]{cpp}{final/command_tour/telecom/task_02/source_1.cpp}

Программа декодирования – поиск блоков и сборка из них исходного сообщения

\inputminted[fontsize=\footnotesize, linenos]{cpp}{final/command_tour/telecom/task_02/source_2.cpp}

Программа компилируется с -lm option

Вариант программы кодирования сигнала помехоустойчивым кодом Хемминга (8,4).

Ипользуется (8,4) так как плотность помех изменяют не большего одного бит в байте

\inputminted[fontsize=\footnotesize, linenos]{cpp}{final/command_tour/telecom/task_02/source_3.cpp}