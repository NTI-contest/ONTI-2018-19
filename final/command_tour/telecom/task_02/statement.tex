\assignementTitle{}{27}{}

\textit{Цель:}

\begin{enumerate}
    \item Получить навыки работы с помехоустойчивым кодированием, научиться использовать алгоритмы кодирования и декодирования сообщения, работа с байтами и битами (работа с бинарными файлами), освоить методику пакетной передачи данных.
    \item Получить навыки работы со статистическим анализом данных.
\end{enumerate}

Результаты решения задачи могут быть использованы при решении задачи 5

\textit{Постановка задачи. Кодирование/Декодирование}

По ИК - каналу (со спутника на радар) передается телеметрическая информация и данные. В канале телеметрии передается координатная информация: время в секундах с момента начала движения и (X, Y) координаты спутника в системе координат стенда. Канал телеметрии очень надежный в нем отсутствуют шумы и помехи, информация передается без искажений. По каналу данных передается фотографии спутника большой планеты в высоком разрешении. В канале данных присутствует шум и помехи, требующие организации помехозащитного кодирования передаваемых данных. Шум сильнее если радар не сопровождает спутник. 

Задачи команд:

Исследовать характер шума в тестовом канале и передать информацию без потерь.
Написать программы: кодер загружается на передатчик (спутник) и декодер загружается на приемник (радар), позволяющее восстановить передаваемое сообщение и исправить возможные ошибки.

Краткое описание подзадач:

\begin{enumerate}
    \item Научиться работать со стендом спутник – радар. Запустить передачу данных со спутника посредством ИК канала и начать приём данных на «радаре».
    \item Исследовать свойства шума, на основе данного исследования выбрать метод помехоустойчивого кодирования.
    \item Разработать программу кодер (действует на стороне «спутника»).
    \item Разработать программу декодер (действует на стороне «радара»).
    \item Провести тестирование программ кодера и декодера в модельных каналах с различным уровнем сложности распределения ошибок. (бонусные очки, максимум 10)
    \item  Используя разработанные и протестированные программы кодера/декодера передать без ошибок тестовое сообщение со спутника (основные очки, максимум 17).
\end{enumerate}

\markSection

Расчет очков (17+10)

\textit{Расчет очков задачи 2а (оценка эффективности кодера и декодера на этапе тестирования)}

Для подсчета верно переданных данных была написана программа, сравнивающая файлы и считающая количествво неисправленных ошибок в переданном файле. Таким образом, параметром $V$, оценивающим точность передачи данных являлось отношение количества байт ($n$), в которых была обнаружена не исправленная ошибка к общему количеству ошибок ($N$) в принятом файле прошедшему модельный канал связи:
$$V =\frac{n}{N}$$
Формула, по которой рассчитывается количество очков ($I$) по точности передачи $V$, имеет вид:
$$I = (1-V) \cdot М \cdot 0.37,$$
где $М$ — максимальное число основных очков за задачу 2а. Коэффициент 0.37 связан с долей основных очков, отведенных на решение задачи 2а.

\textit{Расчет очков задачи 2б (оценка эффективности кодера и декодера)}

Для подсчета верно переданных данных была написана программа, сравнивающая файлы и считающая количество неисправленных ошибок в переданном файле. Таким образом, параметром $V$, оценивающим точность передачи данных являлось отношение количества байт ($n$), в которых была обнаружена не исправленная ошибка к общему количеству ошибок ($N$) в принятом файле прошедшему реальный канал связи:
$$V = \frac{n}{N}$$
Формула, по которой рассчитывается количество очков ($I$) по точности передачи $V$, имеет вид:
$$I = (1-V) \cdot M \cdot 0.63,$$
где $M$ — максимальное число основных очков за задачу 2б. Коэффициент 0.63 связан с долей основных очков, отведенных на решение задачи 2б.