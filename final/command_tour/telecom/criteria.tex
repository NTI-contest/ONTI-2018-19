\markSection

В таблице 1 приведена таблица очков по задачам командного тура, в дополнение к основным очкам за полноценное решение задачи предполагаются бонусные очки, за скорость и точность решения. Бонусные очки введены, для определения победителя при прочих равных условиях, а также, чтобы поощрить командную работу. В правой колонке таблицы 1 приведена формула расчета максимального числа очков за задачу $M+N$, где $M$ – основные очки, $N$ – бонусные очки. Бонусные очки даются за скорость и точность при решении задачи. Количество неудачных попыток сдать задачу снижает количество бонусных очков. Бонусные очки даются за оригинальную идею при решении задачи, более устойчивый алгоритм и т.д.

\begin{table}[H]
    \center
    \caption{Таблица 1. Разбалловка задач по профилю «Технологии беспроводной связи»}
    \begin{tabular}{|p{10cm}|c|}
        \hline
        Задачи командного тура & Очки (внутренняя шкала) \\
        \hline
        1. Работа с узором шестеренки - определение кода Хемминга и декодирование. Определение среднего периода и среднеквадратичного отклонения от среднего – результаты вносятся в диагностическую карту.	& 6+4 \\
        \hline
        2. Передача тестовой картинки со спутника (кодирование/декодирование, сжатие, разбиение на блоки). & 17+10 \\
        Бонус. Исследовать характер шума в тестовом канале и передать информацию без потерь. & \\
        \hline
        3. Разработка алгоритма слежения за спутником.	&10+5 \\
        \hline
        4. Определение параметров траектории&14+4\\
        \hline
        5. Восстановление новой карты. Передача наибольшего количества кратеров.	&25+5\\
        \hline
    \end{tabular}
\end{table}

В таблице 2 представлен вид диагностической карты, карта заполняется при решении первой и четвертой задач.

\begin{table}[H]
    \center
    \caption{Таблица 2. Формат диагностической карты}
    Диагностическая карта

    Название команды: \underline{\hspace{3cm}}
    \begin{tabular}{|p{4cm}|l|l|l|l|}
        \hline
        Стенд с шестернями & Шестеренка 1	&Шестеренка 2&	Шестеренка 3&	Время \\
        \hline
        Код Хэмминга (n,k)	& 1. & 1. & 1. & 1. \\
        & 2. & 2. & 2. & 2. \\
        & 3. & 3. & 3. & 3. \\
        \hline
        Закодированное двоичное слово & 1. & 1. & 1. & 1. \\
        & 2. & 2. & 2. & 2. \\
        & 3. & 3. & 3. & 3. \\
        \hline
        Декодированное & 1. & 1. & 1. & 1. \\
        Двоичное слово & 2. & 2. & 2. & 2. \\
        & 3. & 3. & 3. & 3. \\
        \hline       
        Декодированное слово в десятичной системе 	& 1. & 1. & 1. & 1. \\
        & 2. & 2. & 2. & 2. \\
        & 3. & 3. & 3. & 3. \\
        \hline
        Средний период вращения (мс) & 1. & 1. & 1. & 1. \\
        & 2. & 2. & 2. & 2. \\
        & 3. & 3. & 3. & 3. \\
        \hline 
        Среднеквадратичное отклонение (мс) & 1. & 1. & 1. & 1. \\
        & 2. & 2. & 2. & 2. \\
        & 3. & 3. & 3. & 3. \\
        \hline
    \end{tabular}
\end{table}

\textit{Оценка параметров траектории}

\begin{enumerate}
    \item[1.] Период обращения естественного спутника вокруг планеты (секунды): 
    \item[1.1.]	\underline{\hspace{5cm}},
    \item[1.2.]	\underline{\hspace{5cm}},
    \item[1.3.]	\underline{\hspace{5cm}}.
    \item[2.] Размер большой полуоси (см):
    \item[2.1.]	\underline{\hspace{5cm}},
    \item[2.2.]	\underline{\hspace{5cm}},
    \item[2.3.]	\underline{\hspace{5cm}}.
    \item[3.] Размер малой полуоси (см):
    \item[3.1.]	\underline{\hspace{5cm}},
    \item[3.2.]	\underline{\hspace{5cm}},
    \item[3.3.]	\underline{\hspace{5cm}}.
    \item[4.] Период обращения искусственного спутника вокруг естественно спутника (секунды): 
    \item[4.1.]	\underline{\hspace{5cm}},
    \item[4.2.]	\underline{\hspace{5cm}},
    \item[4.3.]	\underline{\hspace{5cm}}.
    \item[5.] Радиус круговой орбиты искусственного спутника вокруг естественно спутника (см):
    \item[5.1.]	\underline{\hspace{5cm}},
    \item[5.2.]	\underline{\hspace{5cm}},
    \item[5.3.]	\underline{\hspace{5cm}}.
\end{enumerate} 
