\assignementTitle{}{30}{}

\textit{Цель:}
\begin{enumerate}
    \item Получить навыки работы с беспотерьной передачей данных в условиях канала с потерями.
    \item Получить навыки работы с компрессией данных.
\end{enumerate}

Задача является финальным испытанием, для успешного решения необходимо решить четыре предыдущих задачи.

\textit{Постановка задачи. Поиск кратеров и определение их размеров и координат}

Наша задача состоит в том, чтобы отследить появление новых кратеров на естественном спутнике некоторой удаленной планеты. Область, в которой появление новых кратеров минимально, считается относительно безопасной для оборудования исследовательской базы. Для решения задачи на круговую орбиту вокруг естественного спутника планеты выведен исследовательский спутник с хорошей оптической системой на борту. Исследовательский спутник фотографирует поверхность планеты с очень высоким разрешением, однако, так как пропускная способность канала передачи данных низкая, то передать весь объем данных невозможно. На борту предусмотрена возможность запуска программы, преобразующая снимки, полученные в высоком разрешения, в черно-белый формат (‘0’ и ‘1’) для оперативного анализа. Таким образом, на борту спутника необходимо: 
\begin{enumerate}
    \item[a)] проанализировать новый снимок, 
    \item[b)] найти отличия, 
    \item[c)] сформировать список, содержащий: центры найденных областей и их размер (центр прямоугольника и длины его сторон). 
\end{enumerate}
По данному списку штатное бортовое ПО из большой карты вырезает выбранные фрагменты и передает их модулю, обеспечивающему помехоустойчивое кодирование передаваемых данных.

\textit{Краткое описание подзадач:}
\begin{enumerate}
    \item Получить старую карту в черно-белом формате (‘0’ и ‘1’). 
    \item Написать и протестировать программу поиска кратеров на основе данных старой карты.
    \item Загрузить программу на борт исследовательского спутника.
    \item Запустить программу поиска кратеров на борту спутника. Программа должна определить центры и радиусы кратеров на новой карте и данные результаты записать в отдельный файл craters.dat, в отдельных строках которого содержаться координаты центра кратера и его радиус в формате (XYR).
    \item Получить через интерфейс общения со спутником список кратеров, выбрать нужный, что дает команду бортовому ПО – выделить с карты в высоком разрешении нужную область и запустить процесс передачи изображения кратера в высоком разрешении. При этом используется программа, обеспечивающая помехоустойчивое кодирование, разработанная на этапе решения 2-й задачи и программа, обеспечивающая эффективное сопровождение спутника радаром для минимизации шумов, разработанная на этапе решения 3-й задачи.
\end{enumerate}

\markSection

Расчет очков (25 + 5)

При расчете очков за задачу 5 оценивается общая площадь верно определенных и переданных изображений кратеров.

Команда, передавшая кратеры на наибольшую общую площадь, получает 25 очков, очки остальных команд находятся по пропорции относительно максимальных 25 очков. Бонусные 5 очков даются за качество работы программы поиска кратеров и определения их размеров. При этом сравниваются список параметров кратеров, полученный командой и список с заданным расположением кратеров. Параметром $V$, оценивающим точность определения кратеров является количество совпадений ($n$) в сравниваемых списках:
$$V = \frac{n}{N},$$
где $N$ – кол-во кратеров в заданном списке.