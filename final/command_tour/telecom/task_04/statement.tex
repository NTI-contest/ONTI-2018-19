\assignementTitleT{18 очков}

\textit{Цель:}

\begin{enumerate}
    \item Получить навыки работы с функциями, заданными дискретным набором точек.
    \item Получить навыки в анализе параметрических функций.
    \item Получить модель движения спутника и использовать эти знания в задаче 3 – сопровождение спутника
\end{enumerate}

Результаты решения задачи используются при решении задачи 2 и 3.

\textit{Постановка задачи. Вычисление параметров траектории}

Проанализируйте телеметирическую информацию, полученную по техническому каналу. Определите параметры движения: период вращения искусственного спутника вокруг естественного спутника большой планеты и период обращения естественного спутника вокруг большой планеты. Известно, что искусственный спутник вращается вокруг естественного спутника по круговой орбите, определите радиус орбиты. Естественный спутник вокруг планеты вращается по эллиптической орбите, определите большую и малую полуось эллипса. Результаты анализа занесите в диагностическую карту.

\textit{Краткое описание подзадач:}

\begin{enumerate}
    \item Получите как можно больший объем телеметрической информацию по техническому каналу связи.
    \item Постройте и проанализируйте зависимости: $x = x(t), y = y(t), y = y(x)$.
    \item Определите параметры движения спутника: периоды обращения и параметры орбит.
    \item Подставляя найденные параметры в модель движения и сравнивая результаты моделирования с полученными по техническому каналу данными убедитесь, что полученные зависимости соответствуют друг другу и значит параметры движения определены верно.
\end{enumerate}

\subsubsection*{Расчет очков (14 + 4)}

Команды на проверку сдают диагностические карты. Если команда приносит правильный ответ с первой попытки, то получает 100\% (14 очков), если при этом команда решает задачу первой, то добавляются бонусные 4 очка, если команда решает второй, то добавляются бонусные 3 очка и т.д. Если с первой попытки команде не удалось правильно определить все параметры траектории, то бонусные очки не добавляются. Очки рассчитываются исходя из количества правильно определенных параметров, каждый параметр оценивается в 2.8 очка. 

Основные очки рассчитываются следующим образом:
$$\text{Очки} = n \cdot 2.8,$$
где $n$ – количество правильно определенных параметров. Периоды обращения должны быть оценены с точностью 0.1 секунда, параметры орбит с точность 0.5 см.

