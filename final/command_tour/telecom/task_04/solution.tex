\solutionSection

Для траектории построим на графике входные данные.

\putImgWOCaption{15cm}{1}
\putImgWOCaption{15cm}{2}

Из рисунка видно, что обе орбиты близки к круговым, тогда проекции модели движения на $x$ и $y$ можно записать:

$$ \left\{
\begin{aligned}
x = a_1 \cdot \cos(\omega_1\cdot t + \varphi_1)  + a_2 \cdot \cos(\omega_2\cdot t + \varphi_2), \\
y = a_1 \cdot \sin(\omega_1\cdot t + \varphi_1)  + a_2 \cdot \sin(\omega_2\cdot t + \varphi_2). \\
\end{aligned}
\right. $$

где $a_1$ — радиус орбиты естественного спутника вокруг планеты, $a_2$ — радиус орбиты искусственного спутника вокруг естественного, $\omega_1$ и $\omega_2$ — соответствующие им круговые частоты. Если перейти в полярную СК и рассмотреть динамику изменения длины радиус-вектора, тогда получим:

$$r^2 = a_1^2+a_2^2 + 2a_1a_2\cos(t\cdot (\omega_1 - \omega_2) + \varphi_1 + \varphi_2).$$
 
Тогда минимальное и максимальное расстояние:

$$ \left\{
\begin{aligned}
r_{min} = a_1 - a_2, \\
r_{max} = a_1 + a_2. \\
\end{aligned}
\right. $$

Откуда легко найти $a_1$ и $a_2$.
Рассмотрим (на верхний график рисунка) динамику изменения радиус-вектора. В динамике амплитуды видны четкие биения, соответствующие разностной частоте. Период этих биений можно определить, анализируя интервал времени между соседними максимумами.

\putImgWOCaption{15cm}{3}

Найденный период связан с искомыми периодами следующим образом:

$$T_\text{Биений} = \frac{T_1T_2}{T_1 - T_2}.$$

Анализируя таким же образом функцию 

$$\sin\left(\text{atan}\left(\frac{x}{y}\right)\right)$$ 

(нижний график на рисунке), можно определить период «несущей» частоты:

$$\frac{T_\text{Несущей}}{2} = \frac{T_1T_2}{T_1 + T_2}.$$

Решая совместно систему уравнения для $T_\text{Биений}$ и $T_\text{Несущей}$ определяем соответствующие периоды обращения.



