\solutionSection

Вариант программы декодирования кода Хемминга. На вход программы поступает произвольная последовательность из ‘0’ и ‘1’, на выходе программа возвращает декодированную последовательность из ‘0’ и ‘1’, причем если в исходной последовательности была допущена однократная ошибка, прогрмма ее исправляет и сообщает номер битого символа во входной последовательности.

\inputminted[fontsize=\footnotesize, linenos]{cpp}{final/command_tour/telecom/task_01/source_1.cpp}

Пример программы, определяющий период во входной последовательности и среднеквадратичное отклонение периода от среднего, дополнительно решаются следующие задачи: определяется корреляционная функция стабильного сигнала (с постоянным периодом) и измеренным на макете, определяется отношение квадрата невязки между модельным и измеренным сигналом к энергии модельного сигнала (т.е. определяется относительная энергия невязки между сигналами). Входные данные для программы содержатся в конфигурационном файле, имеющего следующий формат: кол-во последовательностей сигналов, длина кодовой последовательности (известна из предыдущей задачи), форма кодовой последовательности (последовательность из ‘0’ и ’1’), длительности переходных процессов (длительность переднего/заднего фронта сигнала, длительность смены фазы – переход от ‘0’ к ‘1’ и наоборот) – вводится для решения задачи в более общем случае, когда время переключения фазы не бесконечно быстрое. На выходе программа возвращает для каждого канала средний период и СКО отклонения от среднего периода.

\inputminted[fontsize=\footnotesize, linenos]{cpp}{final/command_tour/telecom/task_01/source_2.cpp}

Класс, в котором объеденены методы анализа анализа измеренного сигнала(получен на макете спутника)

\inputminted[fontsize=\footnotesize, linenos]{cpp}{final/command_tour/telecom/task_01/source_3.cpp}

Главная программа, из которой вызывается метод для анализа периодичности измеренного сигнала

\inputminted[fontsize=\footnotesize, linenos]{cpp}{final/command_tour/telecom/task_01/source_4.cpp}