\assignementTitleT{10 очков}

\textit{Цель:}

\begin{enumerate}
    \item Получить навыки работы с помехоустойчивым кодом Хемминга, исправляющим однократную ошибку, освоить алгоритмы кодирования и декодирования сообщения, работа с матрицей преобразования, определить синдром, освоить методику исправления однократной ошибки.
    \item Получить навыки работы со статистическим анализом данных: определение среднего и среднеквадратичного отклонения, обработка длинных рядов данных.
\end{enumerate}

Результаты решения задачи могут быть использованы при решении задач 2 и 5.

\textit{Постановка задачи}

Определение кода Хэмминга. Прорези и стенки на шестернях представляют код Хемминга различной длины. По рисунку прорезей на каждой шестеренке необходимо определить код Хемминга: элемент с наименьшим угловым размером – прорезь или стенка имеют смысл ‘0’ или ‘1’. Последовательность прорезей может читаться как по часовой, так и против часовой, признаком того что кодовое слово определено верно является отсутствие ошибок в закодированном слове. Закодированные и декодированные последовательности заносятся в диагностическую карту в двоичном и десятичном представлении.

Определение периода вращения. Получить последовательность сигналов с трех каналов макета (R - красный, G - зеленый, B - синий, далее RGB последовательность, по легенде - телеметрия), динамика сигнала в каждом канале модулируется своей вращающейся шестеренкой. По полученной последовательности сигналов на макете спутника определить средний период вращения каждой шестеренки, оценить стабильность вращения (среднеквадратичное отклонение от среднего периода). Результаты исследования занести в диагностическую карту.

Краткое описание подзадач:

\begin{enumerate}
    \item Научиться работать с макетом спутника: включение, снятие данных.
    \item Получить последовательность сигналов с трех каналов макета (R - красный, G - зеленый, B -синий, далее RGB-последовательность, по легенде - телеметрия), динамика сигнала в каждом канале модулируется своей вращающейся шестеренкой.
    \item По полученной последовательности сигналов на макете спутника команды определяют период вращения каждой шестеренки, оценивают стабильность вращения (среднеквадратичное отклонение от среднего периода). Результаты исследования заносятся в диагностическую карту.
    \item Прорези и стенки на шестернях представляют код Хемминга различной длины (на втором отборочном этапе командам предоставлялись ссылки на методические материалы по работе с кодом Хемминга). Команды по рисунку прорезей на каждой шестеренке определяют код Хемминга: элемент с наименьшим угловым размером – прорезь или стенка имеют смысл ‘0’ или ‘1’. Последовательность может читаться как по часовой, так и против часовой, однако признаком того что кодовое слово определено верно является отсутствие ошибок в закодированном слове (командам об этом сообщается). Закодированные (последовательность прорезей на шестерне) и декодированные последовательности заносятся в диагностическую карту.
\end{enumerate}

Макет спутника, возвращающий эталонную RGB последовательность, представлен на рисунке 2. На шестернях начало кодовой последовательности отмечено радиальной линией. ИК – передатчик излучает сигнал постоянной амплитуды, на пути приемника находятся вращающиеся шестерни, модулирующие ИК сигнал по уровням: ‘0’, ‘1’. Для наглядности каждая шестерня подсвечена своим светодиодом (малая – красным, средняя – зеленым, большая – синим).

\putImgWOCaption{13cm}{1}

\begin{center}
    Рисунок 2. Макет спутника.
\end{center}

На рисунке 3 показана динамика амплитуды ИК – сигнала, прошедшего через вращающиеся шестерни. По полученной последовательности сигналов на макете спутника команды определяют период вращения каждой шестеренки, оценивают стабильность вращения (находят среднеквадратичное отклонение от среднего периода).

\putImgWOCaption{13cm}{2}

\begin{center}
    Рисунок 3. Динамика амплитуды ИК-сигнала, модулированного вращающимися шестернями.
\end{center}

\subsubsection*{Расчет очков (6 + 4)}

Очки за шестерню:

Правильно определен период и СКО – 1 очко;

Последовательность правильно декодирована – 1 очко.

За каждую неудачную попытку снимается 5\% от максимальных 6 очков, т.е. если команда приносит правильный ответ с первой попытки, то получает 100\% (6 очков), если первая попытка неудачная, но ответ верно найден со второй попытки команда получает 95\% от 6 очков, т.е. 5,7, и так далее.