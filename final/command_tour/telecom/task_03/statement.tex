\assignementTitleT{15 очков}

\textit{Цель:}

\begin{enumerate}
    \item Научиться работать с программным интерфейсом управления «радаром» (API).
    \item Получить навыки работы по проектированию предсказательных алгоритмов, в частности алгоритмов сопровождения движущихся объектов.
\end{enumerate}

Результаты решения задачи могут быть использованы при решении задач 2 и 5.

\textit{Постановка задачи}

Необходимо разработать алгоритм сопровождения спутника радаром. Имитатор канала связи по легенде трека назван системой «радар-спутник» и состоит из подвижной каретки «спутник» и вращающейся вокруг вертикальной оси каретки «радар», рисунок 4. Каретка «спутник» может передвигаться вдоль рельсы (ось $X$) и имеет инфракрасный (ИК) передатчик. В зависимости от относительного отклонения спутника от оси наблюдения радара уровень шума в принимаемом канале меняется. Таким образом, моделируется отношение сигнал/шум и диаграмма направленности радара, плотность битового шума зависит от величины углового смещения $\Delta X$, но нигде не превышает одного бита на байт. Для обеспечения наименьших уровней помех в принимаемом сигнале необходимо разработать алгоритм сопровождения спутника радаром.

Управление спутником осуществляется с помощью компьютера RaspberryPi, который получает необходимые для передачи данные от компьютера управления и передает их во время движения. Блок «Радар» оборудован компьютером RaspberryPi и цифровой видеокамерой, с помощью библиотеки OpenCV он распознает положение графического маркера «спутника» в поле зрения камеры и передает угловое положение маркера $\Delta X$ относительно центра камеры вдоль оси $X$ на управляющий компьютер.

Одновременно с этим, ИК-приемник «радара» принимает поток данных, передаваемый «спутником».

\textit{Краткое описание подзадач:}

\begin{enumerate}
    \item Необходимо разработать алгоритм сопровождения спутника радаром. Входными данными для программы является горизонтальное смещение спутника относительно оси радара $\Delta X$. Для управления радаром используются команды, позволяющие регулировать скорость и направление вращения камеры.
    \item Провести тестирование разработанной программы.
    \item Использовать разработанную программу сопровождения спутника во время приема данных при решении задач 2 и 5, так как это позволяет значительно снизить уровень помех.
\end{enumerate}

\putImgWOCaption{9cm}{1}

\begin{center}
    Рисунок 4. Общий вид стенда: 1 – подвижная передающая каретка «спутник» с графическим маркером, 2 – блок управления, 3 – подвижная приемная каретка «радар» с видеокамерой, 4 – макет с шестеренками, 5 – роутер, 6 – компьютер управления стендом.
\end{center}

\subsubsection*{Расчет очков (10 + 5)}

Поскольку для выполнения работы наиболее важным является уровень зашумленности данных, для оценки точности сопровождения используется метод оценки среднего уровня шума.

Во время приема данных по инфракрасному каналу определяется положение спутника ($dx$) в секторе обзора радара. В зависимости от этого положения изменяется степень принимаемых зашумленности данных - чем дальше от центра - тем более зашумлены данные. Уровень зашумленности (N) представляет собой величину в пределах от 0 (отсутствие шума) до 1 (самый сильный шум).

В файле сопровождения сохраняется уровень шума на каждые 32 байта принятых данных.

По каждому блоку данных (условное обозначение - точка) рассчитывается уровень невязки сопровождения:

	если $|dx|< 100$, то уровень шума $N=|dx|/100$

    если $|dx| > 100$, то уровень шума $N=1$.

Таким образом, при невязке $dx$ больше 100 единиц (пикселов видеокамеры) этот уровень равен единице, при невязке $dx$ меньше 100 единиц этот уровень уменьшается до нуля пропорционально невязке.

При оценке точности сопровождения рассчитывается среднее значение и дисперсия шума.  Для оценки используются только случаи, когда было принято достаточно много данных (более 2000 точек или более 64 кБайт данных), это обеспечивает точность оценки порядка 2\%.

Нами рассматривались только результаты участников, в которых количество принятых точек составляло порядка 2000 и более, и рассматривалась средняя невязка сопровождения $V$, как критерий точности сопровождения.

Полученное среднее значение шума $V$ является мерой точности сопровождения, чем эта величина меньше, тем сопровождение точнее.

Расчет основных очков ($I$) по средней невязке ($V$): 
$$I = M \cdot \left(\frac{V_{min}}{V}\right)/100,$$ 
где $M$ — максимальное число основных очков за задачу (10). $V_{min}$ - наилучшая точность, достигнутая всеми командами. Бонусные очки добавлялись за более глубокую адаптацию стандартного PID –алгоритма под решаемую задачу.
