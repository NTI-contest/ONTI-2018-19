\newpage

\markSection

\begin{tabular}{|p{4cm}|p{9cm}|p{1cm}|} 
    \hline
    Функциональные требования & Описание & Макс.
    
    балл \\	
    \hline

    \hline

    Загружаются из файла и отображаются: источники, ресурсы, конвейеры, преобразователи	& 4 балла: 
+1 балл за категорию & 4 \\
    \hline

    Переключение файлов из интерфейса & 2 балла:

    +1 балл - реализовано переключение файлов

    +1 балл - объекты корректно отображаются после переключения на другой файл & 2 \\
    \hline

    Количество видов ресурсов & 2 балла - требуемое количество

    1 балл - больше 1го, но меньше требуемого & 2 \\
    \hline

    
    Количество видов преобразователей & 2 балла - требуемое количество
    
    1 балл - больше 1го, но меньше требуемого & 2 \\
    \hline


    Объекты создаются из интерфейса	& 2 балла - требуемое количество
    
    1 балл - создаются, но меньше требуемого количество & 2 \\
    \hline

    Объекты перемещаются из интерфейса, с корректной привязкой & 2 балла - перемещаются с корректной привязкой

    1 балл - перемещаются & 2 \\
    \hline

    Объекты удаляются из интерфейса & 1 балл & 1 \\
    \hline


    Запуск и остановка симуляции из интерфейса & 1 балл & 1 \\ 
    \hline

    Ресурсы появляются из источников корректно & 2 балла - появляются корректно: при присоединённых конвейерах с интервалом в 2 секунды, первый ресурс появляется спустя 2 секунды после подключения.


    1 балл - появляются с ошибками: при подключении появляются ресурсы, но какое-то из условий корректного появления не выполнено" & 2 \\
    \hline

    Ресурсы движутся между объектами & 2 балла - ресурсы движутся корректно между конвейерами и другими объектами

    1 балл - ресурсы движутся, но есть видимые баги & 2 \\
    \hline

    Преобразователи работают & 2 балла - все работают корректно
    
    1 балл - некоторые работают & 2 \\
    \hline


    Падение ресурсов при съезде с конвейера & 1 балл & 1 \\
    \hline

\end{tabular}

\begin{tabular}{|p{4cm}|p{9cm}|p{1cm}|}
    \hline 
    \multicolumn{3}{|c|}{Автотесты} \\
    \hline

    Пройдены автотесты & 1 балл за каждый пройденный автотест & 8 \\
    \hline
    \hline

    \multicolumn{3}{|c|}{Требования к интерфейсу}	\\	
    \hline

    Перемещения пользователя & 2 балла - перемещение, используя телепорт и позицонирование

    1 балл - перемещение работает & 2 \\
    \hline


    Отображение привязки объектов друг к другу (конвейеры) & 2 балла - отображаются все привязки

    1 балл - отображаются некоторые привязки & 2 \\
    \hline


    Отладочная информация: имя файла, координатная сетка & 2 балла:

    +1 балл - отладочная информация выводится

    +1 балл - включается/выключается через интерфейс & 2 \\
    \hline


    Статус симуляции отображается & 1 балл & 1 \\
    \hline

    Плавное перемещение объектов при анимациях и взаимодействии с пользователем	& 2 балла &2 \\
    \hline

    Интерфейс для загрузки и сохранения файлов & 1 балл & 1 \\
    \hline
    \hline

    \multicolumn{3}{|c|}{Качество кода и структура проекта}\\
    \hline

    Форматирование & 1 балл & 1 \\
    \hline

    Осмысленные идентификаторы & 1 балл & 1 \\
    \hline

    Отсутствие дублирующего и бессмысленного кода & 1 балл & 1\\
    \hline

    Структура проекта & 1 балл & 1\\
    \hline

    Осмысленная иерархия объектов на сцене & 1 балл & 1 \\
    \hline
    \hline

    
    \multicolumn{3}{|c|}{Оформление}	\\
    \hline

    Оформление игровой комнаты (пол, стены, skybox и т.п.) & 2 балла & 2 \\
    \hline

    Оформление игровых объектов & 3 балла -- модели и текстуры качественные и красивые, самостоятельно созданные & 3 \\
    \hline

    Единая тема и стилистика оформления & 2 балла & 2 \\
    \hline

    \multicolumn{3}{|c|}{Дополнительно}	\\
    \hline
	
    Контроль версий & 2 балла & 2 \\
    \hline

    Сборка проекта & 1 балл & 1 \\
    \hline
    \hline

    \multicolumn{2}{|r|}{\textbf{Итого:}}& \textbf{56} \\
    \hline
\end{tabular}