\assignementTitle{}{40}{}

Задача выполняется в бассейне. Характеристики бассейна зависят от места проведения финала. 

Необходимо запрограммировать подводного робот, который должен выполнить под водой две подзадачи.

В расстоянии до 1 метра от старта находится полоска, размеры $14 \times 75$ см, которая направлена на щит. Аппарат должен найти полоску и повернуться по направлению к щиту. Щит расположен на расстоянии от 2 до 5 метров от полоски.

Размеры щита - $60 \times 60$ см, в центре щита имеется отверстие диаметром 25 см. Робот должен выстрелить торпедой с расстояния не менее 10 см от щита так, чтобы торпеда самостоятельно прошла сквозь отверстие.

За поиск полоски, правильный поворот и ориентацию робота дается 10 баллов.

За торпедирование мишени дается 30 баллов.

\markSection

Во время финальных заездов каждой команде будет дано 5 минут на выполнение задачи. В течение отведенного времени робот может погружаться и всплывать неограниченное количество раз. Допускается перепрограммирование. В зачет идет лучшая попытка финального заезда. Время в течение выполнения миссии не останавливается. 

Расстояние от старта до полоски, расстояние от полоски до щита, угол поворота полоски, место старта могут меняться перед финальным заездом. 