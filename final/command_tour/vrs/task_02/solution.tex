\solutionSection

Даная задача состоит из нескольких подзадач:

\begin{enumerate}
    \item Пройти вдоль линии;
    \item Сбросить маркеры в корзины со звездой;
    \item Выстрелить в нижнее отверстие с зеленой стороны мишени;
    \item Выстрелить в верхнее отверстие с желтой стороны мишени;
    \item Сесть на желтую площадку.
\end{enumerate}

Для решения данных задач участнику необходимо реализовать:

\begin{itemize}
    \item регулятор управления курсом и глубиной (подойдет релейный или простой пропорциональный);
    \item алгоритмы поиска линии, определения фигуры в корзинах, поиск мишени.
\end{itemize}

\textbf{Начнем с реализации регуляторов глубины и курса.}

\inputminted[fontsize=\footnotesize, linenos]{cpp}{final/command_tour/vrs/task_02/source_1.cpp}

Наши регуляторы готовы. 

\textbf{Теперь перейдем к поиску линии.} 

Для решения данной задачи мы воспользуемся алгоритмом бинаризации и встроенными в OpenCV алгоритмами определения угла. 

\inputminted[fontsize=\footnotesize, linenos]{cpp}{final/command_tour/vrs/task_02/source_2.cpp}

Алгоритм определения линий готов. 

\textbf{Теперь перейдем к распознаванию звезды и трапеции.}

Данный детектор нам необходим для определения фигуры в корзинах. 

\inputminted[fontsize=\footnotesize, linenos]{cpp}{final/command_tour/vrs/task_02/source_3.cpp}

Алгоритм распознавания зеленого готов. Мы будет условно считать звезду «кругом», а трапецию «квадратом».  

Теперь добавим алгоритм движения к центру корзины. 

Для этого нам достаточно смещаться влево или вправо относительно координаты X на кадре: 

\inputminted[fontsize=\footnotesize, linenos]{cpp}{final/command_tour/vrs/task_02/source_4.cpp}

У нас готовы алгоритмы распознавания корзины и движения к ее центру. 

\textbf{Теперь перейдем к поиску мишени и алгоритму движения к ее центру.} 

Искать мишень мы будем при помощи бинаризации по желтому и зеленому цвету, выбирая самый «высокий» или «низкий» квадрат в кадре, в зависимости от стороны.

\inputminted[fontsize=\footnotesize, linenos]{cpp}{final/command_tour/vrs/task_02/source_5.cpp}

Алгоритм поиска желтого и зеленого готов, теперь перейдем к алгоритму движения к центру мишени:

\inputminted[fontsize=\footnotesize, linenos]{cpp}{final/command_tour/vrs/task_02/source_6.cpp}

Для распознавания зеленой площадки мы будем использовать этот же алгоритм. 

Собираем все вместе и получаем следующий код:

\inputminted[fontsize=\footnotesize, linenos]{cpp}{final/command_tour/vrs/task_02/source_7.cpp}