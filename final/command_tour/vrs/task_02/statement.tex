\assignementTitle{}{30}{}

\putImgWOCaption{7cm}{1}

Робот стартует над началом оранжевой линии. Отдельно баллы за прохождение по линии не начисляются. 
Баллы начисляются за следующие задачи:
\begin{itemize}
    \item Сбросить два маркера в корзины со звездочкой. Порядок маркеров в корзине может меняться. 5 баллов за маркер. Попадание в корзину с трапецией – 5 штрафных баллов. 
    \item Выстрелить в нижнее отверстие с зеленой стороны мишени (7.5 баллов). Выстрелить в верхнее отверстие мишени с желтой стороны (7.5 баллов). 
    \item Сесть аппаратом на желтый квадрат (5 баллов). 
\end{itemize}

Попытка завершается если:
\begin{itemize}
    \item Аппарат сел на желтый квадрат.
    \item Аппарат всплыл.
    \item Закончилось время.
\end{itemize}

Время на выполнение задачи: 2 минуты

Максимальное количество баллов: 30

Количество попыток сдачи решений: 3

\markSection

Команда передают на флешке или отправляют на специальную почту решения в виде проекта MUR IDE. Организаторы проверяют решение на 3 сценах и в течение 30 минут отправляют команде результат тестирования: какие задачи в каждой сцене выполнил робот и сколько баллов команда получила за каждую сцену. Результатом является среднее арифметическое за три сцены. Команда может отправлять решения 3 раза.  

В зачет идет лучшее решение команды. 

Например, за первое решение команда заработала 20 баллов, за второе решение – 11 баллов, за третье – 13 баллов. В зачет пойдет 20 баллов.