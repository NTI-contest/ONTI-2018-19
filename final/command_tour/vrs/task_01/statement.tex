\assignementTitle{}{30}{}

Каждая команда Олимпиады НТИ по профилю Водные робототехнические системы будет иметь в своем распоряжении набор элементов, из которых в течении 4 дней необходимо будет разработать и изготовить устройство для запуска торпед, совместимое с MUR. 

Требования к устройству:
\begin{itemize}
    \item Размеры устройства: не регламентируются
    \item Размеры торпеды (диаметр $\times$ длина), мм: не менее $10 \times 50$
    \item Вес устройства: не регламентируется
    \item Плавучесть: нейтральная ($\pm 10$ грамм)
    \item Количество выстрелов без перезарядки: не менее 1
\end{itemize}

Задача предусматривает выполнение командами следующих подзадач:
\begin{itemize}
    \item Разработка конструкции
    \item Электромонтаж 
    \item Программирование устройства, его тестирование и отладка
\end{itemize}

Список комплектующих и материалов, которые будут выданы каждой команде, приведен в таблице 1.

\begin{table}[H]
    \center
    \caption{Таблица 1. Список комплектующих и материалов}
    \begin{tabular}{|l|l|}
        \hline
        \textbf{Наименование} & \textbf{Кол-во} \\
        \hline
        Перчатки & 1 \\
        \hline
        Холодная сварка & 1 \\
        \hline
        Кабель, 50 см & 1 \\
        \hline
        Мотор коллекторный & 2 \\
        \hline
        Разъем четырехпиновый & 1 \\
        \hline
    \end{tabular}
\end{table}

Команды будут иметь доступ к паяльным станциям, мультиметрам, кусачкам, пинцетам, источникам питаниям, 3Д-принтеру и другим электромонтажным и форомообразующим инструментам. 

В таблице 2 приведена распиновка разъема четырехконтактного, подключаемого к MUR. 

ВНИМАНИЕ: при распайке разъема необходимо проявить максимальное внимание. При неправильной распайке может выйти из строя Модуль бортового компьютера MUR.

\begin{table}[H]
    \center
    \caption{Таблица 2. Распиновка четырехконтактного разъема}
    \begin{tabular}{|l|l|}
        \hline
        1 & - \\
        \hline
        2 & - \\
        \hline
        3 & Контакт двигателя \\
        \hline
        4 & Контакт двигателя \\
        \hline
    \end{tabular}
\end{table}

Информация для программирования устройства.

Конструктор подводного робота MUR на портах Thrust имеет 2 PWM вывода напряжением 12 В, которые используются для управления коллекторными моторами. Управление скважностью PWM происходит с помощью функции из MurAPI – mur.setPortA(B,C,D)(int power), где A, B, C , D – это индекс порта Thrust, а power параметр, принимающий значение от -100 до 100, означающий скважность на одном из выводов PWM.

Примеры:
\begin{enumerate}
    \item Если вызвать функцию mur.setPortA(0) , то это будет означать что на порту Thrust с индексом A будет выставлен уровень PWM в 0, т. е. оба вывода PWM будут держать низкий уровень напряжения и мотор двигаться не будет.
    \item Если вызвать функцию mur.setPortB(50), то это будет означать что на порту Thrust с индексом B будет выставлен уровень PWM со скважностью 50\% на выводе №1, а №2 будет держать низкий уровень. Последствием вызова этой функции будет, то, что двигатель начнет вращение в определённом направление с 50\% скоростью. Для большего понимая, будет считать, что направление движения мотора будет по часовой стрелки. Хотя оно может и отличаться в зависимости от того, как были припаяны контакты на мотор.
    \item Если вызвать функцию mur.setPortC(-50), то это будет означать что на порту Thrust с индексом C будет выставлен уровень PWM со скважность 50\% на выводе №2,  а №1 будет держать низкий уровень. Последствием вызова этой функции будет, то, что двигатель начнет вращение в противоположном направлении (в условиях примера, против часовой стрелки) со скоростью в 50\%.
\end{enumerate}

Таким образом, для того чтобы управлять коллекторным двигателем с помощью конструктора MUR, необходимо подключить его в соответствующий порт Thrust A, B, C или D. Далее на этот порт подать необходимую тягу, где знак тяги определяет направление вращения двигателя, а значение определяет силу с которой вращается двигатель, если подать 0 то мотор двигаться не будет, если подать +\- 100 мотор будет вращаться с максимальной скоростью в направлении заданном знаком тяги.

\markSection

\begin{table}[H]
    \center
    \small
    \begin{tabular}{|c|l|c|}
        \hline
        № & Критери & Баллы  \\
        \hline
        \multicolumn{3}{|c|}{Электроника} \\
        \hline
        1 & Качество пайки разъема (аккуратность, использование термоусадки и др.) & 2 \\
        \hline
        2 & Качество пайки контактов мотора (аккуратность, использование & \\
          & термоусадки и др.) & 2 \\
        \hline
        \multicolumn{3}{|c|}{Конструирование} \\
        \hline
        3 & Качество изготовления, обработки деталей конструкци & 6 \\
        \hline
        4 & Наличие и соответствие 3D-модели реальному устройств & 4 \\
        \hline
        \multicolumn{3}{|c|}{Работоспособность} \\
        \hline
        5 & Устройство «выстреливает» торпедой на воздухе минимум на 50 см. & \\
          & При этом устройство установлено и подключено к аппарату, робот стоит & \\
          & на столе или полу, запуск производит конкурсант.  & 10 \\
        \hline
        6 & Надежность (экспериментатор может повторить подряд «выстрел»& \\
          & не менее 3 раз) & 6 \\
        \hline
    \end{tabular}
\end{table}

