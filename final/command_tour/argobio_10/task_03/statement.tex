\assignementTitle{}{8}{}

Результат приведите в письменном виде. Листок с ответом подпишите (название команды) и проставьте дату.

\textbf{Задание выполняется “online”}

\markSection

\begin{enumerate}
    \item Значение параметров содержания карпа представлены верно -2 балла
    \item Значение параметров содержания стерляди представлены верно – 2 балла
    \item Определён диапазон значений, оптимальный для смешанной аквакультуры карпа и стерляди – 4 балла
\end{enumerate}

Значение параметров представлено с ошибками – 0 баллов (по п.п. 1 и 2)

Не определён оптимальный диапазон для смешанной аквакультуры – 0 баллов (по п.3.)

\solutionSection

\begin{table}[H]
    \caption{Таблица значений параметров для содержания карпов Коя}
    \begin{tabular}{|l|l|l|l|}
        \hline
        Параметр &от&	до&	Оптимум \\
        \hline
        рН&	6,5	&9,5&	7-8&\\
        \hline
        Аммоний-ион моль/куб.м.	&	2,8*10-2	& \\
        \hline
        Нитрит-ион моль/куб.м.	&	4,3*10-4&	\\
        \hline
        Нитрат-ион моль/куб.м.		1,6*10-2	& \\
        \hline
        Диоксид углерода 
        растворенный моль/куб.м.	&	2,3*10-1	& \\
        \hline
        Температура воды, град. С	&+20&	+24	& \\
    \end{tabular}
\end{table}

    \begin{table}[H]
        \caption{Таблица значений параметров для содержания Стерляди}
        \begin{tabular}{|l|l|l|l|}
        \hline
        Параметр &	от &	до &	Оптимум \\
        \hline
        рН &	7,0 &	7,7	& \\
        \hline
        Аммоний-ион моль/куб.м.	&   & 1,25*10^{-2}  & \\
        \hline
        Нитрит-ион моль/куб.м. &	&	2,0*10^{-4} &	\\
        \hline
        Нитрат-ион моль/куб.м. &	&  0 ,9*10^{-2}  & \\
        \hline	
        Диоксид углерода 
        растворенный  моль/куб.м. &    & 1,0*10^{-1}	&\\ 
        \hline
        Температура  воды, град. С	&  +18 & +22	&\\
        \end{tabular}
    \end{table}
\begin{table}[H]
    \caption{Таблица параметров содержания в УЗВ смешанной культуры карп Кои и Стерлядь 
    (оптимизируем по минимальному значению диаппазона – лимитирующим факторам среды)}
    \begin{tabular}{|l|l|l|l|}
        \hline
        Параметр &	от &	до &	Оптимум \\
        \hline
        рН	& 7,0 &	7,7	& \\
        \hline
        Аммоний-ион моль/куб.м.	&  &	1,25*10^{-2}	& \\
        \hline
        Нитрит-ион моль/куб.м. &  &	    2,0*10^{-4}	& \\
        \hline
        Нитрат-ион моль/куб.м. &  &	0,9*10^{-2} & \\
        \hline	
        Диоксид углерода 
        растворенный   моль/куб.м. &  &		1,0*10^{-1}	& \\
        \hline
        Температура , град. С &	+20	 & +22 &	\\  
    \end{tabular}
\end{table}