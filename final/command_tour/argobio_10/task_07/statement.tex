\assignementTitle{}{10}{}

\begin{enumerate}
    \item  Убедиться в правильности работы блока MQTT вход (ИС GRED). Вывести показания всех доступных датчиков в панель отладки. - 1 балл
    \item Создать информационную панель с выводом исторических данных с датчиков в виде графиков, выводом последнего значения с датчиков в виде текста с разбивкой по группам. - 3 балла
    \item Создать оповещение при превышении или понижении концентрации карбонатов до опасного уровня. - 2 балла
    \item Создать оповещение по электронной почте при отключении измерительной системы от сети интернет. - 2 балла
    \item Создать оповещение по электронной почте при поступлении неадекватных показаний с датчиков. - 2 балла
    
\end{enumerate}

\markSection

Задание выполняется с использованием  программирования при помощи построения блок-схем из готовых “кусков” кода на онлайн-платформе GreenPL. 
Успешность выполнения определяется следующими требованиями:
\begin{enumerate}
    \item Все датчики “видятся” онлайн - платформой
    \item Все необходимые оповещения срабатывают с минимальной задержкой
    \item Не происходит ложных срабатываний предупреждений
\end{enumerate}

\subsubsection*{Дополнительные задания}

\textbf{Критерии оценки ведения журнала (10 баллов)}

\begin{enumerate}
    \item	Дата, название команды, состав команды – 1 балл
    \item	Нумерация страниц – 1 балл
    \item	Точность фиксации задачи дня -  1 балл
    \item	Точность фиксации применяемого оборудования, расходных материалов, реагентов – 1 балл
    \item	Наличие плана реализации задачи дня – 1 балл
    \item	Наличие расчётов, комментариев, дополнений и/или примечаний по ходу ведения журнала – 1 балл
    \item	Наличие выводов или предварительных итогов дня – 1 балл
    \item	Заключительный итог по работе с 12 по 15 – 3 балла
\end{enumerate}
    Максимальное количество баллов – 10


\textbf{Инженерная и исследовательская культура при проведении работ с аквапонной установкой (17 баллов)}

\begin{enumerate}
    \item Действие не подкреплённое обоснованием (гипотезой, опорой на знания физики, химии, математики, биологии) – минус 1 балл 
    \item	Действие, противоречащее требованиям к безопасной эксплуатации установки (попытка работать с агрегатами системы (помпы, лампы) в погруженном состоянии, при включенном электропитании – 2 балла
    \item	Организация рабочего места 5 баллов (наличие мусора, бессистемное расположение записей, проливание р-ров на записи -  минус один балл за каждый зафиксированный случай)
    \item	Грамотное инженерное решение задачи по предотвращению перелива из растительноводного фильтра (сифоном, самотёком после прекращения работы помпы) – 5 баллов;
    \item	Грамотное решение по стабилизации скоростей движения воды в двух помпах – 2 балла
    \item	Применение уникального, не стандартного решения в реализации инженерной задачи – 5 баллов.
\end{enumerate}

