\assignementTitle{}{6}{}

Представьте себя на месте заведующего лабораторией: в вашем распоряжении много реактивов, оборудования и персонал. Напишите свой список правил (не менее~5), которые, на ваш взгляд, обеспечивают безопасность в вашей лаборатории.

Варианты правильных ответов (возможны и другие разумные правила), за каждый правильный вариант ставился 1 балл.

\answerMath

\begin{enumerate}
    \item Пройти инструктаж на рабочем месте по технике безопасности и пожарной безопасности
    \item Сообщить о возможных аллергических реакциях и прочих ограничениях по работе
    \item Категорически запрещается: пробовать на вкус любые реактивы и растворы, пить и есть, класть продукты на рабочие столы в кабинетах и лаборантской, принимать пищу в спецодежде
    \item Соблюдать порядок проведения лабораторных и практических работ, демонстрационных экспериментов, а также правила личной гигиены, содержать в чистоте рабочее место
    \item Все опыты с токсичными, летучими веществами производятся только в вытяжном шкафу.
    \item Каждый обязан знать правила оказания первой помощи, а также знать состав аптечки в кабинете (и для чего каждое наименование используется)
    \item При выполнении практической работы в кабинете: при выполнении лабораторной или практической работы, в том числе в процессе демонстрационного эксперимента с использованием реактивов, обязательно наличие надетых халатов у всех находящихся в помещении людей.
    \item Все лица, наблюдающие вблизи (менее 3-4 метров) или проводящие манипуляции с:
    \begin{itemize}
        \item концентрированными (более 10\%) растворами солей и др. веществ;
        \item растворами кислот и щелочей;
        \item любыми реактивами при нагревании, перетирании и дроблении;
        \item токсичными и агрессивными веществами обязаны помимо халатов надевать очки и перчатки.
    \end{itemize}
    \item При работе с
    \begin{itemize}
        \item приборами и оборудованием, оснащенными источником открытого огня;
        \item пылеобразными веществами;
    \end{itemize}
    обязаны помимо халатов надевать очки, для пылеобразных – очки и респираторы.
    \item В канализацию запрещается выбрасывать реактивы, сливать их растворы, легко воспламеняющиеся жидкости (ЛВЖ) и горючие жидкости (ГЖ).
    \item Обеспечить надлежащий вид, соответствующий особенностям планируемой работы (см. пункты 8 и 9), а также:
    \begin{itemize}
        \item не допускается работа на высоких каблуках (допустимая высота каблуков при работе в химической лаборатории не более 2 см);
        \item все лица, имеющие длинные волосы, собирают и заправляют их под халат во избежание случаев возгорания волос во время работы с потенциальными источниками открытого огня;
        \item рукава химических халатов и основные пуговицы должны быть застегнуты.
    \end{itemize}
\end{enumerate}