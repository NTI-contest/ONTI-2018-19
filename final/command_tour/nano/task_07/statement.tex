\assignementTitle{}{6}{}

Объясните, почему ДМФА позволяет получить раствор с солями, в толуоле происходит кристаллизация с образованием наночастиц $CsPbBr_3$, а ацетон способен растворить частицы обратно с образованием раствора.

\explanationSection

ДМФА – Это соединение $HOC-N(CH_3)_2$, которое является полярной молекулой. В полярной среде процесс из вопроса 1б протекает в большей степени, а потому позволяет полностью растворить исходные соли. Толуол – неполярный растворитель, в котором происходит смещение равновесия реакции образования перовскитного соединения вправо: $Cs^+ + Pb^{2+} + 3Br^- \rightarrow CsPbBr_3$, образуются нанокристаллы. Ацетон – полярное соединение, которое имеет более высокую полярность, чем ДМФА. Добавление к нанокристаллам приводит к запуску процесса из вопроса 1б и быстрому растворению.