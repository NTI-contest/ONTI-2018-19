\assignementTitle{}{6}{}

В вашем распоряжении имеется 5 лазеров: фиолетовый (405 нм), синий (450 нм), зеленый (520 нм), лазеров могут желтый (590 нм) и красный (680 нм). Какие из быть использованы в качестве источника возбуждающего излучения в спектроскопии люминесценции, если вы хотите получить спектр люминесценции квантовых точек состава $CsPbBr_3$.

4 балла за правильный ответ, 2 балла за учет ширины запрещенной зоны в квантовых точках

\explanationSection

Ширина запрещенной зоны у объемного $CsPbBr_3$ - 2.3 эВ. Энергия фиолетового, синего, зеленого, желтого и красного лазеров составляет величину 3.06, 2.76, 2.38, 2.1, 1.82 эВ согласно формуле:

$$E, \: \text{эВ}=  1240/(\lambda, \: \text{нм}).$$ 

Лазер может возбудить люминесценцию  только имея энергию выше, чем ширина запрещенной зоны. Правильные ответы, таким образом: фиолетовый (405 нм), синий (450 нм), зеленый (520 нм)

Однако, если получать квантовые точки (имеют более широкую запрещенную зону), то энергии зеленого лазера может и не хватить.