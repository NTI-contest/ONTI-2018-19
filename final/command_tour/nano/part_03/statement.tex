\underline{Приготовление клея из орг. стекла}

Стружка из орг. стекла может быть растворена в толуоле при небольшом нагревании и интенсивном перемешивании. Налейте в стакан (объемом 25 мл) 8 мл толуола, поместите магнитный якорек и поставьте на плитку. Нагрейте плитку до 60-70$^\circ C$ и при интенсивном перемешивании маленькими порциями добавьте 0.5 г порошка орг. стекла. Дождитесь полного растворения орг. стекла и образования прозрачного клея (требуется минут 30-40). Пока клей еще горячий перелейте его в баночку для хранения.

\underline{Подготовка матрицы к нанесению пикселей}

Люминесцентная матрица в задаче изготавливается из фиолетовой матрицы, состоящей из 100 светодиодов (см. рис. 1), которая является источником возбуждающего излучения. Для создания других цветов на матрицу накладывается специальная маска, которую участники изготавливают самостоятельно: из покровного стекла и прорезиненной наклейки с отверстиями, повторяющими расположение светодиодов. Предпочтительная последовательность изготовления:

\begin{enumerate}
    \item прорезиненная наклейка изготавливалась при помощи лазерного плоттера, который вырезал отверстия, необходимо при помощи пинцета удалить мешающие кусочки резины;
    \item смочите салфетку небольшим количеством ацетона и протрите покровное стекло;
    \item удалите бумагу с прорезиненной наклейки и аккуратно наклейте на чистое покровное стекло.
\end{enumerate}

\underline{Приготовление клея с зеленой люминесценцией и нанесение клея на матрицу}

Перенесите 0.3-0.5 мл клея в микроцентрифужную пробирку (эппендорфа) при помощи автоматической пипетки. Добавьте туда квантовые точки $CsPbBr_3$ в твердом виде в количестве, соизмеримом с размером спичечной головки. При помощи наконечника от пипетки как следует размешайте клей с наночастицами до получения однородной смеси.

При помощи автоматической пипетки нанесите полученный люминесцирующий клей на изготовленную ранее матрицу, при этом старайтесь попадать точно в углубления в прорезиненной наклейке. После высыхания повторите нанесение, добиваясь полного заполнения углубления клеем. Для проверки качества изготовления пикселей установите маску на матрицу с фиолетовыми светодиодами и подключите ее питание.

\underline{Приготовление клея с красной и синей люминесценцией и изготовление матриц}

\underline{с желтой, бирюзовой и фиолетовой люминесценцией}

Для получения клея с синей люминесценцией следуйте тем же инструкциям, что и при приготовлении клея с зеленой люминесценцией. При приготовлении клея с красной люминесценцией в большинстве случаев наблюдается затруднения в смешивании клея и квантовых точек, для повышения гомогенизации смеси необходимо добавить пару капель трихлорметана (можно больше).

Для создания матриц с желтой, бирюзовой и фиолетовой люминесценцией необходимо поочередно (чередуя) наносить клей с красной+зеленой, зеленой+синей и синей+красной люминесценцией соответственно в разные углубления прорезиненной наклейки. Для проверки качества изготовления пикселей устанавливайте маски на матрицу с фиолетовыми светодиодами.