\underline{Приготовление растворов для замены $Br^-$ на $I^-$ и $Br^-$ на $Cl^-$}

Рассчитать сколько необходимо взять $PbCl_2$ и $PbI_2$ для приготовления двух растворов объемом 12 мл каждый. Концентрация растворов – 0.018 моль/л, в качестве растворителя используется октадецен-1, также в смесь следует добавить OLA и OLAm. Соотношение по объему между октадеценом-1, OLA и OLAm составляет 10:1:1 соответственно. Результаты вычислений занесите в таблицу:

\begin{table}[H]
    \begin{center}
        \begin{tabular}{|p{2.3cm}|p{2.3cm}|p{2.3cm}|p{2.3cm}|p{2.3cm}|p{2.3cm}|p{2.3cm}|}
            \hline
            &\cellcolor[HTML]{C0C0C0}$PbCl_2$, г	& \cellcolor[HTML]{C0C0C0}$PbI_2$, г	&\cellcolor[HTML]{C0C0C0}OLA, мл	&\cellcolor[HTML]{C0C0C0}OLAm, мл	&\cellcolor[HTML]{C0C0C0}Октадецен-1, мл \\
            \hline
            \cellcolor[HTML]{C0C0C0} Раствор 1	&  & \cellcolor[HTML]{C0C0C0}Отсутствует & & & \\
            \hline
            \cellcolor[HTML]{C0C0C0}Раствор 2& \cellcolor[HTML]{C0C0C0}Отсутствует& & & & \\
            \hline
        \end{tabular}
    \end{center}
\end{table}

Растворы готовить по очереди. Вначале взвесить необходимое количество хлорида свинца, поместить его в плоскодонную колбу на 25 мл, добавить OLA и OLAm, поместить в колбу магнитный якорек. Поставить смесь на плитку и интенсивно перемешать, добавить рассчитанное количество октадецена. Растворить содержимое при перемешивании при температуре плитки 180$^\circ C$ (для PbCl2) (для растворения $PbI_2$ достаточно температуры в 150$^\circ C$). Рекомендуется вести растворение не менее, чем в течение 30 минут. По окончании плитку выключают и дают остудиться раствору при перемешивании. Раствор переносят при помощи шприца в баночку для хранения.

Аналогичным образом получают раствор иодида свинца. Следует отметить, что в процессе растворения солей оба раствора меняют свой цвет, например, раствор иодида свинца становится бурым, однако через некоторое время практически полностью обесцвечивается. Именно отсутствие явно выраженной окраски свидетельствует о получении необходимого раствора. Полученный раствор иодида свинца также помещают в баночку для хранения.

\underline{Получение квантовых точек с синей люминесценцией}

Поместите в плоскодонную колбу на 25 мл некоторое количество концентрированного золя квантовых точек $CsPbBr_3$ (от 1 до 15 мл). Установите колбу на магнитную мешалку, добавьте магнитный якорек и нагрейте до температуры 40-60$^\circ C$. Добавляйте раствор $PbCl_2$ по каплям (2 капли могут уже существенно поменять цвет), между добавлениями ждите не менее 30 секунд (оптимально около минуты) для достижения гомогенности смеси. Если скорость изменения цвета мала, то можно добавлять по 0.1 мл раствора $PbCl_2$ с помощью автоматической пипетки или увеличить скорость прикапывания. По завершении гомогенизации каждой порции необходимо проверять цвет люминесценции при помощи фиолетового лазера и, по достижении нужного цвета, - при помощи спектрометра.

Добавляйте порции $PbCl_2$ до тех пор, пока люминесценция наночастиц не станет синей. Следует понять, что \textbf{процесс анионного обмена протекает «по инерции»}, то есть после добавления порции хлорида свинца цвет может плавно меняться еще в течение минуты и даже больше. По достижении нужной длины волны можно отключить нагрев и перемешивать смесь еще 15 минут в процессе остывания. Повторить контрольное измерение спектра, записать полученную кривую в txt файл и провести процедуру установления длины волны люминесценции в программе Excel (график сохранить), после чего провести процедуру выделения наночастиц.

Провести исследование полученных наночастиц на предмет их взаимодействия с ацетонитрилом. Отберите малую порцию (0.1-0.3 мл) полученных наночастиц с синей люминесценцией в 2 пробирки эппендорфа и добавьте в одну двукратный, а в другую - пятикратный избыток ацетонитрила. Встряхните содержимое пробирок и определите осталась ли люминесценция. Центрифугируйте растворы в течение 2-3 минут, для уравновешивания воспользуйтесь пробирками с водой. Определите, удалось ли осадить наночастицы с синей люминесценцией. Если люминесценция пропала – центрифугируйте всю массу исходных наночастиц без ацетонитрила, если осталась и для вас очевидно положительное влияние ацетонитрила в каком-либо случае – используйте метод с ацетонитрилом.

Супернатант, обедненный квантовыми точками помещается в банку для слива органики, а к осадок сушится на воздухе в течение 15-20 минут и шпателем переносится в маленькую центрифужную пробирку (эппендорфа). Если супернатант содержит большое количество частиц, которые не подвергаются осаждению – также соберите его в отдельную емкость.

\underline{Получение квантовых точек с красной люминесценцией}

Схема действий при синтезе наночастиц схожа со случаем синтеза наночастиц с синей люминесценцией, однако имеются некоторые расхождения.

Поместить от 1 до 15 мл зеленых нанчоастиц в плоскодонную колбу на 25 мл с якорьком. Разогреть до температуры 40-60$^\circ C$ при перемешивании. Добавлять раствор $PbI_2$ порциями по 0.1 мл с интервалом не менее 30 сек, проверяя цвет люминесценции фиолетовой лазерной указкой. Примечание: для достижения красного цвета (680 \textpm 30 нм) вам понадобится существенно больше раствора $PbI_2$, чем до этого требовалось раствора $PbCl_2$ для получения квантовых точек с синей люминесценцией. Если вы установили, что окраска меняется слишком медленно, то можно постепенно увеличить объем вводимой порции в пределе до 0.5 мл. Увеличение объема порции должно также сопровождаться увеличением времени между интервалами (не менее 2 минут для порции 0.5 мл).

При приближении к красному цвету следует проводить контрольные измерения на спектрофотометре. Достигнув требуемой длины волны излучения, можно выключить нагрев и дать колбе остыть, не выключая перемешивание. Повторить контрольное измерение спектра, записать полученную кривую в txt файл и провести процедуру установления длины волны люминесценции в программе Excel (график сохранить), после чего провести процедуру выделения наночастиц.

При выделении наночастиц также следует исследовать их взаимодействие с ацетонитрилом.

Ацетонитрил хорош как осадитель наночастиц, однако зачастую может пагубно влиять на стабильность коллоидной системы и, тем самым, приводить к растворению квантовых точек.