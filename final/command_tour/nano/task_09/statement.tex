\assignementTitle{}{6}{}

Изобразите блок-схему прибора в методе спектроскопии люминесценции, обозначьте основные компоненты, опишите процессы происходящие в процессе измерения, как получается спектр?

\explanationSection

\putTwoImg{7cm}{1}{7cm}{2}

Световой пучок из образца попадает на дифракционную решетку, после чего разлагается на спектральные составляющие – свет с различной длиной волны. Разложенный спектр попадает на кремниевый фотоэлектрический преобразователь (ФЭП), разбитый на каналы (сегменты).  ФЭП умеет превращать падающее на него излучение в электрон-дырочные пары (заряженные частицы), которые, перемещаясь к электродам, создают определенную разность потенциалов в каналах. Чем больше каналов, тем выше разрешающая способность такого детектора (способность различать соседние длины волн). Для повышения чувствительности в схему могут добавляться дополнительные элементы, такие как линзы, или устройство для покачивания дифракционной решетки из стороны в сторону, что также позволяет более тщательно изучить попадающий пучок.

Сигнал с ФЭП обрабатывается микропроцессором и поступает на компьютер, где уже строится спектр излучения в единицах интенсивность (ось Y) – длина волны излучения, нм (ось X). При необходимости спектр люминесценции может быть построен от энергии излучения, которая измеряется в электрон-вольтах (эВ). 

В энергетической шкале спектры излучения приобретают физический смысл и могут сравниваться, например, с шириной запрещенной зоны полупроводника с целью расшифровки природы полос в спектре.

Коллоидный раствор наночастиц помещается в кювету, которая устанавливается в кюветное отделение. Для возбуждения излучения наночастиц используется ультрафиолетовый лазер. Для возбуждения видимого излучения используют фиолетовый лазер на 400 нм, а также лазеры с меньшей длиной волны, (см. спектр излучения [52], пик на 800 нм появляется как артефакт – второй порядок дифракции на дифракционной решетке). Лазерный пучок практически полностью поглощается наночастицами в кювете, остаток излучения проходит дальше. Под углом 90$^\circ$ располагается детектор излучения, собирающий излучение наночастиц. Для соединения различных частей оптической системы используется оптическое волокно (или оптоволокно).