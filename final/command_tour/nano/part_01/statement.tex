\underline{Подготовка осадителя и раствора исходных солей}

В качестве исходных реагентов для получения квантовых точек состава $CsPbBr_3$ используются бромиды цезия и свинца ($CsBr$ и $PbBr_2$). Соли растворяют в диметилформамиде (ДМФА) в присутствии поверхностно-активных веществ (ПАВ), таких как олеиновая кислота (OLA) и олеиламин (OLAm). Рассчитайте какое количество солей (в миллиграммах) и какой объем ДМФА и ПАВ необходимо взять для приготовления раствора с суммарным объемом 7 мл с концентрацией ионов цезия в растворе 35 мМ. При этом учесть, что концентрация ионов цезия должна быть равна концентрации ионов свинца, а соотношение ДМФА:OLA:OLAm = 20 : 2 : 1 по объему. Заполните таблицу ниже:

\begin{table}[H]
    \begin{center}
        \begin{tabular}{|p{2.2cm}|p{2.2cm}|p{2.2cm}|p{2.2cm}|p{2.2cm}|}
            \hline
            Масса CsBr, мг	&Масса PbBr2, мг	&Объем OLA, мл&	Объем OLAm, мл	&Объем ДМФА, мл\\
            \hline
            & & & & \\
            \hline
        \end{tabular}
    \end{center}
\end{table}
    				
Взвесьте рассчитанное количество солей и поместите в плоскодонную колбу на 25 мл, поместите в колбу магнитный якорек малого размера. С использованием автоматической пипетки или шприца на 2 мл добавьте необходимое количество OLA и OLAm. До добавления ДМФА как следует перемешайте смесь на магнитной мешалке без нагревания. Добавить в колбу рассчитанное количество ДМФА при помощи шприца, повысить температуру плитки до 120-130$^\circ C$ (контроль за температурой осуществлять внешней термопарой). При интенсивной работе магнитной мешалки обеспечить растворение большей части солей (на дне останется небольшое количество), для этого разогретый раствор перемешивать в течение 10-15 минут.

При помощи воронки и мерного цилиндра налить в стакан, объемом 250 мл, толуол в десятикратном избытке по отношении к первому раствору – 70 мл. Поместить на дно магнитный якорек.

\underline{Проведение синтеза квантовых точек, выделение наночастиц}

Отключить нагрев и перемешивание на плитке. Аккуратно убрать плоскодонную колбу с плитки (помните, что колба горячая, берите ее за верхнюю часть) и набрать в шприц на 10 мл горячий раствор из колбы. Поставить на плитку стакан с толуолом, включить интенсивное перемешивание раствора и быстро влить раствор из шприца в стакан.

В результате (в зависимости от того насколько хорошо прореагировали соли c OLA и OLAm) можно получить либо раствор желто-зеленого цвета (коллоидный раствор квантовых точек), либо эмульсию (мутный раствор белого цвета). При получении квантовых точек раствор перемешивают еще 1 минуту и снимают с плитки, а при получении эмульсии следует повторить синтез сначала. В случае, если раствор мутный и на фоне белого цвета заметна окраска, то раствор необходимо выдержать при перемешивании при температуре 80$^\circ C$ в течение 40-60 минут.

Далее полученные частицы выделяют методом центрифугирования.

\underline{Выделение и концентрирование квантовых точек $CsPbBr_3$}

Раствор квантовых точек помещают в центрифужную пробирку, и добавляют ацетонитрил в соотношении 1:1 по объему. При необходимости следует использовать несколько центрифужных пробирок, заполняя их на равный объем для уравновешивания центрифуги. Осаждение вести на 6000 об/мин в течение 5 минут.  По окончании центрифугирования супернатант должен не иметь окраски, если это так, то можно его перенести в банку для слива органики.

К осадку добавляют 1 мл толуола с помощью шприца или автоматической пипетки, тщательно перемешивают взбалтыванием и помещают в ультразвуковую ванну на 2-3 секунды. Затем при помощи автоматической пипетки переносят концентрированный золь квантовых точек в толуоле с не растворившемся осадком в микроцентрифужные пробирки (эппендорфа), как следует взбалтывают и повторно помещают в ультразвуковую ванну на 2-3 секунды. Пробирки центрифугируют 6000 об/мин в течение 15 секунд.

Супернатант должен представлять собой концентрированный золь наночастиц, который при помощи шприца переносят в баночку для хранения золя (концентрация проверяется свечением на золь фиолетовым лазером: если лазерный луч поглощается в тонком слое золя, то его можно считать концентрированным). Осадок от центрифугирования оставляют в открытых пробирках до высушивания на воздухе в вытяжном шкафу (как правило требуется около 10 минут). После полного высушивания осадок шпателем собирают со всех эппендорфов в один и оставляют для дальнейшего использования.

\underline{Измерение спектров люминесценции}

Перенесите 0.1 мл концентрированного золя квантовых точек в спектрофотометрическую кювету, после чего добавьте чистый толуол, заполнив кювету до необходимого для измерения уровня. Убедитесь в чистоте кюветы с внешней стороны, при необходимости уберите грязь или разводы салфеткой. Проведите измерение спектра люминесценции согласно инструкции. Сохраните отснятый спектр в формате txt и запишите на Flash-носитель.

Постройте спектр в программе Microsoft Excel, определите положение максимума люминесценции.