\assignementTitle{}{9}{}

Ширина запрещенной зоны $CsPbCl_3$ составляет 3.06 эВ, $CsPbBr_3$ – 2.25 эВ, а $CsPbI_3$ – 1.72 эВ. Последовательное замещение ионов брома на ионы хлора или иода постепенно смещает люминесценцию от зеленой к синей или красной. Напишите гипотетическую формулу соединений в виде $CsPbCl_xBr_y$ или $CsPbBr_xI_y$, у которых длина волны люминесценции составляет 660, 560 и 470 нм.

По формуле $E, \: \text{эВ} = 1240/(\lambda, \: \text{нм})$, рассчитаем длины волн люминесценции чистых соединений: 408.5, 551 и 721 нм соответственно для $CsPbCl_3$, $CsPbBr_3$ и $CsPbI_3$

В предположении линейности изменения свойств от состава перовскитного соединения, получим следующие зависимости:

\putImgWOCaption{10cm}{1}

Подставляя в формулы соответствующие длины волн 660, 560 и 470 нм, получим:

\begin{table}[H]
    \begin{center}
        \begin{tabular}{|c|c|}
            \hline
            $CsPbBr_{1,05}I_{1,95}$& 	660 нм \\
            \hline
            $CsPbBr_{2,83}I_{0,17}$& 	560 нм \\
            \hline
            $CsPbCl_{1,4}Br_{1,6}$& 	470 нм \\
            \hline
        \end{tabular}
    \end{center}
\end{table}