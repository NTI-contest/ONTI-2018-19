\assignementTitle{}{11}{}

Напишите уравнения реакций, протекающих при растворении хлорида и бромида свинца в октадецене в присутствии ПАВ. Предположите, почему температура растворения хлорида свинца выше, чем иодида свинца.

Правильные уравнения (по 1.5 балла за реакцию).

\answerMath

$$PbCl_2 + C_{17}H_{33}COOH \leftrightarrow (C_{17}H{33}COO)PbCl + HCl$$
$$PbCl_2 + C_{17}H_{33}COOH \leftrightarrow (C_{17}H{33}COO)_2Pb + 2HCl$$
$$HCl + C_{18}H_{35}NH_2 \rightarrow C_{18}H_{35}NH_3Cl$$
$$PbI_2 + C_{17}H_{33}COOH \leftrightarrow (C_{17}H{33}COO)PbI + HI$$
$$PbI_2 + C_{17}H_{33}COOH \leftrightarrow (C_{17}H{33}COO)_2Pb + 2HI$$
$$HI + C_{18}H_{35}NH_2 \rightarrow C_{18}H_{35}NH_3I$$

Так как октадецен – неполярное соединение, то ионов в растворе не образуется, баллы снимались за присутствие ионов

\textbf{2 балла за правильный ответ:}

Хлорид свинца – более ионное соединение, чем иодид свинца. Энергия кристаллической решетки ввиду меньшего радиуса хлорид-иона выше у хлорида свинца, чем у иодида свинца. В пользу этого также у иодида ниже температура плавления. Энергия кристаллической решетки напрямую влияет на температуру растворения галогенидов.