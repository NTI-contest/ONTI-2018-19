\assignementTitle{}{3}{}

Какую опасность несут для человека соли цезия и свинца? Почему?

\explanationSection

Свинец и его соединения токсичны [*]. Особенно ядовиты водорастворимые соединения, например, галогениды свинца (II) и летучие, например, тетраэтилсвинец. Токсичны и пары расплавленного свинца.

При остром отравлении наступают боли в животе, в суставах, судороги, обмороки. Свинец может накапливаться в костях, вызывая их постепенное разрушение, концентрируется в печени и почках. Особенно опасно воздействие свинца на детей: при длительном воздействии он вызывает умственную отсталость и хронические заболевания мозга. ПДК соединений свинца в атмосферном воздухе — 0.003 мг/м$^3$, в воде — 0.03 мг/л, почве — 20.0 мг/кг. Сильное отравление взрослого наступает при попадании в организм около 0.5 г, для ребенка – уже 0.1 г.