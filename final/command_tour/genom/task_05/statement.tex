\assignementTitle{Очистка ампликона из реакционной смеси}{8}{}

\subsection*{Необходимое оборудование и реактивы}
\begin{itemize}
    \item Набор для очистки из реакционной смеси
    \item Микропробирки 1,5 мл
    \item Автоматические дозаторы 20 мкл, 200 мкл и 1 мл и наконечники к ним
    \item Настольная центрифуга
    \item Спектрофотометр для определения концентрации ДНК    
\end{itemize}

\subsection*{Протокол процедуры}

Важно! Центрифугирования проводятся в настольной центрифуге при максимальных оборотах, если не указано иное.

\begin{enumerate}
    \item Добавить 5 объемов «Связывающего раствора» (\underline{200} мкл) к 1 объему реакционной смеси, перемешать раствор.
    \item Добавить 2,5 объема изопропанола (iPrOH, \underline{100} мкл) на 1 объем реакционной смеси и перемешать раствор с помощью пипетки.
    \item Поместить спин-колонку в собирательную пробирку.
    \item Перенести пробу в колонку и центрифугировать 30 сек. Удалить фильтрат из собирательной пробирки.
    \item Добавить 700 мкл «Промывочного раствора» в колонку, центрифугировать 30 сек. Удалить фильтрат из собирательной пробирки.
    \item Центрифугировать пустую колонку 60 сек для полного удаления промывочного раствора.
    \item Поместить колонку в новую пробирку 1,5 мл.
    \item Нанести в центр мембраны 20 мкл воды, центрифугировать 30 сек.
    \item Элюат повторно нанести на колонку, центрифугировать 30 сек.
    \item Измерить концентрацию ДНК в полученном растворе при помощи спектрофотометра.
 
    \begin{tabular}{|l|c|c|c|c|}
        \hline
        & Измерение 1 & Измерение 2 & Измерение 3 & Среднее значение \\
        \hline
        $\text{С}_\text{ампликона}$,& & & & \\ 
        контрольная линия & 33 нг/мкл & 37 нг/мкл & 35 нг/мкл & 35 нг/мкл \\
        \hline
        $\text{С}_\text{ампликона}$, & & & & \\ 
        тестовая линия & 41 нг/мкл & 43 нг/мкл & 42 нг/мкл & 42 нг/мкл \\
        \hline
    \end{tabular}    

    \explanationSection
    
    Для постановки одной реакции секвенирования необходимо не менее 6 мкл раствора, содержащего очищенный целевой продукт с концентрацией 20-50 нг/мкл, в зависимости от длины фрагмента.
\end{enumerate}

\markSection

\begin{tabular}{|p{11cm}|p{3cm}|}
    \hline
    \textbf{Описание критерия} & \textbf{Балл} \\
    \hline
    \hline
    Удалены последовательности праймеров & 3 \\
    Собран контиг & 4 \\
    Обозначены рамки трансляции & 3 \\
    \hline
    \hline
    \textbf{Итог} & \textbf{10} \\
    \hline
\end{tabular}