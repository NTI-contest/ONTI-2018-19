\assignementTitle{Анализ секвенограмм, построение контига}{10}{}

\subsection*{Формулировка задания}

В представленных секвенограммах найти и удалить из рассмотрения последовательности, соответствующие праймерам. Интерпретировать неоднозначно распознанные пики. Из полученных последовательностей перекрывающихся участков генома собрать контиг, в котором обозначить рамки трансляции. Файл сохранить в рабочей папке и назвать «N Contig», где N — номер рабочего места вашей команды.

\subsection*{Входные данные}

\begin{enumerate}
    \item Секвенограммы в формате .ab1 
    \item Инструменты для работы – программа Mega X.    
\end{enumerate}

\markSection

\begin{tabular}{|p{11cm}|p{3cm}|}
    \hline
    \textbf{Описание критерия} & \textbf{Балл} \\
    \hline
    Удалены последовательности праймеров & 3 \\
    \hline
    Собран контиг & 4 \\
    \hline
    Обозначены рамки трансляции & 3 \\
    \hline
    \hline
    \textbf{Итог} & \textbf{17} \\
    \hline
\end{tabular}