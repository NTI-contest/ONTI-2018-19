\assignementTitle{Анализ данных нуклеотидных последовательностей}{13}{}

\subsection*{Формулировка задания}
Для выполнения данного задания вам необходимо определить размер ПЦР-продукта, который образуется при использовании данной пары праймеров, а также предоставить информацию о гене, который соответствует данному фрагменту ДНК.

\subsection*{Входные данные}

\begin{enumerate}
\item Последовательность праймеров в файле «Праймеры.txt» на Рабочем столе в папке «ОНТИ»

\putImgWOCaption{15cm}{1}
\item Источник ДНК – клеточная линия HEK 293
\item Онлайн инструменты для работы:

\begin{enumerate}

\item blast.ncbi.nlm.nih.gov (доступен для поиска по запросу «blast»)
\item genome.ucsc.edu, на данном ресурсе доступен инструмент для проведения ПЦР in silico (доступен для поиска по запросу «genome browser»)
\item ncbi.nlm.nih.gov (доступен для поиска по запросу «ncbi»)
\end{enumerate}

\end{enumerate}

\solutionSection

HEK 293 — клеточная линия, полученная из клетки надпочечника абортированного эмбриона человека. Следовательно, организм-источник ДНК - Homo sapiens.
С помощью инструмента “Nucleotide blast” необходимо по последовательности праймеров установить ген, фрагмент которого будет амплифицирован.

\putImgWOCaption{16cm}{2}

Результат выравнивания будет выглядеть следующим образом:

\putImgWOCaption{16cm}{3}

Последовательность праймеров идентична фрагментам гена CCR5, кодирующего  C-C-рецептор хемокина 5.
В БД ncbi в разделе “Gene” по запросу “Homo sapiens CCR5” будет доступна информация для данного гена, в том числе и положение данного гена — 3p21.31.

\putImgWOCaption{16cm}{4}

Далее необходимо провести in silico пцр с помощью онлайн ресурса genome.ucsc.edu. Ниже приведены входные данные для  in silico пцр и  ее результат:

\putImgWOCaption{16cm}{5}

Длина искомого ампликона составляет 761 п.н.

\putImgWOCaption{16cm}{6}

\subsection*{Результаты}

\begin{tabular}{|p{6cm}|p{8cm}|}
    \hline
    Размер ампликона & 761 п.н. \\
    \hline
    Организм-источник ДНК & Homo sapiens \\
    \hline
    Название гена & CCR5 \\
    \hline
    Положение гена & 3p21.31 \\
    \hline
    Краткое описание гена (физиологическая функция его продукта) & C-C-рецептор хемокина 5 — белок человека, кодируемый геном CCR5. CCR5 является членом подкласса рецепторов бета-хемокинов класса интегральных мембранных белков.
    
    CCR5 представляет собой белок адгезии и является рецептором, сопряжённым с G-белком. Белок CCR5 синтезируется преимущественно Т-клетками, макрофагами, дендритными клетками и клетками микроглии. Видимо, CCR5 участвует в процессе воспалительной реакции. Однако роль этого белка в иммунном ответе до конца не ясна.\\
    \hline
\end{tabular}

\markSection

\begin{tabular}{|p{11cm}|p{3cm}|}
    \hline
    \textbf{Описание критерия} & \textbf{Балл} \\
    \hline
    Определен размер ампликона & 2 \\
    \hline
    Определен организм-источник ДНК & 1 \\
    \hline
    Определено положение гена: 
    
    с указанием точного локуса — 4б, 
    
    только номер хромосомы — 2б, 
    
    не указан — 0 & 4 \\
    \hline
    Приведено описание гена или соответствующего ему белка: 6б или 0. Если описание недостаточно или путанно, допускается выставление 3б. & 6 \\
    \hline
    \hline
    \textbf{Итог} & \textbf{13} \\
    \hline    
\end{tabular}

