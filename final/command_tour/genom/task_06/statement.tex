\assignementTitle{Анализ результатов секвенирования целевого гена в модифицированных клеточных линиях}{18}{}

\subsection*{Формулировка задания}

Определить характер мутаций, которые произошли в геноме моноклональной модифицированной линии после воздействия системы редактирования CRISPR/Cas9 с помощью сервиса Tide.

\subsection*{Входные данные}

\begin{enumerate}
    \item На рабочем столе в папке «ОНТИ» — «Задача TIDE» находится папка с номером, соответствующим вашему рабочему месту. В данной папке размещены секвенограммы ампликонов и файл с нуклеотидными последовательностями и ссылками на онлайн-инструменты.
    \putImgWOCaption{15cm}{1}
    \item С помощью сервиса TIDE (\url{https://tide.deskgen.com/}) и иных инструментов  необходимо проанализировать результаты вашего эксперимента и ответить на вопросы таблицы
    \item При решении данной задачи вы можете пользоваться не только рекомендованными сервисами, но и любыми другими онлайн инструментами, которые сочтете нужными. 
\end{enumerate}

\solutionSection

Настройки TIDE для анализа результатов:

\putImgWOCaption{12cm}{2}

В результате анализа тестовой и контрольной секвенограмм TIDE строит спектр инделов:

\putImgWOCaption{15cm}{3}

Ту же информацию можно увидеть ниже в табличной форме:

\putImgWOCaption{8cm}{4}

В результате анализа становится ясно, что ген CCR5 несет делецию 32 нуклеотида. Редактирование прошло не полностью, поскольку в спектре наблюдается пик, соответствующий аллелю дикого типа.
С помощью онлайн инструмента \url{http://molbiotools.com/WebDSV/} можно визуализировать положение протоспейсера и определить PAM-последовательность:

\putImgWOCaption{15cm}{5}

В данном случае PAM будет выглядеть как 5’-TGG-3’.

\subsection*{Результаты}

\begin{tabular}{|p{6cm}|p{8cm}|}
    \hline
    Какие мутации несут аллели полученной клеточной линии & Делеция 32 нуклеотидов \\
    \hline
    Произошел ли полный нокаут гена-мишени?
    Объяснить развернуто ответ.	& Нет, поскольку отредактирован только один из аллелей, соответственно в клетках сохранилась работающая копия гена, полного нокаута не произошло. \\
    \hline
    Определить последовательность PAM, которая была использована для направления sgРНК & 5’-TGG-3’ \\
    \hline
    Выбрать направляющую РНК из списка ниже, которая была использована при редактировании
    (Обосновать ответ) & Для редактирования была использована sgRNA под номером 3, поскольку именно она содержит фрагмент соответствующий  указанному в задании спейсеру, т.е. она сможет связаться с нужным участком ДНК, прилежащим к PAM, и запустить процесс редактирования. \\
    \hline
\end{tabular}

Возможные sgRNA:
\begin{enumerate}
\item 5'-tacagtcagtatcaattctggttttagagctagaaatagcaagttaaaataaggctagtccgttat

caacttgaaaaagtggcaccgagtcggtgc-3'
\item 5'-tttaatgtctggaaattcttgttttagagctagaaatagcaagttaaaataaggctagtccgttat

caacttgaaaaagtggcaccgagtcggtgc-3'
\item 5'-gacattaaagatagtcatctgttttagagctagaaatagcaagttaaaataaggctagtccgttat

caacttgaaaaagtggcaccgagtcggtgc-3'
\item 5'-gaattgatactgactgtatggttttagagctagaaatagcaagttaaaataaggctagtccgttat

caacttgaaaaagtggcaccgagtcggtgc-3'
\end{enumerate}

\markSection

\begin{tabular}{|p{11cm}|p{3cm}|}
    \hline
    \textbf{Описание критерия} & \textbf{Балл} \\
    \hline
    Получен ответ по характеру мутаций с указанием размера инсерций и делеций & 3 \\
    \hline
    Описали результат — произошел ли полный нокаут? 
    
    без объяснения — 3б, 
    
    с объяснением — 6б, 
    
    неверный ответ — 0б & 6 \\
    \hline
    Правильно определен PAM в формате NGG & 3 \\
    \hline
    Выбрана верная направляющая РНК: 
    
    без объяснения — 3б, 
    
    с объяснением — 6б & 6 \\
    \hline
    Команде были выданы параметры настройки tide & (-3) \\    
    \hline
    \hline
    \textbf{Итог} & \textbf{18} \\
    \hline
\end{tabular}