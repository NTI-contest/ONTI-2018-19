\assignementTitle{Выделение геномной ДНК из клеточных лизатов}{17}{}

\subsection*{Необходимое оборудование и реактивы}
\begin{itemize}
\item Набор для выделения геномной ДНК
\item Микропробирки 1,5-2 мл
\item Автоматические дозаторы 20 мкл, 200 мкл и 1 мл и наконечники к ним
\item Настольная центрифуга
\item Спектрофотометр для определения концентрации ДНК
\item Агарозный гель
\item Буфер с красителем для нанесения проб на гель (4X Gel loading dye)
\item ДНК-маркер
\item Камера и источник питания для проведения электрофореза
\item Трансиллюминатор
\end{itemize}

\subsection*{Протокол процедуры}

Важно! Центрифугирования проводятся в настольной центрифуге при максимальных оборотах, если не указано иное.
\begin{enumerate}
\item К 200 мкл клеточного лизата добавить 200 мкл изопропанола (i-PrOH) и перемешать с помощью пипетки.
\item Колонки для выделения ДНК пометить как «контрольная» и «тестовая».
\item Перенести в колонку образец. Центрифугировать 1 мин. Удалить фильтрат.
\item Нанести на фильтр 300 мкл WB1. Центрифугировать 1 мин.
\item Нанести на фильтр 500 мкл WB2. Центрифугировать 1 мин.
\item Вылить содержимое собирательной пробирки. Центрифугировать 1 мин для удаления остатков буфера.
\item Перенести колонку в новую 1,5 мл пробирку. Нанести на фильтр 25 мкл воды. Инкубировать 5 минут. Центрифугировать 1 мин.
\item Измерить концентрацию ДНК в полученном растворе при помощи спектрофотометра, подписать концентрацию на пробирке с образцом.

\begin{tabular}{|l|c|c|c|c|}
    \hline
    & Измерение 1 & Измерение 2 & Измерение 3 & Среднее значение \\
    \hline
    $\text{С}_\text{днк}$,& & & & \\ 
    контрольная линия & 100 нг/мкл & 101 нг/мкл & 99 нг/мкл & 100 нг/мкл \\
    \hline
    $\text{С}_\text{днк}$, & & & & \\ 
    тестовая линия & 52 нг/мкл & 51 нг/мкл & 47 нг/мкл & 50 нг/мкл \\
    \hline
\end{tabular}

\item 100 нг выделенной ДНК необходимо смешать с 4X красителем для нанесения проб на гель (4X Gel loading dye).

\begin{tabular}{|l|c|c|c|}
    \hline
    & $\text{C}_\text{р-ра}$ ДНК & $\text{V}_\text{р-ра}$ ДНК & $\text{V}_\text{красителя}$ 4Х \\
    \hline
    Контрольная линия & 100 нг/мкл & 1 мкл & 0,33 мкл \\
    \hline
    Тестовая линия & 50 нг/мкл & 2 мкл & 0,67 мкл \\
    \hline
\end{tabular}

\item Нанести на гель 6 мкл маркера длин ДНК фрагментов 1kb и 100 нг ДНК образца сравнения (выполняется ассистентом на площадке).
\item Отступив одну лунку от предыдущих проб, нанести на гель в соседние лунки весь объем проб ДНК контрольной и тестовых линий.
\item Провести электрофорез в течение 20 минут.
\item Визуализировать результаты с помощью УФ-трансиллюминатора.

\putImgWOCaption{15cm}{1}
Результат электрофореза в агарозном геле: MW — маркер длин ДНК фрагментов, КЛ — геномная ДНК, выделенная из клеток контрольной линии, ТЛ — геномная ДНК, выделенная из тестовой линии, К-ДНК — контрольное нанесение 100 нг ДНК. Наиболее яркие полосы маркера соответствует 80 нг ДНК, остальные полосы — 40 нг ДНК. Яркость полос выделенной геномной ДНК из контрольной и тестовых линий соответствует яркости контрольного нанесения ДНК и превосходит по яркости полосы маркера по 80 нг. Таким образом, концентрация ДНК на спектрофотометре измерена верно.

\end{enumerate}

\markSection

Примечание: в зависимости от наборов для выделения ДНК выход геномной ДНК может быть различным, поэтому критерии оценки нормировались на каждую площадку по отдельности.

\begin{tabular}{|p{11cm}|p{3cm}|}
    \hline
    \textbf{Описание критерия} & \textbf{Балл} \\
    \hline
    \hline
    Наличие в конце эксперимента 2-х пробирок с объемом растворов ДНК по 25 мкл (наличие ДНК подтверждено форезом) & 2 \\
    \hline
    Концентрация ДНК в контрольной линии (подтверждена форезом):

    Площадка НГУ: 

    >=50нг/мкл 4б, 
    
    25-50 нг/мкл — 3б, 
    
    15-25 нг/мкл — 2б, 
    
    менее 15нг/мкл — 1б, 
    
    менее 10нг/мкл — 0 
    
    Площадка МФТИ: 
    
    >=100нг/мкл 4б, 
    
    50-100 нг/мкл — 3б, 
    
    25-50 нг/мкл — 2б, 
    
    менее 25нг/мкл — 1б, 
    
    менее 10нг/мкл — 0 & 4 \\
    \hline
    Концентрация ДНК в тестовой линии (подтверждена форезом):  
    
    Площадка НГУ: 
    
    >=50нг/мкл 4б, 
    
    25-50 нг/мкл — 3б, 
    
    15-25 нг/мкл — 2б, 
    
    менее 15нг/мкл — 1б, 
    
    менее 10нг/мкл — 0 
    
    Площадка МФТИ: 
    
    >=100нг/мкл 4б, 
    
    50-100 нг/мкл — 3б, 
    
    25-50 нг/мкл — 2б, 
    
    менее 25нг/мкл — 1б, 
    
    менее 10нг/мкл — 0 & 4 \\
    \hline
    Расчет объема 100нг ДНК: по 1б за каждую линию & 2 \\
    \hline
    Расчет объема красителя: по 1б за каждую линию & 2 \\
    \hline
    Результаты фореза: четкая визуализация полос геномной ДНК. Интенсивность соответствует бенду контрольного нанесения 100 нг (3 б), только одна полоса соответствует контрольному нанесению, вторая слабее (2 б), обе полосы очень слабые (1 б) или отсутствуют (0 б) & 3 \\
    \hline
    Команде потребовалась помощь в работе с протоколом & (-2) \\
    \hline
    Команда не уложилась в регламент & (-2) \\
    \hline
    Утеряны пробирки с ДНК, либо ДНК в образце или образцах по результатам фореза отсутствует. Для продолжения работы участникам были выданы пробирки с раствором ДНК.	& штраф 25\% за каждый образец ДНК \\
    \hline
    \hline
    \textbf{Итог} & \textbf{17} \\
    \hline
\end{tabular}