\assignementTitle{Дополнительная задача.
Анализ результатов секвенирования целевого гена в модифицированных клеточных линиях
}{18}{}

\subsection*{Формулировка задания}

Определить характер мутаций, которые произошли в геноме моноклональной модифицированной линии HEK 293 после воздействия системы редактирования CRISPR/Cas9 на ген METTL3 (Homo sapiens methyltransferase like 3).

Для данной клеточной линии была выделена геномная ДНК и получен ПЦР-продукт, соответствующий таргетированному участку. Далее ампликоны были заклонированы в плазмидный вектор. После трансформации и выделения плазмидной ДНК из одиночных колоний были получены плазмиды со вставкой, соответствующей одному либо другому аллелю. Данные плазмиды были так же проанализированы с помощью секвенирования по Сэнгеру.

Вам необходимо провести анализ секвенограмм, полученных для смеси ампликонов, с помощью сервиса TIDE. Далее проанализировать секвенограммы для каждого алелльного варианта с помощью Blast. На основании полученных результатов требуется определить характер мутации и ответить на вопросы в таблице.

\subsection*{Входные данные}

\begin{enumerate}
    \item В папке «ОНТИ» — «Дополнительная задача» — «Сиквенсы» размещены файлы результатов секвенирования исходной клеточной линии «Control\_f» и «Control\_r» и модифицированной линии «Clone\_2-8\_f» и «Clone\_2-8\_r».
    \item В подпапке «Разделенные аллели» находятся результаты сиквенса исходной линии (HEK 293) отдельных аллелей отредактированной линии (Clone\_allel) в формате .ab1 и .fasta.
    \item Ссылка на онлайн инструменты и последовательность протоспейсера размещены в файле «Ссылки и протоспейсер». При работе с Blast необходимо выбрать «Align two or more sequences».
    \putImgWOCaption{15cm}{1}
    \item Помимо указанных инструментов вы можете использовать любые онлайн инструменты, которые сочтете нужными.
\end{enumerate}

\solutionSection

Настройки TIDE для анализа результатов:

\putImgWOCaption{12cm}{2}

Результаты анализа секвенограмм в сервисе TIDE:

\putImgWOCaption{15cm}{3}

Клеточная линия содержит делецию 12 нуклеотидов в одном из аллелей гена METTL3 (примерно в 65.5\% проанализированных молекул ДНК). Информация по второму аллелю недоступна: сервис не отмечает иных мутаций, но и не находит аллель дикого типа (индел = 0).

Анализ секвенограмм для разделенных аллелей с помощью blast, выявляет протяженную мутацию:

\putImgWOCaption{15cm}{4}

По разнице нуклеотидов для верхней последовательности можно определить размер делеции: $227-159-1=67$ п.н.

Поскольку TIDE позволяет анализировать инделы размером до $50$ п.н., данная мутация оказывается “вне поля зрения” инструмента.

\subsection*{Результаты}

\begin{tabular}{|p{6cm}|p{8cm}|}
    \hline
    Какие мутации несут аллели полученной клеточной линии & Делеция 12 п.н., делеция 67 п.н. \\
    Произошел ли полный нокаут гена-мишени? 
    
    Объясните развернуто ответ. & Нет, полный нокаут не произошел, поскольку одна из делеций (-12п.н.) кратна 3, т.е. не приводит к сдвигу рамки считывания и возникновению преждевременного стоп-кодона. \\
    \hline
    Определите последовательность PAM, которая была использована для направления sgРНК & 5’-AGG-3’ \\
    \hline
    Объясните возникшую разницу в результатах анализа и укажите причину, по которой это могло получиться & Одна из делеций (-67п.н.) больше размера индела, который можно задать в параметрах сервиса TIDE, поэтому ее невозможно обнаружить с помощью данного инструмента, но можно найти при анализе секвенограмм разделенных аллелей. \\
    \hline
\end{tabular}

\markSection

\begin{tabular}{|p{11cm}|p{3cm}|}
    \hline
    \textbf{Описание критерия} & \textbf{Балл} \\
    \hline
    Получен ответ по характеру мутаций с указанием размера инсерций и делеций & 3 \\
    \hline
    Получен ответ по характеру мутаций с указанием размера инсерций и делеций & 3 \\
    \hline
    Описали результат — произошел ли полный нокаут? 
    
    без объяснения — 3б, 
    
    с объяснением — 6б, 
    
    неверный ответ — 0б & 6 \\
    \hline
    Правильно определен PAM в формате NGG & 3 \\
    \hline
    Есть объяснение и указание причины расхождения результатов анализа Tide и разделенных аллелей & 6 \\
    \hline
    Команде были выданы параметры настройки tide & (-3) \\
    \hline
    \hline
    \textbf{Итог} & \textbf{18} \\
    \hline
\end{tabular}