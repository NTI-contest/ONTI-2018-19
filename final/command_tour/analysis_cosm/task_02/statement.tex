\section{Задача Б. Оценка точности результатов дешифрирования}

\textbf{\underline{Внимание!} Задача может решаться параллельно с основной задачей и будет оцениваться независимо от неё.}

Чтобы оценить самостоятельно качество вашего результата по картографированию древесных плантаций острова Суматра, необходимо сравнить его с результатами интерпретации набора пробных площадей. Скачайте по данной ссылке (\url{https://drive.google.com/file/d/1ZcUyd-goeIfDIg-2BG_pxPJb2E9AvyfO/}) набор векторных данных, представляющих из себя случайную выборку точек для территории острова Суматра в формате шейп-файла (в zip-архиве). Точки являются маркерами пробных площадей. Пробной площадью являлся пиксель снимка Landsat, в пределы которого попадает данная точка.

В данном случае это – нестратифицированная случайная выборка. Пробные площади расположены в пределах всей зоны работы – округа Оган-Комерин-Илир \linebreak (Kabupaten Ogan Komering Ilir; \url{https://drive.google.com/file/d/1TX8qOilaVFmlx2}\linebreak \url{7_pvGwf0Qr_6Lftr-g/view}) индонезийской провинции Южная Суматра (Sumatera Selatan) – случайным образом.

Вы могли бы и сами создать такой набор данных, расставив случайные точки в пределах границы каждой страты (например, с помощью соответствующего инструмента в QGIS). Однако, мы просим вас воспользоваться именно этим, подготовленном нами набором данных – чтобы мы смогли сравнить ваши результаты с интерпретацией наших экспертов.

Для каждой пробной площади (пикселя Landsat) оцените его принадлежность к тому или иному классу древесных плантаций (или отметьте, что он не является таковыми) по состоянию на момент времени, максимально близкий к сегодняшнему дню.

\underline{Важно!}

Если вы хотите получить реальную оценку качества ваших данных, интерпретируйте пробные площади по доступным космическим снимкам \underline{независимо} от полученных вами границ древесных плантаций! То есть \underline{не пытайтесь} отнести данную пробную площадь к тому или иному классу на основании её попадания/непопадания в пределы выделенных вами в ходе решения основной задачи границ древесных плантаций. \textbf{Совпадение/несовпадение ваших пробных площадей с выделенными вами классами древесных плантаций \underline{НИКАК} не повлияет ни на оценку вашего решения Задачи Б} (ваша интерпретация пробных площадей будет сравниваться с оценкой экспертов, а не с вашей картой), \textbf{ни на оценку вашего решения основной Задачи А} (она будет оцениваться НЕ по ваши пробным площадям).

Смотрите каждый раз на доступные космические снимки и старайтесь определить, чем является каждый такой пиксель. Этим можно (и даже желательно) заниматься \underline{параллельно} с решением основной задачи. В том числе, этим может заниматься и участник(и) команды, не задейстованный(ые) в решении основной задачи.

Результат данной задачи будет оцениваться путём сравнения с проведённой заранее интерпретацией тех же самых пробных площадей нашими экспертами. Максимальный бал за эту задачу будет присваиваться в случае, если отклонение от результатов экспертов не будет превышать 5\%. (Эксперты тоже могут ошибаться, но их ошибка должна заведомо укладываться в данный интервал.)