\solutionSection

Целью задач Б и В было проверить навыки участников по верификации собственных результатов, в данном случае, – карты древесных плантаций, полученной в результате дешифрирования космических снимков.

Задача Б проверяла навыки самостоятельной интерпретации пробных площадей. Она также, в значительной мере, выявляла умение участников понимать, что они видят на космических снимках, и пользоваться различными источниками информации для верного определения, типа растительного покрова, отображённого на снимках.

Для самостоятельной интерпретации всем командам был передан набор векторных данных в точечной топологии, где каждая точка представляла пробную площадь. В условии задачи участникам было разъяснено, что в качестве пробой площади рассматривается отдельный пиксель космических снимков Landsat (30$\times$30 метров), в пределы которого попала данная точка.

Переданный командам набор точек (пробных площадей) являлся подмножеством набора пробных площадей, заранее интерпретированных экспертами. Размер выборки был определён в 500 точек (половину от полного набора пробных площадей), с учётом ограниченного времени, которым располагали участники финала Олимпиады. Результаты интерпретации пробных площадей (класс древесных плантаций или отсутствие плантаций) вносились участниками в соответствующее поле (столбец) атрибутивной таблицы данного точечного набора данных.

\putTwoImg{8cm}{final/command_tour/analysis_cosm/task_02/1}{8cm}{final/command_tour/analysis_cosm/task_02/2}

\begin{center}
    Пробные площади, интерпретированные экспертами (слева) и одной из команд (справа).
\end{center}

Результаты решения задачи сдавались участниками в виде того же набора векторных данных с заполненным соответствующим полем атрибутивной таблицы. Интерпретация пробных площадей участниками сравнивалась с интерпретацией тех же точек экспертами, произведённой заранее. Тот же набор данных (только с добавлением ещё 500 дополнительных точек) использовался и для оценки Задачи А.

Практически сравнение было осуществлено путём пространственного перекрытия точек, сданных участниками с точками, интерпретированными экспертами. Поскольку часть точек была случайно незначительно сдвинута участниками в процессе редактирования набора данных (заполнения атрибутивной таблицы), перекрытие производилось не с самими экспертными точками, а с полигонами, полученными путём построения вокруг исходных точек буфера радиусом 1 метр.

С помощью инструмента «Присоединить атрибуты по местоположению» в QGIS для каждой пробной площади из набора данных участников в атрибутивную таблицу было добавлено значение класса древесных плантаций из атрибутов пробных площадей, интерпретированных экспертами.  В итоге для результатов каждой команды был получен набор векторных данных (точек), для каждой точки которого в атрибутивной таблице имелось два значения: (1) класс древесных плантаций, заранее определённый для соответствующей пробной площади опытными экспертами; и (2) класс древесных плантаций, определённых для соответствующей пробной площади участниками финала. Средствами ГИС подсчитывалось количество точек (пробных площадей) интерпретация которых участниками и экспертами не совпадала.

Если различия в интерпретации пробных площадей участниками и экспертами не превышали 5\% (не совпадали значения классов для 25 или меньше точек из 500), то выставляется высший балл – 20. Если различия превышали 45\% (не совпадали значения классов для 225 точек или более) – выставлялось 0 баллов. В промежутке баллы распределяются обратно пропорционально величине ошибки: 25\% – 10 баллов, 35\% – 5 баллов, 15\% – 15 баллов, 10\% ошибки – 17.5 баллов и т.д.

На практике расчет баллов производился по формуле: $20 \cdot (225-X)/200$, где X – число пробных площадей с несовпадающими значениями классов. Отрицательные значения, полученные при вычислении по данной формуле, приравнивались к 0 баллов. Значения выше 20-ти – уменьшались до 20 баллов.