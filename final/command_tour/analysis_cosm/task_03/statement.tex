\section{Задача В. Расчет матрицы ошибок}

\textbf{\underline{Внимание!} Для решения данной задачи необходимы результаты решения Задач А и Б.}

С помощью инструментов пространственной выборки в QGIS или в другой ГИС-системе, которую вы используете, определите количество пробных площадей, которые совпадают и не совпадают с каждым из ваших классов древесных плантаций, границы которых вы получили в результате решения основной задачи.

Составьте матрицу ошибок (\url{http://gis-lab.info/qa/error-matrix.html}; \linebreak confusion matrix \url{https://en.wikipedia.org/wiki/Confusion_matrix} или error \linebreak matrix; см. также про расчёт матрицы ошибок для пространственных данных здесь (\url{http://gis-lab.info/qa/error-matrix.html}) и здесь (\url{https://habr.com/company/ods/blog/328372/}), однако, в этих статьях речь не идёт о расчете ошибок по пробным площадям). Рассчитайте соответствующие цифры в ней пропорционально доле верно и неверно классифицированных вами пробных площадей каждого класса и доле площади каждого класса от всей территории исследования. Подробнее о таких расчетах можно прочитать, например, в этой статье (\url{http://reddcr.go.cr/sites/default/files/centro-de-documentacion/olofsson_et_al._2014_-_good_practices_for_es}\linebreak \url{timating_area_and_assessing_accuracy_of_land_change.pdf}), к сожалению, только на английском языке.

На основании полученной матрицы ошибок, рассчитайте ошибку производителя для каждого вашего класса, а также для всех классов древесных плантаций вместе. Выразите её в процентах, округлив до десятых долей.

Решение данной задачи будет оцениваться исключительно по правильности расчёта матрицы ошибок на основании \underline{ВАШИХ данных}. Правильность интерпретации вами пробных площадей (Задача Б) и корректность выделения границ древесных плантаций (Задача А) не будут иметь значения для оценки.