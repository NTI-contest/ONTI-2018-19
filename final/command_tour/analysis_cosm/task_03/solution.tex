\solutionSection

Целью задачи было выявить умение участников произвести вычисление ошибки собственных карт по собственным пробным площадям, интерпретированным при решении Задачи Б. Вероятно, 500 пробных площадей является недостаточным количеством для надёжной верификации данных с таким количеством классов и характером распределения в пределах зоны исследования (округа Оган-Комерин-Илир). Именно поэтому для проверки карт, составленных участниками, нами использовался вдвое больший набор из 1000 пробных площадей. Однако, набора из 500 точек достаточно, чтобы проверить умение участников сравнивать пробные площади с границами классов на своих картах и вычислять матрицу ошибок.
Решение данной задачи оценивалось исключительно по правильности расчёта матрицы ошибок и точности на основании \underline{СОБСТВЕННЫХ данных участников}. Правильность интерпретации ими пробных площадей (Задача Б) и корректность выделения границ древесных плантаций (Задача А) не имели значения для оценки.

\putTwoImg{8cm}{final/command_tour/analysis_cosm/task_03/1}{8cm}{final/command_tour/analysis_cosm/task_03/2}

\begin{center}
    Интерпретация пробных площадей одной из команд (слева) и эти точки пробных площадей, наложенные на карту древесных плантаций этой же команды (одни и те же классы окрашены одинаково).
\end{center}

Методика оценки данной задачи практически полностью повторяет последовательность действий при оценки Задачи А. Фактически, единственно отличие состоит в том, что при проверке данной задачи мы использовали набор точечных векторных данных с результатом интерпретации пробных площадей, полученный от самих участников. Он включал в себя 500 точек. При проверке Задачи А мы использовали набор пробных площадей, заранее интерпретированных экспертами (1000 точек). Однако, вся последовательность действий и расчётов – та же самая.

Фактически, при проверке мы повторили те действия, которые должны были сделать участники финала со своими картами и со своим набором точек – пробных площадей. Если бы участники всё сделали правильно, наши расчёты дали бы совершенно одинаковые результаты – с точностью до ошибки округления. Поэтому для данной задачи была принята довольно жёсткая система оценки: если любая из финальных цифр участников отклоняется от результатов, полученных при проверке, более, чем на 0.1, команде присваивается 0 баллов, если отклонение меньше – максимальные 10 баллов. Промежуточных вариантов не было предусмотрено.

К сожалению, в этом году НИ ОДНА команда не смогла справиться с данной задачей. (Хотя в ходе второго этапа Олимпиады, по крайней мере, одна из команд-участниц финала аналогичную задачу решила.) Большинство команд не справилось с расчетом матрицы ошибок: они, в основном, построили матрицу количества совпадающих и несовпадающих значений для классов карты, но не взвесили соответствующие проценты по доле площади каждого класса от всей территории исследования, хотя это было прямо указано в условиях задачи. Единственная команда, проведшая взвешивание по доле площади, ошиблась в расчётах и получила сильно отличающиеся от наших вычислений цифры.