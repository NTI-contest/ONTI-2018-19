\section{Оценка навыков и ведения Лабораторного журнала}

При решении Задачи А каждая команда \underline{должна} была вести лабораторный журнал. По условиям задачи, в лабораторный журнал должны были вноситься следующие записи трёх типов.

\begin{enumerate}
    \item План действий на текущий день.
    
    Вносится ежедневно утром в начале работ по решению Задачи. Вы должны указать:
    \begin{itemize}
        \item план действий на текущий день и последовательность их выполнения;
        \item обоснование (зачем вы планируете предпринять указанные действия);
        \item планируемое распределение ролей: кто из членов команды будет выполнять каждое запланированное действие.
    \end{itemize}
    \item Запись о каждом отдельном действии (операции):
    \begin{itemize}
        \item время начала выполнения операции;
        \item имя члена(ов) команды, производящего(их) операцию;
        \item входные данные (названия файлов), которые сохраняются и передаются жюри;
        \item краткое описание того, что было сделано;
        \item полученный результат;
        \item возникшие проблемы (если были);
        \item выходные данные (названия файлов), которые сохраняются и передаются жюри;
        \item время окончания выполнения операции.
    \end{itemize}
    \item Итоги текущего дня.
    
    Вносится ежедневно вечером, по окончании работ текущего дня. Вы должны указать:
    \begin{itemize}
        \item полученные результаты (итоги дня);
        \item возникшие проблемы (если были);
        \item какие корректировки вносятся в план решения по итогам дня. 
    \end{itemize}
\end{enumerate}

Журнал заполнялся командами онлайн в формате Google Docs по образцу, приведённому в Приложении 1 (см также: \url{https://docs.google.com/document/d/1uAWkWT} \url{rSPvqiM8EVWYTbcw3eY-U8N7-uxcViM-0pfSs/edit?usp=sharing}).

Ежедневно, по окончании работ и заполнению журнала (не позднее 19:30) жюри профиля будет сохранять журнал за прошедший день для последующей оценки ваших действий по решению задачи.

Лабораторный журнал использовался, главным образом, для оценки различных навыков команд. Однако, Лабораторный журнал не был единственным источником информации о навыках участников. Они оценивались также по итогам изучения промежуточных ГИС-данных, которые команды обязаны были ежедневно копировать в определённую папку на сервере во внутренней сети площадки проведения финала.

В частности, оценивались следующие основные навыки, приведённые в таблице ниже (0 – навык отсутствует, 1 – неуверенные навыки, 2 уверенные навыки).

\begin{center}
    \begin{tabular}{|p{14cm}|l|}
        \hline
        Навык & балл \\
        \hline
        Умение подбирать и скачивать космические снимки из общедоступных источников & 0-2 \\
        \hline
        Умение использовать космические снимки с открытых геопорталов без их скачивания & 0-2 \\
        \hline
        Умение искать и находить дополнительную пространственную информацию (не космические снимки), которая может помочь в решении задачи) & 0-2 \\
        \hline
        Умение работать с космическими снимками при подготовке к их классификации (загрузка каналов, склеивание, если необходимо, настройка гистограмм и пр.) & 0-2 \\
        \hline
        Навыки редактирования векторных данных (при рисовании трейнингов, постобработки или в других случаях) & 0-2 \\
        \hline
        Навыки визуального дешифрирования & 0-2 \\
        \hline
        Навыки полуавтоматической классификации космических снимков любым алгоритмом & 0-2 \\
        \hline
        Использование более одного алгоритма / ПО для автоматической обработки (1 балл за каждый новый алгоритм, максимум –24 балла) & 0-2 \\
        \hline
        Навыки постредкатирования результатов дешифрирования (склеивание вместе, корректировка) & 0-2 \\
        \hline
        Навыки протоколирования собственных действий (полнота и аккуратность ведения самого лабораторного журнала) & 0-2 \\
        \hline
    \end{tabular}
\end{center}

Всего можно было набрать до 20 баллов за базовые навыки и знания, продемонстрированные при решении Задачи А.

Пример заполнения Лабораторного журнала одной из команд приведён в Приложении 2.

\section*{Приложение 1. Образец ведения Лабораторного журнала }

\begin{center}
    \textbf{Олимпиада НТИ,\\Профиль “Анализ космических снимков и геопространственных данных”}
\end{center}

\begin{center}
    \textbf{Лабораторный журнал}
\end{center}

\noindent\textbf{\underline{Команда:}} <НАЗВАНИЕ ВАШЕЙ КОМАНДЫ>\\

\noindent\textbf{День 1 (27 марта 2019 г.)}\\

\noindent\underline{Задачи на день: список операций (в скобках - исполнители):}\\
1.\underline{\hspace{6cm}}.(\underline{\hspace{6cm}})\\
2.\underline{\hspace{6cm}}.(\underline{\hspace{6cm}})\\
3.\underline{\hspace{6cm}}.(\underline{\hspace{6cm}})\\

\noindent\underline{Обоснование планируемых действий:}\\
\underline{\hspace{9cm}}.\\
\underline{\hspace{9cm}}.\\
\underline{\hspace{9cm}}.\\

\textbf{Операции (действия)}

\begin{center}
    \begin{longtable}{|l|l|}
        \hline
        № & Описание операции \\
        \hline
        1. & ФИО члена(ов) команды: \underline{\hspace{6cm}} \\
           & Начало выполнения операции: \underline{\hspace{5cm}} \\
           & Входные данные (названия файлов, ссылки): \underline{\hspace{6cm}}\\
           & \underline{\hspace{14cm}}\\
           & Ход работы: \\
           & - что было сделано; \\
           & - полученный результат; \\
           & - возникшие проблемы (если были).\\
           & . \\
           & . \\
           & . \\
           & Окончение выполнения операции: \underline {\hspace{6cm}}\\
           & Входные данные (названия файлов, ссылки): \underline{\hspace{6cm}} \\
           & \underline{\hspace{14cm}}\\
        \hline
        2. & ... \\
        \hline
        3. & ... \\
        \hline
        ... & ... \\
        \hline
    \end{longtable}
\end{center}

\noindent\underline{Итоги дня (27 марта 2019 г.):}\\
\begin{itemize}
    \item полученные результаты (итоги дня);
    \item возникшие проблемы (если были);
    \item какие корректировки вносятся в план решения по итогам дня.
\end{itemize}

\noindent\underline{\hspace{9cm}}.\\
\underline{\hspace{9cm}}.\\
\underline{\hspace{9cm}}.\\
\underline{\hspace{9cm}}.\\
\underline{\hspace{9cm}}.\\
\underline{\hspace{9cm}}.
\clearpage
\section*{Приложение 2. Пример ведения Лабораторного журнала}

\begin{center}
    \textbf{Олимпиада НТИ,\\Профиль “Анализ космических снимков и геопространственных данных”}
\end{center}

\begin{center}
    \textbf{Лабораторный журнал}
\end{center}

\noindent\textbf{\underline{Команда:}} <НАЗВАНИЕ ВАШЕЙ КОМАНДЫ>\\

\noindent\textbf{День 1 (27 марта 2019 г.)}\\

\noindent\underline{Задачи на день: список операций (в скобках - исполнители):}

\begin{enumerate}
    \item Найти и обработать данные, которые понадобятся для решения задачи А (ФИО, ФИО, ФИО)
    \item Обрезать все снимки и векторные данные по границам нужного района (ФИО)
    \item Разобраться, как выглядят те или иные классифицируемые зоны, используя данные прошлых лет и ресурсы GoogleEarth(вся команда)
    \item Выделить с помощью DTClassifier масличные пальмы и отредактировать полученные данные вручную (ФИО)
    \item Выделить с помощью DTClassifier древесные плантации и отредактировать полученные данные вручную (ФИО)
    \item Выделить с помощью DTClassifier резиновые деревья и отредактировать полученные данные вручную (ФИО)
    \item Начало работы над задачей Б (ФИО)  
\end{enumerate}

\noindent\underline{Обоснование планируемых действий:}\\
Данные операции нам понадобятся для дальнейшего решения задач. Операции 4,5,6 нужны для сборки собственной карты распределения территорий под плантации в будущем.

\textbf{Операции (действия)}

\begin{center}
    \begin{longtable}{|l|p{14.5cm}|}
        \hline
        № & Описание операции \\
        \hline
        1. & ФИО члена(ов) команды: ФИО, ФИО, ФИО.\\
           & Начало выполнения операции: 27.03.19 \\
           & Входные данные (названия файлов, ссылки): Tree\_Plantations.zip с сайта GlobalForestWatch.com \\
           & Снимки из папки D:/ONTI/Landsat\_resources\\
           & \underline{Ход работы:} \\
           & Мы открыли и склеили все каналы у каждого снимка(кроме панхроматического) для получения цветного изображения. \\
           & Окончание выполнения операции: 27.03.19 \\
        \hline
        2. & ФИО члена(ов) команды: ФИО \\
           & Начало выполнения операции: 27.03.19 \\
           & Входные данные (названия файлов, ссылки): снимки из папки Landsat\_source, границы округа, скачанные по ссылке из задания. \\
           & \underline{Ход работы:} \\
           & С помощью функции Растр/Прочее/Обрезка обрезала снимки по контуру OKI. Скачала и загрузила в проект файлы из архива Tree\_plantations, после чего с помощью функции Вектор/Обрезка обрезала данные по контуру OKI. \\
           & Окончание выполнения операции: 27.03.19 \\
           & Выходные данные (названия файлов, ссылки): result\_obrezki.shp \\
        \hline
        3. & ФИО члена(ов) команды: вся команда \\
           & Начало выполнения операции: 27.03.2019 \\
           & Входные данные (названия файлов, ссылки): ресурсы с интернет порталов Google Earth, Earth Explorer, Global Forest Watch \\
           & \underline{Ход работы:} \\
           & Просмотрели нужные территории на порталах и определили как выглядит та или иная классифицируемая территория. \\
           & Окончание выполнения операции: 27.03.2019 \\
        \hline
        4. & ФИО члена(ов) команды: ФИО \\
           & Начало выполнения операции: 27.03.19 \\
           & Входные данные (названия файлов, ссылки): Снимки из папки D:/ONTI/Landsat\_resources, Tree\_Plantations.zip с сайта GlobalForestWatch.com \\
           & \underline{Ход работы:} \\
           & Создала отдельный векторный слой для полигонов находящихся на территории масличных пальм и заносили территории туда. Просматривая изображения на портале GoogleEarth по данной территории, определяла те плантации, которые относятся к категории Oil Palm. С помощью классификатора определила зоны масличных пальм и доработала их вручную не для всей территории. \\
           & Окончание выполнения операции: - \\
           & Выходные данные (названия файлов, ссылки): OIL\_PALM.shp \\
        \hline
        5. & ФИО члена(ов) команды: ФИО \\
           & Начало выполнения операции: 27.03.19 \\
           & Входные данные (названия файлов, ссылки): обрезанные снимки и границы плантаций \\
           & \underline{Ход работы:} \\
           & Просматривая изображения на портале GoogleEarth по данной территории, определяла те плантации, которые относятся к категории Wood Fiber(эвкалипты и акации). Создала векторный слой WoodFiber-yes, где отметила 10 полигонов все пиксели которых относятся к данной категории. Создала векторный слой WoodFiber-no, где создала 10 полигонов, не относящихся к плантациям эвкалипта и акации. Произвела первую классификацию по слоям WoodFiber-no и WoodFiber-yes. Проанализировала полученные результаты. Работа ещё не закончена, поэтому поставила задачи на будущее.\\
           & Окончание выполнения операции: - \\
           & Выходные данные (названия файлов, ссылки): WoodFiber.shp \\
        \hline
        6. & ФИО члена(ов) команды: ФИО \\
           & Начало выполнения операции: 27.03.19 \\
           & Входные данные (названия файлов, ссылки): treeplantations.zip, снимки из папки D:/ONTI/Landsat resources, файл с границами: OKI. \\
           & \underline{Ход работы:} \\
           & Я открыла и склеила все каналы у каждого снимка(кроме панхроматического) для получения цветного снимка.Далее загрузила шейп treeplantations и шейп OKI, чтобы ориентироваться, где находятся плантации и границы округа соответственно. Потом создала пустой шейп-файл и ориентируясь на данные расположения плантаций и учитывая границы округа, выделила нные полигоны, которые в будущем должны были мне помочь с классификация каучуковых деревьев(“резиновых” деревьев), а именно Гевей. Далее я воспользовалась функцией классификатора. В итоге получился очень примерный слой, требующий редактирования. \\
           & Окончание выполнения операции: - \\
           & Выходные данные (названия файлов, ссылки): rubber\_trees.shp \\
        \hline
        7. & ФИО члена(ов) команды: ФИО \\
           & Начало выполнения операции: 27.03.19 \\
           & Входные данные (названия файлов, ссылки):Этот компьютер/DATA(D:)/ONTI/Landsat\_source/points, снимки из папки Landsat\_source. \\
           & \underline{Ход работы:} \\
           & Я открыла и склеила все каналы у каждого снимка(кроме панхроматического) для получения цветного изображения. \\
           & Из условия задачи я скачала файл с точками и начала вносить изменения в таблицу атрибутов. \\
           & Окончание выполнения операции: -\\
           & Выходные данные (названия файлов, ссылки):  Этот компьютер/DATA(D:)/ONTI/Landsat\_source/projects/filethatineed \\
        \hline
    \end{longtable}
\end{center}
\underline{Итоги дня (27 марта 2019 г.):}
\begin{itemize}
    \item Малое продвижение в классификации точек для задачи Б
    \item Получили примерный результат обработки классификатора для масленичных пальм, древесного волокна, каучуконосных деревьев
\end{itemize}

\noindent\textbf{День 2 (28 марта 2019 г.)}\\

\noindent\underline{Задачи на день: список операций (в скобках - исполнители):}

\begin{enumerate}
    \item Закончить выделение плантаций для масличных пальм, древесного волокна, фруктовых д и каучуконосных деревьев (ФИО, ФИО и ФИО соответственно)
    \item Продвинуться в решении задачи Б: обработать около 250 точек (ФИО)
    \item Объединить все получившиеся полигоны и заполнить таблицу атрибутов (классификация и исходные данные) для полигонов плантаций масличных пальм, древесного волокна, фруктовых и каучуконосных деревьев (ФИО, ФИО и ФИО соответственно)
\end{enumerate}

\noindent\underline{Обоснование планируемых действий:}\\
Разделение работы для качественного и быстрого результата и параллельное заполнение атрибутивных таблиц, во избежание неорганизованности данных.

\textbf{Операции (действия)}

\begin{center}
    \begin{longtable}{|l|p{14.5cm}|}
        \hline
        № & Описание операции \\
        \hline
        1. & ФИО члена(ов) команды: ФИО, ФИО, ФИО \\
           & Начало выполнения операции: 27.03.19 \\
           & Входные данные (названия файлов, ссылки): файлы вчерашнего дня \\
           & \underline{Ход работы:} \\
           & Примечание: так как были взяты снимки 2018 года, то только засеянные территории были также отнесены к плантациям, так как к сегодняшнему дню наверняка выросли. Поняли мы это благодаря анализу снимков 2018 и 2019 годом: растительность выросла. \\
           & Примечание 2: при обработке находящихся близко к друг другу плантаций использовали параметр прилипания, для уменьшения неточностей. \\
           & Доработала слои WoodFiber-yes и WoodFiber-no, после чего сделала классификации обрезанных снимков. Для каждого снимка отдельные слои WoodFiber-yes и WoodFiber-no(wf-no-1, wf-yes-1, wf-no-2, wf-yes-2 и т.д). Результаты классификаций по каждому снимку(woodfiber, woodf-2, woodf-3) сгладила, а после этого векторизовала. Вручную исправила ошибки и доделала слой WoodFiber-all.(Савельева) \\
           & Опираясь на данные Tree\_Plantations.zip с сайта GlobalForestWatch.com и Google Earth дорабатала полигоны с территориями масличных пальм и в новом shp-файле проделала то же самое с плантациями фруктовых деревьев.( ФИО)\\
           & Доработала результаты классификации каучуконосных деревьев вручную, получила файл Rubber. (ФИО) \\
           & Окончание выполнения операции: 28.03.2019 \\
           & Входные данные (названия файлов, ссылки): Z:/PeaceYes!/Dashas work/nti/WoodFiber-all.rar, Z:/PeaceYes!/Sufiyas work/oil\_palm\_version1/OIL\_PALM.shp, Z:/PeaceYes!/Sufiyas work/Fruit trees\_version1/FRUIT\_TREES.shp \\
        \hline
        2. & ФИО члена(ов) команды: ФИО \\
           & Начало выполнения операции: 27.03.19 \\
           & Входные данные (названия файлов, ссылки): Z:/PeaceYes!I/Yusup work \\
           & \underline{Ход работы:} \\
           & Обработала около 400 точек. Создала отдельный класс для всех не плантаций (вода, здания и прочее). \\
           & Окончание выполнения операции: - \\
           & Выходные данные (названия файлов, ссылки):  Z:/PeaceYes!I/Yusup work \\
        \hline
        3. & ФИО члена(ов) команды: ФИО, ФИО, ФИО\\
           & Начало выполнения операции: 27.03.19 \\
           & Входные данные (названия файлов, ссылки): файлы вчерашнего дня, OIL\_PALM, FRUITS, WoodFiber, Rubber, полученные во время выполнения первого пункта. \\
           & \underline{Ход работы:} \\
           & С помощью функции Объединение соединили слои OIL\_PALM, WoodFiber, FRUIT\_TREES получили слой wf+fr+op. И заполнили таблицу атрибутов: поле “image” для названия исходного файла и поле “spec\_simp” для класса плантаций по выращиваемым там видам деревьев. \\
           & Окончание выполнения операции: 28.03.19\\
           & Входные данные (названия файлов, ссылки): wf+fr+op\\
        \hline
    \end{longtable}
\end{center}
\underline{Итоги дня (28 марта 2019 г.):}\\
Мы получили единый слой с плантациями масличных пальм, древесного волокна и фруктовых деревьев, заполнили таблицу атрибутов, а также обработали около 400 точек для задачи Б. 

\noindent\textbf{День 3 (29 марта 2019 г.)}\\

\noindent\underline{Задачи на день: список операций (в скобках - исполнители):}

\begin{enumerate}
   \item Выделить территории прочих, неопределённых плантаций и свежих расчисток (ФИО и ФИО)
   \item Доделать 100 точек для задачи Б и переделать классы под единый стиль (ФИО)
   \item Соединить все векторы вместе (ФИО, ФИО и ФИО)
\end{enumerate}

\noindent\underline{Обоснование планируемых действий:}\\
Закончить выполнение задач А и Б, для того, чтобы приступить к решению задачи В.

\textbf{Операции (действия)}

\begin{center}
    \begin{longtable}{|l|p{14.5cm}|}
        \hline
        № & Описание операции \\
        \hline
        1. & ФИО члена(ов) команды: ФИО, ФИО.\\
           & Начало выполнения операции: 29.03.2019 \\
           & Входные данные (названия файлов, ссылки): wf+fr+op \\
           & \underline{Ход работы:} \\
           & Мы выделили классы ‘Recently cleared’ и ‘Unknown’. \\
           & Окончание выполнения операции: 29.03.2019 \\
           & Входные данные (названия файлов, ссылки): cleared.shp, UNKNOWN\_OBREZKA\_TABL\_ATR.shp, UNKNOWN\_2\_ATR\_TABL.shp \\
         \hline
         2. & ФИО члена(ов) команды: ФИО \\
            & Начало выполнения операции: 27.03.19 \\
            & Входные данные (названия файлов, ссылки): Z:/PeaceYes!I/Autists\_work/points \\
            & \underline{Ход работы:} \\
            & Обработала все точки для задачи Б и переделала классы под единый стиль. \\
            & Окончание выполнения операции: 29.03.19 \\
            & Входные данные (названия файлов, ссылки): Z:/PeaceYes!I/Autists\_work/points \\
         \hline
         3. & ФИО члена(ов) команды: ФИО, ФИО, ФИО \\
            & Начало выполнения операции: 29.03.2019 \\
            & Входные данные (названия файлов, ссылки): wf+fr+op.shp, Rubber.shp, cleared.shp, UNKNOWN\_OBREZKA\_TABL\_ATR.shp, UNKNOWN\_2\_ATR\_TABL.shp \\
            & \underline{Ход работы:} \\
            & Подкорректировала полигоны каучуковых плантаций, заполнила таблицу атрибутов, получила итоговый файл Rubber (ФИО) \\
            & При помощи функции Объединение мы соединили слои  wf+fr+op, Rubber и Recently cleared, Unknown, полученные во время выполнения первого пункта. Получили слой ALL\_OBR. Начали вручную корректировать этот слой. \\
            & Окончание выполнения операции: 29.03.2019 \\
            & Входные данные (названия файлов, ссылки): ALL\_OBR.shp \\
         \hline
   \end{longtable}
\end{center}
\underline{Итоги дня (29 марта 2019 г.):}\\

\begin{itemize}
   \item Мы получили единый итоговый слой с плантациями масличных пальм, древесного волокна, фруктовых и каучуконосных деревьев, расчисток и неизвестных плантаций;
   \item Завершили решение задачи Б.
\end{itemize}

\noindentВо время объединения векторных слоев столкнулись с проблемой неправильной геометрией, но научились ее быстро и легко исправлять с помощью дополнительных модулей.
   
\noindent\textbf{День 4 (30 марта 2019 г.)}\\

\noindent\underline{Задачи на день: список операций (в скобках - исполнители):}
   
\begin{enumerate}
   \item Подкорректировать все полигоны, заполнить таблицу атрибутов для каждого элемента и отправить задачу А (ФИО и ФИО)
   \item Сделать и отправить задачу В (вся команда)
\end{enumerate}

\noindent\underline{Обоснование планируемых действий:}\\
Так как при объединении предыдущих слоёв мы сталкивались с неорганизованностью таблицы атрибутов,  мы сначала собрали полигоны со всеми плантациями вместе, а потом уже заполняли её.

\textbf{Операции (действия)}

\begin{center}
    \begin{longtable}{|l|p{14.5cm}|}
        \hline
        № & Описание операции \\
        \hline
        1. & ФИО члена(ов) команды: ФИО, ФИО \\
           & Начало выполнения операции: 30.03.2019 \\
           & Входные данные (названия файлов, ссылки): ALL\_OBR.shp \\
           & \underline{Ход работы:} \\
           & Так как при многочисленных объединениях и “разностях” у нас возникли проблемы с взаимным расположением полигонов, мы исправляли геометри как вручную, так и при помощи дополнительных модулей проверки геометрии. После, последовательно выделяя элементы, мы заполнили таблицу атрибутов: “image” для снимка, по которому мы ориентировались при обводке плантаций, “spec\_simp” для “словесного” квалифицирования классов и “id” для числового (он же поможет нам при решении задачи В). Мы запаковали всё в архив и получили PeaceYes\_A.rar \\
           & Окончание выполнения операции: 30.03.2019 \\
           & Входные данные (названия файлов, ссылки): PeaceYes\_A.rar \\
         \hline
         2. & ФИО члена(ов) команды: ФИО, ФИО, ФИО, ФИО \\
            & Начало выполнения операции: 30.03.2019 \\
            & Входные данные (названия файлов, ссылки): PeaceYes\_A.rar, points.zip \\
            & \underline{Ход работы:} \\
            & Так как мы столкнулись с проблемой при решении задачи В в NextGis QGIS, мы решали её в ArcGIS Pro. Перевели векторный слой из задачи А в растровый, после чего с помощью функции Извлечь значения в точки получили исходные данные для матрицы ошибок. \\
            & Составили матрицу ошибок в Excel. \\
            & Окончание выполнения операции: 30.03.2019 \\
            & Входные данные (названия файлов, ссылки): PeaceYes\_B.xlsx и Kiki.xlsx \\
         \hline
   \end{longtable}
\end{center}
\underline{Итоги дня (30 марта 2019 г.):}\\
Решены задачи А и В. Научились составлять матрицу ошибок.