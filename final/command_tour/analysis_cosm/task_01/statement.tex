\section{Задача А. Основная задача финала}

\subsubsection*{Преамбула}

По второму туру вы уже знакомы с набором векторных пространственных данных по древесным плантациям (Tree Plantations), содержащий границы разных типов древесных плантаций для нескольких тропических стран. Подробнее про этот проект картографирования древесных тропических плантаций можно прочитать здесь (\url{http://www.wri.org/sites/default/files/Mapping_Tree_Plantations_with_Multi}\linebreak \url{spectral_Imagery_-_Preliminary_Results_for_Seven_Tropical_Countries.pdf}). \linebreak Скачайте с портала Всемирной лесной вахты (Global Forest Watch; \url{https://www.globalforestwatch.org/}) этот набор данных на территорию острова Суматра (Индонезия).

Данный набор пространственных данных показывает границы разных типов древесных плантаций по состоянию на 2013-2014 гг. Однако, ситуация с плантациями быстро меняется: это очень динамично развивающийся бизнес. На сегодяшний день размещённые на портале Global Forest Watch данные уже устарели. Надо также иметь в виду, что они, как любые данные, содержат ошибки. Согласно независимой оценке, произведённой сотрудниками лаборатории GLAD (Global Land Analysis \& Discovery; \url{https://glad.umd.edu/}) Географического факультета Университета Мэриленда (США), ошибка пропуска (ошибка второго рода, omission error, false negative) данного набора данных для стран Юго-Восточной Азии (Индонезия, Малайзия, Камбоджа) составила 8\% $\pm$2 для всех типов плантаций вместе, за исключением класса свежих расчисток, и 12\% $\pm$3\% для всех классов вместе, включая свежие расчистки. Кроме того, на момент реализации данного проекта не были доступны, например, снимки со спутников Sentinel-2 (\url{https://sentinel.esa.int/web/sentinel/missions/sentinel-2}) Европейского космического агентства, которые имеют более высокое пространственное разрешение, чем снимки со спутников серии Landsat, преимущественно использовавшиеся в этой работе.

\subsubsection*{Суть задания}

Ваша задача состоит в том, чтобы составить \textbf{новую, обновлённую векторную карту древесных плантаций (актуализированный набор пространственных данных) на часть территории острова Суматра, где произошли одни из самых существенных изменений, – округ Оган-Комерин-Илир (Kabupaten Ogan Komering Ilir) индонезийской провинции Южная Суматра (Sumatera Selatan)} на основе космических снимков, доступных из открытых источников. Границы округа в виде набора векторных данных ГИС в формате шайп-файла можно скачать здесь (\url{https://drive.google.com/file/d/1TX8qOilaVFmlx27_pvGwf0Qr_6Lftr-g/view?usp=sharing}).

Для каждого участка плантаций внесите в атрибутивную таблицу вашего набора данных следующую информацию.
\begin{itemize}
    \item Идентификатор сцены космического снимка (название исходного файла), по которому проведён окончательный вариант границ данного участка плантации. В наборе данных с портала Global Forest Watch аналогичное поле называется "image".
    \item Класс плантаций по выращиваемым там видам деревьев. В наборе данных с портала Global Forest Watch аналогичное поле называется "spec\_simp ". В отличие от оригинальных данных, мы предлагаем вам немного упрощённый набор классов. Выделите, по крайней мере, следующие классы древесных плантаций:
    \begin{enumerate}
        \item плантации масличной пальмы ("Oil palm");
        \item плантации быстрорастущих деревьев, выращиваемых для переработки на целлюлозу, – эвкалипты и акации ("Wood fiber");
        \item плантации каучуконосов – гевея, «резиновое дерево» ("Rubber"{}, \linebreak "Rubber tree");
        \item плантации фруктовых деревьев, в том числе, с примесью других типов плантаций и посадок однолетних культур ("Fruit"{}, "Fruit trees");
        \item прочие типы древесных плантаций, в том числе, не определённые вами типы ("Other"{}, "Unknown");
        \item свежие расчистки под плантации, в том числе молодые, ещё не сомкнувшиеся плантации, не отличимые от расчисток ("Recently cleared"{}, "Clearing / Very young plantation").
    \end{enumerate}
    Участки плантаций, где, кроме основной породы, присутствуют примеси других видов деревьев, отнесите к тому или иному классу по преобладающей (занимающей более половины площади) породы деревьев. (В исходном наборе данных такие участки имеют добавление "$\dots$ Mix" в названии класса.)
\end{itemize}

Ваша карта должна отражать границы различных видов древесных плантаций на наиболее поздний возможный момент времени (в идеале – на сегодня). Но, поскольку остров Суматра – одно из самых облачных мест на Земле, вам вряд ли удастся собрать полностью безоблачное покрытие космических снимков, снятых за один день или даже неделю (а, скорее всего, и за один месяц). Поэтому неизбежно придётся пользоваться снимками за более широкий временной интервал. Однако, постарайтесь, чтобы окончательные границы были проведены по снимкам, отражающим состояние местности на дату, максимально приближенную к сегодняшнему дню.

\subsubsection*{Исходные данные и инструменты}

Вы можете пользоваться \underline{любыми} доступными источниками космических снимков и других пространственных данных, \underline{любыми} программными средствами и инструментами и \underline{любыми} методами обработки снимков и картографирования.

В том числе, в качестве источника космических снимков и другой информации вы можете использовать:
\begin{itemize}
    \item любые открытые геопорталы, например, такие как  Яндекс Карты (\url{https://maps.yandex.ru/}), Google Карты (\url{https://www.google.com/maps}), Карты Bing (\url{https://www.bing.com/maps}), Космоснимки.RU (\url{http://kosmosnimki.ru/}), WorldView (\url{https://worldview.earthdata.nasa.gov/}), LandLook \linebreak Viewer (\url{https://landlook.usgs.gov/viewer.html}), Геопортал Роскосмоса \linebreak LandLook Viewer (\url{https://gptl.ru}), портал Всемирной лесной вахты (Global Forest Watch; \url{https://www.globalforestwatch.org/}, портал Глобальные изменения лесного покрова (\url{http://earthenginepartners.appspot.com/scien}\linebreak \url{ce-2013-global-forest}) и др., а также программу Google Планета Земля Pro (\url{https://www.google.com/intl/ru/earth/desktop/});
    \item модули QGIS (и аналогичные плагины других ГИС) QuickMapServices (\url{https://qms.nextgis.com/}) и/или OpenLayers Plugin (\url{https://plugins.qgis.org/plugins/openlayers_plugin/}), которые позволяют использовать космические снимки и другие данные из внешних источников в качестве подложек (слоев) в вашей настольной ГИС;
    \item космические снимки, скачанные с любого из открытых источников, включая порталы EarthExplorer (\url{https://earthexplorer.usgs.gov/}) и GloVis (\url{https://glovis.usgs.gov/}) Геологической службы США (US Geological Survey), Copernicus Open Access Hub (\url{https://scihub.copernicus.eu/dhus/#/home}) Европейского космического агенства, Sentinel Hub (\url{https://apps.sentinel-hub.com/eo-browser/}) и др.
    \item безоблачные композиты снимков Landsat за 2000 и 2018 годы (а также карты лесного покрова и его изменений) с портала Глобальные изменения лесного покрова лаборатории GLAD (Global Land Analysis \& Discovery) Географического факультета Университета Мэриленда (США), скачать которые можно здесь (\url{http://earthenginepartners.appspot.com/science-2013-global-fo} \linebreak \url{rest/download_v1.6.html}). (Помните только, что это не единовременные снимки, а синтез только четырёх каналов множества фрагментов безоблачных снимков за соответствующий год с двух разных спутников, обработанных таким образом, чтобы компенсировать всегда имеющиеся различия между ними, что приводит иногда к разнообразным побочным эффектам.)
\end{itemize}

Помните также, что скачивание снимков может занимать \underline{значительное время}, в том числе, и в связи с ограничениями на сервере-источнике данных. Планируйте скачивание снимков и начинайте его как можно раньше, параллельно с другими операциями. Чтобы вы могли начать работать сразу, мы заранее скачали для вас несколько относительно малооблачных снимков со спутника Landsat 8 на территорию Южной Суматры снятых за последние 8 месяцев. Однако, вам совсем не обязательно ограничиваться только данными сценами для решения настоящей задачи!

\subsubsection*{Какие методы обработки космических снимков использовать?}

Для обработки космических снимков вы можете, в том числе:

Абсолютно \underline{любые}.

\begin{itemize}
    \item воспользоваться модулем DTclassifier для QGIS, подписку на который мы вам оплачиваем на период финала;
    \item использовать другие специализированные модули для QGIS, например, Semi-Automatic Classification Plugin (\url{https://fromgistors.blogspot.com/p/semi-automatic-classification-plugin.html});
    \item использовать иное свободно доступное программное обеспечение для обработки космических снимков и пространственных данных (например, GRASS GIS (\url{https://grass.osgeo.org/}) или MultiSpec (\url{https://engineering.purdue.edu/~biehl/MultiSpec/}));
    \item написать собственную программу обработки космических снимков, реализующую тот или иной алгоритм их классификации (популярные алгоритмы классификации космических снимков, такие как, например, деревья решений (\url{https://basegroup.ru/community/articles/description}) или нейросети хорошо описаны, и вы легко найдёте в сети примеры их реализации в разных программных средах);
    \item наконец, провести обновлённые границы древесных плантаций вручную (полностью или частично, в том числе, улучшив результаты автоматической обработки).
\end{itemize}

Помните только, что конечный результат работы любого автоматического алгоритма обработки космических снимков в очень большой степени зависит от обучающей выборки, а также от подбора и предварительной подготовки космических снимков, которые он будет обрабатывать! В большинстве случаев, это более важно, чем выбор самого алгоритма обработки.

\subsubsection*{Оценка результатов дешифрирования космических снимков}

Ваши результаты будут оцениваться путём сравнения со случайным набором пробных площадей, интерпретированных нашими экспертами. Лучшим будет признан результат с минимальными значениями ошибок обоих типов – ложноположительных результатов (ошибка первого рода, commission error, false positive) и ошибок пропуска (ошибка второго рода, omission error, false negative).

\subsubsection*{Ведение лабораторного журнала}

При решении Задачи А каждая команда \underline{должна} вести лабораторный журнал. Именно по вашим записям в лабораторном журнале вам будет начислено до 20\% вашей общей оценки за ход решения Задачи А.

\underline{Правила заполнения лабораторного журнала.}

В лабораторный журнал вносятся три типа записей.
\begin{enumerate}
    \item План действий на текущий день.
    Вносится ежедневно утром в начале работ по решению Задачи. Вы должны указать:
    \begin{itemize}
        \item план действий на текущий день и последовательность их выполнения;
        \item обоснование (зачем вы планируете предпринять указанные действия);
        \item планируемое распределение ролей: кто из членов команды будет выполнять каждое запланированное действие.
    \end{itemize}
    \item Запись о каждом отдельном действии (операции):
    \begin{itemize}
        \item время начала выполнения операции;
        \item имя члена(ов) команды, производящего(их) операцию;
        \item входные данные (названия файлов), которые сохраняются и передаются жюри;
        \item краткое описание того, что было сделано;
        \item полученный результат;
        \item возникшие проблемы (если были);
        \item выходные данные (названия файлов), которые сохраняются и передаются жюри;
        \item время окончания выполнения операции.
    \end{itemize}
    \item Итоги текущего дня.
    Вносится ежедневно вечером, по окончании работ текущего дня. Вы должны указать:
    \begin{itemize}
        \item полученные результаты (итоги дня);
        \item возникшие проблемы (если были);
        \item какие корректировки вносятся в план решения по итогам дня.
    \end{itemize}
\end{enumerate}

Журнал ведётся онлайн в формате Google Docs по следующему образцу (\url{https://docs.google.com/document/d/1uAWkWTrSPvqiM8EVWYTbcw3eY-U8N7-uxcViM-0pfSs/edit}). (Не редактируйте образец! Мы выдадим каждой команде ссылку на собственную копию лабораторного журнала.)

Ежедневно, по окончании работ и заполнению журнала (не позднее 19:30) жюри профиля будет сохранять журнал за прошедший день для последующей оценки ваших действий по решению задачи.