\solutionSection

Задача участников состояла в том, чтобы по космическим снимкам составить обновлённую векторную карту древесных плантаций на часть территории острова Суматра – округ Оган-Комерин-Илир (Kabupaten Ogan Komering Ilir) индонезийской провинции Южная Суматра (Sumatera Selatan).

Плантации разных типов занимают значительную часть территории этой провинции. При этом их развитие происходит достаточно быстро. С 2014 года, на который имеются пространственные данные по границам разных типов плантаций для Индонезии, на территории округа произошли заметные изменения. Участникам был доступен набор пространственных данных (карта) границ плантаций по состоянию на 2014 год – они использовали его в качестве исходных данных, а также в качестве источника информации о том, как выглядят те или иные типы плантаций на космических снимках. Участникам также были предоставлены несколько космических снимков со спутника Landsat 8 за конец 2018 – начало 2019 года. У них была возможность подобрать и скачать дополнительно любые снимки и другие пространственные данные из открытых источников.

\putTwoImg{8cm}{final/command_tour/analysis_cosm/task_01/1}{8cm}{final/command_tour/analysis_cosm/task_01//2}

\begin{center}
    Композитное изображение из снимков Landsat за 2018 год (слева) и наложенные на него границы различных типов древесных плантаций по состоянию на 2014 год (источник: \url{https://www.globalforestwatch.org/}) (справа).
\end{center}

\putImgWOCaption{12cm}{final/command_tour/analysis_cosm/task_01/3}

\begin{center}
    Композитное изображение, собранное из многих космических снимков Landsat за 2018 год. Источник: \url{http://earthenginepartners.appspot.com/science-2013-global-forest} граница округа Оган-Комерин-Илир выделена красным цветом.
\end{center}

Результаты работы команд участников по Задаче А были представлены в виде набора векторных данных с выделением указанных в условиях задачи классов древесных плантаций.

\putTwoImg{8cm}{final/command_tour/analysis_cosm/task_01/4}{8cm}{final/command_tour/analysis_cosm/task_01/5}

\begin{center}
    Исходная карта древесных плантаций округа Оган-Комерин-Илир на 2014 год (слева) и обновлённая карта на 2018/19 гг., подготовленная одной из команд (справа). Одинаковые классы плантаций обозначены одинаковыми цветами.
\end{center}

Как и в любой задаче на дешифрирование космических снимков (и вообще на составление карт) данная задача не имеет единственного «правильного» решения. Любая карта обладает определённой степенью достоверности и содержит определённую ошибку. Содержит её и исходная карта. В условиях задачи прямо указывается, что, согласно независимой оценке, ошибка пропуска (ошибка второго рода, omission error, false negative) исходных данных составила 8\%$\pm$2\% для всех типов плантаций вместе, за исключением класса свежих расчисток. (И составила 12\%$\pm$3\% для всех классов вместе, включая свежие расчистки.)

Оценка точности карт (полученных при дешифрировании космических снимков наборов векторных данных) производится путем сравнения с тем или иным набором данных, который принимается за надёжный. Распространённым методом является сравнение со случайным или регулярным набором пробных площадей, которые интерпретируются более тщательно, по космическим снимкам лучшего разрешения, более опытными экспертами, или даже проверятся в ходе полевых исследований.

Подробно данные вопросы рассмотрены, например, в следующих научных публикациях:
\begin{itemize}
    \item OlofssonP., Foody G.M., Herold, M. StehmanS.V., Woodcock C.E., WulderM.A. Good practices for estimating area and assessing accuracy of land change. Remote Sensing of Environment 148, 42-57 (2014)
    \item Stehman, S. V. Estimating area and map accuracy for stratified random sampling when the strata are different from the map classes. International Journal of Remote Sensing35.13 (2014)
\end{itemize}

В данном случае для проверки данной задачи мы использовали набор 1000 случайных пробных площадей по округу, который был заранее интерпретирован несколькими экспертами, с учетом доступных космических снимков среднего и высокого разрешения и наземных данных. Набор пробных площадей был получен путем генерации случайных точек в пределах границ территории исследования (округа Оган-Комерин-Илир) с помощью инструмента «Случайные точки...» в QGIS. В качестве пробой площади рассматривался отдельный пиксель космических снимков Landsat (30$\times$30 метров), в пределы которого попала данная точка. Результат экспертной интерпретации (класс древесных плантаций или их отсутствие) записывался для каждой точки в атрибутивную таблицу векторного набора данных с пробными площадями.

Средствами ГИС было произведено сравнение пространственного расположения точек пробных площадей с полигонами (замкнутыми контурами) набора векторных данных, переданных каждой командой как конечный результат её работы по данной задаче.

\putTwoImg{8cm}{final/command_tour/analysis_cosm/task_01/5}{8cm}{final/command_tour/analysis_cosm/task_01/6}

\begin{center}
    Примеры карт древесных плантаций, полученных двумя разными командами, с наложенными на них сверху точками пробных площадей, интерпретированных экспертами. Точки пробных площадей окрашены теми же цветами, что и плантации на карте участников, в зависимости от присвоенного им экспертами класса.
\end{center}

С помощью инструмента «Присоединить атрибуты по местоположению» в QGIS для каждой пробной площади в атрибутивную таблицу было добавлено значение класса древесных плантаций из атрибутов полигона, в пределы которого данная точка попадала на карте команды. В случае, когда точка не попадала в пределы выделенных командами плантаций, добавлялось значение «0». В итоге для результатов каждой команды был получен набор векторных данных (точек), для каждой точки которого в атрибутивной таблице имелось два значения: (1) класс древесных плантаций, заранее определённый для соответствующей пробной площади опытными экспертами; и (2) класс древесных плантаций в соответствии с картой, полученной участниками.

С помощью простого скрипта в Excel строилась матрица соответствия следующего вида (конкретный пример по одной из команд):

\begin{center}
    \footnotesize
    \begin{tabular}{|p{2.8cm}|p{0.8cm}|p{0.8cm}|p{1cm}|p{0.9cm}|p{1.7cm}|p{1.2cm}|p{1.7cm}|p{1.1cm}|}
        \hline
         & \multicolumn{8}{c|}{\textbf{Класс по интерпретации пробных площадей:}} \\
        \hline
        \textbf{Класс по карте команды участников:} & Oil palm & Wood fiber & Rubber trees & 	Fruits trees & Other tree plantations & Clearing & Not plantations (0) &\textbf{Сумма} \\
        \hline
        Oil palm & 32 & 0 & 0 & 1 & 1 & 4 & 2 & \textbf{40} \\
        \hline
        Wood fiber & 1 & 129 & 0 & 0 & 1 & 5 & 33 & \textbf{169} \\
        \hline
        Rubber trees & 7 & 0 & 82 & 3 & 20 & 4 & 52 & \textbf{168} \\
        \hline
        Fruits trees & 1 & 0 & 1 & 5 & 7 & 0 & 3 & \textbf{17} \\
        \hline
        Other plantations & 2 & 0 & 1 & 0 & 2 & 0 & 26 & \textbf{31} \\
        \hline
        Clearing & 10 & 4 & 1 & 0 & 1 & 40 & 12 & \textbf{68} \\
        \hline
        Not plantations (0) & 51 & 2 & 8 & 2 & 8 & 21 & 415 & \textbf{507} \\
        \hline      
    \end{tabular}
\end{center}

По каждому классу вычислялась доля верных (совпадающих с экспертной оценкой) результатов, взвешенная по доле площади каждого класса (включая нулевой) от общей площади территории исследования (округа). Результат представлялся в виде матрицы ошибок (confusion matrix).

Пример матрицы ошибок (confusion matrix) по данным то же команды, что и матрица выше:

\begin{center}
    \footnotesize
    \begin{tabular}{|p{2.8cm}|p{0.85cm}|p{0.85cm}|p{1cm}|p{0.9cm}|p{1.7cm}|p{1.2cm}|p{1.7cm}|p{1.1cm}|}
        \hline
         & \multicolumn{8}{c|}{\textbf{Класс по интерпретации пробных площадей:}} \\
        \hline
        \textbf{Класс по карте команды участников:} & Oil palm & Wood fiber & Rubber trees & 	Fruits trees & Other tree plantations & Clearing & Not plantations (0) & \textbf{Сумма} \\
        \hline
        Oil palm & 5.9\% & 0.0\% & 0.0\% & 0.2\% & 0.2\% & 0.7\% & 0.4\% & \textbf{7.4\%} \\
        \hline
        Wood fiber & 0.1\% & 12.8\% & 0.0\% & 0.0\% & 0.1\% & 0.5\% & 3.3\% & \textbf{16.8\%} \\
        \hline
        Rubber trees & 0.7\% & 0.0\% & 8.1\% & 0.3\% & 2.0\% & 0.4\% & 5.2\% & \textbf{16.7\%} \\
        \hline
        Fruits trees & 0.1\% & 0.0\% & 0.1\% & 0.3\% & 0.5\% & 0.0\% & 0.2\% & \textbf{1.2\%} \\
        \hline
        Other plantations & 0.2\% & 0.0\% & 0.1\% & 0.0\% & 0.2\% & 0.0\% & 3.2\% & \textbf{3.8\%} \\
        \hline
        Clearing & 1.2\% & 0.5\% & 0.1\% & 0.0\% & 0.1\% & 4.7\% & 1.4\% & \textbf{8.0\%} \\
        \hline
        Not plantations (0) & 4.6\% & 0.2\% & 0.7\% & 0.2\% & 0.7\% & 1.9\% & 37.7\% & \textbf{46.1\%} \\
        \hline      
        \textbf{Сумма}: & \textbf{12.9}\% & \textbf{13.5\%} & \textbf{9.2\%} & \textbf{1.0\%} & \textbf{3.8\%} & \textbf{8.3\%} & \textbf{51.4\%} & \textbf{100.0\%} \\
        \hline
    \end{tabular}
\end{center}

По такой матрице можно определить для каждого класса древесных плантаций ошибки обеих типов – ошибка пользователя (ложноположительные результаты, ошибка первого рода, commission error, false positive) и ошибка производителя (ошибки пропуска, ошибка второго рода, omission error, false negative). Для оценки результата мы вычисляли обратные величины – точность пользователя и точность производителя.

Точность пользователя вычислялась для каждого класса в столбцах таблицы (класса по интерпретации пробных площадей) как значение в строке с этим классом по данным карты участников, поделённое на сумму значений по всей строке таблицы (по классу карты, созданной участниками).

Точность производителя вычислялась для каждого класса в строках таблицы (класса карты участников) как значение в столбце с этим классом по данным пробных площадей, поделённое на сумму значений по всему столбцу таблицы (по классу в соответствии с данными пробных площадей).

Для приведённого выше примера значения точности пользователя и точности производителя по классам плантаций следующие:

\begin{center}
    \begin{tabular}{|l|c|}
        \hline
        Точность пользователя по классу Oil palm & 80.0\% \\
        \hline
        Точность производителя по классу Oil palm & 46.1\% \\
        \hline
        Точность пользователя по классу Wood fiber & 76.3\% \\
        \hline
        Точность производителя по классу Wood fiber & 95.1\% \\
        \hline
        Точность пользователя по классу Rubber trees & 48.8\% \\
        \hline
        Точность производителя по классу Rubber trees & 88.7\% \\
        \hline
        Точность пользователя по классу Fruits trees & 29.4\% \\
        \hline
        Точность производителя по классу Fruits trees & 34.2\% \\
        \hline
        Точность пользователя по классу Clearing & 58.8\% \\
        \hline
        Точность производителя по классу Clearing & 57.2\% \\
        \hline
        Точность пользователя по классу Not plantations (0): & 81.9\% \\
        \hline
        Точность производителя по классу Not plantations (0): & 73.4\% \\
        \hline
    \end{tabular}
\end{center}

Для оценки результатов было решено использовать значения точности только для первых трёх классов, которые занимают достаточно большую площадь в пределах округа и результаты по которым наиболее репрезентативны. Для трёх данных классов были посчитаны средние значения для точности пользователя и точности производителя. За каждую из них начислялось от 0 до 25 баллов. Количество баллов вычислялось путям перемножения максимального балла (25) на средний процент точности.

Таким образом, всего за данную задачу можно было набрать от 0 до 50 баллов.