\section{Общая система оценки решений}

Общее максимальное количество баллов за решение всех трёх задач – 100. Их распределение будет следующее:
\begin{itemize}
    \item До 50 баллов – за решение основной Задачи А (результат оценивается путем сравнения с набором пробных площадей, интерпретированных экспертами).
    \item До 20 баллов – за базовые навыки и знания, продемонстрированные при решении Задачи А (оцениваются по итогам изучения лабораторного журнала, а также ежедневных выходных данных).
    \item До 20 баллов – за решение Задачи Б (оценивается путём сравнения результата с результатом экспертов).
    \item До 10 баллов – за решение Задачи В (требует решения как Задачи А, так и Задачи Б), оценивается по совпадению результатов расчётов с результатами нашего расчёта по вашим данным. Отклонение не должно составлять более 0,1.
\end{itemize}

Если ни одна из команд не справится с решением Задач А, Б и В, то результаты будут сравниваться по степени прогресса в решении Задачи А – на основании ваших записей в лабораторном журнале и ежедневных выходных данных.

В случае, если две команды или более двух команд успешно справятся с решением всех трёх задач раньше отведённого времени, им может быть дана дополнительная задача.