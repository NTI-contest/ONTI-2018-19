\section*{Приложение 1. Образец ведения Лабораторного журнала }

\begin{center}
    \textbf{Олимпиада НТИ,\\Профиль “Анализ космических снимков и геопространственных данных”}
\end{center}

\begin{center}
    \textbf{Лабораторный журнал}
\end{center}

\noindent\textbf{\underline{Команда:}} <НАЗВАНИЕ ВАШЕЙ КОМАНДЫ>\\

\noindent\textbf{День 1 (27 марта 2019 г.)}\\

\noindent\underline{Задачи на день: список операций (в скобках - исполнители):}\\
1.\underline{\hspace{6cm}}.(\underline{\hspace{6cm}})\\
2.\underline{\hspace{6cm}}.(\underline{\hspace{6cm}})\\
3.\underline{\hspace{6cm}}.(\underline{\hspace{6cm}})\\

\noindent\underline{Обоснование планируемых действий:}\\
\underline{\hspace{9cm}}.\\
\underline{\hspace{9cm}}.\\
\underline{\hspace{9cm}}.\\

\textbf{Операции (действия)}

\begin{center}
    \begin{longtable}{|l|l|}
        \hline
        № & Описание операции \\
        \hline
        1. & ФИО члена(ов) команды: \underline{\hspace{6cm}} \\
           & Начало выполнения операции: \underline{\hspace{5cm}} \\
           & Входные данные (названия файлов, ссылки): \underline{\hspace{6cm}}\\
           & \underline{\hspace{14cm}}\\
           & Ход работы: \\
           & - что было сделано; \\
           & - полученный результат; \\
           & - возникшие проблемы (если были).\\
           & . \\
           & . \\
           & . \\
           & Окончение выполнения операции: \underline {\hspace{6cm}}\\
           & Входные данные (названия файлов, ссылки): \underline{\hspace{6cm}} \\
           & \underline{\hspace{14cm}}\\
        \hline
        2. & ... \\
        \hline
        3. & ... \\
        \hline
        ... & ... \\
        \hline
    \end{longtable}
\end{center}

\noindent\underline{Итоги дня (27 марта 2019 г.):}\\
\begin{itemize}
    \item полученные результаты (итоги дня);
    \item возникшие проблемы (если были);
    \item какие корректировки вносятся в план решения по итогам дня.
\end{itemize}

\noindent\underline{\hspace{9cm}}.\\
\underline{\hspace{9cm}}.\\
\underline{\hspace{9cm}}.\\
\underline{\hspace{9cm}}.\\
\underline{\hspace{9cm}}.\\
\underline{\hspace{9cm}}.