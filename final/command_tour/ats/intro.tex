На этом этапе участникам необходимо было разработать части автономной транспортной системы: алгоритмы локального, глобального позиционирования и поиска пути для беспилотного автомобиля, алгоритмы навигации коптера внутри помещения с использованием технического зрения, алгоритмы передачи груза, а также смоделировать и изготовить механизм для захвата грузов. Для решения данной задачи участникам необходимо было использовать методы компьютерного зрения и машинного обучения, знания ЗД моделирования и электроники. 

Участникам был предложен ряд задач. Для их решения, необходимо было определить кто занимается коптером, а кто автомобилем. В получившихся парах обязательно должен быть программист и инженер. Инженер решает задачи 3Д моделирования и сборки устройства в оптимальной конфигурации, программист отвечает за управляющие алгоритмы устройства.

Максимальное количество баллов, которые могла заработать команда - 85.5 баллов. Каждый день открывались новые задания, их можно было выполнять в произвольном порядке. Задачи для автомобилей предыдущего дня можно было сдавать в любой момент без штрафов. Задания на день по 3D моделированию и программированию полета коптера каждый день выполнялись в зачет в обозначенное организаторами время. Задание текущего дня в последующие дни не засчитывалось.

В таблице ниже представлен список заданий по дням.

\begin{table}
    \begin{center}
        \begin{tabular}{|p{6.5cm}|p{2.5cm}|p{1.5cm}|}
            \hline
            ДЕНЬ 1 & Максимальный балл & Сумма \\
            \hline
            Создайте 3D модель и gcode универсального захватывающего устройства, способного захватывать и переносить грузы различной формы и сам груз & 4.7 & 6.5 \\
            \hline
            Нарисуйте принципиальную схему захватывающего устройства & 1
            Сохраните все файлы в форматах, пригодных для последующего редактирования и печати на 3D принтере
            0,8
            Запрограммируйте автономный взлёт коптера, сопровождаемый световой индикацией
            1,5
            8,5
            Напишите программу, позволяющую коптеру прилететь в заданную точку, зависнуть на 5 секунд и продемонстрировать достижение точки световой индикацией. Во время полета необходимо показывать, сколько осталось лететь до точки назначения 
            2,5
            Сделайте автономный пролет со световой индикацией над 3 контрольными точками
            4,5




            Научите автомобиль двигаться по прямому участку трассы, не вылетая за пределы своей полосы движения
            1
            15
            Научите автомобиль проходить повороты на трассе, не вылетая за пределы своей полосы движения
            2
            Нацчите автомобиль детектировать стоп-линию и остановиться перед перекрёстком.
            1
            Научите автомобиль пересекать перекрёсток совершая поворот направо
            2
            Научите автомобиль пересекать перекрёсток совершая поворот налево
            2
            Научите автомобиль пересекать перекрёсток не поворачивая
            2
            Научите автомобиль, не вылетая со своей полосы движения, проходить маршрут из точки А в точку Б
            5
            Сумма баллов:
            30
 

        \end{tabular}
    \end{center}
\end{table}

\begin{table}
    \begin{center}
        \begin{tabular}{|p{6.5cm}|p{2.5cm}|p{1.5cm}|}

        \end{tabular}
    \end{center}
\end{table}

\begin{table}
    \begin{center}
        \begin{tabular}{|p{6.5cm}|p{2.5cm}|p{1.5cm}|}

        \end{tabular}
    \end{center}
\end{table}