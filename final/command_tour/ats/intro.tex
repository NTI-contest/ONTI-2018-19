На этом этапе участникам необходимо было разработать части автономной транспортной системы: алгоритмы локального, глобального позиционирования и поиска пути для беспилотного автомобиля, алгоритмы навигации коптера внутри помещения с использованием технического зрения, алгоритмы передачи груза, а также смоделировать и изготовить механизм для захвата грузов. Для решения данной задачи участникам необходимо было использовать методы компьютерного зрения и машинного обучения, знания ЗД моделирования и электроники. 

Участникам был предложен ряд задач. Для их решения, необходимо было определить кто занимается коптером, а кто автомобилем. В получившихся парах обязательно должен быть программист и инженер. Инженер решает задачи 3Д моделирования и сборки устройства в оптимальной конфигурации, программист отвечает за управляющие алгоритмы устройства.

Максимальное количество баллов, которые могла заработать команда - 85.5 баллов. Каждый день открывались новые задания, их можно было выполнять в произвольном порядке. Задачи для автомобилей предыдущего дня можно было сдавать в любой момент без штрафов. Задания на день по 3D моделированию и программированию полета коптера каждый день выполнялись в зачет в обозначенное организаторами время. Задание текущего дня в последующие дни не засчитывалось.

В таблице ниже представлен список заданий по дням.

\begin{longtable}{|p{8.5cm}|p{3.5cm}|p{1.5cm}|}
    \hline
    ДЕНЬ 1 & Максимальный балл & Сумма \\
    \hline
    Создайте 3D модель и gcode универсального захватывающего устройства, способного захватывать и переносить грузы различной формы и сам груз & 4.7 & \multirow{3}{*}{6.5} \\ 
    \cline{1-2}
    Нарисуйте принципиальную схему захватывающего устройства & 1 & \\ 
    \cline{1-2}
    Сохраните все файлы в форматах, пригодных для последующего редактирования и печати на 3D принтере & 0.8 & \\
    \hline
    Запрограммируйте автономный взлёт коптера, сопровождаемый световой индикацией & 1.5 & \multirow{3}{*}{8.5} \\ 
    \cline{1-2}
    Напишите программу, позволяющую коптеру прилететь в заданную точку, зависнуть на 5 секунд и продемонстрировать достижение точки световой индикацией. Во время полета необходимо показывать, сколько осталось лететь до точки назначения & 2.5 & \\ 
    \cline{1-2}
    Сделайте автономный пролет со световой индикацией над 3 контрольными точками & 4.5 & \\ 
    \hline
    Научите автомобиль двигаться по прямому участку трассы, не вылетая за пределы своей полосы движения & 1 & \multirow{7}{*}{15} \\ 
    \cline{1-2}
    Научите автомобиль проходить повороты на трассе, не вылетая за пределы своей полосы движения & 2 & \\ 
    \cline{1-2}
    Нацчите автомобиль детектировать стоп-линию и остановиться перед перекрёстком. & 1 & \\ 
    \cline{1-2}
    Научите автомобиль пересекать перекрёсток совершая поворот направо & 2 & \\ 
    \cline{1-2}
    Научите автомобиль пересекать перекрёсток совершая поворот налево & 2 & \\ 
    \cline{1-2}
    Научите автомобиль пересекать перекрёсток не поворачивая & 2 & \\ 
    \cline{1-2}
    Научите автомобиль, не вылетая со своей полосы движения, проходить маршрут из точки А в точку Б & 5 & \\ 
    \hline
    \textbf{Сумма баллов:} & \multicolumn{2}{|c|}{30} \\
    \hline
\end{longtable}


\begin{longtable}{|p{8.5cm}|p{3.5cm}|p{1.5cm}|}           
    \hline
    ДЕНЬ 2 & Максимальный балл & Сумма \\ 
    \hline
    Изготовьте груз и захватывающее устройство для коптера с использованием 3D принтера & 3.5 & \multirow{2}{*}{5} \\ 
    \cline{1-2}
    Установите захватывающее устройство на Коптер и продемонстрируйте управление им через программу & 1.5 & \\ 
    \hline
    Напишите программу, позволяющую Коптеру подключаться к серверу, получать и выводить полученную миссию на экран & 2 & \multirow{6}{*}{9} \\ 
    \cline{1-2}
    Расшифруйте и переведите в координаты миссию, полученную от сервера & 1 & \\ 
    \cline{1-2}
    Совершите пролет по миссии & 1 & \\ 
    \cline{1-2}
    Произведите захват груза в автономном или ручном режиме с указанного здания & 2.5 & \\ 
    \cline{1-2}
    Произведите взлет с грузом с указанного здания в автономном режиме & 1 & \\ 
    \cline{1-2}
    Произведите сброс груза, сопровождаемый световой индикацией и отправьте серверу сообщение о выполнении миссии & 1.5 & \\ 
    \hline
    Научите автомобиль детектировать светофор & 3 & \multirow{3}{*}{19} \\ 
    \cline{1-2}
    Научите автомобиль проходить перекрёсток в соответствии с сигналом светофора и вести детектирование и распознование сигнала на ходу, не останавливаясь. & 5 & \\ 
    \cline{1-2}
    Научите автомобиль, не вылетая со своей полосы движения и пересекая перекрёсток, в соответствии с сигналом светофора, проходить маршрут из точки А в точку Б & 5 & \\ 
    \hline
    \textbf{Сумма баллов:} & \multicolumn{2}{|c|}{33} \\
    \hline
\end{longtable}


\begin{longtable}{|p{8.5cm}|p{3.5cm}|p{1.5cm}|}
    \hline                       
    ДЕНЬ 3 & Максимальный балл & Сумма \\
    \hline
    Произведите испытания захватывающего устройства и груза & 1.5 & 1.5 \\ 
    \hline
    Выполните контрольный пролет по полигону: Коптер получает от сервера сообщение о том, в какую больницу машина везет пациента, взлетает с отметки H со световой индикацией, долетает до места хранения груза, хватает груз, сбрасывает груз на нужный дом, обозначая доставку световой индикацией, возвращается на базу, производит посадку и посылает серверу сообщение о выполнении миссии & 8 & \multirow{3}{*}{10} \\ 
    \cline{1-2}
    Выполните финальное задание полностью (все действия в связке) & 1 & \\ 
    \cline{1-2}
    Внесите необычное (оригинальное) световое / конструкционное решение в свой проект & 1 & \\ 
    \hline
    Организуйте получение автомобилем данных с сервера. & 2 & \multirow{4}{*}3{11} \\ 
    \cline{1-2}
    Организуйте отправление автомобилем данных на сервер & 2 & \\ 
    \cline{1-2}
    Создайте алгоритм выбора наиболее оптимального пути к заданной цели с учётом загруженных участков трассы & 3 & \\ 
    \cline{1-2}
    Научите автомобиль, не вылетая со своей полосы движения и пересекая перекрёстки, в соответствии с сигналом светофора, проходить маршрут, полученный от сервера, оптимальным образом & 4 & \\ 
    \hline
    \textbf{Сумма баллов:} & \multicolumn{2}{|c|}{22.5} \\
    \hline
\end{longtable}


По всем задачам машинного зрения и локального позиционирования записаны обучающие видео: \url{https://www.youtube.com/watch?v=Y93pvlS_u4Y&list=PLlQE9Jt1}\linebreak \url{MqC5ElGixnnPG1lKZ5WqjJJ86}. В видео-уроках есть небольшая часть решения от разработчиков трека, участники могли использовать эти части, если правильно адаптировали их к своим условиям.