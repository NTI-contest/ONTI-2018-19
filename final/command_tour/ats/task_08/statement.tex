\section{Подтрек “Аэро”}

\subsubsection*{Работа с MQTT}

MQTT – протокол для обмена сообщениями между различными устройствами. Этот протокол используется для отправки команд дрону на Олимпиаде НТИ 2019. Для отправки сообщения оно публикуется в определенный топик, все подписчики этого топика получают это сообщение.

\subsubsection*{Подписка на топики}

В образе Клевера для Олимпиады НТИ 2019 предустановлена библиотека paho-mqtt для Python. Пример работы с этой библиотекой описан ниже:

\inputminted[fontsize=\footnotesize, linenos]{python}{final/command_tour/ats/task_08/source_1.py}

Более подробная документация доступна на странице библиотеки в PyPI (\url{https://pypi.org/project/paho-mqtt/}).

\subsubsection*{Отправка сообщений}

Для отправки сообщений можно использовать метод publish клиента:

\inputminted[fontsize=\footnotesize, linenos]{python}{final/command_tour/ats/task_08/source_2.py}

Данный код опубликует сообщение connected в топик /xxx.

\subsubsection*{Проверка}

Для проверки вы можете опубликовать любое сообщение в топик с помощью команды hbmqtt\_pub:\\
hbmqtt\_pub --url mqtt://192.168.1.199:1883 -t /xxx -m 'сообщение'

Где 192.168.1.199 – IP-адрес MQTT-брокера, сообщение – сообщение для публикации, /xxx~– необходимый топик для публикации.

Чтобы проверить публикацию сообщений от клиента, воспользуйтесь командой hbmqtt\_sub:\\
hbmqtt\_sub --url mqtt://192.168.1.199:1883 -t /xxx

Отправленные в топик /xxx сообщения будут показаны в терминале.

\subsubsection*{Работа с Клевером}

Для выполнения команд на Клевере:
\begin{itemize}
    \item подключитесь в Wi-Fi сети NTI;
    \item подключитесь к вашему Клеверу по SSH по его IP-адресу (подробнее см. подключение по SSH; \url{https://clever.copterexpress.com/ru/ssh.html});
\end{itemize}

После подключения к своему дрону по SSH, смените пароль SSH-доступа, чтобы другие участники не смогли несанкционированно подключаться к нему. Для этого используйте (\url{https://www.raspberrypi-spy.co.uk/2012/10/how-to-change-raspberry-pi-password/})команду passwd.

Для редактирования файлов на Клевере вы можете использовать консольные редакторы nanoили vim. Также вы можете загружать файлы используя PyCharm или WinSCP.

Для автономного полета используйте API модуля simple\_offboard (\url{https://clever.copterexpress.com/ru/simple_offboard.html}).

При использовании русских букв в скрипте на Python 2, добавьте в начало программы следующую строку:\\
\# coding: utf8

Пример программы, выполняющей взлет, полет в точку в системе координат площадки и посадку на Python:

\inputminted[fontsize=\footnotesize, linenos]{python}{final/command_tour/ats/task_08/source_3.py}

\subsubsection*{Описание API}

Незаполненные числовые параметры устанавливаются в значение 0.

\textit{get\_telemetry}

Получить полную телеметрию коптера.
Параметры:
\begin{itemize}
    \item frame\_id – система координат для значений x, y, z, vx, vy, vz. Пример: map, body, aruco\_map. Значение по умолчанию: map.
\end{itemize}
Формат ответа:
\begin{itemize}
    \item frame\_id – система координат;
    \item connected – есть ли подключение к FCU;
    \item armed – состояние armed винтов (винты включены, если true);
    \item mode – текущий полетный режим;
    \item x, y, z – локальная позиция коптера (м);
    \item lat, lon – широта, долгота (градусы), необходимо наличие GPS;
    \item alt – высота в глобальной системе координат (стандарт WGS-84, не AMSL!), необходимо наличие GPS;
    \item vx, vy, vz – скорость коптера (м/с);
    \item pitch – угол по тангажу (радианы);
    \item roll – угол по крену (радианы);
    \item yaw – угол по рысканью (радианы);
    \item pitch\_rate – угловая скорость по тангажу (рад/с);
    \item roll\_rate – угловая скорость по крену (рад/с);
    \item yaw\_rate – угловая скорость по рысканью (рад/с);
    \item voltage – общее напряжение аккумулятора (В);
    \item cell\_voltage – напряжение аккумулятора на ячейку (В).
\end{itemize}

Недоступные по каким-то причинам поля будут содержать в себе значения NaN.

Вывод координат x, y и z коптера в локальной системе координат:
\begin{minted}[fontsize=\footnotesize, linenos]{python}
telemetry = get_telemetry()
print telemetry.x, telemetry.y, telemetry.z
\end{minted}

Вывод высоты коптера относительно карты ArUco-меток:
\begin{minted}[fontsize=\footnotesize, linenos]{python}
telemetry = get_telemetry(frame_id='aruco_map')
print telemetry.z
\end{minted}

Проверка доступности глобальной позиции:
\begin{minted}[fontsize=\footnotesize, linenos]{python}
import math
if not math.isnan(get_telemetry().lat):
	print 'Global position presents'
else:
	print 'No global position'
\end{minted}    

Вывод текущей телеметрии (командная строка):\\
rosservice call /get\_telemetry "\{frame\_id: ''\}"\

\textit{navigate}

Прилететь в обозначенную точку по прямой.

Параметры:
\begin{itemize}
    \item x, y, z – координаты (м);
    \item yaw – угол по рысканью (радианы);
    \item yaw\_rate – угловая скорость по рысканью (применяется при установке yaw в NaN) (рад/с);
    \item speed – скорость полета (скорость движения setpoint) (м/с);
    \item auto\_arm – перевести коптер в OFFBOARD и заармить автоматически (коптер взлетит);
    \item frame\_id – система координат, в которой заданы x, y, z и yaw (по умолчанию: map).
\end{itemize}

Для полета без изменения угла по рысканью достаточно установить yaw в NaN(значение угловой скорости по-умолчанию – 0).

Взлет на высоту 1.5 м со скоростью взлета 0.5 м/с:\\
navigate(x=0, y=0, z=1.5, speed=0.5, frame\_id='body', auto\_arm=True)

Полет по прямой в точку 5:0 (высота 2) в локальной системе координат со скоростью 0.8 м/с (рысканье установится в 0):\\
navigate(x=5, y=0, z=3, speed=0.8)

Полет в точку 5:0 без изменения угла по рысканью (yaw = NaN, yaw\_rate = 0):\\
navigate(x=5, y=0, z=3, speed=0.8, yaw=float('nan'))

Полет вправо относительно коптера на 3 м:\\
navigate(x=0, y=-3, z=0, speed=1, frame\_id='body')

Повернуться на 90 градусов против часовой:\\
navigate(yaw=math.radians(-90), frame\_id='body')

Полет в точку 3:2 (высота 2) в системе координат маркерного поля со скоростью 1 м/с:\\
navigate(x=3, y=2, z=2, speed=1, frame\_id='aruco\_map')

Вращение на месте со скоростью 0.5 рад/c (против часовой):\\
navigate(x=0, y=0, z=0, yaw=float('nan'), yaw\_rate=0.5, frame\_id='body')

Полет вперед 3 метра со скоростью 0.5 м/с, вращаясь по рысканью со скоростью 0.2 рад/с:\\
navigate(x=3, y=0, z=0, speed=0.5, yaw=float('nan'), yaw\_rate=0.2, frame\_id='body')

Взлет на высоту 2 м (командная строка):\\
rosservice call /navigate "\{x: 0.0, y: 0.0, z: 2, yaw: 0.0, yaw\_rate: 0.0, speed: 0.5, frame\_id: 'body', auto\_arm: true\}"\

\textit{navigate\_global}

Полет по прямой в точку в глобальной системе координат (широта/долгота).

Параметры:
\begin{itemize}
    \item lat, lon – широта и долгота (градусы);
    \item z – высота (м);
    \item yaw – угол по рысканью (радианы);
    \item yaw\_rate – угловая скорость по рысканью (при установке yaw в NaN) (рад/с);
    \item speed – скорость полета (скорость движения setpoint) (м/с);
    \item auto\_arm – перевести коптер в OFFBOARD и заармить автоматически (коптер взлетит);
    \item frame\_id – система координат, в которой заданы z и yaw (по умолчанию: map).
\end{itemize}

Для полета без изменения угла по рысканью достаточно установить yaw в NaN(значение угловой скорости по-умолчанию – 0).

Полет в глобальную точку со скоростью 5 м/с, оставаясь на текущей высоте (yaw установится в 0, коптер сориентируется передом на восток):\\
navigate\_global(lat=55.707033, lon=37.725010, z=0, speed=5, frame\_id='body')

Полет в глобальную точку без изменения угла по рысканью (yaw = NaN, yaw\_rate = 0):\\
navigate\_global(lat=55.707033, lon=37.725010, z=0, speed=5, yaw=float('nan'), frame\_id='body')

Полет в глобальную точку (командная строка):\\
rosservice call /navigate\_global "\{lat: 55.707033, lon: 37.725010, z: 0.0, yaw: 0.0, yaw\_rate: 0.0, speed: 5.0, frame\_id: 'body', auto\_arm: false\}"\

\textit{set\_position}

Установить цель по позиции и рысканью. Данный сервис следует использовать при необходимости задания продолжающегося потока целевых точек, например, для полета по сложным траекториям (круговой, дугообразной и т. д.).

Для полета на точку по прямой или взлета используйте более высокоуровневый сервис navigate.

Параметры:
\begin{itemize}
    \item x, y, z – координаты точки (м);
    \item yaw – угол по рысканью (радианы);
    \item yaw\_rate – угловая скорость по рысканью (при установке yaw в NaN) (рад/с);
    \item auto\_arm – перевести коптер в OFFBOARD и заармить автоматически (коптер взлетит);
    \item frame\_id – система координат, в которой заданы x, y, z и yaw (по умолчанию: map).
\end{itemize}

Зависнуть на месте:\\
set\_position(frame\_id='body')

Назначить целевую точку на 3 м выше текущей позиции:\\
set\_position(x=0, y=0, z=3, frame\_id='body')

Назначить целевую точку на 1 м впереди текущей позиции:\\
set\_position(x=1, y=0, z=0, frame\_id='body')

Вращение на месте со скоростью 0.5 рад/c:\\
set\_position(x=0, y=0, z=0, frame\_id='body', yaw=float('nan'), yaw\_rate=0.5)

\textit{set\_velocity}

Установить скорости и рысканье.
\begin{itemize}
    \item vx, vy, vz – требуемая скорость полета (м/с);
    \item yaw – угол по рысканью (радианы);
    \item yaw\_rate – угловая скорость по рысканью (при установке yaw в NaN) (рад/с);
    \item auto\_arm – перевести коптер в OFFBOARD и заармить автоматически (коптер взлетит);
    \item frame\_id – система координат, в которой заданы vx, vy, vz и yaw (по умолчанию: map).
\end{itemize}

Параметр frame\_id определяет только ориентацию результирующего вектора скорости, но не его длину.

Полет вперед (относительно коптера) со скоростью 1 м/с:\\
set\_velocity(vx=1, vy=0.0, vz=0, frame\_id='body')

Один из вариантов полета по кругу:\\
set\_velocity(vx=0.4, vy=0.0, vz=0, yaw=float('nan'), yaw\_rate=0.4, frame\_id='body')

\textit{set\_attitude}

Установить тангаж, крен, рысканье и уровень газа (примерный аналог управления в режиме STABILIZED). Данный сервис может быть использован для более низкоуровнего контроля поведения коптера либо для управления коптером при отсутствии источника достоверных данных о его позиции.

Параметры:
\begin{itemize}
    \item pitch, roll, yaw – необходимый угол по тангажу, крену и рысканью (радианы);
    \item thrust – уровень газа от 0 (нет газа, пропеллеры остановлены) до 1 (полный газ);
    \item auto\_arm – перевести коптер в OFFBOARD и заармить автоматически (коптер взлетит);
    \item frame\_id – система координат, в которой задан yaw (по умолчанию: map).
\end{itemize}

\textit{set\_rates}

Установить угловые скорости по тангажу, крену и рысканью и уровень газа (примерный аналог управления в режиме ACRO). Это самый низкий уровень управления коптером (исключая непосредственный контроль оборотов моторов). Данный сервис может быть использован для автоматического выполнения акробатических трюков (например, флипа).
Параметры:
\begin{itemize}
    \item pitch\_rate, roll\_rate, yaw\_rate – угловая скорость по тангажу, крену и рыканью (рад/с);
    \item thrust – уровень газа от 0 (нет газа, пропеллеры остановлены) до 1 (полный газ).
    \item auto\_arm – перевести коптер в OFFBOARD и заармить автоматически (коптер взлетит);
\end{itemize}

\textit{land}

Перевести коптер в режим посадки (AUTO.LAND или аналогичный).

Для автоматического отключения винтов после посадки параметр\\ PX4COM\_DISARM\_LAND должен быть установлен в значение > 0.

Посадка коптера:
\begin{minted}[fontsize=\footnotesize, linenos]{python}
res = land()
 
if res.success:
	print 'Copter is landing'
\end{minted}

Посадка коптера(команднаястрока):\\
rosservice call /land "{}"

\subsubsection*{Работа с светодиодной лентой}

В используемой версии Клевера LED-лента подключена напрямую к Raspberry Pi. При включении всех светодиодов ленты на полную мощность возможно повреждение цепей питания микрокомпьютера.

Сигнальный провод ленты подключен к GPIO-пину 18.

\subsubsection*{Работа с LED-лентой через ROS}

В образ Клевера для Олимпиады НТИ включена нода ROS, работающая со светодиодной подсветкой. С её помощью можно управлять светодиодами, не запуская свою программу из-под sudo. По умолчанию эта нода выключена, но её можно включить, если в файле /home/pi/catkin\_ws/src/ros\_ws281x/launch/clever4.launch изменить строку \\
<arg name="enable"default="false"/>\\
на\\
<arg name="enable"default="true"/>\\
и перезапустить службу rosled:\\
sudo systemctl restart rosled

Пример работы со светодиодной лентой:

\inputminted[fontsize=\footnotesize, linenos]{python}{final/command_tour/ats/task_08/source_4.py}

\subsubsection*{Алгоритм захвата груза}

Одним из оптимальных алгоритмов для захвата груза с помощью установленного на коптер электромагнитного захвата, свободно подвешенного на коптер с помощью крепления длиной 40 см - летим вниз над грузом до тех пор, пока коптер находится на высоте более 35 см и внутри «цилиндра» радиусом 8 см. Повторяем посадку несколько раз.

\inputminted[fontsize=\footnotesize, linenos]{python}{final/command_tour/ats/task_08/source_5.py}

Для работы с GPIO на Raspberryнеобходимо использовать Pigpio. Сначала включите его так:\\
sudo systemctl start pigpiod.service
 
Для автозапуска pigpio используйте:\\
sudo systemctl enable pigpiod.service

Затем для подачи и снятия сигнала с пинов используйте такой код:

\begin{minted}[fontsize=\footnotesize, linenos]{python}
import pigpio
pi = pigpio.pi()
pi.write(4, 0)
pi.write(4, 1)
\end{minted}

(для 4-го пина)