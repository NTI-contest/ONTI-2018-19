\markSection

Игровые очки, автоматически начисляемые в экономической модели игры, являются интегральной оценкой всех логистических, энергетических, физических, математических решений принятых командой. В финальном подсчете для команд-победителей полуфиналов турнира игровые очки переводятся в баллы от 0 до 100. Команда, набравшая максимальное число очков, получает 100 баллов (и становится командой-победителем), команда, набравшая минимальное число очков — 0 баллов. Остальные команды, получают баллы, нормированные на эти два результата по формуле 

$$S = 100\times \frac{x - MIN}{MAX - MIN},$$

где $x$ — число внутриигровых очков, набранных командой, $MAX$ — число очков, набранное в финале командой-победителем, $MIN$ — число очков, набранное командой, занявшее в финале последнее место. Таким образом, команда-победитель финала получает всегда 100 баллов, команда, занявшая в финале последнее место — всегда 0 баллов, остальные — от 0 до 100, пропорционально показанным ими результатам.