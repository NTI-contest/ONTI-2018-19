\subsection*{Физика}

Задачи из физики присутствовали, однако их физическая составляющая была сведена к минимуму, и их физические модели были тривиальными.

\subsubsection*{Оптимальное расположение ветряка}

Задача возникает на 4-м этапе игры, при монтаже энергосистемы. Кроме того, команда должна знать, насколько эффективно она может решить эту задачу для эффективной работы на этапах 1–3 — при анализе прогнозов погоды и при участии в аукционе.

Первая гипотеза «установить ветряк как можно ближе к анемометру» полностью разрушается двумя факторами:

\begin{enumerate}
    \item При приближении к вентилятору создаваемое им ветровое поле становится всё более неравномерным. Например, на направлении его оси скорость ветра снижается, поскольку проталкивание воздуха осуществляется не всей плоскостью вентилятора, а только его лопастями. На следующем порядке малости на скорость ветра влияет расположение вентилятора относительно стенда и стен помещения, расположения других объектов на стенде (в особенности других ветряков).
    \item Устройство анемометра, установленного на модели ветряка,- вертикально-осевое. На скорость его вращения, в большей степени, чем сама скорость ветра, влияет разность ветрового давления, создаваемого на правую и левую его часть. Получается, что анемометр нужно устанавливать не в зоне большего ветра, а в зоне, в которой силы, вращающие его в «правильную» сторону (против часовой стрелки), будут максимально превосходить силы, вращающие его в противоположную сторону.
\end{enumerate}

Эта задача оптимального расположения весьма эффективно и просто решается при помощи простого наблюдения за скоростью вращения анемометра.

\subsubsection*{Оптимальное расположение солнечной батареи}

Эта задача возникает на 4-м этапе игры, при монтаже энергосистемы. Кроме того, команда должна знать, насколько эффективно она может решить эту задачу для эффективной работы на этапах 1–3 — при анализе прогнозов погоды и при участии в аукционе.

Задача оптимального расположения также достаточно эффективно решается при помощи простой серии экспериментов и наблюдения.

\subsubsection*{Определение взаимосвязи между яркостью освещения и генерацией солнечных батарей}

Эта задача возникает на 5-м этапе игры, при моделировании энергосистемы. Вырабатываемая мощность солнечных батарей зависит от напряжения на солнечных панелях модели солнечной электростанции на стенде. Напряжение на солнечных панелях по отношению к яркости светильников, строго говоря, нелинейно. Однако характеристики солнечных панелей, измеряющих цепей и светильников подобраны так, что отклонение реальных значений от линейной их аппроксимации составляет не более 2%, и в дальнейшем откалиброваны до полной линейности внутренним ПО солнечных батарей.

Время релаксации измерительной системы солнечных батарей в 2,5 раза меньше минимального интервала между изменением яркости и измерением вырабатываемой мощности, поэтому генерация солнечных батарей зависит только от погоды на текущем такте игры.

\subsubsection*{Определение взаимосвязи между скоростью ветра и генерацией ветряков}

Эта задача возникает на 5-м этапе игры, при моделировании энергосистемы. Кроме того, команда должна иметь решение этой задачи для эффективной работы на этапах 1–3 — при проектировании энергосистемы и при участии в аукционе.

Задача вычисления генерации ветровых электростанций по данным погоды похоже на такую же задачу для солнечных батарей, но является более сложной.

Для получения величины генерации используется скорость вращения анемометра, который установлен на модели ветровой электростанции. Устройство анемометра — вертикально-осевой; его лопасти устроены так, что скорость его вращения линейно пропорциональна скорости ветра (если считать поток ветра гомогенным). В случае постоянной негомогенности ветрового потока (когда анемометр не перемещается) скорость вращения анемометра также линейна по отношению к максимальной скорости ветра в потоке.

Измерительная система — инерциальная система вращения с трением. Её параметры нужно вычислять по данным прошедших игр. Время полной остановки анемометра с максимальной скорости вращения составляет около 3 тактов игры (оно зависит от максимальной скорости ветра в месте расположения анемометра). Время полного разгона аналогично составляет 1 такт игры.

Генерируемая мощность пропорциональная кубу скорости вращения анемометра. При достижении максимального уровня генерации (15 МВт) мощность далее не растёт.

Зависимость генерации от скорости ветра можно оценить либо эвристически, либо как линейную комбинацию от погоды за последние 3 такта игры на основании данных прошедших игр.

Ветровую электростанцию возможно установить так, что максимальная мощность будет достигаться даже при сравнительно небольших скоростях ветра (по данным погоды). Вычисление коэффициента в зависимости генерации от скорости ветра необходимо делать экспериментально. Это можно оценить предварительными экспериментами, либо заложить процедуру оценки в управляющий скрипт, чтобы он делал её на каждом такте.

Например, вычисленные «задним числом» коэффициенты линейной комбинации от прогнозов погоды за финальную игру составляют для наилучшего ветрогенератора:

\begin{enumerate}
    \item 0,69 от погоды на такт, для которого вычисляем прогноз.
    \item 0,22 от погоды за предыдущий такт
    \item 0,09 от погоды за пред-предыдущий такт
\end{enumerate}

При вычислениях необходимо учитывать тот факт, что генерация ветровых электростанций ограничена правилами на уровне 15 МВт. Поэтому если прогнозируемая генерация превышает этот уровень, надо считать, что спрогнозировано именно 15.
Средняя величина ошибки такого набора коэффициентов между прогнозируемой и реальной генерацией на играх финала составила 4,2\%, что, с одной стороны, связано с большой инерциальностью физического анемометра. С другой стороны, из-за кубической зависимости генерации от силы ветра, любые погрешности «в середине» диапазона скоростей ветра очень сильно влияют на прогнозируемую мощность. Если скорость ветра мала или велика, погрешности влияют очень слабо.

\includeSolutionIfExistsByPath{final/command_tour/ies/task_01/subtask_01/}