\subsection*{Математика}

\subsubsection*{Вычисление полного энергетического баланса на основании данных прогнозов}

Эта задача возникает на 1-м и 5-м этапах игры, при анализе прогнозов и моделировании энергосистемы.

Всё время игры командам доступны все данные о прогнозах погоды и действующих контрактах. Из них можно вычислить прогноз дефицита или профицита мощности на каждый такт. Для этого нужно на основании составленных заранее игроками моделей вычислить из прогнозов погоды прогнозы генерации. Из результирующего множества случайных величин (вероятная генерация для каждой электростанции и вероятное потребления для каждого потребителя) нужно вычислить их сумму. Она будет представлять собой распределение вероятностей профицита/дефицита мощности в системе. Важно учитывать, что максимальный дефицит или профицит мощности ограничен главной подстанцией и установленными на ней объектами.

\solutionSection

Приведённый ниже модуль на языке Python3 называется powerbalance и будет использоваться почти во всех остальных решениях

\inputminted[fontsize=\footnotesize, linenos]{python}{final/command_tour/ies/task_01/subtask_02/source_1.py}

\subsubsection*{Вычисление баланса энергорайонов энергосистемы}

Каждая энергосистема состоит из набора подстанций и подключённых к ним объектов. Эффективно энергосистема представляет собой дерево, в узлах которого находятся подстанции, а рёбра представляют собой энергорайоны с множеством подключённых объектов и, что очень важно, двумя узлами подстанций, к которым подключен энергорайон (причём если энергорайон физически подключён только к одной подстанции, то второй узел виртуален). По правилам мощность, протекающая через один узел, не может быть больше 15 МВт, иначе узел аварийно отключится. Это приводит к тому, что необходимо прогнозировать энергобаланс в каждом энергорайоне по-отдельности (полностью аналогично нахождению общего баланса), и учитывать складывание токов в энергосистеме при протекании их по дереву. Эта задача похожа на задачу «Раздеревенение» второго заочного тура олимпиады прошлого года, и задачу «Посади электродерево» очного тура по информатике этого года.

\subsubsection*{Вычисление экономического баланса энергосистемы на основании данных прогнозов}

Эта задача возникает на 1-м и 5-м этапах игры, при анализе прогнозов и моделировании энергосистемы. Кроме того, команда должна иметь решение этой задачи для эффективной работы на этапах 2–3 — участии в аукционе.

Далее нужно решить задачу нахождения вероятностного распределения экономических потерь. Это делается почти тривиально при принятии консервативной оценки прибылей и убытков от взаимодействия с внешней энергосистемой: каждый мегаватт непредсказанной избыточной мощности эффективно несёт убыток в 1 у.е., а каждый мегаватт недостаточной — 2 у.е. К этим значениям нужно добавить стоимость электроэнергии, закупленной/проданной во внешней энергосистеме по цене за 1 такт вперёд. Далее математическое ожидание получившейся случайной величины нужно минимизировать, используя параметры направляемой/извлекаемой мощности из аккумуляторов и закупаемой/продаваемой мощности во внешней энергосистеме. Выгоды от торговли с другими командами и ранних закупок во внешней энергосистеме можно учитывать независимо.

Эта задача немного проще, чем кажется из-за того, что использование аккумуляторов экономически намного выгоднее взаимодействия с внешней энергосистемой, имеет смысл закупать/продавать электроэнергию в ней только в случае невозможности сделать это через аккумулятор. Эти два параметра оказываются связаны в один — «балансирующее воздействие на энергосистему», которое можно разбить на аккумуляторы и внешнюю энергосистему «задним числом» после решения этой задачи.

Эту задачу (нахождение балансирующего воздействия, минимизирующего математическое ожидание экономических потерь) недостаточно решить аналитически, алгоритм решения нужно реализовать также в управляющем скрипте (и других вспомогательных программах), что для большинства школьников является нетривиальной и неустойчивой к ошибкам задачей. Немного проще её решить «перебором» — перебрав все значения балансирующего воздействия в размахе случайного распределения дефицита/профицита мощности в системе с фиксированным шагом, например, 0,1. Такой точности будет вполне достаточно, такое решение можно реализовать быстрее, чем точное, и впоследствии при наличии времени его можно точным заменить.

В дальнейшем на основании этих данных можно вычислить распределение вероятностей прибыли за всю игру.

\solutionSection

Программа представляет собой модуль costbalance, который будет использоваться в примере решения задачи нахождения предельной стоимости объекта на аукционе.

\inputminted[fontsize=\footnotesize, linenos]{python}{final/command_tour/ies/task_01/subtask_02/source_2.py}

\subsubsection*{Расчёт предельной цены объекта}

Эта задача возникает на 2-м и 3-м этапах игры, при участи в аукционе.

Это задача нахождения предельной цены контракта объекта на аукционе — такой цены, приобретение контракта по которой ожидаемо не принесёт ни прибылей, ни убытков. Это расширение над задачей нахождения полного экономического баланса энергосистемы — достаточно вычислить суммарную прогнозируемую прибыль энергосистемы с этим объектом и без него. Для электростанций разницу между этими величинами нужно разделить на число тактов в игре. Для потребителей — разделить на суммарное потребление его за всю игру (это случайная величина; для примерной оценки её можно заменить на математическое ожидание, но на этом этапе у команд уже должно быть в избытке опыта вычислений со случайными величинами).

Получившееся число: для электростанций — максимальная осмысленная цена, для потребителей — минимальная.

\solutionSection

\inputminted[fontsize=\footnotesize, linenos]{python}{final/command_tour/ies/task_01/subtask_02/source_3.py}

\subsubsection*{Составление и адаптация стратегии для аукционов}

Эта задача возникает на 2-м и 3-м этапах игры, при участи в аукционе.

Хотя команды могут найти оптимальную энергосистему для любого набора прогнозов погоды, эту энергосистему им ещё нужно «отвоевать» у конкурентов. Эта игра в целом относится к рефлексивным, игры этого класса не имеют устойчивого решения, а использование смешанных стратегий едва ли оправдано на одном прогоне игры. Участникам нужно анализировать поведение оппонентов, предугадывать их цели, и искать способы помешать целям оппонентов и защитить свои. Для этого можно создать несколько вспомогательных инструментов:
\begin{enumerate}
    \item Нахождение эффективных конфигураций энергосистем. Не только оптимальной, но всех, которые достаточно хороши. Все команды будут стремиться к какой-то из них, и для любой промежуточной ситуации на аукционе можно предположить, к какой из них будут стремиться оппоненты.
    \item Нахождение оптимальной ставки на аукционе, исходя из ценности лота для себя и оппонентов. Ценности лота для оппонентов можно оценить из предположения об энергосистеме, к которой они стремятся, с использованием решения задачи расчёта прогноза рентабельности. В этих условиях оптимальной ставкой будет максимальная ставка всех оппонентов, при условии, что она не превышает собственной максимальной ставки. Далее участникам будет интересно от этой ставки отклониться, чтобы рискнуть и увеличить выгоду, либо нарушить стратегию оппонентов (от одного приобретённого по некорректной цене объекта, согласно правилам аукциона, можно избавиться в конце аукциона).
\end{enumerate}

Также в течение всего времени аукциона командам нужно оценивать риск того, что целевую энергосистему составить не удастся, и при необходимости переходить к альтернативным вариантам. Такую задачу решить в общем случае в ограниченное время не представляется возможным, поэтому в ней на первое место выходит смекалка, скорость мышления и скорость принятия решений в команде.

\subsubsection*{Работа с биржей электроэнергии}

Эта задача, тривиальная сама по себе, значительно углубляет задачи вычисления экономического баланса энергосистемы и составления стратегии для аукциона.

Эта задача возникает на 5-м этапе игры, при моделировании энергосистемы.

Таблица цен составлена таким образом, что любое взаимодействие с другим игроком всегда выгоднее его отсутствия (и вынужденного взаимодействия с внешней энергосистемой). Причём более раннее выставление заявки на биржу всегда выгоднее выставление такой же заявки позже.

Таким образом, задача сводится к тому, чтобы делать как можно более ранние закупки на бирже (которые можно распланировать на основании прогнозов даже до начала игры) и их коррекции во время игры на основании действительного состояния энергосистемы (здесь в основном речь идёт о коррекции моделей солнечных батарей и ветряков «на лету»).
