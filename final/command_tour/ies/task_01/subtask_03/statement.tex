\subsection*{Информатика}

\subsubsection*{Система поддержки принятия решений на аукционе}

Это первая точка сборки решений предметных задач, возникающих в командном туре.

Основная задача таких систем — вычисление ценности лота при заданных параметрах энергосистемы.

Самый примитивный вариант такой системы может быть устроен следующим образом:

\begin{enumerate}
    \item Имеются прогнозы погоды и потребления на следующую игру. Мы будем считать, что реальные значения будут точно соответствовать прогнозам.
    \item Для каждого такта игры вычисляем генерацию. Например, солнечные батареи вычисляем из яркости солнца с найденным ранее коэффициентом конверсии солнечной энергии в мощность, например, 1 МВт / 651 лк.
    \item Вычисляем энергобаланс в каждый такт игры.
    \item Вычисляем изменение счёта в каждый такт игры.
    \item Добавляем в энергосистему интересующий нас объект.
    \item Повторяем шаги 1–4.
    \item Сравниваем результаты для энергосистемы при наличии и без наличия интересующего объекта.       
    \begin{enumerate}
        \item Для электростанций оптимальная цена есть их цена плюс разница результатов энергосистем, делённая на число тактов игры.
        \item Для потребителей оптимальная цена есть их цена плюс разница результатов игры, делённая на потреблённую потребителем энергию.  
    \end{enumerate}
\end{enumerate}

Хорошую систему от примитивной могут отличать следующие характеристики:
\begin{itemize}
    \item Учёт погрешностей прогнозов. Расчёт наихудшего варианта, наилучшего, наиболее вероятного и других статистических характеристик.
    \item Использование более точных оценок генерации.
    \item Расчёт стоимости лота для энергосистем оппонентов — чтобы выиграть лот, нужно ставить не собственную цену, а цену, большую, чем у оппонентов. Для этого необходимо оценивать ценность лота для оппонентов.
    \item Прогноз эффективности задуманной энергосистемы.
    \item Расчёт величины риска в зависимости от размера ставки — для этого нужно использовать данные об энергосистемах оппонентов и опыт взаимодействия с ними.
\end{itemize}

Важно, что такая система лишь помогает принимать решения, и не гарантирует того, что команда с наилучшей реализацией этой задачи наиболее эффективно проведёт аукцион.

Разработанные командами программы варьировались по сложности от простых таблиц Excel, до веб-сервисов с элементами ИИ.

\subsubsection*{Управляющие скрипты}

Это вторая точка сборки предметных задач, возникающих в командном туре.

Задача управляющего скрипта — используя возможность управления продажей электроэнергии во внешнюю энергосистему, управления аккумуляторами, прогнозирования генерации, максимизировать число очков, которое получит команда за командный тур.

Если эта задача не решена, то результаты команды будут плохими независимо от результатов аукциона и глубины решения предметных задач командного тура.

Самый простой вариант скрипта может действовать, например, по следующему алгоритму:

\begin{enumerate}
    \item Включить все отключённые линии
    \item Оценить генерацию на следующем такте. Вычислить энергобаланс следующего такта.
    \item ЕСЛИ энергобаланс положителен, ПЕРЕЙТИ к п. 6
    \item Попытаться ликвидировать дефицит из аккумуляторов.
    \item Ликвидировать дефицит из внешней энергосистемы.
    \item ЗАВЕРШИТЬ РАБОТУ
    \item Попытаться ликвидировать профицит, перенаправив мощность в аккумуляторы.
    \item Ликвидировать профицит, продав мощность во внешнюю энергосистему.
    \item ЗАВЕРШИТЬ РАБОТУ 
\end{enumerate}

Хороший скрипт может обладать следующими характеристиками:

\begin{itemize}
    \item Оценки энергобаланса на всю игру вперёд и ранняя закупка электроэнергии.
    \item Коррекция прогнозов генерации на основании реальных данных от электростанций.
    \item Использование распределения вероятных значений энергобаланса, чтобы вычислять средневзвешенную величину его коррекции: дефицитный энергобаланс обходится дороже профицитного, соответственно нужно минимизировать не модуль энергобаланса, а математическое ожидание экономических потерь в результате ошибок прогнозов.
    \item Такое управление аккумуляторами, которое полностью разряжает их к последнему такту игры.
    \item Прогнозирование возможностей аварийных отключений в энергосистеме и использование возможности ограниченного регулирования мощности потребителей и электростанций их избегания.
    \item Использование возможности ограниченного регулирования мощности потребителей и электростанций для увеличения экономической эффективности энергосистемы (в некоторых ситуациях плата за такое регулирование может быть меньше убытков на полноценное снабжение потребителя или плата за повышение генерации — меньше затрат на закупку мощности).
    \item Моделирование состояния энергосистем оппонентов для предсказания будущих состояний биржи.
    \item Уточнение моделей энергосистем оппонентов на основании реального состояния биржи.
    \item Управление риском: в ряде случаев осмысленно использование неоптимальных характеристик управления, которые несмотря на то, что они снижают математическое ожидание счёта в командном этапе, увеличивают вероятность обойти другую команду.    
\end{itemize}

\solutionSection

\inputminted[fontsize=\footnotesize, linenos]{python}{final/command_tour/ies/task_01/subtask_03/source_1.py}

