\subsection*{Финальная командная задача: Моделирование энергосистемы и разработка алгоритмов управления энергообеспечения}

На стенде моделируется два небольших города, энергосистемам которого предстоит быть полностью перестроенной. В том числе, в энергосистемы будет добавлено большое количество ветровой и солнечной генерации.

Энергосистемы объединены в одну и подсоединены к внешней энергосистеме по схеме «микрогрид»: каждая из энергосистем сначала балансируется собственными ресурсами, а затем пытается получить недостающую или передать избыточную энергию энергосистемам оппонентов. Только после этого все энергосистемы полностью балансируются через внешнюю энергосистему.

Каждая будущая энергосистема будет «разделена» между четырьмя конкурирующими компаниями. Каждая команда станет одной из восьми конкурирующих энергокомпаний и будет строить свою собственную энергосистему и управлять ей.

Разделение будет происходить через цепочку аукционов, в которых участники изначально будут в принципиально одинаковых условиях. Часть объектов могла быть установлена на любом стенде (они имели две идентичные копии на обеих площадках проведения), часть была «привязана» к стенду. «Привязанные» объекты имели строго одинаковые характеристики.

После этого команды проектируют собственные энергосистемы (из одних и тех же объектов можно составить очень разные по эффективности энергосистемы) и готовят скрипты для управления ими. Далее происходит натурное моделирование нескольких дней работы энергосистемы, в ходе которого участники управляют своими энергосистемами посредством скриптов; и происходит экспериментальное измерение эффективности построенных энергосистем. Очки, набранные командами во время моделирования, пересчитываются в баллы, которые их участники получат за командную часть олимпиады.

Проведение финала происходило в виде командного турнира. Цель команд участников — набрать наибольшее число баллов в турнире.

Финальный командный тур был распределенный, проходил на двух площадках — в Москве и Иркутске. На каждой площадке игры проводились на идентичных стендах. Связь между стендами и главным сервером игры осуществлялась через сеть Интернет. Возможности стенда позволяли одновременно играть 4-м командам, итого в одной игре одновременно могло принимать участие до 8 команд.

\subsection*{Регламент турнира}

\begin{enumerate}
    \item Подготовка.
    \begin{enumerate}
        \item Проводился уже сформированными командами. Это продолжительный по времени этап (3 дня), в течение которого участники знакомились с правилами, тренировались работать на стенде, изучали предоставленную систему и готовили заготовки (управляющие скрипты, стратегии и вспомогательные программы).
    \end{enumerate}
    \item Два полуфинала турнира
    \begin{enumerate}
        \item Команды делились на две группы по 7 команд (по 4 –Москва, по 3 - Иркутск) случайным образом. С каждой группой проводился отдельный полуфинал.
        \item В полуфинале проводилась одна игра.
    \end{enumerate}
    \item Финал турнира
    \begin{enumerate}
        \item В финал проходили по две команды с каждого стенда с каждого полуфинала (8 команд), набравшие наибольшее число очков (у.е.). 
        \item В финале проходила одна игра.
        \item Победитель определялся по сумме очков (у.е.) за игру.
    \end{enumerate}
    
\end{enumerate}

\subsection*{Этапы одной игры}

\begin{enumerate}
    \item Анализ прогнозов погоды. Минимум за 20 минут до игры участникам выдавались прогнозы погоды и потребления для каждого потребителя в предстоящей игре. За это время участники должны были спроектировать энергосистему, наиболее отвечающую предстоящим условиям.
    \item Основной аукцион. На этом этапе определялось, какой объект к чьей энергосистеме будет подключён. Аукцион закрытого типа, с продолжением в случае одинаковых ставок. Этот этап практически невозможно успешно пройти без глубокого предварительного анализа и заготовленных программ для поддержки принятия решений.
    \item Дополнительный аукцион. Проводился через 1 минуту после основного. Каждая команда имела право повторно «выставить на торги» любой из приобретённых ранее объектов. Этот этап давал командам шанс на исправление одной ошибки, если она не слишком велика.
    \item Монтаж энергосистемы и адаптация стратегии. Участникам было предоставлено на сборку сети 20 минут. За это время они находили оптимальную конфигурацию своей энергосистемы, монтировали её на стенде, включая оптимальную установку электростанций. В это же время команда адаптировала к получившейся энергосистеме составленные заранее управляющие скрипты.
    \item Моделирование. Команды в полном составе находились возле собственных терминалов управления; к стенду запрещено было приближаться ближе, чем на 2 метра всем, включая обслуживающий персонал, во избежание влияния на физические измерения. Участники наблюдали за работой своих скриптов и могли его заменить, например, в случае обнаружения ошибки. Число загружаемых скриптов правилами не было ограничено. Всё управление на этом этапе осуществлялось только скриптами.
\end{enumerate}

\subsection*{Аукцион}

Аукцион проводился по закрытой схеме. Выигрывал предложивший наибольшую цену в аукционе на электростанции, и наименьшую — в аукционе на потребителей. Стартовая цена для электростанций составляла 1, для потребителей — 10.

На ставку отводилось 30 секунд. В случае близости наилучших ставок на 0,5 или менее, проводился дополнительный тур аукциона, в котором участвовали только те, чьи ставки попали в диапазон 0,5 от ставки лидера.

В случае, если один из участников достигает предельной цены, аукцион заканчивается. Если два и более участника достигли предельной цены, для них начинался аукцион по системе «all-pay» («платят все»). В этом аукционе свои ставки выплачивают все: и победители, и проигравшие. Итоговые ставки участников этого аукциона вычитались из их результата.

Порядок выставления лотов фиксирован и известен участникам заранее.

В течение минуты после окончания аукциона каждая команда имела право выставить на торги один из своих объектов. Команда ничего с этого не приобретает, но получает возможность избавиться от лишнего объекта, нарушающего баланс энергосистемы. Команда не имела права делать ставки на объект, который выставила на торги. У каждой команды в наличии по умолчанию имелось один дом и одна солнечная электростанция.

\subsubsection*{Пример результатов аукциона}

\putImgWOCaption{15cm}{1}

\subsection*{Подзадачи финальной задачи}

В ходе выполнения финальной задачи оценивалось комплексное решение финальной задачи профиля. Выделение и формулирование подзадачи является частью интегральной (главной) задачи профиля. Проверкой решения подзадачи являлся вклад её в результат игры. Описанные ниже задачи выделены разработчиками и были известны участником из описания. Распределение усилий между подзадачами принимали сами команды.


