\subsection*{Краткая справка}

Финальный этап профиля проводился на специально разработанном стенде-тренажере «ИЭС» (Подробное устройство стенда представлено в Приложении 1). Участникам предстояло работать с моделью энергосистемы, в которой присутствует большое количество альтернативных источников энергии, накопителей энергии и широкие возможности для экономических действий. В результате участники изучали: физические и экономические параметры энергосетей; сильные и слабые стороны альтернативной энергетики, взаимосвязь инженерных и экономические решений и проблему качества и надежности электросетей.
В финальной задаче участникам требовались навыки и знания из школьной программы, а также самостоятельно приобретенные на хакатонах и из требуемых для решения задач второго этапа:
\begin{itemize}
    \item Умение программировать на языке Python.
    \item Умение работать со случайными величинами в программных вычислениях.
    \item Умение решать базовые оптимизационные задачи.
    \item Навыки реализации решений математических задач в виде программ.
    \item Понимание правил и решения закрытого аукциона первой цены.
    \item Навыки командной работы над программным кодом.
    \item Навыки работы с большими рядами данных в математических задачах.
    \item Навыки работы с рядами данных в алгоритмах.
    \item Понимание принципов работы и характеристик ветрогенераторов.
    \item Теория вероятностей.
\end{itemize}

Успешное решение финальной задачи предполагает освоение и развитие следующих знаний и навыков:
\begin{itemize}
    \item Умение работы в команде.
    \item Умение самостоятельно выделять и формулировать подзадачи.
    \item Понимание принципов оценки риска.
    \item Навыки программирования, в том числе на языке Python.
    \item Навыков работы с рядами данных, как аналитической, так и алгоритмической.
    \item Навыки работы с системами с инерцией.
    \item Практического использования теории вероятностей, в том числе в программных вычислениях.
\end{itemize}
