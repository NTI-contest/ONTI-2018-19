\section*{Приложение 1. Детали правил}

\subsection*{Прогнозы}

Все прогнозы характеризуются пятью числами, которые задают квартили распределения.

У каждого объекта прогноз свой.

Все прогнозы, кроме ветра, строятся как отклонения от паттерна. Прогнозы всегда умещаются в коридор от 80 до 120% значения паттерна.

Паттерн солнца (повторяется периодически с начала игры):

\noindent{0, 0, 0, 0, 0, 0, 0, 0, 0, 0, 0, 0, 1, 1.6, 2.5, 3, 3.7, 4.2, 5, 6, 7, 8, 9, 10, 11, 12, 13, 14, 14.5, 14.8, 15, 15, 15, 15, 14.1, 13.0, 11, 10.5, 9.8, 9, 8, 7, 6, 5, 4.3, 2, 1.2, 0.4}

Паттерн больниц:

\noindent{0.6, 0.6, 0.5, 0.4 , 0.3, 0.3, 0.4, 0.5 , 0.5, 0.7, 0.8, 1.0 , 1.1, 1.2, 1.4, 1.5 , 1.6, 1.7, 1.6, 1.6 , 1.5, 1.5, 1.4, 1.4 , 1.3, 1.3, 1.2, 1.2 , 1.2, 1.1, 1.1, 1.1 , 1.2, 1.2, 1.1, 1.0 , 1.0, 0.9, 0.8, 0.7 , 0.7, 0.7, 0.6, 0.6 , 0.5, 0.4, 0.4, 0.5}

Паттерн заводов:

\noindent{4.2, 4.3, 4.4, 4.5 , 4.6, 4.6, 4.7, 4.8 , 4.9, 4.9, 5.0, 5.1 , 5.3, 5.6, 5.9, 6.2 , 6.3, 6.3, 6.4, 6.3 , 6.5, 6.8, 6.9, 7.0 , 7.4, 8.2, 8.9, 8.8 , 8.7, 8.6, 8.8, 9.0 , 9.1, 9.0, 8.9, 8.8 , 8.0, 7.4, 7.0, 6.6 , 6.3, 6.0, 5.8, 5.2 , 5.0, 4.4, 4.1, 4.0}

Паттерн домов:

\noindent{1.5, 1.1, 0.8, 0.8 , 0.7, 0.7, 0.7, 0.8 , 0.9, 0.9, 1.0, 1.1 , 1.1, 1.2, 1.3, 1.4 , 1.5, 1.6, 1.8, 2.0 , 2.1, 2.1, 2.0, 1.7 , 1.4, 1.3, 1.3, 1.2 , 1.1, 1.2, 1.3, 1.5 , 1.6, 1.9, 2.1, 2.3 , 2.7, 2.9, 3.1, 3.7 , 4.2, 4.4, 4.3, 4.0 , 3.3, 2.8, 2.4, 2.1}

\subsection*{Отключения}

Предельная мощность подключения к внешней сети ограничена и составляет 30 МВт

При перегрузке подключения к внешней сети отключаются ветки главной подстанции в порядке 3, 2, 1, пока сеть не сможет автоматически сбалансироваться.

Предельная мощность, проходящая через узел подстанции, ограничена и составляет 20 МВт

Установка трансформатора увеличивает мощность внешнего подключения подстанции на 15 МВт

Потребителю можно отдать приказ на сокращение потребление, до 25%. За каждый сэкономленный МВт мощности игрок платит потребителю штраф в размере множителя, зависящего от типа потребителя, умноженной на количество недопоставленной электроэнергии и на стоимость единицы энергии по контракту с заданным потребителем. Множитель для больниц составляет 3, для домов ­— 1,5 и для заводов — 1.

Электростанции можно отдать приказ на сокращение или увеличение генерации. За сокращение генерации игрок дополнительно 1% от стоимости по контракту, за увеличение — 5%. Предел изменения мощности электростанции составляет от 0% до 120%

За отключение объекта также платится пропорциональный штраф: пятикратную стоимость МВт*такт для потребителей 1-й категогии, двукратную для 2-й и однократную для 3-й

Обслуживание отключённой электростанции обходится в 2,5 раза дороже работающей.

Максимальная мощность солнечной батареи составляет 15 МВт.

Максимальная мощность ветряка составляет 15 МВт.

\subsection*{Скрипты}

Суммарное время работы скриптов ограничено тремя секундами. Обратите внимание, что получение данных стенда скриптом занимает 0.25 секунды.

\subsection*{Аккумуляторы}

Ёмкость аккумулятора составляет 25 МВт, максимальная скорость заряда — 5 МВт*такт, максимальная скорость разряда — 5 МВт*такт 

\subsection*{Биржа}

Взаимодействие со внешней энергосистемой осуществляется через биржу. Энергия покупается/продаётся из внешней энергосистемы только при невозможности сделать это через других игроков.

Поставки энергии между игроками гарантируются внешней энергосистемой.

Существует три типа биржи, на 0, на 1, на 3 и на 10 ходов вперёд.

Цена, по которой игроки торгуются энергией, зависит от того, на какой бирже купил её покупатель, и на какой продал продавец. Цена будет в пользу того, кто выставил заявку на более ранней бирже. Таблица цен:

\begin{tabular}{|l|l|l|l|l|l|}
    \hline
    Покупатель $rightArrow$ & 0 & 1 & 3 & 10 & Внешняя ЭС \\
    \hline
    0 & 3.5 & 3 & 2.5 & 2 & 1 \\
    \hline
    1 & 4 & 3.5 & 3 & 2.5 & 2 \\
    \hline
    3 & 4.5 & 4 & 3.5 & 3 & 2.5 \\
    \hline
    10 & 5 & 4.5 & 4 & 3.5 & 3 \\
    \hline
    Внешняя ЭС & 7 & 5 & 4.5 & 4 & — \\ 
    \hline   
\end{tabular}

При наличии одной заявки на покупку и многих на продажу, от каждой заявки на продажу берётся одинаковый процент, и наоборот.

Работа на бирже проходит в автоматическом режиме. Игроки могут только составлять 
