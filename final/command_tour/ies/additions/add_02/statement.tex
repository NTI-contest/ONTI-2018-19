\section*{Приложение 2. Справка по системе управления скриптами}

\subsection*{Краткий принцип работы}

\begin{enumerate}
    \item Есть система управления скриптами (далее СУС). 
    \item СУС имеет именованные слоты для хранения управляющих скриптов. 
    \item Скрипт пишется на языке Python версии 3.6 и, помимо стандартных, использует специальную библиотеку для энергостенда. 
    \item Есть назначенная очередь выполнения, указывающая, какие скрипты и в каком порядке должны выполняться в ближайший ход. 
    \item В начале каждого хода помеченные на запуск скрипты перемещаются в текущую очередь и выполняются по порядку до тех пор, как очередь не закончится. 
    \item Скрипт может получить данные со стенда на текущий ход и отправлять на него управляющие приказы. Это основная задача управляющего скрипта и СУС в целом. 
    \item Скрипт может ставить в очередь другой скрипт (по его имени). Он выполнится после всех остальных. 
    \item Скрипт может запускать другой как дочерний, передавая ему на вход произвольные данные в JSON. При этом родительский скрипт блокируется до завершения всех дочерних. Дочерний скрипт может вернуть данные в JSON, они передаются в родительский после разблокировки. 
    \item Количество выполнений для каждого скрипта ограничено в течение хода и равно 100. Запуск скрипта с синтаксической ошибкой также уменьшает счётчик. 
    \item Общее время выполнения всех скриптов ограничено глобальным таймаутом. При его достижении все запущенные скрипты «убиваются», а дальнейший запуск запрещается до начала нового хода. 
    \item Количество вложенных дочерних скриптов тоже ограничено и равно 100. Но из-за пункта 9 это не будет проблемой. 
    \item Скрипт может узнать текущую очередь, количество оставшихся выполнений для любого скрипта и момент глобального таймаута (штамп времени). 
\end{enumerate}

\subsection*{Базовый пример}

\begin{minted}[fontsize=\footnotesize, linenos]{python}
    import ips # 1 
    psm = ips.init() # 2 
    clients = psm.powersystem.get_all_clients() # 3 
    consumption = 0 # прогноз суммарного потребления 
    generation = 0 # прогноз суммарной генерации 
    for client in clients: 
    if client.is_generator(): 
    generation += client.power[-1] 
    else: 
    consumption += client.power[-1]
    shortage = consumption - generation 
    if shortage > 0: psm.orders.trade0.buy(shortage) # 4 
    else: psm.orders.trade0.sell(-shortage) # 4 
\end{minted}

\begin{enumerate}
    \item Импорт библиотеки API стенда. 
    \item Запрос на сервер и создание объекта с данными системы на текущий ход. 
    \item Данные извлекаются путём вызова методов и функций, см. далее. 
    \item Вызываемые приказы сразу же отправляются на сервер и отражаются в интерфейсе.         
\end{enumerate}

\subsection*{Получение данных}

\subsubsection*{Прежде всего:}

\begin{minted}[fontsize=\footnotesize, linenos]{python}
    import ips # импортируем библиотеку 
    # при работе с тестовой версией нужно задать данные вручную 
    # например, из файла (другие способы будут ниже) 
    ips.debug_psm_file("state.json") # убедитесь, что файл лежит рядом со скриптом 
    psm = ips.init() # читаем состояние энергостенда в объект 
    # и продолжаем работать 
\end{minted}

\subsubsection*{Погода}

\begin{minted}[fontsize=\footnotesize, linenos]{python}
    # их структура одинакова и представляет собой список, 
    # каждый элемент которого соответствует своему ходу 
    print(psm.sun) # значения яркости солнца 
    print(psm.wind) # значения скорости ветра 
    # элемент состоит из двух полей: 
    # - реальное значение, которое может быть пусто (ход не наступил) 
    # - прогноз в виде процентилей 
    print(psm.wind[0]) # объект-пара для ветра за первый ход 
    print(psm.wind[0].value) # либо реальное значение, либо None 
    print(psm.wind[0].forecast) # прогноз в виде набора процентилей 
    # актуальное значение погоды берётся по номеру хода 
    # для извлечения номера хода можно использовать это: 
    tick = psm.get_move() # рассмотрим поля для отдельного взятого прогноза 
    q = psm.wind[tick].forecast print(q.lower0) # 0% 
    print(q.lower50) # 25% 
    print(q.median) # 50% 
    print(q.upper50) # 75% 
    print(q.upper0) # 100%                                                       
    # вспомогательные кортежи с парами значений 
    print(q.quart1) # Q1 (0%, 25%) 
    print(q.quart2) # Q2 (25%, 50%) 
    print(q.quart3) # Q3 (50%, 75%) 
    print(q.quart4) # Q4 (75%, 100%) 
    print(q.spread) # (0%, 100%) 
    print(q.spread50) # (25%, 75%) 
\end{minted}

\subsubsection*{Биржа}

\begin{minted}[fontsize=\footnotesize, linenos]{python}
    print(psm.exchange) # все предложения на всех биржах 
    print(psm.exchange[0]) # конкретное предложение

    # для отдельно взятого предложения 
    print(psm.exchange[0].exchange) # тип биржи (через сколько ходов) 
    print(psm.exchange[0].amount) # объём заявки 
    print(psm.exchange[0].issued) # номер хода, на котором она создано 
    print(psm.exchange[0].owner) # контрагент 
\end{minted}

\subsubsection*{Подстанции}

\begin{minted}[fontsize=\footnotesize, linenos]{python}
    print(psm.stations) # все подстанции в виде dict 
    print(list(psm.stations.keys())) # адреса подстанций 

    # возьмём главную подстанцию М0 (у вас будет другой адрес) 
    print(psm.stations["M0"]) # объект главной подстанции 
    print(psm.stations["M0"].addr) # адрес 
    print(psm.stations["M0"].cells) # кол-во аккумуляторов 
    print(psm.stations["M0"].transformers) # кол-во трансформаторов 
    print(psm.stations["M0"].charge) # общий заряд аккумуляторов 
    print(psm.stations["M0"].modules) # список модулей как объектов 
    print(psm.stations["M0"].modules[0].is_cell) # проверка типа модуля 

    # для аккумулятора можно получить 
    # динамический список со значениями заряда за этот и предыдущие ходы 
    print(psm.stations["M0"].modules[0].charge) # аналогичный формат значений используется и ниже 

    # ВНИМАНИЕ! значение за последний ход — конец списка (charge[-1]) 
    # в начале игры некоторые дин. списки пусты, это эквивалент нуля 

    # возьмём произвольную миниподстанцию m5 (у вас их может и не быть) 
    # print(psm.stations["m5"]) # миниподстанция 
    print(psm.stations["m5"].addr) # адрес миниподстанции 
\end{minted}

\subsubsection*{Энергосеть}

\begin{minted}[fontsize=\footnotesize, linenos]{python}
    print(psm.powersystem) # объект энергосети 

    print(psm.powersystem.get_all_lines()) # извлечь все имеющиеся линии 
    print(psm.powersystem.get_all_clients()) # всех имеющихся клиентов 
    print(psm.powersystem.get_all_regions()) # все имеющиеся энергорайоны 
    print(psm.powersystem.line("M0", 1)) # энергорайон на первой линии M0 

    # в системе линия представлена парой (адрес_подстанции, номер) 
    print(psm.powersystem.lines()) # получить все энергорайоны 
    key_list = [("M0", 1), ("M0", 2)] 
    print(psm.powersystem.lines(key_list)) # энергорайоны по ключам в списке 
    # эти две функции возвращают массив пар линия-район 
    # возьмём отдельный энергорайон 
    region = psm.powersystem.line("M0", 1) 
    print(region.get_all_lines()) # аналогичны функциям выше, но выдаются 
    print(region.get_all_clients()) # элементы из себя и соседних районов 
    print(region.get_all_regions()) # 
    print(region.online) # 
    print(region.lines) # линии к соседним энергорайонам (список) 
    print(region.clients) # клиенты в данном энергорайоне (список) 

    # возьмём отдельного клиента 
    client = region.clients[0] 
    print(client.is_consumer()) # ответ на вопрос "это потребитель?" 
    print(client.is_generator()) # или "это генератор?" 
    print(client.addr) # адрес объекта 
    print(client.contract) # тариф по контракту 
    print(client.profits) # доход за каждый ход (дин. список) 
    print(client.losses) # расход за каждый ход (дин. список) 
    print(client.power) # текущая мощность (дин. список) 
    # она же реальное потребление-генерация 
    print(client.influence) # влияние на объект (дин. список) 
    # если клиент — потребитель 
    print(client.preset) # список значений и прогнозов аналогично погоде 
    # это необходимо в основном для работы с прогнозами, 
    # для чтения текущих значений удобнее применять power 
\end{minted}

\subsubsection*{Прочая информация}

\begin{minted}[fontsize=\footnotesize, linenos]{python}
    print(psm.you) # ваш индекс игрока, пара чисел стенд-терминал 
    # не путать с адресом своей подстанции! 
    print(psm.score) # счёт на текущий момент (динамческий список) 
    print(psm.get_move()) # определить номер текущего хода (первый ход = 1) 
    print(psm.humanize()) # полное описание состояния стенда (отладка) 
\end{minted}

\subsubsection*{Приказы}

Существуют в двух форматах: 

\begin{itemize}
    \item строгие приказы, отправляются сразу 
    \begin{minted}[fontsize=\footnotesize]{python} 
    psm.orders 
    \end{minted} 
   
    \item ленивые приказы, сохраняются в виде просматриваемого объекта
    \begin{minted}[fontsize=\footnotesize]{python} 
    psm.orders_lazy 
    \end{minted}    
\end{itemize}

Есть два способа отправки ленивых приказов: 

\begin{itemize}
    \item вызов метода от объекта приказа
    \begin{minted}[fontsize=\footnotesize]{python}
    lazy_order.send() 
    \end{minted} 
     
    \item пакетная отправка нескольких приказов 
    \begin{minted}[fontsize=\footnotesize]{python}
    ips.send_orders([order1, order2]) 
    \end{minted} 
   
\end{itemize}


Приказы обрабатываются в порядке их отправки в систему! 

Все приказы на биржу учитываются и применяются отдельно! 

Для остальных приказов приоритет — у последнего вызванного! 


\begin{minted}[fontsize=\footnotesize, linenos]{python}
    ### Линии (выполняется последний отправленный) 
    psm.orders.line_on("M0", 1) # включение линии 1 подстанции M0 
    psm.orders.line_off("M0", 1) # выключение этой же линии 
    print(psm.orders_lazy.line_on("M0", 1).addr) # адрес подстанции 
    print(psm.orders_lazy.line_on("M0", 1).line) # линия 
    print(psm.orders_lazy.line_on("M0", 1).value) # устанавливаемое значение 

    #### Биржа (накапливается) 
    psm.orders.trade1.buy(5) # купить 5 МВт-ч на следующий ход 
    psm.orders.trade1.sell(5) # продать 5 МВт-ч на следующий ход 
    psm.orders_lazy.trade3.buy(5) # ... через 3 хода 
    psm.orders_lazy.trade10.buy(5) # ... через 10 ходов 
    print(psm.orders_lazy.trade0.buy(5).exchange) # тип биржи (число ходов) 
    print(psm.orders_lazy.trade0.buy(5).value) # объём купли-продажи 

    ### Аккумуляторы (выполняется последний отправленный) 
    # В качестве адреса указывается адрес подстанции! 
    psm.orders.cell_charge("M0", 1) # зарядить на 1 МВт-ч 
    psm.orders.cell_discharge("M0", 1) # разрядить на 1 МВт-ч 
    print(psm.orders_lazy.cell_charge("M0", 1).addr) # адрес 
    print(psm.orders_lazy.cell_charge("M0", 1).power) # мощность 

    ### Влияние на объекты (выполняется последний отправленный) 
    # В качестве адреса указывается адрес объекта! 
    psm.orders.influence("h0", 1) # установить влияние на h0 в значение 1 
    print(psm.orders_lazy.influence("h0", 1).addr) # адрес объекта 
    print(psm.orders_lazy.influence("h0", 1).value) # значение 
\end{minted}

\subsection*{Запуск других скриптов}

Есть два способа вызвать другой скрипт (или самого себя). Команда enqueue добавляет скрипт в конец очереди, т.е. он будет выполнен после завершения остальных скриптов. 
ips.enqueue("eta") 
Также можно выполнять скрипты как функции с входными и выходными данными: 

\begin{minted}[fontsize=\footnotesize, linenos]{python}
    ####### РОДИТЕЛЬСКИЙ СКРИПТ ####### 
    # Пусть у нас есть набор данных для передачи. 
    data = ["horse", "chair", 28] 
    # В дочерний скрипт передаётся data, 
    # скрипт блокируется до завершения дочернего. 
    # Вывод дочернего скрипта запишется в output 
    output = ips.call("beta", data) 

    ####### ДОЧЕРНИЙ СКРИПТ ####### 
    # Чтобы записать выходные данные, в дочернем скрипте пишется команда: 
    ips.exit_with(["horse", "chair", 28]) 
    # После её выполнения скрипт завершится, вернув управление родителю. 
\end{minted}
    
\subsection*{Вспомогательные команды}

\begin{minted}[fontsize=\footnotesize, linenos]{python}
    ips.send_orders(lazy_order_list) # массовая отправка ленивых приказов 
    ips.get_launches("alpha") # получить число оставшихся запусков для скрипта 
    ips.get_timeout() # получить UNIX-timestamp полного завершения выполнения 
    ips.get_queue() # получить текущую очередь выполнения 
    ips.set_order_trace(True) # включить/выключить вывод приказов в консоль 
\end{minted}

\subsection*{Отладка}

В целях упрощения отладки представлены функции для установки заглушек. Они работают исключительно в тестовой версии фреймворка, при выполнении в СУС данные вызовы лишь пишут предупреждение в поток ошибок и пропускаются. 

\begin{minted}[fontsize=\footnotesize, linenos]{python}
    # установка данных энергостенда, перед ips.init() 
    ips.debug_psm_file("state.json") # из файла 
    ips.debug_psm_file("<...>") # из строкового объекта 
    # установка входных данных, перед ips.buffer() 
    ips.debug_buffer_file("buffer.json") # аналогично 
    ips.debug_buffer_json("<...>") # 
    ips.debug_launches(99) # число запусков для get_launches 
    ips.debug_timeout(time.time() + 10) # таймстамп для get_timeout 
    # О, привет. Раз уж ты здесь, рекомендую почитать исходники библиотеки. 
    # В них есть полезные вещи для боевого применения и, возможно, ошибки. 
    # Найдёшь последние — дай знать преподавателям, они передадут.
\end{minted}
