\section*{Приложение 3. Устройство Стенд-тренажер “ИЭС” 2018-2019 года}

Стенд-тренажер “ИЭС” 2018-2019 года состоит из следующих основных частей:

\begin{enumerate}
    \item Модельная поверхность, на которой располагаются модели объектов энергосистемы: электростанций, потребителей и объектов энергетической инфраструктуры.
    \item Терминалы управления энергосистемой. Это персональные компьютеры, объединённые со стендом в единую информационную сеть. Каждая команда работает со своим терминалом.
    \item Модели объектов энергосистемы. Это небольшие стереотипные архитектурные модели, содержащие в себе необходимую управляющую электронику и измерительные системы.
    \item Светильники, моделирующие солнечное освещение. Они способны создавать освещённость в центре стола до 5 клк.
    \item Мощный вентилятор, моделирующий ветровые условия. Он способен создавать ветер со скоростью до 5 м/с на расстоянии не менее 1 метра от плоскости вращения.
    \item Модули расширения главной подстанции — аккумуляторы и трансформаторы.        
\end{enumerate}

\putImgWOCaption{15cm}{1}

\center Скриншот пользовательского ПО стенда (с комментариями)